\section{Kraftförsörjning}

Den elektriska energi, som behövs för elektronikutrustningar, hämtas från det allmänna elektricitetsnätet, ett batteri eller en ackumulator. Vissa batterityper kan återuppladdas och kallas då ackumulator.
Batterier och ackumulatorer avger en
nominell spänning, som beror av de ingående materialen och givetvis av laddningstillståndet. Moderna utrustningar för amatörradio är utförda för 12 V likström och försörjs
vanligen från ett nätanslutet kraftaggregat
På så sätt kan mobila radioutrustningar även
försörjas från startackumulatorn i fordonet.
Handburna radioutrustningarförsörjs från
en inbyggd ackumulator, som laddas från
stationär laddare.
Äldre stationära radioutrustningar drivs
nästan alltid med nätanslutna kraftaggregat
med en eller flera transformatorer och likriktare. Alternativt kan samma transformators
sekundärsida vara försedd med flera lindningar för olika spänningar och strömkretsar.
Det allmänna elnätet i Sverige levererar
växelspänning med frekvensen 50Hz. Nätspänningen för hushållsändamål är numera
400/230 v.
Tidigare importerade utrustningar i marknaden kan vara utförda för andra nätspännings- och skyddsjordningssystem än vad
som nu tillämpas i Sverige. Försiktighet med
sådan utrustning rekommenderas.

Halvm och helvågslikriktning m. m.
Bild II 3-34
Likriktning av spänningar och strömmar i en
krets görs med "elektroniska ventiler", som
släpper igenom ström endast i den s.k. passriktningen och stoppas i spärriktningen. En
sådan strömventil kallas för diod och kan
vara av typen vakuumrör eller halvledare. l
moderna konstruktioner används uteslutande halvledardioder i likriktarkopplingar.

+

!>l

Passriktning

[>l

+

Spärriktning

Bild II 3-35
Halvvågslikriktning
Vid halvvågslikriktning släpps endast varannan halwåg av en växelspänning igenom. l
den strömkrets, som bildas av transformatorns sekundärlindning, dioden och belastningen, flyter därför ström endast under
varannan halvperiod.

Helvågslikriktning
l följande kopplingar med två respektive fyra

dioder släpps varje halwåg av transformatorns växelspänning igenom så att alla halvvågor får samma polaritet. Ström flyter genom belastningen i samma riktning under
varje halvperiod. Följande sätt att anordna
helvågslikriktning är vanliga:
• Med två dioder och mittuttag på transformatorns sekundärlindning. Den ena dioden och ena lindningshalvan släpper igenom ström till belastningen under ena halvperioden. Den andra dioden och andra
lindningshalvan under följande halvperiod
o.s.v.
• Med fyra dioder (s.k. Graetz-brygga) och
inget mittuttag på transformatorns sekundärlindning, släpper dioderna 1 och 3 igenom ström under den ena halvperioden.
Dioderna 2 och 4 släpper igenom ström
under följande halvperiod o.s.v.

Glättningskretsar
Bild II 3-36
Efter likriktningen har växelspänningen omvandlats till en pulserande likspänning som
kan "glättas". Efter likriktarna ansluts då ett
glättningsfilter, som t. ex. kan bestå av laddningskondensatorn CL, induktansen L (s.k.
drossel) och glättningskondensatorn C 8 .
Parallellt över denna kondensator ligger för
elsäkerhetens skull en urladdningsresister
med hög resistans alltid inkopplad.
Säkerhetsresistorn skall ladda ur kondensatorerna, när kraftaggregatet inte är
anslutet till strömförsörjningen och belastningen. Säkerhetsresistorn (eng. bleede()
skall vara av trådlindad typ och kunna tåla
fyra gånger sin egen effektförbrukning.

Bild II 3-34 Halvledardioder
113-25

KRETSAR

HALVVAGSLIKRIKTNING

u~~a:u~
HELVAGSUKRIKTNING

a - med 2 dioder

b- med 4 dioder
(Graetzkoppfing)

1 :a halvvågen

Diod 1 och 3 i passriktning

u
2 :a ha tvvågen

Diod 2 Diod 2 och 4 i passriktning

Bild II 3-35 Halv- och helvågslikriktning

113-26

KRETSAR

HALVVÅGSUKRIKTARE MED GLATTNINGSFILTER

GRAETZKOPPLING MED GLATTNINGSFILTER

Bild II 3-36 Glättning av likspänning
l obelastat tillstånd är spänningen över
gånger högre
laddningskondensatorn
än effektivvärdet på transformatorns sekundärspänning. När en transformator i tomgång har ett effektiwärde av 230 V över
sekundärlindningen, så blir spänningen över
= 325 V.
säkerhetsmotståndet 230

.J2

.J2

Spänningshöjande likriktarkopplingar
Vid likriktning av växelspänningar enligt någon av ovanstående metoder behövs en
sekundärspänning från transformatorn av
minst samma storlek som den önskade likspänningen. Önskas en högre likspänning,
t.ex. den dubbla, men med samma sekundärspänning på transformatorn, så måste en
speciell likriktarkoppling användas.
Bild 113-37
Bilden visar en spänningsfördubblande
koppling. Under 1 :a halwågen laddas kondensator C 1 upp. Under 2:a halwågen laddas kondensator C2 upp. Kondensatorerna
är kopplade i serie och den ena kondensatorn
hinner inte bli urladdad under tiden som den
andra kondensatorn blir uppladdad. Följden
blir att belastningen ser kondensatorernas
spänningar som seriekopplade och därmed

har en spänningsfördubbling erhållits. Det
finns även kopplingar för flerdubbling av
spänningar, vilket bl. a. används för att alstra
accelerationsspänningen för TV-bildrör.

Spänningsstabilisering

Bild II 3-38
Utspänningen från ett kraftaggregat tillåts i
många fall att endast variera mellan vissa
värden, fastän inspänningen och strömuttaget varierar mycket. Ett vanligt sätt att hålla
konstant spänning är att anordna en automatisk spänningsdelare efter glättningsfiltret
Glimlampan och zenerdioden har egenskapen att spänningsfallet över dem är i det
närmaste konstant inom ett visst strömområde. Glimlampor arbetar på högre spänningar och används i utrustningar med elektronrör. Zenerdioder arbetar på de lägre
spänningar som används i dagens elektronik.
stabiliseringen tillgår så att t. ex. zenerdioden får ingå som aktiv del i en spänningsdelare, som består av en resister i serie med
belastningen och zenerdioden parallellt med
den. Zenerdioden tar upp variationerna i
belastningsströmmen, varvid spänningen
113-27

KR

1 :a Halvvågen

2:a Halvvågen

Bild II 3-37 Likriktarkoppling med spänningsdubbling
över spänningsdelarens uttag blir stabiliserad. Vid större strömuttag kan zenerdioden
inte ensam ta upp hela den effekt som den
reglerar bort. l stället tas effekten upp av en
eller flera transistorer som i sin tur regleras
av zenerdioden.
l vissa fall behövs i stället en reglerad
utström från kraftaggregatet Även för detta
ändamål används kopplingar med zenerdioder och transistorer.
Senare utvecklingsformer är s.k. switchade aggregat. l sådana regleras spänningen eller strömmen genom sönderhackning (switching). Genom att förändra förhållandet mellan till- och frånslagstiderna kan
man skapa det önskade medelvärdet. Metoden ger hög verkningsgrad. Switch-frekvensen är i storleksordningen 20 kHz eller högre. På grund av den högre frekvensen
krävs mindre kondensatorer i switchade
aggregat. Sådana kraftaggregat kan emellertid ge störningar, varför effektiv avstörning behövs.

113-28

Ostabiliserad
spänning in

Stabiliserad
spänning ut

Bild II 3-38 Spänningsstabilisering
