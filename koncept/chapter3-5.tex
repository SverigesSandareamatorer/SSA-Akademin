\section{Detektorer -- Demodulatorer}
\harecsection{\harec{a}{3.5}{3.5}}
\label{detektorer}
\index{detektor}
\index{demodulator}

% \mediumminusbotfig{images/cropped_pdfs/bild_2_3-55.pdf}{Dioddetektorn}{fig:BildII3-55}

\subsection{Allmänt}
\label{detektorer_allmänt}

Sändaren omvandlar informationen i lågfrekventa signaler till högfrekvens som
kan strålas ut från en antenn.
I mottagaren återskapas informationen genom att den högfrekventa signalen demoduleras.

Vanligen sker signalbehandlingen i mottagaren i flera steg, där den högfrekventa
radiosignalen först blandas ner till en mellanfrekvens (MF) och sedan
demoduleras till en lågfrekvent signal (LF).

Men det finns även direktblandade mottagare, som blandar ner radiosignalen
direkt till lågfrekvens.

I mottagare som är specialiserade för ett sändningsslag, används bara en typ av
demodulator medan mottagare för flera sändningsslag, AM, SSB/CW, FM et cetera har
flera demodulatorer.
Det finns många typer och namn på demodulatorer, till exempel detektor och diskriminator.
Här beskrivs några av dem.

\subsection{AM-detektorer}
\index{detektor!AM}
\index{amplitudmodulation!detektor}

\subsubsection{Dioddetektorn AM (A3E)}
\harecsection{\harec{a}{3.5.1}{3.5.1}, \harec{a}{3.5.2}{3.5.2}}
\index{dioddetektor}
\index{amplitudmodulation!dioddetektor}
\index{A3E!dioddetektor}

% \mediumbotfig{images/cropped_pdfs/bild_2_3-55.pdf}{Dioddetektorn}{fig:BildII3-55}

Bild \ssaref{fig:BildII3-55} visar en superheterodynmottagare där den sista
MF-kretsen är induktivt kopplad till demoduleringsdioden.
Den amplitudmodulerade MF-signalen visas som ett amplitud/tid-diagram.

Dioden klipper antingen de negativa eller positiva halvvågorna,
beroende på hur den är vänd -- polariserad.

LF-signalen filtreras ut ur de högfrekventa pulserna med ett LF-lågpassfilter.

LF-signalen är nu överlagrad på en likspänning.
I talpauserna sänds bara bärvågen och då lämnar AM-demodulatorn bara
likspänning, som skiljs från LF-förstärkaren med en kondensator.
Kondensatorn släpper bara igenom LF-signalen, som förstärks.

Dioddetektorn följer amplituden och är ett exempel på en amplitudformsdetektor.

\mediumminusbotfig{images/cropped_pdfs/bild_2_3-56.pdf}{Produktdetektor för AM (A3E) och CW (A1A)}{fig:BildII3-56}

\subsubsection{Produktdetektorn SSB (J3E)}
\harecsection{\harec{a}{3.5.3}{3.5.3}}
\index{SSB!produktdetektor}
\index{J3E!produktdetektor}
\index{produktdetektor}


Det finns flera metoder att demodulera en SSB-signal, såsom fasningsmetoden,
filtermetoden och den så kallade tredje metoden.
Filtermetoden är numera den allra vanligaste och beskrivs här samt
illustreras i bild \ssaref{fig:BildII3-56}.
%
En SSB-signal med undertryckt bärvåg består av endast ett sidband.
Det andra sidbandet och bärvågen undertrycks i sändaren.
%
Vid demoduleringen av SSB-signalen alstras i mottagaren en signal som
ersättning för den bärvåg som undertrycktes i sändaren.
Det undertryckta andra sidbandet ersätts inte.

I en mottagare med direktblandning blandas SSB-signalen med VFO-signalen,
varvid en del av blandningsprodukterna faller ut på LF-nivå.

I en superheterodynmottagare däremot, blir SSB-signalen först blandad
med en VFO-signal och som resultat erhålls en mellanfrekvens MF.
Den till MF omvandlade signalen förstärks, filtreras och blandas med en
lokal BFO-signal i ytterligare en blandare, kallad produktdetektor.
Några blandningsprodukterna faller ut på LF-nivå.
Ett lågpassfilter följer efter detektorn för att filtrera ut LF-signalerna.

Numera består produktdetektorn vanligen av en ringblandare, som i ett omvänt
förlopp även kan användas vid DSB-modulering i en sändare.
Bilden visar demoduleringen av en SSB-signal som innehåller tre LF-toner.

\mediumtopfig{images/cropped_pdfs/bild_2_3-57.pdf}{Amplitudbegränsning vid FM-mottagning}{fig:BildII3-57}
\mediumbotfig[0.7]{images/cropped_pdfs/bild_2_3-58.pdf}{Ideal arbetslinje för diskriminator}{fig:BildII3-58}

\subsubsection{CW-/SSB-detektorer CW (A1A)}
\index{CW!detektor}
\index{SSB!detektor}
\index{A1A!detektor}
\index{detektor!CW}
\index{detektor!SSB}
\index{detektor!A1A}

Även telegrafisignaler, även kallat CW, blir demodulerade när MF-signalerna
och BFO-signalen blandas i en produktdetektor.

Till skillnad från SSB är det vid CW inte nödvändigt med en given skillnad
mellan MF- och BFO-frekvenserna.
Frekvensskillnaden påverkar bara överlagringstonens frekvens, men inte
läsligheten av CW-budskapet.

Många moderna mottagare har en fast BFO-fre\-kv\-e\-ns för CW, som ger en
\SI{800}{\hertz}-ton vid rätt frekvensinställning.
I stället för lågpassfiltret för SSB, används ibland ett bandpassfilter, som
bara släpper igenom CW-signaler i frekvensområdet \SI{800}{\hertz} -- en
idealfrekvens för god läsbarhet av morsetecken.

\subsection{FM- och PM-detektorer}
\harecsection{\harec{a}{3.5.4}{3.5.4}, \harec{a}{4.2.4}{4.2.4}, \harec{a}{4.3.5}{4.3.5}}

\index{FM-detektor}
\index{FM!detektor}
\index{detektor!FM}
\index{PM-detektor}
\index{PM!detektor}
\index{detektor!PM}
\index{Automatic Frequency Control (AFC)}
\index{AFC}
\label{fm_detektor}


Vid vinkelmodulering överförs informationen enbart genom frekvens-
eller fasvariationer i bärvågen.
De amplitudvariationer som kan uppstå före demoduleringen är ej önskvärda i
detta sändningsslag.
Av den anledningen finns i FM-mottagare en amplitudbegränsare (eng. \emph{limiter}) före
diskriminatorn (se bild \ssaref{fig:BildII3-57}).
Frekvensvariationerna i den FM-modulerade signalen omvandlas därefter av
detektorn till LF-spän\-ning som motsvarar det utsända talet.

% \mediumfig{images/cropped_pdfs/bild_2_3-58.pdf}{Ideal arbetslinje för diskriminator}{fig:BildII3-58}

Demoduleringen ska ske med mottagaren inställd mitt på avsedd sändarfrekvens.
Ett hjälpmedel för det är en indikator, som vid rätt inställning visar värdet
noll.
Positivt eller negativt utslag anger att inställningen är för högt respektive
för lågt i frekvens, som illustreras i bild \ssaref{fig:BildII3-58}.
En sådan indikator fanns i tidiga FM-mottagare.
Nu används i stället en \emph{Automatic Frequency Control (AFC)} som själv
ställer in mottagaren om sändarfrekvensen är tillräckligt nära.

\newpage
\subsubsection{Slope-detektorn -- Diskriminatorn FM (F3E)}
\index{slope-detektorn}
\index{detektor!slope-detektorn}
\index{FM!slope-detektorn}
\index{FM-diskriminator}
\index{detektor!FM}
\index{detektor!F3E}
\index{FM!detektor}
\index{F3E!detektor}

\mediumfig[0.8]{images/cropped_pdfs/bild_2_3-59.pdf}{Slope-detektorn}{fig:BildII3-59}

Bild \ssaref{fig:BildII3-59} visar två resonanskretsar som är kopplade induktivt
till den sista MF-kretsen.
Resonansfrekvensen för dessa båda kretsar är något högre respektive något lägre
än mellanfrekvensen.
De signalspänningar som uppträder över resonanskretsarna likriktas och
seriekopplas med varandra med motsatt polaritet.

När de båda resonanskretsarna matas med samma frekvens, kommer
likspänningarna att ta ut varandra.
När frekvensen avviker uppåt i frekvens, kommer kretsen med den högre
resonansfrekvensen i kraftigare svängning än den andra kretsen och avger högre
likriktad spänning.
När frekvensen avviker nedåt i frekvens, skiftar de båda kretsarna roller,
och den resulterande likriktade spänningen skiftar till motsatt polaritet.

Vid växelvisa frekvensändringar i MF, över och under vilofrekvensen, blir
resultatet en växelspänning ut från likriktarnas utgångsfilter, som är
LF-signalen.

\subsubsection{Foster-Seeley-diskriminatorn}
\index{Foster-Seeley-diskriminator}
\index{detektor!Foster-Seeley-diskriminator}
\index{FM!Foster-Seeley-diskriminator}

\smallfig{images/cropped_pdfs/bild_2_3-60.pdf}{Foster-Seeley diskriminator}{fig:BildII3-60}

Bild \ssaref{fig:BildII3-60} illustrerar en \emph{Foster-Seeley-diskriminator}.
Denna tidiga demodulator har god linjäritet, om den föregås av en god
amplitudbegränsare, men har tämligen dålig känslighet.

Sista MF-förstärkarsteget avslutas med en transformator vars båda
lindningar ingår i resonanskretsar avstämda till MF.
MF-signalen överförs från primär- till sekundärsidan dels med induktion och dels
med en kondensator till mitten av sekundärlindningen.
Signalen delas på så sätt i två grenar med en fasförskjutning av +90\degree~
respektive \ang{-90}.
Signalerna i grenarna likriktas var för sig och sammanlagras i ett RC-nät.

Om MF-signalen inte devierar är LF-spänningen i grenarna lika.
Men eftersom grenspänningarna har motsatt polaritet tar de ut varandra och
LF-signalen blir noll.
När MF-frekvensen devierar av modulering ökar signalamplituden i den ena
grenen och minskar i den andra.
LF-signalens amplitud blir då proportionell mot frekvensdeviationen.

\subsubsection{Räknardiskriminatorn}
\index{räknardiskriminator}
\index{FM!räknardiskriminator}
\index{detektor!räknardiskriminator}

\mediumfig[0.75]{images/cropped_pdfs/bild_2_3-61.pdf}{Räknardiskriminatorn}{fig:BildII3-61}

Bild \ssaref{fig:BildII3-61} visar räknardiskriminatorn.
En monostabil vippa (eng. \emph{monoflop}) påverkas att slå över av fyrkantspulserna från de
amplitudbegränsade FM-signalerna.

En sådan vippa är en digitalkoppling som, när den matas med en godtyckligt lång
spänningspuls, ändå kommer att leverera en spänningspuls med konstant längd.
För varje positiv halvvåg levererar den monostabila vippan en impuls av konstant längd.
Tidsavstånden mellan pulserna kommer att vara proportionella mot FM-signalens frekvens.
Vid varierande frekvens kommer impulserna med varierande tidsavstånd.
Ett lågpassfilter filtrerar ut lågfrekvensen ur signalen och en pulserande
likspänning kvarstår.
Med denna likspänning laddas kondensatorn upp till ett medelvärde.
Vid en högre frekvens av lika långa pulser blir medelvärdet högre än vid en
lägre pulsfrekvens.

De överlagrade svängningarna på likspänningen utgör LF-signalen.
Utan en monostabil vippa med lika långa pulser hade medelvärdet varit konstant.
Man kan säga att FM-signalen blivit omvandlad till en pulslängdmodulerad signal
(PLM-signal).

\subsubsection{PLL-demodulatorn}
\index{PLL-demodulator}
\index{PLL!demodulator}
\index{FM!PLL-demodulator}
\index{detektor!PLL-demodulator}

\mediumfig[0.75]{images/cropped_pdfs/bild_2_3-62.pdf}{PLL-demodulatorn}{fig:BildII3-62}

Bild \ssaref{fig:BildII3-62} visar PLL-demodulatorn.
Den frekvensmodulerade MF-signalen och en VCO-signal matas in i en
fasjämförare.
VCO-frekvensen följer frekvensändringarna hos FM-signalen.
Avstämningsspänningen för VCO är en likspänning.
Den modulerande LF-spänningen är överlagrad på denna likspänning.

LF-frekvenserna är för låga för att kunna reglera VCO-frekvensen, men
via en kondensator kan de styra LF-förstärkaren.

De båda sista metoderna lämpar sig speciellt för demodulering av FM-signaler.
Det finns ytterligare sätt att demodulera FM-signaler.
Gemensamt för alla är att de fungerar bättre ju lägre mellanfrekvensen är.
Därför utförs de flesta FM-mottagare som dubbel- eller trippelsuprar, med låg
MF.
