\section{Egenskaper i mottagare}
\textbf{HAREC a.\ref{HAREC.a.4.4}\label{myHAREC.a.4.4}}
\index{mottagare!egenskaper}

\subsection{Närliggande kanaler}
\textbf{HAREC a.\ref{HAREC.a.4.4.1}\label{myHAREC.a.4.4.1}}

Närliggande kanaler kan skapa störningar när de läcker in.
Därför gäller det att mottagaren kan undertrycka dem, även när de är starkare
än den valda kanalen, så att man får så god läsbarhet på den valda kanalen.
Närliggande kanaler kan därför anses vara störkällor.
Moderna mottagare medger att flytta både över och undre gräns för att
undertrycka allt för närgående kanaler.
Även begreppet roofing filter förekommer för filter som hjälper till att
filtrera med branta flanker och god undertryckningsförmåga.
Detta är en del av många att ha goda så kallade stor-signal-egenskaper.

\subsection{Selektivitet}
\textbf{HAREC a.\ref{HAREC.a.4.4.2}\label{myHAREC.a.4.4.2}}
\index{selektivitet}
\index{mottagare!selektivitet}
\index{förselektering}
\index{spegelfrekvenser}

Med \emph{selektivitet} (eng. \emph{selectivity}) menas en mottagares förmåga
att skilja ut önskade signaler och undertrycka övriga.
Summariskt beskrivet kallas avståndet mellan yttergränserna för det önskade
frekvensområdet för bandbredd.

När det gäller superheterodynmottagare finns två selektivitetsbegrepp:
\begin{itemize}
  \item Det ena är \emph{förselektering} för att dämpa de \emph{spegelfrekvenser} som
uppstår i samband med blandning av mottagna signaler och oscillatorfrekvenser i
mottagaren.
  \item Det andra är selektiviteten i en superheterodynmottagares MF-steg som används
för att utskilja den önskade signalen efter blandningsförloppen.
\end{itemize}

\subsection{Frekvensstabilitet}
\textbf{HAREC a.\ref{HAREC.a.4.4.4}\label{myHAREC.a.4.4.4}}
\index{frekvensstabilitet}
\index{temperaturkompenserad}
\index{TCXO}
\index{ugnskompenserad}
\index{OCXO}
\index{lokaloscillatorfrekvens}
\index{PLL}
\index{DDS}
\index{Direct Digital Synthesis (DDS)}

\emph{Frekvensstabilitet} (eng. \emph{frequency stability}) är viktigt för
mottagare är viktigt för att kunna hitta sändare på angiven frekvens fort,
kunna stanna på den frekvensen utan att glida ifrån signalen och dessutom
undvika att glida in i närliggande signaler som stör.

Frekvensstabilitet tillgodoses i moderna mottagare med kristalloscillatorer,
och man kan ofta köpa till \emph{temperaturkompenserad} (\emph{TCXO}) eller
\emph{ungskompenserad} (\emph{OCXO}) kristalloscillator för att få en högre
frekvensstabilitet i hela mottagaren.

För att få bäst nytta så genereras alla \emph{lokaloscillatorfrekvenser} låsta
till samma kristalloscillator, något som med modern PLL och DDS teknik blivit
inte bara möjligt utan både kompakt, billigt och med hög prestanda.

\subsection{Spegelfrekvensproblemet vid mottagning}
\textbf{HAREC a.\ref{HAREC.a.4.4.5}\label{myHAREC.a.4.4.5}}
\index{spegelfrekvenser}
\index{mottagare!spegelfrekvenser}
\index{närselektion}
\index{mottagare!närselektion}
\index{förselektion}
\index{mottagare!förselektion}

\begin{figure}
  \includegraphics[width=\textwidth]{images/cropped_pdfs/bild_2_4-22.pdf}
  \caption{Enkelsuper med låg MF och ingen förselektion}
  \label{fig:bildII4-22}
\end{figure}

\textbf{Exempel:}
I bild \ref{fig:bildII4-22} ska en sändning på 3600~kHz ska tas emot och
VFO-frekvensen är 4055~kHz.
Mellanfrekvensfiltret undertrycker sändningar på så närliggande frekvenser
som till exempel 3603 och 3597~kHz.
Denna egenskap kallas för \emph{närselektion}.

Men tyvärr kan en sändning på så avlägsen frekvens som 4510~kHz ändå
störa mottagningen, den goda närselektionen till trots.
Avståndet mellan 4510~kHz och vår mottagningsfrekvens 3600~kHz är 910~kHz.
Frekvensen 4510~kHz och VFO-signalen bildar också en blandningsprodukt,
som har frekvensen 455~kHz.
Vid en VFO-frekvens av 4055~kHz och en mottagningsfrekvens av 3600~kHz benämns
4510~kHz som \emph{spegelfrekvensen}.
Avståndet mellan spegelfrekvens och mottagningsfrekvens är dubbla värdet av
mellanfrekvensen -- i detta exempel \(2 \cdot 455kHz = 910\) kHz.

Signaler på mottagningsfrekvensen och spegelfrekvensen alstrar båda
blandningsprodukter med VFO-frekvensen, som har mellanfrekvensens värde.
Mellanfrekvensfiltret kan därför inte undertrycka en främmande signal på
spegelfrekvensen.

\begin{figure}
  \includegraphics[width=\textwidth]{images/cropped_pdfs/bild_2_4-23.pdf}
  \caption{Enkelsuper med låg MF och med förselektion}
  \label{fig:bildII4-23}
\end{figure}

Däremot kan en mottagaringång med \emph{förselektering} (eng.
\emph{preselection}) undertrycka den.
I bild \ref{fig:bildII4-23} finns en selektiv krets före blandaren släpper
igenom ett smalt frekvensband med mittfrekvensen 3600~kHz, men dämpar till exempel
frekvensen 4510~kHz på grund av den stora frekvensskillnaden.
En förselektion har alltså tillförts som komplement till den närselektion som
erhålls med mellanfrekvensfiltret.

\begin{wrapfigure}{R}{0.5\textwidth}
  \includegraphics[width=0.5\textwidth]{images/cropped_pdfs/bild_2_4-24.pdf}
  \caption{Enkelsuper med hög MF och med förselektion}
  \label{fig:bildII4-24}
\end{wrapfigure}

Ju längre ifrån varandra nyttofrekvens och spegelfrekvens ligger, desto bättre
är förselektionen.
Med en mellanfrekvens av 455~kHz är alltså detta avstånd 910~kHz.
I långvågs- och mellanvågsområdet är det tillräckligt för att man med enkla
medel ska kunna skapa tillräckligt selektiva filter.

\textbf{Exempel:}
Vid den högsta mottagningsfrekvensen på mellanvåg 1605~kHz är spegelfrekvensen
2515~kHz, som ligger 1,57 gånger högre i frekvens och med ett avstånd av
910~kHz.
I kortvågsområdet dämpas inte en spegelfrekvens på avståndet 910~kHz
tillräckligt kraftigt.
Vid den högsta mottagningsfrekvensen på kortvåg 30~MHz ligger nämligen
spegelfrekvensen 30,910~MHz endast 1,03 gånger högre i frekvens.
Med antagandet, att förselektionskretsen har ett Q-värde av 30, blir
bandbredden 53,5~kHz vid frekvensen 1605~kHz.

Med samma Q-värde blir bandbredden 1000~kHz vid frekvensen 30~MHz, vilket
innebär att förkretsen inte längre kan dämpa så närliggande spegelfrekvenser
på ett effektivt sätt.

I mottagare för högre frekvenser används därför högre mellanfrekvens
för att öka avståndet till spegelfrekvensen, som illustreras i bild
\ref{fig:bildII4-24}.
I moderna kortvågsmottagare är det vanligt med en mellanfrekvens av 9~MHz
eller högre.
Vid en mottagningsfrekvens av 30~MHz och en mellanfrekvens av 9~MHz är
spegelfrekvensen 48~MHz, vilket är 1,6~gånger mottagningsfrekvensen.
Detta möjliggör förselektionsfilter med tillräcklig dämpning av
spegelfrekvensen.

\begin{figure}
  \includegraphics[width=\textwidth]{images/cropped_pdfs/bild_2_4-25.pdf}
  \caption{Samtidig för- och närselektion i superheterodynmottagare}
  \label{fig:bildII4-25}
\end{figure}

Bilden \ref{fig:bildII4-25} visar hur när- och förselektion kompletterar
varandra i ett frekvensspektrum.
Märk, att passbandbredden \(b\) i förselektionskretsen anger avståndet mellan
de frekvenser där signalamplituden dämpats till 70~\% av toppvärdet.
I exemplet här ovan har antagits att förkretsen för kortvågsmottagning har
samma Q-värde som förkretsen för mellanvågsmottagning.

Vid högre frekvenser, i VHF- och UHF-området, kan inte önskat Q-värde
erhållas i sådana kretsar som används i KV-området och lägre.
Andra lösningar blir då nödvändiga, till exempel kavitetsfilter och helixfilter.

\subsubsection{MF-bandbredd vid AM (A3E)}
\index{amplitudmodulation}
\index{mottagare!AM}
\index{MF-bandbredd}
\index{mottagare!MF-bandbredd}

\begin{figure}
  \includegraphics[width=\textwidth]{images/cropped_pdfs/bild_2_4-26.pdf}
  \caption{MF-bandbredd vid AM (A3E)}
  \label{fig:bildII4-26}
\end{figure}

Bild \ref{fig:bildII4-26} visar en amplitudmodulerad signals frekvensspektrum
består av bärvågen och två sidfrekvenser -- eller sidband om sidfrekvenserna
är många.

Bandbredden i MF-kretsarna måste vara minst så stor att sidofrekvenserna
längst bort från bärvågen kan passera.
Dessa frekvenser motsvarar de högsta modulerande tonerna.
Vid rundradiosändningar på mellanvåg utsänds alla frekvenser upp till 4,5~kHz.
Detta motsvarar en bandbredd av 9~kHz.
För enbart talöverföring är en bandbredd av 6~kHz tillräcklig, vilket
motsvarar en LF-gränsfrekvens av 3~kHz.

Ett för smalt MF-filter skär bort de yttre delarna av sidbanden.
LF-signalerna kommer då att förlora de höga tonerna (diskanten).
Om däremot filtret är för brett, kommer närliggande utsändningar också att
höras.

I vissa mottagare kan MF-bandbredden anpassas till förhållandena.
Det är alltså en fråga om en kompromiss mellan bättre ljudkvalitet och
mindre störd mottagning.

\subsubsection{MF-bandbredd vid SSB (J3E)}
\index{SSB}
\index{mottagare!SSB}
\index{MF-bandbredd}
\index{mottagare!MF-bandbredd}
\index{snedstämning}
\index{MF-skift}
\index{passband-tuning}

\begin{figure}
  \includegraphics[width=\textwidth]{images/cropped_pdfs/bild_2_4-27.pdf}
  \caption{MF-Bandbredd och passbandtuning vid SSB (J3E)}
  \label{fig:bildII4-27}
\end{figure}

Mellanfrekvensfiltret för SSB-mottagning ska endast släppa igenom
ett av de två sidbanden, så som illustreras i bild \ref{fig:bildII4-27},
vars bredd är skillnaden mellan högsta och lägsta överförda LF-frekvens.
Inom amatörradio är detta 3~kHz - 0,3~kHz = 2,7~kHz, alltså något mindre än
hälften av bandbredden vid AM.

Ett alltför brett MF-filter skulle också släppa igenom oönskade
signaler från angränsande frekvenser.
Å andra sidan skulle ett för smalt MF-filter skära bort signaler i det
önskade frekvensregistet och försvåra mottagningen.
Smala filter kan å andra sidan utnyttjas för att dämpa signaler, till exempel från
en för nära liggande sändare eller en som har för stor bandbredd.

När närliggande sändare stör mottagningen ges följande möjligheter:
\begin{itemize}
\item \emph{Snedstämning.}
  Att göra en liten snedavstämning, uppåt eller nedåt i frekvens.
  Därigenom ändras frekvensläget på det mottagna talet, men vid små
  frekvensavvikelser blir förvrängningen liten.
  Läsligheten blir sämre, men mottagningen på det hela taget bättre.

\item \emph{MF-skift.}
  Som just beskrivits kan en liten snedavstämning göras.
  I vissa mottagare är det ordnat så att också BFO-frekvensen kan förskjutas
  så att frekvensläget på talet blir återställt igen.
  Därmed blir MF-passbandet skenbart förflyttat uppåt eller nedåt i frekvens
  (MF-skift, IF-shift).
  Det verkliga frekvensläget mellan nyttosignal och BFO behålls.
  I alla händelser blir basen eller diskanten på nyttosignalen avskuren,
  beroende på var denna ligger i frekvens.

\item \emph{Passband-tuning.}
  Om det finns störande sändare både över och under i frekvens, går det inte
  att skära bort störningarna med ett enkelt MF-skift, eftersom antingen den
  ena eller den andra störande sändaren ändå skulle höras.
  För det fallet erbjuder några moderna mottagare möjligheten att flytta
  MF-passbandets övre och undre frekvensgräns oberoende av varandra (bandpass
  tuning m.m.).
  Detta förutsätter, att mottagaren är en trippelsuper med branta filter i
  varje MF-steg.
  Vidare måste VFO, 1:a BFO och 2:a BFO kunna ställas in var för sig.
  Frekvensläget på MF I och/eller MF II kan då förskjutas över respektive
  filters passband, oberoende av varandra.
  Därigenom uppstår skenbart effekten att filterkurvorna skjuts emot varandra.
  Samma effekt skulle fås om kristallfiltren gick att avstämma, vilket ju inte
  är möjligt.
  Moderna SDR mottagare kan göra motsvarande genom att justera de digitala
  MF-filtren.
\end{itemize}

\subsubsection{MF-bandbredd vid CW (A1A)}
\index{CW}
\index{mottagare!CW}
\index{MF-bandbredd}
\index{mottagare!MF-bandbredd}

En CW-signal har som bekant inte bandbredden noll Hz, utan det handlar
i grunden om en amplitudmodulerad signal.
Vid en nycklingshastighet av 60 tecken per minut är bandbredden cirka 100~Hz
och vid i 120 tecken per minut den dubbla, cirka 200~Hz.

I vissa mottagare används ett SSB-filter även för mottagning av CW.
En vanlig bandbredd på ett SSB-filter är 2,7~kHz och då kommer även
stationer på närliggande frekvenser att höras, detta illustreras i bild
\ref{fig:bildII4-28}.
Låt vara att de flesta av dessa stationer hörs med olika frekvens.

\begin{figure}
  \includegraphics[width=\textwidth]{images/cropped_pdfs/bild_2_4-28.pdf}
  \caption{Olika MF-bandbreder vid CW (A1A)}
  \label{fig:bildII4-28}
\end{figure}

Fler än 20 CW-stationer får plats inom en bandbredd motsvarande en SSB-kanal.
Den mänskliga hjärnan, kan med någon övning koncentrera sig på en av dessa
signaler medan övriga uppfattas som störande.

Det tidigare nämnda LF-bandpassfiltret skulle emellertid åstadkomma en
bättre selektion och bekvämare avlyssning.
Men om en annan station inom passbandet är mycket starkare än den station
som är av intresse, då blir MF-förstärkaren antingen överstyrd av den
starkare signalen eller AGC reglerar ner förstärkningen så att den svagare
signalen inte längre kan höras trots det smala LF-filtret.
Selektionen i en mottagare bör därför sitta ''så långt fram som möjligt''.
I det skildrade exemplet skulle ett smalt filter i MF vara till bättre nytta
vid CW-mottagning.
Bandbredden på ett sådant filter är 250--500~Hz, således endast något bredare
än CW-signalen.

Med ett ännu smalare CW-filter kan, på grund av bristande frekvensstabilitet hos
sändare och/eller mottagare, svårigheter uppstå att finna den önskade signalen.
Välutrustade mottagare har passband-tuning även för CW, steglös
bandbreddsreglering eller stegvis valbara filterbandbredder.
Då kan mottagaren ställas in på den önskade signalen med en stor bandbredd
som därefter minskas.
För mottagning av RTTY (radiofjärrskrift) med 170~Hz skift mellan de två
frekvenserna, kan ett 500~Hz-filter användas.
Smalare filter går däremot inte så bra.

\subsubsection{Bandbredd vid FM (F3E)}
\index{frekvensmodulation}
\index{mottagare!FM}
\index{MF-bandbredd}
\index{mottagare!MF-bandbredd}
\index{frekvensdeviation}
\index{bandbredd}

En FM-sändare med frekvensdeviationen \(\Delta f_{max}\) och högsta
modulerande LF-moduleringsfrekvensen \(f_{LF_{max}}\) har bandbredden

\[ b = 2 \cdot (\Delta f_{max} + f_{LF_{max}}) \]

Inom amatörradio är det brukligt med en maximal deviation av 3~kHz och
en övre gränsfrekvens av 3~kHz, vilket motsvarar en bandbredd av 12~kHz.

Fullgod mottagning är möjlig endast om MF-filtren i mottagaren har
minst den bandbredd, som sändaren har.
Men vid för stor mottagarbandbredd kan även stationer på närliggande frekvenser
uppfattas.
Sedan 1996 är det av IARU Region~1 rekommenderade kanalavståndet 12,5~kHz
vid FM-trafik på VHF- och UHF-amatörradiobanden.

Det är vanligare med för stor deviation på FM-sändaren än att
mottagaren är alltför smal.
En för stor deviation, avsaknad av deviationsbegränsare och för hög
LF-gränsfrekvens medför en onödigt stor bandbredd på sändaren.
Motstationen får då mottagningssvårigheter och stationer på angränsande
kanaler blir också störda.

Det blir allt vanligare med 12,5~kHz kanalavstånd även för repeatrar,
varför det är viktigt att alla sändare är rätt inställda.

\subsection{Signalkänslighet och brus}
\textbf{HAREC a.\ref{HAREC.a.4.4.3}\label{myHAREC.a.4.4.3}}
\index{signalkänslighet}
\index{mottagare!signalkänslighet}
\index{brus}
\index{mottagare!brus}
\index{SN}
\index{SINAD}

Om man ställer in mottagaren på en ledig frekvens, så hör man vid full
förstärkning ett brus likt det från ett vattenfall.

Bruset kommer från de svaga växelspänningar som uppstår när
laddningsbärarna rör sig genom de material som strömkretsen består av.
Beroende av bruskällan sträcker sig frekvensspektrum från noll
till nära nog oändligt.
På grund av egenskaperna skiljer man mellan en rad specifika bruskällor:

\begin{itemize}
\item Resistorbrus, även kallat ''vitt brus'', som uppstår i resistiva
  komponenter.
  Bruset sträcker sig över hela det mätbara frekvensområdet varvid
  energifördelningen är lika över hela området.

\item Kretsbrus, som uppstår i resistanser i svängningskretsar i resonans.

\item Antennbrus, som är sammansatt av bruset från antennens
  strålnings- och förlustresistanser samt av det galaktiska brus som
  antennen tagit emot.

\item Transistorbrus uppstår av laddningsbärarnas rörelser i
  halvledarmaterial.
\end{itemize}

Mer information om brus i komponenter finns i avsnitt \ref{termisktbrus}

Det bildas en sammanlagd brusspänning som kan bestämmas.
Man talar om ett brustal, som är ett mått på mottagningssystemets egenbrus.
Detta ska ställas mot styrkan på den mottagna signalen.
Man talar om ett förhållande mellan signaleffekt och bruseffekt.
Det finns flera metoder att mäta och uttrycka detta förhållande som kallas
S/N (signal to noise ratio).
För att uppfatta den information som kommer ur en mottagares LF-utgång måste
nyttosignalen vara ett antal gånger starkare än bruset.
Den lägre gränsen för att uppfatta tal i kortvågsmottagare är ett brusavstånd
i storleksordningen 10~dB.

\begin{wrapfigure}[19]{R}{0.5\textwidth}
  \includegraphics[width=0.5\textwidth]{images/cropped_pdfs/bild_2_4-29.pdf}
  \caption{S/N-värde}
  \label{fig:bildII4-29}

  \includegraphics[width=0.5\textwidth]{images/cropped_pdfs/bild_2_4-30.pdf}
  \caption{SINAD-värde}
  \label{fig:bildII4-30}
\end{wrapfigure}

I en broschyr på en kortvågsmottagare kan man till exempel läsa
''Sensitivity SSB, CW: less than 0,25~\(\mu V\) for 10~dB S/N''

Termen S/N betyder Signal/Noise, det vill säga styrkeförhållandet signal/brus
uttryckt i dB.
Det innebär att en signal kan läsas vid 25~\(\mu V\) signalnivå och ett S/N av
mindre än 10~dB.
Utöver brusnivån i mottagaren spelar också distorsionen en roll.

\begin{align*}
  \begin{array}[b]{l}
    \text{Signalbrus-} \\
    \text{förhållande}
  \end{array} &= \frac{S + N + D}{N} \text{ dB} \\
  \text{där} \quad S &= \text{Signalnivå} \\
  N &= \text{Brusnivå} \\
  D &= \text{Distorsionsnivå} \\
\end{align*}

I en broschyr på en VHF-mottagare kan man till exempel läsa
''Sensitivity FM: Less than 0,18~\(\mu V\) for 12~dB SINAD''

Termen SINAD betyder Signal, Noise and Distorsion.
Vid denna definition tar man även hänsyn till distorsionsprodukter som orsakas
av den modulerande signalen.

\[
\text{SINAD} = \frac{S+N+D}{N+D}\text{ dB}
\]

\subsection{Intermodulation, korsmodulation}
\textbf{HAREC
  a.\ref{HAREC.a.4.4.6}\label{myHAREC.a.4.4.6},
  a.\ref{HAREC.a.4.4.7}\label{myHAREC.a.4.4.7}
}
\label{intermodulation}
\index{intermodulation}
\index{korsmodulation}
\index{blocking}
\index{storsignalegenskaper}

Utöver att en bra modern mottagare bör ha tillräcklig frekvensstabilitet,
känslighet och selektivitet bör den även ha goda så kallade
\emph{storsignalegenskaper}.

Med storsignalegenskaper menar man hur bra en relativt svag nyttosignal på
mottagaringången motstår påverkan av starka frekvens nära signaler med hög
fältstyrka.
Störningar av detta slag uppstår genom icke linjära förlopp i komponenter i
mottagarens ingångssteg, varvid mottagna signaler med stor amplitud blir
förvrängda.

Korsmodulation och intermodulation är två begrepp som är förknippade
med storsignalegenskaperna.
Båda kan visserligen definieras och bestämmas entydigt, men de förväxlas ändå
ofta.

En för stark signal klipper dessutom i mixrar och detta gör att allt mindre
signal kan detekteras varvid känsligheten sjunker och till slut kommer den
tänkta signalen vara helt undertryckt, så kallad \emph{blocking}.

\subsubsection{Korsmodulation}
\textbf{HAREC
  a.\ref{HAREC.a.4.4.8}\label{myHAREC.a.4.4.8}
}
\index{korsmodulation}
\index{mottagare!korsmodulation}

Med korsmodulation menas, att den inkommande nyttosignalen amplitudmoduleras
med modulationsprodukter från en annan frekvensnära amplitudmodulerad signal,
varvid korsmodulationen uppstår i olinjära komponenter i mottagaringången
(försteg, blandare).
När man med mottagaren i AM-läge ställt in den på någon bärvåg så hörs också
andra starka, frekvensnära stationer.

Det måste alltså alltid finnas en nyttosignal på den inställda frekvensen för
att det ska uppstå korsmodulation.
När nyttosignalen försvinner så försvinner även korsmodulationen.

För dåligt fasbrus hos mottagaren kan vara en orsak till att starka
grannkanaler mixas in och detekteras.

\subsection{Intermodulation}
\index{intermodulation}
\index{mottagare!intermodulation}

Vid så kallad intermodulation blandas två starka inkommande signaler i olinjära
komponenter i mottagaringången.
Deras blandningsprodukter faller på mottagningsfrekvensen så att den störs,
vare sig det finns en nyttosignal där eller inte.

\subsection{Frekvensstabilitet}

Se avsnitt \ref{oscillatorer}.
