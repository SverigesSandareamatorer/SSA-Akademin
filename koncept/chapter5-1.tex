\section{Sändare}
\label{sändare}
\index{sändare}

För att översiktligt beskriva en komplicerad sändare eller transceiver
behövs ibland ett enklare framställningssätt än detaljrika
principscheman. Då kan ett blockschema vara till stor hjälp.

Hela apparaten kan ses som ett antal funktionsblock. Hur de samverkar
framgår i stort av blockschemat. Där återfinns oscillatorer, blandare,
förstärkare etc. I schemat kan även finnas uppgifter om frekvenser och
spänningar m.m.

Det finns olika slags funktionsblock - kretsar. Kombinationen av block
ger apparater med olika egenskaper. Exempel är s.k. raka sändare med
samma frekvens genom hela sändaren, superheterodynsändare där
frekvensblandning används, frekvensmultiplicerande sändare etc.

\subsection{Rak sändare}
\textbf{
HAREC a.\ref{HAREC.a.5.1.1b}\label{myHAREC.a.5.1.1b},
 a.\ref{HAREC.a.5.2.1}\label{myHAREC.a.5.2.1}
}
\index{rak sändare}
\index{sändare!rak}

\begin{wrapfigure}{R}{0.5\textwidth}
  \includegraphics[width=0.5\textwidth]{images/bild_2_5-01}
  \caption{Enstegs sändare}
  \label{fig:bildII5-1}
\end{wrapfigure}

Bild \ref{fig:bildII5-1}.

Den raka sändaren är det enklaste sändarkonceptet. Då är oscillatorns
frekvens samma som sändningsfrekvensen och ingen frekvensomvandling
sker i signalvägen. Om en antenn kopplas till oscillatorn så blir den
en enkel enstegs sändare.

I flerstegs raka sändare följs oscillatorn av ytterligare funktioner
på samma frekvens som oscillatorn. Buffertsteg, drivsteg och slutsteg
kan vara sådana funktioner.

\begin{figure}
  \includegraphics[width=\textwidth]{images/bild_2_5-02}
  \caption{Flerstegs rak sändare}
  \label{fig:bildII5-2}
\end{figure}

Bild \ref{fig:bildII5-2}.

Bilden visar en rak sändare, som består av oscillator+ buffertsteg 1 +
buffertsteg 2 + drivsteg + effektförstärkare.

Oscillatorn följs av ett avlastande buffertsteg 1. På så sätt blir
oscillatorns frekvensstabilitet bättre. Buffertsteg 2 avlastar
ytterligare och matar dessutom ett effekthöjande drivsteg, som ger
driveffekt till slutsteget, samt slutsteget där den slutliga
effekthöjningen sker.

Raka sändare kan användas för CW, FM, PM och AM, men inte DSB och SSB.
Fördelen med raka sändare är enkelheten.  Nackdelen är att alla steg
arbetar på samma frekvens, varvid risken för återverkan på ett
föregående funktionssteg är större. Oönskad återkoppling kan då bli
följden. Genom att i första hand bygga in VFO och buffertstegen i
metallkapslingar, s.k. skärmar, så minskas denna risk.

\subsection{Sändare med frekvensmultiplicering}
\textbf{
HAREC a.\ref{HAREC.a.5.3.2}\label{myHAREC.a.5.3.2},
 a.\ref{HAREC.a.5.3.5}\label{myHAREC.a.5.3.5},
 a.\ref{HAREC.a.5.3.6}\label{myHAREC.a.5.3.6},
 a.\ref{HAREC.a.5.3.9}\label{myHAREC.a.5.3.9}
}
\index{frekvensmultiplicering}
\index{sändare!frekvensmultiplicering}

Helst väljer man en arbetsfrekvens för oscillatorn där den är mest
frekvensstabil.

Om högre frekvens önskas på nyttosignalen, så kan man
t.ex. multiplicera oscillatorfrekvensen. I olinjära kretsar alstras
övertoner, som ofta utnyttjas i detta syfte.

Endast när kravet på frekvensstabilitet är lågt används den frekvens,
som VFO eller CO arbetar på, även för nyttosignalen.

\begin{figure}
  \includegraphics[width=\textwidth]{images/bild_2_5-03}
  \caption{FM-sändare med frekvensmultiplicering}
  \label{fig:bildII5-3}
\end{figure}

Bild \ref{fig:bildII5-3}.

Oscillatorn svänger här på en låg frekvens, som multipliceras i
olinjära förstärkarsteg till en hög sändningsfrekvens. Oftast
multipliceras frekvensen två eller tre gånger i vart och ett av
förstärkarstegen.

Bilden visar ett blockschema för en FM sändare för 435~MHz (70
cm-bandet). Oscillatorfrekvensen är 8,056~MHz. I fyra av de
efterföljande förstärkarna multipliceras frekvensen 2, 3, 3 respektive
3 gånger, alltså totalt 54 gånger. Sändningsfrekvensen blir då \(8,056
\cdot 54 = 435\) MHz.

Variationer i oscillatorfrekvensen blir också multiplicerade. I detta
exempel blir sändningsfrekvensens deviation 54 gånger större än
oscillatorfrekvensens deviation. En deviation av max 3000Hz från den
nominella sändningsfrekvensen motsvaras av följande deviation från
oscillatorfrekvensen,

\[∆f = \frac{3000}{54} = 55,6\text{ [Hz]}\]

FM-sändare för VHF, UHF och SHF utförs ofta med
frekvensmultiplikation. Jämfört med en rak sändare är komponentbehovet
större, men i stället ger den låga oscillatorfrekvensen god
frekvensstabilitet, vilket är en fördel. Risken för oönskade
självsvängningar är mindre i en frekvensmultiplicerande än i en rak
sändare, eftersom in och utgångsfrekvenserna i flera av stegen är
olika.

De frekvensmultiplicerande stegen i bild \ref{fig:bildII5-3} arbetar i klass C,
d.v.s. olinjärt, vilket medför amplituddistorsion. Vid frekvens-och
fasmodulering saknar emellertid detta betydelse, eftersom amplituden i
det fallet inte är informationsbärande. Övertoner i nyttosignalen bör
dock filtreras bort.

\subsection{Sändare med frekvensblandning - superheterodynsändare}
\textbf{HAREC a.\ref{HAREC.a.5.1.1a}\label{myHAREC.a.5.1.1a}}

\subsubsection{Telegrafisändare (CW) för kortvåg}
\index{CW}
\index{sändare!CW}

\begin{figure}
  \includegraphics[width=\textwidth]{images/bild_2_5-04}
  \caption{2-bands CW-sändare med frekvensblandning}
  \label{fig:bildII5-4}
\end{figure}

Bild \ref{fig:bildII5-4}.

En VFO är mest stabil på låga frekvenser medan en CO har god
stabilitet även på högre frekvenser. När signalerna från dessa
blandas, bildas blandningsprodukter som är skillnaden och summan av
signalernas frekvenser. Bilden visar en telegrafisändare där detta
fenomen används för sändning inom området 14,0--14,5 eller 3,5--4,0
beroende på passbandet i filtret efter blandaren.

Resultatet är en superheterodyn-VFO med både variabel och stabil
signal. På bilden har valts ett filter med passband för det övre av
dessa frekvensområden.

\subsubsection{Telefonisändare (SSB) för kortvåg}
\textbf{
HAREC a.\ref{HAREC.a.5.2.2}\label{myHAREC.a.5.2.2},
 a.\ref{HAREC.a.5.3.1}\label{myHAREC.a.5.3.1},
 a.\ref{HAREC.a.5.3.4}\label{myHAREC.a.5.3.4},
 a.\ref{HAREC.a.5.3.10}\label{myHAREC.a.5.3.10}
}
\index{telefoni}
\index{SSB}
\index{sändare!SSB}

\begin{figure}
  \includegraphics[width=\textwidth]{images/bild_2_5-05}
  \caption{2-bands SSB-sändare med frekvensblandning}
  \label{fig:bildII5-5}
\end{figure}

Bild \ref{fig:bildII5-5}.

Bilden visar en SSB-sändare för två kortvågsband och bygger på
sändaren i bild \ref{fig:bildII5-4}.  Filtermetoden är den mest använda för att bereda
en SSB-signal. Oscillatorsignalen amplitudmoduleras i en balanserad
blandare. I en sådan undertrycks bärvågen medan de båda sidbanden
släpps fram. Det ena sidbandet undertrycks med ett
bandpassfilter. Denna SSB-signal flyttas till avsett frekvensband
genom ännu en frekvensblandning och ytterligare filtrering.

I exemplet är CO-frekvensen 9~MHz. VFO har frekvensområdet 5,0--5,5
MHz. Vid blandningen fås blandningsprodukter inom frekvensområdena
14,0--14,5 och 4,0--3,5~MHz. Genom att välja bandpassfilter kan man
sända i ett av dessa frekvensområden.  Efterföljande driv- och
slutsteg utförs för att arbeta i detta frekvensband, antingen utan
särskild avstämning - s.k. bredbandigt utförande - eller genom
avstämning på en viss frekvens, vilket ger renaste signalen.

Bild \ref{fig:bildII5-6}.

Bilden visar en SSB-sändare som liknar den i bild \ref{fig:bildII5-5}. Den stora
skillnaden är att signalfrekvensen kan flyttas till flera olika band
med hjälp av ännu en frekvensblandning.  Därför används fler valbara
bandpassfilter.

\begin{figure}
  \includegraphics[width=\textwidth]{images/bild_2_5-06}
  \caption{Flerbands SSB-sändare med frekvensblandning}
  \label{fig:bildII5-6}
\end{figure}

I en SSB-signal ligger all information i amplituden, till skillnad
från en FM-signal där all information ligger i frekvensen. En
SSB-signal får alltså inte förvrängas. Det innebär att
förstärkarstegen i SSB-sändare måste arbeta linjärt, d.v.s. en
utsignal ska vara proportionell till insignalen i varje moment.

\subsection{PLL-styrda sändare}
\index{PLL}
\index{sändare!PLL}

PLL-styrning är inte ett sändarkoncept. Det är ett sätt att styra
frekvensen i en oscillator och hålla den stabil med hjälp av en
likspänning från en PLL - Phase Locked Loop vilket är en digitalt
styrd krets.

En PLL kan användas t.ex. i raka sändare och heterodynsändare. I det
första fallet (bild \ref{fig:bildII5-2}) kan frekvensen i den enda ocillatorn
styras av en PLL. I det andra fallet (bild \ref{fig:bildII5-6}) kan frekvensen i
någon av oscillatorerna styras av en PLL.  En närmare beskrivning av
PLL-styrning av dessa två sändarkoncept följer här.

\subsubsection{PLL-styrd FM-sändare för 144--146~MHz}
\textbf{
HAREC a.\ref{HAREC.a.5.2.3}\label{myHAREC.a.5.2.3}
}

\begin{figure}
  \includegraphics[width=\textwidth]{images/bild_2_5-07}
  \caption{PLL-styrd FM-sändare för FM}
  \label{fig:bildII5-7}
\end{figure}

Bild \ref{fig:bildII5-7}.

Bilden visar en PLL-styrd rak sändare med en VCO (spänningsstyrd
oscillator) och ett PA (effektförstärkare).

VCO ingår som det frekvensstyrda elementet i en PLL. Utfrekvensen från
VCO (ärvärdet) avläses och delas periodiskt med talet 10 och matas in
i en programmerbar frekvensdelare. Eftersom frekvensområdet för VCO är
144--146~MHz, kommer infrekvensen till den programmerbara delaren att
ligga i området 14,4--14,6~MHz. Delningstalet i denna delare kan
programmeras in i steg om 1 mellan talen 5760 och 5840.

Med den första delarens divisor 10 och den andra delarens divisor
inställd t.ex. på 5760, så avges ur delarkedjan en puls varje gång som
VCO har genomfört 57600 svängningar. Vid en VCO-frekvens av 144~MHz
(144000~kHz) motsvaras divisorn 57600 (= \(10 \cdot 5760\)) av en
pulsfrekvens av 2,5~kHz ut från räknarkedjan. På samma sätt kommer en
VCO-frekvens av 144025~kHz och divisorn 57610 (= \(10 \cdot 5761\))
också att ge en pulsfrekvens av 2,5~kHz, likaså 146~MHz och divisorn
58400 o.s.v.

VCO-frekvensen låses alltså i intervall om 25~kHz till närmaste
delningstal, för att uppnå en pulsfrekvens av 2,5~kHz. Om
VCO-frekvensen (är-värdet) avviker från det inställda delningstalet
(bör-värdet), så kommer pulsfrekvensen att bli högre eller lägre än
2,5 kHz.

Pulsfrekvensen jämförs i en s.k. fasjämförare med en kristallstyrd
referensfrekvens som efter en delning med 10 också är 2,5~kHz.
Utspänningen från jämföraren är en likspänning, som intar ett
medelvärde då infrekvenserna är lika, men ett högre eller lägre värde
när de skiljer. Denna likspänning används för att kontinuerlig styra
VCO-frekvensen till likhet med börvärdet. Regleringsförloppets
hastighet bestäms av tidskonstanten i ett lågpassfilter, det
s.k. loop-filtret.

Sändningsfrekvensen regleras alltså med styrspänningen. Med samma
spänning går det också att frekvensmodulera oscillatorn.  Det görs så,
att LF-signalen från modulatorn överlagras på styrspänningen genom
additiv blandning (se kapitel \ref{blandare}) via en kondensator. De
variationer i reglerspänningen som kommer av talet är snabbare än
loopfiltrets tidskonstant Variationerna av talet hinner därför inte
uppfattas som frekvensavvikelser och blir därför inte utreglerade.
Drosseln efter loop-filtret förhindrar att moduleringssignalen
kortsluts av filtrets kondensator.

Frekvensinställningen, d.v.s. programmeringen av delaren kan utföras
på flera sätt. Exempel på inställningsorgan är tumhjuls-omkopplare,
logikkretsar i kombination med en knappsats o.s.v.

\begin{figure}
  \includegraphics[width=\textwidth]{images/bild_2_5-08}
  \caption{PLL-styrd SSB-sändare för kortvåg}
  \label{fig:bildII5-8}
\end{figure}

\subsubsection{PLL-styrd sändare för kortvåg}

Bild \ref{fig:bildII5-8}.

Bilden visar ett avancerat koncept för en kortvågssändare.
SSB-signalen alstras på frekvensen 9~MHz och blandas med 61~MHz i 1:a
blandaren.

Summafrekvensen 70~MHz filtreras fram som mellanfrekvens. Den önskade
sändningsfrekvensen fås genom att blanda 70~MHz MF med frekvensen från
VCO och därefter filtrera fram skillnadsfrekvensen.  VCO i detta
exempel täcker frekvensområdet 40--69,5~MHz. Således blir sändarens
täckningsområde 1,5--30~MHz. För att filterfunktionen ska bli
optimal, kan den delas upp på flera valbara filtersektioner, t.ex. ett
per amatörband. Valet kan ske automatiskt och styrt av frekvensläget
på VCO.

Den absoluta ändringen mellan de två extrema sändningsfrekvenserna är
så stor som 28,5~MHz eller 1:20. Frekvensändringen i VCO är 29,5~MHz,
men där är ändringsförhållandet mellan de extrema frekvenserna endast
1:1,74, vilket kan täckas av en enda VCO. Vid en lägre 2:a MF-frekvens
skulle det behövas flera omkopplingsbara VCO för att täcka hela
frekvensområdet

Exempel: Vid en MF på 9~MHz behöver VCO-funktionen täcka 9,5-39~MHz,
d.v.s. 1:4,11, vilket är för mycket för en VCO.

SSB-signalen efter 2:a blandaren är inte lämplig att använda i
regleringsslingan i PLL. Anledningen är att bärvågen är undertryckt i
denna signal och att därför HF-frekvenserna i det resterande sidbandet
varierar i takt med de modulerande LF-frekvenserna.

I konceptet på bilden rekonstrueras bärvågen i en 1:a
kontrollblandare, genom blandning av de två CO-frekvenserna 9 och 61
MHz. Den framfiltrerade bärvågen med frekvensen 70~MHz blandas med
VCO-frekvensen i 2:a kontrollblandare och ur denna signal
framfiltreras den rekonstruerade bärvågen. Denna stämmer perfekt med
den undertryckta bärvågens frekvens och innehåller inga
LF-signaler. Bärvågsfrekvensen delas i en programmerbar frekvensdelare
och jämförs med frekvensen från en kristallstyrd referensoscillator
CO. Ur fasjämföraren erhålls en likspänning som styr VCO via ett
loop-filter. Frekvensen ställs in genom att programmera delaren i PLL.

I en modern sändare finns ofta en mikroprocessor, som erbjuder talrika
möjligheter bl.a. till frekvensinställning, minnen och avsökning av
frekvenser.

Det beskrivna konceptet är avancerat.  Frekvensen i alla oscillatorer
styrs av samma referensoscillator. Frekvensstabiliteten beror alltså
enbart på referensoscillatorns stabilitet.

Omkopplingen mellan LSB och USB kan göras antingen genom att behålla
SSB-filtret och ändra frekvensen 9~MHz med ett värde så att filtret
blir verksamt i det motsatta sidbandet eller genom att behålla
frekvensen 9~MHz och byta till ett SSB-filter som är verksamt i det
motsatta sidbandet.

En PLL-styrd sändare har både kristalloscillatorns stabilitet och
variabel frekvens över ett stort frekvensområde trots ett litet antal
styrkristaller. En sådan sändare kan relativt enkelt styras digitalt.

En principiell nackdel med alla sändare med PLL-oscillator är
fasbruset. En annan nackdel är den stora komponentmängden
(se kapitel \ref{superheterojämförelse}).
