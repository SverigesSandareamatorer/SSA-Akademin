\section{Störningsorsaker}
\label{Störningsorsaker}
\subsection{Störningar från sändare}
\harec{a}{9.2.1}{9.2.1}
\harec{a}{9.2.2}{9.2.2}
\index{störning!av sändare}

HF-förstärkare, till exempel i sändarslutsteg, kan komma i oönskad självsvängning,
vilket kan uppstå av flera orsaker; det kan vara bristande avkoppling av
matningsspänningar, induktiv och/eller kapacitiv återkoppling etc.

Effektförstärkare kan även överstyras.
Då uppstår intermodulation och övertoner som strålas ut på oönskade frekvenser.
I många fall kan störningar undvikas med en eller flera av följande åtgärder:

\begin{itemize}
\item Använd inte mer effekt än vad som behövs.
\item Undvik att överstyra sändarslutsteget (kontrolleras t.ex. med
  ALC-mätaren).
\item Förse sändarutgången med lågpassfilter.
  På så sätt undertrycks övertoner.
\item Anpassa sändarens och antennanläggningens impedanser till varandra.
  Stäm av sändarens \(\pi\)-filter och/eller en separat antennanpassningsenhet
  rätt.
  En felinställd sändare kan medföra oavsiktligt utstrålning.
\item Koppla in balanseringsnät (balun) mellan osymmetriska antennledningar
  (koaxialkablar) och symmetriska antenner.
\item Placera antennen högt och fritt och så långt från personer och
  störningskänslig utrustning som möjligt.
  Fältstyrkan är nämligen högst närmast antennen.
  Se kapitel \ref{EMF} om elektriska fält.
\item Undvik direkt HF-instrålning på elnätet genom att använda nätfilter.
\item Använd ''mjuk'' nyckling av bärvågen (avrundade telegrafitecken).
  Vid hård nyckling alstras övertoner i form av knäppar som hörs långt vid
  sidan av sändningsfrekvensen. Se även kapitel \ref{Nycklingsfilter}.
\end{itemize}

\subsection{Störningar på radiomottagning}
\harec{a}{9.2.3.1}{9.2.3.1}
\harec{a}{9.2.3.2}{9.2.3.2}
\harec{a}{9.2.3.3}{9.2.3.3}
\index{störning!radiomottagning}

I regel uppstår störningar på radiomottagning först när instrålade signaler
uppnått en viss styrka -- immunitetsnivån för HF.
Man kan tala om tre slags HF-immunitet hos mottagare:

\begin{itemize}
\item mot signaler genom antenningången
\item mot signaler genom övriga anslutna ledningar, till exempel högtalar-
  och nätledningar
\item mot elektriska och/eller magnetiska fält som strålar direkt in i
  apparaten.
\end{itemize}

I de båda första fallen kan det hjälpa med komplettering med hög- och/eller
lågpassfilter och skärmströmsfilter.

Störningar orsakade av instrålning är svårast att avhjälpa och fordrar ingrepp i
mottagaren, vilket bör överlåtas till en fackman med tillgång till
tillverkarens serviceinstruktioner.

\subsection{Störningar på TV-mottagning}
\index{störning!TV-mottagning}
\index{störning!digital-TV}

Störningar från radiosändare kan yttra sig till exempel på följande sätt för
digital TV:

Sändningar, främst på 2-meter men även på 70-cm, kan orsaka blockering och
bildstörningar vid mottagning av digital-TV.
TV-bilden tappar då information, det blir pixlingar (fyrkantiga rutor), grönt
brus eller hela bilden fryses eller försvinner kortvarigt.
För analog TV i till exempel kabel-TV-nät kan störningarna yttra sig på
följande sätt:

\begin{itemize}
\item Vid sändning av amplitudmodulerade signaler, till exempel AM och SSB,
  uppstår ljudförvrängning i ljudkanalen samt ränder med mera i bilden.
\item Vid sändning av FM och CW uppstår ljudstörningar samt
  kontrastvariationer, interferensmönster (moire-effekter) med mera i bilden.
\end{itemize}

Problem med störningar av den här typen har minskat betydligt sedan digital-TV
infördes och de flesta TV-sändningar numera sker på VHF- och UHF-banden.

Störningar i TV som orsakas av sändare på lägre frekvenser kan i många fall
avhjälpas med frekvensfilter.
Ett lågpassfilter efter en kortvågsändare kan till exempel dimensioneras att
endast släppa igenom signaler under cirka 35~MHz. Läs mer om lågpassfilter i
kapitel \ref{Lågpassfilter}.

Ett högpassfilter före en TV-mottagare kan till exempel dimensioneras att
endast släppa igenom signaler med frekvenser över cirka 35~MHz. Läs mer om
högpassfilter i kapitel \ref{Högpassfilter}.

Om inte mottagning i TV-band I och II är av intresse, så kan ett
högpassfilter med en gränsfrekvens av cirka 160~MHz sättas in.
Det dämpar den oönskade utstrålning från sändare i HF- och lägre VHF-området,
det vill säga upp till och med 144--146~MHz amatörband.
Däremot släpps TV-band III (174--230~MHz) och TV-banden IV och V igenom
(470--890~MHz).

Ytterligare avstörningsmedel kan sättas in om det uppstår störningar av
amatörradiosändningar.
Det kan vara skärmströmsfilter på antennkablar, bandspärrar samt sug- och
spärrkretsar avstämda till störfrekvensen, bandpassfilter avstämt till
nyttofrekvensen. Läs mer om filter i kapitel \ref{spärrfilter}.

Ett vanligt störningsfall är att en bredbandig antennförstärkare blir
överstyrd eller blockerad av en sändare. Se även kapitel \ref{blockering}.

\begin{itemize}
\item Försök att undvika antennförstärkare.
\item Försök att undvika dåligt skärmade skarvar och antennkontakter.
\item Skaffa en bättre TV-antenn som även kan ta emot TV-sändningar på VHF.
  Många hushåll har idag endast en UHF-antenn och har därför dålig
  antennsignal på VHF-bandet där sändningar för HD-TV sker i många områden.
\end{itemize}

\subsection{Störningar på LF-apparater}
\index{störning!LF-apparater}

Störningar av HF-instrålning i ljudbandspelare, LF-förstärkare, telefonapparater
etc. kan ofta stoppas med avkopplingskondensatorer och HF-drosslar.
Moderna avstörningsdrosslar innehåller oftast något ferritmaterial i form av
rör, stavar eller ringar.
