\section{Störningsorsaker}

\subsection{Störningar från sändare}
\textbf{
HAREC a.\ref{HAREC.a.9.2.1}\label{myHAREC.a.9.2.1},
 a.\ref{HAREC.a.9.2.2}\label{myHAREC.a.9.2.2}
}

HF-förstärkare, till exempel i sändarslutsteg, kan komma i oönskad självsvängning,
vilket kan uppstå av flera orsaker; det kan vara bristande avkoppling av
matningsspänningar, induktiv och/eller kapacitiv återkoppling etc.

Effektförstärkare kan även överstyras.
Då uppstår intermodulation och övertoner som strålas ut på oönskade frekvenser.

I många fall kan störningar undvikas med en eller flera av följande åtgärder:
\begin{itemize}
\item Undvik att överstyra sändarslutsteget (kontrolleras till exempel med
  ALC-mätaren).
\item Förse sändarutgången med lågpassfilter.
  På så sätt undertrycks övertoner.
\item Anpassa sändarens och antennanläggningens impedanser till varandra.
  Stäm av sändarens \(\pi\)-filter och/eller en separat antennanpassningsenhet
  rätt.
  En felinställd sändare kan medföra oavsiktligt utstrålning.
\item Koppla in balanseringsnät (balun) mellan osymmetriska antennledningar
  (koaxialkablar) och symmetriska antenner.
\item Placera antennen högt och fritt och så långt från personer och
  störningskänslig utrustning som möjligt.
  Fältstyrkan är nämligen högst närmast antennen.
  Se kapitel \ref{EMF} om elektriska fält.
\item Undvik direkt HF-instrålning på belysningsnätet genom att använda
  nätfilter.
\item Använd ''mjuk'' nyckling av bärvågen (avrundade telegrafitecken).
  Vid hård nyckling alstras övertoner i form av knäppar som hörs långt vid
  sidan av sändningsfrekvensen.
\end{itemize}

\subsection{Störningar på radiomottagning}
\textbf{
HAREC a.\ref{HAREC.a.9.2.3.1}\label{myHAREC.a.9.2.3.1},
 a.\ref{HAREC.a.9.2.3.2}\label{myHAREC.a.9.2.3.2},
 a.\ref{HAREC.a.9.2.3.3}\label{myHAREC.a.9.2.3.3}
}

I regel uppstår störningar på radiomottagning först när utstrålade signaler
uppnått en viss styrka -- immunitetsnivån för HF.

Man kan tala om tre slags HF-immunitet hos mottagare:
\begin{itemize}
\item mot signaler genom antenningången
\item mot signaler genom övriga anslutna ledningar, till exempel högtalar-
  och nätledningar
\item mot elektriska och/eller magnetiska fält som strålar direkt in i
  apparaten.
\end{itemize}

I de båda första fallen kan det hjälpa med komplettering med hög- och/eller
lågpassfilter och skärmströmsfilter.

Störningar orsakade av instrålning är svårast att avhjälpa och fordrar ingrepp i
mottagaren, vilket bör överlåtas till en fackman med tillgång till
tillverkarens serviceinstruktioner.

\subsection{Störningar på TV-mottagning}

Störningar från radiosändare kan yttra sig till exempel på följande sätt:
\begin{itemize}
  uppstår ljudförvrängning i ljudkanalen samt ränder m.m. i bilden,
\item Vid sändning av amplitudmodulerade signaler, till exempel AM och SSB,
\item Vid sändning av FM och CW uppstår ljudstörningar samt
  kontrastvariationer, interferensmönster (moire-effekter) m.m. i bilden.
\item Sändningar, främst på 2-meter men även på 70-cm, kan orsaka blockering och
  bildstörningar vid mottagning av digital-TV.
\end{itemize}

Störningar i TV som orsakas av sändare på lägre frekvenser kan i många fall
avhjälpas med frekvensfilter.
Det kan t.ex. uppstå TV-störningar, när en amatörradiostation sänder på
21~MHz-bandet.
Dess 3:e överton hamnar då på TV-kanal E-4 (61--68~MHz).
Problem med störningar av den här typen har minskat betydligt sedan digital-TV
infördes och de flesta TV-sändningar numera sker på VHF- och UHF-banden.

Ett lågpassfilter efter en kortvågsändare kan t.ex. dimensioneras att endast
släppa igenom signaler under ca 35~MHz.

Ett högpassfilter före en TV-mottagare kan t.ex. dimensioneras att endast släppa
igenom signaler med frekvenser över ca 35~MHz.

Om inte mottagning i TV-band I (40--68~MHz) är av intresse, så kan ett
högpassfilter med en gränsfrekvens av ca 160~MHz sättas in.
Det dämpar den oönskade utstrålning från sändare i HF- och VHF-området,
dvs. upp t.o.m. 144--146~MHz amatörband.
Däremot släpps TV-band III (174--230~MHz) och TV-banden IV och V igenom
(470--890~MHz).

Ytterligare avstörningsmedel kan sättas in om det uppstår störningar av
amatörradiosändningar.
Det kan vara skärmströmsfilter på antennkablar, bandspärrar samt sug- och
spärrkretsar avstämda till störfrekvensen, bandpassfilter avstämt till
nyttofrekvensen.

Ett vanligt störningsfall är att en dåligt skärmad och bredbandig
antennförstärkare blir överstyrd av starka sändare.

\subsection{Störningar på LF-apparater}

Störningar av HF-instrålning i ljudbandspelare, LF-förstärkare, telefonapparater
etc. kan ofta stoppas med avkopplingskondensatorer och HF-drosslar.
Moderna avstörningsdrosslar innehåller oftast något ferritmaterial i form av
rör, stavar eller ringar.
