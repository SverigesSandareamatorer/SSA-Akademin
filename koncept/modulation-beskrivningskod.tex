\section{Beskrivningskod för sändningsslagen}
\index{sändningsslag}
\label{modulation_beskrivningskod}

Vid 1979~års radioförvaltningskonferens (WARC~79) i Geneve reviderades det
internationella radioreglementet (RR), som i huvudsak trädde i kraft 1982.
Däri ingår bland annat ett nytt system för klassindelning och beteckning av
sätten att utsända information över radio med mera.
Reglementet har reviderats senare, men i detta stycke gäller det ännu.

Indelningen i \emph{sändningsslag} behövs för att känneteckna utsändningarna,
till exempel i frekvenslistor, författningar och föreskrifter.
Indelningen är också av stort värde vid teknisk beskrivning av apparater och
system för radiokommunikation.

Emellertid används av många även äldre benämningar, vilka lever kvar i
litteraturen, i märkning av manöverdonen på sändare och mottagare.

Dessa äldre benämningar är dock inte entydiga och skapar lätt missförstånd,
varför beskrivningskoden enligt WARC~79 bör användas för tydlighetens skull.

Här följer avkortade koder enligt WARC~79 för några av de sändningsslag som
amatörer använder mest, samt för jämförelse även de benämningar som fortfarande
används jämsides (se vidare i bilaga~\ssaref{saendslag}).

\mediumfig[0.67]{images/cropped_pdfs/bild_2_1-23.pdf}{Modulerande signaler}{fig:BildII1-23}

\begin{description}
\item[NON] Bärvåg utan modulerande signal. Ingen information.

\item[A1A] Bärvåg med dubbla sidband. En enda kanal med kvantiserad bärvåg.
Ingen modulerande underbärvåg. Telegrafi. Även kallat nycklad bärvåg (CW).

\item[A3E] Linjärt modulerad huvudbärvåg. Dubbla sidband. En enda kanal med
analog information. Telefoni. Även kallat amplitudmodulation (AM).

\item[J3E] Linjärt modulerad huvudbärvåg. Ett sidband med undertryckt bärvåg.
  En enda kanal med analog information. Telefoni.
  Även kallat enkelt sidband, Single Side Band (SSB).

\item[F3E] Vinkelmodulerad bärvåg. Frekvensmodulering. En enda kanal med analog
information. Telefoni. Även kallat frekvensmodulering (FM).

\item[G3E] Vinkelmodulerad bärvåg. Fasmodulering. En enda kanal med analog
information. Telefoni. Även kallat fasmodulering (PM).
\end{description}

Såväl A1A, A3E som J3E är sändningsslag där amplituden moduleras.
Därför är termen \emph{amplitudmodulation} inte tillräcklig för att beskriva
flera likartade sändningsslag.
