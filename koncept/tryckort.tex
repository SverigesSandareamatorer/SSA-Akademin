%% Tryckortssidan, denna behöver uppdateras med ny information så småningom.
\onecolumn\newgeometry{left=4cm,right=4cm}
\pagenumbering{gobble}
\vspace{10em}
\title{KonCEPT för amatörradiocertifikat}
\begin{center}
\Large{KonCEPT FÖR AMATÖRRADIOCERTIFIKAT}

Föreningen Sveriges Sändareamatörer\\[2\baselineskip]
\end{center}

\noindent \textbf{Andra upplagan, femte tryckningen}

\noindent ISBN: 987-91-86368-23-4

\noindent
\\
\noindent Det här verket är licenserat under Creative Commons:\newline
\noindent Erkännande, Icke kommersiell, Dela lika\\
\noindent (CC BY-NC-SA) 4.0 Internationell.\\
\\
\\
\noindent Version \revision

\begin{figure}[h]
    \includegraphics[width=4cm]{images/cc-by-nc-sa}
\end{figure}

%Tryckt i Sverige / Printed in Sweden 1997
%Smegraf, Smedjebacken

\vfill

\noindent Denna faktabok omfattar det av Post- och tele\-styrel\-sen anvisade
kompetensområdet för amatörradiocertifikat.\\

\noindent Innehållet är delat i två ämnesgrupper: grundläggande radioteknik samt regler och trafikmetoder.
Det finns även inlärningsanvisningar för morsesignalering för den
som vill lära sig telegrafi. \\

\noindent I bilagorna finns bland annat grundläggande matematik
och frekvensplaner för amatörradiotrafik. 

\vfill

\noindent
\textbf{Förlag}

\noindent
\textit{Föreningen Sveriges Sändareamatörer (SSA)}\\
Box 45, SE-191 21 Sollentuna\\
Telefon +46 8 585 702 73\\
E-post \href{mailto:hq@ssa.se}{hq@ssa.se}\\[\baselineskip]
\restoregeometry\twocolumn
\pagenumbering{arabic}
