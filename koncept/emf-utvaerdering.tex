\section{Utvärdering av EMF}
\index{EMF!utvärdering}

För att kunna utvärdera att den egna radiostationen vid sändning ger
elektromagnetiska fält som understiger referensvärdena behöver man känna till
de parametrar som är avgörande för styrkan på det elektromagnetiska fältet:

\begin{itemize}
  \item Antennens förstärkning (G).
  \item Sändningens medeleffekt (P).
  \item Transmissionsledningens förluster (k).
  \item Distansen (d).
\end{itemize}

\subsection{Antennen}
Antennen tar emot signalen från sändaren via en matningskabel och omvandlar
denna signal till ett elektromagnetiskt fält.
Hur effektivt antennen omvandlar signalen från sändaren kan enklast förklaras
med begreppen förstärkning eller antennvinst.

Man måste alltså känna till vilken förstärkning antennen har uttryckt i linjära
faktorer i förhållande till en isotrop antenn.

Antennförstärkning i förhållande till en isotrop antenn uttrycks vanligen i dBi.
Detta medför att en vanlig dipolantenn som används som referens för 0\,dBd har
en förstärkning på 2,15\,dBi jämfört med en isotrop antenn.

Alla värden på antennförstärkning uttryckt i dBd ska därför ökas med 2,15 för
att kunna användas i tabell~\ssaref{tab:forst} som visar förhållandet mellan
förstärkning i \unit{\decibel} och linjära faktorer.

\begin{table*}[ht]
  \begin{center}
    \begin{tabular}{|l|ccccccccccc|}
    \hline
    dB     &  0  &  1  &  2 & 2,15 &  3  &  4  &  5  &  6  &  7  &  8  &  9  \\ \hline
    G & 1,0 & 1,3 & 1,6 & 1,64 & 2,0 & 2,5 & 3,2 & 4,0 & 5,0 & 6,3 & 7,9 \\ \hline\hline
    dB     &  10  &  11  &  12  &  13  &  14  &  15  &  16  &  17  &  18  &  19  &  20 \\ \hline
    G & 10,0 & 12,6 & 15,8 & 20,0 & 25,1 & 31,6 & 39,8 & 50,1 & 63,1 & 79,4 & 100,0 \\ \hline
    \end{tabular}
    \caption{G = Antennens förstärkning i linjära faktorer}
    \label{tab:forst}
  \end{center}
\end{table*}

För en antenn med förstärkningen 7\,dBi ska alltså värdet 5,0 användas.

\subsection{Sändareffekten}
Alla SAR-värden ska beräknas som ett medelvärde under en period av sex minuter.
För att kunna utföra en beräkning av effektens medelvärde behövs utöver
PEP-effekt kännedom om de två faktorer som påverkar medeleffekten.
Faktorerna har därför betydelse för nivån på det elektromagnetiska fältet och
påverkar därigenom den genomsnittliga exponeringen för EMF.

\subsubsection{Modulationsfaktor}
\index{EMF!modulationsfaktor}
\index{modulationsfaktor}

Beroende på trafiksätt så blir medeleffekten olika.
Används FM så medför det modulationssättet att man använder max uteffekt
kontinuerligt jämfört med SSB där medeleffekten beror på hur man talar.

Tabell~\ssaref{tab:modfakt} ger de faktorer som enligt OET bulletin 65 supplement b,
\cite{OETbul65b} används i USA för att räkna ut medeleffekten på grund
av modulationsfaktorn.

\begin{table}[H]
  \begin{center}
    \begin{tabular}{lc}
	\textbf{Trafiksätt} & \textbf{Modulationsfaktor} \\ 
	\hline
	SSB & 0,2 \\
	CW & 0,4 \\
	SSB med processing & 0,5 \\
	FM & 1,0 \\
	MGM (t.ex. RTTY, PSK) & 1,0 \\
	Bärvåg & 1,0 \\
    \end{tabular}
    \caption{Modulationsfaktor per trafiksätt}
    \label{tab:modfakt}
  \end{center}
\end{table}

\subsubsection{Intermittensfaktor}
\index{EMF!intermittensfaktor}
\index{intermittensfaktor}

Vid vanlig amatörradioanvändning sänder man inte kontinuerligt då växling
mellan sändning och lyssning sker regelbundet.
Sänder man och tar emot lika mycket under en sexminutersperiod så blir faktorn
0,5 men om man lyssnar mycket mer och sänder sällan blir faktorn mindre.
Se tabell~\ssaref{tab:intfakt} för fler exempel.

\begin{table}[H]
  \begin{center}
    \begin{tabular}{|c|c|c|}
	\hline
	Sändning  & Mottagning & Intermittensfaktor \\
	(minuter) & (minuter)  & \\ \hline
	1 & 5 & 0,17 \\ \hline
	2 & 4 & 0,33 \\ \hline
	3 & 3 & 0,50 \\ \hline
	4 & 2 & 0,67 \\ \hline
	5 & 1 & 0,83 \\ \hline
	6 & 0 & 1,00 \\ \hline
    \end{tabular}
    \caption{Intermittensfaktor}
    \label{tab:intfakt}
  \end{center}
\end{table}

\subsubsection{Medeleffekt}
\index{EMF!medeleffekt}

För att räkna ut vilken medeleffekt som används ska man ta hänsyn
till både modulationsfaktor och intermittensfaktor enligt följande

\(\textit{Medeleffekt} = \textit{Maxeffekten} \cdot \textit{Modulationsfaktor} \cdot \textit{Intermittensfaktor}\)

\noindent\textbf{P = Medeleffekten under en sexminutersperiod}

\subsection{Kabeldämpning}
\index{EMF!kabeldämpning}

När uteffekten mäts vid sändaren och fältet genereras av effekten som
når antennen måste även den dämpning som matarledaren har vara känd.
Annars överskattas den genererade fältstyrkan.

Även här måste linjära faktorer användas.
Förlusterna i en kabel har negativa värden uttryckt i \unit{\decibel} vilket
medför att faktorerna i tabell~\ssaref{tab:feedannut} blir mindre än ett.

\begin{table*}[ht]
  \begin{center}
    \begin{tabular}{|l|c|c|c|c|c|c|c|c|c|c|c|}
	\hline
	dB & 0,0  & 0,5  & 1,0  & 1,5  & 2,0  & 2,5  & 3,0  & 3,5  & 4,0  & 4,5  & 5,0 \\ \hline
	k  & 1,00 & 0,89 & 0,79 & 0,71 & 0,63 & 0,56 & 0,50 & 0,45 & 0,40 & 0,35 & 0,32 \\ \hline
    \end{tabular}
    \caption{k = Matarkabels dämpning i linjära termer}
    \label{tab:feedannut}
  \end{center}
\end{table*}

För en kabel med dämpningen \qty{2,5}{\decibel} ska alltså värdet 0,56 användas.

\subsection{Distans}
\index{EMF!distans}

För att kunna beräkna nivån på det elektromagnetiska fältet på en utvald plats
behöver man veta distansen till den sändande antennen.

Enligt Strålsäkerhetsmyndighetens allmänna råd så bör inte referensvärdena
överskridas på platser där allmänheten vistas.
En bedömning bör därför göras över distanserna från den sändande antennen till
platser allmänheten riskerar att exponeras för elektromagnetiska fält.
\\[1ex] % layout
\noindent\textbf{d = Distansen från antennen till platsen där fältstyrkan ska bestämmas}

\subsection{Beräkning}
\index{EMF!beräkning}

Beräkning av det elektromagnetiska fältet kan med enkelhet bara
genomföras i fjärrfältet från en antenn.
I fjärrfältet vet vi sedan tidigare att man antingen kan utvärdera det
elektriska eller det magnetiska fältet.
Av denna anledning beskrivs här enbart beräkning av det elektriska fältets del av
det elektromagnetiska fältet.
Ett vedertaget avstånd från antennen där fjärrfältsberäkningar kan genomföras är
\(d=\lambda / 6\).
Se tabell~\ssaref{tab:fjfltgr}.

Följande formler gäller enbart för beräkning av korrekt fältstyrka i
fjärrfältet men kan för enklare antenner användas för att uppskatta den
fältstyrka som uppträder i närfältet.

%% k7per: Where is this table referenced?
\begin{table*}[ht]
  \begin{center}
    \begin{tabular}{|l|c|c|c|c|c|c|c|c|c|c|}
	\hline
	Band [m] & 160 & 80 & 40 & 30 & 20 & 17 & 15 & 12 & 10 & 6 \\ \hline
	Fjärrfältsgräns [m] & 27 & 13 & 6,7 & 5 & 3,3 & 2,8 & 2,5 & 2 & 1,7 & 1 \\ \hline
    \end{tabular}
    \caption{Fjärrfältsgräns per band}
    \label{tab:fjfltgr}
  \end{center}
\end{table*}

\noindent\textbf{E = Det elektromagnetiska fältets storlek i fjärrfältet}

Det elektromagnetiska fältets storlek (i fjärrfältet) räknas ut från
effekten (medelvärde), antennförstärkningen, matarledningens dämpning
och avståndet enligt följande förenklade formel.
%%
\[E=\dfrac{\sqrt{30 \cdot P \cdot G \cdot k}}{d}\]
%%
Genom enkel matematik kan man då använda samma formel för att räkna
ut på vilket avstånd man genererar en viss fältstyrka.
%%
\[d=\dfrac{\sqrt{30 \cdot P \cdot G \cdot k}}{E}\]
%%
Denna beräkning är enbart relevant för huvudloben.
Fältet under antennen beräknas inte, och därför kan resultatet inte användas
för att bedöma höjd på eller säkerhetsavstånd till antenntorn.
Använd datorprogram för att få bra bedömning på hur en antenn beter sig,
särskilt med avseende på antenner med riktverkan.

\begin{exempelbox}
\textbf{Exempel 1} \\
En riktantenn för \qty{144}{\mega\hertz} med förstärkning enligt databladet på
14,92\,dBi (31 gånger).
Max uteffekt är \qty{1000}{\watt} och trafiksättet är MGM (t.ex. RTTY, PSK) med
30~sekunders intervaller.
Den valda matarledningen har en dämpning på \qty{2,5}{\decibel} (0,56~gånger).
Avståndet från antennen till beräkningspunkten är \qty{15}{\metre}.
Vilken medelfältstyrka genererar man på ett visst avstånd från antennen?
\tcblower
\noindent
\[P_{medel} = P_{pep} \cdot k_{mod} \cdot k_{if}
= 1000 \cdot 1 \cdot 0,5 = \qty{500}{\watt}\]
\[k_{mod} = modulationsfaktor\]
\[k_{if} = intermittensfaktor\]
\[G = 31 \quad k = 0,56 \quad d = 15\]
% \[E = \dfrac{\sqrt{30 \cdot P \cdot G \cdot k}}{d}
% = \dfrac{\sqrt{30 \cdot 500 \cdot 31 \cdot 0,56}}{15}
% = \qty{34,02}{\volt\per\metre}\]
\begin{align*}
  E &= \dfrac{\sqrt{30 \cdot P \cdot G \cdot k}}{d} =\\
&= \dfrac{\sqrt{30 \cdot 500 \cdot 31 \cdot 0,56}}{15}
= \qty{34,02}{\volt\per\metre}
\end{align*}

Då referensvärdet på denna frekvens är \qty{28}{\volt\per\metre}, överskrider
amatörradiosändningen referensvärdet på detta avstånd.
\end{exempelbox}

\begin{exempelbox}
\textbf{Exempel 2} \\
En riktantenn för \qty{144}{\mega\hertz} med förstärkning enligt databladet på
14,92\,dBi (31 gånger).
Max uteffekt är \qty{1000}{\watt} och trafiksättet är MGM (t.ex. RTTY, PSK) med
30~sekunders intervaller.
Den valda matarledningen har en dämpning på \qty{2,5}{\decibel} (0,56~gånger).
Referensvärdet för \qty{144}{\mega\hertz} är \qty{28}{\volt\per\metre}.
På vilket avstånd från antennen når man referensvärdet?
\tcblower
\noindent
\[P_{medel} = P_{pep} \cdot k_{mod} \cdot k_{if}
= 1000 \cdot 1 \cdot 0,5 = \qty{500}{\watt}\]
\[k_{mod} = \textit{modulationsfaktor}\]
\[k_{if} = \textit{intermittensfaktor}\]
\[G = 31 \quad k = 0,56 \quad E = 28\]
%% \[d = \dfrac{\sqrt{30 \cdot P \cdot G \cdot k}}{d}
%% = \dfrac{\sqrt{30 \cdot 500 \cdot 31 \cdot 0,56}}{28}
%% = \qty{18,22}{\metre}\]
\begin{align*}
  d &= \dfrac{\sqrt{30 \cdot P \cdot G \cdot k}}{d} =\\
&= \dfrac{\sqrt{30 \cdot 500 \cdot 31 \cdot 0,56}}{28}
  = \qty{18,22}{\metre}
  \end{align*}

För att följa de allmänna råden bör allmänheten inte kunna vistas i huvudloben
framför antennen på ett avstånd mindre än \qty{19}{\metre} då sändning utförs
enligt exemplet.
\end{exempelbox}

\begin{exempelbox}
\textbf{Exempel 3} \\
En dipolantenn för \qty{3,75}{\mega\hertz} har jämfört med en isotrop antenn
förstärkningen 2,15\,dBi (cirka 1,6 gånger).
Max uteffekt är \qty{100}{\watt} och trafiksättet är SSB med normala TX/RX
intervaller.
Den valda matarledningen har en dämpning på \qty{0,5}{\decibel} (0,89 gånger).
Referensvärdet för \qty{3,75}{\mega\hertz} är \qty{45}{\volt\per\metre}.
På vilket avstånd från antennen når man referensvärdet?
\tcblower
\noindent
\[P_{medel} = P_{pep} \cdot k_{mod} \cdot k_{if}
= 100 \cdot 0,5 \cdot 0,5 = \qty{25}{\watt}\]
\[k_{mod} = modulationsfaktor\]
\[k_{if} = intermittensfaktor\]
\[G = 1,6 \quad k = 0,89 \quad E = 45\]
\[d = \dfrac{\sqrt{30 \cdot P \cdot G \cdot k}}{E} = \dfrac{\sqrt{30 \cdot 25 \cdot 1,6 \cdot 0,89}}{45}
= \qty{0,74}{\metre}\]

Här konstaterar vi att det uträknade avståndet ligger i närfältet från antennen
(inom 13~meter).
En dipol är en enklare antenntyp så vi kan anta att värdet är användbart för att
kunna utvärdera exponeringen.

För att följa de allmänna råden bör människor inte ha tillträde till nån del av
antennen närmare än \qty{0,74}{\metre} då sändning utförs.

Om man i exemplet höjer uteffekten till \qty{1000}{\watt} blir avståndet 2,3
meter.
\end{exempelbox}
