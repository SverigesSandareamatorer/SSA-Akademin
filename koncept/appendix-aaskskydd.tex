\chapter{Att skydda sin amatörradiostation mot blixten}
\label{app:aaskskydd}

Ursprungliga författare: Rolf och Bengt Högberg.
Ursprungligen publicerad i tidningen QTC~1992:4.
Uppdaterad för KonCEPT år~2025.

\section{Inledning}
De radioutrustningar som användes av sändaramatörer på 1960-talet och tidigare
var ofta hembyggda och rörbestyckade.
Komponenterna som användes var tåliga mot överspänningar och gick de sönder, var
det i regel inga stora kostnader för att få tag på nya komponenter.
Reparationerna kunde oftast göras av sändaramatören själv.
Å andra sidan kom ofta elmatning och tele in via luftledningar (landsbygden),
vilket kunde hjälpa till att leda in åsköverspänningar till husen och därmed
till radiostationerna.

De utrustningar som används av sändaramatörerna i dag är synnerligen
sofistikerade, ofta mikroprocessorbaserade, och med en komponenttäthet som många
amatörer knappt vågar titta på, än mindre stoppa in lödkolven i.
Även om mycket har gjorts av tillverkarna vad gäller skydd mot atmosfäriska
överspänningar blir reparationskostnaderna, vid t.ex. ett slutstegshaveri,
synnerligen kostbara och relativt få amatörer har kunskap eller utrustning för
att göra jobbet själva.
Det finns alltså goda skäl för att försöka skydda sin utrustning (och kanske
även livhanken) mot blixten.

\mediumfigpad{images/aaskskydd-1.png}{Skydd av station och antenn.}{fig:aaska-skydd-antenn}

\section{Bakgrund}
Om vi med direktträff av blixten menar att blixtkanalen ska direkt träffa din
antenn eller byggnaden där din utrustning finns så är sannolikheten för detta
liten.
Man brukar räkna med att statistiskt sett så sker ett direktnedslag ca 0,2--0,3
gånger per \unit{\kilo\metre\squared} och år vilket för ''medelradioamatören''
innebär ett direkt blixtnedslag mindre än en gång vart hundrade år.
Betydligt mera sannolikt är att blixtens verkningar når din anläggning via de
ledningar som försörjer ditt hus med el eller tele endera genom att dessa
ledningar fungerar som antenner och plockar upp en kraftig störimpuls genom
induktion eller genom att ledningarna själva träffas av blixten.
Även i dina antenner kan givetvis kraftiga störningar induceras.
De är ju avsedda att plocka upp radiosignaler och blixtens frekvensspektrum
sträcker sig från \unit{\kilo\hertz} upp till över \qty{10}{\giga\hertz} med ett
maximum vid ca \qty{10}{\kilo\hertz}.
Faktum är att den radiosignal som sänds ut när blixten går från moln till mark
används för att pejla blixtens nedslagspunkt med hjälp av minst två mätstationer
och att signalen från en kraftig blixturladdning kan vara så kraftig att
samtliga pejlstationer i Sverige registrerar mätbara signaler från en och samma
blixt.
Dessa blixtpejlar använder sig i första hand av frekvensområdet upp till ca
\qty{0,5}{\mega\hertz}.
De strömmar som kan förekomma varierar från ca \qty{1}{\kilo\ampere} till flera
hundra \unit{\kilo\ampere}.
Med ca \qty{100}{\mega\volt} spänning i molnet och något hundratal
\unit{\kilo\ampere} ström i blixtkanalen kan man förstå att den
''sändareffekten'' räcker för att nå rätt långt.

\section{Grundregler för åskskydd}
\subsection{Placering av antenner}
Placera om möjligt antennerna i en fristående mast.
Detta förenklar skyddet av ditt hus som i annat fall måste skyddas av en
byggnadsåskledare, utförd t.ex. enligt den svenska åskskyddsstandarden
SS-EN~62305-1.
Båda dessa standarder kan beställas från SIS i Stockholm, https://www.sis.se.
Om man nöjer sig med att skydda mot inducerade effekter (ej direktinslag) kan
ett enklare skydd anordnas, se nedan.
Observera att skyddsåtgärderna enligt dessa standarder syftar till att förhindra
personskada och brand men inte garanterar att utrustningen överlever ett
blixtnedslag.

\mediumfigpad{images/aaskskydd-2.png}{Jordning av antenn och station.}{fig:aaska-jordning}

\subsection{Placering av din station}
Din station kan lättast skyddas mot blixtens verkningar om den ligger lågt i
byggnaden och därigenom ligger nära jordtaget.
Källaren eller bottenvåningen nära inkommande el är alltså bäst.
Det blir något svårare att skydda stationen effektivt ju högre upp i byggnaden
(och därigenom längre från huvudelcentralen) stationen är installerad.
I princip beror detta på att man vill göra spänningsutjämning mot inkommande
nolla i huvudelcentralen.
Sitter antennen mastmonterad på taket fordras dessutom alltid normalt
byggnadsåskskydd enligt ovanstående normer om man vill skydda byggnaden vid
direkta blixtnedslag.

\subsection{Avled blixtströmmen till jord nära det träffade objektet}
För att minska skadorna vid blixtnedslag i en antenn oavsett om den sitter
monterad i separat mast eller på en byggnad så gäller samma grundregel: Avled så
stor del av blixtströmmen som möjligt nära inslagspunkten!
Sitter antennen monterad på en fristående metallmast kan en stor del av strömmen
avledas genom att ansluta jordlinor, \qty{25}{\milli\metre\squared} 7-trådiga
glödgade kopparlinor, till masten ungefär i marknivå.
Linorna dras ut från masten i olika riktningar och avslutas med vertikala
jordspett till 2--4~meter djup eller helst mera.
Om antennen sitter monterad direkt på en villa eller ett hyreshus bör byggnaden
förses med normalt åskskydd enligt tidigare nämnda standarder vilket innebär att
jordningen normalt utgörs av en lina ca 0,5~meter ner i marken runt byggnaden,
en s.k. ringlina, någon meter ut från husgrunden.
Antennen ansluts till denna ringlina via de normala nedledare som
åskskyddsanläggningen har.
Principen är även här att sprida strömmen i olika riktningar, vilket minskar
induktansen.
Skulle marken vara dåligt elektriskt ledande kan det löna sig att till ringlinan
ansluta s.k. utlöpare, en eller flera \qty{25}{\milli\metre\squared}
åskskyddslinor i radiella riktningar ut från ringlinan.
Det lönar sig inte att ha alltför långa jordlinor eftersom
avledningseffektiviteten avtar med längden på linan.
Som tumregel kan man säga att längre jordlina än som motsvarar ca kvadratroten
ur markresistiviteten i meter räknat inte är verksamt.
För berg där resistiviteten är ca \qty{10000}{\watt\metre} innebär detta att det
kan högst kan vara lönsamt att gå upp till ca \qty{100}{\metre} från
anläggningen med en eller flera utlöpare för att hitta gynnsammare
jordningsförhållanden.
Av praktiska skäl och av kostnadsskäl får man i regel nöja sig med mindre vilket
innebär att övriga strömvägar (framför allt till elnätet) kommer att avleda en
något större andel av blixtströmmen än vad annars skulle varit fallet.
För pinnmo respektive lera är motsvarande längder \qtyrange{10}{30}{\metre}
respektive \qtyrange{5}{15}{\metre}.
Användning av vertikala jordningar, s.k. jordspett är ofta effektivare än
horisontella jordlinor.

När blixten träffar och strömpulsen stiger från noll upp mot sitt maximivärde
kommer strömmen att fördela sig ungefär lika på de vägar som finns tillgängliga.
Strömmen i masten, i varje jordtagsledare, i varje antennkabel, i
telefonledningen, i matningsledningen till din antennrotor och i inkommande
elledning kommer alltså under någon eller några mikrosekunder att fördela sig
relativt jämnt, nästan oberoende av ledningarnas area.
Strömdelningen styrs här av induktansen i de olika ledningarna.
Under denna tid när blixtströmmen växer till sitt maximivärde uppstår också de
största inducerade spänningarna.
När strömpulsen passerat sitt maximivärde kommer däremot resistanserna att styra
avledningen.
En låg total avledningsresistans till marken kan därför leda bort den stora
delen av den blixtström och den energi som matas in och minska strömmarna i
övriga kablar vilket i sin tur minskar risken för att utrustningen skadas.
Antennkabeln bör dras nedgrävd i marken från masten in till stationsplatsen.
Tätt intill koaxialkabel och ev. styrkabel (gärna hoptejpad med dessa) bör en
oisolerad \qty{25}{\milli\metre\squared} jordlina dras samma väg.
Detta minskar den s.k. kopplingsimpedansen, åtminstone för lägre frekvenser,
vilket t.ex. minskar störspänningen mellan innerledare och skärm och minskar
strömbelastningen på övriga kablar.

Kom också ihåg att se till att avledningsresistansen går att mäta, vilket endast
är möjligt om den del man vill mäta på kan kopplas loss från övriga delar så att
den är isolerad från övriga jordtag.
Detta innebär också att närliggande jordlinor i mark bör isoleras t.ex. med
polyetenrör så att inte överledning i mark stör mätningen.
Denna isolation görs på så lång sträcka att sträckan mellan oisolerade ledare i
mark blir \qty{1}{\metre} eller mera.
Mätning kan ske med en s.k. jordresistansmätbrygga.
Det är i princip önskvärt att få ett så lågt värde på avledningsresistansen som
möjligt.
Genom att marken joniseras runt jordtagen, speciellt för blixtar med stora
strömmar, kommer emellertid avledningsresistansen att minska till värden
betydligt lägre än den uppmätta avledningsresistansen.
För riktigt stora blixtströmmar kan avledningsresistansen minska ända till ca en
tiondel av det med jordresistansbrygga uppmätta värdet.
Där så är möjligt bör eventuell armerad betong i byggnaden utnyttjas för att
ytterligare förbättra stationens jordning.
Är husets bottensula av platsgjuten armerad betong väljer man en lämplig plats
nära radioutrustningen, borrar in till armeringen och ansluter minst tre
armeringsjärn till det övriga jordsystemet med hjälp av svetsning eller
klämkopplingar.
Hålet gjuts sedan igen.
Linan som anslutits till armeringen kan sedan dras närmaste lämpliga väg till
radioutrustningen.

Avledningsresistansen till jord för armerad betong kan approximativt beräknas
enligt följande:

\begin{exempelbox}
Om volymen av den betong under marknivå som är aktiv för att avleda
blixtströmmen är

V~[\unit{\cubic\metre}]

(endast betong i kontakt med marken räknas och alltså inte betong i
mellanväggar) är avledningsresistansen

\(R = r/D\)

där \(D = 1,57 (V)^{0,333}\) och r är markresistiviteten i \unit{\ohm\metre}.

för \qty{8}{\cubic\metre} betong är \(D = 1,57 \cdot 2 = 3,14\).
För r = \qty{200}{\ohm\metre} (torr åkermark) erhålls R = \qty{20}{\ohm}
\end{exempelbox}

\subsection{Skydda din stationsplats}
Installationen vid din radioutrustning bör vara sådan att blixtströmmen inte
passerar utrustningen utan ''shuntas'' förbi utrustningen.
Detta innebär att alla kablar inklusive åskskyddsjordar samt el- och andra
ledningar om möjligt bör föras in i byggnaden på ett gemensamt ställe och där
anslutas till en gemensam referenspunkt direkt eller via överspänningsskydd om
det finns signaler eller t.ex. nätspänning på någon ledare.
Helst bör denna gemensamma referenspunkt väljas mycket nära huvudelcentralen.
Att praktiskt genomföra detta för olika typer av installationer i villa,
lägenhet, höghus, i båt etc. har alla sina speciella svårigheter och
kompromisser är nästan alltid nödvändiga.
Principerna för skydd är dock relativt allmängiltiga varför du säkert själv kan
fundera vidare och hitta en lösning för just din station.
Kom emellertid ihåg att varje avsteg från ''bästa installation'' innebär att
skyddet försämras.
Några praktiska erfarenheter kan kanske vara på sin plats:

Om det nu inte går att anordna kablaget enligt konstens alla regler och
kompromisser alltså måste göras har det visat sig fungera bra att göra en
gemensam referenspunkt direkt vid din station.
En \qty{1}{\milli\metre} kopparplåt kan t.ex. placeras något under stationen och
så att den skjuter ut över bordskanten.
Alla inkommande kablar ansluts till plåten endera direkt eller via
överspänningsskydd.
Skärmen på koaxialkablarna ansluts till plåten via chassikontakter.
Mittledaren ansluts via gasurladdningsrör till plåten.
Hållare för skydd av andra än koaxialkablar tillverkar du lätt själv.
Gasurladdningsrörens statiska tändspänning (den spänning som ofta står tryckt på
rören) väljs så att rören ej tänder för den spänning som under normal drift kan
förekomma.
T.ex. om uteffekten på din sändare är \qty{1}{\kilo\watt} (\qty{50}{\ohm}
impedans) så är \(1000 = U^{2}/50\) vilket ger U = strax under \qty{320}{\volt}.
Eftersom gasurladdningsrör har en tolerans på \qty{+-20}{\percent} vilket ger
\(U_{tänd} \approx \qty{380}{\volt}\) väljs närmaste högre standardvärde
\qty{470}{\volt}.
Den statiska tändspänningen för gasurladdningsröret mellan mittledare och skärm
monterat i sändaränden bör alltså vara \qty{470}{\volt} vilket inte bör ge
obefogad tändning vid sändning.
Även utan skydd (bara skärmen jordad) så kommer kabelkontakterna att begränsa
överspänningarna till någonstans mellan ca \qtyrange{3}{8}{\kilo\volt} (beroende
på vilken kontakttyp som används).
Bäst är emellertid om skydd används vilken ger lägre påkänningar på din utrustning.

Kraftmatningen till sändaren skyddas förslagsvis med ventilavledare, helst i
huvudcentalen om den finns nära.
Finns huvudelcentralen mer än ca \qty{10}{\metre} bort från stationen kan
avledarna placeras i en undercentral.

\mediumfigpad{images/aaskskydd-3.png}{Principskiss för skydd av stationsplatsen.}{fig:aaska-skydd-station}

Observera att montering av ventilavledare ska utföras av behörig elektriker.

\section{Sammanfattande praktiska råd}

Sändaramatörer i villa eller sommarstugor med yagi-/quad- och trådantenner:
\begin{itemize}
    \item Skaffa bra jordtag anslutet till din antennmast och station!
    \item Använd om möjligt byggnadens armering som kan ge en effektiv minskning av anläggningens avledningsresistans! Även denna bör anslutas till stationens ''referens''.
    \item Använd en \qty{25}{\milli\metre\squared} jordledare in till ''referensen'' vid din station, gärna ''hopbuntad'' med antennkablarna som går in till huset (fristående mast)!
    \item Dra antennledningar från antenner på taket på husets utsida, och ej via ventilationkanaler eller dylikt (brandrisk)! Normalt byggnadsåskskydd erfordras om skydd ska anordnas för direkta blixtsträffar.
    \item Dra om möjligt in kablaget så att det passerar nära huvudelcentralen.
    \item Anslut ventilavledare till inkommande nolla i huvudelcentralen (elbehörighet erfordras)! Ventilavledare bör dessutom finnas i närmaste elcentral och vara anslutna mellan faser och skyddsjord om inte huvudcentralen är närmast.
    \item Använd överspänningsskydd på inkommande antennledningar!
    \item Bygg upp jordsystem och övrigt kablage stjärnformat med en gemensam lokal referenspunkt till vilken överspänningsskydden ansluts!
\end{itemize}

Sändaramatörer i lägenhetshus med trådantenn och ev. vertikal antenn (GP) på taket:
\begin{itemize}
    \item Använd så bra jord som går att åstadkomma och anslut din station. Eget ''riktigt jordtag'', armering, vattenledning, fjärrvärmeledningar, värmeledningssystem! Att åstadkomma ett tillfredsställande jordning i ett hyreshus är ofta det svåraste problemet att lösa.
    \item Dra antennledningar från antenner på taket på husets utsida, och ej via ventilationkanaler eller dylikt, se ovan!
    \item Installera ventilavledare vid elcentralen och gasurladdningsrör som skydd mot överspänningar från antenner och telenät, se ovan!
    \item Anordna en gemensam referens vid din station!
    \item Bygg upp jordsystem och övrigt kablage stjärnformat med en gemensam lokal referenspunkt till vilken överspänningsskydden ansluts!
    \item Sätt överspänningsskydd på inkommande antennledningar!
\end{itemize}

För i stort sett alla system oavsett om de är monterade på en villa eller
lägenhetshus eller om antennen har fristående mast eller är monterad på taket på
en byggnad gäller att för att vara mycket säker att blixten inte skadar din
station så bör du: helst dra ur alla kablar till radiostationen och jorda
antennledningarna när du är bortrest eller när åskan närmar sig.
