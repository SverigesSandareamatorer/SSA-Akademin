\section{Allmänna råd}
\index{EMF!allmänna~råd}

SSM har gett ut allmänna råd för begränsning av allmänhetens exponering
för elektromagnetiska fält SSMFS~2008:18~\cite{SSMFS2008:18}.
Syftet med råden är att skydda allmänheten från akuta
skadliga biologiska effekter vid exponering för elektromagnetiska fält.
I råden anges grundläggande begränsningar och härledda referensvärden.

\begin{quote}
	De grundläggande begränsningarna säkerställer att elektriska eller
	magnetiska fenomen som kan uppträda i kroppen inte stör funktioner i
	nervsystemet eller ger upphov till skadlig värmeutveckling.
\end{quote}

De grundläggande begränsningarna är, enligt internationella rekommendationer,
satta vid ungefär två procent av de nivåer vid vilka akuta biologiska effekter
är vetenskapligt säkerställda.

Från de grundläggande begränsningarna har härletts referensvärden som utgörs
av storheter som är mätbara utanför människokroppen.
Referensvärdena ska säkerställa att de grundläggande begränsningarna inte
överskrids.

\begin{quote}
	Om uppmätta värden överstiger referensvärdena, innebär detta inte nödvändigtvis
	att de grundläggande begränsningarna överskrids. I sådana fall gäller enligt
	dessa allmänna råd de grundläggande begränsningarna.
\end{quote}

I EU-direktivet 1999/519/EG~\cite{1999/519/EG} skrivs att i sådana fall skall det
göras en bedömning huruvida exponeringsnivån ligger under den grundläggande
begränsningen.

Referensvärdena i de allmänna råden bör inte överskridas i något område där
allmänheten kan vistas under sådana tider att begränsningarna är av betydelse.

\index{EMF!akuta biologiska effekter}
Det finns två huvudsakliga akuta biologiska effekter som kan förekomma vid
kraftig exponering för elektromagnetiska fält.
Fält med frekvens upp till cirka \qty{10}{\mega\hertz} kan om strömtätheten blir
hög i kroppen påverka det centrala nervsystemet.
Fält med frekvenser från \qty{100}{\kilo\hertz} till \qty{10}{\giga\hertz} kan
vid höga nivåer leda till en uppvärmning av kroppen.

\index{Specific Absorption Rate (SAR)}
\index{SAR}
När elektromagnetisk strålning absorberas i biologisk vävnad kan vävnaden värmas
upp.
Detta benämns ''Specific Absorption Rate'' (SAR) som mäts i enheten watt per
kilogram (\unit{\watt\per\kilo\gram}) eller milliwatt per gram
(\unit{\milli\watt\per\gram}).
SAR definieras som den energi, medelvärdesbildad över hela kroppen eller delar
av kroppen som absorberas per tidsenhet och per massenhet biologisk vävnad.

Då uppvärmningen av kroppsvävnad inte går snabbt räknar man med den medeleffekt
som under en viss tid orsakar uppvärmning.
För frekvenser mellan \qty{100}{\kilo\hertz} och \qty{10}{\giga\hertz} beräknas
SAR-värdet som medelvärdet under en sexminutersperiod.
För beräkning av SAR-värde på frekvenser överstigande \qty{10}{\giga\hertz}
hänvisas till formler för beräkning enligt SSMFS 2008:18.

Beroende på kroppens storlek i förhållande till det elektromagnetiska fältets
riktning och våglängd skapas resonansfenomen på grund av att kroppen fungerar
som en antenn.
Detta påverkar uppvärmningen på så sätt att vid frekvenser som är nära kroppens
eller kroppsdelens elektriska resonansfrekvens absorberas effekten lättare och
maximal uppvärmning uppstår.
Hos vuxna ligger denna resonansfrekvens mellan 70 och \qty{90}{\mega\hertz} om
personen står upp och är isolerad från något som kan jämföras med ett jordplan.
Även de olika kroppsdelarna kan vara resonanta.
En vuxen persons huvud är till exempel resonant vid cirka \qty{400}{\mega\hertz}.

Kroppens storlek avgör alltså vid vilken frekvens den absorberar mest effekt och
vid frekvenser över och under resonansfrekvensen så minskar uppvärmningen
orsakad av det elektromagnetiska fältet.

\index{EMF!referensvärden}
Referensvärdena tar hänsyn till detta faktum och det mest restriktiva
frekvensområdet ligger inom området 10 till \qty{400}{\mega\hertz} där effekt
lättast absorberas av kroppen.

I frekvensområdet 10 till \qty{110}{\mega\hertz} finns även en begränsning till
\qty{45}{\milli\ampere} för inducerad ström i varje extremitet i syfte att
begränsa det lokala SAR-värdet.

\mediumfig[0.87]{images/emfbild-000}{Referensvärden för begränsning av elektriska fält (100~kHz--10~GHz)}{fig:emf1}
\mediumfig[0.87]{images/emfbild-001}{Referensvärden för begränsning av magnetiska fält (100~kHz--10~GHz)}{fig:emf2}

Bild~\ssaref{fig:emf1} illustrerar referensvärden för begränsning av elektriska
fält på platser där allmänheten kan vistas (100~kHz--10~GHz), med amatörband
och fältstyrkenivå angivna, till exempel \qty{10,15}{\mega\hertz} har en högsta
tillåtna elektriskt fältstyrka på \qty{28}{\volt\per\metre}.

Bild~\ssaref{fig:emf2} illustrerar referensvärden för begränsning av magnetiska
fält på platser där allmänheten kan vistas (100~kHz--10~GHz), med amatörband
och fältstyrkenivå angivna, till exempel \qty{10,15}{\mega\hertz} har en högsta
tillåtna magnetisk fältstyrka på \qty{73}{\milli\ampere\per\metre}.
