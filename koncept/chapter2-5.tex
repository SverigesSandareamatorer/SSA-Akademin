\section{Halvledardioden}
\textbf{HAREC a.\ref{HAREC.a.2.5}\label{myHAREC.a.2.5}}
\index{halvledardiod}
\index{diod}

\subsection{Allmänt}
I en strömkrets kan av olika anledningar ström tillåtas att flyta i en riktning
men kanske inte i den motsatta. En anordning med en sådan funktion kallas för
diod.

Först bestod en diod av två elektroder i vakuum (se avsnitt
\ref{vakuumdioden}). Därav namnet vakuumdiod.
Numera består en diod oftast av någon halvledare. Därav namnet halvledardiod.

\begin{figure}
\includegraphics[width=\textwidth]{images/bild_2_2-12}
\caption{Spärrskiktet i en halvledardiod}
\label{fig:BildII2-12}
\end{figure}

Bild \ref{fig:BildII2-12} överst

En halvledardiod består av ett P-ledande och ett N-ledande materialskikt som
fogats samman.

Mellan de båda skikten utbildas ett tunt gränsskikt, som inte innehåller
laddningsbärare. Detta skikt kan vara ledande eller icke ledande - ett
spärrskikt - beroende på polariseringen.

\subsection{Halvledardiodens karaktär}
\textbf{HAREC a.\ref{HAREC.a.2.5.1.2}\label{myHAREC.a.2.5.1.2}}

\emph{Framström, temperatur, förlusteffekt, passriktning}

Bild \ref{fig:BildII2-12} mitten

Förbinder man den positiva polen på en spänningskälla med P-skiktet i en diod
och den negativa polen med N-skiktet så är dioden polariserad i passriktningen.
Spärrskiktet upplöses då och ström flyter genom dioden. Elektronerna flyter till
den positiva polen och hålen till den negativa polen.

\emph{Backspänning, backström, läckström, spärriktning}

Bild \ref{fig:BildII2-12} underst

Förbinder man i stället den negativa polen på en spänningskälla med P-skiktet i
en diod och den positiva polen med N-skiktet så är dioden polariserad i
spärriktningen. Spärrskiktet blir då ännu kraftigare.

Endast en obetydlig ström \(I_{SP}\) flyter genom dioden i den s.k.
spärriktningen även vid ökande spänning \(U_{SP}\). Men över en viss spänning
ökar strömmen snabbt - den s.k. zenereffekten uppstår. Dioden kan då lätt
förstöras av en alltför hög ström.

\begin{figure}
\includegraphics[width=\textwidth]{images/bild_2_2-13}
\caption{Halvledardiodens karaktäristik}
\label{fig:BildII2-13}
\end{figure}

Bild \ref{fig:BildII2-13}

Strömmen \(I_D\) börjar att flyta när spänningen \(U_D\) har nått ett
tröskelvärde (vid kiseldioder 0.6 V). När spänningen ökar ytterligare däröver,
så ökar även strömmen.

\begin{wrapfigure}[13]{R}{0.5\textwidth}
\includegraphics[width=0.5\textwidth]{images/bild_2_2-14}
\caption{Schemasymboler för dioder}
\label{fig:BildII2-14}
\end{wrapfigure}

Produkten av spänningsfallet över dioden och strömmen genom den kallas
förlusteffekt. Denna värmer upp dioden. Vid för hög temperatur förstörs
kristallstrukturen. En kiselkristall kan klara upp till 200 medan en
germaniumkristall klarar bara 75 °C.

Bild \ref{fig:BildII2-14}

\subsection{Diodtillämpningar}
\textbf{HAREC a.\ref{HAREC.a.2.5.1.1}\label{myHAREC.a.2.5.1.1}}

Likriktning är det vanligaste tillämpningen (se kapitel \ref{kraftaggregat}).
Halvledardioder görs även för en rad andra ändamål och finns i en mängd
utföranden, såsom

\subsubsection{Dioder för spänningsstabilisering (zenerdiod).}
\index{zenerdiod}
\index{diod!zener}

  Inom ett visst område är spänningsfallet över en zenerdiod i en strömkrets
  i det närmaste konstant medan strömmen varierar. Denna egenskap kallas
  zenereffekt och används för konstanthållning av spänning.

  Det finns zenerdioder många olika spänningar och effekter.

\subsubsection{Dioder som variabel kondensator, s.k. kapacitansdiod (VariCap).}
\label{varicap}
\index{varicap}
\index{diod!varicap}

  När en diod är polariserad i spärriktningen så bildas det ett spärrskikt.
  Olika polariseringsspänning alstrar olika tjocka spärrskikt En spärrad diod
  har på så sätt egenskaper som liknar dem i en variabel kondensator. Det finns
  därför dioder där reglerbarheten av kapacitansen är speciellt utvecklad.

\subsubsection{Lysdioder (LED).}
\index{lysdioder}
\index{diod!lysdiod}
\index{LED}
\index{diod!LED}

  Energi frigörs i spärrzonen i en diod som är polariserad i passriktningen. Det
  sker genom rekombination av par av laddningsbärare, varvid det normalt avgår
  energi i form av värme. Vid en viss inblandning av främmande atomer avgår
  istället ljus. Spänningfallet över en lysdiod är ungefär dubbelt så stort som
  över en kiseldiod, d.v.s. ungefär 1.5 volt. Strömmen är i proportion med
  önskad ljusstyrka och mellan 10 och 50 mA.

  \hilight{TODO:  här borde vi även ta upp moderna powerledar som kräver
  konstantströmsmatning, har betydligt högre framspänningsfall m..
  Dessa är ju billiga och populära för experiment. // Hans}

\subsection{Vakuumdioden jämfört halvledardioden}

\begin{figure}
\includegraphics[width=\textwidth]{images/bild_2_2-15}
\caption{Dioders polarisering i kretsen}
\label{fig:BildII2-15}
\end{figure}

Bild \ref{fig:BildII2-15}

Bilden visar principen för hur de båda diodtyperna ingår i en strömkrets. Den
stora skillnaden är att arbetsspänningen för en vakuumdiod är mångfalt högre än
den för en halvledardiod samt att vakuumdiodens ena elektrod (katoden) behöver
hettas upp för att avge elektroner.
