\section{Frekvens- och fasmodulation jämförs}

\begin{itemize}
\item Frekvensmodulation (FM) alstras genom att sändarens oscillatorfrekvens
  varieras (devieras) i takt med den modulerande signalen (t.ex. tal).
  Det gör man genom att variera resonansfrekvensen i den resonanskrets som
  styr oscillatorfrekvensen.

\item Fasmodulation (PM) alstras vanligen genom att efter sändaroscillatorn
  variera den modulerande signalens fasläge i förhållande till en omodulerad
  bärvåg -- så kallad fasmodulering.
  Det gör man genom att variera resonansfrekvensen i en resonanskrets efter
  oscillatorn, dvs. utan att påverka oscillatorfrekvensen.

\item I båda fallen ändrar man alltså resonansfrekvensen i en resonanskrets i
  takt med frekvensen i den modulerande spänningen, men denna krets har
  olika placering i FM-sändare respektive PM-sändare.

\item I sändaren alstras det i båda fallen utfrekvenser som devierar från
  oscillatorns vilofrekvens.
  Graden av deviation skiljer emellertid vid FM och PM.
  Vid FM är deviationen proportionell mot amplituden på den modulerande
  underbärvågen medan deviationen vid PM är proportionell mot produkten av den
  modulerande underbärvågens amplitud och frekvens.

\item Den hörbara skillnaden mellan FM och PM är därför en annorlunda
  frekvensgång.
  Vid samtidig användning av PM-sändare och FM-mottagare är det alltså lämpligt
  att justera frekvensgången i PM-sändarens modulator, lämpligen
  med \qty{6}{\decibel} dämpning per oktav ökad frekvens.
\end{itemize}
