\section{Sändningsslaget A3E (AM)}
\harecsection{\harec{a}{1.8.2}{1.8.2}, \harec{a}{1.8.6b}{1.8.6b}, \harec{a}{1.8.7b}{1.8.7b}}
\index{amplitudmodulation}
\index{A3E}
\index{AM|see {amplitudmodulation}}
\label{modulation_am}

\mediumfig{images/cropped_pdfs/bild_2_1-24.pdf}{Sidband vid A3E-modulation}{fig:BildII1-24}

Bild~\ssaref{fig:BildII1-24} visar frekvensspektrum av en signal vid
amplitudmodulation med

\begin{enumerate}[label=\alph*.,noitemsep]
\item en sinuston,
\item en blandning av tre sinustoner,
\item ett frekvensspektrum.
\end{enumerate}

\noindent\textbf{Försök}
%
Modulera en A3E-sändare med en \qty{3}{\kilo\hertz}-signal.
Med en mottagare utrustad med ett smalt filter för telegrafi, kan man urskilja
och påvisa bärvågen och de båda sidbanden.

\subsection{A3E-modulation med en ton}

\mediumfig{images/cropped_pdfs/bild_2_1-25.pdf}{A3E-modulation med toner med olika styrka och frekvens}{fig:BildII1-25}

Bild~\ssaref{fig:BildII1-25} visar A3E-modulation med toner av olika styrka och
frekvens.
En omodulerad bärvåg har konstant amplitud.
En amplitudmodulerad signal är i grunden resultatet av svävning mellan
frekvenser eller av icke linjär blandning av frekvenser.
När bärvåg och basband blandas är särskilt tre blandningsprodukter av intresse.
Dessa är
\begin{itemize}
\item bärvågen
\item det lägre sidbandet (förkortat LSB)
\item det övre sidbandet (förkortat USB).
\end{itemize}

AM-signalen består således inte bara av bärvågsfrekvensen \(f_{HF}\) utan även
av övre och nedre sidofrekvenser, vilka är summan och skillnaden av
bärvågsfrekvensen \(f_{HF}\) och den modulerande frekvensen \(f_{LF}\).
Alltså \(f_{HF} + f_{LF}\) (övre sidofrekvens) och skillnadsfrekvensen
\(f_{HF} - f_{LF}\) (undre sidofrekvens).

Eftersom tal inte bara omfattar en enda frekvens utan ett helt frekvensspektrum
(ca \SIrange{0,3}{3}{\kilo\hertz}) uppstår inte bara två sidofrekvenser utan två
sidband, det lägre sidbandet (LSB, Lower Side Band) och det övre (USB, Upper
Side Band).

LF-signalens frekvens bestämmer sidofrekvensens avstånd från bärvågen.
Bandbredden på en amplitudmodulerad signal med full bärvåg och två sidband är
dubbelt så stor som den högsta modulerande LF-frekvensen:
\(b= 2 \cdot f_{LFmax}\)

Om de modulerande LF-frekvenserna är mellan 0,3 och \qty{3}{\kilo\hertz} blir
sändningens totala bandbredd \qty{6}{\kilo\hertz}.

LF-signalernas amplitud påverkar sidbandens och sidofrekvensernas amplitud.
Vid maximal modulation (\qty{100}{\percent} modulationsgrad) varierar
signalamplituden mellan noll och dubbla värdet av det för en omodulerad bärvåg.

Som mest kan vardera sidbandet överföra en fjärdedel så mycket effekt som
bärvågen, dvs. en sjättedel av den totalt utsända effekten.
Då avger sändaren dubbelt så stor medeleffekt som utan modulation.
Toppeffekten (PEP, Peak Envelope Power) är till och med fyra gånger så stor.

Slutförstärkaren och kraftförsörjningen måste dimensioneras för toppeffekten vid
full modulation eller att modulationsgraden anpassas så att överbelastning inte
sker.

\subsection{Fördelar med A3E-modulation}

En A3E-sändare är enkel jämfört med en J3E-sändare, vilken har en mer
komplicerad signalbehandling.

\pagefig{images/cropped_pdfs/bild_2_1-26.pdf}{Amplitudmodulation med morsetecken}{fig:BildII1-26}

\subsection{Nackdelar med A3E-modulation}

Eftersom samma information finns i båda sidbanden och ingen finns i bärvågen,
så sänds effekten i bärvågen och ett av sidbanden ut till ingen nytta.
I talpauser sänds endast bärvågseffekten och till ingen nytta.
Även frekvensutrymme slösas bort.
Då en annan, alltför närliggande sändares bärvåg blandas med den egna,
alstras interferenstoner i mottagarna.

\mediumplustopfig{images/cropped_pdfs/bild_2_1-27.pdf}{Sidband vid DSB}{fig:BildII1-27}
