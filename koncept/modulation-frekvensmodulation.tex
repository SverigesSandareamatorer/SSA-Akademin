\section{Frekvensmodulation (FM)}
\harecsection{\harec{a}{1.8.3b}{1.8.3b}, \harec{a}{1.8.6d}{1.8.6d}}
\index{frekvensmodulation}
\index{FM|see {frekvensmodulation}}
\label{modulation_fm}

\mediumfig[0.8]{images/cropped_pdfs/bild_2_1-30.pdf}{Frekvensmodulation}{fig:BildII1-30}

Bild~\ssaref{fig:BildII1-30} (överst och i mitten) visar frekvensmodulation.

Vid frekvensmodulation varierar bärvågens frekvens i takt med den modulerande
signalens amplitud och polaritet.
På bilden ökar bärvågens frekvens när den modulerande signalen är positiv
(första halvperioden) och minskar när den modulerande signalen är negativ
(andra halvperioden).
Bilden visar att perioderna i den modulerade bärvågen tar kortare tid (har
högre frekvens), när den modulerande signalen är positiv, och mer tid (har lägre
frekvens) när den modulerande signalen är negativ.
Bärvågen kommer alltså att pendla omkring ett medelvärde, dvs. vara
frekvensmodulerad.

Frekvensavvikelsen \(\Delta f\) (deviationen) från bärvågens vilofrekvens är
vid varje tillfälle proportionell mot den modulerande signalens amplitud.
Sålunda är deviationen liten när den modulerande signalens amplitud är liten
och störst när amplituden når sitt toppvärde, antingen amplituden är positiv
eller negativ.
Vid en modulationsfrekvens av \qty{300}{\hertz} varierar bärvågsfrekvensen 300
gånger per sekund, vid \qty{3}{\kilo\hertz} varierar den 3000 gånger per sekund.

Likspänningsnivåer kan överföras med FM, eftersom en motsvarande
frekvensavvikelse kan framställas.

Bilden visar också vad som oftast sägs, att bärvågsamplituden inte ändras av
modulationen.
Detta är emellertid bara delvis sant, eftersom såväl bärvågsamplitud som
sidbandsamplitud varierar med modulationsindex, vilket förklaras nedan.

\subsection{Sidbanden vid vinkelmodulation}

Vid AM produceras endast ett sidbandspar med samma innehåll, ett över och ett
under bärvågsfrekvensen.
Vid vinkelmodulation, både vid FM och PM, produceras däremot flera sidbandspar
över och under bärvågsfrekvensen.
Dessa sidband uppträder på multiplerna av varje modulerande frekvens.
Vid basband med samma frekvensomfång har därför en vinkelmodulerad signal
större bandbredd än en AM-signal.

Vid vinkelmodulation beror antalet sidband på sambandet mellan den modulerande
frekvensen, frekvensdeviationen och modulationsindex.

\mediumtopfig{images/cropped_pdfs/bild_2_1-31.pdf}{Sidbandsspektrum vid FM-modulering med 1 sinuston}{fig:BildII1-31}

\subsection{Bandbredden vid vinkelmodulation}

Bild~\ssaref{fig:BildII1-30} (nederst) visar bandbredd på vinkelmodulation.
Vi gör tankeexperimentet att en FM-sändare moduleras med en fyrkantsvåg.
Frekvensen kommer då att hoppa växelvis mellan frekvenserna \(f\) och
\(f + \Delta f\).
Sättet kallas FSK (frekvensskiftnyckling) och används till exempel vid sändning
av radiofjärrskrift (RTTY, AMTOR, Paketradio etc.).

Vi föreställer oss två sändare, som sänder varannan gång, varav den ena sänder
frekvensen \(f\) och den andra sänder \(f + \Delta f\).
Båda sändarnas HF-signaler kommer då att bilda ett frekvensspektrum, som
förutom \(f\) och \(f + \Delta f\) även innehåller sidofrekvenser.

Bredden på detta spektrum beror bland annat på nycklingsfrekvensen.
Eftersom en fyrkantsvåg innehåller summan av dess grundfrekvens och övertoner,
kommer alla dessa toner att modulera vardera sändaren.
De högsta modulerande LF-frekvenserna alstrar sidofrekvenserna längst ut från
vilofrekvensen.
LF-signalens frekvensspektrum påverkar alltså HF-signalens bandbredd.

Spektrum nederst i bilden är en förenklad framställning av
frekvensskiftnyckling.

Vid modulation med en sinussignal istället för med en fyrkantssignal, uppstår
ett frekvensspektrum som på överst i bilden.


\subsection{Frekvensdeviation och modulationsindex}
\harecsection{\harec{a}{1.8.4}{1.8.4}}
\index{frekvensdeviation}
\index{modulationsindex}
\index{symbol!\(m\) modulationsindex}

%% k7per: Find a solution for words that already have a hyphen. quote-dash?
Bild~\ssaref{fig:BildII1-31} visar sidbandsspektrum vid FM-moduler\-ing med 1
sinuston.
Vid vinkelmodulation uppstår talrika sidofrekvenser, som beror av den
modulerande frekvensen \(f_{LF}\).
Amplitudfördelningen mellan sidofrekvenserna står i förhållande till
deviationen, varvid deras amplitud blir mindre ju längre bort från bärvågen
de är.

I praktiken anses en sidofrekvens försumbar när dess amplitud är mindre än
\qty{1}{\percent} av amplituden för omodulerad bärvåg.

För beräkning av bandbredden används begreppet modulationsindex \(m\), vilket är
kvoten av maximal deviation \(\Delta f\) och högsta frekvensen \(f_{LF}\).
%%
\[m = \dfrac{\Delta f_{max}}{f_{LFmax}}\]
%%
Inom amatörradion är det vanligt att arbeta med \(\Delta f_{max} =
\qty{3}{\kilo\hertz}\) och \(f_{LFmax} = \qty{3}{\kilo\hertz}\), dvs. \(m = 1\).

Vid modulationsindex \(m = 1\), gäller följande formel för bandbredden \(b\)

% k7per: Make this a formula?
\medskip
\(b = 2 \cdot ( \Delta f_{max} + f_{LFmax}) = 2 \cdot \Delta f_{max}
 + 2 \cdot f_{LFmax}\)
 \medskip
 
Med ovan nämnda värden blir bandbredden \(b = 2 \cdot (\qty{3}{\kilo\hertz} +
\qty{3}{\kilo\hertz}) = \qty{12}{\kilo\hertz}\)

Bandbredden ökar således både med ökande deviation och ökande modulerande
frekvens.
För att inte interferera med trafik på grannkanalerna måste såväl deviation som
frekvensen på den modulerande signalen begränsas.
En deviationsbegränsare begränsar amplituden på denna signal.
Ett lågpassfilter reducerar den distorsion, som uppstår av begränsningen.
Vidare undertrycks modulerande frekvenser högre än \qty{3}{\kilo\hertz}, vilket
är tillräckligt för överföring av tal.

\paragraph{Jämförelse}

En VHF-rundradiosändare är tilldelad ett större frekvensutrymme och kan därför
använda mycket större bandbredd.

Där är \(\Delta f_{max} = \qty{75}{\kilo\hertz}\) och \(f_{LFmax} =
\qty{15}{\kilo\hertz}\), därmed är \(m = \frac{75}{15} = 5\) och \(b = 2 \cdot
(75 + 15) = \qty{180}{\kilo\hertz}\).

Som framgår av tabell~\ssaref{tab:ampmod} varierar bärvågens liksom
sidofrekvensernas inbördes amplitud med modulationsindex.
Detta ska jämföras med AM där bärvågens amplitud är konstant och endast
sidbandens amplitud varierar.

Vid vinkelmodulation utsläcks bärvågen \(A_0\) vid modulationsindex 2,404.
Den blir sedan ''negativ'' vid högre index, vilket betyder att den återkommer,
men att dess fasläge blir omvänt.
I vinkelmodulation tas energin i sidbanden från bärvågen, vilket innebär att
den totala effekten förblir densamma oavsett modulationsindex.

%\paragraph{Kännetecken för sändningsslaget F3E (FM)}
%\index{F3E}

\paragraph{Fördelar med sändningsslaget F3E (FM)}
F3E-sän\-daren är enkel till sin uppbyggnad och hög överföringskvalitet
uppnås vid stor bandbredd, störningar från amplitudmodulerade signaler såsom
tändgnistor undertrycks i mottagaren.

\paragraph{Nackdelar med sändningsslaget F3E (FM)}
En relativt stor bandbredd behövs för överföring av ett basband med stort
frekvensomfång.
Sändaren måste avge full effekt, även när modulation inte sker.

\begin{table*}[ht]
\begin{center}
  %\begin{tabular}{ll|S|S[table-format=-1.3]|S[table-format=-1.3|S[table-format=-1.3]|S[table-format=-1.3]|S[table-format=-1.3]|S[table-format=-1.3]|S[table-format=-1.3]|}
  \begin{tabular}{ll|S[table-format=-1.3]|S[table-format=-1.3]|S[table-format=-1.3]|S[table-format=-1.3]|S[table-format=-1.3]|S[table-format=-1.3]|S[table-format=-1.3]|l|}
\cline{3-9}
&\multicolumn{1}{l}{}  & \multicolumn{7}{|c|}{Modulationsindex} \\ \cline{3-9}
&\multicolumn{1}{l|}{}  &  \multicolumn{1}{c|}{1}   &   \multicolumn{1}{c|}{2}   &    \multicolumn{1}{c|}{3}   &    \multicolumn{1}{c|}{4}   &    \multicolumn{1}{c|}{5}   &    \multicolumn{1}{c|}{6}   &    \multicolumn{1}{c|}{7}   \\ \hline
\multicolumn{1}{|c|}{\multirow{11}{*}{\rotatebox[origin=c]{90}{Relativ amplitud på}}}&\(A_0\) & 0,765 & 0,224 & \num{-0,260} & \num{-0,397} & \num{-0,178} &  0,151 &  0,300 \\
\multicolumn{1}{|c|}{}&\(A_1\) & 0,440 & 0,577 &  0,334 & \num{-0,066} & \num{-0,328} & \num{-0,277} & -0,005 \\
\multicolumn{1}{|c|}{}&\(A_2\) & 0,115 & 0,353 &  0,486 &  0,364 &  0,047 & \num{-0,243} & -0,301 \\
\multicolumn{1}{|c|}{}&\(A_3\) & 0,020 & 0,129 &  0,309 &  0,430 &  0,365 &  0,115 & -0,168 \\
\multicolumn{1}{|c|}{}&\(A_4\) &       & 0,034 &  0,132 &  0,281 &  0,391 &  0,358 &  0,158 \\
\multicolumn{1}{|c|}{}&\(A_5\) &       & 0,016 &  0,043 &  0,132 &  0,261 &  0,362 &  0,348 \\
\multicolumn{1}{|c|}{}&\(A_6\) & \multicolumn{2}{c|}{} &  0,011 &  0,049 &  0,131 &  0,246 &  0,339 \\
\multicolumn{1}{|c|}{}&\(A_7\) & \multicolumn{3}{c|}{} &  0,015 &  0,053 &  0,130 &  0,234 \\
\multicolumn{1}{|c|}{}&\(A_8\) & \multicolumn{4}{c|}{}           &  0,018 &  0,057 &  0,128 \\
\multicolumn{1}{|c|}{}&\(A_9\) & \multicolumn{4}{c}{} &        &  0,021 &  0,059 \\
\multicolumn{1}{|c|}{}&\(A_{10}\) & \multicolumn{5}{c}{Tomma fält för \(A_n\) under 0,01 (1 \%)} &  &  0,024 \\ \hline
\end{tabular}
\end{center}
\caption{Relativa amplituden på bärvåg $A_0$ och sidofrekvenser $A_1$--$A_{10}$ vid
modulationsindex 1--7. (Vid omodulerad bärvåg är modulationsindex 0. Då är
bärvågens relativa amplitud 1,0.)}
\label{tab:ampmod}
\end{table*}
