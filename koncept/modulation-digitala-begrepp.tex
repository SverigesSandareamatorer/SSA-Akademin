\section{Begrepp vid digital modulation}
\harecsection{\harec{a}{1.8.9}{1.8.9}}
\index{digital modulation}

Digital modulation innebär också att signalerna som sänds har lite andra
egenskaper än de analoga.
Istället för varierande spänningsnivåer som för till exempel tal skickar vi
diskreta fixa nivåer, ofta i form av bitar.
Det är därför lämpligt att diskutera några grundläggande begrepp kring digital
modulation.

\subsection{Bit rate}
\harecsection{\harec{a}{1.8.9a}{1.8.9a}}
\index{bit}
\index{byte}
\index{informationsmängd}
\index{informationsöverföringskapacitet}
\index{bit rate}

Informationen som vi skickar har vi kodat i bitar (eng. \emph{bit, b}),
\emph{informationsmängden} vi har är därför ett visst antal bitar och takten på
denna informationsmängd blir därmed \emph{informationsöverföringskapaciteten}
(eng. \emph{bit rate}) i bitar per sekund.

Ofta brukar vi referera till informationsmängden som mängden \emph{byte (B)}
som till exempel att en fil är 2\,kB eller en bild är 1,25\,MB.
Då en byte innehåller åtta bitar motsvarar det 16\,kb respektive 10\,Mb.
I dagligt tal talar vi då om storleken på en fil.

Överföringskapaciteten, eller i dagligt tal hastigheten, brukar vi ofta prata
om i termer av \emph{bit rate} som 10\,Mb/s (ofta skrivet \emph{bps -- bits per
second}), dvs. man klarar av att överföra upp till 10 miljoner bitar per sekund.

Det är ofta som man talar om den råa överföringskapaciteten, medan den
verkliga överföringskapaciteten för nyttotrafik är något lägre på grund av
olika former av packningsformat och protokollbehov, så kallad \emph{overhead}.
Man ska därför vara noga med att skilja dessa åt.

\subsection{Symboltakt -- Baud rate}
\harecsection{\harec{a}{1.8.9b}{1.8.9b}}
\index{symbol}
\index{symboltakt}
\index{symbol rate}
\index{Baud rate}
\index{Baudot, Emile}
\index{enheter!baud (Bd)}

Som vi redan sett exempel på kan ibland bitar skickas en och en, eller
ihopklumpade.
Varje sådan ihopklumpning kallas \emph{symbol}, och en symbol kan bära en eller
flera bitar, ibland inte ens ett jämnt antal.

Om man kan artikulera något i två olika \emph{nivåer} (av amplitud, fas,
frekvens eller kombination), så kan man representera en bit.
Om man kan artikulera något i fyra olika nivåer, kan man representera två bitar.
På samma sätt ger åtta nivåer support för tre bitar.
Varje representation kallar man en symbol och varje symbol bär alltså en, två
eller tre bitar information.
Strikt räknat är det logaritmen med bas två (2-logaritm eller $\log_{2}$) av
antalet nivåer som anger antalet bitar som en symbol kan bära.
Tre nivåer brukar sägas kunna bära 1,5~bitar, vilket är en slarvig approximation
men visar principen.

Den takt varmed symboler överförs \emph{symboltakten} (eng. \emph{symbol rate}),
benämns även \emph{Baud rate}, efter Emile Baudot, med enheten \emph{baud (Bd)}.
Enheten baud (förkortat Bd) anger antalet symboler per sekund.
Genom att multiplicera antalet symboler per sekund med antalet bitar per symbol
fås överföringskapaciteten bitar per sekund.

\subsection{Bandbredd}
\harecsection{\harec{a}{1.8.9c}{1.8.9c}}
\index{bandbredd}
\index{Nyquist-Shannons samplingsteorem}

Genom att justera antalet bitar per symbol kan man ändra antalet symboler
per sekund utan att ändra överföringskapaciteten. En anledning till att man
vill göra det är att bandbredden som används av en överföring är ungefär
proportionerlig mot symboltakten, det vill säga hur många baud man överför.
Detta påverkar hur stor del av radiospektrat man upptar, och därmed också hur
nära en annan signal man kan ligga i spektrat utan att störa varandra, dvs.
det påverkar frekvensplaneringen av bandet ifråga.

Ofta används begreppet bandbredd synonymt med överföringskapaciteten, eftersom
det finns en proportionell relation dem emellan, men bandbredden är inte den
enda parametern som krävs, så i mer strikta sammanhang ska dessa begrepp
hanteras som separata för att undvika missförstånd.

Bandbredden för en digital ström är relaterad till nyquistteoremet, som säger
att samplingstakten måste vara minst dubbelt så hög som den högsta frekvens
som överförs.
