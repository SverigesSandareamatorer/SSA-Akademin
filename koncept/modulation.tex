\chapter{Modulation}
\harecsection{\harec{a}{1.8}{1.8}}
\label{ch:modulation}
\index{modulation}
\index{modulerande signal}
\index{basband}
\index{modulerad signal}
\index{bärvåg}

\emph{Modulera} (lat. \emph{modulari}, rytmiskt avmäta, eng. \emph{modulate})
är att med hjälp av en oftast högfrekvent elektrisk signal (bärvågen) överföra
informationen i en lågfrekvent signal.
På så sätt kan lågfrekvens, till exempel tal och musik, först omvandlas till en
elektrisk signal, som får påverka (modulera) en högfrekvent elektrisk signal.
Denna modulerade signal strålas ut från antennen som ett elektromagnetiskt fält.

Den signal som innehåller informationen kallas \emph{modulerande signal},
\emph{basband} eller \emph{underbärvåg}.

Den signal som informationen överförts till kallas \emph{modulerad signal},
\emph{bärvåg} eller \emph{huvudbärvåg}.
