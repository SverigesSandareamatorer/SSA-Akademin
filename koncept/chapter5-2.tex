\section{Raka mottagare}
\harecsection{\harec{a}{4.1.2}{4.1.2}}
\index{rak mottagare}
\index{mottagare!rak}

\subsection{Mottagare med kristalldetektor}
\index{kristalldetektor}
\index{mottagare!kristall}

\mediumplusbotfig{images/cropped_pdfs/bild_2_4-01.pdf}{Detektormottagare}{fig:bildII4-1}

Detektormottagaren består av ett mycket litet antal komponenter.
Princip och arbetssätt framgår av bild~\ssaref{fig:bildII4-1}.
Samma princip används även i mer komplicerade mottagare, mätinstrument etc.
Antennkretsen består av antenn, jordtag och däremellan en induktor
(kopplingsspole), som överför energin från antennen till en resonanskrets.
Resonanskretsen används för att välja ut (selektera) en bärvåg med önskad
frekvens.
Bärvågen kan naturligtvis inte höras, men av kurvformen på bilden framgår
att bärvågen är amplitudmodulerad med en LF-signal.

För att återvinna LF-signalen utför man en så kallad demodulering med hjälp
av dioden.
Dioden klipper bort antingen de positiva eller negativa halvvågorna i den
mottagna signalen, beroende på hur dioden är vänd, polariserad.
Kondensatorn, som är kopplad parallellt över hörtelefonen, glättar de
högfrekventa spänningstopparna till ett amplitudmedelvärde (jämför med
entaktsblandare i avsnitt~\ssaref{detektorer}).
Detta spänningsvärde varierar på ett sätt, som motsvarar den modulerande
spänning i sändaren som kommer av tal, musik etc.
Vi har nu demodulerat bärvågen, återställt LF-signalen och kan höra den i
mottagaren.

Signalspänningen över resonanskretsen är störst när dess resonansfrekvens och
antennströmmens frekvens är lika.

\mediumfig[0.45]{images/cropped_pdfs/bild_2_4-02.pdf}{Selektion i detektormottagare}{fig:bildII4-2}

Överst i bild~\ssaref{fig:bildII4-2} ser man att mottagaren är inställd på
samma frekvens som sändare 2.
Även sändare 3 hörs eftersom bandbredden i resonanskretsen är stor.
Nederst i bilden är resonanskretsen inställd på sändare 3, men man hör
också sändare 2 och 4.

Bandbredden i resonanskretsen blir mindre ju mindre den belastas,
det vill säga dämpas.
I bild~\ssaref{fig:bildII4-1} består belastningen av antennen (via
kopplingsspolen), hörtelefonen och avkopplingskondensatorn (via dioden).

Mindre belastning kan åstadkommas på två sätt; dels med ''lösare''
koppling mellan antennkrets och resonanskrets och dels med bättre
impedansanpassning mellan resonanskrets och diod.
Båda sätten tillämpas i bild~\ssaref{fig:bildII4-3}.
Hur selektionen då förbättras visas i bild~\ssaref{fig:bildII4-4}, vilket ska
jämföras med bild~\ssaref{fig:bildII4-2}.

\mediumplusbotfig[0.77]{images/cropped_pdfs/bild_2_4-03.pdf}{Detektormottagare med LF-förstärkare}{fig:bildII4-3}

\mediumplustopfig{images/cropped_pdfs/bild_2_4-05.pdf}{Förbättrade HF-egenskaper i detektormottagare}{fig:bildII4-5}

\subsection{Detektormottagare med förstärkare}

Om man vill höra sändningarna över högtalare, behövs högre effekt än
vad som kan fångas upp genom antennen.
För ändamålet används en LF-förstärkare, som drivs av en annan energikälla,
till exempel ett batteri.
LF-förstärkaren kan även minska belastningen på resonanskretsen.

I bild~\ssaref{fig:bildII4-3} har ett LF-lågpassfilter satts in efter
HF-avkopplingskondensatorn.
Det dämpar LF-signaler med högre frekvens än vad som behövs för god mottagning.

\subsubsection{Mottagare med bättre HF-egenskaper}
\label{mottagare_bättre_hf}

Ett sätt att minska bandbredden i en detektormottagare är att koppla
flera resonanskretsar med samma frekvens efter varandra, så som illustreras
i bild~\ssaref{fig:bildII4-5}.
Den större dämpningen av fler kretsar kan kompenseras med en HF-förstärkare.

Sådana mottagare används för speciella ändamål, till exempel för övervakning
av en enda frekvens.
I sådana fall är resonanskretsarna fast avstämda.
Kanske utnyttjas till och med en kvartskristall som filter för den speciella
frekvensen.
Se bild~\ssaref{fig:bildII4-6} om hög selektion.

\smallfig{images/cropped_pdfs/bild_2_4-04.pdf}{Förbättrad selektion}{fig:bildII4-4}

\smallfigpad{images/cropped_pdfs/bild_2_4-06.pdf}{Hög HF-selektion}{fig:bildII4-6}

\smallfig{images/cropped_pdfs/bild_2_4-07.pdf}{CW i detektormottagare}{fig:bildII4-7}

\subsection{Detektormottagare och sändningsslag}

I huvudsak fungerar detektormottagaren endast vid amplitudmodulering.
Det innebär sändningsslagen A3E och A2A, det vill säga amplitudmodulerad
telefoni respektive tonmodulerad telegrafi, båda med full bärvåg.

Däremot fungerar detektormottagaren inte vid A1A, det vill säga telegrafi med
endast bärvåg.
En omodulerad bärvåg alstrar nämligen endast en likström i en
detektormottagare.
Vid nyckling hörs då endast knäppningar i hörtelefonen vid början och
slutet av teckendelarna, så som illustreras i bild~\ssaref{fig:bildII4-7}.

Detektormottagaren fungerar inte heller vid J3E, det vill säga SSB och övriga
sändningsslag med undertryckt bärvåg.
Ljud såsom tal förvrängs nämligen kraftigt i en J3E-signal eftersom
bärvågskomponenten saknas.

I båda ovannämnda fall kan talet återställas med tillsats av en bärvåg.
Slutligen kan sändningsslag som innebär frekvens- och fasmodulering i
princip inte demoduleras med detektormottagare.

\mediumtopfig{images/cropped_pdfs/bild_2_4-08.pdf}{Mottagare med direkt frekvensblandning}{fig:bildII4-8}

\subsection{Mottagare med direkt frekvensblandning}
\harecsection{\harec{a}{4.2.2}{4.2.2}, \harec{a}{4.3.2}{4.3.2}, \harec{a}{4.3.3}{4.3.3}, \harec{a}{4.3.6}{4.3.6}, \harec{a}{4.3.7}{4.3.7}}
\index{frekvensblandning}
\index{mottagare!direkt frekvensblandare}
\index{svävningston}
\index{BFO}
\index{Beat Frequency Oscillator (BFO)}
\index{A1A}
\index{J3E}

För att demodulera \emph{A1A} och \emph{J3E} i en rak mottagare --
detektormottagare måste den kompletteras med en oscillator som alstrar en
intern bärvåg.
Denna blandas med den mottagna signalen.
Det uppstår då en \emph{svävningston} (eng. \emph{beat frequency}).
Därav namnet \emph{Beat Frequency Oscillator (BFO)}.

Förfarandet har givit mottagartypen sitt namn -- direktblandad mottagare.

Ett sätt att komplettera den raka mottagaren med BFO framgår av bild
\ssaref{fig:bildII4-8}.
När BFO kopplas till och ställs in på en frekvens tillräckligt
nära mottagningsfrekvensen så uppstår en hörbar ton.

Demodulatordioden tillförs alltså två HF-signaler, dels den från antennen och
dels den från BFO.
Dessa båda signaler blandas i dioden och skillnadsfrekvensen är den hörbara
tonen.
Övriga blandningsprodukter dämpas av ett lågpassfilter.

\mediumtopfig{images/cropped_pdfs/bild_2_4-09.pdf}{Demodulering i mottagare med direkt frekvensomvandling -- CW-signaler}{fig:bildII4-9}

\mediumherefig{images/cropped_pdfs/bild_2_4-10.pdf}{Demodulering i mottagare med direkt frekvensomvandling -- SSB-signaler}{fig:bildII4-10}

\newpage
\subsubsection{Mottagning av telegrafi (CW)}
\harecsection{\harec{a}{4.2.1}{4.2.1}}
\index{CW}
\index{telegrafi}
\index{mottagare!CW}

Bild~\ssaref{fig:bildII4-9} illustrerar blandning av CW-signal och BFO-signal
för ett antal fall.

Då BFO (VFO) är inställd på frekvensen \(f_2\) = \qty{1831}{\kilo\hertz} och den
mottagna signalen \(f_1\) har frekvensen \qty{1830}{\kilo\hertz} så hörs en
svävningston med frekvensen \qty{1000}{\hertz}.
Samma resultat fås om BFO ställs in på frekvensen \(f_2\) =
\qty{1829}{\kilo\hertz}.

Med BFO på frekvensen \(f_2\) = \qty{1830}{\kilo\hertz} hörs ingenting av
signalen \(f_1\) = \qty{1830}{\kilo\hertz} från sändaren.
Frekvensskillnaden är noll hertz.

Med BFO på frekvensen \(f_2\) = \qty{1849}{\kilo\hertz} hörs nästan ingenting av
signalen \(f_1\) = \qty{1830}{\kilo\hertz} från sändaren, då mixprodukten
\qty{19}{\kilo\hertz} knappt är hörbar.

De flesta föredrar en ton med frekvensen cirka \qty{800}{\hertz} för mottagning
av telegrafi.
BFO-frekvensen skulle i så fall ställas in på 1830,8 eller
\qty{1829,2}{\kilo\hertz} om \(f_1\) vore en telegrafisändning.

\mediumfig{images/cropped_pdfs/bild_2_4-11.pdf}{Selektionen i direktblandade mottagare}{fig:bildII4-11}

\subsubsection{Mottagning av J3E (SSB)}
\harecsection{\harec{a}{4.2.3}{4.2.3}}
\index{J3E}
\index{SSB}
\index{mottagare!SSB}

När en SSB-sändare sägs arbeta till exempel på frekvensen
\qty{1835}{\kilo\hertz}, så innebär det frekvensen på den bärvåg som
undertryckts i sändaren redan före utsändningen.

Vad som uppfattas av mottagarens ingångskretsar är alltså det utsända sidbandet.
När en SSB-signal demoduleras, så blandas den lokala bärvågen i mottagaren med
de mottagna modulationsprodukterna.
Vid blandningen uppstår blandningsprodukter som består dels av LF, dels av
andra högre frekvenser som dämpas i ett lågpassfilter.

Bild~\ssaref{fig:bildII4-10} illustrerar en undertryckt bärvåg på
\qty{1835}{\kilo\hertz} och dess lägre sidband LSB som sträcker sig från
\qty{1832}{\kilo\hertz} till \qty{1834,7}{\kilo\hertz}.
Det demodulerade sidbandet sträcker sig från \qty{300}{\hertz} till
\qty{3}{\kilo\hertz}.

Inom amatörradio används för SSB det lägre sidbandet vid frekvenser
under \qty{10}{\mega\hertz}.
Med en frekvens av till exempel \qty{1835}{\kilo\hertz} och ett talspektrum av
\SIrange{300}{3000}{\hertz} kommer det lägre sidbandet att finnas mellan 1834,7
och \qty{1832,0}{\kilo\hertz}.
Tre modulerande frekvenser 300, 1000 och \qty{3000}{\hertz} visas på bilden.

Med en bärvågsfrekvens av \qty{1835}{\kilo\hertz} motsvaras de modulerande
frekvenserna av utfrekvenserna 1834,7; 1834 och \qty{1832}{\kilo\hertz}.
VFO ersätter SSB-sändarens bärvåg och ska ha samma frekvens --
\qty{1835}{\kilo\hertz} -- för att kunna återge 300, 1000 och \qty{3000}{\hertz}.

\newpage
\subsection{Selektionen i direktblandade mottagare}
\index{selektion}
\index{mottagare!selektion}
\label{selektion_direktblandade}

Direktblandade mottagare kan ses som en typ av detektormottagare, även
kallad ''rak'' mottagare.
Begreppet ''rak'' kommer av att HF-signalen från antennen passerar genom en
selektiv krets och en eventuell HF-förstärkare rakt fram till detektorn,
utan att frekvensen omvandlas.

I en detektormottagare är bandbredden oftast rätt stor.
Flera sändare hörs därför samtidigt.

På grund av att blandningsdioden i en direktblandad mottagare även fungerar
som AM-demodulator, så hörs faktiskt alla sändare inom förkretsens bandbredd.
Detta kan undvikas till en del genom att dioden, som fungerar som
entaktsblandare, byts till en mottaktsblandare eller ännu hellre till en
ringblandare.
Sådana blandare undertrycker ingångsfrekvenserna och släpper endast igenom
blandningsprodukter.
Bara den sändarsignal hörs då, vars frekvens tillsammans med VFO-frekvensen
ger blandningsprodukter, som faller inom LF-filtrets passband.
Mottagningsfrekvensen är VFO-frekvensen.
Resonanskretsen fungerar som en ställbar förselektor och LF-lågpassfiltret
ger den egentliga frekvensselektionen.

Vilka HF-signaler bildar blandningsprodukter med VFO-frekvensen och
vilka av dessa passerar sedan genom lågpassfiltret efter blandning ner
till LF-nivå?

\begin{exempelbox}
En CW-sändare med en \qty{1830}{\kilo\hertz} frekvens tas emot genom att
mottagarens VFO ställs in på frekvensen \qty{1829,2}{\kilo\hertz}.
Från blandarutgången kommer då en ton med frekvensen \qty{800}{\hertz}.

Men sändaren är inte ensam på bandet.
Kommer till exempel SSB-sändaren på 1835, som moduleras med 300, 1000 och
\qty{3000}{\hertz}, att störa mottagningen?
Se bild~\ssaref{fig:bildII4-11}.

Förkretsen i mottagaren är så bred att denna sändning passerar.
SSB-sändarens signalfrekvenser i det utsända sidbandet är 1834,7; 1834,0 och
\qty{1832}{\kilo\hertz}.
Dessa frekvenser blandas med mottagarens VFO-frekvens \qty{1829,2}{\kilo\hertz}
och alstrar blandningsprodukterna 5,5; 4,8 och \qty{2,8}{\kilo\hertz}.
Eftersom lågpassfiltret i mottagarens LF-förstärkare har bandbredden
\SIrange{0}{3000}{\hertz}, så kommer endast blandningsprodukten
\qty{2,8}{\kilo\hertz} att vara störande.
För att förbättra CW-mottagningen, så kan lågpassfiltret bytas ut mot ett
bandpassfilter, som endast släpper igenom ett smalt frekvensområde omkring
mittfrekvensen \qty{800}{\hertz}.
\end{exempelbox}

\mediumbotfig{images/cropped_pdfs/bild_2_4-12.pdf}{Passbandbredd och spegelfrekvenser i direktblandade mottagare}{fig:bildII4-12}
\subsection{Passband och spegelfrekvenser i direktblandare}
\index{passband}
\index{spegelfrekvenser}
\index{direktblandare}
\index{blandare!spegelfrekvenser}
\label{passband_spegelfrekvens}

I exemplet i förra stycket blev problemet med en störande ton löst med
ett bandpassfilter med annan frekvensgång.
Men vilka frekvenser kan tas emot genom ett lågpassfilter,
\SIrange{0}{3000}{\hertz}, om VFO-frekvensen är till exempel
\qty{1829,2}{\kilo\hertz}?

\textbf{Experiment:}
Ändra frekvensen på en CW-sändare långsamt från 1820 till
\qty{1840}{\kilo\hertz}.
Se bild~\ssaref{fig:bildII4-12}

Sändarfrekvensen \qty{1820}{\kilo\hertz} hörs knappast eftersom
blandningsprodukten har frekvensen \qty{9,2}{\kilo\hertz} och den dämpas
kraftigt av lågpassfiltret.
Först när sändarfrekvensen är \qty{1826,2}{\kilo\hertz} hörs en tydlig ton med
frekvensen \qty{3000}{\hertz}.
Fortsätter man att ändra sändarfrekvensen, så sjunker tonens frekvens för att
bli noll (svävningsnoll), när sändarfrekvensen är lika med mottagarens
VFO-frekvens \qty{1829,2}{\kilo\hertz}.
Om man nu fortsätter med att höja frekvens, så blir blandningsproduktens
frekvens åter högre.
Vid sändarfrekvensen 1832,2 är den \qty{3000}{\hertz}.
Vid ännu högre sändarfrekvens dämpas blandningsprodukten igen av lågpassfiltret.

Slutsatsen av experimentet blir följande:
Vid en direktblandande mottagare med VFO-frekvensen \\ \qty{1829,2}{\kilo\hertz}
och ett \qty{3}{\kilo\hertz} lågpassfilter blir varje sändare hörbar, som har en
sändningsfrekvens mellan 1826,2 och \qty{1832,2}{\kilo\hertz}, varvid
blandningsprodukten har frekvenser från \qty{3000}{\hertz}, ner genom noll och
upp till \qty{3000}{\hertz} igen.

Vår mottagare har bandbredden \qty{6}{\kilo\hertz}.
Varje annan sändare inom denna \emph{passbandbredd} kommer att höras eller --
om man så tycker -- störa mottagningen.

Tillbaka till exemplet med bandpassfiltret.
Vilka frekvenser kan tas emot med ett bandpassfilter \SIrange{700}{900}{\hertz}
(mittfrekvens \qty{800}{\hertz}), om VFO-frekvensen är \qty{1829,2}{\kilo\hertz}?
Jo, vi kan lyssna rätt ostört till vår CW-sändares \qty{800}{\hertz}-ton på
frekvensen \qty{1830}{\kilo\hertz}.
Ändå kan en annan sändare med frekvensen \qty{1828,4}{\kilo\hertz} störa
mottagningen därför att denna är \emph{spegelfrekvens} (eng. \emph{mirror
frequency}) till mottagningsfrekvensen \qty{1830}{\kilo\hertz}.
Vid VFO-frekvensen \qty{1829,2}{\kilo\hertz} uppstår en blandningsprodukt, inte
bara vid sändarfrekvensen \qty{1830}{\kilo\hertz} utan också vid
\qty{1828,2}{\kilo\hertz}.
Även denna andra sändarfrekvens, liksom nyttofrekvensen, släpps igenom
bandpassfiltret.

Spegelfrekvensmottagning är en principiell nackdel i mottagare med
direktblandning.
Nytto- och spegelfrekvens i det senaste exemplet ligger
\qty{1,6}{\kilo\hertz} ($2\cdot 800$~\unit{\hertz}) ifrån varandra, alltså
dubbla värdet av bandpassfiltrets mittfrekvens.

Vid SSB-mottagning måste naturligtvis hela LF-området upp till
\qty{3000}{\hertz} kunna släppas igenom.
Utöver det önskade frekvensområdet \SIrange{1832}{1835}{\kilo\hertz}, kommer
även spegelfrekvenser i området \SIrange{1835}{1838}{\kilo\hertz} att kunna tas
emot.

Vid en LF-bandbredd av \qty{3}{\kilo\hertz} har således den direktblandade
mottagaren en bandbredd av \qty{6}{\kilo\hertz}, vilket är en god
avstämningsskärpa i jämförelse med den \qty{300}{\kilo\hertz} breda förkretsen.

\subsection{För- och nackdelar med direkt\-blandare}

Enkel uppbyggnad, men trots det en god känslighet och hygglig avstämningsskärpa.
VFO kan även användas till att styra en sändare.

Spegelfrekvensmottagning är tyvärr oundviklig.
Vidare kan signaler från starka sändare stråla in i den känsliga LF-förstärkaren
och orsaka LF-detektering, om mottagaren är otillräckligt skärmad.
Förbättrad isolering mellan antenn och VFO kan dock fås med en HF-förstärkare.

Entakts diodblandare är olämplig i en direktblandad mottagare.
Den tar emot alla sändare inom förkretsens passband och en del av VFO-signalen
kommer att strålas ut i antennen.
Ingen av dessa nackdelar finns i en mottakts- eller ringblandare.
