\section{Radiovågornas egenskaper}
\label{radiovågornasegenskaper}

Ett elektromagnetiskt fält, som alstras i ett givet tidsmoment, breder
ut sig åt alla håll i rymden likt en ständigt växande sfär.

Fältstyrkan inom ett givet avsnitt av sfärens yta sjunker därför
alltefter som avståndet från sändaren ökar. Det är därför som en
sändare hörs svagare ju mera avlägsen den är ifrån mottagaren. Jämför
med ljuset från en rundstrålande lampa.

I rymden breder radiovågor ut sig mycket långt. Det uppstår dockäven
där utbredningsförluster i materia som finns i vägen.

När radiovågorna passerar genom jordatmosfärens olika skikt uppstår
mycket större utbredningsförluster än i rymden och därmed blir
räckvidden kortare.

Elektromagnetiska fält från alla slags sändare (emittörer) genomkorsar
alla slags material och alstrar strömmar i dem som är elektriskt
ledande.

Radiovågorna
\begin{itemize}
\item breder ut sig rätlinjigt i alla riktningar i rymden med ljusets
  hastighet som är ca 300 000 km/s (se även avsnitt \ref{ljushastigheten}),
\item tränger igenom fasta kroppar, som inte är elektriskt ledande,
\item dämpas eller reflekteras, bl.a. av metaller, joniserade vätskor
  och joniserade atmosfärskikt,
\item är polariserade,
\item förstärker eller motverkar varandra.
\end{itemize}
Radiovågorna breder ut sig
\begin{itemize}
\item utmed jordytan,
\item upp från jordytan,
\item upp från jordytan efter en första reflexion mot denna.  Det
  första sättet kallas för markvåg och de två senare kallas med ett
  samlingsbegrepp för rymdvåg.
\end{itemize}

Radiovågornas riktning kan böjas av genom
\begin{itemize}
\item reflexion eller splittring mot naturliga reflektorer i
  atmosfären och i jordytan,
\item konstgjorda såväl passiva som aktiva reflektorer (relästationer)
  på jordytan och i rymden.  Radiovågorna kan dämpas
\item i jordytan,
\item i topografin,
\item i atmosfärsskikten.
\end{itemize}

Vågutbredningens natur är mycket sammansatt och kan inte enkelt
beskrivas. Några starkt påverkande faktorer på vågutbredningen kan
ändå urskiljas, t.ex.
\begin{itemize}
\item utbredningsvägens höjd över jordytan,
\item radiovågens frekvens,
\item solstrålningens jonisering av jordatmosfären,
\item väderförhållandena..
\end{itemize}

\subsection{Olika slags vågavböjning}

Olika faktorer påverkar vågutbredningen inom olika avsnitt i
frekvensspektrum. Här följer de viktigaste:

\subsubsection{Reflexion}

Reflexion innebär att vågorna böjs tillbaka från den yta som de
träffar. Ljus- och radiovågor reflekteras på samma villkor eftersom
att båda är elektromagnetiska till sin natur.  Den stora skillnaden är
vågfrekvensen.

Reflektorns storlek uttrycks i termer av antal våglängder vid den
aktuella frekvensen. En 80-metersvåg reflekteras inte bra mot en yta
med bara någon meters sida.  Däremot reflekteras en 2-metersvåg mycket
bättre mot en lika stor yta och en ljusvåg (med våglängden 4 - 7.7 µm)
ojämförligt mycket bättre.

Olika materials förmåga att reflektera en infallande radiovåg beror av
vågens frekvens samt av materialets tjocklek och elektriska
ledningsförmåga. Vågen tränger djupare in i materialet vid låg
frekvens respektive vid låg ledningsförmåga.

\subsubsection{Refraktion}

Refraktion (brytning) innebär att vågen ändrar riktning, när den
passerar gränsen mellan två media eller material med olika
ledningsförmåga. När ledningsförmågan ändras successivt t.ex. i ett
atmosfärskikt, blir vågens avböjning mjuk.

\subsubsection{Diffraktion}

Diffraktion innebär att vågens infallsriktning splittras upp i flera
nya riktningar, närvågen passerar nära över ett hinder. Det är p.g.a.
detta fenomen som radiosignaler i viss mån kan höras även bortom en
berg rygg. Diffraktionen tilltar med minskande frekvens.
