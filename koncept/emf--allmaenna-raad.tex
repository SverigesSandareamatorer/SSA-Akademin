\section{Allmänna råd}
\index{EMF!allmänna~råd}
\index{Strålsäkerhetsmyndigheten}
\index{SSM|see {Strålsäkerhetsmyndigheten}}

Strålsäkerhetsmyndigheten (SSM) är den myndighet som är sammanhållande för
allmänhetens exponering för elektromagnetiska fält.
Dess allmänna råd, SSMFS~2008:18~\cite{SSMFS2008:18}, anger vilka referensvärden
som gäller i Sverige.

Syftet med de allmänna råden är att skydda allmänheten från akuta skadliga
biologiska effekter vid exponering för elektromagnetiska fält.
De allmänna råden är en svensk anpassning av ett EU-direktiv~\cite{1999/519/EG}
som grundar sig på ICNIRP:s riktlinjer.

Grunden till riktlinjerna är baserad på hur kroppen värms upp av radiovågorna
och definieras som:
hur stor effekt per kilogram kroppsvikt som absorberas under en definierad tid.
Den tekniska benämningen på detta värde är ''Specific Absorption Rate'' (SAR)
och mäts i \unit{\watt\per\kilogram}.
De grundläggande begränsningarna är, enligt internationella rekommendationer,
satta vid ungefär två procent av de nivåer vid vilka akuta biologiska effekter
är vetenskapligt säkerställda.

Då uppvärmningen av kroppsvävnad inte går snabbt räknar man med den medeleffekt
som under en viss tid orsakar uppvärmning.
De allmänna råden definierar SAR-värdet som medelvärdet under en
sexminutersperiod.

Eftersom ett elektromagnetiskt fält inte går att mäta i enheten
\unit{\watt\per\kilogram} innehåller de allmänna råden även en tabell på
referensvärden framräknade för att motsvara SAR-värdet.
Referensvärdena utgörs av storheter som är mätbara utanför människokroppen.

Referensvärdena är angivna i bland annat elektrisk- och magnetisk-fältstyrka,
vilka är mätbara storheter.
De är också de värden som en amatörradiostation inte bör överskrida i områden
där allmänheten kan vistas.

Vid frekvenser som är nära kroppens egen resonansfrekvens absorberas effekten
lättast och maximal upphettning uppstår.
Hos vuxna ligger den frekvensen på ungefär \qty{35}{\mega\hertz} om personen är
jordad och vid ungefär \qty{70}{\mega\hertz} om personen är isolerad från jord.
Även de olika kroppsdelarna kan vara resonanta.
Till exempel är en vuxens huvud resonant vid ca \qty{400}{\mega\hertz} medan
ett mindre barns huvud är resonant vid ca \qty{700}{\mega\hertz}.

Kroppens storlek avgör alltså vid vilken frekvens den absorberar mest effekt och
vid frekvenser över och under resonansfrekvensen så minskar uppvärmningen från
fältet.

Referensvärdena tar hänsyn till detta faktum och det mest restriktiva
frekvensområdet ligger inom 10 till \qty{400}{\mega\hertz} vilket är där
effekten absorberas lättast av kroppen.

\largetikz{
	\begin{tikzpicture}
	\begin{loglogaxis}[
		xlabel={Frekvens [MHz]},
		ylabel={Elektrisk fältstyrka (E) [\unit{V/m}]},
		xmin=0.1, xmax=10000,
		ymin=10, ymax=100,
		legend pos=south west,
		grid=major, grid style=dashed,
	]

	\addplot [forget plot, domain=0.003:1, samples=10, color=black,
	]{87};

	\addplot [forget plot, domain=1:10, samples=100, color=black,
	]{8.7*10^4/(x*10^6)^(1/2)};

	\addplot [forget plot, domain=10:399.9, samples=10, color=black,
	]{28};

	\addplot [forget plot, domain=400:1999.9, samples=100, color=black,
	]{1.375*(x*10^6)^(1/2)/1000};

	\addplot [forget plot, domain=2000:10000, samples=10, color=black,
	]{61};

	% Definiera punktlista som frekvens/värde-par (slash-separerade) och återanvänd
	\def\emfEpoints{0.1378/87,0.479/87,1.85/64,3.8/45,7.2/32,10.15/28,14.175/28,21.225/28,28.85/28,50/28,145/28,435/29,1296/50,2450/61,5750/61}

	% Rita markörer genom att bygga (x,y)-par från listan
	\addplot[only marks, color=gray, mark=*, nodes near coords, point meta=explicit symbolic]
		coordinates {\foreach \x/\y in \emfEpoints { (\x,\y) } };

	% Lägg till etiketter för varje punkt från samma lista: "<frekvens> MHz, <värde> V/m"
	\foreach \x/\y in \emfEpoints {
		\node[anchor=south west, font=\scriptsize, fill=white, inner sep=1pt] at (axis cs:\x,\y) {\x~MHz,~\y~V/m};
	}
		\legend{Amatörradioband}
	\end{loglogaxis}
	\end{tikzpicture}
}{Referensvärden för begränsning av elektriska fält på platser där allmänheten kan vistas. Amatörradioband och fältstyrkenivå angivna.}{fig:emf1}

\largetikz{
	\begin{tikzpicture}
	\begin{loglogaxis}[
		xlabel={Frekvens [MHz]},
		ylabel={Magnetisk fältstyrka (H) [\unit{A/m}]},
		xmin=0.1, xmax=10000,
		ymin=0.01, ymax=10,
		legend pos=north east,
		grid=major, grid style=dashed,
	]

	\addplot [forget plot, domain=0.1:0.15, samples=10, color=black,
	]{5};

	\addplot [forget plot, domain=0.15:10, samples=100, color=black,
	]{(7.3*10^5)/(x*10^6)};

	\addplot [forget plot, domain=10:399.9, samples=10, color=black,
	]{0.073};

	\addplot [forget plot, domain=400.0:1999.9, samples=100, color=black,
	]{0.0037*(x*10^6)^(1/2)/1000};

	\addplot [forget plot, domain=2000.0:10000, samples=10, color=black,
	]{0.16};

	% Definiera punktlista som frekvens/värde-par (slash-separerade) och återanvänd
	\def\emfHpoints{0.1378/5.0,0.479/1.524,1.85/0.395,3.8/0.192,7.2/0.101,10.15/0.073,14.175/0.073,21.225/0.073,28.85/0.073,50/0.073,145/0.073,435/0.077,1296/0.1332,2450/0.16,5750/0.16}

	% Rita markörer genom att bygga (x,y)-par från listan
	\addplot[only marks, color=gray, mark=*, nodes near coords, point meta=explicit symbolic]
		coordinates {\foreach \x/\y in \emfHpoints { (\x,\y) } };

	% Lägg till etiketter för varje punkt från samma lista: "<frekvens> MHz, <värde> A/m"
	\foreach \x/\y in \emfHpoints {
		\node[anchor=south west, font=\scriptsize, fill=white, inner sep=1pt] at (axis cs:\x,\y) {\x~MHz,~\y~A/m};
	}
		\legend{Amatörradioband}
	\end{loglogaxis}
	\end{tikzpicture}
}{Referensvärden för begränsning av magnetiska fält på platser där allmänheten kan vistas. Amatörradioband och fältstyrkenivå angivna.}{fig:emf2}

Bild~\ssaref{fig:emf1} illustrerar referensvärden för begränsning av elektriska
fält på platser där allmänheten kan vistas (100~kHz--10~GHz), med amatörband
och fältstyrkenivå angivna, till exempel \qty{10,15}{\mega\hertz} har en högsta
tillåtna elektriskt fältstyrka på \qty{28}{\volt\per\metre}.

Bild~\ssaref{fig:emf2} illustrerar referensvärden för begränsning av magnetiska
fält på platser där allmänheten kan vistas (100~kHz--10~GHz), med amatörband
och fältstyrkenivå angivna, till exempel \qty{10,15}{\mega\hertz} har en högsta
tillåtna magnetisk fältstyrka på \qty{73}{\milli\ampere\per\metre}.
