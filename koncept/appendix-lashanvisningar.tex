\onecolumn

\chapter{Läsanvisning för certifikatprov}

\section{Teknikdelens läsanvisningar}


\begin{table}[H]
	\small
\begin{tabular}{rll}
\textbf{Nr} & \textbf{Innehåll} & \textbf{Avsnitt}\\ \hline\hline
T1 & ledare, halvledare och isolatorer & 
\ssaref{konduktivitet}, \ssaref{ledare}, \ssaref{isolator}, \ssaref{halvledare}\\ \hline
T2 & potential, spänningsfall, resistans, ohms lag &
\ssaref{subsec:spaenning}, \ssaref{spänning.symboler}, \ssaref{elektrisk_ström}, \ssaref{subsec:stroemkrets}, \ssaref{resistans},\ssaref{ohms_lag}\\ \hline
T3 & elektrisk effekt, joules lag &
\ssaref{elektrisk_effekt}, \ssaref{joules_lag}\\ \hline
T4 & batterier, och batterikapacitet & 
\ssaref{batterikapacitet}\\ \hline
T5 & inre resistans, kortslutningsström & 
\ssaref{inre_resistans}, \ssaref{subsec:kortslutningsstroem}\\ \hline
T6 & serie- och parallellkopplade kraftkällor &
hela \ssaref{kraftkällor_serie_parallell}\\ \hline
T7 & elektriska fält och fältstyrka &
\ssaref{elektrisk_fälststyrka}, \ssaref{elektrostatik skärmning}\\ \hline
T8 & magnetiska fält och fältstyrka &
\ssaref{magfält_ström}, \ssaref{magnetisk_fältstyrka}\\ \hline
T9 & våglängd och frekvens &
\ssaref{utbredningsmodeller}, \ssaref{subsec:elektromagnetiska-faelt}\\ \hline
T10 & toppvärde, amplitud, effektivvärde &
\ssaref{subsec:toppvaerde}, \ssaref{peak-to-peak-värde}, \ssaref{subsec:effektivvaerde}\\ \hline
T11 & periodtid och frekvens&
\ssaref{period}, \ssaref{frekvens}\\ \hline
T12 & övertoner &
\ssaref{subsec:oevertoner}\\ \hline
T13 & analog modulation och analoga sändningsslag&
\ssaref{sec:modulationssystem}, \ssaref{sändningsslag}, \ssaref{kännetecken_modulerade_signaler}, 
\ssaref{bandbredd_modulation},  \ssaref{modulation_beskrivningskod}\\
 && \ssaref{modulation_am}, \ssaref{modulation_cw}, \ssaref{modulation_ssb}, 
 \ssaref{modulation_vinkel}, \ssaref{modulation_fm}\\ \hline
T14 & digital modulation och digitala sändningsslag &
hela \ssaref{modulation_digital}, \ssaref{bitfel_detektion}, \ssaref{modulation_aprs}, 
\ssaref{modulation_psk31}\\ \hline
T15 & decibel, dBm, verkningsgrad &
\ssaref{effekt_db}, \ssaref{dBm}, \ssaref{verkningsgrad}\\ \hline
T16 & digital signalbehandling, sampling, kvantisering & \\
   & samplingsfrekvens, D/A och A/D-omvandlare &
\ssaref{sec:DSP}, \ssaref{sampling}, \ssaref{nyquist}, \ssaref{ADC-DAC}\\ \hline
T17 & resistorer (motstånd) & 
\ssaref{enheten_ohm}, \ssaref{fasta_resistorer_linjära}, \ssaref{fasta_resistorer_olinjära}\\
& kondensatorer & 
\ssaref{resistor_temperaturkoefficient}, \ssaref{kondensator_allmänt}--\ssaref{kapacitiv_reaktans}\\ 
& induktorer (spolar) &
\ssaref{induktor_allmänt}, \ssaref{enheten_henry}--\ssaref{induktiv_reaktans} \\
& tranformatorer & 
\ssaref{ideal_transformator} \\ \hline
T18 & dioder och diodtillämpningar &
\ssaref{dioden_allmänt}, \ssaref{subsec:zenerdiod}, \ssaref{diod_led}\\ \hline
T19 & transistorer och förstärkningsfaktor, strömställare &
\ssaref{transistor_allmänt}, \ssaref{transistor_förstärkningsfaktor}, \ssaref{transistor_pnp}, 
\ssaref{transistor_strömställare} \\ \hline
T20 & Operationsförstärkare (jmfr buffertsteg) & 
\ssaref{op-amp} \\ \hline
T21 & serie- och parallellkopplade resistorer &
\ssaref{seriekopplade_resistorer}, \ssaref{parallellkopplade_resistorer}\\ \hline
T22 & spänningsdelare & 
\ssaref{spänningsdelare}\\ \hline
T23 & serie- och parallellkopplade kondensatorer & 
\ssaref{parallellkopplade kondensatorer}, \ssaref{seriekopplade_kondensatorer} \\ \hline
T24 & galvaniskt kopplade induktorer & 
\ssaref{galvaniskt_kopplade_induktorer}, \ssaref{induktor_urkoppling}\\ \hline
T25 & impedans och ohms lag vid växelström & 
\ssaref{impedans}, \ssaref{ohms_lag_växelström}, \ssaref{impedans_resonant_krets}\\ \hline
T26 & filter & 
\ssaref{filter} (inl.), \ssaref{lågpassfilter}, \ssaref{bandfilter_kristall} \\ \hline
T27 & kraftförsörjning och likriktare &
\ssaref{sec:kraftfoersoerjning} (inl.), \ssaref{likriktning}, \ssaref{glättningskretsar} (inl.), \ssaref{spänningsstabilisering}\\ \hline
T28 & förstärkarsteg & 
\ssaref{förstärkarsteg_allmänt}, \ssaref{förstärkare_grundkoppling}, 
\ssaref{förstärkare_utstyrningskontroll}\\ \hline
T29 & detektorer och demodulatorer & 
\ssaref{detektorer_allmänt}, \ssaref{fm_detektor} (inl.)\\ \hline
T30 & svängningar o oscillator, kristallosc., PLL, buffert & 
\ssaref{svängningar_alstring}--\ssaref{svängningar_LC-oscillator}, 
\ssaref{kristalloscillator}, \ssaref{PLL}, \ssaref{buffertsteg}\\ \hline
T31 & obalans i antennsystem & 
\ssaref{obalans_antennsystem}\\ \hline
T32 & mottagare, raka, superheterodyn, AGC & 
\ssaref{mottagare_bättre_hf}, \ssaref{selektion_direktblandade}, \ssaref{passband_spegelfrekvens}, 
hela \ssaref{superheterodynmottagaren}, \ssaref{AGC} (inl.)\\ \hline
T33 & brusspärr och selektivitet & 
\ssaref{subsec:brusspaerr}, \ssaref{tonöppning}, \ssaref{subton}\\ \hline
T34 & selektivitet, spegelfrekvenser, mottagarkänslighet & 
\ssaref{selektivitet}, \ssaref{spegelfrekvens_mottagare} (inl.), 
\ssaref{bandbredd_fm}, \ssaref{signalkänslighet_brus}\\ \hline 
T35 & sändare, blockscheman, egenskaper, duplex & 
\ssaref{sändare_blockschema}-\ssaref{sändare_frekvensblandning}, \ssaref{utgångsimpedans}, \ssaref{cw-klickar}, \ssaref{splatter}, \ssaref{duplex}\\ \hline
T36 & antenner och antennvinst, polarisation &
\ssaref{sec:antennsystem-allmaent}--\ssaref{antenner_impedans}, 
\ssaref{antenner_ståendevåg}--\ssaref{antenner_antennvins}, 
\ssaref{polarisation_hf}, \ssaref{polarisation_vhf}\\ \hline
T37 & antenner (kortvåg, VHF, UHF och SHF) &
\ssaref{ändmatad_halvvågsantenn}, \ssaref{jordplanantenn}, 
\ssaref{antenner_vhf_allmänt}, \ssaref{antenner_vhf_yagi}\\ \hline
T38 & transmissionsledningar & 
\ssaref{avstämd_matarledning}, \ssaref{oavstämd_matarledning}, \ssaref{stående_vågor}-
\ssaref{antenner_balansering}, Tabell \ssaref{tab:kabeldaempning}\\ \hline
T39 & vågutbredning och jonosfären & 
\ssaref{vågutbredning_reflektion}, \ssaref{vågutbredning_jonosfärskikten}--
\ssaref{d-skiktet}, \ssaref{e-skiktet}--\ssaref{sporadiskt_e}, hela 
\ssaref{solens_inverkan_jonosfären}\\ \hline
T40 & vågutbredning på kortvåg &
\ssaref{subsec:markvaag}-\ssaref{rymdvåg}, \ssaref{subsec:faedning}--\ssaref{om_kortvågsbanden}, hela \ssaref{vågutbredning_vhf}\\ \hline
T41 & mätning av likström, ståendevåg & 
\ssaref{mäta_likspänning}, \ssaref{mäta_ståendevåg}\\ \hline
T42 & konstlast, fältstyrkemätare & 
\ssaref{konstlast}, \ssaref{fältstyrkemätare}\\ \hline 
\end{tabular}
\normalsize
\end{table}

\newpage

\section{Reglementesdelens läsanvisningar}

\begin{table}[H]
\small
\begin{tabular}{rll}
\textbf{Nr} & \textbf{Innehåll} & \textbf{Avsnitt}\\ \hline\hline
R1 & fonetiska alfabeten & 
hela \ssaref{fonetiska_alfabet}, tabellen \ssaref{tab:bokstavering-svenska}\\ \hline
R2 & Q-koden &
hela \ssaref{q-koden}, tabellen \ssaref{tab:q-kod}\\ \hline
R3 & trafikförkortningar & 
hela \ssaref{trafikförkortningar}, tabellen \ssaref{tab:trafikforkortningar}\\ \hline
R4 & nödsignaler & 
hela \ssaref{subsec:noedsignaler} utom \ssaref{nödfrekvens}\\ \hline
R5 & anropssignaler & 
\ssaref{anropssignaler}--\ssaref{cq dx och split}, \ssaref{innehåll i förbindelse}--\ssaref{kryptering av radiomeddelande} \\ \hline
R6 & bandplaner &
hela \ssaref{bandplaner}, appendix \ssaref{bandplaner2} HF--UHF (veta gränser för CW/SSB)\\ \hline
R7 & bestämmelser, radioreglementen & 
\ssaref{ITU radioreglemente}, \ssaref{amatörradio definitioner}, \ssaref{regioner}\\ \hline
R8 & CEPT &
hela \ssaref{sec:CEPT} \\ \hline
R9 & svensk lag och föreskrift & 
hela \ssaref{svensk lag och föreskrift} \\ \hline
\end{tabular}
\normalsize
\end{table}

\section{Säkerhet läsanvisningar}

\begin{table}[H]
	\small
	\begin{tabular}{rll}
		\textbf{Nr} & \textbf{Innehåll} & \textbf{Avsnitt}\\ \hline\hline
		S1 & EMC elektromagnetisk kompatibilitet & 
		hela kapitel  \ssaref{ch:EMC}\\ \hline
		S2 & EMF elektromagnetiska fält &
		hela kapitel  \ssaref{EMF}\\ \hline
		S3 & Elsäkerhet & 
		hela kapitel  \ssaref{ch:elsakerhet}\\ \hline
	\end{tabular}
	\normalsize
\end{table}

\twocolumn
