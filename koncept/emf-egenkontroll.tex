\section{Egenkontroll}
\index{EMF!egenkontroll}
\index{EMF!utvärdering}

För att utvärdera sin egen station så finns det några olika vägar att gå:

\begin{itemize}
\item Räkna ut fältstyrkan eller säkerhetsavståndet med sina egna
  parametrar enligt exemplen ovan.
\item Jämföra med andras utvärderingar.
\item Använda programvara som är speciellt gjort för att räkna ut på
  vilket avstånd referensvärdet nås under givna förutsättningar enligt
  exempel 2 ovan.
\item Använda värden från tabeller där olika typiska antenner är beskrivna.
\item Använda antennsimuleringsprogram som har möjlighet att även
  beräkna fältstyrka.
\item Mäta fältstyrkan (speciellt då man utvärderar i närfältet från
  antennen).
\end{itemize}

Man bör då tänka på vilket avstånd man har till platser där allmänheten har
tillträde, sin effektanvändning, vilka antenntyper och vilka trafiksätt man
använder.

\subsection{Räkna manuellt}

Enligt exemplen ovan är det ganska enkelt att göra en uppskattning av
de fältstyrkor som genereras av sin egen amatörradioanvändning.

\subsection{Räkna med specialprogram}

Istället för att själv använda miniräknaren kan man använda program
som är speciellt framtagna för detta ändamål.

\mediumfig{images/IcnirpCalc.pdf}{Det grafiska gränssnittet till datorprogrammet
ICNIRPcalc där värdena matas in samt resultaten visas.}{fig:icnirpcalc}

Ett exempel på ett sådant program är ICNIRPcalc (bild~\ssaref{fig:icnirpcalc})
som är framtaget av en representant från den tyska amatörradioföreningen (DARC).
I programmet finns redan olika antenntyper och det finns även möjlighet att
lägga in egna antenner för att göra korrekta beräkningar.
Detta program finns att ladda ner från SSA:s webbplats för EMC/EMF-frågor.

\subsection{Tabellvärden}
Utifrån den typ av antenn man själv använder kan man jämföra med
typiska värden från andras beräkningar och göra en hyfsad uppskattning
av sig egen situation.

\subsection{Antennsimulering}
Vissa program för antennsimulering har även funktioner för att beräkna
fältstyrkenivåer runt antennen och kan i vissa fall beräkna fältstyrkan
även i närfältet.

\subsection{Mäta fältstyrka}
Att mäta fältstyrka kräver tillgång till kalibrerad mätutrustning som
ger mätvärden som är tillförlitliga nog för att med säkerhet kunna användas
vid utvärdering av fältstyrkenivån.
