\section{Bandbredd vid olika sändningsslag}
\harecsection{\harec{a}{1.8.5}{1.8.5b}}
\index{bandbredd}
\index{frekvenseffektivitet}
\label{bandbredd_modulation}

Varje radiosändning tar upp plats omkring den nominella bärvågsfrekvensen --
tillsammans \emph{bandbredden}.

Radioamatören måste veta detta ''platsbehov'', främst för att inte sända utanför
de frekvensband som är tilldelade för amatörradioanvändning, men även för att
kunna umgås med annan trafik inom banden.

I alla sändningsslag ökar den använda bandbredden med ökad modulation.
Eftersom största \emph{frekvenseffektivitet} alltid ska eftersträvas så upptar
en sändare med kraftigare modulation än vad som behövs för en överföring alltid
onödigt frekvensutrymme.
