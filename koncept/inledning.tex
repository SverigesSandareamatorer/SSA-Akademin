\chapter*{Inledning}

\section*{Amatörradio}

Amatörradio är en teknisk hobby med inriktning på kommunikation och experiment
med radioanläggningar samt radiovågors utbredning.
Det är en verksamhet som utövas över hela världen av licensierade radioamatörer,
även kallade sändaramatörer.

Syftet med amatörradio är att främja personlig utveckling och internationell
förståelse samt teknisk färdighet och erfarenhetsutbyte inom området.
Amatörradio kan därtill vara en tillgång då samhällets normala resurser för
radiokommunikation behöver förstärkas.

\section*{En hobby med krav}

För att använda en radiosändare, och i vissa fall inneha, i ett land, krävs
tillstånd (licens) från dess teleadministration.
För ett amatörradiotillstånd föreskrivs i det internationella radioreglementet
\cite{ITU-RR} bland annat handhavandemässiga och tekniska kvalifikationer hos
varje person som önskar använda en amatörradiostation.
De nationella teleadministrationerna tillser detta genom kompetensprov.
För att få sända med amatörradiosändare måste man ha amatörradiocertifikat.

CEPT är ett samarbetsorgan mellan europeiska länders teleadministrationer
(myndigheter).
En av dem är svenska Post- och telestyrelsen (PTS).

Dessa administrationer har antagit rekommendationer om sinsemellan
harmoniserade krav på radioamatörers kompetens.

Sverige har antagit CEPT-rekom\-men\-da\-ti\-on\-en
 Ha\-rmonised Amateur Radio Examination
Certificate, Vilnius 2004, version 5 februari 2016, T/R~61-02~\cite{TR6102}.
Vid genomförandet av kompetensprov ska de i den rekommendationen
angivna kraven särskilt beaktas.

För den som godkänts i ett sådant prov utfärdas ett harmoniserat
amatörradiocertifikat (HAREC).

Det svenska certifikatet bygger på CEPT HAREC krav~\cite{TR6102},
med anpassning till svensk frekvensplan i Bilaga~\ssaref{svensk frekvensplan}.

%% De detaljerade CEPT HAREC-kraven finns i Bilaga \ssaref{CEPT HAREC}, där även
%% referenser till den eller de delkapitel som avses uppfylla utbildningskraven.
%% Alla de delkapitel som är märkta med HAREC ingår alltså i den internationellt
%% överenskomna kunskapsmängden som utbildningen ska inkludera.

\section*{Utbildning}

Man kan antingen söka sig till någon av de klubbar som har kurs eller skaffa
SSA:s utbildningspaket och studera på egen hand.
SSA har dessutom övningsprov online som man kan testa sina kunskaper på, något
som varmt rekommenderas för alla studerande oavsett studieform.

Amatörradioklubbarna bedriver huvuddelen av utbildningen med
amatörradiocertifikat som mål.
Också vissa skolor, militära förband, FRO-förbund med flera har amatörradio på
programmet.
Se SSA:s webbplats <\href{https://www.ssa.se}{www.ssa.se}> för aktuella
kurstillfällen.

När man är mogen för att avlägga certifikatprov skriver man för någon av de
provförrättare som finns.
De klubbar som har utbildning brukar planera prov med den grupp elever de har.

Efter avlagt och godkänt prov kan man sedan ansöka om anropssignal och
certifikat, något som SSA sköter enligt delegation från Post- och telestyrelsen.

Till tillståndet knyts en internationellt unik anropssignal.
Man har möjlighet att föreslå en anropssignal, men i brist på förslag tas en
ledig anropssignal ur serien.

\section*{Föreningen Sveriges Sändareamatörer -- SSA}

SSA är en ideell förening för personer med intresse för amatörradio.
Verksamheten är religiöst och politiskt obunden.
Ett av syftena är att bland medlemmarna verka för ökade tekniska kunskaper och
god radiotrafikkultur för att därigenom skapa en kår av kunniga radioamatörer.
SSA representerar Sverige som nationell förening i
International Amateur Radio Union (IARU), Region~1.

\section*{Internationell samverkan}

De nationella föreningarna inom IARU samarbetar över nationsgränserna.
Ett exempel är när DARC (Deutscher Amateur-Radio-Club e.V.) för ett antal år
sedan ställde sina Ausbildungsunterlagen~\cite{DARCaus} till SSA:s förfogande
som källmaterial till föregångaren till denna bok.

\section*{Denna bok}

Denna bok omfattar hela teorin för CEPT HAREC och PTS krav.
Den ingår i det utbildningspaket som kan köpas från SSA.

Så till vida innehåller boken ämnen såsom grundläggande ellära, elektronik,
komponenter, kretsar, radioteknik, elsäkerhet, regler, bestämmelser, bandplaner
och trafikmetoder.
Det finns även inlärningsanvisningar för morsesignalering för den som vill lära
sig telegrafi.

I bilagorna finns bland annat grundläggande matematik och frekvensplaner för
amatörradiotrafik.

Rekrytering av handledare för terminslånga kurser är en nyckelfråga för
kursarrangören, liksom målinriktade, anpassade läromedel.

Tanken med denna bok är att leverera ett material som kan vara grunden till
denna utbildning samt även för viss fördjupning och förståelse för de begrepp
som man vanligtvis stöter på inom hobbyn.

\section*{VAD behöver en radioamatör kunna?}
\balance

CEPT är ett samarbetsorgan mellan europeiska länders teleadministrationer
(myndigheter).
En av dem är svenska Post- och telestyrelsen (PTS).

Dessa administrationer har antagit rekommendationer om sinsemellan
harmoniserade krav på radioamatörers kompetens.

Sverige har antagit CEPT-rekommendationen T/R~61-02~\cite{TR6102}.
Vid genomförandet av kompetensprov ska de i rekommendationen angivna kraven
särskilt beaktas.

För den som godkänts i ett sådant prov utfärdas ett harmoniserat
amatörradiocertifikat (HAREC).
Rekommendationen anger kompetensnivån HAREC.
%
Det svenska certifikatet bygger på CEPT HAREC krav~\cite{TR6102},
med anpassning till svensk bandplan i bilaga~\ssaref{svensk frekvensplan}.
%De detaljerade CEPT HAREC kraven finns i Bilaga \ssaref{CEPT HAREC}, där även
%referenser till den eller de del-kapitel som avses uppfylla utbildningskraven.

\section*{HUR blir man radioamatör?}

För att få sända med amatörradiosändare måste man ha amatörradiocertifikat.
Man kan antingen söka sig till någon av de klubbar som har kurs, eller skaffa
SSA:s utbildningspaket och studera på egen hand.
SSA har övningsprov online som man kan testa sina kunskaper på.
När man är mogen för att avlägga certifikatprov så skriver man för någon av de
provförrättare som finns.
De klubbar som har utbildning brukar planera prov med den grupp elever de har.

Efter avlagt och godkänt prov kan man sedan ansöka om signal och certifikat,
något som SSA sköter enligt delegation från Post- och telestyrelsen.

Till tillståndet knyts en internationellt unik anropssignal.
Man har möjlighet att föreslå en anropssignal, men i brist på förslag så tas en
ledig anropssignal ur serien.

\section*{VAR hålls det certifikatskurser?}

Vissa amatörradioklubbar, militära förband, FRO-förbund och andra
sammanslutningar håller certifikatskurser.
Det går också att studera på egen hand.

\section*{VILKA läromedel behöver man?}

Denna bok omfattar hela teorin för CEPT HAREC och PTS krav.
Den ingår i det utbildningspaket som kan köpas från SSA.
