\section{Digitala kretsar}
\index{digitala kretsar}
\label{digitala kretsar}

Digital elektronik förekommer i all modern utrustning för radio- och
telekommunikation.
Ämnet är mycket omfattande och här redogörs endast för några grundläggande
digitala funktioner.

I \emph{analogtekniken} kan under ett förlopp förekomma oändligt många nivåer,
till exempel spänningar mellan noll och ett högsta värde.

I digitaltekniken förekommer bara ett bestämt antal tillstånd.
I det enklaste digitala systemet finns två tillstånd, till exempel 0 och 1 eller
Till och Från eller Hög och Låg eller Fel och Rätt.
Ett system med två tillstånd kallas binärt.
En lampa som tänds eller släcks med en enkel strömställare är ett binärt system.
Strömställaren kan ha olika utföranden.
Den kan vara en mekanisk kontakt som är styrd för hand eller av en reläspole.
Den kan också vara en transistor eller annan anordning.

\subsection{Transistorn som strömställare}
\label{transistor_strömställare}

\mediumfigpad{images/cropped_pdfs/bild_2_2-35.pdf}{Transistorn som analog förstärkare respektive digital strömställare}{fig:BildII2-35}

Bild~\ssaref{fig:BildII2-35} visar två transistorkopplingar.
Den till vänster är en analog förstärkare för växelspänning.
Om det på grund av en viss basspänning flyter en kollektorström av
\qty{1}{\milli\ampere} och kollektorresistorn har värdet \qty{5}{\kilo\ohm},
blir spänningsfallet över denna resistor \qty{5}{\volt}.
Eftersom matningsspänningen är \qty{12}{\volt}, blir spänningen \qty{7}{\volt}
mellan kollektorn och minuspolen.

Kopplingen till höger fungerar som en binär strömställare.
Antag att insignalen intar ett av två spänningstillstånd, antingen
\qty{0}{\volt} (låg) eller \qty{5}{\volt} (hög).
När inspänningen är till exempel \qty{5}{\volt}, flyter så mycket basström genom
basresistorns \qty{10}{\kilo\ohm} att transistorn blir fullt utstyrd.

Därmed är spänningen mellan kollektor och emitter, det vill säga utspänningen,
nära \qty{0}{\volt} (0,1 till \qty{0,2}{\volt} beroende på transistortyp).
Man säger då att utgången är låg (L) eller 0 (noll).

Om däremot inspänningen är \qty{0}{\volt}, spärras kollektorströmmen och
utspänningen blir nära \qty{5}{\volt}.
Man säger då att utgången är hög (H) eller 1.

För NPN-transistorn i bilden gäller att hög inspänning ger låg utspänning och
vice versa.

Denna logiska funktion kallas inverterande.

\subsubsection{NOT-gate eller inverterande grind}
\index{inverterande grind}
\index{NOT-gate}

\smallfig{images/cropped_pdfs/bild_2_2-36.pdf}{NOT-gate}{fig:BildII2-36}

Logiska funktioner beskrivs med internationella symboler.
En ring vid utgången betyder att utspänningens nivå är motsatt
inspänningens vilket illustreras i bild~\ssaref{fig:BildII2-36}.
Sambandet mellan in- och utnivåerna beskrivs med en \emph{sanningstabell}.


\subsection{Villkorskretsar -- s.k. grindar}

Det finns olika sätt att bygga grindar.
Idag är de flesta grindarna elektroniska lösningar.
Därutöver finns elektromekaniska grindar i form av strömbrytare och
reläkontakter.

Föregångarna till de elektroniska televäxlarna (AXE med flera) var stora system
av mestadels elektromekaniska reläer.

Att överskådligt förklara arbetssättet i de vanligaste grindarna görs enklast
med reläsymboler.
En reläkontakt kan då motsvara en transistor eller en diod.
Reläspolar kan motsvara logiska nivåer i insignaler.

Elektriska kontakter kan vara normalt öppna och sluter vid påverkan (slutande
kontakt).
Alternativt kan de vara normalt slutna och öppnar vid påverkan (brytande
kontakt).
I kretsscheman visas kontaktlägena vid systemet i vila.

\newpage
\mediumtopfig[0.65]{images/cropped_pdfs/bild_2_2-37.pdf}{OCH-grind (AND-gate)}{fig:BildII2-37}
\mediumtopfig[0.65]{images/cropped_pdfs/bild_2_2-38.pdf}{ELLER-grind (OR-gate)}{fig:BildII2-38}

Av bild~\ssaref{fig:BildII2-37} framgår att samma villkor kan skapas med
slutande alternativt brytande kontakter.
Observera placeringen av resistorn på kretsens utgångssida i respektive fall.
När resistorn ligger närmast pluspolen kallas den pull-up.
När den ligger närmast minuspolen kallas den pull-down.
I båda fallen definierar resistorn den logiska nivån.

\subsubsection{OCH-grind eller AND-gate}
\index{AND-gate}

Sanningstabellen i bild~\ssaref{fig:BildII2-37} säger att när alla insignaler
är 1 så är utsignalen också 1.

\subsubsection{ELLER-grind eller OR-gate}
\index{OR-gate}


Sanningstabellen i bild~\ssaref{fig:BildII2-38} säger att när en eller flera av
insignalerna är 1 så är utsignalen också 1.
När alla insignaler är 0 är utsignalen 0.

\subsubsection{OCH INTE-grind eller NAND-gate}
\index{NAND-gate}

Sanningstabellen i bild~\ssaref{fig:BildII2-39} säger att när ingen eller någon
insignal är 1, men inte alla, så är utsignalen 1.
När alla insignaler är 1 är utsignalen 0.

\subsubsection{INTE ELLER-grind eller NOR-gate}
\index{NOR-gate}

Sanningstabellen i bild~\ssaref{fig:BildII2-40} säger att när någon eller alla
insignaler är 1 är utsignalen 0.
När alla insignaler är 0 är utsignalen 1.

\subsubsection{Inverterad ingång}

En ingång kan behöva ha en inverterad funktion i förhållande till de övriga
(\emph{low active}).
Man kan då göra som i exemplet med en OCH-grind i bild~\ssaref{fig:BildII2-41}.

\mediumtopfig[0.66]{images/cropped_pdfs/bild_2_2-39.pdf}{OCH INTE-grind (NAND-gate)}{fig:BildII2-39}

\mediumtopfig[0.66]{images/cropped_pdfs/bild_2_2-40.pdf}{INTE ELLER-grind (NOR-gate)}{fig:BildII2-40}

\mediumherefig[0.65]{images/cropped_pdfs/bild_2_2-41sbs.pdf}{Inverterad ingång}{fig:BildII2-41}

%% k7per: De följande två små styckena är delade i två för att få
%% siddelninging and figurplacering OK, lägg inte ihop dem om du inte
%% håller på att fixa layout.
\newpage
Möjligheten att ha en ingång inverterad gör det ibland svårt att läsa
kretsscheman, för läslighet är det oftast bäst att ha med en explicit NOT-gate.

\newpage
Meningarna går naturligtvis isär om vad som är mest läsligt, så kolla
noga vilka grindar som används i ett visst kretsschema.

\newpage

\smallfig[0.25]{images/cropped_pdfs/bild_2_2-42.pdf}{Exklusiv ELLER-grind (EXOR-gate)}{fig:BildII2-42}
\smallfig[0.27]{images/cropped_pdfs/bild_2_2-43.pdf}{Exklusiv INTE ELLER-grind (EXNOR-gate)}{fig:BildII2-43}


\subsubsection{Exklusiv ELLER-grind (XOR-gate)}
\index{XOR-gate}

Sanningstabellen i bild~\ssaref{fig:BildII2-42} säger att när alla insignaler
antingen är 1 eller 0, så är utsignalen 0.
När någon insignal är 1, men inte alla, så är utsignalen 1.

\subsubsection{Exklusiv INTE ELLER-grind (XNOR-gate)}
\index{XNOR-gate}

Sanningstabellen i bild~\ssaref{fig:BildII2-43} säger att när alla insignaler
antingen är 1 eller 0, så är utsignalen 1.
När en insignal är 1, men inte alla, så är utsignalen 0.

\subsection{Grindar med dioder och transistorer}

I stället för reläer eller diskreta halvledare i grindar använder man nu ytterst
sällan något annat än integrerade digitala kretsar (se avsnitt
\ssaref{integrerade kretsar}).

\smalltikz{
  \bigskip
  \begin{circuitikz}
    \draw (-2,1) to[short,-o] (1,1);
    \draw
    (0,-2) node[npn](T) {}
    ;
    \draw (T.collector) to[R] (0,1) to (0,1);
    \draw (-2,-2) to[D] (-1,-2) to (T.base);
    \draw (-2,-2) to[D] (-4,-2) to node[label=A] {} (-4,-2);
    \draw (-2,1) to[R] (-2,-1) to (-2,-2);
    \draw (-2,-2) to (-2,-3);
    \draw (-2,-3) to[D] (-4,-3) to node[label=B] {} (-4,-3);
    \draw (T.emitter) to (0,-3) to node[ground] {} (0,-3);
    \draw (T.collector) to node[label=C] {} (1,-1.235);
  \end{circuitikz}
}{DTL-logik}{fig:BildII2-44}

%%\smallfig[0.4]{images/cropped_pdfs/bild_2_2-44.pdf}{DTL-logik}{fig:BildII2-44}

Bild~\ssaref{fig:BildII2-44} visar en NAND-grind.
Den egentliga grinden består av tre dioder och en resistor.
Två av dioderna är ingångar och den tredje är utgång.
Grinden styr en digitalt arbetande transistor liksom den i bild
\ssaref{fig:BildII2-35}.
Resultatet är en så kallad DTL-logik (eng. \emph{Diode-Transistor Logic}).

\smallfig[0.4]{images/cropped_pdfs/bild_2_2-45.pdf}{TTL-logik}{fig:BildII2-45}

Bild~\ssaref{fig:BildII2-45} visar en NAND-grind.
Här består den egentliga grinden av en ingångstransistor med två emittrar,
vilka motsvarar dioderna vid A och B i föregående bild.
Kollektorn i denna transistor motsvarar ingångsdioden till transistorn i bild
\ssaref{fig:BildII2-44}.
De övriga tre transistorerna i bild~\ssaref{fig:BildII2-45} bildar en switch
(digital strömställare, jämför bild~\ssaref{fig:BildII2-35}), som ger snabb
övergång mellan väl definierade logiska nivåer.
Resultatet är en så kallad TTL-logik (eng. \emph{Transistor-Transistor Logic}).
