\section{Kännetecken för modulerade signaler}
\label{kännetecken_modulerade_signaler}
\harecsection{\harec{a}{1.8.5}{1.8.5a}}
\index{amplitudmodulation}
\index{frekvensmodulation}
\index{fasmodulation}
\index{pulsmodulation}

\mediumfig{images/cropped_pdfs/bild_2_1-22.pdf}{Modulerade signaler}{fig:BildII1-22}

Bild~\ssaref{fig:BildII1-22} illustrerar modulerade signaler.
En modulerad signal kännetecknas av dess amplitud, frekvens och fasläge.

Vid \emph{amplitudmodulation} påverkas huvudbärvågens amplitud, så att den i
varje tidpunkt motsvarar den modulerande signalens variation.

Vid \emph{frekvensmodulation} påverkas huvudbärvågens frekvens, så att den i
varje tidpunkt motsvarar den modulerande signalens variation.

Vid \emph{fasmodulation}, som är besläktad med frekvensmodulation, påverkas i
stället för frekvensen huvudbärvågens fasläge i förhållande till en
referenssignal, så att fasläget i varje tidpunkt motsvarar den modulerande
signalens variation.

Frekvens- och fasmodulation liknar varandra och kan sammanfattas som
vinkelmodulation, eftersom fasvinkeln mellan bärvågens spänning och ström
varierar i båda fallen.

Vid \emph{pulsmodulation} används pulståg (korta upprepade bärvågspaket), till
exempel pulsamplitud-, pulslängds-, pulsläges- och pulskodmodulation.
Pulskodmodulation används till exempel vid samtidig överföring av flera
telesamtal på samma linje, bärvåg etc.
