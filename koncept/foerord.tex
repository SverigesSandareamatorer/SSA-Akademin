\chapter*{Förord}

\balance

\section*{Förord till tredje upplagan}

Denna bok omfattar hela teorin för CEPT HAREC och PTS krav.

SSA har sammanlagt fyra egna böcker för radioamatörer:
\emph{Bli sändaramatör -- Vägen till instegscertifikat},
\emph{Bli sändaramatör -- del 2 -- från instegscertifikat till internationellt amatörradiocertifikat},
\emph{KonCEPT för amatörradiocertifikat} (denna bok) och
\emph{Trafikhandbok}.

Så till vida innehåller boken ämnen såsom grundläggande ellära, elektronik,
komponenter, kretsar, radioteknik, elsäkerhet, regler, bestämmelser, bandplaner
och trafikmetoder.
Det finns även inlärningsanvisningar för morsesignalering för den som vill lära
sig telegrafi.

I bilagorna finns bland annat grundläggande matematik och frekvensplaner för
amatörradiotrafik.

Rekrytering av handledare för terminslånga kurser är en nyckelfråga för
kursarrangören, liksom målinriktade, anpassade läromedel.

Tanken med denna bok är att leverera ett material som kan vara grunden till
denna utbildning samt även för viss fördjupning och förståelse för de begrepp
som man vanligtvis stöter på inom hobbyn.

\bigskip

\begin{tcolorbox}[sidebyside,colback=white,title=Två sätt att lämna återkoppling]
\begin{center}
\qrcode[hyperlink,height=2.5cm,version=5]{mailto:akademin@ssa.se}
\end{center}

Skicka mejl till \texttt{akademin@ssa.se}.
\tcblower
\begin{center}
\qrcode[hyperlink,height=2.5cm,version=5]{https://github.com/SverigesSandareamatorer/SSA-Akademin/issues/new}
\end{center}

Skapa ett ärende på GitHub.
\end{tcolorbox}
