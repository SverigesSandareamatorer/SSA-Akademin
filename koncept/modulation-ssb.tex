\section{Sändningsslaget J3E (SSB)}
\harecsection{\harec{a}{1.8.3c}{1.8.3c}, \harec{a}{1.8.6c}{1.8.6c}, \harec{a}{1.8.7c}{1.8.7c}}
\index{Single Side Band (SSB)}
\index{J3E}
\index{SSB}
\label{modulation_ssb}

\subsection{Princip}

Som sagts är det onödigt att sända ut två sidband, eftersom båda innehåller
samma information.

Signaler med endast ett sidband och undertryckt bärvåg kan alstras på flera
sätt.
Numera är den så kallade filtermetoden i särklass vanligast och den enda som
behandlas här.

Bild~\ssaref{fig:BildII1-27} illustrerar sidband vid DSB-modulation.
Med filtermetoden blandas HF- och LF-signalerna i en speciell blandare.
Där undertrycks båda dessa signaler medan blandningsprodukterna med deras summa-
och skillnadsfrekvenser blir kvar, dvs. det övre och nedre sidbandet.

Utsignalen från blandaren benämns DSB-signal (Double Side Band).
Till skillnad från i A3E-signalen saknas dock bärvågen i DSB-signalen.
För att även undertrycka det ena sidbandet före sändningen följs blandaren
av ett bandpassfilter med bandbredd och frekvensläge för avsett sidband.

Den signal som sänds ut innehåller därför endast ett sidband (Single Side Band).

\mediumtopfig{images/cropped_pdfs/bild_2_1-28.pdf}{Sidbandsval vid SSB}{fig:BildII1-28}

\paragraph{Exempel}

Bild~\ssaref{fig:BildII1-28} illustrerar sidbandsval vid SSB-modulering.
Ett SSB-filter har ett passband av \SIrange{9000,3}{9003}{\kilo\hertz}.
Vid bärvågsfrekvensen \qty{9000}{\kilo\hertz} sträcker sig det övre sidbandet
från \SIrange{9000,3}{9003}{\kilo\hertz} och släpps igenom.
Däremot blir bärvågsfrekvensen undertryckt.

Det undre sidbandet \SIrange{8997}{8999,7}{\kilo\hertz} faller utanför filtrets
passband och blir också undertryckt.

Ska däremot det undre sidbandet kunna passera igenom samma filter, så måste
bärvågsfrekvensen höjas med \qty{3}{\kilo\hertz}, alltså till
\qty{9003}{\kilo\hertz}.
Då faller det undre sidbandet, \SIrange{9002,7}{9000,0}{\kilo\hertz} inom
filtrets passband.

Det övre sidbandet \SIrange{9003,3}{9006,0}{\kilo\hertz} faller nu utanför
passbandet och blir undertryckt.

%% k7per: Make this bigger.
\mediumtopfig{images/cropped_pdfs/bild_2_1-29.pdf}{Sidbandslägen vid SSB}{fig:BildII1-29}

Bild~\ssaref{fig:BildII1-29} illustrerar sidbandslägen vid SSB.
LF-signalens amplitud bestämmer amplituden på sidofrekvensen.

LF-signalens frekvens bestämmer sidofrekvensens avstånd från bärvågsfrekvensen
(bärvågen undertryckt).

Bandbredden på den utsända signalen är skillnaden mellan högsta och lägsta
modulerande frekvens i signalen:

till exempel \(b = \qty{3}{\kilo\hertz} - \qty{0,3}{\kilo\hertz} =
\qty{2,7}{\kilo\hertz}\)

\subsection{Fördelar med J3E-modulation}
Bra verkningsgrad vid J3E-modulation jämfört med vid A3E-modulation
(traditionell AM).
Effekten i det utsända sidbandet motsvarar den i ett av sidbanden vid A3E.
Hela den utsända effekten finns alltså i ett enda sidband,
som överför hela informationen.

I sändningspauserna sänds ingen effekt ut.
Bandbredden är mindre än hälften av den vid A3E.
Vid mottagning av en J3E-sändning (SSB) är det mindre besvär med
interferenstoner från J3E-sändningar på närliggande frekvenser, eftersom ingen
bärvåg och endast ett sidband sänds ut.

\subsection{Nackdelar med J3E-modulation}
J3E-modulation medför mera komplicerade apparater, både för mottagning och
sändning.
En J3E-signal blir förvrängd och hörs i fel tonläge om mottagaren inte är
inställd på exakt rätt frekvens.
