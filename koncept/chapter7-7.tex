\section{Brus och länkbudget}

\subsection{Allmänt}

Den mottagna signalens kvalite kan ofta sammanfattas med dess signal-brus
förhållande.
För att kunna estimera det behöver man dels förstå själva länk-budgeten som
ger en uppfattning om hur stark signal man får, men även förstå de olika
bridragen av brus som sätter det effektiva brusgolvet.

\subsection{Brus}
\textbf{HAREC a.\ref{HAREC.a.7.17}\label{myHAREC.a.7.17}, a.\ref{HAREC.a.7.18}\label{myHAREC.a.7.18}}
\index{brus}
\index{atmosfäriskt brus}
\index{brus!atmosfäriskt}
\index{galaktiskt brus}
\index{brus!galaktiskt}
\index{termiskt brus}
\index{brus!termiskt}

Det finns flera källor till brus, atmosfäriskt brus, galaktiskt brus samt
termiskt brus.

Atmosfäriskt brus (eng. atmospheric noise) uppstår på grund av blixtar som
sedan sprids.
Över hela jorden sker hela tiden blixtnedslag, och dess starka impulser sprider
sig precis som radiovågor och ger en grundläggande störning i kortvågsbandet.
Atmosfäriskt brus identifierades 1925 av Karl Jansky.

Galaktiskt brus (eng. galactic noise) kommer huvudsakligen från centrum av
Vintergatan, och är huvudsakligen termiskt brus från den stora ansamlingen av
stjärnor i mitten av Vintergatan.
Galaktiskt brus kommer från den delen av himlen som för stunden har mitten av
Vintergatan, så det är riktningskänsligt.

Termiskt brus är mottagarens interna brus, se \ref{termisktbrus}.
