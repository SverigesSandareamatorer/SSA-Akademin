\section{Fält}
\index{elektriskt fält (E)}
\index{magnetiskt fält (H)}
\index{fjärrfält}
\index{närfält}

Antennens uppgift är att omvandla den ledningsbundna signalen i matarkabel på
ett så effektivt sätt som möjligt till en elektromagnetisk våg som utbreder sig
i luften.

Den elektromagnetiska vågen uppträder inte direkt vid antennen utan uppstår
efter ett visst avstånd från antennen som man kallar fjärrfältet.
Detta sker genom växelverkan mellan det elektriska och magnetiska fältet som
antennen genererar.

Teorierna som beskriver hur denna växelverkar fungerar är komplicerade men det
viktiga att förstå är att det finns en gräns mellan vad man kallar fjärrfältet
längre bort från antennen och närfältet nära antennen.
Av denna anledning måste man nära antenner mäta både det elektriska och
magnetiska fältet för att utvärdera maximal fältstyrka.
I fjärrfältet kan man på grund av växelverkan mellan dem mäta antingen det ena
eller det andra.

Beroende på den antenntyp som genererar fältet är det antingen ett elektriskt
eller magnetiskt fält som dominerar närfältet.
Elektrisk fältdominans genereras av antenntyper som bygger på
spänningsskillnader (t.ex. dipol) och magnetisk fältdominans av antenner med
strömflöde (t.ex. små loopar).

Det elektriska fältet (E) anges i ''Volt per meter'' (\unit{\volt\per\metre})
och det magnetiska fältet (H) i ''Ampere per meter'' (\unit{\ampere\per\metre}).

I och med att fältet utbreder sig i luften åt alla håll samtidigt så avtar
styrkan i fältet.
Detta beror på att effekten som genererar fältyrkan täcker ett större och större
tänkt klot runt antennen ju längre bort man kommer.
Effekten per ytenhet blir då mindre ju längre från antennen man kommer.

Fältet avtar alltså i styrka på samma sätt som att jämföra ytan på två klot med
olika radier.
Genom matematisk analys av dessa klots ytor med olika radier kommer man fram
till att om avståndet dubbleras halveras fältstyrkan.

Det spelar ingen roll om antennen är helt rundstålande eller koncentrerar
effekten i en riktning så avtar fältet på samma sätt i alla fall.

Eftersom alla elektriska ledare kan betraktas som antenner kommer dessa att
kunna generera fält, oavsett om det är tänkt att det ska vara en antenn eller
inte.
Detta är tydligt då både apparater och ledningar kan leda högfrekvent ström och
genom detta generera fält.
Man bör ha detta i åtanke vid installation av matarledning till antennen för att
undvika att ström flyter tillbaka till stationen på utsidan av ledningen.
Även de apparater man använder för att generera de signaler man vill skicka ut
kan ha dålig skärmning och leda högfrekvent ström på utsidan.

Det finns alltså en risk att fältstyrkorna i närheten av sändare, och framför
allt slutsteg med tillhörande kablage, kan vara betydande.
