\section{Fält}
\index{elektriskt fält (E)}
\index{magnetiskt fält (H)}
För att ange nivån på det elektriska fältet (E) används enheten
''volt per meter'' (V/m).
Det magnetiska fältet (H) nivå anges i enheten ''ampere per meter'' (A/m).

Antennens uppgift är att så effektivt som möjligt omvandla den högfrekventa
strömmen i matarkabeln till en elektromagnetisk våg som utbreder sig i luften.

Den sammansatta elektromagnetiska vågen uppträder inte direkt vid antennen utan
uppstår i det som man kallar fjärrfältet.
Detta sker genom växelverkan mellan de elektriska och magnetiska fält som
utgår från antennen.
Teorierna som beskriver hur denna växelverkan fungerar är komplicerade
men det viktiga att förstå är att det finns en gräns mellan vad man
kallar fjärrfältet, längre bort från antennen och närfältet nära antennen.

\index{fjärrfält}
I fjärrfältet kan man tack vare växelverkan mellan det elektriska- och det
magnetiska fältet mäta vilket som helst av dem.
I och med att det elektromagnetiska fältet sprider ut sig över en större yta så
avtar styrkan i fältet med avståndet från antennen.
Det sammansatta elektromagnetiska fältet som passerat gränsen till fjärrfältet
avtar linjärt med avståndet, dubbleras avståndet halveras fältstyrkan.
Det spelar ingen roll om antennen är helt rundstrålande eller koncentrerar
effekten i en riktning, det elektromagnetiska fältet avtar på samma sätt.

\index{närfält}
I närfältet behöver man på grund av fältens komplicerade inbördes förhållande
mäta både det elektriska och det magnetiska fältet för att få en uppfattning
om storleken på det radiofrekventa fältet.
I antennens närhet varierar nivåerna på de olika fälten kraftigt och på vissa
punkter kan höga fältstyrkenivåer mätas upp.

Om antennen har stor utsträckning i förhållande till använd våglängd kan ibland
fjärrfältsformler användas för att överslagsmässigt beräkna fältstyrkenivå i
antennens närfält.
För kompakta antenner (t.ex. små loopar) krävs komplicerade beräkningar
med hjälp av antennsimuleringsprogram.

Beroende på den antenntyp som genererar fältet är det antingen ett elektriskt
eller magnetiskt fält som dominerar i närfältet.
Elektrisk fältdominans genereras av antenntyper som bygger på
spänningsskillnader (t.ex. dipol) och magnetisk fältdominans av antenner
med strömflöde (t.ex. små loopar).

Eftersom alla elektriska ledare kan betraktas som antenner kommer dessa att
kunna generera fält, oavsett om det är tänkt att det ska vara en antenn eller
inte.
Man bör ha detta i åtanke vid installation av matarledning till antennen för
att undvika att högfrekvent ström flyter tillbaka till stationen på utsidan av
ledningen.
Även de apparater man använder för att generera radiosignaler kan ha dålig
skärmning och därigenom leds högfrekvent ström till apparaternas utsida.

Det finns alltså en risk att fältstyrkorna kan vara betydande i närheten av
sändare och framför allt vid slutsteg med tillhörande kablage.
