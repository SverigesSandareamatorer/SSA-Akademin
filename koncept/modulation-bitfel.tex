\section{Bitfel -- detektion och korrigering}

Hittills har vi diskuterat digital modulation utan att ta hänsyn till störningar
och hur dessa påverkar våra överförda data.
Precis som vår CW eller SSB kan vara störd av atmosfäriska störningar, andra
sändare eller helt enkelt vara svaga signaler så att det interna bruset blir en
begränsning, så kommer mottagningen av digitala signaler att bli störd.
Vi ska titta på dessa grundläggande begrepp såsom bitfel, bitfelssannolikhet,
felupptäckt samt korrigering med återsändning eller korrigeringskoder.

\subsection{Bitfel}
\index{bitfel}
\index{bit error}

Av olika orsaker kommer en eller flera bitar ofta att bli fel.
Vi kallar varje sådant fel för att \emph{bitfel} (eng. \emph{bit error}).
Störningar kan göra att vi tolkar en symbol fel, vilket kan resultera i en eller
flera felaktiga bitar.

Om vi i till exempel 16QAM-koden i kapitel~\ssaref{QAM} får in +0.2 i I och +1.1
i Q, ser vi i tabell~\ssaref{tab:16QAM} att närmaste symbolen är symbol 5 med +1
i I och +1 i Q.
Vi skulle kunna anta att om I är större än 0 och mindre än 2, samt Q är större
än 0 och mindre än 2 så är symbol 5 den enda vettiga symbolen, och det är precis
den tolkning vi i allmänhet gör, för det är den symbolen vars avstånd är lägst
och därmed rimligast.
Det kan dock vara så att man egentligen sände symbol 9 med \num{-1} i I och +1 i
Q, och därmed fick för stor störning på I för att man ska tolka det som rätt
symbol.
Vi kommer då lägga ut 9 istället för 5, vilket innebär att två bitar har
ändrats.

Genom att granska tabell~\ssaref{tab:16QAM} vidare ser man att värdena för
I och Q för de olika symbolerna är gjorda så att minsta avstånd är 2 mellan
alla närliggande symboler, i respektive I- och Q-riktning.
Det förenklar tolkning av symbolerna.
Är dock störningen större än 1 i någon riktning kommer man tolka den symbolen
fel, och det kan då leda till 1 eller fler bitfel.

\subsection{Bitfelssannolikhet}
\index{bitfelssannolikhet}
\index{bit error rate}
\index{BER|see {bit error rate}}
\index{gaussiskt brus}
\index{brus!gaussiskt}
\index{gaussian noise}
\index{effektiv-värde}
\index{Root Mean Square}
\index{RMS|see {Root Mean Square}}
\index{Error Function (erf)}
\index{erf}

Om vi antar att vi inte har störning från några andra signaler, utan enbart har
brus som störning, så kan vi estimera \emph{bitfelssannolikheten} (eng.
\emph{bit error rate, BER}) ur hur starkt bruset är i förhållande till vårt
steg.
Eftersom bruset antas vara vitt brus, så har det egenskaperna av \emph{Gaussiskt
brus} (eng. \emph{Gaussian noise}).

Gaussiskt brus har en statistisk fördelning med hög sannolikhet nära
medelvärdet och avtar sedan med avståndet.
Sannolikheten att man tolkar en signal som vara på ena eller andra sidan av en
gräns beror på hur långt bort från medelvärdet den gränsen, ofta benämnd
kvantiseringsgränsen, är i förhållande till den effektiva värdet (eng.
\emph{Root Mean Square, RMS}) i amplitud hos bruset.
Detta kan uttryckas i form av den matematiska funktionen \emph{error function
(erf)}.

När gränsen är 1~sigma, det vill säga 1 gånger RMS-värdet för brusamplituden,
från medelvärdet så är det \qty{67}{\percent} sannolikhet att värdet ligger inom
gränsvärdet, det vill säga en bitfelssannolikhet på \qty{33}{\percent}.
Ligger det inom 2~sigma har sannolikheten ökat till \qty{97}{\percent}, en
bitfelssannolikhet på \qty{3}{\percent}, och vid 3 sigma är den
\qty{99,7}{\percent} med en bitfelssannolikhet på ringa \qty{0,3}{\percent},
vilket ofta används för många ingenjörsapplikationer.
Dock, för överföring av information har vi högre krav.
För en bitfelssannolikhet på \(10^{-12}\), ofta benämnt BER på 1E-12, behövs
det 14~sigma bort till gränsen, dvs. brusmängden får max vara \(1/14\) av
kvantiseringsgränsen.
Den råa radiokanalen uppvisar dock sällan så bra egenskaper, men det kan uppnås
i kabel och fiber.

\subsection{Detektion}
\harecsection{\harec{a}{1.8.10a}{1.8.10a}}
\index{bitfelsdetektion}
\index{paritet}
\index{CRC}
\label{bitfel_detektion}

Eftersom störningar förekommer och man har behov av lägre bitfelssannolikhet
än vad den råa kanalen medger är det lämpligt att identifiera när det har
blivit bitfel.
Detta kan utföras på många sätt, men ett sätt är att räkna fram checksummor som
skickas med datat.
Det kräver visserligen en del av informationsöverföringskapaciteten, men
tjänsten det medger är att försäkra sig om att informationen är rimligt korrekt.

En enkel form av checksumma är paritet, där bitarna i ett ord har summerats ihop
binärt (med XOR) för att bilda en checksumma.
I mottagaränden görs samma kombination och sedan jämförs det med paritetsbiten,
och om de överensstämmer så har inget bitfel upptäckts.
Denna enkla metod har en svaghet i att ett jämnt antal bitfel kommer att
kompensera varandra, varvid det döljer bitfel från upptäckt.
Det är med andra ord inte en särdeles stark checksumma.
Paritet används till exempel i seriekommunikation så som RS-232.

Ett flertal checksummor finns, för olika ändamål, olika mängd fel och olika
typer av fel.
För lite större meddelanden är det vanligt att summera bytes till en checksumma
antingen additivt eller med XOR.
För större meddelanden används en lite mer intrikat metod som heter Cyclic
Redundancy Check (CRC) där man återmatar överskjutande del på checksumman till
sig själv och får en starkare kod den vägen.
CRC används till exempel i Ethernet.

\subsection{Omsändning}
\harecsection{\harec{a}{1.8.10b}{1.8.10b}}
\index{omsändning}
\index{ARQ}
\index{TCP}

En enkel åtgärd att vidta när man konstaterat att ett block data man tagit emot
har fel, är att begära omsändning.
Genom att sändaren håller en buffert med meddelanden som den skickat, och
mottagaren meddelar sändare om den mottagit meddelandet eller behöver ha det
omskickat, så kan omsändning realiseras.
Automatisk omsändningsbegäran (eng. \emph{Automatic Repeat reQuest, ARQ}) är en
typ av protokoll som gör automatisk omsändningsbegäran om ett enskilt datablock,
även kallat paket, inte kommit fram rätt eller helt försvunnit.
Ett sådant protokoll är TCP, som ingår i internetsviten av TCP/IP-protokollet.

\subsection{Korrigeringskod -- FEC}
\harecsection{\harec{a}{1.8.10c}{1.8.10c}}
\index{korrigeringskod}
\index{felrättandekod}
\index{FEC}
\index{AMTOR}
\index{Hamming-koder}
\index{paritet}
\index{Reed-Solomon (RS)}

En annan form av korrigering är att helt enkelt skicka för mycket data redan
från början, som mottagaren kan använda för att korrigera meddelandet utan att
skicka någon begäran till sändaren.
Detta är praktiskt antingen om det skulle ta för mycket tid eller om det helt
enkelt inte finns någon kommunikation från mottagaren till sändaren, till
exempel för satellitmottagare.

En enkel form av felrättande kod används i AMTOR FEC, där man helt enkelt
sänder samma tecken två gånger.
Liknande används i Bluetooth där meddelandet sänds tre gånger, varvid man kan
göra majoritetsröstning.

Andra system för FEC är Hamming-koder, paritets-paket och Reed-Solomon (RS).
