\section{Sammanfattning}
Strålsäkerhetsmyndigheten (SSM) har i sina allmänna råd angett referensvärden
som ska begränsa allmänhetens exponering för elektromagnetiska fält (EMF).

Dessa begränsningar och sändaramatörens möjligheter att generera kraftiga
elektromagnetiska fält innebär att vi som sändaramatörer måste förstå
och kunna hantera området elektromagnetiska fält (EMF).

Alla sändande antenner kommer att ha ett elektromagnetiskt fält (EMF)
runt sig.
Detta elektromagnetiska fält (EMF) är beroende på vilken typ av antenn som
används och den signal som skickas in i antennen.
Hur man bedömer storleken på dessa fält är avgörande för att kunna
begränsa exponeringen av EMF från en amatörradiostation.

En egenkontroll bör genomföras för att kunna bedöma den fältbild som
amatörradioutövandet orsakar runt sin station.
Eftersom amatörradio är en experimentell verksamhet så måste alla förstå hur
olika förändringar i sin installation och användning påverkar denna fältbild.

Vilken metod man än väljer för sin egenkontroll är det lämpligt att
göra den tydligt och lättförståelig.
Detta är viktigt eftersom man bör spara sina resultat och då ha möjlighet att
göra om sin utvärdering när man har förändrat något eller några av de värden
som skulle kunna påverka resultatet.

\subsection{Praktisk hantering}
Vid all användning av amatörradioutrustning måste man göra en bedömning
av vilka fältstyrkor man genererar och vilka som kan bli exponerade.
Det kan vara frågan om människor i omedelbar närhet eller människor på
längre avstånd.
I alla fall bör man fundera på om man valt rätt sätt att generera den
elektromagnetiska fältstyrka som man behöver, eller om det finns ett bättre
och effektivare sätt som möjliggör att man når motstationen utan att onödigtvis
exponera någon annan för elektromagnetiska fält.

Det finns vissa installationer som man bör undvika och andra som kan
rekommenderas för att hålla nivåerna på exponering så låga som möjligt:

\begin{itemize}
\item Antenner som sitter nära människor, exempelvis balkongantenner, kan ge
  mycket högre exponering än antenner som sitter högt monterade i en mast.

\item Riktantenner för höga frekvenser har ofta hög förstärkning, och
  kan ge höga fältstyrkor i huvudriktningen.
  Då måste man se till att det inte är möjligt att rikta denna typ av antenn
  mot platser där människor kan exponeras.

\item Inomhusantenner hamnar alltid nära människor och bör enbart användas med
  låg effekt då de kan ge mycket hög exponering.
  De kommer också ta emot störningar från hemelektronik (nätadaptrar, datorer
  etc.) vilket också gör antennplaceringen mycket olämplig.

\item Antenner ovanför huskroppar bör endast användas med låg effekt.
  Trådantenner för lägre frekvenser rakt ovanför bostadshus kommer att
  vara nära människor i byggnaden.

\item Om man har behov av att använda hög effekt så måste man också se
  till att effekten används så bra som möjligt.
  Det är direkt olämpligt att kompensera en dålig antenn med högre effekt då
  det oftast resulterar i höga fältstyrkor på fel ställe.

\item Högre fältstyrka kan för det mesta enklast åstadkommas med en
  antenn som riktar signalen i den riktning man vill kommunicera.
  Det är oftast mycket dyrare och mer komplicerat att öka uteffekten för att nå
  samma resultat.

\item Osymmetriska antenner kan ge mantelströmmar i matningsledningen.
  Det innebär att en HF-ström flyter från antennen tillbaka på matarledningen
  och kan ge höga fältstyrkor längs hela kabellängden.
  Bättre är det då att använda symmetriska antenner, exempelvis en mittmatad
  halvvågsdipol.
  En strömbalun (även common mode choke, RF-choke) där antennen ansluts till
  matarledningen undertrycker denna HF-ström och därmed kommer
  matningsledningen sluta att agera radierande element, varvid fältstyrkorna
  längs matningsledningen sjunker.

\item Vissa antenner, så som T-antenn, använder dock obalansen då
  matningsledningen agerar radierande element.
  I dessa fall ska den delen av matningsledningen som agerar radierande element
  betraktas som sådant även i EMF-sammanhang och säkerhetsavstånd ska iakttas.
  Det är rekommenderat att använda en strömbalun för att isolera antennen från
  radiostationen med avseende på mantelburen HF-ström.

\item Även symmetriska antenner kan ha strömmar på utsidan av matarledningen.
  Dra därför matarledningen så långt bort från människor som möjligt.

\item Använd inte effektförstärkare eller antennavstämningsenhet utan
  hölje då fältstyrkorna runt utrustningen kan nå höga nivåer.

\item Vid antennplaceringar nära människor så kan det vara omöjligt att
  använda hög effekt.
\end{itemize}

Det finns som synes många sätt att göra rätt men också många sätt att göra fel
när det gäller att hantera den fältstyrka vi vill generera för att upprätthålla
radiokommunikation.
Innan man börjar sin amatörradiosändning är det viktigt att ha förståelse för
de fält som genereras och att kunna begränsa dem där så behövs.
Det går att finna mer information om elektromagnetiska fält på:

\begin{itemize}
\item Strålsäkerhetsmyndighetens webbplats där återfinns även SSMFS~2008:18~\cite{SSMFS2008:18}.

\item Arbetsmiljöverkets webbplats där finns även AFS~2016:3 Arbetsmiljöverkets
föreskrifter om elektromagnetiska fält och allmänna råd om tillämpningen av
föreskrifterna.

\item Folkhälsomyndighetens webbplats.

\item Federal Communications Commission (FCC) OET bulletin 65 supplement B~\cite{OETbul65b}.

\item EU-direktiv 1999/519/EG~\cite{1999/519/EG}
\end{itemize}

