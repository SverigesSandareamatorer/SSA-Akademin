\section{Resistorn}
\harecsection{\harec{a}{2.1}{2.1}}
\index{resistor}
\index{resistans}
\label{sec:resistorn}

\subsection{Allmänt}

Strömkretsar består av komponenter med olika egenskaper.
Den vanligaste egenskapen, åtminstone i likströmskretsar, är resistansen.
För att få avsedd funktion, så anpassar man resistansen i komponenterna.

\textbf{Exempel:} En krets med strömkälla, lampa, kopplingsledningar och smältsäkring.
Kopplingsledningarna mellan komponenterna bör ha låg resistans och därför lågt
spänningsfall (små förluster).
Lampan ska däremot ha hög resistans och därmed höga förluster för att kunna bli
het och lysa.
Smältsäkringen ska skydda ledningarna från för hög ström.
Säkringen ges därför en resistans som gör att den smälter när strömmen
överstiger ett tillåtet värde.

Som hjälpmedel för att fördela spänningar och strömmar i en krets används
en komponenttyp kallad \emph{resistor}.
Dess utmärkande egenskap är \emph{resistans} (eng. \emph{resistance}) --
även kallad ohmskt motstånd.

\subsection{Enheten ohm}
\harecsection{\harec{a}{2.1.1}{2.1.1}}
\index{ohm (\unit{\ohm})}
\index{enheter!ohm (\unit{\ohm})}
\index{symbol!\(R\) resistans}
\label{enheten_ohm}

%%(Se även kapitel \ssaref{ch:ellaera}.)

Resistansen mellan två punkter i en strömkrets är 1~ohm som även skrives
\qty{1}{\ohm} (uttalas ''en åm''), när spänningen \qty{1}{\volt} mellan
punkterna gör att en ström av \qty{1}{\ampere} (en ampere) flyter i kretsen.

Inom elektroniken används höga resistansvärden och därför även följande
multipler av enheten

\begin{center}
\begin{tabular}{lll}
  1 kiloohm & (\qty{1}{\kilo\ohm}) & = \(10^3\) ohm \\
  1 megaohm & (\qty{1}{\mega\ohm}) & = \(10^6\) ohm \\
\end{tabular}
\end{center}

\subsection{Resistans i strömledare}
\harecsection{\harec{a}{2.1.2}{2.1.2}}
\index{resistivitet}
\index{specifik resistans}
\index{symbol!\(\rho\) resistivitet}

För att bestämma resistansen i exempelvis en tråd, behöver man veta dess 
resistivitet, tvärsnittsyta, längd och temperatur.

\emph{Resistivitet} (eng. \emph{resistivity}) är ett materials
strömledningsegenskaper.
Ett annat namn för resistivitet är \emph{specifik resistans}.
Symbolen för resistivitet är \(\rho\) (uttalas ''rå'').
Formeln för resistivitet är:
%%
\[\rho = \dfrac{R A}{l}\qquad \left[\dfrac{ohm \cdot mm^2}{m}\right]\]
%%
där resistansen \(R\) på en längd \(l\) av en strömledare med en
genomsnittsarea \(A\) (som oftast anges i kvadratmillimeter).

Resistiviteten för material finns ofta att finna i tabeller i formelsamlingar.
I tabellen~\ssaref{table:metaller} finns ett antal vanliga metallers
resistivitet angivna.

Följande formel gäller för beräkning av resistansen i en strömledare med linjär
ström/spänningskaraktär.
%%
\[\begin{array}{c}
    R = \rho \dfrac{l}{A} \qquad \left[\rho = \frac{\unit{\ohm} \cdot A}{m} \right] \qquad l=\text{meter}; A=\unit{\milli\metre\squared}
\end{array}\]
%%
\noindent
\begin{exempelbox}
Beräkna resistansen för denna ledare.

\(l = \qty{4}{\metre}\) koppartråd

\(A = \qty{2}{\milli\metre\squared}\)

\(\rho \text{ (koppar)} = 0,017\)
\tcblower
\noindent
\[\begin{array}{c}
R = \dfrac{\rho \cdot l}{A} \qquad R = \dfrac{0,017 \cdot 4}{2} = \qty{0,034}{\ohm}
\end{array}\]

\noindent
\textbf{Not.} \emph{Förväxla inte A [tvärsnittsytan] i denna formel med enheten ampere.}
\end{exempelbox}

\subsection{Resistiva material}

Resistorer kan utföras med olika typer av resistiva material, vilket bestämmer
användningsområdet.
En resistor, vars resistans är oberoende av ström, spänning och annan yttre
påverkan, till exempel temperatur och ljus, sägs ha linjär karaktär.
Om resistansen däremot beror av yttre påverkan sägs resistorn ha olinjär
karaktär.
Man skiljer mellan tre huvudgrupper av resistiva material.
Det kan vara en kropp av pressat kol eller ett ledande ytskikt på ett isolerande
underlag eller en metalltråd på en isolerande stomme.
På senare tid har det tillkommit resistornät med integrerade resistorer, det
vill säga flera resistorer av resistiva skikt på ett gemensamt isolerande
underlag.
Här beskrivs i korthet olika typer av resistorer.

\subsection{Utförandeformer}

Resistorer kan utföras med fast eller ställbart resistansvärde.
Här följer först en översikt över resistorer med olika resistiva material och
fast resistansvärde.

\subsection{Fasta resistorer med linjär karaktär}
\label{fasta_resistorer_linjära}

\subsubsection{Massaresistor}
\index{massaresistor}
\index{resistor!massa-}

Det resistiva materialet består av kolmassa med bindemedel (kolkomposit).
Massan är bakad till en stav eller ett rör.
Anslutningsledningarna är inbakade i materialet.
\emph{Massaresistorer} är lämpliga för lik- och växelströmskretsar med
låga krav på temperaturberoende och egenbrus.
Den homogena kroppen gör att egeninduktansen är låg.
Å andra sidan uppstår vid höga frekvenser en skineffekt, det vill säga en
strömkoncentration vid ytan, som medför viss resistansökning.

\subsubsection{Kolfilmsresistor}
\index{kolfilmresistor}
\index{resistor!kolfilm-}

Det resistiva materialet består av ett kolskikt, som genom förångning överförts
till ett keramiskt rör.
Resistansen bestäms av tjockleken på skiktet samt av spiralformade spår i
detta.
Genom spiraliseringen tillförs en induktans, som dock i någon mån uppvägs av
egenkapacitansen.

\subsubsection{Metallfilmresistor}
\index{metallfilmresistor}
\index{resistor!metallfilm-}

I denna typ är kolfilmen ersatt av ett metallskikt.
Eftersom egenkapacitansen är liten är typen lämpad för höga frekvenser.

\subsubsection{Tjockfilmsresistor}
\index{tjockfilmsresistor}
\index{resistor!tjockfilm-}

Det resistiva materialet består av en film av bland annat metalloxid, som
screentrycks på ett keramiskt underlag.
Typen har god tålighet mot pulser och höga temperaturer, men har relativt högt
egenbrus.
Ytmonterade resistorer är oftast tillverkade av tjockfilm.

\subsubsection{Tunnfilmsresistor}
\index{tunnfilmresistor}
\index{resistor!tunnfilm-}

Det resistiva materialet består av en tunn metallfilm, som genom förångning
överförts till ett underlag av glas eller keramik. Denna resistortyp har över
lag god stabilitet och används ofta i apparater med hög precision.
Egenskaperna vid höga frekvenser är dock inte så goda.

\subsubsection{Metalloxidresistor}
\index{metalloxidresistor}
\index{resistor!metalloxid-}

Denna resistortyp har ett spiralformat skikt av metalloxid.
Temperatur- och spänningsberoendet är måttligt.
Tåligheten mot pulser och höga temperaturer är stor.
Typen kan i någon mån ersätta trådlindade resistorer.

\subsubsection{Resistornät}
\index{resistornät}
\index{resistor!-nät}

Resistornät (integrerade resistorer) består av flera resistiva skikt på ett
gemensamt isolerande underlag, det vill säga en liknande teknik som för tjock- 
och tunnfilmsresistorer.

\subsubsection{Trådlindad resistor}
\index{trådlindad resistor}
\index{resistor!trådlindad}

Det resistiva materialet är en metalltråd lindad på en stomme som tål hög
temperatur.
Stommen kan vara av keramik, glas eller liknande.
Tåligheten mot pulser och höga temperaturer är stor.

\subsection{Fasta resistorer med olinjär karaktär}
\harecsection{\harec{a}{2.1.3}{2.1.3}}
\index{olinjära resistorer}
\index{resistor!olinjär}
\label{fasta_resistorer_olinjära}

Vanligast är att materialet i resistorer har linjär ström- och
spänningskaraktär, men det finns även sådana med olinjär karaktär.
I resistorer med olinjär karaktär är det ingående materialet av halvledartyp.

\subsubsection{Spänningsberoende resistor -- Voltage Dependent Resistor (VDR)}
\index{spänningsberoende resistor}
\index{resistor!spänningsberoende}
\index{VDR}
\index{resistor!VDR}

Linjära resistorer påverkas knappast av den pålagda spänningen.
Resistorer av kiselkarbid har däremot en hög resistans vid låg spänning och
omvänt en låg resistans vid hög spänning.
Sådana spänningsberoende resistorer används till exempel för begränsning av
spänningstoppar.

\subsubsection{Ljusberoende resistor, fotoresistor -- Light Dependent Resistor (LDR)}
\index{ljusberoende resistor}
\index{resistor!ljusberoende}
\index{fotoresistor}
\index{resistor!foto-}
\index{LDR}
\index{resistor!LDR}

Ledningsförmågan i halvledare påverkas inte bara av värme utan även av ljus.
Halvledare av germanium och särskilt sammansatta halvledare av kadmiumoxid,
blysulfid och indiumantimonid har särskilt stor ljuskänslighet. Kadmiumsulfid
är känsligast för synligt ljus medan andra material är känsligast i det
infraröda området.

\subsubsection{Magnetfältberoende resistor (fältplatta)}
\index{magnetfältberoende resistor}
\index{resistor!magnetfältberoende}
\index{hallresistor}
\index{resistor!Hall-}

Resistansen ökar med längden på strömledaren. Denna egenskap används i
\emph{magnetfältsberoende fältplattor} som utnyttjar \emph{halleffekten}, även
kända som \emph{hallresistor}. En sådan består av en keramisk bärarplatta med
en yta av indiumantimonid.
I ytan är ytterst smala parallella metallbanor inlagda på ett avstånd av någon
\unit{\micro\metre}.
Normalt går strömmen kortaste vägen tvärs över banorna, men när ett magnetfält
träffar vinkelrätt mot plattans yta avlänkas elektronerna.
De får då längre väg över till nästa metallbana och den totala resistansen
ökar.

\subsubsection{Temperaturberoende resistor}
\index{temperaturberoende resistor}
\index{resistor!temperaturberoende}
\index{NTC}
\index{resistor!NTC}
\index{PTC}
\index{resistor!PTC}

Se nedan om NTC och PTC i resistorer.

\subsection{Temperaturkoefficienten för resistorer}
\index{temperaturkoefficient i resistor}
\index{resistor!temperaturkoefficient}
\index{NTC}
\index{resistor!NTC}
\index{PTC}
\index{resistor!PTC}
\label{resistor_temperaturkoefficient}

Resistansen i ingående material påverkas av temperaturen, varvid det skiljer
mellan materialen.

Amorft kol och de flesta halvledande material leder bättre när de är varma -- de
har en negativ temperaturkoefficient (NTC). Sådana material finns till exempel i
dioder och transistorer.

Däremot leder metaller och speciella halvledarmaterial bättre när de är kalla
-- de har en positiv temperaturkoefficient (PTC). Glödtråden i glödlampor och
elektronrör är resistorer med positiv temperaturkoefficient (PTC).

I vissa metallegeringar kan resistansen däremot vara nästan konstant vid
varierande temperatur.
Ett exempel är konstantan, som är en legering mellan koppar, nickel och mangan.

Alla material har en temperaturkoefficient, som anger hur mycket resistansen
ändras per grad. Resistansen vid någon annan temperatur kan därför beräknas med
följande formel, där man sätter in begynnelsetemperaturen [\(\vartheta\)]
(\unit{\degreeCelsius}), temperaturändringen [\(\Delta \vartheta\)] och
temperaturkoefficienten [\(\alpha\)].
%%
\[R_{varm} = R_{kall} \pm \alpha \cdot \Delta \vartheta \cdot R_{kall}\]
%%
Resistansändringen är ledet
%%
\[ \Delta R = \pm \alpha \cdot \Delta \vartheta \cdot R_{kall}\]
%%
Temperaturkoefficienten kan vara positiv (PTC) eller negativ (NTC).
I principscheman har PTC- respektive NTC-resistorer symboler som i bild~\ssaref{fig:BildII2-1}.

\mediumfig{images/cropped_pdfs/bild_2_2-01.pdf}{Schemasymboler för resistorer}{fig:BildII2-1}

\subsection{Variabla resistorer}
\index{variabla resistorer}
\index{resistor!variabel}

En resistor kan även utföras med variabelt resistansvärde. Då används endast
den andel av det resistiva materialet som finns mellan en resistors ena ände
och ett uttag någonstans mellan ändarna. En sådan anordning kallas för reostat.
Om en variabel resistor används som spänningsdelare kallas den för
potentiometer.

I en potentiometer används dels hela resistansen mellan ändpunkterna och dels
andelen mellan uttaget och någon av ändpunkterna.
Uttagets mekaniska utförande beror oftast av hur bekvämt inställningen ska
kunna ske.
En potentiometer, där det resistiva materialet är lagt på en cirkulär bana och
uttaget är fäst vid en axel i banans centrum, medger enkel inställning med
mejsel, ratt eller liknande.
Ett enklare slags uttag är en släpkontakt eller ett spännband som kan flyttas
utmed en stavformad resistor.

\subsubsection{Resistiva material i variabla resistorer}

Banan i en variabel resistor består i princip av liknande resistiva material som
i en fast resistor.
Billigast och enklast är en bana av kol, som är tryckt på ett enkelt underlag.
Nackdelar är låg effekttålighet, dålig upplösning och linjäritet, högt brus och
kort livslängd. Fördelen är lågt pris.
Bättre än en kolbana är en bana av kolkomposit, det vill säga kolpulver med
bindemedel, som är tryckt på ett underlag.
Nackdel är högre pris och låg effekttålighet, medan fördelarna är god
upplösning, lågt brus och lång livslängd.
Vill man ha god effekttålighet och temperaturstabilitet, utöver kolkompositens
egenskaper, så erbjuder en bana av cermet sådana fördelar.
En cermetbana består av en blandning av metaller och keramik, som trycks på ett
underlag.
Trådlindad bana har främst god tålighet mot hög effekt.
Tålighet vid hög ström genom uttaget är en annan fördel.

\subsubsection{Linjära och olinjära potentiometrar}

En potentiometers resistansändring som funktion av uttagets rörelseväg utmed
resistansbanan kan beskrivas med en kurva.
Kurvformen kan utföras linjär, logaritmisk, eller på något annat sätt.
Olinjära kurvor består oftast av en följd av linjära segment, som tillsammans
någorlunda motsvarar den önskade olinjära formen.

\subsection{Effektutveckling i resistorer}
\harecsection{\harec{a}{2.1.4}{2.1.4}}
\index{effektutveckling i resistorer}
\index{resistor!effektutveckling}

I resistorer utvecklas värme av den ström som flyter igenom dem.
Värmeutvecklingen sker enligt Joules lag, som återges i
kapitel~\ssaref{ch:ellaera}.
Hur mycket effekt i form av värme som strålas ut från resistorn beror på
storleken på dess yta och egentemperatur samt på omgivningens temperatur.
Det finns en övre gräns för hur mycket värme det ingående materialet tål innan
det förstörs och eventuellt fattar eld.
En resistors effekttålighet framgår i vissa fall av påstämplade värden.
I övriga fall är man hänvisad till kataloguppgifter eller en bedömning, som
eventuellt kan grundas på höljets utseende och dimensioner.

\subsection{Standardiserade komponentvärden}
\index{resistor!standardiserade värden}

Resistorer tillverkas vanligen med standardiserade värden från en talserie.

\subsection{Märkning av resistorer}
\index{färgmärkning}
\index{färgmärkning!resistor}
\index{färgmärkning!kapacitans}
\label{subsec:faergmaerkning}

Resistorer märks med hjälp av siffror och bokstäver eller med en färgkod så att
resistorns huvuddata kan avläsas.
Ofta finns märkningen förklarad i komponentleverantörernas kataloger.

\subsubsection{Färgmärkning av resistorer}

Ett vanligt sätt att märka resistorer är genom att ha färger på ringar runt
kroppen.
Detta var vanligare på den tiden man hade hålmonterade resistorer och i dag med
ytmonterade brukar man i stället använda siffror tryckta på motståndskroppen.

Färgkoden är dock bra att känna till för att kunna identifiera resistorer
och ibland även andra komponenter.

Färgkoden består av tre olika scheman, de kan finnas 4, 5 eller 6 band runt
komponenten.
Första banden ger värdet hos komponenten och det två sista banden har särskild
betydelse, det näst sista är en multiplikator och det sista bandet är
toleransen.
Ofta är det sista bandet också tryckt med en viss distans från de andra banden.

Om färgkoden har n stycken band kan man beskriva den med
tabell~\ssaref{tab:rcolors} nedan.

\begin{table}[H]
\begin{tabular}{lrrr}
	\textbf{Färg}    & \textbf{Sifferkod} &     \textbf{Multiplikator} 
	&     \textbf{Tolerans} \\ \hline \hline
	Svart   &         0 &    $10^0$ &              \\ \hline
	Brun    &         1 &    $10^1$ &    $\pm 1\%$ \\ \hline
	Röd     &         2 &    $10^2$ &    $\pm 2\%$ \\ \hline
	Orange  &         3 &    $10^3$ &              \\ \hline
	Gul     &         4 &    $10^4$ &              \\ \hline
	Grön    &         5 &    $10^5$ &  $\pm 0,5\%$ \\ \hline
	Blå     &         6 &    $10^6$ & $\pm 0,25\%$ \\ \hline
	Violett &         7 &    $10^7$ &  $\pm 0,1\%$ \\ \hline
	Grå     &         8 &    $10^8$ & $\pm 0,05\%$ \\ \hline
	Vit     &         9 &    $10^9$ &              \\ \hline
	Guld    &           & $10^{-1}$ &    $\pm 5\%$ \\ \hline
	Silver  &           & $10^{-2}$ &   $\pm 10\%$ \\ \hline
	Saknas  &           &           &   $\pm 20\%$ \\ \hline
\end{tabular}
\caption{Färgmärkning av resistorer och deras betydelse}
\label{tab:rcolors}
\end{table}

\begin{exempelbox}
En resistor har färgbanden: gul, violett, orange, silver.
\tcblower
Första är siffran 4, nästa är siffran 7 och den tredje orange är
multiplikatorn $10^3$.
Resultatet blir då 47\,kOhm.
Till sist har vi toleransen som är silver och innebär $\pm 10\%$.
\end{exempelbox}
