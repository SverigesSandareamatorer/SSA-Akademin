% Kapitel 1 Ellära
\chapter{Ellära}
\label{ch:ellaera}

% Avsnitt 1.1 Elektriska grundbegrepp
\input{koncept/ellaera--elektriska-grundbegrepp}
% Avsnitt 1.2 Elektriska kraftkällor
\input{koncept/ellaera--elektriska-kraftkallor}
% Avsnitt 1.3 Elektriskt fält
\input{koncept/ellaera--elektriskt-faelt}
% Avsnitt 1.4 Magnetiskt fält
\input{koncept/ellaera--magnetiskt-faelt}
% Avsnitt 1.5 Elektromagnetiska vågor
\input{koncept/ellaera--elektromagnetiska-vaagor}
% Avsnitt 1.6 Sinusformade signaler
\input{koncept/ellaera--sinusformade-signaler}
% Avsnitt 1.7 Icke sinusformade signaler
\input{koncept/ellaera--icke-sinusformade-signaler}
% Avsnitt 1.8 Effekt och energi
\input{koncept/ellaera--effekt-och-energi}
%
%
% Kapitel 2 Komponenter
\chapter{Komponenter}
\label{ch:komponenter}
\harecsection{\harec{a}{2}{2}}
\index{komponenter}

Komponenter är samlingsnamnet för de delar som tillsammans eller var för sig
bestämmer hur spänning och ström påverkar funktionen i en elektrisk eller
elektronisk apparat.
För att kunna förstå de enskilda komponenternas funktion är det oftast
nödvändigt att studera deras uppbyggnad.

% Avsnitt 2.1 Resistorn
\input{koncept/komponenter-resistorn}
% Avsnitt 2.2 Kondensatorn
\input{koncept/komponenter-kondensatorn}
% Avsnitt 2.3 Induktorn
\mediumfig{images/cropped_pdfs/bild_2_2-03a.pdf}{Försök 1 med induktion}{fig:BildII2-3a}
\smallfig{images/cropped_pdfs/bild_2_2-03b.pdf}{Försök 2 med induktion}{fig:BildII2-3b}

\section{Induktorn}
\harecsection{\harec{a}{2.3}{2.3}}
\index{induktor}

\subsection{Allmänt}
\label{induktor_allmänt}

När elektrisk ström flyter genom en ledare alstras ett magnetfält omkring den.
Så snart strömmens styrka eller riktning ändras uppstår en motsvarande så kallad
elektromotorisk kraft (EMK) som motverkar ändringen.
Kraften finns i magnetfältet i form av lagrad magnetisk energi.


\subsection{Självinduktion -- induktans}
\harecsection{\harec{a}{2.3.1}{2.3.1}}
\index{induktans}
\index{självinduktion}
\index{elektromotorisk kraft (EMK)}
\index{EMK}
\index{induktor}

Magnetfältets förmåga att alstra en motverkande EMK kallas
\emph{självinduktion} (eng. \emph{self inductance}) eller
\emph{induktans} (eng. \emph{inductance}).
%% k7per: Macro for latin?
Ordet induktans kommer från latinets \emph{inducere}, som betyder införa.

När en ledare som ingår i en sluten krets rör sig i ett magnetfält, kommer
en ström att flyta genom ledaren på grund av den EMK (spänning) som alstras.
Varje ändring av strömmen motverkas av det magnetfält som strömmen själv
alstrar.

När det uppstår självinduktion i en ledare kallas ledaren \emph{induktor}
(eng. \emph{inductor}).
Självinduktionen är jämnt utbredd över ledarens hela längd. När ett större
induktansvärde behövs på något särskilt ställe i strömkretsen, kan ledarens
längd ökas just där och lindas upp till en spole med lämplig form.
Hela spolen kallas då för induktor.

När ett motverkande magnetiskt fält alstras omkring en ledare genom att strömmen
i den ändras, påverkas kretsens egenskaper och därmed dess utformning på olika
sätt.
Vid snabba strömändringar, exempelvis vid hög frekvens, är motverkan större än
vid långsamma ändringar.
Vid konstant likström uppstår däremot ingen motverkan -- självinduktion.

Induktansen är efter resistansen och kapacitansen den vanligaste egenskapen i
en strömkrets.

\subsection{Försök med induktion}

%\mediumfig{images/cropped_pdfs/bild_2_2-03.pdf}{Försök med induktion}{fig:BildII2-3}

\paragraph{Försök 1:}
I bild~\ssaref{fig:BildII2-3a} är ett känsligt vridspoleinstrument kopplat
till en induktor.
Instrumentet bör ha noll på skalans mitt, så att strömriktningen syns.
En permanentmagnet används för att visa att självinduktion uppstår när
magneten förs fram och tillbaka genom induktorn.

Instrumentet ger utslag när magneten är i rörelse.
Utslaget blir större vid snabbare hastighetsändring.
Utslagsriktningen växlar när magneten förs in i respektive dras ut ur induktorn
-- det uppstår en växelström.

En växelspänning uppstår över induktorn även när den ingår i en strömkrets som
sluts och bryts -- alltså utan en magnet som rör sig.

\paragraph{Försök 2:}
I bild~\ssaref{fig:BildII2-3b} har permanentmagneten bytts mot ännu en
induktor.
Utöver den första induktorn, som vi nu kallar sekundärlindning, kallar vi den
nya induktorn för primärlindning.

När vi släpper ström genom primärlindningen alstrar den ett magnetfält.
Först är strömmen noll för att sedan ändras till ett högt värde och därefter
återgå till noll. Det blir en strömstöt.

Varje ändring alstrar en mot-EMK, som bygger upp ett magnetfält, först i en
riktning och sedan i den andra. I båda fallen passerar fältet genom båda
lindningarna. Fältet från primärlindningen inducerar en spänningsstöt i
sekundärlindningen. Stöten har en riktning när primärlindningens strömkrets
sluts och motsatt riktning när den bryts -- en växelspänning alstras.
När sekundärlindningen ingår i en sluten krets uppstår en växelström genom
sekundärlindningen.

\smallfig{images/cropped_pdfs/bild_2_2-03c.pdf}{Försök 3 med induktion}{fig:BildII2-3c}
\paragraph{Försök 3:}
I bild~\ssaref{fig:BildII2-3c} ställer vi oss frågan vad som händer när
primärlindningen i försök 2 ansluts till en växelspänning, till exempel
med nätfrekvensen \qty{50}{\hertz}.
Använd för säkerhets skull en skyddstransformator mellan nätet och lindningen!

I sekundärlindningen uppstår då spänningsstötar vars polaritet i detta fall
växlar 100 gånger per sekund.
Det uppstår alltså en växelspänning över sekundärlindningen och om denna ingår i
en sluten strömkrets uppstår det en motsvarande växelström.

\mediumfig{images/cropped_pdfs/bild_2_2-04.pdf}{Schemasymboler för induktorer}{fig:BildII2-4}

\subsection{Olika utföranden}
\index{drossel}

Bild~\ssaref{fig:BildII2-4} visar schemasymboler för ett antal vanliga induktorer.
Utöver dessa finns elektromagneter, drosslar, induktorer för resonanskretsar,
ramantenner och så vidare.

En drossel (eng. \emph{choke}) är en induktans, ofta lindad kring en 
magnetisk 
kärna, med uppgift att begränsa strömmen i en ledare.

\subsection{Enheten henry (H)}
\harecsection{\harec{a}{2.3.2}{2.3.2}}
\index{henry (H)}
\index{enheter!henry (H)}
\index{symbol!\(L\) induktans}
\label{enheten_henry}

Måttenheten för självinduktion är \emph{henry} (\unit{\henry}).
1~henry (\qty{1}{\henry}) är självinduktionen i en induktor som alstrar en
motspänning av 1~volt vid en strömändring av 1~ampere under 1~sekund.
I formler betecknas induktans med symbolen L.
Sambandet är:
%%
\[\textit{volt} = \textit{henry} \cdot \textit{ampere}/\textit{sekund}\]
%%
\qty{1}{\henry} är en stor måttenhet.
För elektroniktillämpningar används därför ett mer hanterligt format.
Se bilaga~\ssaref{app:mattenheter}.

\noindent\textbf{Exempel:}

\begin{center}
\begin{tabular}{ll}
\qty{1}{\henry} & = \qty{1000}{\milli\henry} \\
\qty{1}{\milli\henry} & = \(1 \cdot 10^{-3}\)\,H \\
\qty{1}{\milli\henry} & = \qty{1000}{\micro\henry} \\
\qty{1}{\micro\henry} & = \(1 \cdot 10^{-3}\)\,mH = \(1 \cdot 10^{-6}\)\,H
\end{tabular}
\end{center}

\subsection{Hur induktansen påverkas}
\harecsection{\harec{a}{2.3.3}{2.3.3}}
\index{permeabilitet}
\index{relativa permeabiliteten}
\index{symbol!\(\mu_0\) permeabilitetskonstanten}
\index{symbol!\(\mu_r\) relativa permeabiliten}

Induktansen beror på induktorns mekaniska dimensioner, antalet lindningsvarv och
materialet i kärnan.

Induktansen i en cylindrisk induktor är proportionell mot tvärsnittsytan, omvänt
proportionell mot längden och proportionell mot kvadraten på lindningsvarvtalet.

Induktansen ökar om induktorn förses med en kärna av järn och minskar med en
kärna av omagnetisk, ledande metall, till exempel koppar, mässing eller
aluminium.

Precis som för kondensatorn har materialet i en induktors kärna betydelse,
då dess \emph{permeabilitet} kan anta olika värden. Den absoluta permeabiliteten
\(\mu\) brukar delas upp i permeabiliteten för vakuum \(\mu_0\) och den
\emph{relativa permeabiliteten} \(\mu_r\) som gives av
%%
\[\mu = \mu_0\mu_r\]
%%
Den relativa permeabiliteten går att hitta i tabeller och varierar med material.
Permeabiliteten för vakuum är definierad som
%%
\[\mu_0 = 4\pi 10^{-7} \approx 1,256637 \cdot 10^{-6}\]

\subsection{Induktiv reaktans}
\harecsection{\harec{a}{2.3.4}{2.3.4}}
\index{induktiv reaktans}
\index{reaktans!induktiv}
\index{symbol!\(X_L\) induktiv reaktans}
\label{induktiv_reaktans} 

Till skillnad från när en resistor ansluts till en spänning, så blir
strömökningen i en induktor fördröjd. Orsaken är att en induktor inte bara har
en resistans, vilken ju inte påverkas av strömvariationer, utan har även en
\emph{induktiv reaktans} (eng. \emph{inductive reactance}) \(X_L\).
Ordet reaktans kommer från latinets re (åter) agere (verka).

\emph{Reaktans} -- växelströmsmotstånd eller skenbart motstånd -- uppträder så
länge som strömmen genom induktorn ändras.
En induktor gör således också motstånd mot varje strömändring och detta motstånd
ökar med ökande ändringshastighet.

En fullbordad pendling i en växelström kan ses som ett varv i en cirkel --
\ang{360} -- och en fullbordad sådan pendling kallas en period.

En period motsvarar omkretsen i en cirkel med radien r, där omkretsen är
\(2 \cdot \pi  \cdot r\). När strömmen växlar 1 gång per sekund har
pendlingen en frekvens [f] av 1~hertz [Hz].
Vid 50 växlingar per sekund har pendlingen en frekvens av \qty{50}{\hertz}.

Den \emph{Induktiva reaktansen \(X_L\)} -- växelströmsmotståndet i en induktor
-- är en funktion av strömmens så kallade vinkelhastighet
\(\omega = 2 \cdot \pi  \cdot f\) och av induktansen L.

Den induktiva reaktansen är proportionell mot strömmens frekvens och mot
induktorns induktansvärde.
Inga förluster uppstår i en ideal induktor, det vill säga en induktor som
teoretiskt saknar resistans.
Sambandet är:
%%
\[X_L = 2\pi fL = \omega L\]
\[X_L [\unit{\ohm}] \quad f [\unit{\hertz}] \quad L [\unit{\henry}]\]
%%
\begin{exempelbox}
\[L = \qty{1}{\henry} \quad f = \qty{50}{\hertz} \quad X_L = ?\]
\tcblower
\[X_L = 2\pi fL = 2\pi \cdot 50 \cdot 1 = \qty{314}{\ohm}\]
\end{exempelbox}

\begin{exempelbox}
\[L = \qty{1}{\henry} \quad f = \qty{5}{\kilo\hertz} \quad X_L = ?\]
\tcblower
\[X_L = 2\pi fL = 2\pi  \cdot 5 \cdot 10^3 \cdot 1 = \qty{31400}{\ohm}\]
\end{exempelbox}

\subsection{Fasförskjutning mellan spänning och ström i en induktor}
\harecsection{\harec{a}{2.3.5}{2.3.5}}
\index{induktor!fasförskjutning}

Med fasförskjutning menas den tidsmässiga förskjutningen mellan ström- och
spänningsförlopp.
Strömmen genom en induktor når inte sitt toppvärde samtidigt
som spänningen över den.
Orsaken är växlingarna mellan elektrisk och magnetisk energi i induktorn.
Detta illustreras i bild~\ssaref{fig:BildII3-11}.

I en ideal induktor är spänningen fasförskjuten \ang{90} före strömmen.
I praktiken är dock förskjutningen något mindre än \ang{90} på grund av
resistansen i induktorn.

\subsection{Q-faktor -- godhetstal}
\harecsection{\harec{a}{2.3.6}{2.3.6}}
\index{Q-faktor!induktor}

\emph{Q-faktorn} kan avse två olika saker, som inte ska förväxlas.
Dessa är Q-faktorn för en komponent respektive Q-faktorn för en hel strömkrets.

Q-faktorn för en induktor är kvoten av dess reaktans och dess serieresistans.

\[Q_{\text{komponent}} = \dfrac{X_{\text{komponent}}}{R_{\text{komponent}}}\]

Q-faktorn för en hel resonanskrets beror däremot på bredden på det
frekvensband som en viss komponentkombination ger.
Q-faktorn för en resonanskrets är därför ett mått på dess
selektivitet (se kapitel~\ssaref{Q-faktor}).

Q-faktorn för en ingående komponent påverkar Q-faktorn för en hel krets.
Däremot gäller inte det omvända.

\subsection{Yteffekt -- skin-effect}
\index{yteffekt}
\index{skin-effect}

I en ledare av homogent material fördelar sig en likström lika över hela
tvärsnittet.
Men för en växelström minskar strömtätheten i ledarens mitt och ökar i stället
vid ytan.
Ju högre frekvensen är desto större är strömtätheten vid ytan.
Fenomenet kallas \emph{yteffekt} (eng. \emph{skin effect}) och uppträder i alla
ledare.

Det djup i ledarmaterialet där laddningstätheten sjunkit till \qty{37}{\percent}
av värdet vid ytan kallas \emph{skin depth}.
För koppar är detta djup ca \qty{70}{\milli\metre} vid \qty{100}{\hertz}.
Vid \qty{1}{\mega\hertz} har djupet minskat till \qty{0,07}{\milli\metre} och
vid \qty{100}{\mega\hertz} till \qty{0,0067}{\milli\metre}.
På grund av yteffekten är alltså materialet i mitten av homogena ledare
elektriskt mindre verksamt vid höga frekvenser.
Resistansen för en viss ledare blir alltså större för växelström än för likström.

Utöver frekvensen påverkas yteffekten av ledarmaterialets elektriska och
magnetiska ledningsförmåga.
För att få låg resistans i ledare för högfrekvent ström är det viktigt att
omkretsen är stor och att materialskiktet vid ytan har hög ledningsförmåga.
Därför är induktorerna i sändarslutsteg ofta försilvrade och består av rör med
stor diameter eller av breda band.

\subsection{Temperaturkoefficient}

Liksom med resistorer påverkas även induktansen av temperaturen.
Att sambandet mellan induktans och temperatur är viktigt förstås av att
temperaturkoefficienten i den frekvensbestämmande induktorn i en oscillatorkrets
påverkar frekvensstabiliteten.

Eftersom metallen koppar utvidgar sig vid temperaturökning och induktorns
tvärsnittsyta då blir större, är temperaturkoefficienten vanligen positiv.
Temperaturkoefficienten \(\alpha_L\) anger induktansändringen per grad
temperaturändring.

Induktansändringen blir då $\Delta L = \pm \alpha _L \cdot L_k \cdot \Delta\vartheta$
där \(L_k\) är induktansvärdet vid den lägre temperaturen (oftast
\qty{20}{\degreeCelsius}) och \(\Delta\vartheta\) är temperaturändringen i
kelvin.

Kelvin [K] är den normerade måttenheten för absolut temperatur.
En ändring med \qty{1}{\kelvin} motsvarar en ändring med \qty{1}{\degreeCelsius}.

Induktorer kan innehålla kärnor av någon metallegering vars egenskaper också är
temperaturberoende.

I praktiken kan man knappast påverka temperaturkoefficienten i en induktor.
Eftersom en resonanskrets för det mesta även innehåller kondensatorer kan
man kompensera en positiv temperaturkoefficient i induktorn med en negativ
temperaturkoefficient i en kondensator.

\subsection{Förluster i kärnmaterial}

När ett magnetiskt växelfält passerar ett kärnmaterial kommer atomerna (som
är permanentmagneter) att ständigt inta nya lägen i materialet i takt med
fältets frekvens.
Då uppstår virvelströmmar, så kallade järnförluster, som dels påverkar
materialets ledningsförmåga och som dels höjer temperaturen i kärnan och därmed
i hela induktorn.

% Avsnitt 2.4 Transformatorn
\mediumtikz{
      \begin{circuitikz}
        \draw
        (1,1) node[transformer](T1) {}
        (T1.base) node{1}
        ;
        \draw[european]
        (4,1) node[transformer](T2) {}
        (T2.base) node{2}
        ;
        \draw
        (7,1) node[transformer core](T3) {}
        (T3.base) node{3}
        ;
      \end{circuitikz}
%%      \\
%%      \begin{tabular}{rl}
%%        1, 2 & Allmänna symboler \\
%%        3 & Transformator med kärna
%%      \end{tabular}
}{Schemasymboler för transformatorer: 1 och 2 är allmänna symboler och 3 transformator med kärna.}{fig:BildII2-5}

\mediumfigpad{images/cropped_pdfs/bild_2_2-06.pdf}{Obelastad transformator}{fig:BildII2-6}

\section{Transformatorn}
\harecsection{\harec{a}{2.4}{2.4}}
\label{sec:transformator}
\index{primärlindning}
\index{transformator!primärlindning}
\index{sekundärlindning}
\index{transformator!sekundärlindning}

\subsection{Allmänt}

En \emph{transformator} (eng. \emph{transformer}) består av en eller flera
lindningar eller spolar av elektriska ledare.
Lindningarna är magnetiskt kopplade till varandra.
Det innebär att de är anordnade så att ett magnetfält som alstrats i någon
av lindningarna även passerar genom övriga lindningar.

När en växelspänning läggs över en lindning kallas den \emph{primärlindning}
(eng. \emph{primary coil}).
I och omkring primärlindningen alstras då ett magnetiskt fält som växlar i takt
med spänningen. Primärfältet passerar även genom övriga lindningar --
\emph{sekundärlindningarna} (eng. \emph{secondary coil}) -- och alstrar där
spänningar och strömmar.

Den så kallade kopplingsfaktorn mellan lindningarna varierar för olika frekvenser.
Den är lägre vid låga frekvenser (hundratals \unit{\hertz}) och högre vid höga
frekvenser (tusentals \unit{\hertz}).
Speciellt vid låga frekvenser behövs en större kopplingsfaktor för att avsedd
effekt ska kunna överföras mellan lindningarna. Då kan ledningsförmågan i den
magnetiska flödesvägen ökas med hjälp av en järnkärna.

Bild~\ssaref{fig:BildII2-5} illustrerar flera vanligt förekommande schemasymboler
för transformatorer med två lindningar.

\subsection{Utföranden}
\index{spänningstransformator}
\index{transformator!spännings-}
\index{strömtransformator}
\index{transformator!ström-}
\index{impedanstransformator}
\index{transformator!impedans-}

Transformatorn kan utföras för olika ändamål, till exempel som
\emph{spänningstransformator} (eng. \emph{voltage transformer}),
\emph{strömtransformator} (eng. \emph{current transformer}) eller
\emph{impedanstransformator} (eng. \emph{impedance transformer}).

Utförandet påverkas även av frekvens och av vilken effekt som ska överföras.

\subsection{Terminologi}
\index{varvtalsomsättning}
\index{transformator!varvtalsomsättning}
\index{impedansomsättning}
\index{transformator!impedansomsättning}

\begin{center}
\begin{tabular}{ll}
primärkrets & sekundärkrets \\
primärlindning & sekundärlindning \\
primärspänning \(u_1\) &  sekundärspänning \(u_2\) \\
primärström \(i_1\) & sekundärström \(i_2\) \\
lindningsvarvtal n & primärt \(n_1\) sekundärt \(n_2\)
\end{tabular}
\end{center}

\begin{tabular}{rcl}
varvtalsomsättning &=& \(\dfrac{n_1}{n_2}\) eller \(\dfrac{n_2}{n_1}\) \\
&&\\
impedansomsättning &=& \(\dfrac{Z_1}{Z_2}\) eller \(\dfrac{Z_2}{Z_1}\) \\
\end{tabular}

\subsection{Den ideala (förlustfria) transformatorn}
\harecsection{\harec{a}{2.4.1}{2.4.2.1}, \harec{a}{2.4.2.2}{2.4.1}, \harec{a}{2.4.2.1}{2.4.2.2}}
\index{varvtalsomsättning}
\index{transformator!varvtalsomsättning}
\index{impedansomsättning}
\index{transformator!impedansomsättning}
\label{ideal_transformator}

% \mediumfig{images/cropped_pdfs/bild_2_2-06.pdf}{Obelastad transformator}{fig:BildII2-6}

I bild~\ssaref{fig:BildII2-6} är transformatorn obelastad när sekundärkretsen är
bruten.

När primärlindningen ansluts till en växelspänning induceras växelspänningar
både över primär- och sekundärlindningarna.
Det uppstår även en ström i primärlindningen, men däremot inte i
sekundärlindningen när sekundärkretsen är bruten.
För den obelastade transformatorn gäller sambandet
%%
\[\dfrac{u_1}{u_2} = \dfrac{n_1}{n_2}\]
%%
det vill säga att spänningen över lindningarna är proportionell mot lindningsvarvtalen.

\mediumplustopfig{images/cropped_pdfs/bild_2_2-07.pdf}{Belastad transformator}{fig:BildII2-7}
% \mediumminustopfig{images/cropped_pdfs/bild_2_2-08.pdf}{Sparkopplad transformator}{fig:BildII2-8}
I bild~\ssaref{fig:BildII2-7} är transformatorn belastad när sekundärkretsen
är sluten.

När någon av transformatorns sekundärlindningar ingår i en sluten strömkrets 
uppstår en sekundärström där.
%
Sekundärströmmen alstrar ett magnetfält som motverkar primärströmmens fält,
hindrar dess växlingar och tar ut energi från primärkretsen.
%
Strömförbrukningen på primärsidan ökar således i proportion mot
strömförbrukningen på sekundärsidan. Transformatorn reglerar själv hur mycket
energi som den tar från strömkällan och lagrar i fältet för att föra över
till sekundärkretsen.

För den belastade transformatorn gäller att strömmen genom lindningarna är
omvänt proportionell mot lindningsvarvtalet, det vill säga omvänt proportionell 
mot varvtalsomsättningen.
%%
\[\dfrac{i_1}{i_2} = \dfrac{n_2}{n_1}\]
%%
Av föregående formler följer att:
%%
\[\dfrac{u_1}{u_2} = \dfrac{i_2}{i_1}\]
%%
Av \(P_1 = u_1 \cdot i_1\) och \(P_2 = u_2 \cdot i_2\) följer att \(P_1 = P_2\).

Om man bortser från förlusterna i transformatorn, är den effekt som den tar
från kraftkällan lika med den effekt som transformatorn avger.

Eftersom transformatorn transformerar både spänningar och strömmar, kommer
även impedansen att transformeras genom transformatorn.
Denna impedanstransformation följer impedansomsättningen, det vill säga
%%
\[\dfrac{Z_1}{Z_2} = \dfrac{n_1^2}{n_2^2}~.\]

\subsection{Transformatortillämpningar}
\harecsection{\harec{a}{2.4.2.4}{2.4.2.4}}

\subsubsection{Sparkopplade transformatorer}
\index{galvanisk förbindelse}
\index{spartransformator}
\index{transformator!spar-}

\mediumfig{images/cropped_pdfs/bild_2_2-08.pdf}{Sparkopplad transformator}{fig:BildII2-8}
\mediumfigpad[0.9]{images/cropped_pdfs/bild_2_2-09.pdf}{Strömtransformator}{fig:BildII2-9}

I bild~\ssaref{fig:BildII2-7} har transformatorn beskrivits så att primär- och
sekundärlindningarnas enda förbindelse med varandra är över ett magnetfält,
alltså utan galvanisk förbindelse.

Varje lindning kan förses med godtyckliga uttag. Mellan uttagen finns 
då en spänning som är proportionell mot antalet lindningsvarv.

Detta är en metod för att spara in på antalet lindningar.
För att till exempel omsätta nätspänningen \qty{230}{\volt} till
\qty{115}{\volt} används ibland en \emph{spartransformator}.

Med en spartransformator kommer olika strömkretsar i galvanisk förbindelse med
varandra, vilket visas i bild~\ssaref{fig:BildII2-8}.
Särskild försiktighet ska därför iakttas vid användning av sparkopplade
transformatorer, på grund av risken för elolycksfall.
Spartransformatorer bör därför inte användas i amatörradiosammanhang.
Säkrast är skyddstransformatorer med galvaniskt skilda ledningar och dessutom
med speciellt bra isolering och kapsling.

\subsubsection{Strömtransformatorer}
\index{strömtransformator}
\index{transformator!ström-}

Hög sekundärström under låg sekundärspänning kännetecknar en
\emph{strömtransformator} (eng. \emph{current transformer}),
som illustreras i bild~\ssaref{fig:BildII2-9}.
Strömtransformatorer används i elektriska svetsningsutrustningar,
induktionsugnar och liknande.
Strömtransformatorer används även för mätning av höga växelströmmar.

%% \mediumplustopfig{images/cropped_pdfs/bild_2_2-09.pdf}{Strömtransformator}{fig:BildII2-9}

\subsubsection{Högspänningstransformatorer}
\index{spänningstransformator}
\index{transformator!spännings-}
\index{högspänningstransformator}
\index{transformator!högspännings-}

Hög sekundärspänning under förhållandevis låg sekundärström kännetecknar en
\emph{spänningstransformator} (eng. \emph{voltage transformer}).
Bild~\ssaref{fig:BildII2-10} visar en transformator med ett gnistgap i
sekundärkretsen för tändning av gas.

\emph{Högspänningstransformatorer} (eng. \emph{high voltage transformer})
används i distributionsnät, neonskyltar, tändsystem för förbränningsmotorer,
anodspänningsaggregat för sändare och så vidare.

\mediumplustopfig[0.9]{images/cropped_pdfs/bild_2_2-10.pdf}{Högspänningstransformator}{fig:BildII2-10}

\subsubsection{Låg- och klenspänningstransformatorer}
\index{spänningstransformator}
\index{transformator!spännings-}
\index{lågspänningstransformator}
\index{transformator!lågspännings-}
\index{skyddstransformator}
\index{transformator!skydds-}

\mediumfigpad[0.9]{images/cropped_pdfs/bild_2_2-11.pdf}{Klenspänningstransformator}{fig:BildII2-11}

En \emph{lågspänningstransformator} (eng. \emph{low voltage transformer}) med
spänningen 400/\qty{230}{\volt} används i lokala distributionsnät, det vill säga
de elektriska ledningar som går från en transformator till vanliga bostäder och
kontor.

För ökad säkerhet mot elektrisk chock krävs dock att vissa apparater drivs med
klenspänning via en \emph{skyddstransformator}
(eng. \emph{safety isolating transformer}).
Det är en transformator med skyddsseparation mellan primär- och
sekundärlindningarna.
Sekundärspänningen i en klenspänningstransformator,
bild~\ssaref{fig:BildII2-11}, får inte överstiga \qty{50}{\volt}.

\subsection{Sambandet mellan varvtal och impedans}
\harecsection{\harec{a}{2.4.2.3}{2.4.2.3}}
\index{impedans!transformator varvtal}
\index{impedansomsättning}
\index{transformator!impedansomsättning}
\index{impedanstransformator}
\index{transformator!impedans-}

Transformatorn kan även användas för anpassning av impedanser.
Impedansen Z i en lindning är proportionell mot kvadraten av dess
lindningsvarvtal n.

Om effekten i sekundärlindningen är lika stor som i primärlindningen, gäller
formeln:
%
\[\dfrac{Z_p}{Z_s} = \dfrac{n_p^2}{n_s^2}~.\]

% Avsnitt 2.5 Halvledardioden
\input{koncept/komponenter-halvledardioden}
% Avsnitt 2.6 Transistorn
\input{koncept/komponenter-transistorn}
% Avsnitt 2.7 Elektronrör
\input{koncept/komponenter-elektronroer}
% Avsnitt 2.8 Digitala kretsar
\input{koncept/komponenter--digitala-kretsar}
% Avsnitt 2.9 Integrerade kretsar (IC)
\newpage
\section{Integrerade kretsar (IC)}
\label{integrerade kretsar}

\subsection{Allmänt om IC}
\index{integrerad krets}
\index{IC}

Att integrera betyder att samla till en enhet, det kan vara komponenter,
funktioner eller verksamheter.
Integration kan ske på olika nivåer och i många olika sammanhang.

Med integration avses här integration av komponenter för elektroniska
strömkretsar.
Särskilt halvledarelement av olika slag samt resistorer och kondensatorer med
små värden kan framställas med små dimensioner.
Många komponenter kan då samlas i samma hölje.

Komponenter inom ett hölje, avsedda för en viss funktion kallas
\emph{integrerad krets} (eng. \emph{Integrated Circuit -- IC}).

Komponenterna i en IC kan i sin tur vara del av komponenterna en hel strömkrets.
Redan inom höljet kan komponenter kopplas samman för en viss funktion eller som
en del av strömkretsen.
Skrymmande eller effektkrävande komponenter, såsom induktorer, transformatorer
och så vidare får emellertid inte plats, varför även yttre kopplingar behövs.
Det kan också behövas flera IC i en strömkrets -- kanske med innehåll för en
annan funktion.

En integrerad krets är uppbyggd på en basplatta av halvledarmaterial -- ett
chipp.
På plattan framställs, med fototeknik eller etsning, kompletta eller nästan
kompletta dioder, transistorer, resistorer och kondensatorer.
Metoden, som kallas planarteknik, medger att många komponenter kan få plats på
samma platta.

\subsection{Olika slags integrerade kretsar}

Det finns stora sortiment av både standardiserade och speciella IC, varav det
finns två huvudtyper:
\begin{itemize}
  \item digitala integrerade kretsar
  \item analoga integrerade kretsar.
\end{itemize}

\subsection{Digitala IC}

Digitala IC arbetar som framgår av namnet med digitala signalnivåer.
De enklaste typerna innehåller en eller flera digitala grindar (se avsnitt
\ssaref{digitala kretsar}).
Genom att koppla samman grindar kan man skapa kretsar för ett visst ändamål.
I början av 70-talet byggdes komplicerade system av grindar i SSI- och
MSI-teknik.
Ett sådant system är emellertid inte flexibelt eftersom eventuella ändringar
måste göras ''hårdvarumässigt''.
Det innebär att kopplingsledningar måste ändras, kanske hela kretsar bytas ut
och så vidare.

I dagens digitala system används IC i form av en mikroprocessor eller till och
med flera.
En mikroprocessor är en avancerad krets som kan programmeras (konfigureras)
mjukvarumässigt inte bara för ett ändamål utan för många olika.
I system med mikroprocessorer behövs också minnesfunktioner.
Mikroprocessorn är hjärtat i en dator.
Styrd av ett program (mjukvaran) kontrollerar den kringutrustningar med uppgift
att inhämta och avge information -- att kommunicera.

\subsection{Analoga IC}

Analoga IC arbetar med analoga signalnivåer, det vill säga spänningar och
strömmar med kontinuerligt varierande nivåer och frekvenser.
En analog IC kan även arbeta med digitala signaler.

Analoga IC innehåller en eller flera balanserade förstärkare samt olika slags
hjälpkretsar.
Med yttre komponenter kan en analog IC ges olika förstärkning och frekvensgång.
Med ett gemensamt namn kallas dessa förstärkare för operationsförstärkare
(OP-amp).
Operationsförstärkare utförs vanligen i SSI- eller möjligen MSI-teknik.

\subsection{Kombinerade och speciella IC}

Utöver standardiserade IC finns kombinerade och speciella IC.
Exempel på speciella digitala IC är sådana för telekommunikationsändamål.
Ett annat exempel på digitala IC är sådana för signalbehandling, såväl på HF
som LF-nivå.
Exempel på speciella analoga IC är sådana för radiokommunikationsändamål.

Bortsett från vissa skrymmande komponenter och manöverdon kan numera till
exempel en IC innehålla en komplett radiomottagare.
Ett annat exempel på speciella analoga IC är sådana för hörapparater.
Genom programmering anpassas de för det personliga behovet.

\subsection{Utvecklingen}

Det kan sägas hur ofta som helst.
Genom den fantastiska utvecklingen av mikroelektronik öppnas även för
radioamatören möjligheter som tidigare inte var tänkbara.

Denna utveckling har vidgat utrymmet för den experimentella verksamhet som
amatörradio i grunden innebär.
Hobbyn får sålunda med tiden en allt större teknisk spännvidd.

\subsection{Aktuell litteratur}

Ökat teknikomfång inom amatörradio ställer motsvarande krav på litteratur.
På senare tid inbegripes även digitalteknik.
Mest av utrymmesskäl behandlas i denna faktabok digitaltekniken mycket
kortfattat, men ändå så mycket som nämns i CEPT-rekommendationen T/R 61-02.
För djupare studium hänvisas till andra läromedel samt till
leverantörskataloger.

% Avsnitt 2.10 Operationsförstärkare
\input{koncept/komponenter--operationsfoerstarkare}
% Avsnitt 2.11 Värmeutveckling
\input{koncept/komponenter--vaermeutveckling}
%
%
% Kapitel 3 Kretsar
\chapter{Kretsar}
\label{ch:kretsar}
\index{kretsar}

När flera likadana eller olika komponenter kopplas ihop bildas en elektrisk
krets.
Den sammansatta funktionen i en krets är beroende på hur de enskilda
komponenterna påverkar varandra.

% Avsnitt 3.1 Serie och parallellt
\input{koncept/kretsar--serie-och-parallellt}
% Avsnitt 3.2 Filter
\input{koncept/kretsar-filter}
% Avsnitt 3.3 Kraftförsörjning
\input{koncept/kretsar--kraftfoersorjning}
% Avsnitt 3.4 Förstärkare
\input{koncept/kretsar--foerstarkare}
% Avsnitt 3.5 Detektorer -- Demodulatorer
\input{koncept/kretsar-detektorer-demodulatorer}
% Avsnitt 3.6 Oscillatorer
\input{koncept/kretsar-oscillatorer}
% Avsnitt 3.7 Kristalloscillatorer
\input{koncept/kretsar-kristalloscillatorer}
% Avsnitt 3.8 Frekvensblandare
\input{koncept/kretsar-frekvensblandare}
% Avsnitt 3.9 Modulatorer
\mediumtopfig[0.5]{images/cropped_pdfs/bild_2_3-89.pdf}{A3E-modulator}{fig:BildII3-89}

\section{Modulatorer}
\index{modulatorer}

\subsection{Allmänt}
\index{modulation}

När en signal (bärvåg) påverkas så att den överför informationen i en annan
signal, sägs bärvågen bli modulerad.
Detta förlopp kallas modulation.
Vad som då händer behandlas främst i kapitel~\ssaref{ch:modulation}, med
tillämpningar i kapitel~\ssaref{sändare} och delvis i
kapitel~\ssaref{ch:mottagare}.

En anordning för modulation kallas för modulator.
En modulator kan ingå som en funktion i sändare, mottagare med flera system.
Beroende på modulationsmetoden används olika kombinationer av delkretsar som
tillsammans utgör modulatorn.

I detta avsnitt ges exempel på några vanliga modulatorer för sändare.

\subsection{Amplitudmodulatorer}
\index{amplitudmodulator}
\index{amplitudmodulation}

Med en amplitudmodulator påverkas bärvågens amplitud proportionellt
mot den modulerande signalens amplitud.

\index{A1A}
\textbf{Vid sändningsslaget A1A} är amplituden på den modulerande signalen
antingen maximal eller ingen.
Då består modulatorn av en nycklingskrets, som påverkar till exempel ett
drivsteg i sändaren så att bärvågen släpps fram helt eller inte alls.

\index{A3E}
\textbf{Vid sändningsslaget A3E} har den modulerande signalens amplitud
ett analogt förlopp, till exempel tal, med vilket bärvågens amplitud moduleras.
Här beskrivs amplitudmodulation i en förstärkare med ett katodkopplat
elektronrör.
En emitterkopplad transistorförstärkare kan moduleras på ett liknande sätt.
I båda fallen moduleras förstärkarens arbetsspänning (anodspänning respektive 
kollektorspänning) med den modulerande signalen.
Det som då händer är att två signaler blandas på ett sätt som beskrivs i
kapitel~\ssaref{ch:modulation} med tillämpning på A3E.
I vila är då bärvågsamplituden halva den möjliga inom arbetskurvans linjära del.
Vid modulation kommer bärvågens amplitud att variera mellan noll
och den möjliga amplituden.

% \mediumtopfig[0.5]{images/cropped_pdfs/bild_2_3-89.pdf}{A3E-modulator}{fig:BildII3-89}

Bild~\ssaref{fig:BildII3-89} visar ett sändarslutsteg med en triod.
I serie med tilledningen för anodspänningen finns sekundärlindningen av en
modulationstransformator för LF-signalen.

Den LF-förstärkare som driver transformatorn måste för 100~\% modulationsgrad
kunna avge bärvågens halva effekt.
Eftersom uteffekten från en fullt utmodulerad A3E-sändare är 150~\% av den i
vila, måste slutsteget dimensioneras därefter.
Utöver den egna signalspänningen måste modulationstransformatorn även klara
slutstegets arbetsspänning.

\index{arbetsklass}
Om som på bilden anodspänningen i ett förstärkarsteg amplitudmoduleras,
kan förstärkarsteget arbeta olinjärt, till exempel i klass~C.
Varje följande förstärkarsteg måste däremot arbeta linjärt, till exempel i
klass~A.

På grund av den låga verkningsgraden och det stora bandbreddsbehovet används i
dagens amatörradiosändare knappast ''äkta'' amplitudmodulering,
det vill säga A3E.
I stället används i läget ''AM'' nästan alltid H3E, det vill säga enkelt
sidband med full eller reducerad bärvåg (se nästa stycke).
Trots det lägre effektbehovet på grund av endast ett sidband och eventuellt
reducerad bärvågsamplitud kan av dimensioneringsskäl ändå inte de flesta
H3E-sändare avge sin fulla effekt kontinuerligt!

Som redan sagts i kapitel~\ssaref{ch:modulation}, är det onödigt sända ut två
sidband, eftersom båda innehåller samma information.
Det räcker med ett sidband.
Bärvågen innehåller inte någon information.
Den kan därför undertryckas redan i sändaren för att ersättas i mottagaren.
Därmed uppstår sändningsslaget J3E.

\newpage
\subsection{Sändningsslaget J3E (SSB)}
\index{Single Side Band (SSB)}
\index{SSB}
\index{J3E}

Vid sändningsslaget J3E (SSB) sänds således endast ett sidband.
Det andra sidbandet och bärvågen undertrycks, vilket kan göras på flera sätt.
Numera är den så kallade filtermetoden allra vanligast och den enda som
behandlas här.

\mediumfig{images/cropped_pdfs/bild_2_3-90.pdf}{Alstring av J3E (SSB)}{fig:BildII3-90}

Bild~\ssaref{fig:BildII3-90} visar alstring av J3E (SSB).
\index{Upper Side Band (USB)}
\index{USB}
\index{Lower Side Band}
\index{LSB|see {Lower Side Band}}
Med filtermetoden blandas HF- och LF-signalerna i en balanserad blandare där de
undertrycks medan blandningsprodukterna med deras summa- och
skillnadsfrekvenser blir kvar, det vill säga det övre och undre sidbandet.

För att undertrycka det ena sidbandet före sändningen följs blandaren
av ett bandpassfilter med bandbredd och frekvensläge för avsett sidband.
Den signal som sänds ut innehåller på så sätt endast ett sidband (Single Side
Band).

Valet mellan USB och LSB kan göras på två sätt.
Antingen genom att välja mellan ett separat passbandfilter för respektive
sidband eller genom att använda ett enda filter och flytta HF-signalen från ena
sidan till den andra av det filtret (se bild~\ssaref{fig:BildII1-28} i
kapitel~\ssaref{ch:modulation}).

En J3E-modulator enligt filtermetoden består således av en balanserad blandare
ofta en så kallad ringblandare (se bild~\ssaref{fig:BildII3-87} i avsnitt
\ssaref{blandare}) samt ett bandpassfilter.
För att SSB-signalen ska få avsedd sändarfrekvens kan ytterligare
frekvensblandning behövas (se kapitel~\ssaref{sändare}).

\subsection{Vinkelmodulation}
\index{vinkelmodulation}

Vinkelmodulation är samlingsnamnet för frekvensmodulation (FM) och
fasmodulation (PM).

\subsection{Frekvensmodulation}
\index{frekvensmodulation}
\index{F3E}

Vid sändningsslaget F3E (även kallat FM) varierar bärvågens frekvens i
takt med den modulerande signalens amplitud.
Bärvågen kommer på så sätt att pendla omkring en nominell frekvens, det vill
säga vara frekvensmodulerad.
Bärvågsamplituden ändras däremot inte vid frekvensmodulation.

Likspänningsnivåer kan således överföras eftersom en frekvensavvikelse
(deviation) i bärvågen endast påverkas av den modulerande signalens amplitud.

Vid F3E påverkas resonansfrekvensen i den resonanskrets i oscillatorn som
bestämmer dess arbetsfrekvens.
Det görs enklast genom att tillföra en kondensator med variabelt
kapacitansvärde, en kapacitansdiod (se avsnitt~\ssaref{varicap}).

\mediumfig[0.8]{images/cropped_pdfs/bild_2_3-91.pdf}{Alstring av F3E (FM)}{fig:BildII3-91}
\mediumfig[0.8]{images/cropped_pdfs/bild_2_3-92.pdf}{Alstring av G3E (PM)}{fig:BildII3-92}

Bild~\ssaref{fig:BildII3-91} visar en LC-resonanskrets där det ingår en
kapacitansdiod som styrs av en likspänning med en överlagrad modulerande LF-signal.
En likspänning tjänar som en ställbar förspänning till kapacitansdioden.
På så sätt kan man påverka arbetsfrekvensen.
Med den överlagrade LF-signalen påverkas arbetsfrekvensen i takt med
signalamplituden.

\subsection{Fasmodulation}
\index{fasmodulation}
\index{phasemodulation (PM)}
\index{PM}
\index{G3E}

Vid sändningsslaget G3E (även kallat PM) varierar bärvågens fasläge i
förhållande till en omodulerad referens.
Bärvågens amplitud ändras däremot inte.
Fasändringen -- deviationen -- är direkt proportionell mot hur snabbt fasläget
ändras och till den totala fasändringen.
Hastigheten på fasändringen är direkt proportionell mot frekvensen på den
modulerande signalen och till dess amplitud.

Det betyder att deviationen vid fasmodulation ökar både med amplituden
och frekvensen på den modulerande signalen.
Ändringar i likspänningsnivåer kan därför överföras endast om en fasreferens
används.

Fasmodulation kan alstras till exempel genom att påverka resonansfrekvensen i
en resonanskrets någonstans efter oscillatorn, det vill säga där
oscillatorfrekvensen inte påverkas.
Denna resonanskrets har i viloläge samma resonansfrekvens som oscillatorn.
När kretsen bringas ur resonans genom modulation -- samtidigt som kretsen
påtrycks oscillatorsignalen -- uppstår i kretsen omväxlande en induktiv och
kapacitiv reaktans -- detta inom tiden för varje halv period.
Reaktansen skapar därvid den fasförskjutning som innebär fasmodulation.
Se även avsnitt~\ssaref{parallellresonans} och \ssaref{serieresonans}, bilderna
\ssaref{fig:BildII3-18} och \ssaref{fig:BildII3-19}.

% \mediumfig[0.8]{images/cropped_pdfs/bild_2_3-92.pdf}{Alstring av G3E (PM)}{fig:BildII3-92}

Bild~\ssaref{fig:BildII3-92} visar alstring av G3E (PM).
Liksom vid frekvensmodulation kan till exempel en kapacitansdiod användas för
att med en modulerande signal påverka resonansfrekvensen i en krets.

% Avsnitt 3.10 Digital signalbehandling
\input{koncept/kretsar--digital-signalbehandling}
%
%
% Kapitel 4 Isolation och jordning
\chapter[Isolation och jord]{Isolation och jordning}

Isolation och jordning är samlingsbegrepp för ett antal viktiga koncept för
att reducera störningar, som även berör EMC och elsäkerhet.
Detta är viktiga koncept både när man bygger en installation och när man
designar utrustning.
Det skapar också förståelse för hur utrustning är designad, vilket gör det
enklare att använda den på ett korrekt sätt.

% Avsnitt 4.1 Isolation
% Avsnitt 4.2 Jordning
% Avsnitt 4.3 Gemensam och diff
\section{Isolation}
\index{isolation}
\index{isolator}
\index{galvanisk isolation}
\index{isolation!galvanisk}

\emph{Isolation} (eng. \emph{isolation}) är ett samlingsbegrepp för att separera
olika signaler.
Den första enkla separationen är den hos en \emph{isolator}, det vill säga ett
material som inte leder ström så bra.
Det är den mest grundläggande formen av isolation som förhindrar elektrisk
ledning mellan ledningar.

Man brukar prata om \emph{galvanisk isolation} (eng. \emph{galvanic isolation})
för en isolation som inte kan leda likström.
Transformatorer används ofta för att åstadkomma galvanisk isolation.

Nu är isolation inte begränsat till enbart likström, utan även växelspänning
kan behöva isoleras.
Hur god isolationen är beror kraftigt på frekvensen, och de åtgärder man gör
bör anpassas för hur god isolation man behöver eller vill ha för olika
frekvenser.
Man kan till exempel vilja ha god isolation vid sändar- och mottagarfrekvensen
\qty{14}{\mega\hertz}, men vill inte ha galvanisk isolation för det gemensamma
\qty{12}{\volt} kraftaggregatet.

\section{Jordning}
\index{jordning}
\index{bonding}
\index{jordnät}
\index{jord!jordnät}
\index{bonding network}
\index{jordpotential}
\index{jord!jordpotential}
\index{blomjord}
\index{jord!blomjord}
\index{skyddsjord}
\index{jord!skyddsjord}
\index{nolla}
\index{jord!nolla}

\emph{Jordning} (eng. \emph{bonding}) eller dagligt tal \emph{jord} (eng.
\emph{ground}, \emph{earth}) är en kopplingsstrategi för att få samma
referenspotential i olika delar av en elektrisk koppling.
Man bygger ett \emph{jordnät} (eng. \emph{bonding network, BN})
\cite[kap 3.2.1]{K27-1991} och \emph{earthing network})
\cite[kap 3.1.3]{K27-1991} för att koppla samman de olika jordpunkterna.

Den engelska termen \emph{bonding} och även \emph{bondning network} ger en
indikation på vad det handlar om, nämligen en metod att knyta samman flera
olika delar av en design eller installation för att få en gemensam
referensspänning.
Det är helt enkelt en galvanisk sammankoppling.

Många gånger kallas den referenspotentialen för \emph{jordpotential} för att
det är väldigt behändigt att använda jorden som referens, helt enkelt gräva ned
en ledare i marken, till exempel jordspett (eng. \emph{earth electrode})
\cite[kap 3.1.2]{K27-1991}, för att den vägen få tillgång till jordpotentialen.

Begreppen jord och jordning är dock ofta missförstådda då det finns en övertro
på att man kan ta ned störningar med enbart jordning.
Det förekommer också att man upplever att man har problem där jorden upplevs
skapa störningar, varvid en del felaktigt bryter jorden, och därmed
\emph{skyddsjorden}, något man inte får göra av elsäkerhetsskäl.

På samma sätt tror många att man kan göra sig av med en stor växelström i
jorden.
Detta kallas ibland lite skämtsamt för \emph{blomjordning}, för att man inte
tagit hänsyn till jordledarens resistans och induktans, vilket gör att en
växelström inte kan ta sig så långt då ledaren motarbetar den.
Man hade lika gärna kunnat lägga ned sin jordanslutning i en blomkruka för där
gör den lika god nytta.

Inom elkraft förekommer även termen \emph{nolla} (eng. \emph{neutral}), den
kan lätt förväxlas med jorden, men ska hanteras separat från skyddsjord utom
där elsäkerhetsföreskrifter föreskriver att de ska vara sammankopplade.
Nollan är den ledare som är returledare för strömmen.
I det vanligaste elsystemet \emph{TN-C}, är nollan sammankopplad med skyddsjorden
i elcentralen, men ut från elcentralen hanteras den som en separat ledare.
Man får inte koppla ihop dem för att spara ledare!
Skyddsjorden ska ha väldigt lite ström på sig, och därmed även ha väldigt
låg spänningsskillnad från jordpotentialen, men i praktiken kommer det ändå
finnas skillnader.

\smalltikz{
  \begin{circuitikz}[american voltages]
    % Ground reference
    \draw (0,0) node[ground]{};
    % Source 1 ground
    \draw (0,0) to [R, l^=$Z_1$] (2,0);
    \draw (2,1) to [short, i^=$I_1$] (2,0);
    \draw (1.75,-1) to [short, i=$I_1+I_2+I_3$] (0.25,-1);
    % Source 2 ground
    \draw (2,0) to [R, l^=$Z_2$] (4,0);
    \draw (4,1) to [short, i^=$I_2$] (4,0);
    \draw (3.75,-1) to [short, i=$I_2+I_3$] (2.25,-1);
    % Source 3 ground
    \draw (4,0) to [R, l^=$Z_3$] (6,0);
    \draw (6,1) to [short, i^=$I_3$] (6,0);
    \draw (5.75,-1) to [short, i=$I_3$] (4.25,-1);
  \end{circuitikz}
}{Seriekopplat jordsystem}{fig:kap4-1}

\subsection{Seriekoppling av jord}

Den enklaste uppkopplingen av jordförbindelse är att seriekoppla jorden
\cite[kap 3]{ott1988} mellan ett antal strömförbrukare.
Detta förekommer till exempel i en serie av eluttag matade från samma säkring
eller flera eluttag i en skarvdosa.

I bild~\ssaref{fig:kap4-1} att vi har tre strömförbrukare som var och en
bidrar med en ström \(I_1\), \(I_2\) och \(I_3\), och att dessa är
seriekopplade till en jordanslutning.
Från jordanslutningen till strömbidraget \(I_1\) har vi impedansen \(Z_1\),
och från den punkten har vi impedansen \(Z_2\) fram till strömbidragen \(I_2\)
och slutligen impedansen \(Z_3\) fram till \(I_3\).

En naiv tolkning är att spänningen \(U_1\) för strömbidraget \(I_1\) blir
\(U_1 = Z_1 I_1\), vidare \(U_2 = (Z_1 + Z_2) I_2\) och
\(U_3 = (Z_1 + Z_2 + Z_3) I_3\) för det blir det ju om varje ström ansluts var
och en för sig, det vill säga normal seriekoppling av impedanserna.
Denna analys är dock för enkel för att ta hänsyn till fallet när strömmarna
ansluts samtidigt, eftersom strömmar och spänningar kommer samverka.

Den totala strömmen genom första impedansen \(Z_1\) blir ju summan av de tre
strömmarna, därför måste också spänningen höjas med det bidraget.
Den första spänningen blir därför \(U_1=Z_1 (I_1 + I_2 + I_3)\).
På liknande sätt beräknas den andra spänningen med de bägge strömmarna \(I_2\)
och \(I_3\) plus spänningen \(U_1\) och därför blir
\(U_2 = U_1 + Z_2 (I_2 + I_3)\).
Slutligen blir den sista spänningen \(U_3 = U_2 + Z_3 I_3\).
Med förenkling får vi
%%
\[
\begin{array}{ll}
U_1 & = Z_1 I_1 + Z_1 I_2 + Z_1 I_3 \\
U_2 & = Z_1 I_1 + (Z_1 + Z_2) I_2 + (Z_1 + Z_2) I_3 \\
U_3 & = Z_1 I_1 + (Z_1 + Z_2) I_2 + (Z_1 + Z_2 + Z_3) I_3
\end{array}
\]
%%
Vi ser då att störningen blir
%%
\[
\begin{array}{ll}
\Delta U_1 & =  Z_1 I_2 + Z_1 I_3 \\
\Delta U_2 & = Z_1 I_1 + (Z_1 + Z_2) I_3 \\
\Delta U_3 & = Z_1 I_1 + (Z_1 + Z_2) I_2
\end{array}
\]
%%
Vilket är ett tydligt exempel på hur strömmarna stör varandras spänningar och
därmed har avsaknad av isolation.

Fördelen med seriekopplad jord är förstås att man får flera korta anslutningar
men däremot kommer summeringen av de olika strömmarna göra att man får dålig
isolation mellan de olika jordströmmarna och hur nollpotentialen upplevs.

\subsection{Parallellkoppling av jord}
\index{stjärnjordning}
\index{jord!stjärn-}
\index{skyddsjord}
\index{nolla}

Om vi istället ansluter våra tre laster med individuella ledare till jord
kommer de olika strömmarna inte att samverka, detta är en parallellkoppling
av jord~\cite[kap 3]{ott1988}, se bild~\ssaref{fig:kap4-2}.
Vi har därmed åstadkommit en isolation mellan strömmarna med avseende på
jordanslutningen.

\smalltikz{
    \begin{circuitikz}[american voltages]
      % Ground reference
      \draw (3,0) node[ground]{};
      \draw (1,0) to (5,0);
      % Source 1 ground
      \draw (1,0) to [R, l^=$Z_1$] (1,2);
      \draw (1,3) to [short, i^=$I_1$] (1,2);
      % Source 2 ground
      \draw (3,0) to [R, l^=$Z_2$] (3,2);
      \draw (3,3) to [short, i^=$I_2$] (3,2);
      % Source 3 ground
      \draw (5,0) to [R, l^=$Z_3$] (5,2);
      \draw (5,3) to [short, i^=$I_3$] (5,2);
    \end{circuitikz}
}{Parallellkopplat jordsystem}{fig:kap4-2}

% \noindent
Dock kommer varje strömkälla uppleva en förskjutning i spänningen av
sin jord som beror på dess egen ström och impedansen den har till jord.
För att minska denna effekt kan en minskad strömförbrukning användas
eller oftare en förbättrad jordanslutning.

Givetvis kan även varje strömförbrukare ha två jordar, parallellt.
Elkraftsystemens användning av både \emph{skyddsjord} och \emph{nolla} är
just ett sådant system, där nollan är den som har strömmen och tillåts få
åka runt i spänning, medan skyddsjorden i allmänhet enbart har små strömmar.
Skyddsjordens funktion är också att kunna hantera stora strömmar vid fel,
för att kunna bryta tillförseln.
Skyddsjorden har egentligen det som sitt huvudsyfte, men ger ofta en bra
jordreferens.

I apparater och även inne på kretskort kan man ha parallellkoppling.
Det är även känt som \emph{stjärnjordning} (eng. \emph{star grounding})
eftersom kopplingsschemat ser ut att ha en stjärna från en gemensam punkt.
Det kan vara nyttigt att isolera jord för analoga signaler från digitala eller
rent av reläer, PA med mera.
Man försöker sätta stjärnan direkt vid anslutningen till kraftaggregatet för
att hålla dem så gemensamt som möjligt men med så lite påverkan av
seriejordning som möjligt.
Samma teknik används ofta för själva kraftdistributionen av samma skäl.

\subsection{Sammankoppling av apparater}
\label{sammankopplingavapparater}
\index{jordbrum}
\index{jord!brum}

I ett system där man har gjort parallella jordar i matningen,
bild~\ssaref{fig:kap4-3}, vill man nu koppla samman två apparater för att
överföra en signal.
En första naiv lösning är ju att helt enkelt bara dra en tråd \(Z_{signal}\)
från den ena apparaten över till den andra.
Eftersom de har jordanslutning så har de ju en gemensam jordreferens.

\smalltikz{
    \begin{circuitikz}[american voltages]
      % Ground reference
      \draw (2,0) node[ground]{};
      \draw (1,0) to (3,0);
      % Source ground
      \draw (1,2) to [R, l_=$Z_1$, v^=$U_1$] (1,0);
      \draw (0,2) to [short, i^=$I_1$] (1,2);
      % Source output
      \draw (1,4) to [american voltage source, l_=$U_{ut}$] (1,2);
      % Interconnect and load
      \draw (1,4) to [R, l^=$Z_{signal}$] (3,4)
      to [R, l_=$Z_{load}$, v^=$U_{in}$] (3,2);
      % Destination ground
      \draw (3,2) to [R, l_=$Z_2$, v^=$U_2$] (3,0);
      \draw (4,2) to [short, i_=$I_2$] (3,2);
    \end{circuitikz}
}{Sammankopplat system}{fig:kap4-3}

% \noindent
Problemet är att när strömmen \(I_1\) till den första apparaten går igenom
anslutningsimpedansen \(Z_1\) till jord så ger det en spänning
\(U_1 = Z_1 I_1\) på den jordanslutningen.
På samma sätt kommer den andra apparaten att uppleva jorden med en förskjutning
av jordspänningen på \(U_2 = Z_2 I_2\).
Om den tänka utspänningen är \(U_{ut}\) så kommer den egentliga utspänningen
vara \(U_{ut} + U_1\) i förhållande till jord.
Om vi för stunden antar att det inte går någon anmärkningsvärd ström i ledaren
över till den andra apparaten så kommer den uppleva det som en inspänning
\(U_{in}\) i förhållande till sin jordpotential \(U_2\) det vill säga
\(U_{in} = U_{ut} + U_1 - U_2\).

Vi ser här att skillnaden i jordpotential kommer förskjuta den upplevda
inspänningen \(U_{in}\) från den avsedda spänningen \(U_{ut}\) med skillnaden i
jordpotential, det vill säga \(U_1 - U_2\) som i sin tur beror på
anslutningarnas impedans och strömmarna.
Överföringen kan därför ha problem med sin isolation av \(I_1\) och \(I_2\)
till \(U_{in}\).

Om de bägge strömmarna inte har någon starkt frekvensinnehåll för det
frekvensband som man observerar på mottagaren, så fungerar dock detta fint.
Inte helt sällan råkar dock isolationen bli ett bekymmer antingen direkt eller
genom att det stör funktionen indirekt.

Ett försök att minska störningen är förstås att försöka minska \(Z_1\) och
\(Z_2\) genom att göra motståndet mindre, till exempel genom kortare kablar
eller grövre kablar.
Detta fungerar givetvis, men enbart till en viss praktisk gräns.

Det här illustrerar grunden i hur \emph{jordbrum} (eng. \emph{hum}) brukar
uppstå när man kopplar ihop två apparater.
Själva jordbrummet kommer från kraftaggregaten och då deras strömmar delar
krets med nyttosignalen så kommer \emph{överhörning} (eng. \emph{crosstalk})
göra att brummet blir märkbart.
Det finns givetvis många vägar för brum att störa en signal.

\subsection{Isolerad jordning}
\index{isolerad jordning}
\index{jordning!isolerad}
\index{IBN|see {isolerad jordning}}
\index{signaljord}
\index{flytande}
\index{jord!flytande}
\index{jordbrum}
\index{jord!brum}
\index{chassijordning}
\index{jord!chassi}
\index{ledningsbunden störning}

En strategi för att skapa isolation från jordvägen är att helt enkelt
isolera signalerna och deras jord från kraftförsörjningens jord, detta kallas
för \emph{isolerad jordning} (eng. \emph{isolated bonding} även \emph{isolated
 bonding network, IBN})~\cite[kap 3.2.4]{K27-1991}.
Man börjar plötsligt prata om \emph{skyddsjord} skilt från \emph{signaljord}
(eng. \emph{signal ground}).

\smalltikz{
    \begin{circuitikz}[american voltages]
      % Ground reference
      \draw (3.5,0) node[ground]{};
      \draw (1,0) to (6,0);
      % Source ground
      \draw (1,0) to [R, l^=$Z_1$] (1,2);
      \draw (0,2) to [short, i^=$I_1$] (1,2);
      % Source output
      \draw (1,4) to [american voltage source, l^=$U_{ut}$] (1,2);
      % Interconnect and load
      \draw (1,4) to [R, l^=$Z_{signal}$] (4,4)
      to [R, l^=$Z_{load}$, v_=$U_{in}$] (4,2);
      \draw (1,2) to [R, l^=$Z_{GND}$] (4,2);
      % Destination isolation
      \draw (4,2) to [R, l^=$Z_{iso2}$, v_=$U_5$] (6,2);
      % Destination ground
      \draw (6,0) to [R, l^=$Z_2$] (6,2);
      \draw (7,2) to [short, i_=$I_2$] (6,2);
    \end{circuitikz}
}{Sammankopplat system med utjämningsledare}{fig:kap4-4}

\noindent
I apparater med växelströmsmatning har man redan en transformator som
tillhandahåller en galvanisk isolation mellan primärsidan (elkraft) och
sekundärsidan (elektroniken).
Genom att helt enkelt hålla signaljorden \emph{flytande} (eng.
\emph{floating}), det vill säga utan någon galvanisk koppling till skyddsjord,
så kan man istället koppla samman signaljord på två apparater med separata
ledare \(Z_{GND}\).
I bild~\ssaref{fig:kap4-4} är isolationen hos mottagande apparat representerad
av \(Z_{iso2}\) där spänningen \(U_5\) representerar spänningen mellan primär
och sekundärsida.
På liknande sätt kan isolationen på den sändande apparatens sida moduleras som
 \(Z_{iso1}\), men för detta resonemang räcker \(Z_{iso2}\).

Den galvaniska åtskillnaden gör att isolationen för likström kan variera från
megaohm till gigaohm, men på grund av den kapacitiva kopplingen mellan primär
och sekundär sida i transformatorn sjunker isolationen med stigande frekvens.
I praktiken kan transformatorn på grund av sin obalans driva spänningen \(U_5\)
och därför så kan man behöva lasta dess kapacitiva källan med ett motstånd,
varvid \(Z_{iso2}\) snarare kan vara i kiloohm.

Om vi återgår till de bägge två apparaterna, så kan vi nu istället för att
använda oss av elnätets skyddsjord låta apparaternas signaljord vara kopplad
med en kabel \(Z_{GND}\) parallell med signalledaren \(Z_{signal}\).
Har vi en förhållandevis låg ström genom den impedans som kabeln har så
kommer det fungera fint och \(U_{ut}\) kommer att representeras hyfsat bra som
spänningen \(U_{in}\) över \(Z_{load}\).

Eftersom \(Z_{GND}\) kan vara några fåtal ohm medan \(Z_{iso2}\) för låga
frekvenser är i storleksordningen megaohm så kommer kabeln att koppla väl.
För högre frekvenser kan vi förvänta oss att induktansen i kabeln ökar
impedansen \(Z_{GND}\) samtidigt som kapacitansen gör att impedansen
\(Z_{iso2}\) sjunker varvid för högre frekvenser kommer \(Z_{load}\) vara mer
kopplad lokalt mot \(Z_{2}\) snarare än \(Z_{1}\).

Det här scenariot liknar till exempel det hos en normal hemmastereo och ändå kan
det uppstå \emph{jordbrum} i denna koppling.
Det finns flera skäl.
Ett skäl är att transformatorer visserligen erbjuder en galvanisk isolation,
men de är även kapacitiva spänningsdelare för den spänning som finns över
primärlindningen, med 230~VAC spänning så behövs bara lite läckage över för att
man ska uppleva att isolationen brister.
Det brukar vara rekommendabelt att helt enkelt lasta denna spänningsdelare med
ett motstånd, så att signaljord och skyddsjord sitter ihop med ett någorlunda
högt motstånd, ofta med en kondensator parallellt, för att se till att reducera
det bidraget utan att få för mycket störningar från den ström som kommer flyta
mellan jordarna.

Ett annat scenario som skapar jordbrum är när man i någon ände råkar hårt
koppla samman signaljord och skyddsjord, typiskt att det blir oavsiktlig
kontakt mot chassi, som ska vara skyddsjordad.
Själva chassit brukar man prata om som \emph{chassijordat}, men det är
egentligen bara skyddsjord på de flesta system.

För att isolationsjordning ska fungera måste alla kontakter vara isolerade från
chassit.
Detta gäller även signaljord som inte får ha kontakt med chassi inuti apparaten.
Man behöver alltså försäkra sig om bra isolationsavstånd, vilket väldigt lätt
kan missas av att man har en skruv som råkar skrapa sig igenom skyddslack till
exempel.

En annan nackdel med isolationsjordning är att den gör det svårare att designa
för god EMC-täthet.
För \emph{ledningsbunden störning} (eng. \emph{conductive emission}) så vill
man helst att kontaktens och kabelns skärm sitter i chassijorden med så låg
impedans (induktans) som möjligt.
Isolationsjordning kräver då att man monterar kondensatorer som kopplar ihop
ledarens jord med chassijord och helst runt om för att få lägsta induktans.

Isolationsjordning rekommenderas inte för större system, då den blir svår
att upprätthålla.

Det förekommer att man för att minska störningarna i ett isolationsjordat
system väljer att koppla bort skyddsjorden, för att på det sättet ha mindre
störningar.
Detta är oftast inte tillåtet göra då man normalt inte bryter mot
elsäkerhetsregler och anläggningen riskerar bli farlig, då personskyddet
sätts ur spel.

\begin{center}
\begin{minipage}{0.19\columnwidth}
\Huge{\warningsymbol}
\end{minipage}
\begin{minipage}{0.7\columnwidth}
Varje gång som skyddsjorden kopplas bort för att lösa ett problem så
har man skaffat sig ett större problem, nämligen signifikant sänkt
elsäkerhet, vilket indikerar att man valt en felaktig lösning.
\end{minipage}
\end{center}

\subsection{Sammankopplad jordning}
\index{sammankopplad jordning}
\index{jord!sammankopplad}
\index{jordloop}
\index{jord!loop}
\index{vagabonderande jordström}
\index{jordbrum}
\index{jord!brum}
\index{chassijord}
\index{skyddsjord}

En annan strategi är \emph{sammankopplad jordning} (eng. \emph{mesh bonding}
och \emph{mesh bonding network, mesh-BN})~\cite[kap 3.2.3]{K27-1991}
där man istället för att isolera satsar på att koppla samman jordarna, hårt.
Varje signalkabel sitter ansluten mot \emph{chassijord} och därmed
\emph{skyddsjord} och man låter därmed jordarna sammankopplas.
Varje apparat har en ordentlig jordanslutning som man ansluter till stativjord
eller jordskenor.
Kablar läggs på kabelstegar som jordas.
I detta system kommer varje extra kabel att koppla samman jordarna hårdare,
eftersom man parallellkopplar många impedanser.
Denna strategi väljs ofta i telekommunikationssystem.

%% k7per: Varifrån ska denna figur refereras?
\smalltikz{
  \begin{circuitikz}[american voltages]
    % Ground reference
    \draw (2.5,0) node[ground]{};
    \draw (1,0) to (4,0);
    % Source ground
    \draw (1,0) to [R, l^=$Z_1$] (1,2);
    \draw (0,2) to [short, i^=$I_1$] (1,2);
    % Source output
    \draw (1,4) to [american voltage source, l^=$U_{ut}$] (1,2);
    % Interconnection
    \draw (1,4) to [R, l^=$Z_{signal}$] (4,4)
    to [R, l^=$Z_{load}$, v_=$U_{in}$] (4,2);
    \draw (1,2) to [R, l^=$Z_{GND}$, v_=$U_{GND}$] (4,2);
    % Destination ground
    \draw (4,0) to [R, l^=$Z_2$] (4,2);
    \draw (5,2) to [short, i_=$I_2$] (4,2);
  \end{circuitikz}
}{Sammankopplat system med utjämningsledare}{fig:kap4-5}

\noindent
I ett system som har sammankopplad jord kommer man ofrånkomligen att behöva
hantera vad man kallar för \emph{jordloop} (eng. \emph{ground loop}) eller även
\emph{vagabonderande jordströmmar}.
Många gånger förklaras det som att man får en loop som agerar antenn för ett
magnetfält.
Det är dock sällan som ett magnetfält är så starkt att det inducerar flera
ampere av vanlig \qty{50}{\hertz} ström.

Om vi går tillbaka till sammankoppling av apparater (kapitel
\ssaref{sammankopplingavapparater}) där vi fick en skillnad av spänning \(U_{GND}\)
mellan jordpunkterna så kommer vi ha den även här, men nu ansluter vi ju en
ledning \(Z_{GND}\) mellan dessa punkter, och då kommer det gå en ström som
försöker utjämna potentialen mellan de bägge jordanslutningarna, som då kommer
närmare varandra.
Det är impedansen \(Z_{GND}\) på kabeln som kommer att avgöra hur stor strömmen
blir och hur nära de kommer varandra spänningsmässigt.
Denna ström kan bli ansenlig och har man då en kabel som har till exempel tunn
skärm så kommer kabeln helt enkelt bli varm.
Det är därför lämpligt att lägga en jordkabel parallellt med signalkabeln, för
att låta den med sin större tvärsnittsarea ta merparten av strömmen och därmed
undviker man värme och ström i signalkabeln.

Med en större kabel mellan kommer spänningen sjunka och den vägen kommer
\emph{jordbrummet} minska.

Fördelen med sammankoppling av jordar är att det blir enklare (och billigare)
att designa ur EMC-perspektiv, då man direkt kopplar jordströmmarna i chassit.
Man har inte heller problem med att man skulle råka jorda eller att man skulle
tappa den enda jordvägen.
Istället försöker man koppla ihop jordarna väl.

Ett vanligt problem är om man låter jordströmmarna gå genom kretskort, vilket
gör att man skapar lokala problem med seriejordning.
Man ska se till att jordströmmarna knyter hårt till chassit, men svagt genom
kortet för att på det sättet få bästa möjliga isolation.
Denna princip är också lämplig för att kunna hantera till exempel ESD-skador.

En annan fördel är att man bygger en vana att jorda allt, och för varje
kompletterande jordning gör man systemet starkare.

En självklar fördel är att man dessutom inte bryter skyddsjord, och därmed inte
sänker elsäkerheten på utrustningen och installationen.

\subsection{Balanserad signal}
\index{balanserad signal}
\index{jordbrum}
\index{jord!brum}
\index{galvanisk isolation}
\index{gemensam spänning}
\index{differentiell spänningen}

För att ytterligare få isolation från jordbrum kan man använda en
\emph{balanserad signal} (eng. \emph{balanced signal}).
Grundprincipen är att man skickar samma signal två gånger, men med omvänt
tecken, och sedan ta emot den och bara titta på skillnaden mellan dem.
Skulle nu en störning introducera sig på dessa ledare gemensamt så påverkar
detta inte skillnaden i spänning mellan dem.

\begin{figure}
  \begin{center}
    \begin{circuitikz}[american voltages]
      % Ground reference
      \draw (4,1) node[ground]{};
      \draw (1,1) to (7,1);
      % Source ground
      \draw (1,1) to [R, l^=$Z_1$] (1,4);
      \draw (0,4) to [short, i^=$I_1$] (1,4) to (2,4);
      % Source diff
      \draw (2,6) to [american voltage source, l^=$U_{ut}$] (2,4)
      to [american voltage source, l^=$U_{ut}$] (2,2);
      % Wires and load
      \draw (2,6) to [R, l^=$Z_{signal+}$] (5,6)
      to [R, l^=$Z_{load}/2$, v_=$U_{in+}$] (5,4)
      to [R, l^=$Z_{load}/2$, v_=$U_{in-}$] (5,2);
      \draw (2,2) to [R, l^=$Z_{signal-}$] (5,2);
      \draw (2,4) to [R, l^=$Z_{GND}$] (5,4);
      % Destination isolation
      \draw (5,4) to [R, l^=$Z_{iso2}$, v_=$U_5$] (7,4);
      % Destination ground
      \draw (7,1) to [R, l^=$Z_2$] (7,4);
      \draw (8,4) to [short, i_=$I_2$] (7,4);
    \end{circuitikz}
  \end{center}
  \caption{Sammankopplat system med utjämningsledare och differentiell signal}
  \label{fig:kap4-6}
\end{figure}

Redan tidigare har vi gjort liknande och försökt efterlikna egenskaperna, för
redan när vi skickade en signal på en enkel ledare så skickar vi en spänning
i förhållande till en referensspänning och vi tittar på den inkommande
spänningen i förhållande till referensspänningen.
Dock har vi haft problem att ha en bra gemensam sådan, och det är uppenbart
att vi egentligen observerar skillnaden i spänning.

Med balanserad signal tar vi steget fullt ut och separerar nollreferens från
signal och skickar en signal som vars summa är en fix spänning medan
skillnaden är nyttosignalen.
Det är som om signalen är neutral.
Ofta är dock signalen av praktiska skäl förskjuten spänningsmässigt.

Den balanserade signalen har jord, \emph{pluspol} och \emph{minuspol}.
\emph{Pluspolen} kallas även +, \emph{positiv polaritet}, \emph{het} (eng.
\emph{positive pole}, \emph{positive polarity} och \emph{hot}) medan
\emph{minuspolen} kallas även -, \emph{negativ polaritet}, \emph{kall} (eng.
\emph{negative pole}, \emph{negative polarity} och \emph{cold}).
Utöver dessa har man oftast en \emph{spänningsreferens} som ofta betäcknas som
\emph{jord} (eng. \emph{ground, GND}) eller \emph{nolla} (eng.
\emph{neutral}).

I bild~\ssaref{fig:kap4-6} visas hur ut-spänningen \(U_{ut}\) är dubblerad och
matar på var sin sida om jordpotentialen som  \(I_{1}\) och \(Z_{1}\) ger.
De bägge utspänningarna är kopplade över var sin ledare \(Z_{singal+}\) och
\(Z_{singal-}\) för att över var sin \(Z_{load}/2\) resultera i \(U_{in+}\)
respektive \(U_{in-}\), som i sin tur sitter mot signaljorden på samma sätt
som tidigare.
Den egentliga in-spänningen är från \(U_{+}\) till \(U_{-}\) det vill säga
\(U_{in} = U_{+} - U_{-} = U_{in+}+U_{in-}\)

Transformatorer passar väl för att både generera och ta emot balanserade
signaler, då de har en \emph{galvanisk isolation} för \emph{gemensam spänning}
men transformerar den \emph{differentiella spänningen}.
Detta kan även göras med aktiv elektronik så som op-ampar men även färdiga
kretsar finns.

Transformatorer har fördelen att man kan få den galvaniska skillnaden genom
att helt enkelt bryta jordförbindelsen på ledaren.
Dock, transformatorer har inte fulländad isolation men kan däremot ofta hantera
ganska stora spänningar, vilket kan krävas i besvärliga sammanhang.
För RF är dock transformatorer inte balanserade och ger dålig isolation.
Förbättrad isolation hos transformatorer kan uppnås med ett eller två
skärmlager mellan lindningarna.
Skärmlagren kan anslutas till respektive sidas jord.
För RF krävs dock en strömbalun/RF-choke för att undertrycka den
gemensamma strömmen.

Aktiv elektronik för balansering har sällan galvanisk isolation, men däremot
kan man upprätthålla hög impedans för den gemensamma spänningen, vilket kan
vara nog så tillräckligt.

Differentiell signal i RF kan uppnås genom att använda en RF-choke som
undertrycker den gemensamma spänningen i RF men inte i likspänning.

\section[Gemensam och diff]{Gemensam och differentiell spänning och ström}

När man har ett treledarsystem som vi har med differentiell matning eller
även om man bara har två ledare men mellan system som har gemensam jord
(gäller också om de bara har RF-koppling en annan väg) så kan man betrakta
de två signalledarna antingen som att de har sin individuella spänning och
ström, eller som att de har gemensam och differentiell spänning och ström.

\subsection{Gemensam och differentiell spänning}
\label{comdiffv}

Gemensam spänning och differentiell spänning är ett alternativt sätt att
betrakta spänning på de bägge ledarna, där man delar upp spänningen i det som
är gemensamt för de bägge spänningarna och det som skiljer dem åt. Man kan
alltså betrakta dem på detta alternativa och oberoende (ortogonala) sättet.

\begin{figure}
  \begin{center}
    \begin{circuitikz}[american voltages]
      % Ground reference
      \draw (3.5,1) node[ground]{};
      \draw (0,1) to (7,1);
      % Source ground
      \draw (0,4) to [american voltage source, l^=$U_{g}$] (0,1);
      \draw (0,4) to (2,4);
      % Source diff
      \draw (2,6) node[anchor=east] {$V_{ut+}$} to [american voltage source, l^=$U_{d}/2$] (2,4)
      to [american voltage source, l^=$U_{d}/2$] (2,2) node[anchor=east] {$V_{ut-}$};
      % Wires and load
      \draw (2,6) to [R, l^=$Z_{signal+}$] (5,6) node[anchor=west] {$V_{in+}$}
      to [R, l^=$Z_{d}/2$, v_=$U_{d+}$] (5,4)
      to [R, l^=$Z_{d}/2$, v_=$U_{d-}$] (5,2) node[anchor=west] {$V_{in-}$};
      \draw (2,2) to [R, l^=$Z_{signal-}$] (5,2);
      \draw (2,4) to [R, l^=$Z_{GND}$] (5,4);
      % Destination isolation
      \draw (5,4) to (7,4);
      % Destination ground
      \draw (7,4) to [R, l^=$Z_{g}$, v_=$U_{g}$] (7,1);
    \end{circuitikz}
  \end{center}
  \caption{Sammankopplat system med utjämningsledare och differentiell signal}
  \label{fig:kap4-7}
\end{figure}

I bild~\ssaref{fig:kap4-7} har man den gemensamma spänningskällan \(U_g\), som
från ersatt de förskjutna jordpunkterna i tidigare exempel.
Den differentiella spänningen \(U_d\), det vill säga den drivande spänningen
mellan \(V_{ut+}\) och  \(V_{ut-}\) är fördelad på två spänningskällor som
levererar halva spänningen var.
%%
\[V_{ut+} = U_g + \dfrac{U_d}{2}\]
\[V_{ut-} = U_g - \dfrac{U_d}{2}\]
%%
Omvänt kan man formulera uttrycken för gemensam spänningen \(U_g\) samt
den differentiella spänningen \(U_d\) som \(V_{ut+}\) och \(V_{ut-}\):
%%
\[U_g = \dfrac{V_{ut+}+V_{ut-}}{2}\]
\[U_d = {V_{ut+}-V_{ut-}}\]
%%
På motsvarande sätt på ingången kan man skriva uttrycken för den gemensamma
mottagna spänningen \(U_{g,in}\) och den mottagna differentiella spänningen
\(U_{d,in}\) baserat på inspänningarna \(V_{in+}\) och \(V_{in-}\), man får då
%%
\[V_{g,in} = \dfrac{V_{in+} + V_{in-}}{2}\]
\[V_{in+} = V_{in+}-V_{in-}\]
%%
Ett sätt att illustrera skillnaden är till exempel med en transformator.
En transformator med 1:1 lindning kopplas in mellan två balanserade signaler.
Transformatorns primärlindning kommer att omvandla den differentiella spänningen
\(V_d\) till en motsvarande spänning på utgången.
Däremot kommer den gemensamma spänningen inte att överföras.
Transformatorn blir då en isolator för den gemensamma spänningen precis som vi
förväntar oss av en galvanisk isolation.

Isolationen för den gemensamma spänningen i en transformator är dock främst ett
likströmsbeteende, så ju högre frekvens desto bättre koppling, det vill säga
sämre isolation.
Detta beror på den kapacitiva kopplingen mellan lindningarna som skapar en
ström, som sammankopplar sidorna och resulterar i att den gemensamma spänningen
ändå går igenom transformatorn.
För högre frekvenser är kopplingen väldigt god och transformatorn gör ingen
nytta för att undertrycka den gemensamma spänningen.

Eftersom nyttosignalen är differentiell kan man ibland medvetet använda den
gemensamma spänningen för att överföra matningsspänning till exempel en
mikrofon.
Denna form av matningsspänning kallas för \emph{fantommatning}
(eng. \emph{phantom power}).
En vanligt förekommande spänning är \qty{48}{\volt}, som då symboliseras med P48.
Det förekommer även på modern Ethernet-utrustning och kallas då för
\emph{Power over Ethernet (PoE)}.

\subsection{Gemensam och differentiell ström}
\label{comdiffi}
\index{RF-choke}
\index{strömbalun}
\index{current balun}

Precis som för spänning kan man beskriva strömmarna i samma ledare som
gemensam och differentiell ström.
Vi kan därför återanvända formlerna och bara byta ut V mot I genomgående och
får då:
%%
\begin{eqnarray*}
I_+ = & I_g + I_d\\
I_- = & I_g - I_d\\
I_g = & \dfrac{I_+ + I_-}{2}\\
I_d = & \dfrac{I_+ - I_-}{2}
\end{eqnarray*}
%%
Om vi återgår till transformatorexemplet så kommer det vara den differentiella
strömmen på primärlindningen som ger upphov till magnetfältet i transformatorn
och som sedan inducerar en differentiell ström i sekundärlindningen.

Isolationen mellan lindningarna förhindrar att det går en ström mellan dem,
och därför förhindras den gemensamma strömmen vid låga frekvenser.
Vid högre frekvenser kommer dock den kapacitiva kopplingen mellan de två
sidorna att ske varvid en gemensam ström kommer uppstå för högre frekvenser,
det vill säga för högre frekvenser kommer isolationen att bli sämre.

Ett intressant specialfall är om vi sätter en ringkärna på vår kabel, lindar
kabeln flera varv genom den, eller bara lindar den runt luft.
Då kommer strömmen i den ena ledaren inducera en ström i den andra ledaren och
vice versa.
Denna koppling kan liknas vid att vrida en 1:1 transformator 90~grader fel.
Eftersom den inducerade strömmen har motsatt riktning så kommer den motverka
den gemensamma strömmen, men inte den differentiella strömmen.
Dessutom kommer denna koppling bli starkare för högre frekvenser (i den fina
teorin) och därmed skapa en högre isolation för gemensam ström.
Detta kallas för bland annat \emph{RF-choke} (eng. \emph{RF-choke}) och
\emph{strömbalun} (eng. \emph{current balun}).
Den kompletterar isolationen hos en transformator eller löser den nödvändiga
isolationen helt på egen hand.

RF-choke är ett oerhört användbart verktyg för att undertrycka RF-strålning
och det man ofta i EMC sammanhang kallar ledningsbunden strålning, som är en
gemensam ström ut på ledarna.
Att det är den gemensamma strömmen förstås lätt eftersom den differentiella
strömmen från de bägge ledarna kommer att motverka varandra i utstrålat
magnetfält medan den gemensamma strömmen samverkar och därför är det enbart
den som ger ett utstrålat magnetfält.

Det är därför man ofta hittar klumpar som sitter på kablar till exempel skärmar.
Dessa klumpar är helt enkelt en ringkärna som förstärker kopplingen mellan
ledarna för att undertrycka den gemensamma strömmen för RF och därmed minska
störningen.

\subsection{Generell gemensam och differentiell analys}
\label{comdiffgeneric}
\index{mod!gemensam}
\index{gemensam strömöverföring}
\index{mod!differentiell}
\index{differentiell strömöverföring}
\index{Common Mode (CM)}
\index{CM}
\index{Differential Mode (DM)}
\index{DM}

Efter att ha studerat gemensam och differentiell spänning (kapitel
\ssaref{comdiffv}) och gemensam och differentiell ström (kapitel~\ssaref{comdiffi})
kan vi sammanfattningsvis konstatera att den grundläggande metoden att omvandla
de individuella spänningarna och strömmarna till \emph{gemensam överföring}
(eng. \emph{Common Mode, CM}) och \emph{differentiell överföring}
(eng. \emph{Differential Mode, DM}) är en kraftfull metod både för att
förstå och avhjälpa problem och uppnå isolation.
För spänning har vi ekvationerna
%%
\begin{eqnarray*}
V_+ = & V_{CM} + V_{DM}\\
V_- = & V_{CM} - V_{DM}\\
V_{CM} = & \dfrac{V_+ + V_-}{2}\\
V_{DM} = & \dfrac{V_+ - V_-}{2}
\end{eqnarray*}
%%
För ström har vi ekvationerna
%%
\begin{eqnarray*}
I_+ = & I_{CM} + I_{DM}\\
I_- = & I_{CM} - I_{DM}\\
I_{CM} = & \dfrac{I_+ + I_-}{2}\\
I_{DM} = & \dfrac{I_+ - I_-}{2}
\end{eqnarray*}

\subsection{Gemensam och differentiell impedans}

Precis som man har impedans på ingångar så har man det på ingångar i
treledarsystem.
Det som är den normala impedansen för en transmissionsledare till exempel är
egentligen den differentiella impedansen, det vill säga förhållande mellan den
differentiella spänningen och differentiella strömmen.
Den gemensamma impedansen är på samma sätt förhållandet mellan gemensam
spänning och gemensam ström
%%
\begin{eqnarray*}
Z_{DM} = & \dfrac{U_{DM}}{I_{DM}}\\
Z_{CM} = & \dfrac{U_{CM}}{I_{CM}}
\end{eqnarray*}
%%
Egentligen är det inte så konstigt, om man har en koaxialkabel i ett 50~ohm
system så har sändare och mottagare idealt 50~ohm som differentiell impedans.
I ett system som har isolerad jordning så kan den gemensamma impedansen vara
många megaohm eller högre, eftersom den är isolerad.

\subsection{Obalans}
\index{strömbalun}
\index{obalans}

Så här långt har huvudsakligen antagit att vi har balans, det vill säga att
transformatorer, induktorer med mera är ideala och ger lika bra koppling till
bägge sidor.
Givetvis finns inte detta i verkligheten, och man har en obalans.
Vid obalans får man en signal som är gemensam att läcka över till den som är
differentiell och omvänt att differentiell läcker över till den gemensamma.
Det resulterar dels i minskad isolation och dels i minskad signal.
I allmänhet är den minskade isolationen värre än förlusten av signal, som i
allmänhet är försumbar.

I en transformator ligger lindningarna ofta så att den kapacitiva kopplingen
från ena polen på en spole är starkare än från den andra polen.
Det ger därför en obalans i hur de kopplar kapacitivt.
Genom att lägga ett skärmlager mellan lindningarna kan den kapacitiva
kopplingen jämnas ut, då de kopplar kapacitivt till skärmlagret istället,
som kan lågresistivt hindra koppling.
En ännu bättre lösning är att ha dubbla lager med isolation, för då
kan de kopplas mot respektive sidas jord, och kvar blir bara den kapacitiva
kopplingen mellan jordarna, som oftast är ett mindre problem.
Med dessa metoder fås bättre isolation än vad en oskärmad transformator kan
erbjuda, på grund av just obalans.

Den kapacitiva kopplingen har väldigt hög impedans vid \qty{50}{\hertz}, så man
kan använda relativt höga motståndsvärden för att lasta ned den hårt.
Fördelen är att man kan undvika direkt koppling, vilket kan skapa andra
problem som när man vill ha relativ isolation galvaniskt.

I en strömbalun kan den ena ledaren ha något lite längre varv runt kärnan än
den andra.
Det ger inte en perfekt 1:1 relation i kopplingen och därmed en obalans.

I en transformator med mitt-tapp kan mitt-tappen sitta lite förskjuten från
riktiga mitten, så att anslutningen av mitt-tappen till jord skapar en
obalans.

Dessa exempel på brister i konstruktionen ska man vara medveten om, så att man
inte tillskriver en transformator eller strömbalun att ha egenskaperna av en
perfekt isolation.
Snarare ska man förvänta sig att den inte är perfekt och anpassa sin design
efter det.
Många gånger kan en kombination av åtgärder ge fullgott resultat utan att vara
särdeles dyrt eller klumpigt, men det kräver eftertanke och helhetssyn.

Ett enkelt fall i ljudsammanhang är \qty{50}{\hertz} \qty{230}{\volt} men man
vill hålla störningen mindre än säg \qty{1}{\milli\volt}.
Det kräver mer än \qty{106}{\decibel} isolation mellan \qty{230}{\volt}
differentiellt på primärlindningen och \qty{1}{\milli\volt} gemensamt på
sekundärlindningen.
Så god balans kan vara svår att finna i enskilda komponenter.
Principen återkommer oavsett spänning och frekvens, det är en brist
man behöver lära sig att förstå och hantera.

\subsection{Obalans i antennsystem}
\index{obalans!antennsystem}
\index{mantelström}
\index{obalans!mantelström}
\index{balun}
\index{strömbalun}
\label{obalans_antennsystem}

Obalans kan även förekomma i antennsystem, där en obalanserad antenn omvandlar
den utsända signalen, som är differentiell, till att delvis bli gemensam.
Detta gör att via reflektion från den obalanserade antennen går en ström i
matningsledningen som gör att den strålar.

Detta har traditionellt uttryckts som att strömmen vänder och går på utsidan av
skärmen, men det som hänt är att den differentiella strömmen, som ju motverkar
utstrålning plötsligt får en pålagd gemensam komponent som då kommer stråla.
Man kan uppleva det om man berör ledningen så kan man känna denna som en ström,
vilket man upplever går på utsidan.
Kabeln har då blivit en strålande del av antennen, något som för vissa
antenntyper är en medveten design.

Det är också denna ström som behöver motverkas för att operatören inte ska
skada sig.
Detta görs med en strömbalun, lämpligtvis en kvartsvåg ned från anslutningen
till antennen.
Strömbalunen motverkar den gemensamma strömmen utan att nämnvärt påverka
den differentiella, så det är ett fint exempel på en bra åtgärd.

De allra flesta antenner har en annan impedans i matningspunkten än vad dess
matarledning har.
Detta kräver en impedansanpassning för optimal energiöverföring.
En annan aspekt är att för en koaxial matning så överförs energin enkelsidigt
(single-ended) det vill säga att det är mittledaren i förhållande till
skärm/jord som överför energi.
När vi ansluter denna ledare till en dipolantenn vill vi se till att strömmen
går balanserat ut i de bägge ledarna, så att mittpunkten är nära noll, så att
det inte går en ström med gemensam mod ut i matarledningen.

Vi har alltså dels behovet att omvandla obalanserad signal till balanserad
samt undertrycka gemensam signal i ledaren, och därtill impedanskonvertera den.
Detta brukar man låta en balun (balanced-unbalanced) göra, vilket som namnet
anger bara ger indikation på konverteringen, men den gör alltså flera saker.
Eftersom ingen balun är perfekt designad så kommer den i sig själv ha en
obalans, varvid den ändå kommer ge viss gemensam ström.
För högre effekter kan därför en separat spärr komma att behövas.

Utöver balun finns även unun (unbalan\-ced-un\-balan\-ced) som gör
impedanskonvertering enbart.

Även om man har en bra balun riskerar man att få mantelströmmar, ty antennen
kan vara av en obalanserad typ, till exempel Off-Center-Feed (OCF)/Windom, eller
för att den kopplar olika med miljön som träd och torn med mera.

Att undvika att det går gemensam ström, även kallad \emph{mantelström} kan
krävas av många olika anledningar, och det är viktigt dels för att få ut
energin där den ska, det vill säga radierat ut i luften på ett korrekt sätt,
men även av säkerhetsskäl så att inte utrustnings- eller personskada uppstår.

%
%
% Kapitel 5 Modulation
\chapter{Modulation}
\harecsection{\harec{a}{1.8}{1.8}}
\label{ch:modulation}

\index{modulation}
\index{modulerande signal}
\index{basband}
\index{modulerad signal}
\index{bärvåg}
\emph{Modulera} (lat. \emph{modulari}, rytmiskt avmäta, eng. \emph{modulate})
är att med hjälp av en oftast högfrekvent elektrisk signal (bärvågen) överföra
informationen i en lågfrekvent signal.
På så sätt kan lågfrekvens, till exempel tal och musik, först omvandlas till en
elektrisk signal, som får påverka (modulera) en högfrekvent elektrisk signal.
Denna modulerade signal strålas ut från antennen som ett elektromagnetiskt fält.

Den signal som innehåller informationen kallas \emph{modulerande signal},
\emph{basband} eller \emph{underbärvåg}.

Den signal som informationen överförts till kallas \emph{modulerad signal},
\emph{bärvåg} eller \emph{huvudbärvåg}.

% Avsnitt 5.1 Modulationssystem
\section{Modulationssystem}
\label{sec:modulationssystem}

Den största gruppen av modulationssystem är definierad med avseende på hur
huvudbärvågen är modulerad.
Vanligast är då amplitud- och vinkelmodulation.
Av vinkelmodulation finns främst två slag, frekvensmodulation och fasmodulation.
Därutöver finns system för pulsmodulation.

% Avsnitt 5.2 Sändningsslag
\input{koncept/modulation-saendningsslag}
% Avsnitt 5.3 Kännetecken för modulerade signaler
\input{koncept/modulation-kaennetecken}
% Avsnitt 5.4 Bandbredd vid olika sändningsslag
\input{koncept/modulation-bandbredd}
% Avsnitt 5.5 Beskrivningskod för sändningsslagen
\section{Beskrivningskod för sändningsslagen}
\index{sändningsslag}
\label{modulation_beskrivningskod}

Vid 1979~års radioförvaltningskonferens (WARC~79) i Geneve reviderades det
internationella radioreglementet (RR), som i huvudsak trädde i kraft 1982.
Däri ingår bland annat ett nytt system för klassindelning och beteckning av
sätten att utsända information över radio med mera.
Reglementet har reviderats senare, men i detta stycke gäller det ännu.

Indelningen i \emph{sändningsslag} behövs för att känneteckna utsändningarna,
till exempel i frekvenslistor, författningar och föreskrifter.
Indelningen är också av stort värde vid teknisk beskrivning av apparater och
system för radiokommunikation.

Emellertid används av många även äldre benämningar, vilka lever kvar i
litteraturen, i märkning av manöverdonen på sändare och mottagare.

Dessa äldre benämningar är dock inte entydiga och skapar lätt missförstånd,
varför beskrivningskoden enligt WARC~79 bör användas för tydlighetens skull.

Här följer avkortade koder enligt WARC~79 för några av de sändningsslag som
amatörer använder mest, samt för jämförelse även de benämningar som fortfarande
används jämsides (se vidare i bilaga~\ssaref{saendslag}).

\mediumfig[0.67]{images/cropped_pdfs/bild_2_1-23.pdf}{Modulerande signaler}{fig:BildII1-23}

\begin{description}
\item[NON] Bärvåg utan modulerande signal. Ingen information.

\item[A1A] Bärvåg med dubbla sidband. En enda kanal med kvantiserad bärvåg.
Ingen modulerande underbärvåg. Telegrafi. Även kallat nycklad bärvåg (CW).

\item[A3E] Linjärt modulerad huvudbärvåg. Dubbla sidband. En enda kanal med
analog information. Telefoni. Även kallat amplitudmodulation (AM).

\item[J3E] Linjärt modulerad huvudbärvåg. Ett sidband med undertryckt bärvåg.
  En enda kanal med analog information. Telefoni.
  Även kallat enkelt sidband, Single Side Band (SSB).

\item[F3E] Vinkelmodulerad bärvåg. Frekvensmodulering. En enda kanal med analog
information. Telefoni. Även kallat frekvensmodulering (FM).

\item[G3E] Vinkelmodulerad bärvåg. Fasmodulering. En enda kanal med analog
information. Telefoni. Även kallat fasmodulering (PM).
\end{description}

Såväl A1A, A3E som J3E är sändningsslag där amplituden moduleras.
Därför är termen \emph{amplitudmodulation} inte tillräcklig för att beskriva
flera likartade sändningsslag.

% Avsnitt 5.6 Modulerande signaler
\section{Modulerande signaler}
\harecsection{\harec{a}{1.7.1}{1.7.1}}
\index{modulerande signaler}

\subsection{Basband}
\index{basband}

Basband är ett frekvensområde för en modulerande signal.
Det finns ett basband för alla slags modulerande signaler, vare sig de är
analoga eller digitala.
Det kan finnas mer än ett basband i en komplett modulationsprocess.
Till exempel är en nycklad ton, som går till sändaren genom mikrofoningången,
dess analoga basband medan nycklingspulserna till tongeneratorn är dess
digitala basband.

Bild~\ssaref{fig:BildII1-23} illustrerar modulerade signaler.
Ett vanligt sätt att överföra information över radio är med telefoni, det vill
säga tal.

Frekvensområdet \SIrange{300}{3000}{\hertz} räcker för god förståelighet av tal.
Dels är örat känsligast inom det området och dels finns där den mesta energin
i talet.

Mikrofonen tar upp de lufttrycksvariationer som uppstår när man talar och
omvandlar dem till elektriska svängningar.
Svängningarna varierar mellan positiva och negativa spänningsvärden.

\bigskip

\textbf{Försök}

\begin{enumerate}
\item Anslut en mikrofon till ett oscilloskop och studera spänningsförloppen
  för olika slags ljud, toner, tal osv. som funktion av tiden.
  På bilden är dessa svängningar mycket förenklade, till exempel sinusformade.

\item Anslut en högtalare och ett oscilloskop till en LF-generator, vars
frekvens och amplitud kan ändras. Lyssna på ljud med låg och hög frekvens samt
på svaga och starka ljud.
En baston har låg frekvens och en diskantton har hög frekvens.
En svag ton har liten amplitud och en stark ton har stor amplitud.
\end{enumerate}

% Avsnitt 5.7 Sändningsslaget A3E (AM)
\input{koncept/modulation-amplitudmodulation}
% Avsnitt 5.8 Sändningsslaget A1A (CW)
\input{koncept/modulation-cw}
% Avsnitt 5.9 Sändningsslaget J3E (SSB)
\section{Sändningsslaget J3E (SSB)}
\harecsection{\harec{a}{1.8.3c}{1.8.3c}, \harec{a}{1.8.6c}{1.8.6c}, \harec{a}{1.8.7c}{1.8.7c}}
\index{Single Side Band (SSB)}
\index{J3E}
\index{SSB}
\label{modulation_ssb}

\subsection{Princip}

Som sagts är det onödigt att sända ut två sidband, eftersom båda innehåller
samma information.

Signaler med endast ett sidband och undertryckt bärvåg kan alstras på flera
sätt.
Numera är den så kallade filtermetoden i särklass vanligast och den enda som
behandlas här.

Bild~\ssaref{fig:BildII1-27} illustrerar sidband vid DSB-modulation.
Med filtermetoden blandas HF- och LF-signalerna i en speciell blandare.
Där undertrycks båda dessa signaler medan blandningsprodukterna med deras summa-
och skillnadsfrekvenser blir kvar, dvs. det övre och nedre sidbandet.

Utsignalen från blandaren benämns DSB-signal (Double Side Band).
Till skillnad från i A3E-signalen saknas dock bärvågen i DSB-signalen.
För att även undertrycka det ena sidbandet före sändningen följs blandaren
av ett bandpassfilter med bandbredd och frekvensläge för avsett sidband.

Den signal som sänds ut innehåller därför endast ett sidband (Single Side Band).

\mediumtopfig{images/cropped_pdfs/bild_2_1-28.pdf}{Sidbandsval vid SSB}{fig:BildII1-28}

\paragraph{Exempel}

Bild~\ssaref{fig:BildII1-28} illustrerar sidbandsval vid SSB-modulering.
Ett SSB-filter har ett passband av \SIrange{9000,3}{9003}{\kilo\hertz}.
Vid bärvågsfrekvensen \qty{9000}{\kilo\hertz} sträcker sig det övre sidbandet
från \SIrange{9000,3}{9003}{\kilo\hertz} och släpps igenom.
Däremot blir bärvågsfrekvensen undertryckt.

Det undre sidbandet \SIrange{8997}{8999,7}{\kilo\hertz} faller utanför filtrets
passband och blir också undertryckt.

Ska däremot det undre sidbandet kunna passera igenom samma filter, så måste
bärvågsfrekvensen höjas med \qty{3}{\kilo\hertz}, alltså till
\qty{9003}{\kilo\hertz}.
Då faller det undre sidbandet, \SIrange{9002,7}{9000,0}{\kilo\hertz} inom
filtrets passband.

Det övre sidbandet \SIrange{9003,3}{9006,0}{\kilo\hertz} faller nu utanför
passbandet och blir undertryckt.

%% k7per: Make this bigger.
\mediumtopfig{images/cropped_pdfs/bild_2_1-29.pdf}{Sidbandslägen vid SSB}{fig:BildII1-29}

Bild~\ssaref{fig:BildII1-29} illustrerar sidbandslägen vid SSB.
LF-signalens amplitud bestämmer amplituden på sidofrekvensen.

LF-signalens frekvens bestämmer sidofrekvensens avstånd från bärvågsfrekvensen
(bärvågen undertryckt).

Bandbredden på den utsända signalen är skillnaden mellan högsta och lägsta
modulerande frekvens i signalen:

till exempel \(b = \qty{3}{\kilo\hertz} - \qty{0,3}{\kilo\hertz} =
\qty{2,7}{\kilo\hertz}\)

\subsection{Fördelar med J3E-modulation}
Bra verkningsgrad vid J3E-modulation jämfört med vid A3E-modulation
(traditionell AM).
Effekten i det utsända sidbandet motsvarar den i ett av sidbanden vid A3E.
Hela den utsända effekten finns alltså i ett enda sidband,
som överför hela informationen.

I sändningspauserna sänds ingen effekt ut.
Bandbredden är mindre än hälften av den vid A3E.
Vid mottagning av en J3E-sändning (SSB) är det mindre besvär med
interferenstoner från J3E-sändningar på närliggande frekvenser, eftersom ingen
bärvåg och endast ett sidband sänds ut.

\subsection{Nackdelar med J3E-modulation}
J3E-modulation medför mera komplicerade apparater, både för mottagning och
sändning.
En J3E-signal blir förvrängd och hörs i fel tonläge om mottagaren inte är
inställd på exakt rätt frekvens.

% Avsnitt 5.10 Vinkelmodulation
\section{Vinkelmodulation}
\harecsection{\harec{a}{1.8.3a}{1.8.3a}}
\index{vinkelmodulation}
\label{modulation_vinkel}

Termen vinkelmodulation är samlingsnamnet för frekvensmodulation (FM) och
fasmodulation (PM).
Ofta sägs utrustningar vara för frekvensmodulation när de antingen är för
frekvens- eller fasmodulation.
Det finns alltså skillnader och likheter mellan dessa system, vilka emellertid
inte är oberoende av varandra, eftersom frekvensen i en signal inte kan
varieras utan att fasen också varieras, och vice versa.

Hur effektiv kommunikationen då är beror mest på mottagningsmetoderna.
I båda fallen uppfattas ändringar i den mottagna signalens frekvens och fasläge.
Amplitudändringar uppfattas däremot inte.
De flesta störningar -- särskilt pulserande sådana som från tändningssystem --
kommer därför att skiljas bort.

För att effektivt utnyttja fördelarna med vinkelmodulation, antingen det är
frekvens eller fasmodulation, behövs tillräckligt frekvensutrymme.
Det innebär att främst högre frekvensband kommer i fråga.

% Avsnitt 5.11 Frekvensmodulation (FM)
\input{koncept/modulation-frekvensmodulation}
% Avsnitt 5.12 Fasmodulation (PM)
\section{Fasmodulation (PM)}
\index{fasmodulation}
\index{PM}

Vid fasmodulation varierar bärvågens fasläge i förhållande till ett
referensvärde.
Vid PM är frekvensändringen -- deviationen -- direkt proportionell mot hur
snabbt fasläget på den modulerande frekvensen ändras och till den totala
fasändringen.
Hastigheten på fasändringen är direkt proportionell mot frekvensen på den
modulerande frekvensen och till den momentana amplituden på den modulerande
signalen.

Det betyder att deviationen i PM-system ökar både med den momentana amplituden
och frekvensen på den modulerande signalen.
Detta att jämföras med FM-system där deviationen är proportionell mot den
momentana amplituden på den modulerande signalen.

I PM-system uppfattar demodulatorn i mottagaren endast momentana ändringar i
bärvågsfrekvensen.
Till skillnad från vid FM, så kan därför ändringar i likspänningsnivåer
överföras endast om en fasreferens används.

Med konstant amplitud på insignalen till modulatorn är vid PM
modulationsindex konstant oavsett modulerande frekvens, medan vid FM
modulationsindex varierar med den modulerande frekvensen.

% Avsnitt 5.13 Frekvens- och fasmodulation jämförs
\section{Frekvens- och fasmodulation jämförs}

\begin{itemize}
\item Frekvensmodulation (FM) alstras genom att sändarens oscillatorfrekvens
  varieras (devieras) i takt med den modulerande signalen (t.ex. tal).
  Det gör man genom att variera resonansfrekvensen i den resonanskrets som
  styr oscillatorfrekvensen.

\item Fasmodulation (PM) alstras vanligen genom att efter sändaroscillatorn
  variera den modulerande signalens fasläge i förhållande till en omodulerad
  bärvåg -- så kallad fasmodulering.
  Det gör man genom att variera resonansfrekvensen i en resonanskrets efter
  oscillatorn, dvs. utan att påverka oscillatorfrekvensen.

\item I båda fallen ändrar man alltså resonansfrekvensen i en resonanskrets i
  takt med frekvensen i den modulerande spänningen, men denna krets har
  olika placering i FM-sändare respektive PM-sändare.

\item I sändaren alstras det i båda fallen utfrekvenser som devierar från
  oscillatorns vilofrekvens.
  Graden av deviation skiljer emellertid vid FM och PM.
  Vid FM är deviationen proportionell mot amplituden på den modulerande
  underbärvågen medan deviationen vid PM är proportionell mot produkten av den
  modulerande underbärvågens amplitud och frekvens.

\item Den hörbara skillnaden mellan FM och PM är därför en annorlunda
  frekvensgång.
  Vid samtidig användning av PM-sändare och FM-mottagare är det alltså lämpligt
  att justera frekvensgången i PM-sändarens modulator, lämpligen
  med \qty{6}{\decibel} dämpning per oktav ökad frekvens.
\end{itemize}

% Avsnitt 5.14 Pulsmodulation
\input{koncept/modulation-pulsmodulation}
% Avsnitt 5.15 Digital modulation
\section{Digital modulation}
\harecsection{\harec{a}{1.8.8}{1.8.8}}
\index{digital modulation}
\label{modulation_digital}

Utöver de klassiska analoga modulationsmetoderna finns ett antal digitala
modulationsformer.
De är anpassade för transmission av binära data.
I viss mån kan CW ses som digital modulation där 0 moduleras utan bärvåg och 1
moduleras med bärvåg.
Det finns dock flera andra modulationsmetoder som FSK, 2-PSK/BPSK, 4-PSK och
QAM vilka presenteras i följande delavsnitt.

\subsection{Frekvensskiftsmodulation -- FSK}
\harecsection{\harec{a}{1.8.8a}{1.8.8a}}
\index{frekvensskiftsmodulation}
\index{Frequency Shift Keying (FSK)}
\index{FSK}
\index{frekvensmodulation}
\index{GFSK}
\index{Gaussian Frequency Shift Keying (GFSK)}
\index{Gaussiskt filter}
\index{C4FM}
\index{JT65}
\index{JT9}

\emph{Frekvensskiftsmodulation} (eng. \emph{Frequency Shift Keying, FSK})
skiljer sig från CW-modulationen genom att den ändrar frekvensen, dvs. är en
variant av frekvensmodulation.
I den enklaste formen, binär FSK växlar man mellan två frekvenser, där en
frekvens får representera 0 och den andra får representera 1.
Denna metod har används för modem på telefonförbindelser, såsom Bell~103.

Eftersom varje växling mellan frekvenser ger avbrott i bägge signalerna, likt
nycklingen i CW, så kommer de att skapa sidband.
Av det skälet filtrerar man gärna signalen, och använder man ett Gaussiskt
filter får man \emph{Gaussian Frequency Shift Keying (GFSK)} som används av till
exempel GSM-telefoni.

Man kan använda fler än två frekvenser, till exempel används fyra frekvenser i
Continuous 4 level FM (C4FM), i Phase 1 radios, i Project~25 samt Fusion.

Frekvensskift används även för att sända långsamma meddelanden där JT65
använder 65 frekvenser som den skiftar mellan, medan JT9 använder 9~frekvenser.

\subsection{Binär fasskiftsmodulation -- 2-PSK \& BPSK}
\harecsection{\harec{a}{1.8.8b}{1.8.8b}}
\index{binär fasskift modulation}
\index{fasskift modulation!binär}
\index{2-PSK}
\index{fasskift modulation!2-PSK}
\index{BPSK}
\index{fasskift modulation!BPSK}
\index{Costas loop}

Istället för att modulera frekvensen kan man modulera polariteten eller fasen.
En sådan modulationsform är \emph{binär fasskift modulation} (eng.
\emph{Binary Phase Shift Keying (BPSK)} eller \emph{2-state Phase Shift Keying
(2-PSK)}.
Förenklat kan man säga att bärvågen moduleras med \num{+1} eller \num{-1}, ofta
med \num{+1} representerande \num{0} och \num{-1} representerande \num{1}.

En nackdel med BPSK är att om polariteten blir förväxlad kommer meddelandet att
bli inverterat, dvs. 0 blir 1 och 1 blir 0.
BPSK behöver därför också kompletteras med annan digital modulation för att
hantera polariteten, något som i allmänhet kan åstadkommas enkelt.

BPSK används av satellitnavigationssystem som GPS, GLONASS och Galileo.
För att återvinna BPSK behöver man ofta en speciell variant av PLL-loop känd
som \emph{Costas loop}, eftersom en normal PLL-loop inte klarar av
teckenändringarna på signalen.

\subsection{Fyrnivå fasskiftmodulation -- 4-PSK}
\harecsection{\harec{a}{1.8.8c}{1.8.8c}}
\index{4-PSK}
\index{fasskift modulation!4-PSK}
\index{kvadratur-modulering}
\index{quadrature modulation}
\index{In phase (I)}
\index{Quadrature (Q)}
\index{I/Q modulation}

Fasskiftmodulation kan även göras med flera nivåer.
När fyra olika faslägen används kallas det för \emph{fyrnivå fasskiftmodulation}
(eng. \emph{4-state Phase Shift Keying, 4-PSK}).

Istället för 180~graders fasskift (0 och 180~grader) som används vid 2-PSK/BPSK
så använder man \(360/4\) det vill säga 90~graders fasskift mellan symbolerna.
Ett effektivt sätt att avkoda det är att göra \emph{kvadraturmodulering} (eng.
\emph{quadrature modulation}) där man modulerar en signal till två komponenter,
i \emph{fas} (eng. \emph{In Phase, I}) och förskjuten 90~grader \emph{kvadratur}
(eng. \emph{Quadrature, Q}), ofta kallat I/Q modulering.

De fyra faslägena kan nu enkelt förklaras som amplituder i de olika faslägena
som anges av tabell~\ssaref{tab:4-PSK}.

\begin{table}[t]
\begin{center}
\begin{tabular}{|r|r|r|r|}
\hline
Symbol & Vinkel & I & Q \\ \hline
0 &   0 & +1       &  0 \\
1 &  90 &  0       & +1 \\
2 & 180 & \num{-1} &  0 \\
3 & 270 &  0       & \num{-1} \\ \hline
\end{tabular}
\end{center}
\caption{4-PSK i kvadratur-modulering}
\label{tab:4-PSK}
\end{table}

Amplituden är densamma för alla fyra symbolerna, men med olika vinkel.
I likhet med 2-PSK/BPSK behöver man återvinna fasen och sedan kunna avgöra
vad som är 0~grader, men givet att det görs i den övriga modulationen så
kan informationen avkodas korrekt.

\subsection{Kvadratur-amplitudmodulation -- QAM}
\harecsection{\harec{a}{1.8.8d}{1.8.8d}}
\label{QAM}
\index{kvadratur-amplitudmodulation}
\index{QAM}
\index{16QAM}
\index{DAB}
\index{DVB-T}
\index{DVB-T2}
\index{Wi-Fi}

Medan fasskiftning kan göras för fler fassteg har man funnit att det inte är
lika enkelt för högre upplösningar.
Redan vid åtta steg behöver man ha I- och Q-värden som är \(\sqrt{1/2}\), vilket
i och för sig går att approximera.
En smidigare modulationsform är istället att låta även amplituden variera, och
genom att låta några bitar modulera I och några bitar modulera Q kan man enkelt
få ett symbolmönster som är effektivt att implementera.
Denna modulationsform kallar man \emph{kvadratur-amplitudmodulation}
(eng. \emph{Quadrature Amplitude Modulation, QAM}).

Ofta benämner man olika varianter med antalet olika positioner, så att 16QAM
har 16 olika lägen i fas och amplitud tillsammans.
Ett exempel på hur 16QAM kan moduleras finns i tabell~\ssaref{tab:16QAM}.

\begin{table*}[ht]
\begin{center}
\begin{tabular}{|r|r|r|r|r|r|r|}
\hline
Symbol & Isym & Qsym & Amplitud      & Vinkel &  I &   Q \\ \hline
     0 &    0 &    0 & \(3\sqrt{2}\) &    +45 & +3 &  +3 \\
     1 &    0 &    1 & \(\sqrt{10}\) &    +72 & +3 &  +1 \\
     2 &    0 &    2 & \(\sqrt{10}\) &   +108 & +3 &  \num{-1} \\
     3 &    0 &    3 & \(3\sqrt{2}\) &   +135 & +3 &  \num{-3} \\
     4 &    1 &    0 & \(\sqrt{10}\) &    +18 & +1 &  +3 \\
     5 &    1 &    1 &  \(\sqrt{2}\) &    +45 & +1 &  +1 \\
     6 &    1 &    2 &  \(\sqrt{2}\) &   +135 & +1 &  \num{-1} \\
     7 &    1 &    3 & \(\sqrt{10}\) &   +162 & +1 &  \num{-3} \\
     8 &    2 &    0 & \(\sqrt{10}\) &   +342 & \num{-1} &  +3 \\
     9 &    2 &    1 &  \(\sqrt{2}\) &   +315 & \num{-1} &  +1 \\
    10 &    2 &    2 &  \(\sqrt{2}\) &   +225 & \num{-1} &  \num{-1} \\
    11 &    2 &    3 & \(\sqrt{10}\) &   +198 & \num{-1} &  \num{-3} \\
    12 &    3 &    0 & \(3\sqrt{2}\) &   +225 & \num{-3} &  +3 \\
    13 &    3 &    1 & \(\sqrt{10}\) &   +252 & \num{-3} &  +1 \\
    14 &    3 &    2 & \(\sqrt{10}\) &   +288 & \num{-3} &  \num{-1} \\
    15 &    3 &    3 & \(3\sqrt{2}\) &   +315 & \num{-3} &  \num{-3} \\ \hline
\end{tabular}
\end{center}
\caption{Exempel på 16QAM i kvadraturmodulering}
\label{tab:16QAM}
\end{table*}

Medan både amplituder och vinklar kan kännas udda, så är det enkelt att mappa
bitarna över till I- och Q-amplituder och faslägen via Isym- och Qsym-delarna av
symboler.

QAM-modulering används av DAB, DVB-T, DVB-T2, IEEE~802.11 (Wi-Fi),
mikrovågslänkar och system för mobilt bredband, där man bland annat använder
64QAM, 256QAM och 1024QAM.

En fördel med QAM-moduleringen är att det är enkelt att få samma avstånd mellan
de olika symbolpositionerna, och därmed kan också modulationen anpassas till
störningen.
Detta nyttjas av många moderna modulationssystem så att QAM-modulationen
anpassas utifrån mottagarens rapportering om störning.
Denna dynamiska anpassning gör att kommunikationen kan upprätthållas även om
kapaciteten tillåts variera.

% Avsnitt 5.16 Begrepp vid digital modulation
\input{koncept/modulation-digitala-begrepp}
% Avsnitt 5.17 Bitfel - detektion och korrigering
\section{Bitfel -- detektion och korrigering}

Hittills har vi diskuterat digital modulation utan att ta hänsyn till störningar
och hur dessa påverkar våra överförda data.
Precis som vår CW eller SSB kan vara störd av atmosfäriska störningar, andra
sändare eller helt enkelt vara svaga signaler så att det interna bruset blir en
begränsning, så kommer mottagningen av digitala signaler att bli störd.
Vi ska titta på dessa grundläggande begrepp såsom bitfel, bitfelssannolikhet,
felupptäckt samt korrigering med återsändning eller korrigeringskoder.

\subsection{Bitfel}
\index{bitfel}
\index{bit error}

Av olika orsaker kommer en eller flera bitar ofta att bli fel.
Vi kallar varje sådant fel för att \emph{bitfel} (eng. \emph{bit error}).
Störningar kan göra att vi tolkar en symbol fel, vilket kan resultera i en eller
flera felaktiga bitar.

Om vi i till exempel 16QAM-koden i kapitel~\ssaref{QAM} får in +0.2 i I och +1.1
i Q, ser vi i tabell~\ssaref{tab:16QAM} att närmaste symbolen är symbol 5 med +1
i I och +1 i Q.
Vi skulle kunna anta att om I är större än 0 och mindre än 2, samt Q är större
än 0 och mindre än 2 så är symbol 5 den enda vettiga symbolen, och det är precis
den tolkning vi i allmänhet gör, för det är den symbolen vars avstånd är lägst
och därmed rimligast.
Det kan dock vara så att man egentligen sände symbol 9 med \num{-1} i I och +1 i
Q, och därmed fick för stor störning på I för att man ska tolka det som rätt
symbol.
Vi kommer då lägga ut 9 istället för 5, vilket innebär att två bitar har
ändrats.

Genom att granska tabell~\ssaref{tab:16QAM} vidare ser man att värdena för
I och Q för de olika symbolerna är gjorda så att minsta avstånd är 2 mellan
alla närliggande symboler, i respektive I- och Q-riktning.
Det förenklar tolkning av symbolerna.
Är dock störningen större än 1 i någon riktning kommer man tolka den symbolen
fel, och det kan då leda till 1 eller fler bitfel.

\subsection{Bitfelssannolikhet}
\index{bitfelssannolikhet}
\index{bit error rate}
\index{BER|see {bit error rate}}
\index{gaussiskt brus}
\index{brus!gaussiskt}
\index{gaussian noise}
\index{effektiv-värde}
\index{Root Mean Square}
\index{RMS|see {Root Mean Square}}
\index{Error Function (erf)}
\index{erf}

Om vi antar att vi inte har störning från några andra signaler, utan enbart har
brus som störning, så kan vi estimera \emph{bitfelssannolikheten} (eng.
\emph{bit error rate, BER}) ur hur starkt bruset är i förhållande till vårt
steg.
Eftersom bruset antas vara vitt brus, så har det egenskaperna av \emph{Gaussiskt
brus} (eng. \emph{Gaussian noise}).

Gaussiskt brus har en statistisk fördelning med hög sannolikhet nära
medelvärdet och avtar sedan med avståndet.
Sannolikheten att man tolkar en signal som vara på ena eller andra sidan av en
gräns beror på hur långt bort från medelvärdet den gränsen, ofta benämnd
kvantiseringsgränsen, är i förhållande till den effektiva värdet (eng.
\emph{Root Mean Square, RMS}) i amplitud hos bruset.
Detta kan uttryckas i form av den matematiska funktionen \emph{error function
(erf)}.

När gränsen är 1~sigma, det vill säga 1 gånger RMS-värdet för brusamplituden,
från medelvärdet så är det \qty{67}{\percent} sannolikhet att värdet ligger inom
gränsvärdet, det vill säga en bitfelssannolikhet på \qty{33}{\percent}.
Ligger det inom 2~sigma har sannolikheten ökat till \qty{97}{\percent}, en
bitfelssannolikhet på \qty{3}{\percent}, och vid 3 sigma är den
\qty{99,7}{\percent} med en bitfelssannolikhet på ringa \qty{0,3}{\percent},
vilket ofta används för många ingenjörsapplikationer.
Dock, för överföring av information har vi högre krav.
För en bitfelssannolikhet på \(10^{-12}\), ofta benämnt BER på 1E-12, behövs
det 14~sigma bort till gränsen, dvs. brusmängden får max vara \(1/14\) av
kvantiseringsgränsen.
Den råa radiokanalen uppvisar dock sällan så bra egenskaper, men det kan uppnås
i kabel och fiber.

\subsection{Detektion}
\harecsection{\harec{a}{1.8.10a}{1.8.10a}}
\index{bitfelsdetektion}
\index{paritet}
\index{CRC}
\label{bitfel_detektion}

Eftersom störningar förekommer och man har behov av lägre bitfelssannolikhet
än vad den råa kanalen medger är det lämpligt att identifiera när det har
blivit bitfel.
Detta kan utföras på många sätt, men ett sätt är att räkna fram checksummor som
skickas med datat.
Det kräver visserligen en del av informationsöverföringskapaciteten, men
tjänsten det medger är att försäkra sig om att informationen är rimligt korrekt.

En enkel form av checksumma är paritet, där bitarna i ett ord har summerats ihop
binärt (med XOR) för att bilda en checksumma.
I mottagaränden görs samma kombination och sedan jämförs det med paritetsbiten,
och om de överensstämmer så har inget bitfel upptäckts.
Denna enkla metod har en svaghet i att ett jämnt antal bitfel kommer att
kompensera varandra, varvid det döljer bitfel från upptäckt.
Det är med andra ord inte en särdeles stark checksumma.
Paritet används till exempel i seriekommunikation så som RS-232.

Ett flertal checksummor finns, för olika ändamål, olika mängd fel och olika
typer av fel.
För lite större meddelanden är det vanligt att summera bytes till en checksumma
antingen additivt eller med XOR.
För större meddelanden används en lite mer intrikat metod som heter Cyclic
Redundancy Check (CRC) där man återmatar överskjutande del på checksumman till
sig själv och får en starkare kod den vägen.
CRC används till exempel i Ethernet.

\subsection{Omsändning}
\harecsection{\harec{a}{1.8.10b}{1.8.10b}}
\index{omsändning}
\index{ARQ}
\index{TCP}

En enkel åtgärd att vidta när man konstaterat att ett block data man tagit emot
har fel, är att begära omsändning.
Genom att sändaren håller en buffert med meddelanden som den skickat, och
mottagaren meddelar sändare om den mottagit meddelandet eller behöver ha det
omskickat, så kan omsändning realiseras.
Automatisk omsändningsbegäran (eng. \emph{Automatic Repeat reQuest, ARQ}) är en
typ av protokoll som gör automatisk omsändningsbegäran om ett enskilt datablock,
även kallat paket, inte kommit fram rätt eller helt försvunnit.
Ett sådant protokoll är TCP, som ingår i internetsviten av TCP/IP-protokollet.

\subsection{Korrigeringskod -- FEC}
\harecsection{\harec{a}{1.8.10c}{1.8.10c}}
\index{korrigeringskod}
\index{felrättandekod}
\index{FEC}
\index{AMTOR}
\index{Hamming-koder}
\index{paritet}
\index{Reed-Solomon (RS)}

En annan form av korrigering är att helt enkelt skicka för mycket data redan
från början, som mottagaren kan använda för att korrigera meddelandet utan att
skicka någon begäran till sändaren.
Detta är praktiskt antingen om det skulle ta för mycket tid eller om det helt
enkelt inte finns någon kommunikation från mottagaren till sändaren, till
exempel för satellitmottagare.

En enkel form av felrättande kod används i AMTOR FEC, där man helt enkelt
sänder samma tecken två gånger.
Liknande används i Bluetooth där meddelandet sänds tre gånger, varvid man kan
göra majoritetsröstning.

Andra system för FEC är Hamming-koder, paritets-paket och Reed-Solomon (RS).

% Avsnitt 5.18 Digitala sändningsslag
\input{koncept/modulation-digitala-saendningsslag}
%
%
% Kapitel 6 Mottagare
\chapter{Mottagare}
\label{ch:mottagare}
\index{mottagare}

Energin i de elektromagnetiska magnetfält, som omger oss, alstrar högfrekventa
strömmar i alla metallföremål.
För att effektivt fånga upp dessa fält används antenner.
Fastän energin i fälten kan få en lampa att lysa om sändarantennen är
tillräckligt nära, så går det ändå inte att uppfatta den information som fälten
också kan innehålla.
För det behövs en radiomottagare för att dels förstärka de oftast mycket svaga
signalerna och dels uttyda informationen i dem.

Lyssna på amplitudmodulerade rundradiosändningar på mellanvåg kan man enklast
göra med hjälp av en detektormottagare.
Speciellt under dygnets mörka timmar vintertid kan man höra utländska sändare
med denna enkla mottagare, låt vara att det hörs mycket svagt.
I detektormottagaren omvandlas fältens energi till elektricitet och sedan till
ljud.
Så länge som ingen förstärkare används, förbrukas ingen annan energi än den som
fångas ur fälten -- radiovågorna.

% Avsnitt 6.1 Raka mottagare
\input{koncept/mottagare--raka-mottagare}
% Avsnitt 6.2 Superheterodynmottagare
\input{koncept/mottagare-superheterodynmottagare}
% Avsnitt 6.3 Jämförelse superheterodyn
\input{koncept/mottagare--jaemfoerrelse-superheterodyn}
% Avsnitt 6.4 Panoramamottagare
\input{koncept/mottagare-panoramamottagare}
% Avsnitt 6.5 Mottagningskonvertern
\input{koncept/mottagare-mottagningskonvertern}
% Avsnitt 6.6 Transvertern
\input{koncept/mottagare-transvertern}
% Avsnitt 6.7 AGC
\input{koncept/mottagare-agc}
% Avsnitt 6.8 Egenskaper i mottagare
\input{koncept/mottagare--egenskaper-i-mottagare}
%
%
% Kapitel 7 Sändare och transceivers
\chapter{Sändare och transceivrar}
\label{ch:saendare}
\label{ch:transceiver}
\index{sändare}
\index{transceiver}

Sändaren har till uppgift att skapa en högfrekvent elektrisk energi som sedan
skickas till antennen via transmissionsledningen.
När energin från sändaren når antennen omvandlas den till ett elektromagnetiskt
fält.
Hur fältet sprider sig från antennen beskrivs i
kapitel~\ssaref{ch:vaagutbredning}.

% Avsnitt 7.1 Sändare
% Avsnitt 7.2 Egenskaper i sändare
\input{koncept/saendare--egenskaper-i-saendare}
% Avsnitt 7.3 Transceiver
\input{koncept/saendare--transceiver}
%
%
% Kapitel 8 Antennsystem
\chapter{Antennsystem}
\index{antenn}
\label{ch:antennsystem}

Aldrig så förnämliga radioapparater kommer inte till sin fulla rätt utan ett
effektivt antennsystem.
Det är en huvudförutsättning för framgångsrik radiokommunikation.

Antennen omsätter elektrisk energi från sändaren till elektromagnetiska fält
som strålas ut, det vill säga radiovågor.

Vid mottagning fångar antennen upp radiovågorna och omsätter dem till
elektriska signaler som förs till mottagaren.

Antennsystemet består av den egentliga antennen och transmissionsledningen
mellan denna och sändaren respektive mottagaren.
I antennsystemet ingår även impedansanpassningar, antennkopplare med mera.

% Avsnitt 8.1 Allmänt
\input{koncept/antennsystem--allmant}
% Avsnitt 8.2 Polarisation
\input{koncept/antennsystem-polarisation}
% Avsnitt 8.3 Antenner för kortvåg
\input{koncept/antennsystem--antenner-foer-kortvag}
% Avsnitt 8.4 Riktantenner för kortvåg
\input{koncept/antennsystem--riktantenner-foer-kortvag}
% Avsnitt 8.5 Antenner för VHF/UHF/SHF
\input{koncept/antennsystem--antenner-foer-vhf-uhf-shf}
% Avsnitt 8.6 Transmissionsledningar
\input{koncept/antennsystem-transmissionsledningar}
%
%
% Kapitel 9 Vågutbredning
\chapter{Vågutbredning}
\label{ch:vaagutbredning}
\harecsection{\harec{a}{7}{7}}
\index{vågutbredning}

Elektromagnetisk vågutbredning är energitransport och förutsättningen för all
radiokommunikation.
Radiovågornas utbredning på vägen mellan sändare och mottagare påverkas
emellertid på många sätt.
Med vetskap om radiovågornas utbredningssätt kan man mer metodiskt försöka uppnå
önskade radioförbindelser.

% Avsnitt 9.1 Kraftfält antenner
\section[Kraftfält antenner]{Kraftfälten omkring antenner}
\index{kraftfält}
\index{antenner!kraftfält}

För att sända ut och ta emot radiovågor behövs antenner.
Mycket förenklat är en antenn en elektrisk krets, som består av en induktor
och en kondensator som illustreras i bild~\ssaref{fig:BildII7-01}.

Med kondensatorns elektroder helt isärdragna och förminskade har
resonanskretsen fått ett mycket annorlunda mekaniskt utseende.
Sedan induktorn i LC-kretsen tagits bort, så återstår mekaniskt sett endast
en enkel ledare, men elektriskt sett finns kretsen ändå kvar.
Ledaren med sin utsträckning är fortfarande en induktor och ytorna på dess
motstående halvor är fortfarande elektroderna i kondensatorn med
omgivningen som dielektrikum.

En elektrisk ledare, en stång, tråd etc. är alltså en elektrisk
resonanskrets, vars resonansfrekvens mest bestäms av längden och
tjockleken. Ledaren (antennen) kan kallas dipol -- den har två poler,
detta är grunden för alla typer av antenner.

\mediumplusbotfig{images/cropped_pdfs/bild_2_7-01.pdf}{Från sluten LC-resonanskrets till antenn}{fig:BildII7-01}
\mediumfig{images/cropped_pdfs/bild_2_7-02.pdf}{Pendlingen mellan E-fält och H-fält}{fig:BildII7-02}

Det finns vissa likheter mellan en mekanisk pendel och en elektrisk
resonanskrets.
Energin i en mekanisk pendel växlar mellan två ytterlighetstillstånd.
Det ena är när pendeln just vänder i ett ytterläge.
Då innehåller den enbart lägesenergi och ingen rörelseenergi.
När pendeln rör sig mot mittläget, så omvandlas lägesenergin till rörelseenergi.
I mittläget, som är det andra ytterlighetstillståndet, innehåller pendeln enbart
rörelseenergi och ingen lägesenergi etc.

\subsubsection{Elektrisk resonanskrets}
\index{kraftfält!elektrisk resonanskrets}
\index{elektrisk resonanskrets}
\index{Maxwell}

Den elektriska resonanskretsen kan jämföras med den mekaniska pendeln där det
hela tiden pågår en pendling eller omvandling mellan lägesenergi och
rörelseenergi.
Se bild~\ssaref{fig:BildII7-02}.

När strömmen i den elektriska resonanskretsen just upphört för att vända så
innehåller kondensatorn mest laddning, det vill säga, det starkaste elektriska
fältet mellan elektroderna.
Detta fält kan jämföras med pendelns lägesenergi.
Den utjämningsström som följer från den ena elektroden över till den andra
omges av ett magnetiskt fält som kan jämföras med pendelns rörelseenergi.

Förloppet visas i bild~\ssaref{fig:BildII7-02}, där det framgår att dipolen omges
av det starkaste elektriska fältet vid tidpunkten \(t=0\) samt vid
\(t=1/2T\) med omvänd polaritet, där T är periodtiden.
Vidare att dipolen omges av det starkaste magnetiska fältet vid tidpunkten
\(t=1/4T\) samt vid \(t=3/4T\) med omvänd strömriktning och fältpolaritet.


\smallfig{images/cropped_pdfs/bild_2_7-03.pdf}{Elementär dipol}{fig:BildII7-03}

Med förklaringen av E- och H-fälten som bakgrund följer nu en enkel
framställning av hur radiovågor uppstår ur dessa fält.

Maxwell påvisade i sina ekvationer bland annat sambandet mellan elektroner
i rörelse i en ledare och elektromagnetiska vågor i rummet.
Vidare, att elektroner som rör sig med avtagande eller tilltagande hastighet
avger elektromagnetisk energi.

Hur energi strålar från en ledare kan förklaras med en (tänkt)
elementär dipol, som genomflyts av växelström (Bild~\ssaref{fig:BildII7-03}).

Dipolen består av två lika stora elektriska laddningar med motsatt polaritet.
När den matas med en växelström, så rör sig laddningarna ständigt,
omväxlande emot respektive ifrån varandra.
Tänk på två kulor i var sin ände av en spiralfjäder.
Avståndet mellan laddningarna ändras i takt med styrkan och riktningen på
strömmen.
Systemet är alltså under ständig hastighetsändring (ökning respektive
minskning), vilket är förutsättningen för att energi ska strålas ut.

Först är laddningarna nära varandra på grund av liten laddning.
Vid ökande ström ökar avståndet mellan laddningarna och det byggs upp ett
mer utbrett och energirikt E-fält.
Samtidigt byggs även ett H-fält upp omkring dipolen, vinkelrätt mot E-fältet
och så vidare.
Detta gäller både för en elementär dipol och en elektrisk ledare med många fria
elektroner (verklig antenn).

Formeln för det resulterande S-fältet är \(\overline{S} =
\overline{E}\times\overline{H}\), vilket visar att den lagrade energin
i dipolens närmaste omgivning ökar när avståndet (potentialen) mellan
dipolens laddningar ökar.

Bild~\ssaref{fig:BildII7-04} visar hur ett E-fält byggs upp omkring en dipol och
avskiljs från den.
De visade kraftlinjerna är E- fältet.
H-fältet visas inte, men ligger vinkelrätt mot E-fältet, i cirklar omkring
antennen. Se bild~\ssaref{fig:BildII7-05}.

\mediumfig{images/cropped_pdfs/bild_2_7-04.pdf}{Ett självständigt E-fält skapas}{fig:BildII7-04}

\mediumfig{images/cropped_pdfs/bild_2_7-05.pdf}{E-, H- och S-fälten omkring en antenn (förenklad framställning)}{fig:BildII7-05}

När dipolens laddningar ändrar riktning och åter börjar att röra sig
emot varandra, börjar det E-fält som byggts upp att också byta riktning.
Men det kommer inte att falla tillbaka till dipolens mitt
utan sluts till ett eget kretslopp -- Maxwells första ekvation.
Jämför med en såpbubbla som lämnat blåsröret.
Omkring dipolen har det nu bildats ett självständigt E-fält, som sin tur
alstrar ett eget H-fält.

En period av en elektromagnetisk våg (ett S-fält) har alstrats och
fortsätter att utvidga sig.
För varje följande period alstras ett nytt E-fält, som separeras från antennen
och bildar ett H-fält och så vidare.
Varje gång bildas alltså en ny ''fältbubbla'' inne i den föregående, vilken
håller på att utvidgas.
Resultatet är ett elektromagnetiskt fält, det vill säga en radiovåg.

Som nämnts består en radiovåg av ett högfrekvent elektromagnetiskt fält (S).
Det är i sin tur sammansatt av två andra fält, det elektriska E-
och det magnetiska H-fältet.
Energin i S-fältet fördelas lika mellan E-fältet och H-fälten,
vars krafter korsar varandra vinkelrätt.
S-fältet ligger i plan med både E- och H-fälten och breder ut sig vinkelrätt
mot dem.
S-fältets riktning beror av den inbördes riktningen på E- och H-fälten.

När E-fältet är vertikalt, sägs vågen vara vertikalt polariserad.
När samma fält är horisontellt sägs vågen vara horisontellt polariserad.
När E-fältet roterar i vågfrontens plan, och därmed även H-fältet, sägs vågen
vara cirkulärt polariserad.

Fälten framställs i text och bild som så kallade kraftlinjer med pilar som
föreställer kraftriktningen.
Linjernas längd föreställer fältets styrka.
Bild~\ssaref{fig:BildII7-06} visar ett avsnitt av en vågfront S med vertikal
polarisation.

\smallfig{images/cropped_pdfs/bild_2_7-06.pdf}{E-, H- och S-fält}{fig:BildII7-06}

% Avsnitt 9.2 Radiovågornas egenskaper
\input{koncept/vaagutbredning--radiovaagornas-egenskaper}
% Avsnitt 9.3 Jonosfärskikten
\input{koncept/vaagutbredning-jonosfaerskikten}
% Avsnitt 9.4 Solens inverkan
\input{koncept/vaagutbredning-solens-inverkan}
% Avsnitt 9.5 Vågutbredning på kortvåg
\input{koncept/vaagutbredning--vaagutbredning-paa-kortvag}
% Avsnitt 9.6 Vågutbredning på VHF-EHF
\input{koncept/vaagutbredning--vaagutbredning-paa-vhf-ehf}
% Avsnitt 9.7 Brus och länkbudget
\section{Brus och länkbudget}

\subsection{Allmänt}

Den mottagna signalens kvalitet kan ofta sammanfattas med dess signal-brus
förhållande.
För att kunna estimera det behöver man dels förstå själva länk-budgeten som
ger en uppfattning om hur stark signal man får, men även förstå de olika
bidragen av brus som sätter det effektiva brusgolvet.

\subsection{Brus}
\harecsection{\harec{a}{7.17}{7.17}, \harec{a}{7.18}{7.18}}
\index{brus}
\index{atmosfäriskt brus}
\index{brus!atmosfäriskt}
\index{galaktiskt brus}
\index{brus!galaktiskt}
\index{termiskt brus}
\index{brus!termiskt}

Det finns flera källor till brus, atmosfäriskt brus, galaktiskt brus samt
termiskt brus.

\emph{Atmosfäriskt brus} (eng. \emph{atmospheric noise}) uppstår på grund av
blixturladdningar.
Över hela jorden sker hela tiden blixtnedslag, och dess starka impulser sprider
sig precis som radiovågor och ger en grundläggande störning i kortvågsbandet.
Atmosfäriskt brus identifierades 1925 av Karl Jansky.

\emph{Galaktiskt brus} (eng. \emph{galactic noise}) kommer huvudsakligen från
centrum av Vintergatan, och är huvudsakligen termiskt brus från den stora
ansamlingen av stjärnor i mitten av Vintergatan.
Galaktiskt brus kommer från den delen av himlen som för stunden har mitten av
Vintergatan, så det är riktningskänsligt.

Termiskt brus är mottagarens interna brus, se \ssaref{termisktbrus}.

\subsection{Länkbudget}
\harecsection{\harec{a}{7.20}{7.20}}
\index{länkbudget}

För att kunna estimera den upplevda signalkvaliteten så gör man en så kallad
\emph{länkbudget} (eng. \emph{link budget}).
Länkbudgeten sammanställer hur signalstyrkan respektive brus varierar längs en
länk med dess förstärkningar och dämpningar.
I slutet av länkbudgeten kan sedan det upplevda signal-brus-förhållandet
enkelt estimeras.
En väl utförd länkbudget kan därför skapa god förståelse för länkens brister
så att förbättringar kan göras.

\subsubsection{Dominant bruskälla}
\harecsection{\harec{a}{7.20.1}{7.20.1}}
\index{bandbrus}
\index{band noise}
\index{brus!band}
\index{interntbrus}
\index{brus!internt}

I en noggrann modell så ska alla bruskällor, från källa till mottagare,
listas, justeras för gain och sammanställas.
I praktiken så har man en dominant störkälla, typiskt bruset på bandet eller
kort \emph{bandbruset} (eng. \emph{band noise}) eller
\emph{mottagarens interna brus} (eng. \emph{receiver noise}),
varvid de övriga bidragen har liten påverkan på den totala uträkningen.
Det är därför praktiskt att fort estimera om det är brus på bandet eller
mottagarens brus som dominerar, vartefter man enbart räknar med den
dominerande bruskällan.

Som tumregel kan man säga att för kortvåg är oftast bruset på bandet
den dominerande bruskällan, medan för högre band så kommer det interna
bruset att dominera, och dämpningar i kablar bli allt mer märkbart.

\subsubsection{Signal-brus-förhållande}
\harecsection{\harec{a}{7.1.2}{7.1.2}}
\index{signal-brus-förhållande}
\index{brus!signal-brus-förhållande}
\index{S/N}
\index{brus!S/N}

Upplevelsen av en signals kvalitet kan mätas på många sätt, dock är just
signalens brusmängd en viktig sådan relation och därför så använder man
begreppet \emph{signal-brus-förhållande} (eng.
\emph{signal to noise ratio, S/N}).

Signal-brus förhållandet uttrycks oftast i \unit{\decibel} och kan enkelt räknas
fram som skillnaden i nivå på signal och på brus, det vill säga signal minus
brus, räknat i \unit{\decibel}.
Med en signalnivå på \qty{45}{\decibel} och brusnivå på \qty{22}{\decibel} har
man således \qty{+23}{\decibel} S/N.

\subsubsection{Minimal signal-brus-förhållande}
\harecsection{\harec{a}{7.20.2}{7.20.2}}

Det är också till stor hjälp att fort etablera det \emph{minimala
signal-brus-förhållandet} (eng. \emph{minimum signal to noise ratio})
som man kan tolerera.
Genom att jämföra länkbudgeten mot detta kan man fort avgöra om det är
tillräckligt bra eller behöver ändras.

Har man för lågt signal-brus-förhållande mot minimum, så behöver man öka
förstärkningen eller oftast minska förlusterna i länkbudgeten.

\subsubsection{Minimal mottagen signalstyrka}
\harecsection{\harec{a}{7.20.3}{7.20.3}}

Om den dominanta störkällan är det interna bruset och man har för lågt
signal-brus-förhållande, måste signalstyrkan in till mottagaren ökas
tills signalen är stark nog för att ge tillräckligt högt
signal-brus-förhållande.
Detta ger nivån för \emph{minimalt mottagen signalstyrka} (eng.
\emph{minimum receiver signal power}) som mottagaren kräver.

\subsubsection{Signaldämpning}
\harecsection{\harec{a}{7.1.1}{7.1.1}}
\index{dämpning}

Så väl kablar, filter, kopplingar och vågutbredning innebär
\emph{signaldämpning}.
Det innebär att man tappar energi i förhållande till den tillförda energin.
Förhållandet mellan uttagen och inmatad energi uttrycks oftast i form av
\unit{\decibel}.
Man ska vara noga att notera dämpningen är oftast relaterad till frekvensen,
så den ska uppskattas eller mätas för den frekvens som avses.

\subsubsection{Brusfaktor}
\label{brusfaktor}
\index{brus}
\index{brusfaktor}
\index{brus!brusfaktor}
\index{noise factor}
\index{NF}
\index{brus!noise factor}
\index{brus!NF}
\index{LNA}

För högre frekvenser tenderar brus domineras av mottagarens interna brus,
samtidigt som kabelförluster börjar bli märkbara.
För sådana fall kan det vara lämpligt att installera en
\emph{lågbrusig förstärkare} (eng. \emph{low noise amplifier, LNA}) före
mottagaren.
Även den förstärkaren har dock egenbrus som sedan förstärks.
\emph{Brusfaktor} (eng. \emph{noise factor, NF}) ger förhållandet mellan en
förstärkares egenbrus i förhållande till det termiska bruset för ett motstånd
på dess ingång.
Brusfaktor redovisas oftast i \unit{\decibel}, som varande dB över brusgolvet.

En förstärkares egenbrus kommer givetvis att förstärkas, och därför kommer
bruset på utgången vara brusfaktorn plus förstärkningen, räknat i \unit{\decibel}.
Exempelvis kommer en förstärkare med \qty{4,5}{\decibel} i brusfaktor och
\qty{20}{\decibel} förstärkning ha brus på \qty{24,5}{\decibel} över brusgolvet.
En efterföljande förstärkare med \qty{10}{\decibel} brusfaktor kommer inte
signifikant bidra med brus, eftersom föregående steg har \qty{14,5}{\decibel}
högre brus än egenbruset.
För detta fall är den första förstärkaren dominant.

Eftersom kabeldämpning kan vara signifikant, så kommer signalen dämpas genom
kabeln.
Givet att vi har \qty{15}{\decibel} dämpning i kabeln, en signal
\qty{40}{\decibel} över brusgolvet och en \qty{20}{\decibel} förstärkare med
brusfaktor \qty{4,5}{\decibel}, var ska vi sätta förstärkaren?

Om förstärkaren sitter efter kabeln så kommer signalen att dämpas först i
kabeln, bruset kommer att läggas på och sedan kommer det att förstärkas
\qty{20}{\decibel}.
Det ger \qty{40}{\decibel} minus \qty{15}{\decibel} plus \qty{20}{\decibel} för
signalen, det vill säga \qty{45}{\decibel}.
För bruset får vi \qty{4,5}{\decibel} plus \qty{20}{\decibel} det vill säga
\qty{24,5}{\decibel}.
För detta fallet får vi ett signal-brus-förhållande på \qty{45}{\decibel} minus
\qty{24,5}{\decibel}, det vill säga \qty{20,5}{\decibel}, givet att det är det
interna bruset som är dominerande.

Om förstärkaren sitter före kabeln så kommer signalen först förstärkas och
sedan dämpas i kabeln.
Det ger \qty{40}{\decibel} plus \qty{20}{\decibel} minus \qty{15}{\decibel} för
signalen, det vill säga \qty{45}{\decibel}.
För bruset får vi \qty{4,5}{\decibel} plus \qty{20}{\decibel} minus
\qty{15}{\decibel} det vill säga \qty{9,5}{\decibel}.
För detta fallet får vi ett signal-brus-förhållande på \qty{45}{\decibel} minus
\qty{9,5}{\decibel}, det vill säga \qty{35,5}{\decibel}, givet att det är det
interna bruset som är dominerande.

Med detta exempel ser vi hur en länkbudget hjälper oss att få
signal-brus-förhållandet att gå från \qty{20,5}{\decibel} till
\qty{35,5}{\decibel} enbart genom att ändra placeringen av förstärkaren i
systemet.

\subsubsection{Vägförlust}
\harecsection{\harec{a}{7.20.4}{7.20.4}}
\index{vägförlust}
\index{path loss}
\index{fresnelzon}
\index{närfält}

Den dämpning som signalen har i fri-rymds förlust, på grund av dess avtagande
fältstyrka kallas även för \emph{vägförlust} (eng. \emph{path loss}).
Fri-rymds förlusten beror förenklat på frekvens och avstånd, en enkel model
\cite[\S 19.1.2]{ARRLHDB2015}:
%%
\[L_{fs} = 32.45 + 20\log d + 20\log f\]
%%
där \(d\) är avståndet i \unit{\kilo\metre} och \(f\) är frekvensen i
\unit{\mega\hertz} och det ger \(L_{fs}\) är vägförlusten i \unit{\decibel}.
Fri-rymds förlusten avtar med kvadraten på avståndet, det vill säga
\qty{6}{\decibel} på dubblat avstånd och det är i allmänhet den dominerande
effekten när man lämnat antennens närfält.

Ytterligare förluster kan förekomma på grund av vegetation, delvis täckt
\emph{Fresnelzon}, studsar mot jonosfär med mera.
Fresnelzonen är den zon som befinner sig inom den ellips vars form definieras
av en våglängd längre väg än direkt väg mellan sändare och mottagare.
Merparten av energin mellan två antenner rör sig i denna fresnelzon och
således så om den regionen är påverkad av hinder så kommer signalen dämpas
märkbart.

\subsubsection{Antennförstärkning och kabelförluster}
\harecsection{\harec{a}{7.20.5}{7.20.5}}
\index{antennvinst}
\index{antenn!vinst}
\index{antennförstärkning}
\index{antenn!förstärkning}
\index{kabelförluster}

En antenns direktivitet ger antennen en \emph{antennförstärkning} (eng.
\emph{antenna gain}) då den i en viss riktning har en förmåga att ha högre
förstärkning än en enkel dipol.

Antennens förmåga att undertrycka andra signaler, till exempel som mätt med
fram-back-relationen, ger också en undertryckning av oönskade signaler och
atmosfäriskt brus.
Det kan därför vara värt att inte enbart mäta antennens förstärkning av
önskad signal, utan även räkna på dess förmåga att ta in oönskat brus och
störande signaler.

Anslutna kablar kan ha signifikant påverkan på både sänd- och mottagen
signalstyrka då \emph{kabelförlusterna} (eng. \emph{transmission line losses})
kommer dämpa signaler.
Kabelförluster beror på hur lång kabeln är, vilken kabel det är samt vid
vilken frekvens man använder.
Som regel har högre frekvenser högre dämpning.
Både storleken på kabeln och val av dielektrium påverkar förlusten i kabeln.

\subsubsection{Minsta sända signalstyrkan}
\harecsection{\harec{a}{7.20.6}{7.20.6}}

En sändare har en varierande uteffekt, och eftersom man försöker åstadkomma en
uppskattning på sämsta signal-brus-förhållandet så är det inte maxeffekten
eller ens medel effekten som blir den intressanta, utan den \emph{minsta sända
signalstrykan} (eng. \emph{minimum transmitter power}).
Genom att använda sig av den i beräkningen på sin länk-budget så försäkrar man
sig om att länk-budgeten hanterar sämsta tänkbara fall, och för de fall som
sändaren är starkare så får man alltså bättre signal än den lägsta man
tolererar.
På detta sätt bygger man sig marginaler i beräkningen.

\subsubsection{Sammanställning av länkbudget}

En komplett länkbudget fås genom att räkna på signalstyrka respektive brusnvå
för varje steg i kedjan, genom att gå igenom alla förstärkningar och förluster
längs vägen.
När man sedan har räknat fram mottagarens upplevda signalstyrka och brusnivå så
kan man räkna fram signal-brus förhållandet, se exemplet i \ssaref{brusfaktor}.

För att försäkra sig om att det fungerar brukar man räkna konservativt, det
vill säga man väljer de sämsta siffrorna, till exempel minsta mottagen effekt
och minsta sända effekt.

En väl utförd länkbudget ger god förståelse över var den svaga länken är,
och genom att experimentera med olika alternativa lösningar så kan man
förstå var man ska börja göra åtgärder och var det är lönlöst eller har
ringa påverkan.

%
%
% Kapitel 10 Mätteknik
\chapter{Mätteknik}

I forskning, utveckling och produktion är mätning en hörnpelare i verksamheten.
Även inom mättekniken sker en snabb utveckling och digitaltekniken kommer
alltmer till användning, men grunderna för mätning är desamma.
I detta kapitel behandlas de viktigaste mättekniska begreppen som radioamatörer
kan behöva känna till.

% Avsnitt 10.1 Att mäta
\input{koncept/maetteknik--att-maeta}
% Avsnitt 10.2 Mätinstrument
\section{Mätinstrument}
\harecsection{\harec{a}{8.2}{8.2}}

\subsection{Att mäta är att veta}

Mätinstrument används för att, under kontrollerade former, testa och bekräfta
en funktion eller avsaknad av densamma, i en utrustning.
Det används också för att mäta olika komponenter för att verifiera dess
funktion och egenskaper.

Med mätinstrument vill man åstadkomma en så snarlik testmiljö som man kan
förvänta sig, att utrustningen som man testar, ska kunna hantera i
verkligheten.

De mätinstrument vi kommer att nämna nedan, används bland annat i sluttester,
som när, i vårt fall, radioutrustningen är hopmonterad och ska testas, innan den
levereras till kund.

Att tillverkarna använder mätinstrument enligt ovan, är sålunda en förutsättning
för att kunna veta att den utrustning man konstruerat uppfyller de krav som man
specificerat.

Dessa typer av instrument är också av intresse för sändaramatören, som kan
använda dessa för felsökning eller service.

\subsection{Presentation av mätvärden}

\tallfig{images/cropped_pdfs/bild_2_8-02.pdf}{Presentation av mätvärden}{fig:bildII8-2}

Mätvärden kan presenteras på olika sätt som illustreras i
bild~\ssaref{fig:bildII8-2}.
De vanligaste sätten är optiska och då med digital eller analog visning.
Mätresultat kan även överföras till dator för vidare bearbetning och visning.

\subsection{Multimeter}
\harecsection{\harec{a}{8.2.1.1}{8.2.1.1}}
\index{multimeter}

Flera mätfunktioner kan utföras med samma basinstrument, som visas i
bild~\ssaref{fig:bildII8-2}, denna egenskap kallas för en \emph{multimeter}.
Genom omkoppling mellan olika tillsatser väljer man mätfunktion och mätområde.
Instrumentskalan utformas så att olika slags mätvärden kan avläsas.
Kombinationer med elektroniska förstärkare och digital visning etc. är nu
vanligt.

\subsection{Vridspoleinstrument}
\index{vridspoleinstrument}

\mediumfig{images/cropped_pdfs/bild_2_8-03.pdf}{Vridspoleinstrument}{fig:bildII8-3}

\emph{Vridspoleinstrument}, som illustreras i bild~\ssaref{fig:bildII8-3}, kan
bara användas för likströmsmätning, eftersom visarutslaget beror av
strömriktningen.
Instrumentet har låg effektförbrukning och stor noggrannhet.
Visningen är vanligen linjär, men kan göras annorlunda.

\paragraph{Funktion}
En spole är upplagrad i fältet av en hästskomagnet.
När den ström, som ska mätas, passerar genom den vridbara spolen så alstras ett
magnetfält även i denna.
De två magnetfälten påverkar varandra så att spolen vrider sig.
Spolen förses med en visare och en returfjäder.
Ju större ström det flyter genom spolen desto större blir visarutslaget.

\smallfig[0.3]{images/cropped_pdfs/bild_2_8-05.pdf}{Konstlast}{fig:bildII8-5}
\mediumbotfig{images/cropped_pdfs/bild_2_8-06.pdf}{Fältstyrkemätare}{fig:bildII8-6}

\subsection{Konstlast}
\label{konstlast}
\index{konstlast}
\index{dummy load}

En \emph{konstlast} (eng. \emph{dummy load}) är en alternativ last som kan
hantera en viss mängd effekt, konstlast illustreras i bild~\ssaref{fig:bildII8-5}.
En konstlast bör ingå i varje amatörradiostation.
Vid mätning och inställning av till exempel modulation och uteffekt, är det
lämpligt att belasta sändaren med dess nominella utgångsimpedans.
För att då undvika att energi strålas ut bör en väl skärmad konstlast användas.

I moderna amatörradiosändare med koaxialkabel utgång är utgångsimpedansen
\qty{50}{\ohm}.
Konstlasten ska då vara en \qty{50}{\ohm} resistor utan reaktiva egenskaper för
det intressanta frekvensområdet.
Den kan bestå av en eller flera sammankopplade resistorer, ofta parallellt för
att minska den induktiva komponenten.

Sändareffekten ska kunna tas upp utan att resistansen förändras nämnvärt.
Det är viktigt att resistorerna kyls effektivt med luft eller vätska i ett kärl
med tillräckligt utrymme, även när vätskan expanderar av värmen.
Vätskan får inte vara lättantändlig eller miljöfarlig.
Till exempel är oljor med PCB förbjudna!

\subsection{Fältstyrkemätare}
\index{fältstyrkemätare}
\label{fältstyrkemätare}

% \mediumbotfig{images/cropped_pdfs/bild_2_8-06.pdf}{Fältstyrkemätare}{fig:bildII8-6}

Styrkan av elektromagnetiska fält kan bestämmas med \emph{fältstyrkemätare}.

En fältstyrkemätare är en högfrekvensdetektor, vars utspänning visas med ett
instrument med skala.
Den selektiva kretsen kan bestå enbart av den avstämda antennen, men även av
ytterligare selektiva kretsar.
Instrumentet visar endast relativa värden och används till exempel för att
bestämma strålningsegenskaperna i sändarantenner och för antennjustering.
Mätresultatet påverkas även av utstrålning från andra sändare inom mätarens
bandbredd.
Bild~\ssaref{fig:bildII8-6} visar en sändare och en fältstyrkemätare.
Dessutom två enkla fältstyrkemätare.

\newpage
\subsection{Kalibreringsoscillator}
\index{kalibreringsoscillator}

\smallfig{images/cropped_pdfs/bild_2_8-07.pdf}{Kalibreringsoscillator i mottagare}{fig:bildII8-7}

En \emph{kalibreringsoscillator} (eng. \emph{calibration oscillator}) används
för att frekvenskalibrera andra apparaters inställningsskalor, som illustreras
i bild~\ssaref{fig:bildII8-7}.
Den är kristallstyrd och avger särskilt precisa och frekvensstabila signaler.

Oscillatorsignalen förvrängs avsiktligt, så att det utöver grundfrekvensen även
skapas harmoniska övertoner.
En oscillator med till exempel grundfrekvensen \qty{25}{\kilo\hertz} avger på så
sätt även frekvenserna \qty{50}{\kilo\hertz}, \qty{75}{\kilo\hertz},
\qty{100}{\kilo\hertz}, \qty{125}{\kilo\hertz} och så vidare.
Man får således en ''kalibreringsfrekvens'' för varje \qty{25}{\kilo\hertz}.

Detta övertonsspektrum kan sträcka flera \qty{100}{\mega\hertz} upp.
Man ''nollsvävar'' apparat mot närmaste kalibreringsfrekvens och kan kalibrera
till exempel VFO-skalan.

Användningsområdet är huvudsakligen kalibrering av äldre mottagare och gradering
av nya skalor och så vidare för densamma.
Dagens mottagare och sändare har syntesoscillator och då behövs normalt ingen
kalibreringsoscillator.

\paragraph{Not}
Äldre trafikmottagare har VFO med LC-krets och ofta en inbyggd
kalibreringsoscillator, vilken i sin tur kan behöva kalibreras.
Det enklaste sättet är då, att jämföra frekvensen på en känd rundradiosändare
på mellanvåg med kalibreringsoscillatorn.

\subsection{Brusmätbrygga}
\index{brusmätbrygga}
\index{Wheatstones brygga}

\smallfigpad{images/cropped_pdfs/bild_2_8-08.pdf}{Brusmätbrygga}{fig:bildII8-8}

\emph{Brusmätbryggan} används vid mätning i antennsystem, så som illustreras i
bild~\ssaref{fig:bildII8-8}.
Den består av en brusgenerator och en Wheatstonebrygga för mätning av
resistans och reaktans.

Till bryggan ansluts en antenn som mätobjekt och en mottagare som
nollindikeringsinstrument för brussignalen.
Mottagaren ställs in på den frekvens där mätvärden önskas.
Bruset hörs svagast när bryggan är injusterad.
Man kan då avläsa mätvärdena för \(R\) och \(X\).
Mäter man vid flera frekvenser, kan till exempel ett impedansdiagram upprättas.
Detta är med andra ord en äldre förlaga till en nätverksanalysator.

\subsection{Ståendevågmeter (SVF-meter)}
\harecsection{\harec{a}{8.2.1.3}{8.2.1.3}}
\index{ståendevågmeter}
\index{SWR-meter|see {ståendevågmeter}}
\index{ståendevåg-förhållande (SVF)}
\index{Standing Wave Ratio}
\index{framåtgående effekt}
\index{forward power|see {framåtgående effekt}}
\index{bakåtgående effekt}
\index{backward power, reflected power}
\index{SVF|see {ståendevåg-förhållande (SVF)}}
\index{SWR|see {Standing Wave Ratio}}
\label{SVF}

\mediumfig{images/cropped_pdfs/bild_2_8-09.pdf}{SVF-meter, princip och inkoppling}{fig:bildII8-9}

När en transmissionsledning eller apparat ansluts till en annan med
avvikande impedans, kommer HF-energi att reflekteras i övergången.

Denna reflekterade energi kan mätas med en \emph{ståendevågmeter}
(eng. \emph{SWR-meter}) så som illustreras i bild~\ssaref{fig:bildII8-9}.
Med \emph{ståendevåg-förhållande (SVF)} (eng. \emph{Standing Wave Ratio, SWR})
menas förhållandet mellan den effekt som flyter framåt respektive bakåt i en
transmissionsledning.
Användningsområden för SVF-meter är:

\begin{itemize}
\item Mätning av \emph{framåtgående effekt} (eng. \emph{forward power}).
\item Mätning av \emph{bakåtgående effekt} (eng. \emph{backward power, reflected power}).
\item Bestämning av \emph{SVF} (eng. \emph{SWR}).
\item Bestämning av resulterande, relativ effekt.
\end{itemize}

\paragraph{Anmärkning} Vid bestämning av absolut effekt måste
anslutningsimpedansen vara lika i instrument och transmissionsledning.

SVF-metern är ett av de mest användbara instrumenten vid HF-mätningar.
En SVF-meter kan ha separata instrument för fram- respektive backeffekt eller
ett gemensamt.

SVF-metern kan vara ständigt inkopplad till exempel mellan sändare och antenn,
men ska då kunna tåla effektutvecklingen.
En SVF-meter kan alstra övertoner, vilka kan medföra störningar.
Orsaken är olinjäriteten hos halvledardioderna i instrumentet.

\subsection{Frekvensräknare}
\harecsection{\harec{a}{8.2.1.5}{8.2.1.5}}
\index{frekvensräknare}

\smallfig{images/cropped_pdfs/bild_2_8-10.pdf}{Frekvensräknare}{fig:bildII8-10}

\emph{Frekvensräknaren} (eng. \emph{frequency counter}), som är ett digitalt
instrument, används för att bestämma oscillatorfrekvensen i sändare,
mottagare med mera.

Bild~\ssaref{fig:bildII8-10} illustrerar den schematiska bilden av en
frekvensräknare.
I frekvensräknaren räknas antalet svängningar \(E\) (från engelskans events)
i den aktuella inkommande signalen under en bestämd tidsenhet \(t\).
Först förstärks signalen i en analog förstärkare och omvandlas till
kantvågspulser i ingångsstegets triggerenhet.
När varje mätning börjar så kommer en räknare räkna hur många triggerpulser
som passerat fram tills dess den inställda tiden löpt ut.
Moderna räknare mäter även hur den första pulsen (eng. \emph{start event})
respektive sista pulsen (eng. \emph{stop event}) skiljer i tid, så att den
egentliga tiden kan användas, för att få en hög upplösning.
Frekvensen kan nu \emph{estimeras} med formeln:
%%
\[f_{est} = \dfrac{E}{t}\]
%%
Gamla frekvensräknare hade en fix tid som den räknade över, och utan justering
av egentlig tid.
Dessa har en enkel princip och tiden valdes ofta för att få en enkel skalfaktor
mellan räknarens värde och frekvensen, genom att ha steg om
\qty{100}{\milli\second}, \qty{1}{\second}, \qty{10}{\second} och så vidare.
Dessa räknare har dock problemet att för låga frekvenser så krävs lång
observationstid för att försäkra sig om en tillräckligt bra numerisk precision.
En variant av detta som framkom var den så kallade reciproka räknaren, som är
den nu förhärskande principen när man behöver precision, i den så mäts både
hur många event och tiden.
Detta gör att man kan låta tidbasen enkelt varieras med en pot eller extern
signal.

Ytterligare en förbättring som kom är att interpolera tiden för den inledande
och den avslutande pulsen gentemot tidbasens klocka, för att därför kunna
justera mätningen med ett bättre estimat på tiden det egentligen tog för de
räknade eventen att hända.
Med tidsupplösning på 1~ps-nivå kan därför 12~siffrors noggrannhet presenteras
för en mätning över 1~sekund, medan för gamla frekvensräknare med sin
\qty{10}{\mega\hertz} oscillator gav \qty{100}{\nano\second} upplösning och
därmed enbart 7~siffrors noggrannhet för samma 1~sekund mätning.

De moderna frekvensräknarna har nu mer även filter som sammanställer flera
mätningar till en, och presenterar resultat överlappande.
Detta ger en uppfattad högre avläsningshastighet, men mätningarna är inte helt
oberoende.
Vissa frekvensräknare använder även linjärregression för att ytterligare
filtrera bort mätbrus.

Resultatet visas som siffror i ett fönster.
Noggrannheten i den så kallade tidbasen erhålls med en kristallstyrd oscillator
eller för dyrare instrument med en rubidiumnormal.
Man kan ofta ansluta en extern frekvensnormal med frekvens på \qty{10}{\mega\hertz},
vilket gör att man med moderna GPS-styrda oscillatorer kan få tillgång till
SI-definitionen av hertz till en nu mer modest kostnad även i ett hobbylabb.

\subsection{Dipmeter}
\index{dipmeter}

\smallfig{images/cropped_pdfs/bild_2_8-12.pdf}{Dip-meter}{fig:bildII8-12}

\smallfig{images/cropped_pdfs/bild_2_8-13.pdf}{Mätning med dip-meter}{fig:bildII8-13}

\emph{Dipmetern} är i princip en oscillator med variabel frekvens och utbytbara
induktorer för olika frekvensområden, så som visas i bild~\ssaref{fig:bildII8-12}.
Den används för att bestämma resonansfrekvensen på passiva och aktiva
resonanskretsar samt vid bestämning av induktanser och kapacitanser.
Noggrannheten är cirka \qty{3}{\percent}.

\paragraph{Funktion}
Instrumentet avger alternativt reagerar för en HF-signal med viss frekvens.
Frekvensen i dipmeterns resonanskrets är steglöst variabel och frekvensvärdet
kan avläsas på en skala.

Vid mätning av resonansfrekvensen i en passiv resonanskrets kopplas dipmeterns
induktor induktivt till kretsen så som visas i bild~\ssaref{fig:bildII8-13}.
När resonansfrekvensen i kretsen och dipmetern överensstämmer, ändras 
belastningen på dipmeterns resonanskrets varvid instrumentet uppvisar en
strömminskning -- en ''dip''.
Frekvensen avläses då på skalskivan.

Vid mätning på en aktiv resonanskrets, det vill säga som drivs av någon
HF-källa, uppstår i stället en strömökning vid resonans vilket också visas på
instrumentet.

Induktansen i en resonanskrets kan bestämmas med dip-metern, om kapacitansen
är bekant.
På motsvarande sätt kan en obekant kapacitans bestämmas om induktansen i
resonanskretsen är bekant.

Namnet griddipmeter kommer från elektronrörsepoken.
Ändringar i gallerströmmen (grid current) i ett oscillatorkopplat elektronrör
används som indikation på att en resonanskrets är i resonans.
Då minskar gallerströmmen -- det blir en ''ström-dip''.
Numera används en transistor i stället för röret och instrumentet benämns
dip-meter.

\subsection{Oscilloskop}
\harecsection{\harec{a}{8.2.1.6}{8.2.1.6}}
\index{oscilloskop}

\mediumfig{images/cropped_pdfs/bild_2_8-14.pdf}{Oscilloskop}{fig:bildII8-14}

\emph{Oscilloskopet} (eng. \emph{oscilloscope}) är ett mycket användbart
instrument.
Mycket snabba förlopp kan med fördel studeras på en oscilloskopskärm.

Spänningsförlopp kan visas som funktion av tiden.
Tillsammans med andra instrument kan frekvenskaraktäristiken i filter,
modulationskvalitet och så vidare åskådliggöras.

Äldre oscilloskop består av ett katodstrålerör, där styrningen av katodstrålen
sker med hjälp av X- och Y-förstärkare och en så kallad triggerförstärkare.
Den signal som ska mätas ansluts vanligen till Y-förstärkaren medan en
tidbasgenerator som alstrar en sågtandsformad signal ansluts till X-förstärkaren.
Bild~\ssaref{fig:bildII8-14} visar ett blockschema på oscilloskop.

Moderna oscilloskop digitaliserar signalen efter ingångsförstärkaren,
och läggs sedan i minne, där den sågtandsformade signalen är ersatt med en
räknare som placerar det i minnet.
Sen presenteras bilden på bildskärmen eller på en ansluten dator.

Gemensamt för analoga och digitala oscilloskop är i stort samma handhavande.
Man ansluter en eller flera signaler till ingångarna, justerar ingångssteget
så att hela vågformen fångas och att det är god amplitud, så den syns men inte
klipps.
Ibland väljer man att göra vågformerna mindre, för att man ska kunna
arrangera dem på ett bra sätt på skärmen.
Ibland väljer man att klippa vågen för att man enbart vill se tiden för en
kant tydligt.

En viktig sak för att få en tydlig bild på bildskärmen är att triggpunkten,
den punkt där mätningen av signalen börjar, är vald så att man inte får dubbla
eller otydliga bilder.
Valet av triggpunkt sker ofta automatiskt men om signalen som ska mätas är
väldigt flack eller har många nollgenomgångar kan triggpunkten bli felaktig.
För att lösa problemet med suddiga eller dubbla bilder så brukar man manuellt
justera triggpunkten så att man får en tydlig bild.

Man justerar också tidbasen för att ha rätt skala på tidsaxeln, så man ser en
eller ett fåtal cykler, eller ibland över längre tid för att se variationer.
Man kan också fördröja svepet för att kunna se en viss del efter triggpunkten.

Oscilloskop har nu mera ofta inbyggda funktioner för mätning.
Det är behändigt att snabbt få en uppfattning om periodtider, frekvenser,
amplituder med mera men dessvärre blir mätprecisionen ofta kraftigt lidande,
och det ska inte övertolkas.
Till exempel är frekvensmätningen inte bättre än hur bra placeringen av
markörerna är, så det är sällan tillförlitligheten är bättre än 2 siffrors
noggrannhet.

Rätt använt är oscilloskop dock ett fantastiskt mätinstrument.

\subsection{Spektrumanalysator}
\harecsection{\harec{a}{8.2.1.7}{8.2.1.7}}
\label{spektrumanalysator}

En \emph{spektrumanalysator} (eng. \emph{spectrum analyzer}) visar amplituden
för olika frekvenser över ett visst frekvensområde.
Detta är som kontrast till oscilloskopet som visar amplitud för en signal
över tiden.

Spektrumanalysatorn kan liknas vid en mottagare, men med en viktig skillnad:
där en mottagare har ett eller flera avstämda ingångssteg, som ska förhindra
mottagaren att påverkas av signaler som ligger mer eller mindre nära den
önskade signalen, har spektrumanalysatorn en vidöppen ingång.

Spektrumanalysatorn är ett mätinstrument, och ska kunna presentera de
signaler som matas in på dess ingång, utan att signalerna ska påverkas på
något sätt.
Detta ställer höga krav på spektrumanalysatorns konstruktion.
Den måste kunna tåla starka signaler, utan att en mätning på en svag signal i
närheten påverkas.

Spektrumanalysatorerna finns också i två olika typer.
Den ena arbetar med svepteknik, och sveper över ett viss del av
frekvensspektrumet.
Den andra typen kallas för \emph{realtidsanalysator}, och är kapabel att för
ett givet ögonblick spela in ett spektrum digitalt för senare analys av
innehållet.

Det vi fortsättningsvis beskriver i detta kapitel, är den svepande
spektrumanalysatorn, vilket också är den vanligaste typen för
service tillämpningar.

En spektrumanalysator består förenklat av en svepbar oscillator, variabelt
filter på mellanfrekvensen, en detektor och ett ställbart filter från detektorn.

Den variabla oscillatorn sveper så att det tänkta frekvensområdet täcks, ofta
med ett bestämt antal punkter, till exempel 801, där varje punkt är mätning på
en specifik frekvens.
Ibland kan man anpassa antalet punkter för att få mätningen gå snabbare,
mot att man får en lägre upplösning.

Filtret sätter bandbredden på mätningen, och det kan gå i 1--3 steg, till
exempel \qty{3}{\hertz}, \qty{10}{\hertz}, \qty{30}{\hertz}, osv. till
\qty{3}{\mega\hertz}.
För vissa specifika mätändamål, som till exempel för EMC mätningar, behöver man
ha filter av rätt bandbredd och dessa är extra optioner.

Filtret, vars inställning brukar kallas för \emph{Resolution Bandwidth},
fungerar som ett fönster, där fönstret släpper in de signaler som finns i det
spektrum som man önskar studera.

Om analysatorn inte skulle ha något filter skulle den ta in alla signaler som
finns inom dess specificerade frekvensområde.

Ett brett filter släpper igenom ett större frekvensområde och är användbart för
signaler med större bandbredd.
Ett smalare filter är att föredra för signaler med smalare bandbredd.

Ett brett filter innebär också att analysatorn kan svepa snabbare.
Det är då lättare att kunna detektera signaler som är kortvariga.
Ett smalare filter innebär att analysatorn måste svepa långsammare, men kan då
istället hitta signaler som inte skulle ha synts med det bredare filtret.

Detta filter har också en annan egenskap som är viktig -- ett brett filter
släpper också igenom mer brus, vilket påverkar den lägsta brusnivån som
analysatorn kan presentera.

För att nå högre känslighet kan man välja smalare filter.
Ju smalare filter, desto högre känslighet, men också ett långsammare svep över
det aktuella frekvensområdet.

\subsubsection{Fördjupning}

På en spektrumanalysator ställs stora krav på känslighet och förmåga att hantera
starka signaler i närheten av en svag men önskad signal utan att falska signaler
påverkar instrumentet.
Kraven har medfört att kostnaden för instrumentets uppbyggnad och ingående
komponenter under lång tid har varit mycket hög.

Under många decennier har sändaramatörer därför varit hänvisade till
marknaden för begagnade instrument som haft minst tio till tjugo år på nacken.

Det är bara för några år sedan som det har kommit produkter på marknaden som nu
kommit ner i kostnadsnivåer som inneburit att radioamatörer kan köpa dessa
instrument, i nyskick.

En modern spektrumanalysator av dyrare snitt erbjuder också en möjlighet att
analysera den modulerade signalen.
Sålunda förekommer instrument på marknaden som är specialdesignade att
analysera signaleringsinnehållet i olika system, till exempel Bluetooth, olika
Wi-Fi- och mobiltelefonsystem.

För en specifik mätning över ett visst frekvensspektrum, kan man ställa in en
så kallad startfrekvens, samt motsvarande stoppfrekvens för spektrumet ifråga.
Analysatorn kommer då att svepa, från startfrekvensen, till stoppfrekvensen.
Man kan också att välja en frekvens mitt i spektrumet, och därefter ett
valfritt så kallat span.
\emph{Span} betecknar det frekvensspann som är önskvärd för att man ska
kunna studera signalerna inom det aktuella spektrumet.

För att kunna göra bättre avläsningar kan man sätta markörer, så att man kan
avläsa frekvens och amplitud för den punkten.
Ibland sätter man dubbla frekvensen för att avläsa skillnaden i amplitud,
vilket kan vara relevant för filter eller avläsa den relativa styrkan på ett
sidband.

Detektorn kan vara topp-detektor eller RMS-detektor; det kan finnas flera.
För specifika mätändamål som till exempel EMC-mätningar behöver man en
specifik detektor.
Det är viktigt att välja rätt detektor när signalen ska presenteras.
Valet av detektor kan påverka den presenterade nivån med flera \unit{\decibel}.

Det finns olika detektorer beroende på hur man vill ha signalen presenterad.
Det finns toppvärdesindikerande (Auto-peak, max peak, min peak) detektorer som
indikerar signalens toppvärden.
Det finns medelvärdesbildande (\emph{Average}, \emph{Sample}) detektorer som, om
instrumentet arbetar med digitalteknik, plockar ögonblickliga mätvärden
slumpmässigt.
Det finns också speciella detektorer (s.k. Quasi-Peak) som används för att
mäta enligt specifika EMC-mätningar.

En viktig detektor att komma ihåg är den så kallade RMS-detektorn.
Den utvecklades för att kunna mäta på digitalt modulerade signaler, ofta med
varierande fas- och amplitudinformation.

Denna detektor är att rekommendera för att mäta på digitalt modulerade signaler.
Den finns oftast i lite modernare analysatorer.

En vanlig Average- eller Sample-detektor enligt ovan, förväntar sig att
RF-signalens förlopp är i stort sett återkommande konstant, vilket är fallet
med analoga signaler som består av en bärvåg.

En digitalt modulerad signal har ett innehåll som förändras hela tiden, oftast
både till fas och amplitud.

RMS-detektorn läser in -- samplar -- den digitalt modulerade signalen och tar
konstant mätvärden ur den fasvarierande RF-signalen.
Den följer RF-signalens förändrade innehåll.

Denna detektor är därför utmärkt att använda för att mäta på digitalt modulerade
signaler, till exempel Bluetooth- eller Wi-Fi-signaler, men även de digitala
system vi har inom amatörradion, då dessa signaler innehåller förändringar i fas
och amplitud.

Den kan självklart också användas för att mäta på analoga signaler.
Att den plockar ögonblicksmätvärden även på en analog signal gör alltså inget.

Efter detektorn finns ett filter, ofta benämnt med videobandbredd, som
medelvärdesbildar detektorns amplitudestimat över tiden.
Oftast regleras det automatiskt med bandbredden på filtret, eftersom smalare
filter behöver proportionerligt längre tid för att ge ett bra resultat.
Sveptiden beror därför på antalet punkter för frekvenser, bandbredd på filtret
och videobandbredden.
Ibland kan man styra videobandbredden manuellt för att få en längre tid, då
det kan vara gynnsamt för att få en tydligare bild, det vill säga ta bort brus
och störningar som enbart skapar variationer för att man inte observerar ett bra
medelvärde.
Videobandbredden påverkar alltså inte själva mätresultatet, utan är enbart till
för att användaren lättare ska kunna avläsa mätningen.

\subsection{Signalgeneratorn}
\harecsection{\harec{a}{8.2.1.4}{8.2.1.4}}

\emph{Signalgeneratorn} (eng. \emph{signal generator}) är ett instrument, som
namnet antyder, genererar en signal, i detta fall en radiofrekvent signal.

Detta instrument kan användas för att till exempel testa mottagare, eller för
att generera en eller flera kontrollerade signaler för att till exempel testa
förstärkarsteg.

Äldre signalgeneratorer var oftast uppbyggda runt en resonanskrets, och drev
ofta i frekvens när de värmdes upp.
De var sålunda inte stabila.
Senare generatorer arbetade med frekvenssyntes, och var att föredra i detta
sammanhang.

Den kan också användas för att generera en testsignal, som man matar in på en
mottagares ingång, för att sedan kunna följa signalen med en spektrumanalysator
(se \ssaref{spektrumanalysator}).

En bra signalgenerator ska ha förmågan att ge en så ren signal som möjligt,
där övertoner och sidband av olika slag är så låga som möjligt.
Ofta genererar generatorn ett egenbrus runt den inställda signalen.
Detta brus avtar, ju längre bort man kommer från den inställda signalen.
Detta brus ska förstås vara så lågt som möjligt.

En annan önskvärd parameter är möjligheten att kunna reglera den radiofrekventa
utnivån över ett stort område.
Signalgeneratorer innehåller oftast någon form av mätfunktion för att kunna
mäta nivån.

En fördel är om signalgeneratorn har en möjlighet att själv skapa modulation,
till exempel AM- eller FM-signaler.
Det kan emellanåt finna en inbyggd lågfrekvensgenerator, där man kan ställa in
önskad frekvens för att till exempel generera en ton.
I detta sammanhang brukar det också finnas möjlighet att justera
modulationsgraden för AM, eller deviationen för FM-signalen.

\subsection{Nätverksanalysator}
\index{nätverksanalysator}
\index{network analyzer}
\index{antennanalysator}
\index{skalär nätverksanalysator}
\index{Scalar Network Analyzer (SNA)}
\index{SNA}
\index{tracking generator}
\index{vektornätverksanalysator}
\index{Vector Network Analyzer (VNA)}
\index{VNA}

En \emph{nätverksanalysator} (eng. \emph{network analyser}) används för att
mäta hur mycket signal som går igenom en koppling, till exempel filter eller
förstärkare, eller hur mycket signal som reflekteras tillbaka från till exempel
en antenn.
Ibland kallas detta även för \emph{antennanalysator} i amatörradiosammanhang.

En nätverksanalysator som enbart kan mäta amplituder kallas ibland för
\emph{skalär nätverksanalysator} (eng. \emph{Scalar Network Analyzer, SNA}).
En spektrumanalysator med en så kallad \emph{tracking generator}, som genererar
en signal med samma frekvens som man analyserar, kan agera SNA.
En signalgenerator med svepfunktion kan också agera SNA.

En nätverksanalysator som mäter fasen både på utgående och inkommande signal
kan även mäta fasförskjutningen, och då kan man representera fasen både som
komplex storhet eller med polära koordinater, det vill säga amplitud och fas.
En sådan nätverksanalysator kallas för \emph{vektornätverksanalysator} (eng.
\emph{Vector Network Analyzer, VNA}).

Användningsmässigt liknar en nätverksanalysator en spektrumanalysator, men
med flera väsentliga skillnader.
För att få korrekt mätning av amplitud och fas läggs större vikt vid att göra
\emph{kalibrering} (eng. \emph{calibration}), något som görs för att kompensera
varierande amplitud och fas för olika frekvenser.
Vid kalibrering använder man ofta \emph{last} (eng. \emph{load}),
\emph{kortslutning} (eng. \emph{short}) samt \emph{öppen port} (eng.
\emph{open}) mätning av kalibreringsreferenser.
För två-ports mätning använder man även en \emph{överföring} (eng.
\emph{through}) för att få port-till-port egenskaperna korrekt.
Efter kalibrering av instrumentet så korrigeras mätningarna, och ibland kan
skillnaderna vara drastiska.

\newpage %layout
Nätverksanalysatorn har därtill ofta ett stort antal olika sätt att presentera
mätresultaten så att man kan mäta enligt \emph{scatter-modellen},
\emph{return loss (RL)}, \emph{VSWR}, \emph{Smith-diagram} och så vidare.
Det gör att en nätverksanalysator kan vara ett kraftfullt verktyg som korrekt
använt kan ge god insikt i hur en krets beter sig.

%
%
% Kapitel 11 Elektromagnetisk kompatibilitet
\chapter{Elektromagnetisk kompatibilitet}
\label{ch:EMC}
\index{EMC}

Det moderna samhället blir alltmer tekniskt avancerat och antalet elektroniska
apparater i hemmen och på arbetsplatserna ökar kraftigt.
Den ökande mängden och komplexiteten hos apparaterna kräver därför regler, som
styr både utförande och användning med rimligt bibehållen säkerhet och funktion.
Internationella och nationella väl preciserade regler för radio- och
teletekniskt samexistens är numera helt nödvändiga.

% Avsnitt 11.1 Störningar och störkänslighet
\input{koncept/emc--stoerningar-och-stoerkanslihet}
% Avsnitt 11.2 Störningar i elektronik
\input{koncept/emc--stoerningar-i-elektronik}
% Avsnitt 11.3 Störningsorsaker
\section{Störningsorsaker}
\label{sec:storningsorsaker}

\subsection{Störningar från sändare}
\harecsection{\harec{a}{9.2.1}{9.2.1}, \harec{a}{9.2.2}{9.2.2}}
\index{störning!av sändare}

HF-förstärkare, till exempel i sändarslutsteg, kan komma i oönskad självsvängning,
vilket kan uppstå av flera orsaker; det kan vara bristande avkoppling av
matningsspänningar, induktiv och/eller kapacitiv återkoppling etc.

Effektförstärkare kan även överstyras.
Då uppstår intermodulation och övertoner som strålas ut på oönskade frekvenser.
I många fall kan störningar undvikas med en eller flera av följande åtgärder:

\begin{itemize}
\item Använd inte mer effekt än vad som behövs.
\item Undvik att överstyra sändarslutsteget (kontrolleras t.ex. med
  ALC-mätaren).
\item Förse sändarutgången med lågpassfilter.
  På så sätt undertrycks övertoner.
\item Anpassa sändarens och antennanläggningens impedanser till varandra.
  Stäm av sändarens \(\pi\)-filter och/eller en separat antennanpassningsenhet
  rätt.
  En felinställd sändare kan medföra oavsiktligt utstrålning.
\item Koppla in balanseringsnät (balun) mellan osymmetriska antennledningar
  (koaxialkablar) och symmetriska antenner.
\item Placera antennen högt och fritt och så långt från personer och
  störningskänslig utrustning som möjligt.
  Fältstyrkan är nämligen högst närmast antennen.
  Se kapitel~\ssaref{ch:emf} om elektriska fält.
\item Undvik direkt HF-instrålning på elnätet genom att använda nätfilter.
\item Använd ''mjuk'' nyckling av bärvågen (avrundade telegrafitecken).
  Vid hård nyckling alstras övertoner i form av knäppar som hörs långt vid
  sidan av sändningsfrekvensen. Se även avsnitt~\ssaref{Nycklingsfilter}.
\end{itemize}

\subsection{Störningar på radiomottagning}
\harecsection{\harec{a}{9.2.3.1}{9.2.3.1}, \harec{a}{9.2.3.2}{9.2.3.2}, \harec{a}{9.2.3.3}{9.2.3.3}}
\index{störning!radiomottagning}

I regel uppstår störningar på radiomottagning först när instrålade signaler
uppnått en viss styrka -- immunitetsnivån för HF.
Man kan tala om tre slags HF-immunitet hos mottagare:

\begin{itemize}
\item mot signaler genom antenningången
\item mot signaler genom övriga anslutna ledningar, till exempel högtalar-
  och nätledningar
\item mot elektriska och/eller magnetiska fält som strålar direkt in i
  apparaten.
\end{itemize}

I de båda första fallen kan det hjälpa med komplettering med hög- och/eller
lågpassfilter och skärmströmsfilter.

Störningar orsakade av instrålning är svårast att avhjälpa och fordrar ingrepp i
mottagaren, vilket bör överlåtas till en fackman med tillgång till
tillverkarens serviceinstruktioner.

\subsection{Störningar på TV-mottagning}
\index{störning!TV-mottagning}
\index{störning!digital-TV}

Störningar från radiosändare kan yttra sig till exempel på följande sätt för
digital TV:

Sändningar, främst på 2-meter men även på 70-cm, kan orsaka blockering och
bildstörningar vid mottagning av digital-TV.
TV-bilden tappar då information, det blir pixlingar (fyrkantiga rutor), grönt
brus eller hela bilden fryses eller försvinner kortvarigt.
För analog TV i till exempel kabel-TV-nät kan störningarna yttra sig på
följande sätt:

\begin{itemize}
\item Vid sändning av amplitudmodulerade signaler, till exempel AM och SSB,
  uppstår ljudförvrängning i ljudkanalen samt ränder med mera i bilden.
\item Vid sändning av FM och CW uppstår ljudstörningar samt
  kontrastvariationer, interferensmönster (moire-effekter) med mera i bilden.
\end{itemize}

Problem med störningar av den här typen har minskat betydligt sedan digital-TV
infördes och de flesta TV-sändningar numera sker på VHF- och UHF-banden.

Störningar i TV som orsakas av sändare på lägre frekvenser kan i många fall
avhjälpas med frekvensfilter.
Ett lågpassfilter efter en kortvågsändare kan till exempel dimensioneras att
endast släppa igenom signaler under cirka \qty{35}{\mega\hertz}.
Läs mer om lågpassfilter i avsnitt~\ssaref{subsec:laagpassfilter}.

Ett högpassfilter före en TV-mottagare kan till exempel dimensioneras att
endast släppa igenom signaler med frekvenser över cirka \qty{35}{\mega\hertz}.
Läs mer om högpassfilter i avsnitt~\ssaref{subsec:hoegpassfilter}.

Om inte mottagning i TV-band I och II är av intresse, så kan ett högpassfilter
med en gränsfrekvens av cirka \qty{160}{\mega\hertz} sättas in.
Det dämpar den oönskade utstrålning från sändare i HF- och lägre VHF-området,
det vill säga upp till och med \SIrange{144}{146}{\mega\hertz} amatörband.
Däremot släpps TV-band III (\SIrange{174}{230}{\mega\hertz}) och TV-banden IV
och V igenom (\SIrange{470}{890}{\mega\hertz}).

Ytterligare avstörningsmedel kan sättas in om det uppstår störningar av
amatörradiosändningar.
Det kan vara skärmströmsfilter på antennkablar, bandspärrar samt sug- och
spärrkretsar avstämda till störfrekvensen, bandpassfilter avstämt till
nyttofrekvensen.
Läs mer om filter i avsnitt~\ssaref{subsec:spaerrfilter}.

Ett vanligt störningsfall är att en bredbandig antennförstärkare blir
överstyrd eller blockerad av en sändare. Se även kapitel~\ssaref{blockering}.

\begin{itemize}
\item Försök att undvika antennförstärkare.
\item Försök att undvika dåligt skärmade skarvar och antennkontakter.
\item Skaffa en bättre TV-antenn som även kan ta emot TV-sändningar på VHF.
  Många hushåll har idag endast en UHF-antenn och har därför dålig
  antennsignal på VHF-bandet där sändningar för HD-TV sker i många områden.
\end{itemize}

\subsection{Störningar på LF-apparater}
\index{störning!LF-apparater}

Störningar av HF-instrålning i ljudbandspelare, LF-förstärkare, telefonapparater
etc. kan ofta stoppas med avkopplingskondensatorer och HF-drosslar.
Moderna avstörningsdrosslar innehåller oftast något ferritmaterial i form av
rör, stavar eller ringar.

% Avsnitt 11.4 Avstörningsmetoder
\input{koncept/emc--avstoerningsmetoder}
%
%
% Kapitel 12 Elektromagnetiska fält

\chapter{ELEKTROMAGNETISKA FÄLT}

\section{INLEDNING}
\index{elektromagnetiska fält (EMF)}
\index{EMF}
En amatörradiostation genererar signaler för att kommunicera trådlöst
med hela världen. Detta sker med signaler som kallas elektromagnetiska
fält (EMF). Runt alla sändande antenner bildas EMF av den energi som
skickas in i dem från sändaren, fälten kallas vanligen radiovågor.

Radiovågorna (EMF) är klassificerade som \emph{icke-joniserande strålning}.

Icke-joniserande elektromagnetisk strålning, vilket är samma sak som
elektromagnetiska fält (EMF), uppträder i många former. Som radiovågor,
mikrovågor, infraröd strålning, synligt ljus, ultraviolett strålning.
Det som skiljer är våglängden. Radiovågor har längst våglängd och
ultraviolett kortast.

\index{joniserande strålning}
\index{strålning!joniserande}
Joniserande strålning är så energirik att den kan rycka loss
elektroner från de atomer som den träffar och förvandla dem till
positivt laddade joner, jonisering. Exempel på joniserande
strålning är röntgenstrålning och strålning från radioaktiva ämnen.
Energin hos den joniserande strålningen kan vara så hög att den kan
tränga in i och påverka cellstrukturen i biologiskt material.

\index{icke-joniserande strålning}
\index{strålning!icke-joniserande}
Energin hos icke-joniserande strålning, som optisk strålning och
elektromagnetiska fält, är normalt inte lika energirik som de
joniserande och kan därför inte jonisera material utan orsakar enbart
uppvärmning av kroppens vävnad. I allmänhet har studier visat att de
nivåer av EMF som allmänheten utsätts för ligger långt under de värden
där kroppstemperaturen skulle öka.

En amatörradiostation genererar således enbart icke-joniserande
strålning vilken absolut inte får blandas ihop med vad som kallas
joniserande strålning. Det är därför mycket viktigt att förstå
skillnaden mellan dem.

\index{WHO}
\index{World Health Organizatio (WHO)}
\index{International Commission on Non-Ionizing Radiation Protection (ICNIRP)}
\index{ICNIRP}
Inom World Health Organization (WHO) finns ett program som kallas
’The International EMF Project’ och där samlas all vetenskaplig
information som finns om biologiska effekter av elektromagnetiska fält.
’International Commission on Non-Ionizing Radiation Protection’, (ICNIRP)
är en fristående organisation (erkänd av WHO) som bland annat använder denna
information för att utveckla riktlinjer för begränsning exponering för EMF.
Dessa riktlinjer används av många länder.

\index{Strålsäkerhetsmyndigheten (SSM)}
\index{SSM}
Strålsäkerhetsmyndigheten (SSM) är den myndighet som arbetar med
ovanstående frågor i Sverige har tagit fram allmänna råd som är
en tolkning av ett EU-direktiv som bygger på ICNIRP:s riktlinjer för
allmänhetens exponering för elektromagnetiska fält.

Eftersom grunden i amatörradioutövandet är att generera
elektromagnetiska fält för att kommunicera via radio så är kunskapen
om EMF viktig. Med de befogenheter radioamatörer har, måste de
allmänna råden för EMF följas och förståelsen för hur fält uppstår
och hur de kan begränsas är fundamental.

\section{FÄLT}
Antennens uppgift är att omvandla den ledningsbundna signalen i
matarkabel på ett så effektivt sätt som möjligt till en
elektromagnetisk våg som utbreder sig i luften.

Den elektromagnetiska vågen uppträder inte direkt vid antennen utan
uppstår i det som man kallar fjärrfältet. Detta sker genom växelverkan
mellan det elektriska och magnetiska fältet som antennen genererar.

\index{närfält}
\index{fjärrfält}
Teorierna som beskriver hur denna växelverkan fungerar är komplicerade
men det viktiga att förstå är att det finns en gräns mellan vad man
kallar fjärrfältet längre bort från antennen och närfältet nära
antennen. Av denna anledning måste man nära antenner mäta både det
elektriska- och det magnetiska fältet för att utvärdera maximal fältstyrka.
I fjärrfältet kan man på grund av växelverkan mellan dem
mäta antingen det ena eller det andra.

Beroende på den antenntyp som genererar fältet är det antingen ett
elektriskt eller magnetiskt fält som dominerar i närfältet.
Elektrisk fältdominans genereras av antenntyper som bygger på
spänningsskillnader (t.ex. dipol) och magnetisk fältdominans av antenner
med strömflöde (t.ex. små loopar).

\index{elektriskt fält (E)}
\index{magnetiskt fält (H)}
Det elektriska fältet (E) anges i ”volt per meter” (V/m) och det
magnetiska fältet (H) i ”ampere per meter” (A/m).

I och med att fältet utbreder sig i luften åt alla håll samtidigt så
avtar styrkan i fältet. Detta beror på att effekten som genererar
fältstyrkan runt antennen täcker ett större och större klot ju
längre bort man kommer. Effekten per ytenhet minskar därmed på
längre avstånd från antennen.

Fältet avtar alltså i styrka på samma sätt som att jämföra ytan på två
klot med olika radier. Genom matematisk analys av dessa klots ytor med
olika radier kommer man fram till att om avståndet dubbleras halveras
fältstyrkan.

Det spelar ingen roll om antennen är helt rundstrålande eller
koncentrerar effekten i en riktning, fältet avtar på samma sätt i
alla fall.

Eftersom alla elektriska ledare kan betraktas som antenner kommer
dessa att kunna generera fält, oavsett om det är tänkt att det ska
vara en antenn eller inte. Detta är tydligt då både apparater och
ledningar kan leda högfrekvent ström och genom detta generera fält.
Man bör ha detta i åtanke vid installation av matarledning till
antennen för att undvika att ström flyter tillbaka till stationen på
utsidan av ledningen. Även de apparater man använder för att generera
de signaler man vill skicka ut kan ha dålig skärmning och leda
högfrekvent ström på utsidan.

Det finns alltså en risk att fältstyrkorna i närheten av sändare, och
framför allt slutsteg med tillhörande kablage, kan vara betydande.

\section{ALLMÄNNA RÅD}

\index{Strålsäkerhetsmyndigheten (SSM)}
\index{SSM}
Strålsäkerhetsmyndigheten (SSM) är den myndighet som är sammanhållande
för allmänhetens exponering för elektromagnetiska fält. Dess allmänna
råd, SSMFS 2008:18, anger vilka referensvärden som gäller i Sverige.

\index{ICNIRP}
Syftet med de allmänna råden är att skydda allmänheten från akuta
skadliga biologiska effekter vid exponering för elektromagnetiska fält.
De allmänna råden är en svensk anpassning av ett EU-direktiv som
grundar sig på ICNIRP’s riktlinjer.

\index{Specific Absorption Rate (SAR)}
\index{SAR}
Riktlinjerna baseras på hur kroppen värms upp av radiovågorna och
definieras som: hur stor effekt per kilogram kroppsvikt som absorberas
under en definierad tid. Den tekniska benämningen på detta värde är
’Specific Absorption Rate’ (SAR) och mäts i W/kg.
De grundläggande begränsningarna är, enligt internationella
rekommendationer, satta vid ungefär två procent av de nivåer vid
vilka akuta biologiska effekter är vetenskapligt säkerställda.

Då uppvärmningen av kroppsvävnad inte går snabbt räknar man med den
medeleffekt som under en viss tid orsakar uppvärmning. De allmänna
råden definierar SAR-värdet som medelvärdet under en sexminutersperiod.

Eftersom ett elektromagnetiskt fält inte går att mäta i enheten W/kg
innehåller de allmänna råden även en tabell på referensvärden
framräknade för att motsvara SAR-värdet. Referensvärdena utgörs av
storheter som är mätbara utanför människokroppen.

Referensvärdena är angivna i bland annat elektrisk- och
magnetisk-fältstyrka, vilka är mätbara storheter. De är också de
värden som en amatörradiostation inte bör överskrida i områden där
allmänheten kan vistas.

Vid frekvenser som är nära kroppens egen resonansfrekvens absorberas
effekten lättast och maximal uppvärmning uppstår. Hos vuxna ligger
den frekvensen på ungefär 35~MHz om personen är jordad och vid
ungefär 70~MHz om personen är isolerad från jord. Även de olika
kroppsdelarna kan vara resonanta. T.ex. är en vuxens huvud resonant
vid ca 400~MHz medan ett mindre barns huvud är resonant vid ca 700~MHz.

Kroppens storlek avgör alltså vid vilken frekvens den absorberar mest
effekt och vid frekvenser över och under resonansfrekvensen så minskar
uppvärmningen från fältet.

Referensvärdena tar hänsyn till detta faktum och det mest restriktiva
frekvensområdet ligger inom 10 till 400~MHz vilket är där effekten
absorberas lättast av kroppen.

\begin{figure*}[h]
\begin{center}
\includegraphics[width=14cm]{images/emfbild-000}
\caption{Referensvärden för allmänhetens exponering för elektromagnetiska fält (0~Hz -- 300~GHz)}
\label{fig:emf1}
\end{center}
\end{figure*}

Bild \ref{fig:emf1} illustrerar referensvärden för allmänhetens
exponering för elektriska fält (0~Hz -- 300~GHz), med amatörband
och fältstyrkenivå angivna, t.ex. 10,15~MHz har en högsta tillåtna
elektriskt fältstyrka på 28~V/m.

\begin{figure*}[h]
\begin{center}
\includegraphics[width=14cm]{images/emfbild-001}
\caption{Referensvärden för allmänhetens exponering för elektromagnetiska fält (0~Hz -- 300~GHz)}
\label{fig:emf2}
\end{center}
\end{figure*}

Bild \ref{fig:emf2} illustrerar referensvärden för allmänhetens
exponering för magnetiskt fält (0~Hz -- 300~GHz), med amatörband
och fältstyrkenivå angivna, t.ex. 10,15~MHz har en högsta tillåtna
magnetisk fältstyrka på 73~mA/m.

\section{UTVÄRDERING AV EMF}

Hur är det då med den egna stationen, överensstämmer fältstyrkorna som
den kan generera med referensvärdena?

För att kunna utvärdera detta måste man känna till vilka parametrar
som är avgörande vid en beräkning av fältstyrkan. Mycket beror på
vilken antenn man använder och hur den är placerad.
Man måste även förstå egenskaperna signalen från sändaren har.

\subsection{Antennen}

Antennen tar emot signalen från sändaren via en matningskabel och
omvandlar denna signal till en elektromagnetisk våg. Hur denna
omvandling går till är komplicerat men kan enklast förklaras med
begreppet förstärkning eller antennvinst. Man måste alltså känna
till vilken förstärkning antennen har. Formeln som används för
uträkning av fältstyrkan förutsätter att man benämner antennens
förstärkning relativt en isotrop antenn (dBi) och inte en
dipolantenn (dBd), samt att man räknar med linjära värden (gånger)
och inte logaritmiska (dB).

\begin{tabular}{|l|ccccccccccc|}
\hline
dB     &  0  &  1  &  2  &  3  &  4  &  5  &  6  &  7  &  8  &  9  & \\ \hline
Gånger & 1,0 & 1,3 & 1,6 & 2,0 & 2,5 & 3,2 & 4,0 & 5,0 & 6,3 & 7,9 & \\ \hline
dB     &  10  &  11  &  12  &  13  &  14  &  15  &  16  &  17  &  18  &  19  &  20 \\ \hline
Gånger & 10,0 & 12,6 & 15,8 & 20,0 & 25,1 & 31,6 & 39,8 & 50,1 & 63,1 & 79,4 & 100,0 \\ \hline
\end{tabular}

För en antenn med förstärkningen 7~dBi ska alltså värdet 5,0 användas.

\textbf{G = Antennens förstärkning i linjära termer}

\subsection{Sändarsignalen}

Då referensvärdet enligt de allmänna råden är definierade som
medelvärden under en sexminutersperiod är det medeleffekten som ska
användas vid beräkningarna.

Effektens medelvärde under en sexminutersperiod beror på två olika saker.

\subsubsection{Modulationsfaktor:}

Beroende på trafiksätt så blir medeleffekten olika. Används FM så medför
det modulationssättet att man använder max uteffekt kontinuerligt
jämfört med SSB där medeleffekten beror på hur man talar.

Nedanstående tabell är de värden som regelverket i USA använder för
att räkna ut medeleffekten på grund av moduleringen.

\begin{tabular}{|l|c|}
\hline
Trafiksätt & Modulationsfaktor \\ \hline
SSB & 0,2 \\ \hline
CW & 0,4 \\ \hline
SSB med processing & 0,5 \\ \hline
FM & 1,0 \\ \hline
MGM (t.ex. RTTY,PSK) & 1,0 \\ \hline
Bärvåg & 1,0 \\ \hline
\end{tabular}

\subsubsection{Intermittensfaktor:}

Vid vanlig amatörradioanvändning sänder man inte kontinuerligt utan
både lyssnar och sänder växelvis. Sänder man och tar emot lika mycket
under en sexminutersperiod så blir faktorn 0,5 men om man lyssnar
mycket mer och sänder sällan blir faktorn mindre.

\begin{tabular}{|c|c|c|}
\hline
Sändning  & Mottagning & Intermittensfaktor \\
(minuter) & (minuter)  & \\ \hline
1 & 5 & 0,17 \\ \hline
2 & 4 & 0,33 \\ \hline
3 & 3 & 0,50 \\ \hline
4 & 2 & 0,67 \\ \hline
5 & 1 & 0,83 \\ \hline
6 & 0 & 1,00 \\ \hline
\end{tabular}

\subsubsection{Medeleffekt:}

För att räkna ut vilken medeleffekt som används ska man ta hänsyn
till både modulationsfaktorn och intermittensfaktorn enligt följande

\(Medeleffekte = Maxeffekten \cdot Modulationsfaktorn \cdot Intermittensfaktor\)

\textbf{P = Medeleffekten under en sexminutersperiod}

\subsection{Kabeldämpning}

När uteffekten mäts vid sändaren och fältet genereras av effekten som
når antennen måste även den dämpning som matarledaren har vara känd.
Annars överskattas den genererade fältstyrkan.

Även här måste linjära enheter användas (gånger).

\begin{tabular}{|l|c|c|c|c|c|c|c|c|c|c|c|}
\hline
dB & 0,0  & 0,5  & 1,0  & 1,5  & 2,0  & 2,5  & 3,0  & 3,5  & 4,0  & 4,5  & 5,0 \\ \hline
k  & 1,00 & 0,89 & 0,79 & 0,71 & 0,63 & 0,56 & 0,50 & 0,45 & 0,40 & 0,35 & 0,32 \\ \hline
\end{tabular}

För en kabel med dämpningen 2,5~dB ska alltså värdet 0,56 användas.

\textbf{k = Matarkabels dämpning i linjära termer}

\subsection{Avstånd}

På vilket avstånd är det intressant att veta vilken fältstyrka som genereras?
De allmänna råden definierar detta på följande sätt

\emph{”1.3 Dessa allmänna råd omfattar områden där allmänheten kan vistas under sådana tider att begränsningarna är av betydelse.”}

Ett bra utgångsläge är då att utvärdera området där antennen är placerad och bedöma var allmänheten kan exponeras.

\textbf{d = Avståndet från antennen till platsen där fältstyrkan ska bestämmas}

\subsection{Beräkning}

Beräkning av det elektromagnetiska fältet kan med enkelhet bara
genomföras i fjärrfältet från en antenn. I fjärrfältet vet vi sedan
tidigare att man antigen kan utvärdera det elektriska eller
magnetiska fältet. Av denna anledning beskrivs enbart beräkning av
det elektriska fältets del av det elektromagnetiska fältet.

Ett vedertaget avstånd från antennen där fjärrfältsberäkningar kan
genomföras är \(\lambda/6\) för enklare antenner (sådana med låg
antennvinst). Följande formler gäller enbart för beräkning av korrekt
fältstyrka i fjärrfältet men kan för enklare antenner approximera
(eller uppskatta) den fältstyrka som uppträder i närfältet.

Korrekt analys av fältstyrkan i närfältet av mer komplicerade antenner
(t.ex. små loopar) kräver mer komplicerade beräkningar eller mätningar.

\begin{tabular}{|l|c|c|c|c|c|c|c|c|c|c|}
\hline
Band [m] & 160 & 80 & 40 & 30 & 20 & 17 & 15 & 12 & 10 & 6 \\ \hline
Fjärrfältsgräns [m] & 27 & 13 & 6,7 & 5 & 3,3 & 2,8 & 2,5 & 2 & 1,7 & 1 \\ \hline
\end{tabular}

\textbf{E = Det elektromagnetiska fältets storlek i fjärrfältet}

Det elektromagnetiska fältets storlek (i fjärrfältet) räknas ut från
effekten (medelvärde), antennförstärkningen, matarledningens dämpning
och avståndet enligt följande förenklade formel.

\(E=\frac{\sqrt{30 \cdot P \cdot G \cdot k}}{d}\)

Genom enkel matematik kan man då använda samma formel för att räkna
ut på vilket avstånd man genererar en viss fältstyrka.

\(d=\frac{\sqrt{30 \cdot P \cdot G \cdot k}}{E}\)

Denna beräkning är enbart relevant för loben. Fältet under antennen
beräknas inte, och därför är detta inte relevant för att bedöma höjd
på antenntorn eller säkerhetsavstånd till antenntorn. Använd mjukvara för
att få bra bedömning på hur en antenn beter sig, särskilt med avseende på
antenner med riktverkan.

\subsubsection{Exempel 1:}

Vilken medelfältstyrka genererar man på ett visst avstånd från antennen?

En riktantenn för 144~MHz med förstärkning enligt databladet på
14,92~dBi (31 gånger).
Max uteffekt är 1000~W och trafiksättet är MGM (t.ex. RTTY, PSK) med
30 sekunders intervaller.
Den valda matarledningen har en dämpning på 2,5~dB (0,56~gånger).
Avståndet från antennen till beräkningspunkten är 15~m.

\(G = 31\)

\(P = W_{pep} \cdot modulationsfaktor \cdot intermittensfaktorn
= 1000 \cdot 1 \cdot 0,5 = 500\)

\(k = 0,56\)
\(d = 15\)

\(E = \frac{\sqrt{30 \cdot P \cdot G \cdot k}}{d}
= \frac{\sqrt{30 \cdot 500 \cdot 31 \cdot 0,56}}{15}
= 34,02\ V/m\)

Då referensvärdet på denna frekvens är 28~V/m, överskrider
amatörradiosändningen referensvärdet på detta avstånd.

\subsubsection{Exempel 2:}

På vilket avstånd från antennen når man referensvärdet?

En riktantenn för 144~MHz med förstärkning enligt databladet på
14,92~dBi (31 gånger).
Max uteffekt är 1000~W och trafiksättet är MGM (t.ex. RTTY, PSK) med
30~sekunders intervaller.
Den valda matarledningen har en dämpning på 2,5~dB (0,56~gånger).
Referensvärdet för 144~MHz är 28~V/m.

\(G = 31\)

\(P = W_{pep} \cdot modulationsfaktor \cdot intermittensfaktor
= 1000 \cdot 1 \cdot 0,5 = 500\)

\(k = 0,56\)

\(E = 28\)

\(d = \frac{\sqrt{30 \cdot P \cdot G \cdot k}}{E}
= \frac{\sqrt{30 \cdot 500 \cdot 31 \cdot 0,56}}{28}
= 18,22\ m\)

För att följa de allmänna råden bör allmänheten inte ha tillträde framför
huvudloben närmare än 19~m från denna antenn då sändning utförs
enligt exemplet.

\subsubsection{Exempel 3:}

På vilket avstånd från antennen når man referensvärdet?

En dipolantenn för 3,75~MHz utan förstärkning innebär 2,15~dBi (ca 1,6 gånger).
Max uteffekt är 100~W och trafiksättet är SSB med normala TX/RX intervaller.
Den valda matarledningen har en dämpning på 0,5~dB (0,89 gånger).
Referensvärdet för 3,75~MHz är 45~V/m.

\(G = 1,6\)

\(P = W_{pep} \cdot modulationsfaktor \cdot intermittensfaktor
= 100 \cdot 0,5 \cdot 0,5 = 25\)

\(k = 0,89\)

\(E = 45\)

\(d = \frac{\sqrt{30 \cdot P \cdot G \cdot k}}{E} = \frac{\sqrt{30 \cdot 25 \cdot 1,6 \cdot 0,89}}{45}
= 0,74\ m\)

Här konstaterar vi att det uträknade avståndet ligger i närfältet från
antennen (inom 13~meter). En dipol är en enklare antenntyp så antar vi
att värdet är tillräckligt nära för att kunna utvärdera exponeringen.

För att följa de allmänna råden bör människor inte ha tillträde till
nån del av antennen närmare än 0,74~m då sändning utförs enligt exemplet.

\section{EGENKONTROLL}

För att utvärdera sin egen station så finns det några olika vägar att gå.

\begin{itemize}
\item Räkna ut fältstyrkan eller säkerhetsavståndet med sina egna
parametrar enligt exemplen ovan.
\item Använda mjukvara som är speciellt gjort för att räkna ut på
vilket avstånd referensvärdet nås under givna förutsättningar enligt
exempel 2 ovan.
\item Använda värden från tabeller där olika typiska antenner är beskrivna.
\item Använda antennsimuleringsprogram som har möjlighet att även
beräkna fältstyrka.
\item Mäta fältstyrkan (speciellt då man utvärderar i närfältet från antennen)
\end{itemize}

Man bör då tänka på vilket avstånd man har till platser där allmänheten 
har tillträde, sin effektanvändning, vilka antenntyper och vilka
trafiksätt man använder.

\subsection{Räkna manuellt}

Enligt exemplen ovan är det ganska enkelt att göra en uppskattning av
de fältstyrkor som genereras av sin egen amatörradioanvändning.

\subsection{Räkna med specialprogram}

Istället för att själv använda miniräknaren kan man använda program
som är speciellt framtagna för detta ändamål.

Ett exempel på ett sånt program är \textbf{IcnirpCalc} som är framtaget av en
representant från den tyska amatörradioföreningen (DARC). I programmet
finns redan olika antenntyper och det finns även möjlighet att lägga
in egna antenner för att göra korrekta beräkningar.
Detta program finns att ladda ner från SSA web plats för EMC/EMF frågor.

\subsection{Tabellvärden}

Utifrån den typ av antenn man själv använder kan man jämföra med
typiska värden från andras beräkningar och göra en hyfsad uppskattning
av sig egen situation.

\subsection{Antennsimulering}

Vissa program för antennsimulering har även funktioner för att beräkna
fältstyrkenivåer runt antennen och kan i vissa fall beräkna fältstyrkan
även i närfältet.

\subsection{Mäta fältstyrka}

Att istället för att räkna ut vilken fältstyrka man genererar låter
det enkelt att mäta upp den med ett enkelt mätinstrument. Problemet
med detta är att det är svårt att få tillgång till mätutrustning som är
kalibrerad och noggrant nog att korrekt kunna bestämma fältstyrkenivån.

\section{SAMMANFATTNING}

Enligt ett nu gällande EU direktiv som är omsatt av
Strålsäkerhetsmyndigheten (SSM) till svenska allmänna råd finns det
referensvärden som gäller allmänhetens exponering av elektromagnetiska
fält (EMF).

Dessa nivåer och sändaramatörens befogenheter att generera
elektromagnetiska fält innebär att vi som sändaramatörer måste förstå
och kunna hantera området elektromagnetiska fält (EMF).

Alla sändande antenner kommer att ha ett elektromagnetiskt fält (EMF)
runt sig. Detta elektromagnetiska fält (EMF) är beroende på vilken
typ av antenn som används och den signal man skickar in i antennen.
Hur man bedömer storleken på dessa fält är avgörande för att kunna
begränsa exponeringen från sin amatörradiostation.

En egenkontroll bör genomföras för att kunna bedöma den fältbild som
amatörradioutövandet orsakar runt sin station. Eftersom amatörradio
är en experimentell verksamhet så måste alla förstå hur olika
förändringar i sin installation och användning påverkar denna fältbild.

Vilken metod man än väljer för sin egenkontroll är det lämpligt att
göra den tydligt och lättförståelig. Detta är viktigt eftersom man
bör spara sina resultat och då ha möjlighet att göra om sin utvärdering
när man har förändrat något eller några av de värden som skulle kunna
påverka resultatet.

\subsection{Praktisk hantering}

Vid all användning av amatörradioutrustning måste man göra en bedömning
av vilka fältstyrkor man genererar och vilka som blir exponerade. Det
kan vara frågan om människor i sin omedelbara närhet eller människor
på längre håll. I alla fall bör man fundera på om man väljer rätt sätt
att generera den fältstyrka som man behöver för att kommunicera eller
om det finns ett bättre sätt som innebär att fältstyrkan effektivast
möjligt når motstationen och inte exponerar nån annan.

Det finns alltså vissa installationer som bör undvikas eller förordas
för att hålla nivåerna på exponering så låga som möjligt.

\begin{itemize}

\item Antenner som sitter nära människor, exempelvis balkongantenner,
kan ge mycket högre exponering än antenner som sitter högt monterade i en mast.

\item Riktantenner för höga frekvenser har ofta hög förstärkning, och
kan ge höga fältstyrkor i huvudriktningen. Då måste man se till att
det inte är möjligt att rikta denna typ av antenn mot platser där
människor kan exponeras.

\item Inomhusantenner hamnar alltid nära människor och bör enbart
användas med låg effekt då de kan ge mycket hög exponering. De kommer
också ta emot störningar från hemelektronik (nätadaptrar, datorer etc.)
vilket också gör antennplaceringen mycket olämplig.

\item Antenner ovanför huskroppar bör endast användas med låg effekt.
Tråd-antenner för lägre frekvenser rakt ovanför bostadshus kommer att
vara nära människor i byggnaden.

\item Om man har behov av att använda hög effekt så måste man också se
till att effekten används så bra som möjligt. Det är direkt olämpligt
att kompensera en dålig antenn med högre effekt då det oftast
resulterar i höga fältstyrkor på fel ställe.

\item Högre fältstyrka kan för det mesta enklast åstadkommas med en
antenn som riktar signalen i den riktning man vill kommunicera. Det är
oftast mycket dyrare och mer komplicerat att öka uteffekten för att nå
samma resultat.

\item Osymmetriska antenner kan ge mantelströmmar i matningsledningen.
Det innebär att en gemensam överföring (common-mode) HF-ström flyter från
antennen tillbaka på matarledningen och kan ge höga fältstyrkor längs hela
kabellängden. Bättre är det då att använda symmetriska antenner, exempelvis en
mittmatad halvvågsdipol. En ström-balun (även common mode choke, RF-choke)
där antennen ansluts till matarledningen undertrycker denna gemensamma
ström och därmed kommer matningsledningen sluta att agera radierande
element, varvid fältstyrkorna sjunker signifikant.

\item Vissa antenner, så som T-antenn, använder dock obalansen för att
matningsledningen agerar radierande element. I dessa fall ska den delen
av matningsledningen som agerar radierande element betraktas som sådant även
i EMF-sammanhang och säkerhetsavstånd ska iakttas. Det är rekommenderat att
använda ström-balun för att isolera antennen från radio-stationen med
avseende på gemensam ström.

\item Även symmetriska antenner kan ha strömmar på utsidan av
matarledningen. Dra därför matarledningen så långt bort från människor
som möjligt.

\item Använd inte effektförstärkare eller antennavstämningsenhet utan
hölje då fältstyrkorna runt utrustningen kan nå höga nivåer.

\item Vid antennplaceringar nära människor så kan det bli omöjligt att
använda hög effekt.
\end{itemize}

Det finns som synes många sätt att göra rätt men också många sätt att
göra fel när det gäller att hantera den fältstyrka vi vill generera
för att upprätthålla radiokommunikation. Innan man börjar sin
amatörradiosändning är det viktigt att ha förståelse för de fält
som genereras och kunna begränsa dem där så behövs.

Detta ämne är mycket komplicerat och omfattande då det krävs hög
utbildning inom främst matematik och antennteori för att förstå alla
detaljer. Detta innebär att det finns stort utrymme för
vidareutbildning inom området om man vill fördjupa sig mer.

\hilight{TODO: Flytta referenser till BibTex}

Referenser
1999/519/EG, Rådets rekommendation av den 12~juli 1999 om begränsning av allmänhetens exponering för
elektromagnetiska fält (0 Hz--300 GHz)
SSMFS 2008:18, Strålsäkerhetsmyndighetens allmänna råd om begränsning av allmänhetens exponering för
elektromagnetiska fält
OET Bulletin 65 Supplement B, Evaluating Compliance with FCC Guidelines for Human exposure to
Radiofrequency Electromagnetic Fields, Additional Information for Amateur Radio Stations

% Avsnitt 12.1 Fält
% Avsnitt 12.2 Allmänna råd
% Avsnitt 12.3 Utvärdering av EMF
\section{Fält}
\index{elektriskt fält (E)}
\index{magnetiskt fält (H)}
För att ange nivån på det elektriska fältet (E) används enheten
''volt per meter'' (V/m).
Det magnetiska fältet (H) nivå anges i enheten ''ampere per meter'' (A/m).

Antennens uppgift är att så effektivt som möjligt omvandla den högfrekventa
strömmen i matarkabeln till en elektromagnetisk våg som utbreder sig i luften.

Den sammansatta elektromagnetiska vågen uppträder inte direkt vid antennen utan
uppstår i det som man kallar fjärrfältet.
Detta sker genom växelverkan mellan de elektriska och magnetiska fält som
utgår från antennen.
Teorierna som beskriver hur denna växelverkan fungerar är komplicerade
men det viktiga att förstå är att det finns en gräns mellan vad man
kallar fjärrfältet, längre bort från antennen och närfältet nära antennen.

\index{fjärrfält}
I fjärrfältet kan man tack vare växelverkan mellan det elektriska- och det
magnetiska fältet mäta vilket som helst av dem.
I och med att det elektromagnetiska fältet sprider ut sig över en större yta så
avtar styrkan i fältet med avståndet från antennen.
Det sammansatta elektromagnetiska fältet som passerat gränsen till fjärrfältet
avtar linjärt med avståndet, dubbleras avståndet halveras fältstyrkan.
Det spelar ingen roll om antennen är helt rundstrålande eller koncentrerar
effekten i en riktning, det elektromagnetiska fältet avtar på samma sätt.

\index{närfält}
I närfältet behöver man på grund av fältens komplicerade inbördes förhållande
mäta både det elektriska och det magnetiska fältet för att få en uppfattning
om storleken på det radiofrekventa fältet.
I antennens närhet varierar nivåerna på de olika fälten kraftigt och på vissa
punkter kan höga fältstyrkenivåer mätas upp.

Om antennen har stor utsträckning i förhållande till använd våglängd kan ibland
fjärrfältsformler användas för att överslagsmässigt beräkna fältstyrkenivå i
antennens närfält.
För kompakta antenner (t.ex. små loopar) krävs komplicerade beräkningar
med hjälp av antennsimuleringsprogram.

Beroende på den antenntyp som genererar fältet är det antingen ett elektriskt
eller magnetiskt fält som dominerar i närfältet.
Elektrisk fältdominans genereras av antenntyper som bygger på
spänningsskillnader (t.ex. dipol) och magnetisk fältdominans av antenner
med strömflöde (t.ex. små loopar).

Eftersom alla elektriska ledare kan betraktas som antenner kommer dessa att
kunna generera fält, oavsett om det är tänkt att det ska vara en antenn eller
inte.
Man bör ha detta i åtanke vid installation av matarledning till antennen för
att undvika att högfrekvent ström flyter tillbaka till stationen på utsidan av
ledningen.
Även de apparater man använder för att generera radiosignaler kan ha dålig
skärmning och därigenom leds högfrekvent ström till apparaternas utsida.

Det finns alltså en risk att fältstyrkorna kan vara betydande i närheten av
sändare och framför allt vid slutsteg med tillhörande kablage.

\section{Allmänna råd}
\index{EMF!allmänna~råd}

SSM har gett ut allmänna råd för begränsning av allmänhetens exponering
för elektromagnetiska fält SSMFS~2008:18~\cite{SSMFS2008:18}.
Syftet med råden är att skydda allmänheten från akuta
skadliga biologiska effekter vid exponering för elektromagnetiska fält.
I råden anges grundläggande begränsningar och härledda referensvärden.

\begin{quote}
	De grundläggande begränsningarna säkerställer att elektriska eller
	magnetiska fenomen som kan uppträda i kroppen inte stör funktioner i
	nervsystemet eller ger upphov till skadlig värmeutveckling.
\end{quote}

De grundläggande begränsningarna är, enligt internationella rekommendationer,
satta vid ungefär två procent av de nivåer vid vilka akuta biologiska effekter
är vetenskapligt säkerställda.

Från de grundläggande begränsningarna har härletts referensvärden som utgörs
av storheter som är mätbara utanför människokroppen.
Referensvärdena ska säkerställa att de grundläggande begränsningarna inte
överskrids.

\begin{quote}
	Om uppmätta värden överstiger referensvärdena, innebär detta inte nödvändigtvis
	att de grundläggande begränsningarna överskrids. I sådana fall gäller enligt
	dessa allmänna råd de grundläggande begränsningarna.
\end{quote}

I EU-direktivet 1999/519/EG~\cite{1999/519/EG} skrivs att i sådana fall skall det
göras en bedömning huruvida exponeringsnivån ligger under den grundläggande
begränsningen.

Referensvärdena i de allmänna råden bör inte överskridas i något område där
allmänheten kan vistas under sådana tider att begränsningarna är av betydelse.

\index{EMF!akuta biologiska effekter}
Det finns två huvudsakliga akuta biologiska effekter som kan förekomma vid
kraftig exponering för elektromagnetiska fält.
Fält med frekvens upp till cirka \qty{10}{\mega\hertz} kan om strömtätheten blir
hög i kroppen påverka det centrala nervsystemet.
Fält med frekvenser från \qty{100}{\kilo\hertz} till \qty{10}{\giga\hertz} kan
vid höga nivåer leda till en uppvärmning av kroppen.

\index{Specific Absorption Rate (SAR)}
\index{SAR}
När elektromagnetisk strålning absorberas i biologisk vävnad kan vävnaden värmas
upp.
Detta benämns ''Specific Absorption Rate'' (SAR) som mäts i enheten watt per
kilogram (\unit{\watt\per\kilo\gram}) eller milliwatt per gram
(\unit{\milli\watt\per\gram}).
SAR definieras som den energi, medelvärdesbildad över hela kroppen eller delar
av kroppen som absorberas per tidsenhet och per massenhet biologisk vävnad.

Då uppvärmningen av kroppsvävnad inte går snabbt räknar man med den medeleffekt
som under en viss tid orsakar uppvärmning.
För frekvenser mellan \qty{100}{\kilo\hertz} och \qty{10}{\giga\hertz} beräknas
SAR-värdet som medelvärdet under en sexminutersperiod.
För beräkning av SAR-värde på frekvenser överstigande \qty{10}{\giga\hertz}
hänvisas till formler för beräkning enligt SSMFS 2008:18.

Beroende på kroppens storlek i förhållande till det elektromagnetiska fältets
riktning och våglängd skapas resonansfenomen på grund av att kroppen fungerar
som en antenn.
Detta påverkar uppvärmningen på så sätt att vid frekvenser som är nära kroppens
eller kroppsdelens elektriska resonansfrekvens absorberas effekten lättare och
maximal uppvärmning uppstår.
Hos vuxna ligger denna resonansfrekvens mellan 70 och \qty{90}{\mega\hertz} om
personen står upp är och isolerad från något som kan jämföras med ett jordplan.
Även de olika kroppsdelarna kan vara resonanta.
En vuxen persons huvud är till exempel resonant vid cirka \qty{400}{\mega\hertz}.

Kroppens storlek avgör alltså vid vilken frekvens den absorberar mest effekt och
vid frekvenser över och under resonansfrekvensen så minskar uppvärmningen
orsakad av det elektromagnetiska fältet.

\index{EMF!referensvärden}
Referensvärdena tar hänsyn till detta faktum och det mest restriktiva
frekvensområdet ligger inom området 10 till \qty{400}{\mega\hertz} där effekt
lättast absorberas av kroppen.

I frekvensområdet 10 till \qty{110}{\mega\hertz} finns även en begränsning till
\qty{45}{\milli\ampere} för inducerad ström i varje extremitet i syfte att
begränsa det lokala SAR-värdet.

\mediumfig[0.87]{images/emfbild-000}{Referensvärden för begränsning av elektriska fält (100~kHz--10~GHz)}{fig:emf1}
\mediumfig[0.87]{images/emfbild-001}{Referensvärden för begränsning av magnetiska fält (100~kHz--10~GHz)}{fig:emf2}

Bild~\ssaref{fig:emf1} illustrerar referensvärden för begränsning av elektriska
fält på platser där allmänheten kan vistas (100~kHz--10~GHz), med amatörband
och fältstyrkenivå angivna, till exempel \qty{10,15}{\mega\hertz} har en högsta
tillåtna elektriskt fältstyrka på \qty{28}{\volt\per\metre}.

Bild~\ssaref{fig:emf2} illustrerar referensvärden för begränsning av magnetiska
fält på platser där allmänheten kan vistas (100~kHz--10~GHz), med amatörband
och fältstyrkenivå angivna, till exempel \qty{10,15}{\mega\hertz} har en högsta
tillåtna magnetisk fältstyrka på \qty{73}{\milli\ampere\per\metre}.

\clearpage
\section{Utvärdering av EMF}
\index{EMF!utvärdering}

För att kunna utvärdera att den egna radiostationen vid sändning ger
elektromagnetiska fält som understiger referensvärdena behöver man känna till
de parametrar som är avgörande för styrkan på det elektromagnetiska fältet:

\begin{itemize}
  \item Antennens förstärkning (G).
  \item Sändningens medeleffekt (P).
  \item Transmissionsledningens förluster (k).
  \item Distansen (d).
\end{itemize}

\subsection{Antennen}
Antennen tar emot signalen från sändaren via en matningskabel och
omvandlar denna signal till en elektromagnetisk våg.
Hur effektivt antennen omvandlar signalen från sändaren kan enklast förklaras
med begreppen förstärkning eller antennvinst.

Man måste alltså känna till vilken förstärkning antennen har uttryckt i linjära
faktorer i förhållande till en isotrop antenn.

Antennförstärkning i förhållande till en isotrop antenn uttrycks vanligen i dBi.
Detta medför att en vanlig dipolantenn som används som referens för 0\,dBd har
en förstärkning på 2,15\,dBi jämfört med en isotrop antenn.

Alla värden på antennförstärkning uttryckt i dBd ska därför ökas med 2,15 för
att kunna användas i tabell~\ssaref{tab:forst} som visar förhållandet mellan
förstärkning i \unit{\decibel} och linjära faktorer.

\begin{table*}[ht]
  \begin{center}
    \begin{tabular}{|l|ccccccccccc|}
    \hline
    dB     &  0  &  1  &  2 & 2,15 &  3  &  4  &  5  &  6  &  7  &  8  &  9  \\ \hline
    G & 1,0 & 1,3 & 1,6 & 1,64 & 2,0 & 2,5 & 3,2 & 4,0 & 5,0 & 6,3 & 7,9 \\ \hline\hline
    dB     &  10  &  11  &  12  &  13  &  14  &  15  &  16  &  17  &  18  &  19  &  20 \\ \hline
    G & 10,0 & 12,6 & 15,8 & 20,0 & 25,1 & 31,6 & 39,8 & 50,1 & 63,1 & 79,4 & 100,0 \\ \hline
    \end{tabular}
    \caption{G = Antennens förstärkning i linjära faktorer}
    \label{tab:forst}
  \end{center}
\end{table*}

För en antenn med förstärkningen 7\,dBi ska alltså värdet 5,0 användas.

\subsection{Sändareffekten}
Alla SAR-värden ska beräknas som ett medelvärde under en period av sex minuter.
För att kunna utföra en beräkning av effektens medelvärde behövs utöver
PEP-effekt kännedom om de två faktorer som påverkar medeleffekten.
Faktorerna har därför betydelse för nivån på det elektromagnetiska fältet och
påverkar därigenom den genomsnittliga exponeringen för EMF.

\subsubsection{Modulationsfaktor}
\index{EMF!modulationsfaktor}
\index{modulationsfaktor}

Beroende på trafiksätt så blir medeleffekten olika.
Används FM så medför det modulationssättet att man använder max uteffekt
kontinuerligt jämfört med SSB där medeleffekten beror på hur man talar.

Tabell~\ssaref{tab:modfakt} ger de faktorer som enligt OET bulletin 65 supplement b,
\cite{OETbul65b} används i USA för att räkna ut medeleffekten på grund
av modulationsfaktorn.

\begin{table}[H]
  \begin{center}
    \begin{tabular}{lc}
	\textbf{Trafiksätt} & \textbf{Modulationsfaktor} \\ 
	\hline
	\emph{SSB} & 0,2 \\ 
	\emph{CW} & 0,4 \\ 
	\emph{SSB med processing} & 0,5 \\ 
	\emph{FM} & 1,0 \\ 
	\emph{MGM (t.ex. RTTY,PSK)} & 1,0 \\ 
	\emph{Bärvåg} & 1,0 \\ 
    \end{tabular}
    \caption{Modulationsfaktor per trafiksätt}
    \label{tab:modfakt}
  \end{center}
\end{table}

\subsubsection{Intermittensfaktor}
\index{EMF!intermittensfaktor}
\index{intermittensfaktor}

Vid vanlig amatörradioanvändning sänder man inte kontinuerligt då växling
mellan sändning och lyssning sker regelbundet.
Sänder man och tar emot lika mycket under en sexminutersperiod så blir faktorn
0,5 men om man lyssnar mycket mer och sänder sällan blir faktorn mindre.
Se tabell~\ssaref{tab:intfakt} för fler exempel.

\begin{table}[H]
  \begin{center}
    \begin{tabular}{|c|c|c|}
	\hline
	Sändning  & Mottagning & Intermittensfaktor \\
	(minuter) & (minuter)  & \\ \hline
	1 & 5 & 0,17 \\ \hline
	2 & 4 & 0,33 \\ \hline
	3 & 3 & 0,50 \\ \hline
	4 & 2 & 0,67 \\ \hline
	5 & 1 & 0,83 \\ \hline
	6 & 0 & 1,00 \\ \hline
    \end{tabular}
    \caption{Intermittensfaktor}
    \label{tab:intfakt}
  \end{center}
\end{table}

\subsubsection{Medeleffekt}
\index{EMF!medeleffekt}

För att räkna ut vilken medeleffekt som används ska man ta hänsyn
till både modulationsfaktor och intermittensfaktor enligt följande

\(\textit{Medeleffekt} = \textit{Maxeffekten} \cdot \textit{Modulationsfaktor} \cdot \textit{Intermittensfaktor}\)

\noindent\textbf{P = Medeleffekten under en sexminutersperiod}

\subsection{Kabeldämpning}
\index{EMF!kabeldämpning}

När uteffekten mäts vid sändaren och fältet genereras av effekten som
når antennen måste även den dämpning som matarledaren har vara känd.
Annars överskattas den genererade fältstyrkan.

Även här måste linjära faktorer användas.
Förlusterna i en kabel har negativa värden uttryckt i \unit{\decibel} vilket
medför att faktorerna i tabell~\ssaref{tab:feedannut} blir mindre än ett.

\begin{table*}[ht]
  \begin{center}
    \begin{tabular}{|l|c|c|c|c|c|c|c|c|c|c|c|}
	\hline
	dB & 0,0  & 0,5  & 1,0  & 1,5  & 2,0  & 2,5  & 3,0  & 3,5  & 4,0  & 4,5  & 5,0 \\ \hline
	k  & 1,00 & 0,89 & 0,79 & 0,71 & 0,63 & 0,56 & 0,50 & 0,45 & 0,40 & 0,35 & 0,32 \\ \hline
    \end{tabular}
    \caption{k = Matarkabels dämpning i linjära termer}
    \label{tab:feedannut}
  \end{center}
\end{table*}

För en kabel med dämpningen \qty{2,5}{\decibel} ska alltså värdet 0,56 användas.

\subsection{Distans}
\index{EMF!distans}

För att kunna beräkna nivån på det elektromagnetiska fältet på en utvald plats
behöver man veta distansen till den sändande antennen.

Enligt Strålsäkerhetsmyndighetens allmänna råd så bör inte referensvärdena
överskridas på platser där allmänheten vistas.
En bedömning bör därför göras över distanserna från den sändande antennen till
platser allmänheten riskerar att exponeras för elektromagnetiska fält.
\\[1ex] % layout
\noindent\textbf{d = Distansen från antennen till platsen där fältstyrkan ska bestämmas}

\subsection{Beräkning}
\index{EMF!beräkning}

Beräkning av det elektromagnetiska fältet kan med enkelhet bara
genomföras i fjärrfältet från en antenn.
I fjärrfältet vet vi sedan tidigare att man antingen kan utvärdera det
elektriska eller det magnetiska fältet.
Av denna anledning beskrivs här enbart beräkning av det elektriska fältets del av
det elektromagnetiska fältet.
Ett vedertaget avstånd från antennen där fjärrfältsberäkningar kan genomföras
är
%% \(d=\dfrac{\lambda}{6}\).
\(d=\lambda / 6\). Se tabell~\ssaref{tab:fjfltgr}.

Följande formler gäller enbart för beräkning av korrekt fältstyrka i
fjärrfältet men kan för enklare antenner användas för att uppskatta den
fältstyrka som uppträder i närfältet.

%% k7per: Where is this table referenced?
\begin{table*}[ht]
  \begin{center}
    \begin{tabular}{|l|c|c|c|c|c|c|c|c|c|c|}
	\hline
	Band [m] & 160 & 80 & 40 & 30 & 20 & 17 & 15 & 12 & 10 & 6 \\ \hline
	Fjärrfältsgräns [m] & 27 & 13 & 6,7 & 5 & 3,3 & 2,8 & 2,5 & 2 & 1,7 & 1 \\ \hline
    \end{tabular}
    \caption{Fjärrfältsgräns per band}
    \label{tab:fjfltgr}
  \end{center}
\end{table*}

\noindent\textbf{E = Det elektromagnetiska fältets storlek i fjärrfältet}

Det elektromagnetiska fältets storlek (i fjärrfältet) räknas ut från
effekten (medelvärde), antennförstärkningen, matarledningens dämpning
och avståndet enligt följande förenklade formel.
%%
\[E=\dfrac{\sqrt{30 \cdot P \cdot G \cdot k}}{d}\]
%%
Genom enkel matematik kan man då använda samma formel för att räkna
ut på vilket avstånd man genererar en viss fältstyrka.
%%
\[d=\dfrac{\sqrt{30 \cdot P \cdot G \cdot k}}{E}\]
%%
Denna beräkning är enbart relevant för huvudloben.
Fältet under antennen beräknas inte, och därför kan resultatet inte användas
för att bedöma höjd på eller säkerhetsavstånd till antenntorn.
Använd datorprogram för att få bra bedömning på hur en antenn beter sig,
särskilt med avseende på antenner med riktverkan.

\begin{exempelbox}
En riktantenn för \qty{144}{\mega\hertz} med förstärkning enligt databladet på
14,92\,dBi (31 gånger).
Max uteffekt är \qty{1000}{\watt} och trafiksättet är MGM (t.ex. RTTY, PSK) med
30~sekunders intervaller.
Den valda matarledningen har en dämpning på \qty{2,5}{\decibel} (0,56~gånger).
Avståndet från antennen till beräkningspunkten är \qty{15}{\metre}.
Vilken medelfältstyrka genererar man på ett visst avstånd från antennen?
\tcblower
\noindent
\[P_{medel} = P_{pep} \cdot k_{mod} \cdot k_{if}
= 1000 \cdot 1 \cdot 0,5 = \qty{500}{\watt}\]
\[k_{mod} = modulationsfaktor\]
\[k_{if} = intermittensfaktor\]
\[G = 31 \quad k = 0,56 \quad d = 15\]
% \[E = \dfrac{\sqrt{30 \cdot P \cdot G \cdot k}}{d}
% = \dfrac{\sqrt{30 \cdot 500 \cdot 31 \cdot 0,56}}{15}
% = \qty{34,02}{\volt\per\metre}\]
\begin{align*}
  E &= \dfrac{\sqrt{30 \cdot P \cdot G \cdot k}}{d} =\\
&= \dfrac{\sqrt{30 \cdot 500 \cdot 31 \cdot 0,56}}{15}
= \qty{34,02}{\volt\per\metre}
\end{align*}

Då referensvärdet på denna frekvens är \qty{28}{\volt\per\metre}, överskrider
amatörradiosändningen referensvärdet på detta avstånd.
\end{exempelbox}

\begin{exempelbox}
En riktantenn för \qty{144}{\mega\hertz} med förstärkning enligt databladet på
14,92\,dBi (31 gånger).
Max uteffekt är \qty{1000}{\watt} och trafiksättet är MGM (t.ex. RTTY, PSK) med
30~sekunders intervaller.
Den valda matarledningen har en dämpning på \qty{2,5}{\decibel} (0,56~gånger).
Referensvärdet för \qty{144}{\mega\hertz} är \qty{28}{\volt\per\metre}.
På vilket avstånd från antennen når man referensvärdet?
\tcblower
\noindent
\[P_{medel} = P_{pep} \cdot k_{mod} \cdot k_{if}
= 1000 \cdot 1 \cdot 0,5 = \qty{500}{\watt}\]
\[k_{mod} = \textit{modulationsfaktor}\]
\[k_{if} = \textit{intermittensfaktor}\]
\[G = 31 \quad k = 0,56 \quad E = 28\]
%% \[d = \dfrac{\sqrt{30 \cdot P \cdot G \cdot k}}{d}
%% = \dfrac{\sqrt{30 \cdot 500 \cdot 31 \cdot 0,56}}{28}
%% = \qty{18,22}{\metre}\]
\begin{align*}
  d &= \dfrac{\sqrt{30 \cdot P \cdot G \cdot k}}{d} =\\
&= \dfrac{\sqrt{30 \cdot 500 \cdot 31 \cdot 0,56}}{28}
  = \qty{18,22}{\metre}
  \end{align*}

För att följa de allmänna råden bör allmänheten inte kunna vistas i huvudloben
framför antennen på ett avstånd mindre än \qty{19}{\metre} då sändning utförs
enligt exemplet.
\end{exempelbox}

\begin{exempelbox}
En dipolantenn för \qty{3,75}{\mega\hertz} har jämfört med en isotrop antenn
förstärkningen 2,15\,dBi (cirka 1,6 gånger).
Max uteffekt är \qty{100}{\watt} och trafiksättet är SSB med normala TX/RX
intervaller.
Den valda matarledningen har en dämpning på \qty{0,5}{\decibel} (0,89 gånger).
Referensvärdet för \qty{3,75}{\mega\hertz} är \qty{45}{\volt\per\metre}.
På vilket avstånd från antennen når man referensvärdet?
\tcblower
\noindent
\[P_{medel} = P_{pep} \cdot k_{mod} \cdot k_{if}
= 100 \cdot 0,5 \cdot 0,5 = \qty{25}{\watt}\]
\[k_{mod} = modulationsfaktor\]
\[k_{if} = intermittensfaktor\]
\[G = 1,6 \quad k = 0,89 \quad E = 45\]
\[d = \dfrac{\sqrt{30 \cdot P \cdot G \cdot k}}{E} = \dfrac{\sqrt{30 \cdot 25 \cdot 1,6 \cdot 0,89}}{45}
= \qty{0,74}{\metre}\]

Här konstaterar vi att det uträknade avståndet ligger i närfältet från antennen
(inom 13~meter).
En dipol är en enklare antenntyp så vi kan anta att värdet är användbart för att
kunna utvärdera exponeringen.

För att följa de allmänna råden bör människor inte ha tillträde till nån del av
antennen närmare än \qty{0,74}{\metre} då sändning utförs.
\end{exempelbox}

% Avsnitt 12.4 Egenkontroll
\input{koncept/emf-egenkontroll}
% Avsnitt 12.5 Sammanfattning
\input{koncept/emf-sammanfattning}
%
%
% Kapitel 13 Elsäkerhet
\chapter{Elsäkerhet}
\label{ch:elsakerhet}
\index{elsäkerhet}

Människokroppen är ett komplicerat elektrokemiskt system, som främst
kontrolleras av hjärnan.
Musklerna styrs av svaga elektriska strömimpulser genom nervsystemet.
Främmande strömmar genom kroppen kan störa kroppsfunktioner och kan i olyckliga
fall göra stor skada.
Styrkan och frekvensen på strömmarna avgör skadans art och omfattning.

% Avsnitt 13.1 Människokroppen
\input{koncept/elsaekerhet--maenniskokroppen}
% Avsnitt 13.2 Allmänna elnätet
\section{Allmänna elnätet}
\harecsection{\harec{a}{10.2}{10.2}}
\label{jordning}
\index{elnätet}

Elektrisk energi levereras till förbrukarna över transformatorstationer där
högspänning först transformeras till lågspänning.
Från transformatorstationerna förgrenas lågspänningsnätet till serviceskåp ute
i kvarter och byar.

I Sverige är fördelningstransformatorns sekundärlindningar oftast sammankopplade
till ett Y (s.k. Y- eller stjärnkoppling) där mittpunkten är jordad.

De i Sverige vanligast förekommande 3-fas lågspänningsnäten har huvudspänningen
\qty{400}{\volt} och fasspänningen \qty{230}{\volt}.
Spänningen mellan fasledarna kallas för huvudspänning och spänningen mellan
respektive fasledare och nolledaren kallas för fasspänning.

Bruksföremålen i huset ansluts oftast 1-fasigt, det vill säga mellan någon av
fasledarna och nolledaren.
Någorlunda lika belastning mellan faserna är önskvärd.
Mer effektkrävande apparater som el-pannor och spisar ansluts därför till alla
tre faserna (3-fasigt).
Amatörradioutrustningar ansluts oftast 1-fasigt.

Nybyggnad, förändring eller reparation av starkströmsanläggning,
fast anslutning av elektrisk utrustning till en starkströmsanläggning
eller att koppla loss fast ansluten elektrisk utrustning från en
starkströmsanläggning, klassas som elinstallationsarbete och får endast
utföras av person som har auktorisation som elinstallatör eller av
yrkesverksam som omfattas av ett elinstallationsföretags egenkontroll.

Om du har \emph{erforderlig kunskap} om elsäkerhet får du

\begin{itemize}
\item byta ut en elkopplare (strömbrytare) för högst \qty{16}{\ampere} \qty{400}{\volt}
\item byta ut ett anslutningsdon (vägguttag, lamputtag, stickpropp,
skarvuttag eller liknande) för högst \qty{16}{\ampere} \qty{400}{\volt}
\item byta ut en ljusarmatur i torrt, icke brandfarligt utrymme i bostäder
\item utföra, ändra eller reparera en starkströmsanläggning som ingår i en
skyddsklenspänningskrets med nominell spänning om högst \qty{50}{\volt} med
effekt om högst \qty{200}{VA} och ström begränsad av säkring på högst \qty{10}{\ampere}
\item byta säkring
\item byta ljuskälla (lampa, lysrör eller liknande)
\item reparera apparater
\item reparera och tillverka apparatkablar och skarvsladdar.
\end{itemize}

\begin{center}
\begin{minipage}{0.19\columnwidth}
\Huge{\warningsymbol}
\end{minipage}
\begin{minipage}{0.7\columnwidth}
\textbf{Kom ihåg, att auktoriserad installatör ska anlitas för arbete
i fasta installationer.}
\end{minipage}
\end{center}

\subsection{Hembyggd elektronik}
\label{hembyggd elektronik}
\label{praktiska rad för sjalvbyggaren}
\index{hembyggd elektronik}
\index{praktiska råd för självbyggaren}

När en elektrisk eller elektronisk apparat konstrueras eller byggs finns det
ett antal punkter som ska uppmärksammas för att apparaten ska vara säker att
använda oavsett hur den är avsedd att strömförsörjas.
Som stöd för hur en apparat kunde byggas för att uppfylla kraven gav
dåvarande SEMKO ut \emph{Praktiska råd för självbyggaren}.
Nedanstående punkter bygger på dessa praktiska råd:

\begin{itemize}
\item Höljet ska vara anpassat till apparaten och inte vara öppningsbart
  utan verktyg.

\item Höljet ska vara försett med nödvändiga ventilationshål för att
  undvika överhettning.
  Observera att spänningsförande delar inte får vara nåbara genom
  ventilationshålen.

\item Höljet får inte bli så varmt att skada kan uppstå på människa
  eller egendom.

\item Är höljet eller chassiet till en elnätsansluten apparat av ledande
  material och apparaten inte har förstärkt isolering så ska \emph{utsatta}
  delar som riskerar att spänningssättas vid fel anslutas till skyddsjord.

\item Kabeln för nätanslutning ska vara försedd med en för ändamålet lämplig
  dragavlastning som även skyddar kabeln mot nötning när den passerar höljet.

\item Komponenter i apparaten ska vara dimensionerade och godkända
  för den effekt de utvecklar och för den spänning och strömstyrka de
  ansluts till.
  \emph{Not: Ett tips är att ha god marginal vad gäller värmetålighet då det
    ger ökad livslängd och större säkerhetsmarginaler.}

\item Apparaten ska vara försedd med korrekt dimensionerad säkring
  som skydd mot kortslutning och överbelastning.

\item Elnätsansluten apparat ska vara försedd med 2-polig nätströmbrytare.

\item Spänningsförande delar i apparaten ska vara försedda med
  beröringsskydd som skyddar mot oavsiktlig beröring.

\item Komponenter i apparaten ska monteras fast och placeras på lämpliga
  inbördes avstånd så att risken för störningar, överslag, kortslutning eller
  överhettning minimeras.

\item Kablar och ledningar för starkström ska skyddas mot varma komponenter,
  nötning och skarpa kanter samt förläggas separerade från ledningar för
  klenspänning och signaler.
\end{itemize}

\begin{center}
\begin{minipage}{0.19\columnwidth}
\Huge{\warningsymbol}
\end{minipage}
\begin{minipage}{0.7\columnwidth}
\textbf{Sträva efter att alltid ansluta din apparat via vägguttag
	skyddade av jordfelsbrytare.}
\end{minipage}
\end{center}

När det gäller hembyggda radiosändare gäller även radioutrustningslagen.
Läs mer om hembyggd radiosändare i avsnitt~\ssaref{hembyggd radiosandare}.


\subsection{Strömbrytare}

Kraftförsörjningen av radiostationens apparater bör ske över en
gemensam huvudströmbrytare, som lätt kan nås.
En indikatorlampa får gärna markera att den brytaren är tillslagen och att
stationen är under spänning.
Informera familjen och övriga i din omgivning om hur den brytaren fungerar.
Det är en säkerhetsåtgärd om något skulle hända.

Apparaternas nätströmbrytare ska vara utförda för den aktuella arbetsspänningen
och ha ett godkänt utförande.

\begin{description}
\item[Vid 1-fassystem] ska nätströmbrytaren i apparaterna vara 2-polig och bryta fas-
och N-ledare, men aldrig PE-ledaren.

\item[Vid 3-fassystem] ska nätströmbrytaren vara 3-polig och bryta fasledarna, men
aldrig N-ledare och PE-ledare.
\end{description}

\begin{center}
\begin{minipage}{0.19\columnwidth}
\Huge{\warningsymbol}
\end{minipage}
\begin{minipage}{0.7\columnwidth}
\textbf{Kom ihåg, att auktoriserad installatör ska anlitas för arbete
i fasta installationer.}
\end{minipage}
\end{center}

\subsection{Liten terminologi vid elinstallationer}
\begin{description}[style=nextline]
\item[Gruppcentral] Den säkringscentral som följer efter elmätaren,
  till exempel i villor och lägenheter.

\item[Gruppledningar] Ledningar efter en gruppcentral, dvs.
  ledningar till belysning, el-spisar, uttag med mera.

\item[Fasledare] En ledare som för fasspänning.

\item[Nolledare (N-ledare)] En ledare som är ansluten till elnätets så kallade
  nollpunkt (nollskena) och som normalt inte ska föra spänning till jord.

\item[Skyddsledare (PE-ledare)] De ledare i kablar och sladdar, som är
  speciellt avsedda för skyddsjordning.

\item[Bruksföremål] Ett i princip flyttbart elanslutet föremål,
  till exempel handverktyg och radioapparater.

\item[Förstärkt isolering] Vissa bruksföremål tillverkas med en så god
  isolering att de inte behöver skyddsjordas.
  Så isolerade får anslutningsledningen förses med en speciell stickpropp,
  som passar i vägguttag, såväl med som utan jorddon.
  Sådana bruksföremål är märkta med Fi-märket bild~\ssaref{fig:Fi-mark} och får
  inte ändras så att de kan skyddsjordas.
\end{description}

\smallfig[0.1]{images/cropped_pdfs/Fi-mark.pdf}{Dubbel isolering, Fi-märke}{fig:Fi-mark}

Bild~\ssaref{fig:Fi-mark} visar Fi-märket, symbolen som finns på all elektrisk
utrustning som har dubbel isolering.

\subsection{Färgkoder för fas, noll- och skyddsledare}

Isoleringsmaterialet omkring gruppledarna i fasta elinstallationer har
färger som fyller en viktig funktion.
Tyvärr har användningen av dessa färger ändrats flera gånger under årens lopp,
vilket skapar risker för förväxling.
Ledarnas färger och funktion får aldrig förväxlas då det kan medföra fara för
allvarlig skada genom brand, elchock eller ljusbåge.

Fasledaren har numera brun färg vid nyinstallation, men har tidigare varit
både svart, grå, vit eller röd.
N-ledaren (nollan) har numera blå färg vid nyinstallation, men har tidigare
varit både svart och vit.
Skyddsledaren (PE-ledare) med gul/grön längsgående randig färgmärkning är
alltid en skyddsjordledare och får endast användas för det ändamålet.
I äldre installationer kan emellertid skyddsledarens isolering vara till
exempel röd.

Det är till fas och N-ledarna i vägguttagen, som man kopplar apparaterna för
att få ström.
Helst ska uttagen vara i skyddsjordat utförande det vill säga med ett
jordningsbleck.
Detta bleck är anslutet till den gul/gröna ledaren för skyddsjord.

\subsection{Uttag och stickproppar med jorddon}

Jorddonet ger förbindelse med elsystemets skyddsjord (PE).
Det är tidigare rummets utförande som avgjorde om vägg- och lamputtagen skulle
ha uttag med jorddon.
Kök och tvättstugor med ledande plåtbänkar, vattenkranar och så vidare anses
som riskfyllda rum och måste ha uttag med jorddon.
Samma gäller källare och liknande andra rum med ledande golv, väggar och
inredningar.
Bostadsrum var klassade som inte särskilt riskfyllda och har därför tidigare
inte försetts med lamp- och vägguttag med jorddon.

Vid nybyggnation är emellertid numera alla uttag av skyddsjordat utförande!
Det rekommenderas att installera skyddsjordade vägguttag för radiostationen.
Observera då, att alla uttag i det rummet ska vara skyddsjordade!

\subsection{Skyddsjordning}

Att jorda är det vanliga uttrycket för att ansluta ett föremål till skyddsjord.
Men uttrycket används även lite slarvigt i andra fall utan att syfta på
skyddsjordning av elsäkerhetsskäl.

Metallhöljen på elektrisk utrustning kan av olika anledningar bli
spänningsförande och är då en elsäkerhetsrisk.
För att minska risken för farlig spänningssättning av metallhöljet ansluts
höljet till skyddsjord.

\begin{center}
\begin{minipage}{0.19\columnwidth}
\Huge{\warningsymbol}
\end{minipage}
\begin{minipage}{0.7\columnwidth}
Om det blir isolationsfel mellan en strömförande del och höljet kommer
säkringen att bryta strömtillförseln och risken för skada minskar.
\textbf{PE-ledaren får därför aldrig brytas!}
\end{minipage}
\end{center}


\noindent\emph{För skyddsjordning finns särskilda föreskrifter.
  Kontakta därför en auktoriserad elinstallatör.}

\subsection{Jordfelsbrytare}
\index{jordfelsbrytare}

Jordfelsbrytare är en automatisk strömbrytare som snabbt bryter strömmen
då strömmen till och från en apparat är olika.
Detta kan inträffa vid ett jordfel eller vid överledning i en skyddsjordad
apparat eller i andra fall när inkommande ström och utgående ström genom
jordfelsbrytaren inte är lika stora.
Jordfelsbrytaren kan skydda dig

\begin{itemize}
\item vid isolations- och jordfel
\item om chassiet på en apparat blir strömförande
\item om du kommer åt spänningsförande delar och jord samtidigt
\item om vägguttagen saknar skyddsjord
\item om du använder en apparat på ett felaktigt sätt i våtutrymmen
\item om du installerat en apparat på ett felaktigt sätt
\item om apparatens kabel skadats
\item mot och minimera risken för brand.
\end{itemize}

Jordfelsbrytaren \textbf{skyddar inte} för strömmar som går genom fasledare
och neutralledare eller genom fas till fasledare (3-fas).

Jordfelsbrytare får inte ersätta skyddsjordning, men kan under särskilda
förutsättningar komplettera skyddsjordningen som en extra säkerhetsåtgärd.
Vid nyinstallation av bostäder är det numera krav på att minst en
jordfelsbrytare ska installeras.
Beställ gärna installation av jordfelsbrytare i äldre anläggningar!

\subsection{Särjordning}
\index{särjordning}

Särjordning är ett uttryck för att jorda apparater till en separat jordpunkt,
det görs via separat jordlina till ett jordtag, det vill säga jordplåt eller
jordspett.
Särjordning ska ske på rätt sätt eftersom det avsedda skyddet annars kan bli en
fara.

\emph{Om du har planer på särjordning, fråga en auktoriserad installatör.}

\subsection{Jordning av antennsystem}

I brist på annan jordpunkt är det frestande att ansluta antennjordledaren till
PE-ledarens anslutningsbleck i vägguttaget eller till ett värmeelement med
förhoppning att på så sätt få ett bättre HF-jordplan för antennen.
Detta är emellertid ett dåligt exempel på särjordning, som både kan innebära
säkerhetsrisker och medföra störningsproblem.

\subsection{Snabba och tröga säkringar}
\index{säkringar}

Det finns snabba och tröga säkringar.
Snabba säkringar är det som normalt används.
Tröga säkringar för samma strömstyrka kan behövas för apparater som har
speciellt hög startström, till exempel stora nättransformatorer med toroidkärna.

Säkringarna ska kunna bryta tillräcklig hög spänning, annars blir det
en kvarstående ljusbåge i dem vid säkringsbrott.
Använd säkringar med rätta strömvärden och välj en säkring med lite marginal
till belastningsströmmen så att säkringen inte löser ut under normal drift.

Det är förbjudet att laga säkringar då det kan orsaka brand.

% Avsnitt 13.3 Faror
\input{koncept/elsaekerhet-faror}
% Avsnitt 13.4 Åska
\section{Åska}
\harecsection{\harec{a}{10.4}{10.4}}
\index{åska}

\subsection{Faror}

Vid åska utvecklas det mycket starka, elektromagnetiska fält, som breder ut sig
och alstrar mycket korta spänningsstötar i alla metallföremål, till exempel i
antenner.
Stötarna vandrar genom kablarna in i apparaterna.
Är stötströmmen hög, kommer saker i strömvägen att förstöras på något sätt.
Förbränning och nersmältning är vanligt.

Men om blixturladdningen sker på långt håll, kan stötströmmen bli så låg att
man någorlunda kan undgå skada på apparater och hus.

Om blixturladdningen däremot sker mycket nära antennen eller som direkt nedslag,
då uppstår definitivt stora skador.

\newpage % layout
\subsection{Skydd och jordning}

Antenner och antennkablar kan man aldrig skydda mot blixtnedslag.
De är till sin natur en slags åskledare.
Det man kan försöka att göra är att leda en eventuell blixturladdning i ett
antennsystem bort från hus och människor.
Observera, att man inte får ''haka på'' husets ordinarie åskledare.
Då gäller inte husförsäkringen.

Antennkabeln, som fungerar som en (för klent dimensionerad) åskledare,
ska naturligtvis inte i onödan dras in i huset utan kortaste vägen
utanför huset till en avgrening.

Från avgreningen fortsätter dels kabeln in till apparaterna genom ett
överspänningsskydd och dels en jordlina kortaste vägen ner till
jordtaget över en gniststräcka.
Det bästa sättet att skydda apparaterna mot åska är fortfarande att koppla bort
dem helt från antennkabeln och vägguttag.

Om man bor i ett hyreshus är det tyvärr oftast svårt att få vidta åtgärder som
dem här ovan.
Då får man nöja sig med att koppla bort antennledningarna från apparaterna och
lägga dem väl åt sidan -- gärna utanför husväggen.

Som permanent, men otillräckligt skydd kan man förse de olika
anslutningsställena med lämpliga överspänningsskydd.

Mer information om hur man kan skydda sin amatörradiostation mot blixten
finns på webbplatsen för Uppsala universitet, avdelningen för elektricitetslära.
Dokumentet \emph{Att skydda sin amatörradiostation mot blixten}
<\url{http://www.hvi.uu.se/meny/m5.html}>.

%
%
% Kapitel 14 Trafikreglemente
\chapter{Trafikreglemente}
\label{ch:trafikreglemente}

\index{trafikreglemente}

För att underlätta kommunikation mellan radioamatörer oavsett vilket språk de
har som modersmål finns det överenskommelser om betydelsen av bestämda uttryck
och förkortningar.
Det finns även överenskommelser om hur radiospektrum bör användas för att så
många som möjligt ska kunna utöva sin hobby utan att störa andra.
Alla dessa överenskommelser och regler för uppförande bildar tillsammans med
lagar och radioamatörens hederskod det som vardagligt kallas trafikreglemente.

% Avsnitt 14.1 Fonetiska alfabet
\input{koncept/trafikreglemente--fonetiska-alfabet}
% Avsnitt 14.2 Q-koden
\input{koncept/trafikreglemente-q-koden}
% Avsnitt 14.3 Trafikförkortningar
\input{koncept/trafikreglemente--trafikfoerkortningar}
% Avsnitt 14.4 Internationell nödtrafik
\input{koncept/trafikreglemente--internationell-noedtrafik}
% Avsnitt 14.5 Anropssignaler
\section{Anropssignaler}
\label{anropssignaler}

\subsection{Anropssignalernas syfte}

Alla radiosändare ska vara identifierbara, så att man kan veta vem som
sänder~\cite[\S19.1]{ITU-RR}.
Identifiering görs med hjälp av en anropssignal, som är en kombination av
bokstäver, (A--Z) och siffror (0--9).~\cite[\S19.45]{ITU-RR}.
Ett tecken är antingen en bokstav eller siffra.
Nationella bokstäver som Å, Ä och Ö samt andra specialtecken används inte.
Anropssignaler är internationellt koordinerade och unika, vilket är nödvändigt
när signalerna kan komma att höras över hela världen.
Systemet är gemensamt för kommersiell trafik och amatörradio, men vi kommer
enbart beröra de anropssignaler som är aktuella för amatörradio.

\begin{itemize}
\item Alla sändningar med falsk eller missledande identifiering är förbjuden
\cite[\S19.2]{ITU-RR}!

\item Alla amatörradiosändningar ska vara identifierade~\cite[\S19.4, \S19.5]{ITU-RR}.
\end{itemize}

Identifiering sker normalt i tal eller på morsetelegrafi, men även andra former
kan förekomma som är anpassade till modulationsmetoden som används.

Det finns flera sätt på vilka personen bakom en anropssignal kan identifieras.
För svenska anropssignaler tillhandahåller SSA en Callbook
<\href{https://www.ssa.se/}{\texttt{www.ssa.se}}>.
En annan populär variant är QRZ
<\href{https://www.qrz.com/}{\texttt{www.qrz.com}}> där man kan registrera sig.
Anropssignalen används även för online-loggning av kontakter, så som Logbook of
the World (LoTW) <\href{https://lotw.arrl.org/}{\texttt{lotw.arrl.org}}>.

\subsection{Anropssignalernas sammansättning}

Varje land disponerar en eller flera serier med unika anropssignaler för all
sin radiotrafik.
Dessa utformas enligt ITU Radioreglemente (RR)~\cite[\S19]{ITU-RR} på sätt,
som beror på syftet med varje särskild radiostation.
I RR finns definitioner för olika slags stationer, till exempel stationer för
fast radio, landmobila stationer, stationer i fartyg, i sjöräddningsfarkoster,
i flygplan, amatörradiostationer och så vidare.

\subsection{Identifiering av amatörradiostationer}
\harecsection{\harec{b}{5.1}{5.1}, \harec{b}{5.3}{5.3}}

En radiostation ska identifieras med den anropssignal, som tilldelats av det
egna landets teleadministration (myndighet).
I Sverige är det Post- och telestyrelsen (PTS) som har ansvaret och som genom
beslut har delegerat handläggningen av amatörradiosignaler till Föreningen
Sveriges Sändareamatörer (SSA).
Anropssignalen meddelas i det amatörradiocertifikat som erhålls efter godkänt
kompetensprov.

Anropssignaler för amatörradio är uppbyggda av ett prefix, en siffra och ett
suffix på följande sätt~\cite[\S19.68, \S19.69]{ITU-RR}:

\begin{itemize}
\item Prefixet består vanligtvis av två tecken, exempelvis SM~(Sverige), 9A~(Kroatien)
eller S5~(Slovenien).
\item Prefixet kan ibland bestå av en ensam bokstav, som i så fall måste vara någon
av B, F, G, I, K, M, N, R eller W.
\end{itemize}

Sverige är tilldelat prefix i serierna SA--SM, 7S och 8S
\cite[Appendix 42]{ITU-RR}, se tabell~\ssaref{tab:seprefix}.

Prefixet följs av en siffra och ett suffix. Suffixet består av minst ett och
högst fyra tecken, där det sista tecknet inte får vara en siffra.

Anropssignaler för speciella ändamål, exempelvis för att fira något jubileum,
kan ha suffix som består av fler än fyra tecken~\cite[\S19.68A]{ITU-RR}.
Sådana anropssignaler, eller andra som inte följer formatmallen, behöver i så
fall godkännas av PTS innan de kan tilldelas av SSA.

\begin{exempelbox}
\signal{DL65DARC} är en eventsignal för tyska (DL) amatörradioföreningen
DARC:s 65-års jubileum.
\end{exempelbox}

PTS regler för tilldelning av svenska anropssignaler kan skilja sig från
grundreglerna i RR som anges ovan, men följer i allmänhet dessa.

Anropssignaler för svenska amatörradiostationer är uppbyggda på följande
sätt, varvid med distrikt avses amatörradiodistrikt.

\begin{table*}[ht]
  \begin{center}
    \begin{tabular}{lll}
      \emph{enskilda radioamatörer} & \textbf{SA} &
      + distriktssiffra + treställigt suffix (grundsignal) \\
      \emph{enskilda radioamatörer} & \textbf{SM} &
      + distriktssiffra + två- eller treställigt suffix (grundsignal) \\
      \emph{amatörradioklubbar} & \textbf{SA} &
      + distriktssiffra + tvåställigt suffix \\
      \emph{amatörradioklubbar} & \textbf{SK} &
      + distriktssiffra + tvåställigt suffix \\
      \emph{militära förband och FRO} & \textbf{SL} &
      + distriktssiffra + två- eller treställigt suffix \\
    \end{tabular}
    \caption{Svenska anropssignalprefix}
    \label{tab:seprefix}
  \end{center}
\end{table*}

Signalserien SM är tilldelad av Televerket och sedermera PTS fram till 2009.
Signalserien SA är tilldelad av SSA från 2004.
Äldre anropssignaler i SM-serien är tilldelade med tvåställiga suffix, medan
nyare SM- och SA-signaler har treställiga suffix.

Utöver grundsignalen finns även extra anropssignaler tilldelade i de övriga
tillgängliga serierna.

\begin{exempelbox}
	\begin{itemize}
		\item \signal{SM0XXX} är en radioamatör som fått sin tilldelning av PTS.
		\item \signal{SA0XXX} är en radioamatör som fått sin tilldelning av SSA.
		\item \signal{SK2XX} är en amatörklubb.
		\item \signal{SM7X} är en radioamatör med kort anropssignal.
	\end{itemize}
\end{exempelbox}

\medskip

Sverige är indelat i amatörradiodistrikt med följande numrering och
utsträckning:

\begin{center}
\begin{tabular}{rp{6cm}}
\emph{Distrikt} & \emph{Utsträckning} \\
\textbf{0} & Stockholms (AB) län \\
\textbf{1} & Gotlands (I) län \\
\textbf{2} & Västerbottens (AC) och Norrbottens (BD) län \\
\textbf{3} & Gävleborgs (X), Jämtlands (Z) och Västernorrlands (Y) län \\
\textbf{4} & Örebro (T), Värmlands (S) och Dalarnas (W) län \\
\textbf{5} & Östergötlands (E), Södermanlands (D), Västmanlands (U) och Uppsala (C) län\\
\textbf{6} & Hallands (N) och Västra Götalands (O) län \\
\textbf{7} & Skåne (M), Blekinge (K), Kronobergs (G), Jönköpings (F) och Kalmar (H) län.\\
\end{tabular}
\end{center}

Distriktssiffran i anropssignalen bestäms av det län som hemadressen är belägen inom.
Vid sändning utanför hemadressen bör det framgå av tillägg till anropssignalen.

\begin{exempelbox}
	\begin{itemize}
		\item \signal{SA0XXX} är en radioamatör hemmahörande i Stockholms län.
		\item \signal{SM7YYY} är en radioamatör hemmahörande i Jönköpings län.
		\item \signal{SK7AX} är en amatörklubb hemmahörande i Jönköpings län.
	\end{itemize}
\end{exempelbox}
\medskip

I Post- och telestyrelsens föreskrifter sägs dock inte vilken distriktssiffra
som ska användas, när sändning sker från annan plats än hemortsadressen.

Med stöd av praxis rekommenderar dock SSA att följande regler tillämpas:

\begin{itemize}
\item Vid trafik från en regelbundet använd fritidsbostad kan i
  anropssignalen användas den distriktssiffra som utvisar var
  fritidsbostaden är belägen.

\item Vid trafik från annan tillfällig plats bör anropssignalen
  åtföljas av snedstreck och siffran för det distrikt varifrån
  sändningen görs. Till exempel \signal{SM0XYZ} i distrikt 6 blir \signal{SM0XYZ/6}
  vilket låter som ``S M nolla X Y Z streck sexa.''

\item Vid trafik från mobil station bör den ordinarie anropssignalen
  även åtföljas av \signal{/M}. Till exempel \signal{SM0XYZ} mobil i distrikt 6 blir \signal{SM0XYZ/6/M} vilket låter som ``S M nolla X Y Z streck sexa mobil.''

\item Vid trafik från mobil station inom hemorten kan dock den extra
  distriktssiffran utelämnas.  Till exempel \signal{SM9XYZ} mobil hemma vid blir \signal{SM0XYZ/M} vilket låter som ``S M nolla X Y Z mobil.''

%% k7per???   
\item Vid trafik från sjöfarkost bör den ordinarie anropssignalen
  åtföljas av \signal{/MM} vilket låter som ``maritime mobil.''

%% k7per???   
\item Vid trafik från luftfarkost bör den ordinarie anropssignalen
  åtföljas av \signal{/AM} vilket låter som ``aeromobil.''

\item Vid trafik från svensk farkost på internationellt territorium
 kan distriktssiffran 8 användas.

\item Vid sändning från ett annat lands territorium gäller det landets
  bestämmelser.
  Vid osäkerhet -- vänd dig till SSA!

\item Utländsk radioamatör på besök i Sverige ska använda sin
  anropssignal från det egna landet, föregånget av \signal{SM/}. Till exempel \signal{SM/LA9XX} vilket låter som ``S M streck L A nia X X''~\cite{TR6101}.
\end{itemize}

\subsection{Nationella prefix}
\harecsection{\harec{b}{5.4}{5.4}}

Tabell~\ssaref{tab:landsprefix} visar några viktiga nationella prefix att kunna.

\begin{table*}[ht]
  \begin{center}
    \begin{minipage}{.3\linewidth}
      \begin{tabular}{ll}
        \emph{Prefix} & \emph{Land} \\
        \hline
        DL            & Tyskland    \\
        EA            & Spanien        \\
        EA8           & Kanarieöarna   \\
        ES            & Estland     \\
        F             & Frankrike   \\
        G             & Storbritannien \\
        HB            & Schweiz     \\
        HS            & Thailand    \\
        I             & Italien     \\
        JA            & Japan          \\
      \end{tabular}
    \end{minipage}
    \begin{minipage}{.3\linewidth}
      \begin{tabular}{ll}
        \emph{Prefix} & \emph{Land} \\
        \hline
        K             & USA         \\
        LA            & Norge       \\
        LU            & Argentina      \\
        LY            & Litauen        \\
        OH            & Finland     \\
        OH0           & Åland          \\
        OK            & Tjeckien    \\
        ON            & Belgien     \\
        OZ            & Danmark     \\
        PA            & Holland        \\
      \end{tabular}
    \end{minipage}
    \begin{minipage}{.3\linewidth}
      \begin{tabular}{ll}
        \emph{Prefix} & \emph{Land} \\
        \hline
        PY            & Brasilien   \\
        S5            & Slovenien   \\
        SP            & Polen       \\
        SV            & Grekland       \\
        UA            & Ryssland    \\
        VE            & Kanada      \\
        VK            & Australien  \\
        YL            & Lettland    \\
        ZL            & Nya Zeeland    \\
        ZS            & Sydafrika   \\
      \end{tabular}
    \end{minipage}
    \caption{Landsprefix}
    \label{tab:landsprefix}
  \end{center}
\end{table*}

\subsection{Användning av anropssignal}
\harecsection{\harec{b}{5.2}{5.2}, \harec{b}{7.2.2}{7.2.2}}

Både motstationens och den egna anropssignalen ska användas i början
och slutet av varje sändning.
Under sändningen ska anropssignalen upprepas ''med korta mellanrum'', utan
närmare precisering av mellanrummet.
Även om man inte har kontakt med en motstation, ska den egna anropssignalen
anges vid varje sändning.
Se vidare i PTS föreskrifter.

% Avsnitt 14.6 Exempel på kontakt
\section{Exempel på kontakt}
\harecsection{\harec{b}{7.2.1}{7.2.1}}

Det finns många sätt att genomföra en radiokontakt, men det finns några
grundregler för hur man uppträder och utväxlar samtal.
Ett trevligt och kamratligt uppträdande är en hederssak inom amatörradion.
Det behöver inte bli stelt för den skull!

Allmänt anrop är ett sätt att kalla på någon
-- vem som helst -- att kommunicera med.
På telegrafi låter det så här:

-- CQ CQ CQ de SM0XYZ SM0XYZ K

Det vill säga anropet först och därefter den egna anropssignalen.
På telefoni låter det så här:

-- Allmänt anrop, allmänt anrop, allmänt anrop från SM0XYZ Kom

Glöm inte Kom i slutet.
Riktat anrop gör man, när man vill tala med någon särskild station.
Då sänder man först anropssignalen på den station, som man vill tala med och
därefter sin egen anropssignal.
På telegrafi låter det så här:

-- SM0ZYX SM0ZYX de SM0XYZ SM0XYZ K

På telefoni låter det så här:

-- SM0ZYX SM0ZYX från SM0XYZ SM0XYZ Kom

Motstationen svarar förhoppningsvis på anropet, alltså

-- SM0XYZ från SM0ZYX Kom

\subsection{Upprättad förbindelse}

När en station svarat på anrop, lämnar man först sin signalrapport enligt
RST-koden och presenterar sig med sitt förnamn och berättar var man finns.
Motstationen kvitterar troligen med sina motsvarande uppgifter.

När man överlämnar ordet till motstationen avslutar man meningen med Kom och
lyssnar.
Om man har en telegrafiförbindelse och bara vill att den station man har
förbindelse med ska svara kan man sända KN (eng. \emph{come named station}).

Om förbindelsen varar länge, är det lämpligt att upprepa anropssignalerna
ungefär var tionde minut vid överlämning.

-- SM0ZYX från SM0XYZ Kom

\subsection{Avsluta förbindelse}

När man så småningom avslutar kontakten tackar man för sig på och utbyter
avskedshälsningar. Då kan det låta så här:

-- Tack för en trevlig förbindelse och på återhörande. SM0ZYX från
SM0XYZ. Klart Slut.

Träna med din instruktör på att klara olika slags trafiksituationer!

\subsection{Second operator}
\index{Second operator}
\label{secondoperator}

Den som självständigt använder en amatörradiosändare ska ha ett
amatörradiocertifikat.
Det finns ett undantag från kravet på amatörradiocertifikat då en person
tillfälligt använder en amatörradiosändare under uppsikt av någon som har ett
amatörradiocertifikat.
Detta kallas \emph{second operator} och innebär att en person som saknar
amatörradiocertifikat kan agera operatör jämte en person som har ett.

I Sverige är det reglerat i undantagsföreskriften PTSFS 2022:19 som tas upp i
avsnitt~\ssaref{PTSFS2022:19}.
Detta medger att man kan förevisas hobbyn och även träna under kontrollerade
förhållanden.
För att detta ska fungera krävs att den med amatörradiocertifikat instruerar
om hur man ska bete sig i etern, hur handhavandet går till och kan övervaka
att detta följs.

Självklart används anropssignalen för innehavaren av amatörradiocertifikatet.
Det är bra att det tydligt framgår att det är en second operator som är aktiv.
Antingen ropar amatören upp och sedan berättar att han lämnar över till second
operator Simon.
Alternativt kan en second operator göra anropen själv och då ropa till exempel
''SM5XYZ second operator Anna''.

Möjligheten att använda second operator ska användas med klokhet, och kan rätt
använd skapa en god förståelse för hobbyn och utgöra en morot för att få både
ungdomar och vuxna intresserade av amatörradio.

\subsection{CQ DX och split}
\label{cq dx och split}
\index{CQ DX}
\index{split}
\index{DX expedition}
\index{rar DX}
\index{pile-up}

Det förekommer att man hör någon ropa \emph{''CQ DX''}, vilket betyder att
stationen söker långväga kontakter, i allmänhet utanför sin egen världsdel.

-- ''CQ DX, CQ DX, CQ DX, SM0XYZ calling CQ DX and standing by''

I detta fallet är det SM0XYZ som söker att nå någon utanför Europa.
Är du själv inte ett DX, det vill säga om du befinner dig i samma världsdel så
ska du undvika att svara.

Ibland genomförs så kallade \emph{DX expeditioner} då man beger sig till en
plats som sällan aktiveras.
Man brukar tala om \emph{rara DX} (eng. \emph{rare DX}), då ett ovanligt
landområde aktiveras, som många vill ha i sin logg.

En station som ropar CQ kan få svar från många stationer samtidigt.
Då uppstår ett sammelsurium av signaler som kallas för \emph{pile-up}.

När stationen betar av en pile-up kan stationen även fråga \emph{''QRZ?''},
alltså ''vem där''?

Ett rart DX kan drabbas av enorma pile-ups, och det kan bli svårt för
motstationerna att höra DX-stationen bland alla andra som ropar samtidigt.
Det kan kan också vara svårt för DX-stationen att urskilja vilka motstationer
som svarar, om alla svarar på samma frekvens.
En strategi för att få detta att fungera effektivare är att köra split
\cite{LowBandDX}, det vill säga att DX-stationen sänder och lyssnar på olika
frekvenser (men fortfarande inom samma frekvensband).

Oftast väljer DX-stationen att lyssna på en frekvens som ligger några kilohertz
högre än den egna sändningsfrekvensen, och anger detta genom att sända
exempelvis \emph{''listening up''} eller \emph{''listening five up''}.

Genom att använda sig av split undviker DX-stationen att störas ut av sin egen
pile-up.
DX-stationen kan också välja sprida ut sin pile-up, genom att inte lyssna på
endast en frekvens, utan genom att svepa över ett lite större område.

\emph{''Listening five to ten up''} betyder då att DX-stationen lyssnar i ett
område mellan 5 och \qty{10}{\kilo\hertz} över den egna sändningsfrekvensen, och
motstationerna får försöka gissa var i detta frekvensområde som DX-stationen
lyssnar just för tillfället.

För trafik på morsetelegrafi använder man vanligtvis ett mindre avstånd mellan
sändar- och mottagarfrekvens än för trafik på SSB-telefoni, eftersom
telegrafisignalerna upptar mindre bandbredd.

Moderna transceivrar har nästan alltid möjlighet att ställa in split genom att
man använder ''VFO A/B'', ''RIT/XIT'' eller ''clarifier''.
Mer avancerade transceivrar kan ha möjlighet att separera de två frekvenserna i
hörlurarnas vänster- respektive högerkanal.

När en station ropar CQ och gör paus för anropande stationer, ange då din egen
signal kort och tydligt en gång i varje pass.
Istället för att ropa flera gånger varje pass, skrika eller på annat sätt
ta utrymme, ha tålamod och vänta ut bra tillfälle.
Ropa inte under den tid som DX-stationen sänder sitt CQ. Då hör han dig ju ändå inte.

Det kan vara nyttigt att lyssna in sig på operatörens stil.
Var medveten om att DX-stationen kan höra helt andra stationer starkare än vad
du hör, eftersom konditionerna kan vara helt annorlunda för DX-stationen.

Vid stora pile-ups kan operatören välja att bara lyssna efter vissa stationer,
och därför fråga efter ''only number five stations please'' eller efter ''only
European stations please''.
Detta syftar till att dela upp en stor pile-up för en chans att lättare
uppfatta vilka som anropar.

Vid uppdelning efter nummer kommer operatören avverka några stationer med ett
visst nummer i anropssignalen, för att sedan gå vidare till nästa och så
vidare, tills 0 till 9 är genomgångna.

Alternativet att gå efter regioner eller landsprefix kan vara att föredra om
operatören upplever att konditionerna dit snart försvinner och därför vill ge
dem en extra förtur innan de helt tappar chansen.

\paragraph{Lär dig ''DX:arens ordningsregler'':}

\begin{itemize}
\item Jag ska lyssna, och lyssna, och sedan lyssna lite till.
\item Jag ska ropa endast om jag kan läsa DX-sta\-tion\-en ordentligt.
\item Jag ska icke lita på cluster-information, utan vara helt klar över DX-stationens anropssignal innan jag ropar.
\item Jag ska icke störa DX-stationen, eller någon som anropar denne, och jag ska aldrig stämma av på DX:ets egen frekvens, eller i det segment där denne lyssnar.
\item Jag ska vänta tills DX:et avslutat föregående kontakt innan jag ropar själv.
\item Jag ska alltid ange min fullständiga anropssignal.
\item Jag ska ropa och lyssna med lämpliga intervaller.
\item Jag ska icke ropa kontinuerligt.
\item Jag ska icke ropa då DX:et svarar någon annan än mig.
\item Jag ska icke ropa då DX:et frågar efter en anropssignal som icke liknar min egen.
\item Jag ska icke ropa då DX:et söker efter ett annat geografiskt område än mitt eget.
\item När DX:et svarar mig så ska jag icke upprepa min anropssignal, annat än om jag tror att denne ej uppfattat den korrekt.
\item Jag ska vara tacksam om och när jag får kontakt.
\item Jag ska respektera mina amatörkamrater och uppträda så jag förtjänar deras respekt.
\end{itemize}

Försök inte agera polis och rätta andra stationer som du anser bryter mot reglerna!

% Avsnitt 14.7 Innehåll i förbindelse
\input{koncept/trafikreglemente--innehaall-i-foerbindelse}
% Avsnitt 14.8 Loggbok
\input{koncept/trafikreglemente-loggbok}
% Avsnitt 14.9 Hederskod
\newpage % layout
\section[Hederskod]{Radioamatörens hederskod}
\harecsection{\harec{b}{7.1.1}{7.1.1}}
\index{Radioamatörens hederskod}

Radioamatören är

\begin{description}
  \item[HÄNSYNSFULL] Han agerar aldrig medvetet på ett sätt som minskar nöjet för andra.

  \item[LOJAL] Han erbjuder lojalitet, uppmuntran och stöd åt andra amatörer, lokala klubbar,
    IARU organisationen i hans land genom vilken amatörradio i hans land
    representeras nationellt och internationellt.

  \item[PROGRESSIV] Han håller sin station på en hög teknisk nivå. Den är välbyggd och effektiv.
    Hans operationsteknik är oantastlig.

  \item[VÄNLIG] Han kommunicerar sakta och tålmodigt när så begärs; erbjuder kamratligt stöd och ger nybörjaren goda råd; vänlig assistans, samarbete och omtanke i andras intresse. Detta är kännetecknen för amatörandan.

  \item[BALANSERAD] Radio är en hobby och får aldrig orsaka konflikt i förpliktelser gentemot
    familj, arbete, skola eller samhälle.

  \item[PATRIOTISK] Hans station och hans kunnande står alltid till förfogande för att
    assistera land och samhälle.
\end{description}

%% \begin{tabular}{lp{6cm}}
%%   \textbf{HÄNSYNSFULL} &
%%   Han agerar aldrig medvetet på ett sätt som minskar nöjet för andra. \\
%%   & \\
  
%%   \textbf{LOJAL} &
%%   Han erbjuder lojalitet, uppmuntran och stöd åt andra amatörer, lokala klubbar,
%%   IARU organisationen i hans land genom vilken amatörradio i hans land
%%   representeras nationellt och internationellt.\\
%%   & \\
  
%%   \textbf{PROGRESSIV} &
%%   Han håller sin station på en hög teknisk nivå.
%%   Den är välbyggd och effektiv.
%%   Hans operationsteknik är oantastlig.\\
%%   & \\
  
%%   \textbf{VÄNLIG} &
%%   Han kommunicerar sakta och tålmodigt när så begärs;
%%   erbjuder kamratligt stöd och ger nybörjaren goda råd;
%%   vänlig assistans, samarbete och omtanke i andras intresse.
%%   Detta är kännetecknen för amatörandan.\\
%%   & \\
  
%%   \textbf{BALANSERAD} &
%%   Radio är en hobby och får aldrig orsaka konflikt i förpliktelser gentemot
%%   familj, arbete, skola eller samhälle.\\
%%   & \\
  
%%   \textbf{PATRIOTISK} &
%%   Hans station och hans kunnande står alltid till förfogande för att
%%   assistera land och samhälle.\\
%% \end{tabular}

\emph{-- anpassad från den ursprungliga Amateur's Code, skriven av Paul M. Segal, W9EEA, 1928.}

% Avsnitt 14.10 Ordningsregler
\input{koncept/trafikreglemente-ordningsregler}
% Avsnitt 14.11 Bandplaner
\input{koncept/trafikreglemente-bandplaner}
%
%
% Kapitel 15 Bestämmelser
\chapter{Bestämmelser}

Tekniskt sett kan radioamatörerna världen över, med hjälp av sina
radiostationer, tämligen lätt skapa kontakt med varandra.
Därvid krävs att reglerna i de länder som berörs vid kontakten respekteras.

En hel serie både internationella och nationella regler styr
radiokommunikationerna i en nation.
Varje radioamatör ska känna till och följa dessa regler så långt de har
anslutning till amatörradio.
Vissa länder -- till exempel CEPT-länderna -- har i någon utsträckning
harmoniserat sina bestämmelser inbördes.
Nationella avvikelser förekommer likväl och reglerna i det land, som man gör
radiosändningar ifrån, ska alltid följas.

% Avsnitt 15.1 ITU RR
\input{koncept/bestaemmelser-itu-rr}
% Avsnitt 15.2 CEPT
\input{koncept/bestaemmelser-cept}
% Avsnitt 15.3 Svensk lag och föreskrift
\section{Svensk lag och föreskrift}
\label{svensk lag och föreskrift}

Lagar, föreskrifter och anvisningar tillämpas för
  amatörradioanvändning.
Märk, att ändringar kan förekomma.
\textbf{Använd därför aktuella versioner!}


\subsection{Lag om elektronisk kommunikation}
\harecsection{\harec{c}{3.1}{3.1}}
\index{Lag om elektronisk kommunikation}
\index{LEK|see {Lag om elektronisk kommunikation}}

\emph{Lag (2022:482) om elektronisk kommunikation}~\cite{SFS2022:482} reglerar
all radiokommunikation Sverige.
Tillstånd behövs för all radiosändning som inte är undantagen tillståndsplikt.
Lagen förkortas ofta LEK.

\emph{Post- och telestyrelsen} (PTS) är enligt förordning (2022:511) om
elektronisk kommunikation den svenska myndighet som handlägger ärenden gällande
telekommunikation.
PTS ska bland annat svara för att möjligheterna till radiokommunikationer
utnyttjas effektivt och har därvid att beakta den internationella regleringen
inom området.
Regleringen av amatörradioanvändningen begränsas nu till den minsta omfattning
som följer av internationella avtal och europeiska rekommendationer,
CEPT-rekommendationer.

\newpage %layout
\subsection{Post- och telestyrelsens föreskrifter om undantag från tillståndsplikt för användning av vissa radiosändare}
\harecsection{\harec{c}{3.2}{3.2}}
\index{amatörradiocertifikat}
\index{amatörradiosändare}
\index{amatörradiotrafik}
\index{antennvinst}
\index{antennförstärkning}
\index{ERP|see {effektivt utstrålad effekt}}
\index{effektivt utstrålad effekt (ERP)}
\index{antenn!effektivt utstrålad effekt (ERP)}
\index{Effective Radiated Power (ERP)}
\index{antenn!Effective Radiated Power (ERP)}
\index{antenn!ERP}
\index{ekvivalent isotropiskt utstrålad effekt (EIRP)}
\index{antenn!ekvivalent isotropiskt utstrålad effekt (EIRP)}
\index{Equivalent Isotropically Radiated Power (EIRP)}
\index{antenn!Equivalent Isotropically Radiated Power (EIRP)}
\index{EIRP}
\index{antenn!EIRP}
\index{PEP}
\label{PTSFS2022:19}
\index{T/R 61-02}
\index{PTSFS 2022:19}

Post- och telestyrelsen föreskriver i PTSFS~2022:19~\cite{PTSFS2022:19} med stöd
av 5~\S{} Förordningen (2022:511)~\cite{SFS2022:511} om elektronisk kommunikation
att användningen av amatörradiosändare är undantagen tillståndsplikt.
Notera att PTS med viss regelbundenhet uppdaterar undantagsföreskrifterna,
och därför bör man kontrollera på PTS webbplats vad som är den senaste versionen
och använda den när den trätt i kraft.

I undantagsföreskriften~\cite{PTSFS2022:19} finns följande definitioner som är
relevanta för amatörradiotjänsten:

\begin{description}
\item[amatörradiocertifikat] kunskapsbevis utfärdat eller godkänt av
Post- och telestyrelsen, som utvisar att godkänt kunskapsprov avlagts.

\item[amatörradiosändare] radiosändare som är avsedd att användas av personer
som har amatörradiocertifikat, för sändning på frekvenser som är avsedda för
amatörradiotrafik.

\item[amatörradiotrafik] icke yrkesmässig radiotrafik för övning,
kommunikation och tekniska undersökningar, bedriven i personligt radiotekniskt
intresse och utan vinstsyfte.

\item[antennvinst] förstärkning i förhållande till en referensantenn som
antingen är isotropisk eller en dipol och som mäts i dBi eller dBd.
Antennvinsten anger hur bra riktverkan en antenn har.

\item[\eirp] equivalent isotropically radiated power (ekvivalent
isotropiskt utstrålad effekt).

\item[\erp] effective radiated power (effektivt utstrålad effekt relativt en
halvvågsdipol).

\item[\pep] peak envelope power.
\end{description}
%%
Vidare anges ytterligare villkor i 3~kap. 26~\S{} av undantagsföreskriften
\cite{PTSFS2022:19}:
%%

De tekniska egenskaperna hos amatörradiosändaren ska anpassas så att de inte
stör användningen av andra radioanläggningar.
Den som använder en amatörradiosändare ska ha ett amatörradiocertifikat.
För att få ett amatörradiocertifikat krävs kunskaper i enlighet med Annex 6 i
CEPT Rekommendation T/R~61-02~\cite{TR6102}.

Undantag från kravet på amatörradiocertifikat gäller för den som under en
tidsbegränsad period utbildar sig för att få ett sådant certifikat och för den
som under en förevisning tillfälligt använder amatörradiosändare, under
förutsättning att användningen av radiosändaren sker under uppsikt av en
innehavare av amatörradiocertifikat.
(Läs mer om användningen i avsnitt~\ssaref{secondoperator})

Den som innehar amatörradiocertifikat ska ha en egen anropssignal.
Denna framgår av certifikatet, eller tidigare av amatörradiotillståndet.
Mottagare- och sändarestationens anropssignaler ska sändas i början och i
slutet av varje radioförbindelse.
Anropssignalerna ska också upprepas med korta mellanrum under pågående
radioförbindelse. Under de utbildnings- och förevisningstillfällen som anges i
stycket ovan ska anropssignal användas som tillhör den innehavare av
amatörradiocertifikat som har uppsikt över användningen av radiosändaren.
Vid dessa tillfällen får även anropssignal som tillhör den amatörradioförening
eller institution som anordnar utbildnings- eller förevisningstillfället
användas om företrädare för föreningen eller institutionen har uppsikt över
användningen av radiosändaren.

Automatiska amatörradiosändare, till exempel en radiofyr, repeater eller
sändare för positionering ska alltid kunna identifieras genom att en
anropssignal regelbundet sänds med morsetelegrafi, röstmeddelande eller
på annat sätt.
Anropssignalen ska ange vem som är ansvarig för den automatiska sändaren.
Den som startar eller använder automatiska amatörradiosändare ska ha eget
amatörradiocertifikat och ska använda egen anropssignal.
Sådan start och användning får även utföras av den som inte har
amatörradiocertifikat, om det sker under uppsikt av en innehavare av
amatörradiocertifikat och dennes anropssignal används.

\subsection{Litteraturhänvisning om lagar och föreskrifter}

\begin{itemize}
\item CEPT rekommendation T/R~61-01~\cite{TR6101}
\item CEPT rekommendation T/R~61-02~\cite{TR6102}
\item Lag (2022:482) om elektronisk kommunikation~\cite{SFS2022:482}
\item Förordning (2022:511) om elektronisk kommunikation~\cite{SFS2022:511}
\item Post- och telestyrelsens föreskrifter om undantag från tillståndsplikt för
användning av vissa radiosändare PTSFS 2022:19~\cite{PTSFS2022:19}
\end{itemize}

%
%
