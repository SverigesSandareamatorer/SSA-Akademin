\section{Anropssignaler}
\label{anropssignaler}

\subsection{Anropssignalernas syfte}

Alla radiosändare ska vara identifierbara, så att man kan veta vem som
sänder~\cite[\S19.1]{ITU-RR}.
Identifiering görs med hjälp av en anropssignal, som är en kombination av
bokstäver, (A--Z) och siffror (0--9).~\cite[\S19.45]{ITU-RR}.
Ett tecken är antingen en bokstav eller siffra.
Nationella bokstäver som Å, Ä och Ö samt andra specialtecken används inte.
Anropssignaler är internationellt koordinerade och unika, vilket är nödvändigt
när signalerna kan komma att höras över hela världen.
Systemet är gemensamt för kommersiell trafik och amatörradio, men vi kommer
enbart beröra de anropssignaler som är aktuella för amatörradio.

\begin{itemize}
\item Alla sändningar med falsk eller missledande identifiering är förbjuden
\cite[\S19.2]{ITU-RR}!

\item Alla amatörradiosändningar ska vara identifierade~\cite[\S19.4, \S19.5]{ITU-RR}.
\end{itemize}

Identifiering sker normalt i tal eller på morsetelegrafi, men även andra former
kan förekomma som är anpassade till modulationsmetoden som används.

Det finns flera sätt på vilka personen bakom en anropssignal kan identifieras.
För svenska anropssignaler tillhandahåller SSA en Callbook
<\href{https://www.ssa.se/}{\texttt{www.ssa.se}}>.
En annan populär variant är QRZ
<\href{https://www.qrz.com/}{\texttt{www.qrz.com}}> där man kan registrera sig.
Anropssignalen används även för online-loggning av kontakter, så som Logbook of
the World (LoTW) <\href{https://lotw.arrl.org/}{\texttt{lotw.arrl.org}}>.

\subsection{Anropssignalernas sammansättning}

Varje land disponerar en eller flera serier med unika anropssignaler för all
sin radiotrafik.
Dessa utformas enligt ITU Radioreglemente (RR)~\cite[\S19]{ITU-RR} på sätt,
som beror på syftet med varje särskild radiostation.
I RR finns definitioner för olika slags stationer, till exempel stationer för
fast radio, landmobila stationer, stationer i fartyg, i sjöräddningsfarkoster,
i flygplan, amatörradiostationer och så vidare.

\subsection{Identifiering av amatörradiostationer}
\harecsection{\harec{b}{5.1}{5.1}, \harec{b}{5.3}{5.3}}

En radiostation ska identifieras med den anropssignal, som tilldelats av det
egna landets teleadministration (myndighet).
I Sverige är det Post- och telestyrelsen (PTS) som har ansvaret och som genom
beslut har delegerat handläggningen av amatörradiosignaler till Föreningen
Sveriges Sändareamatörer (SSA).
Anropssignalen meddelas i det amatörradiocertifikat som erhålls efter godkänt
kompetensprov.

Anropssignaler för amatörradio är uppbyggda av ett prefix, en siffra och ett
suffix på följande sätt~\cite[\S19.68, \S19.69]{ITU-RR}:

\begin{itemize}
\item Prefixet består vanligtvis av två tecken, exempelvis SM~(Sverige),
9A~(Kroatien) eller S5~(Slovenien).
\item Prefixet kan ibland bestå av en ensam bokstav, som i så fall måste vara
någon av B, F, G, I, K, M, N, R eller W.
\end{itemize}

Sverige är tilldelat prefix i serierna SA--SM, 7S och 8S
\cite[Appendix 42]{ITU-RR}, se tabell~\ssaref{tab:seprefix}.

Prefixet följs av en siffra och ett suffix. Suffixet består av minst ett och
högst fyra tecken, där det sista tecknet inte får vara en siffra.

Anropssignaler för speciella ändamål, exempelvis för att fira något jubileum,
kan ha suffix som består av fler än fyra tecken~\cite[\S19.68A]{ITU-RR}.
Sådana anropssignaler, eller andra som inte följer formatmallen, behöver i så
fall godkännas av PTS innan de kan tilldelas av SSA.

\begin{exempelbox}
\signal{DL65DARC} är en eventsignal för tyska (DL) amatörradioföreningen
DARC:s 65-års jubileum.
\end{exempelbox}

PTS regler för tilldelning av svenska anropssignaler kan skilja sig från
grundreglerna i RR som anges ovan, men följer i allmänhet dessa.

Anropssignaler för svenska amatörradiostationer är uppbyggda på följande
sätt, varvid med distrikt avses amatörradiodistrikt.

\begin{table*}[ht]
  \begin{center}
    \begin{tabular}{lll}
      \emph{enskilda radioamatörer} & \textbf{SA} &
      + distriktssiffra + treställigt suffix (grundsignal) \\
      \emph{enskilda radioamatörer} & \textbf{SM} &
      + distriktssiffra + två- eller treställigt suffix (grundsignal) \\
      \emph{amatörradioklubbar} & \textbf{SA} &
      + distriktssiffra + tvåställigt suffix \\
      \emph{amatörradioklubbar} & \textbf{SK} &
      + distriktssiffra + tvåställigt suffix \\
      \emph{militära förband och FRO} & \textbf{SL} &
      + distriktssiffra + två- eller treställigt suffix \\
    \end{tabular}
    \caption{Svenska anropssignalprefix}
    \label{tab:seprefix}
  \end{center}
\end{table*}

Signalserien SM är tilldelad av Televerket och sedermera PTS fram till 2009.
Signalserien SA är tilldelad av SSA från 2004.
Äldre anropssignaler i SM-serien är tilldelade med tvåställiga suffix, medan
nyare SM- och SA-signaler har treställiga suffix.

Utöver grundsignalen finns även extra anropssignaler tilldelade i de övriga
tillgängliga serierna.

\begin{exempelbox}
	\begin{itemize}
		\item \signal{SM0XXX} är en radioamatör som fått sin tilldelning av PTS.
		\item \signal{SA0XXX} är en radioamatör som fått sin tilldelning av SSA.
		\item \signal{SK2XX} är en amatörklubb.
		\item \signal{SM7X} är en radioamatör med kort anropssignal.
	\end{itemize}
\end{exempelbox}

\medskip

Sverige är indelat i amatörradiodistrikt med följande numrering och
utsträckning:

\begin{center}
\begin{tabular}{rp{6cm}}
\emph{Distrikt} & \emph{Utsträckning} \\
\textbf{0} & Stockholms (AB) län \\
\textbf{1} & Gotlands (I) län \\
\textbf{2} & Västerbottens (AC) och Norrbottens (BD) län \\
\textbf{3} & Gävleborgs (X), Jämtlands (Z) och Västernorrlands (Y) län \\
\textbf{4} & Örebro (T), Värmlands (S) och Dalarnas (W) län \\
\textbf{5} & Östergötlands (E), Södermanlands (D), Västmanlands (U) och Uppsala (C) län\\
\textbf{6} & Hallands (N) och Västra Götalands (O) län \\
\textbf{7} & Skåne (M), Blekinge (K), Kronobergs (G), Jönköpings (F) och Kalmar (H) län.\\
\end{tabular}
\end{center}

Distriktssiffran i anropssignalen bestäms av det län som hemadressen är belägen inom.
Vid sändning utanför hemadressen bör det framgå av tillägg till anropssignalen.

\begin{exempelbox}
	\begin{itemize}
		\item \signal{SA0XXX} är en radioamatör hemmahörande i Stockholms län.
		\item \signal{SM7YYY} är en radioamatör hemmahörande i Jönköpings län.
		\item \signal{SK7AX} är en amatörklubb hemmahörande i Jönköping län.
	\end{itemize}
\end{exempelbox}
\medskip

I Post- och telestyrelsens föreskrifter sägs dock inte vilken distriktssiffra
som ska användas, när sändning sker från annan plats än hemortsadressen.

Med stöd av praxis rekommenderar dock SSA att följande regler tillämpas:

\begin{itemize}
\item Vid trafik från en regelbundet använd fritidsbostad kan i
  anropssignalen användas den distriktssiffra som utvisar var
  fritidsbostaden är belägen.

\item Vid trafik från annan tillfällig plats bör anropssignalen
  åtföljas av snedstreck och siffran för det distrikt varifrån
  sändningen görs. Till exempel \signal{SM0XYZ} i distrikt 6 blir \signal{SM0XYZ/6}
  vilket låter som ``S M nolla X Y Z streck sexa.''

\item Vid trafik från mobil station bör den ordinarie anropssignalen
  även åtföljas av \signal{/M}. Till exempel \signal{SM0XYZ} mobil i distrikt 6 blir \signal{SM0XYZ/6/M} vilket låter som ``S M nolla X Y Z streck sexa mobil.''

\item Vid trafik från mobil station inom hemorten kan dock den extra
  distriktssiffran utelämnas.  Till exempel \signal{SM9XYZ} mobil hemma vid blir \signal{SM0XYZ/M} vilket låter som ``S M nolla X Y Z mobil.''

%% k7per???   
\item Vid trafik från sjöfarkost bör den ordinarie anropssignalen
  åtföjas av \signal{/MM} vilket låter som ``maritime mobil.''

%% k7per???   
\item Vid trafik från luftfarkost bör den ordinarie anropssignalen
  åtföljas av \signal{/AM} vilket låter som ``aeromobil.''

\item Vid trafik från svensk farkost på internationellt territorium
 kan distriktssiffran 8 användas.

\item Vid sändning från ett annat lands territorium gäller det landets
  bestämmelser.
  Vid osäkerhet -- vänd dig till SSA!

\item Utländsk radioamatör på besök i Sverige ska använda sin
  anropssignal från det egna landet, föregånget av \signal{SM/}. Till exempel \signal{SM/LA9XX} vilket låter som ``S M streck L A nia X X''~\cite{TR6101}.
\end{itemize}

\subsection{Nationella prefix}
\harecsection{\harec{b}{5.4}{5.4}}

Tabell~\ssaref{tab:landsprefix} visar några viktiga nationella prefix att kunna.

\begin{table*}[ht]
  \begin{center}
    \begin{minipage}{.3\linewidth}
      \begin{tabular}{ll}
        \emph{Prefix} & \emph{Land} \\
        \hline
        DL            & Tyskland    \\
        EA            & Spanien        \\
        EA8           & Kanarieöarna   \\
        ES            & Estland     \\
        F             & Frankrike   \\
        G             & Storbritannien \\
        HB            & Schweiz     \\
        HS            & Thailand    \\
        I             & Italien     \\
        JA            & Japan          \\
      \end{tabular}
    \end{minipage}
    \begin{minipage}{.3\linewidth}
      \begin{tabular}{ll}
        \emph{Prefix} & \emph{Land} \\
        \hline
        K             & USA         \\
        LA            & Norge       \\
        LU            & Argentina      \\
        LY            & Litauen        \\
        OH            & Finland     \\
        OH0           & Åland          \\
        OK            & Tjeckien    \\
        ON            & Belgien     \\
        OZ            & Danmark     \\
        PA            & Holland        \\
      \end{tabular}
    \end{minipage}
    \begin{minipage}{.3\linewidth}
      \begin{tabular}{ll}
        \emph{Prefix} & \emph{Land} \\
        \hline
        PY            & Brasilien   \\
        S5            & Slovenien   \\
        SP            & Polen       \\
        SV            & Grekland       \\
        UA            & Ryssland    \\
        VE            & Kanada      \\
        VK            & Australien  \\
        YL            & Lettland    \\
        ZL            & Nya Zeeland    \\
        ZS            & Sydafrika   \\
      \end{tabular}
    \end{minipage}
    \caption{Landsprefix}
    \label{tab:landsprefix}
  \end{center}
\end{table*}

\subsection{Användning av anropssignal}
\harecsection{\harec{b}{5.2}{5.2}, \harec{b}{7.2.2}{7.2.2}}

Både motstationens och den egna anropssignalen ska användas i början
och slutet av varje sändning.
Under sändningen ska anropssignalen upprepas ''med korta mellanrum'', utan
närmare precisering av mellanrummet.
Även om man inte har kontakt med en motstation, ska den egna anropssignalen
anges vid varje sändning.
Se vidare i PTS föreskrifter.
