\chapter*{Introduktion}
\section*{Amatörradio}
Amatörradio är en teknisk hobby med inriktning på kommunikation och experiment
med radioanläggningar samt radiovågors utbredning. Det är en verksamhet som
utövas över hela världen av licensierade radioamatörer, även kallade
sändaramatörer.

Syftet med amatörradio är att främja personlig utveckling och internationell
förståelse samt teknisk färdighet och erfarenhetsutbyte inom området.
Amatörradio kan därtill vara en tillgång då samhällets normala resurser för
radiokommunikation behöver förstärkas.

\section*{En hobby med krav}

För att använda en radiosändare, och i vissa fall inneha, i ett land, krävs
tillstånd (licens) från dess teleadministration.
För ett amatörradiotillstånd föreskrivs i det internationella radioreglementet
\cite{ITU-RR} bland annat handhavandemässiga och tekniska kvalifikationer hos
varje person som önskar använda en amatörradiostation.
De nationella teleadministrationerna tillser detta genom kompetensprov.
För att få sända med amatörradiosändare måste man ha amatörradiocertifikat.

CEPT är ett samarbetsorgan mellan europeiska länders teleadministrationer
(myndigheter).
En av dem är svenska Post- och telestyrelsen -- PTS.

Dessa administrationer har antagit rekommendationer om sinsemellan
harmoniserade krav på radioamatörers kompetens.

Sverige har antagit CEPT-rekommendationen T/R 61-02 \cite{TR6102}.
Vid genomförandet av kompetensprov ska de i den rekommendationen
angivna kraven särskilt beaktas.

För den som godkänts i ett sådant prov utfärdas ett harmoniserat
amatörradiocertifikat (HAREC).
Rekommendationen anger kompetensnivån HAREC.
Den svenska certifikatet bygger på CEPT HAREC krav \cite{TR6102},
med anpassning till svensk frekvensplan i Bilaga \ref{svensk frekvensplan}.

De detaljerade CEPT HAREC kraven finns i Bilaga \ref{CEPT HAREC}, där även
referenser till den eller de del-kapitel som avses uppfylla utbildningenkraven.
Alla de delkapitel som är märkta med HAREC ingår alltså i den internationellt
överenskomna kunskapsmängden som utbildningen ska inkludera.

\section*{Utbildning}

Man kan antingen söka sig till någon av de klubbar som har kurs eller skaffa
SSA:s utbildningspaket och studera på egen hand.
Post- och telestyrelsen har dessutom övningsprov online som man kan testa sina
kunskaper på, något som varmt rekommenderas för alla studerande oavsett
studieform.

Amatörradioklubbarna bedriver huvuddelen av utbildningen med
amatör\-radio\-certi\-fikat som mål.
Även vissa skolor, militära förband, FRO-förbund med flera har amatörradio på
programmet.
Se SSA:s webbplats (\href{http://www.ssa.se}{www.ssa.se}) för aktuella kurstillfällen.

När man är mogen för att avlägga certifikatprov skriver man för någon av de
provförrättare som finns. De klubbar som har utbildning brukar planera prov
med den grupp elever de har.

Efter avlagt och godkänt prov kan man sedan ansöka om anropssignal och
certifikat, något som SSA sköter enligt delegation från Post- och telestyrelsen.

Till tillståndet knyts en internationellt unik anropssignal.
Man har möjlighet att föreslå en anropssignal, men i brist på förslag tas en ledig
anropssignal ur serien.

\section*{Föreningen Sveriges Sändareamatörer -- SSA}

SSA är en ideell förening för personer med intresse för amatörradio.
Verksamheten är religiöst och politiskt obunden.
Ett av syftena är att bland medlemmarna verka för ökade tekniska kunskaper och
god radiotrafikkultur för att därigenom skapa en kår av kunniga radioamatörer.
SSA representerar Sverige som nationell förening i
International Amateur Radio Union (IARU), Region~1.

\section*{Internationell samverkan}

De nationella föreningarna inom IARU samarbetar över nationsgränserna.
Ett exempel är när DARC (Deutscher Amateur-Radio-Club e.V.) för ett antal år
sedan ställde sina Ausbildungsunterlagen \cite{DARCaus} till SSA:s förfogande
som källmaterial till föregångaren till denna bok.

\section*{Denna bok}

Denna bok omfattar hela teorin för CEPT HAREC och PTS krav.
Den ingår i det utbildningspaket som kan köpas från SSA.

Innehållet är delat i två ämnesgrupper; grundläggande radioteknik
samt regler och trafikmetoder.
Det finns även inlärningsanvisningar för morsesignalering för den
som vill lära sig telegrafi.

I bilagorna finns bland annat grundläggande matematik
och frekvensplaner för amatörradiotrafik.

Rekrytering av handledare för terminslånga kurser är en nyckelfråga för
kursarrangören, liksom målinriktade, anpassade läromedel.

Tanken med denna bok är att leverera ett material som kan vara grunden till
denna utbildning samt även för viss fördjupning och förståelse för de begrepp
som man vanligtvis stöter på inom hobbyn.



\clearpage

\section*{Förord till andra upplagan}

Boken bygger till mycket stor del på det arbete som till första upplagan
utfördes av Lennart Wiberg SM7KHF, med flera.

Med tiden har uppstått ett behov av att bredda det existerande
utbildningsmaterialet och att anpassa det till ett modernare sätt att utbilda,
inte minst för att kunna utnyttja moderna webbaserade utbildningssystem.

En viktig aspekt har varit att materialet ska täcka hela CEPT HAREC,
som uppdaterats över åren, och vara spårbart till dessa krav.

Till denna andra upplaga har allt tidigare material granskats och uppdaterats.
Nya kapitel har lagts till, bland annat om elektromagnetiska fält, digitala
trafiksätt och digital signalbehandling.
Avsnitten om elsäkerhet och nödtrafik har omarbetats och samtliga referenser
till lagar och föreskrifter är i skrivande stund aktuella.

Den nu föreliggande andra upplagan finns tillgänglig i digitalt format.
Detta underlättar inte bara för läsaren att söka efter specifik information,
men utgör också en grund för kommande webbaserad utbildning.



TACK!

Ett stort tack till alla dem, som på olika sätt bidragit till att förverkliga
boken.
Ett särskilt tack riktas till Magnus Danielson SA0MAD, Petter Karkea SA2PKA,
Lorentz Björklund SM7NTJ, Jonas Hultin SM5PHU och Philip Eriksson.

\emph{Författarna}
