\section{Isolation}
\index{isolation}
\index{isolator}
\index{galvanisk isolation}
\index{isolation!galvanisk}

\emph{Isolation} (eng. \emph{isolation}) är ett samlingsbegrepp för att separera
olika signaler.
Den första enkla separationen är den hos en \emph{isolator}, det vill säga ett
material som inte leder ström så bra.
Det är den mest grundläggande formen av isolation som förhindrar elektrisk
ledning mellan ledningar.

Man brukar prata om \emph{galvanisk isolation} (eng. \emph{galvanic isolation})
för en isolation som inte kan leda likström.
Transformatorer används ofta för att åstadkomma galvanisk isolation.

Nu är isolation inte begränsat till enbart likström, utan även växelspänning
kan behöva isoleras.
Hur god isolationen är beror kraftigt på frekvensen, och de åtgärder man gör
bör anpassas för hur god isolation man behöver eller vill ha för olika
frekvenser.
Man kan till exempel vilja ha god isolation vid sändar- och mottagarfrekvensen
\qty{14}{\mega\hertz}, men vill inte ha galvanisk isolation för det gemensamma
\qty{12}{\volt} kraftaggregatet.

\section{Jordning}
\index{jordning}
\index{bonding}
\index{jordnät}
\index{jord!jordnät}
\index{bonding network}
\index{jordpotential}
\index{jord!jordpotential}
\index{blomjord}
\index{jord!blomjord}
\index{skyddsjord}
\index{jord!skyddsjord}
\index{nolla}
\index{jord!nolla}

\emph{Jordning} (eng. \emph{bonding}) eller dagligt tal \emph{jord} (eng.
\emph{ground}, \emph{earth}) är en kopplingsstrategi för att få samma
referenspotential i olika delar av en elektrisk koppling.
Man bygger ett \emph{jordnät} (eng. \emph{bonding network, BN})
\cite[kap 3.2.1]{K27-1991} och \emph{earthing network})
\cite[kap 3.1.3]{K27-1991} för att koppla samman de olika jordpunkterna.

Den engelska termen \emph{bonding} och även \emph{bondning network} ger en
indikation på vad det handlar om, nämligen en metod att knyta samman flera
olika delar av en design eller installation för att få en gemensam
referensspänning.
Det är helt enkelt en galvanisk sammankoppling.

Många gånger kallas den referenspotentialen för \emph{jordpotential} för att
det är väldigt behändigt att använda jorden som referens, helt enkelt gräva ned
en ledare i marken, till exempel jordspett (eng. \emph{earth electrode})
\cite[kap 3.1.2]{K27-1991}, för att den vägen få tillgång till jordpotentialen.

Begreppen jord och jordning är dock ofta missförstådda då det finns en övertro
på att man kan ta ned störningar med enbart jordning.
Det förekommer också att man upplever att man har problem där jorden upplevs
skapa störningar, varvid en del felaktigt bryter jorden, och därmed
\emph{skyddsjorden}, något man inte får göra av elsäkerhetsskäl.

På samma sätt tror många att man kan göra sig av med en stor växelström i
jorden.
Detta kallas ibland lite skämtsamt för \emph{blomjordning}, för att man inte
tagit hänsyn till jordledarens resistans och induktans, vilket gör att en
växelström inte kan ta sig så långt då ledaren motarbetar den.
Man hade lika gärna kunnat lägga ned sin jordanslutning i en blomkruka för där
gör den lika god nytta.

Inom elkraft förekommer även termen \emph{nolla} (eng. \emph{neutral}), den
kan lätt förväxlas med jorden, men ska hanteras separat från skyddsjord utom
där elsäkerhetsföreskrifter föreskriver att de ska vara sammankopplade.
Nollan är den ledare som är returledare för strömmen.
I det vanligaste elsystemet \emph{TN-C}, är nollan sammankopplad med skyddsjorden
i elcentralen, men ut från elcentralen hanteras den som en separat ledare.
Man får inte koppla ihop dem för att spara ledare!
Skyddsjorden ska ha väldigt lite ström på sig, och därmed även ha väldigt
låg spänningsskillnad från jordpotentialen, men i praktiken kommer det ändå
finnas skillnader.

\smalltikz{
  \begin{circuitikz}[american voltages]
    % Ground reference
    \draw (0,0) node[ground]{};
    % Source 1 ground
    \draw (0,0) to [R, l^=$Z_1$] (2,0);
    \draw (2,1) to [short, i^=$I_1$] (2,0);
    \draw (1.75,-1) to [short, i=$I_1+I_2+I_3$] (0.25,-1);
    % Source 2 ground
    \draw (2,0) to [R, l^=$Z_2$] (4,0);
    \draw (4,1) to [short, i^=$I_2$] (4,0);
    \draw (3.75,-1) to [short, i=$I_2+I_3$] (2.25,-1);
    % Source 3 ground
    \draw (4,0) to [R, l^=$Z_3$] (6,0);
    \draw (6,1) to [short, i^=$I_3$] (6,0);
    \draw (5.75,-1) to [short, i=$I_3$] (4.25,-1);
  \end{circuitikz}
}{Seriekopplat jordsystem}{fig:kap4-1}

\subsection{Seriekoppling av jord}

Den enklaste uppkopplingen av jordförbindelse är att seriekoppla jorden
\cite[kap 3]{ott1988} mellan ett antal strömförbrukare.
Detta förekommer till exempel i en serie av eluttag matade från samma säkring
eller flera eluttag i en skarvdosa.

I bild~\ssaref{fig:kap4-1} att vi har tre strömförbrukare som var och en
bidrar med en ström \(I_1\), \(I_2\) och \(I_3\), och att dessa är
seriekopplade till en jordanslutning.
Från jordanslutningen till strömbidraget \(I_1\) har vi impedansen \(Z_1\),
och från den punkten har vi impedansen \(Z_2\) fram till strömbidragen \(I_2\)
och slutligen impedansen \(Z_3\) fram till \(I_3\).

En naiv tolkning är att spänningen \(U_1\) för strömbidraget \(I_1\) blir
\(U_1 = Z_1 I_1\), vidare \(U_2 = (Z_1 + Z_2) I_2\) och
\(U_3 = (Z_1 + Z_2 + Z_3) I_3\) för det blir det ju om varje ström ansluts var
och en för sig, det vill säga normal seriekoppling av impedanserna.
Denna analys är dock för enkel för att ta hänsyn till fallet när strömmarna
ansluts samtidigt, eftersom strömmar och spänningar kommer samverka.

Den totala strömmen genom första impedansen \(Z_1\) blir ju summan av de tre
strömmarna, därför måste också spänningen höjas med det bidraget.
Den första spänningen blir därför \(U_1=Z_1 (I_1 + I_2 + I_3)\).
På liknande sätt beräknas den andra spänningen med de bägge strömmarna \(I_2\)
och \(I_3\) plus spänningen \(U_1\) och därför blir
\(U_2 = U_1 + Z_2 (I_2 + I_3)\).
Slutligen blir den sista spänningen \(U_3 = U_2 + Z_3 I_3\).
Med förenkling får vi
%%
\[
\begin{array}{ll}
U_1 & = Z_1 I_1 + Z_1 I_2 + Z_1 I_3 \\
U_2 & = Z_1 I_1 + (Z_1 + Z_2) I_2 + (Z_1 + Z_2) I_3 \\
U_3 & = Z_1 I_1 + (Z_1 + Z_2) I_2 + (Z_1 + Z_2 + Z_3) I_3
\end{array}
\]
%%
Vi ser då att störningen blir
%%
\[
\begin{array}{ll}
\Delta U_1 & =  Z_1 I_2 + Z_1 I_3 \\
\Delta U_2 & = Z_1 I_1 + (Z_1 + Z_2) I_3 \\
\Delta U_3 & = Z_1 I_1 + (Z_1 + Z_2) I_2
\end{array}
\]
%%
Vilket är ett tydligt exempel på hur strömmarna stör varandras spänningar och
därmed har avsaknad av isolation.

Fördelen med seriekopplad jord är förstås att man får flera korta anslutningar
men däremot kommer summeringen av de olika strömmarna göra att man får dålig
isolation mellan de olika jordströmmarna och hur nollpotentialen upplevs.

\subsection{Parallellkoppling av jord}
\index{stjärnjordning}
\index{jord!stjärn-}
\index{skyddsjord}
\index{nolla}

Om vi istället ansluter våra tre laster med individuella ledare till jord
kommer de olika strömmarna inte att samverka, detta är en parallellkoppling
av jord~\cite[kap 3]{ott1988}, se bild~\ssaref{fig:kap4-2}.
Vi har därmed åstadkommit en isolation mellan strömmarna med avseende på
jordanslutningen.

\smalltikz{
    \begin{circuitikz}[american voltages]
      % Ground reference
      \draw (3,0) node[ground]{};
      \draw (1,0) to (5,0);
      % Source 1 ground
      \draw (1,0) to [R, l^=$Z_1$] (1,2);
      \draw (1,3) to [short, i^=$I_1$] (1,2);
      % Source 2 ground
      \draw (3,0) to [R, l^=$Z_2$] (3,2);
      \draw (3,3) to [short, i^=$I_2$] (3,2);
      % Source 3 ground
      \draw (5,0) to [R, l^=$Z_3$] (5,2);
      \draw (5,3) to [short, i^=$I_3$] (5,2);
    \end{circuitikz}
}{Parallellkopplat jordsystem}{fig:kap4-2}

% \noindent
Dock kommer varje strömkälla uppleva en förskjutning i spänningen av
sin jord som beror på dess egen ström och impedansen den har till jord.
För att minska denna effekt kan en minskad strömförbrukning användas
eller oftare en förbättrad jordanslutning.

Givetvis kan även varje strömförbrukare ha två jordar, parallellt.
Elkraftsystemens användning av både \emph{skyddsjord} och \emph{nolla} är
just ett sådant system, där nollan är den som har strömmen och tillåts få
åka runt i spänning, medan skyddsjorden i allmänhet enbart har små strömmar.
Skyddsjordens funktion är också att kunna hantera stora strömmar vid fel,
för att kunna bryta tillförseln.
Skyddsjorden har egentligen det som sitt huvudsyfte, men ger ofta en bra
jordreferens.

I apparater och även inne på kretskort kan man ha parallellkoppling.
Det är även känt som \emph{stjärnjordning} (eng. \emph{star grounding})
eftersom kopplingsschemat ser ut att ha en stjärna från en gemensam punkt.
Det kan vara nyttigt att isolera jord för analoga signaler från digitala eller
rent av reläer, PA med mera.
Man försöker sätta stjärnan direkt vid anslutningen till kraftaggregatet för
att hålla dem så gemensamt som möjligt men med så lite påverkan av
seriejordning som möjligt.
Samma teknik används ofta för själva kraftdistributionen av samma skäl.

\subsection{Sammankoppling av apparater}
\label{sammankopplingavapparater}
\index{jordbrum}
\index{jord!brum}

I ett system där man har gjort parallella jordar i matningen,
bild~\ssaref{fig:kap4-3}, vill man nu koppla samman två apparater för att
överföra en signal.
En första naiv lösning är ju att helt enkelt bara dra en tråd \(Z_{signal}\)
från den ena apparaten över till den andra.
Eftersom de har jordanslutning så har de ju en gemensam jordreferens.

\smalltikz{
    \begin{circuitikz}[american voltages]
      % Ground reference
      \draw (2,0) node[ground]{};
      \draw (1,0) to (3,0);
      % Source ground
      \draw (1,2) to [R, l_=$Z_1$, v^=$U_1$] (1,0);
      \draw (0,2) to [short, i^=$I_1$] (1,2);
      % Source output
      \draw (1,4) to [american voltage source, l_=$U_{ut}$] (1,2);
      % Interconnect and load
      \draw (1,4) to [R, l^=$Z_{signal}$] (3,4)
      to [R, l_=$Z_{load}$, v^=$U_{in}$] (3,2);
      % Destination ground
      \draw (3,2) to [R, l_=$Z_2$, v^=$U_2$] (3,0);
      \draw (4,2) to [short, i_=$I_2$] (3,2);
    \end{circuitikz}
}{Sammankopplat system}{fig:kap4-3}

% \noindent
Problemet är att när strömmen \(I_1\) till den första apparaten går igenom
anslutningsimpedansen \(Z_1\) till jord så ger det en spänning
\(U_1 = Z_1 I_1\) på den jordanslutningen.
På samma sätt kommer den andra apparaten att uppleva jorden med en förskjutning
av jordspänningen på \(U_2 = Z_2 I_2\).
Om den tänka utspänningen är \(U_{ut}\) så kommer den egentliga utspänningen
vara \(U_{ut} + U_1\) i förhållande till jord.
Om vi för stunden antar att det inte går någon anmärkningsvärd ström i ledaren
över till den andra apparaten så kommer den uppleva det som en inspänning
\(U_{in}\) i förhållande till sin jordpotential \(U_2\) det vill säga
\(U_{in} = U_{ut} + U_1 - U_2\).

Vi ser här att skillnaden i jordpotential kommer förskjuta den upplevda
inspänningen \(U_{in}\) från den avsedda spänningen \(U_{ut}\) med skillnaden i
jordpotential, det vill säga \(U_1 - U_2\) som i sin tur beror på
anslutningarnas impedans och strömmarna.
Överföringen kan därför ha problem med sin isolation av \(I_1\) och \(I_2\)
till \(U_{in}\).

Om de bägge strömmarna inte har någon starkt frekvensinnehåll för det
frekvensband som man observerar på mottagaren, så fungerar dock detta fint.
Inte helt sällan råkar dock isolationen bli ett bekymmer antingen direkt eller
genom att det stör funktionen indirekt.

Ett försök att minska störningen är förstås att försöka minska \(Z_1\) och
\(Z_2\) genom att göra motståndet mindre, till exempel genom kortare kablar
eller grövre kablar.
Detta fungerar givetvis, men enbart till en viss praktisk gräns.

Det här illustrerar grunden i hur \emph{jordbrum} (eng. \emph{hum}) brukar
uppstå när man kopplar ihop två apparater.
Själva jordbrummet kommer från kraftaggregaten och då deras strömmar delar
krets med nyttosignalen så kommer \emph{överhörning} (eng. \emph{crosstalk})
göra att brummet blir märkbart.
Det finns givetvis många vägar för brum att störa en signal.

\subsection{Isolerad jordning}
\index{isolerad jordning}
\index{jordning!isolerad}
\index{IBN}
\index{signaljord}
\index{flytande}
\index{jord!flytande}
\index{jordbrum}
\index{jord!brum}
\index{chassijordning}
\index{jord!chassi}
\index{ledningsbunden störning}

En strategi för att skapa isolation från jordvägen är att helt enkelt
isolera signalerna och deras jord från kraftförsörjningens jord, detta kallas
för \emph{isolerad jordning} (eng. \emph{isolated bonding} även \emph{isolated
 bonding network, IBN})~\cite[kap 3.2.4]{K27-1991}.
Man börjar plötsligt prata om \emph{skyddsjord} skilt från \emph{signaljord}
(eng. \emph{signal ground}).

\smalltikz{
    \begin{circuitikz}[american voltages]
      % Ground reference
      \draw (3.5,0) node[ground]{};
      \draw (1,0) to (6,0);
      % Source ground
      \draw (1,0) to [R, l^=$Z_1$] (1,2);
      \draw (0,2) to [short, i^=$I_1$] (1,2);
      % Source output
      \draw (1,4) to [american voltage source, l^=$U_{ut}$] (1,2);
      % Interconnect and load
      \draw (1,4) to [R, l^=$Z_{signal}$] (4,4)
      to [R, l^=$Z_{load}$, v_=$U_{in}$] (4,2);
      \draw (1,2) to [R, l^=$Z_{GND}$] (4,2);
      % Destination isolation
      \draw (4,2) to [R, l^=$Z_{iso2}$, v_=$U_5$] (6,2);
      % Destination ground
      \draw (6,0) to [R, l^=$Z_2$] (6,2);
      \draw (7,2) to [short, i_=$I_2$] (6,2);
    \end{circuitikz}
}{Sammankopplat system med utjämningsledare}{fig:kap4-4}

\noindent
I apparater med växelströmsmatning har man redan en transformator som
tillhandahåller en galvanisk isolation mellan primärsidan (elkraft) och
sekundärsidan (elektroniken).
Genom att helt enkelt hålla signaljorden \emph{flytande} (eng.
\emph{floating}), det vill säga utan någon galvanisk koppling till skyddsjord,
så kan man istället koppla samman signaljord på två apparater med separata
ledare \(Z_{GND}\).
I bild~\ssaref{fig:kap4-4} är isolationen hos mottagande apparat representerad
av \(Z_{iso2}\) där spänningen \(U_5\) representerar spänningen mellan primär
och sekundärsida.
På liknande sätt kan isolationen på den sändande apparatens sida moduleras som
 \(Z_{iso1}\), men för detta resonemang räcker \(Z_{iso2}\).

Den galvaniska åtskillnaden gör att isolationen för likström kan variera från
megaohm till gigaohm, men på grund av den kapacitiva kopplingen mellan primär
och sekundär sida i transformatorn sjunker isolationen med stigande frekvens.
I praktiken kan transformatorn på grund av sin obalans driva spänningen \(U_5\)
och därför så kan man behöva lasta dess kapacitiva källan med ett motstånd,
varvid \(Z_{iso2}\) snarare kan vara i kiloohm.

Om vi återgår till de bägge två apparaterna, så kan vi nu istället för att
använda oss av elnätets skyddsjord låta apparaternas signaljord vara kopplad
med en kabel \(Z_{GND}\) parallell med signalledaren \(Z_{signal}\).
Har vi en förhållandevis låg ström genom den impedans som kabeln har så
kommer det fungera fint och \(U_{ut}\) kommer att representeras hyfsat bra som
spänningen \(U_{in}\) över \(Z_{load}\).

Eftersom \(Z_{GND}\) kan vara några fåtal ohm medan \(Z_{iso2}\) för låga
frekvenser är i storleksordningen megaohm så kommer kabeln att koppla väl.
För högre frekvenser kan vi förvänta oss att induktansen i kabeln ökar
impedansen \(Z_{GND}\) samtidigt som kapacitansen gör att impedansen
\(Z_{iso2}\) sjunker varvid för högre frekvenser kommer \(Z_{load}\) vara mer
kopplad lokalt mot \(Z_{2}\) snarare än \(Z_{1}\).

Det här scenariot liknar till exempel det hos en normal hemmastereo och ändå kan
det uppstå \emph{jordbrum} i denna koppling.
Det finns flera skäl.
Ett skäl är att transformatorer visserligen erbjuder en galvanisk isolation,
men de är även kapacitiva spänningsdelare för den spänning som finns över
primärlindningen, med 230~VAC spänning så behövs bara lite läckage över för att
man ska uppleva att isolationen brister.
Det brukar vara rekommendabelt att helt enkelt lasta denna spänningsdelare med
ett motstånd, så att signaljord och skyddsjord sitter ihop med ett någorlunda
högt motstånd, ofta med en kondensator parallellt, för att se till att reducera
det bidraget utan att få för mycket störningar från den ström som kommer flyta
mellan jordarna.

Ett annat scenario som skapar jordbrum är när man i någon ände råkar hårt
koppla samman signaljord och skyddsjord, typiskt att det blir oavsiktlig
kontakt mot chassi, som ska vara skyddsjordad.
Själva chassit brukar man prata om som \emph{chassijordat}, men det är
egentligen bara skyddsjord på de flesta system.

För att isolationsjordning ska fungera måste alla kontakter vara isolerade från
chassit.
Detta gäller även signaljord som inte får ha kontakt med chassi inuti apparaten.
Man behöver alltså försäkra sig om bra isolationsavstånd, vilket väldigt lätt
kan missas av att man har en skruv som råkar skrapa sig igenom skyddslack till
exempel.

En annan nackdel med isolationsjordning är att den gör det svårare att designa
för god EMC-täthet.
För \emph{ledningsbunden störning} (eng. \emph{conductive emission}) så vill
man helst att kontaktens och kabelns skärm sitter i chassijorden med så låg
impedans (induktans) som möjligt.
Isolationsjordning kräver då att man monterar kondensatorer som kopplar ihop
ledarens jord med chassijord och helst runt om för att få lägsta induktans.

Isolationsjordning rekommenderas inte för större system, då den blir svår
att upprätthålla.

Det förekommer att man för att minska störningarna i ett isolationsjordat
system väljer att koppla bort skyddsjorden, för att på det sättet ha mindre
störningar.
Detta är oftast inte tillåtet göra då man normalt inte bryter mot
elsäkerhetsregler och anläggningen riskerar bli farlig, då personskyddet
sätts ur spel.

\begin{center}
\begin{minipage}{0.19\columnwidth}
\Huge{\warningsymbol}
\end{minipage}
\begin{minipage}{0.7\columnwidth}
Varje gång som skyddsjorden kopplas bort för att lösa ett problem så
har man skaffat sig ett större problem, nämligen signifikant sänkt
elsäkerhet, vilket indikerar att man valt en felaktig lösning.
\end{minipage}
\end{center}

\subsection{Sammankopplad jordning}
\index{sammankopplad jordning}
\index{jord!sammankopplad}
\index{jordloop}
\index{jord!loop}
\index{vagabonderande jordström}
\index{jordbrum}
\index{jord!brum}
\index{chassijord}
\index{skyddsjord}

En annan strategi är \emph{sammankopplad jordning} (eng. \emph{mesh bonding}
och \emph{mesh bonding network, mesh-BN})~\cite[kap 3.2.3]{K27-1991}
där man istället för att isolera satsar på att koppla samman jordarna, hårt.
Varje signalkabel sitter ansluten mot \emph{chassijord} och därmed
\emph{skyddsjord} och man låter därmed jordarna sammankopplas.
Varje apparat har en ordentlig jordanslutning som man ansluter till stativjord
eller jordskenor.
Kablar läggs på kabelstegar som jordas.
I detta system kommer varje extra kabel att koppla samman jordarna hårdare,
eftersom man parallellkopplar många impedanser.
Denna strategi väljs ofta i telekommunikationssystem.

%% k7per: Varifrån ska denna figur refereras?
\smalltikz{
  \begin{circuitikz}[american voltages]
    % Ground reference
    \draw (2.5,0) node[ground]{};
    \draw (1,0) to (4,0);
    % Source ground
    \draw (1,0) to [R, l^=$Z_1$] (1,2);
    \draw (0,2) to [short, i^=$I_1$] (1,2);
    % Source output
    \draw (1,4) to [american voltage source, l^=$U_{ut}$] (1,2);
    % Interconnection
    \draw (1,4) to [R, l^=$Z_{signal}$] (4,4)
    to [R, l^=$Z_{load}$, v_=$U_{in}$] (4,2);
    \draw (1,2) to [R, l^=$Z_{GND}$, v_=$U_{GND}$] (4,2);
    % Destination ground
    \draw (4,0) to [R, l^=$Z_2$] (4,2);
    \draw (5,2) to [short, i_=$I_2$] (4,2);
  \end{circuitikz}
}{Sammankopplat system med utjämningsledare}{fig:kap4-5}

\noindent
I ett system som har sammankopplad jord kommer man ofrånkomligen att behöva
hantera vad man kallar för \emph{jordloop} (eng. \emph{ground loop}) eller även
\emph{vagabonderande jordströmmar}.
Många gånger förklaras det som att man får en loop som agerar antenn för ett
magnetfält.
Det är dock sällan som ett magnetfält är så starkt att det inducerar flera
ampere av vanlig \qty{50}{\hertz} ström.

Om vi går tillbaka till sammankoppling av apparater (kapitel
\ssaref{sammankopplingavapparater}) där vi fick en skillnad av spänning \(U_{GND}\)
mellan jordpunkterna så kommer vi ha den även här, men nu ansluter vi ju en
ledning \(Z_{GND}\) mellan dessa punkter, och då kommer det gå en ström som
försöker utjämna potentialen mellan de bägge jordanslutningarna, som då kommer
närmare varandra.
Det är impedansen \(Z_{GND}\) på kabeln som kommer att avgöra hur stor strömmen
blir och hur nära de kommer varandra spänningsmässigt.
Denna ström kan bli ansenlig och har man då en kabel som har till exempel tunn
skärm så kommer kabeln helt enkelt bli varm.
Det är därför lämpligt att lägga en jordkabel parallellt med signalkabeln, för
att låta den med sin större tvärsnittsarea ta merparten av strömmen och därmed
undviker man värme och ström i signalkabeln.

Med en större kabel mellan kommer spänningen sjunka och den vägen kommer
\emph{jordbrummet} minska.

Fördelen med sammankoppling av jordar är att det blir enklare (och billigare)
att designa ur EMC-perspektiv, då man direkt kopplar jordströmmarna i chassit.
Man har inte heller problem med att man skulle råka jorda eller att man skulle
tappa den enda jordvägen.
Istället försöker man koppla ihop jordarna väl.

Ett vanligt problem är om man låter jordströmmarna gå genom kretskort, vilket
gör att man skapar lokala problem med seriejordning.
Man ska se till att jordströmmarna knyter hårt till chassit, men svagt genom
kortet för att på det sättet få bästa möjliga isolation.
Denna princip är också lämplig för att kunna hantera till exempel ESD-skador.

En annan fördel är att man bygger en vana att jorda allt, och för varje
kompletterande jordning gör man systemet starkare.

En självklar fördel är att man dessutom inte bryter skyddsjord, och därmed inte
sänker elsäkerheten på utrustningen och installationen.

\subsection{Balanserad signal}
\index{balanserad signal}
\index{jordbrum}
\index{jord!brum}
\index{galvanisk isolation}
\index{gemensam spänning}
\index{differentiell spänningen}

För att ytterligare få isolation från jordbrum kan man använda en
\emph{balanserad signal} (eng. \emph{balanced signal}).
Grundprincipen är att man skickar samma signal två gånger, men med omvänt
tecken, och sedan ta emot den och bara titta på skillnaden mellan dem.
Skulle nu en störning introducera sig på dessa ledare gemensamt så påverkar
detta inte skillnaden i spänning mellan dem.

\begin{figure}
  \begin{center}
    \begin{circuitikz}[american voltages]
      % Ground reference
      \draw (4,1) node[ground]{};
      \draw (1,1) to (7,1);
      % Source ground
      \draw (1,1) to [R, l^=$Z_1$] (1,4);
      \draw (0,4) to [short, i^=$I_1$] (1,4) to (2,4);
      % Source diff
      \draw (2,6) to [american voltage source, l^=$U_{ut}$] (2,4)
      to [american voltage source, l^=$U_{ut}$] (2,2);
      % Wires and load
      \draw (2,6) to [R, l^=$Z_{signal+}$] (5,6)
      to [R, l^=$Z_{load}/2$, v_=$U_{in+}$] (5,4)
      to [R, l^=$Z_{load}/2$, v_=$U_{in-}$] (5,2);
      \draw (2,2) to [R, l^=$Z_{signal-}$] (5,2);
      \draw (2,4) to [R, l^=$Z_{GND}$] (5,4);
      % Destination isolation
      \draw (5,4) to [R, l^=$Z_{iso2}$, v_=$U_5$] (7,4);
      % Destination ground
      \draw (7,1) to [R, l^=$Z_2$] (7,4);
      \draw (8,4) to [short, i_=$I_2$] (7,4);
    \end{circuitikz}
  \end{center}
  \caption{Sammankopplat system med utjämningsledare och differentiell signal}
  \label{fig:kap4-6}
\end{figure}

Redan tidigare har vi gjort liknande och försökt efterlikna egenskaperna, för
redan när vi skickade en signal på en enkel ledare så skickar vi en spänning
i förhållande till en referensspänning och vi tittar på den inkommande
spänningen i förhållande till referensspänningen.
Dock har vi haft problem att ha en bra gemensam sådan, och det är uppenbart
att vi egentligen observerar skillnaden i spänning.

Med balanserad signal tar vi steget fullt ut och separerar nollreferens från
signal och skickar en signal som vars summa är en fix spänning medan
skillnaden är nyttosignalen.
Det är som om signalen är neutral.
Ofta är dock signalen av praktiska skäl förskjuten spänningsmässigt.

Den balanserade signalen har jord, \emph{pluspol} och \emph{minuspol}.
\emph{Pluspolen} kallas även +, \emph{positiv polaritet}, \emph{het} (eng.
\emph{positive pole}, \emph{positive polarity} och \emph{hot}) medan
\emph{minuspolen} kallas även -, \emph{negativ polaritet}, \emph{kall} (eng.
\emph{negative pole}, \emph{negative polarity} och \emph{cold}).
Utöver dessa har man oftast en \emph{spänningsreferens} som ofta betäcknas som
\emph{jord} (eng. \emph{ground, GND}) eller \emph{nolla} (eng.
\emph{neutral}).

I bild~\ssaref{fig:kap4-6} visas hur ut-spänningen \(U_{ut}\) är dubblerad och
matar på var sin sida om jordpotentialen som  \(I_{1}\) och \(Z_{1}\) ger.
De bägge utspänningarna är kopplade över var sin ledare \(Z_{singal+}\) och
\(Z_{singal-}\) för att över var sin \(Z_{load}/2\) resultera i \(U_{in+}\)
respektive \(U_{in-}\), som i sin tur sitter mot signaljorden på samma sätt
som tidigare.
Den egentliga in-spänningen är från \(U_{+}\) till \(U_{-}\) det vill säga
\(U_{in} = U_{+} - U_{-} = U_{in+}+U_{in-}\)

Transformatorer passar väl för att både generera och ta emot balanserade
signaler, då de har en \emph{galvanisk isolation} för \emph{gemensam spänning}
men transformerar den \emph{differentiella spänningen}.
Detta kan även göras med aktiv elektronik så som op-ampar men även färdiga
kretsar finns.

Transformatorer har fördelen att man kan få den galvaniska skillnaden genom
att helt enkelt bryta jordförbindelsen på ledaren.
Dock, transformatorer har inte fulländad isolation men kan däremot ofta hantera
ganska stora spänningar, vilket kan krävas i besvärliga sammanhang.
För RF är dock transformatorer inte balanserade och ger dålig isolation.
Förbättrad isolation hos transformatorer kan uppnås med ett eller två
skärmlager mellan lindningarna.
Skärmlagren kan anslutas till respektive sidas jord.
För RF krävs dock en strömbalun/RF-choke för att undertrycka den
gemensamma strömmen.

Aktiv elektronik för balansering har sällan galvanisk isolation, men däremot
kan man upprätthålla hög impedans för den gemensamma spänningen, vilket kan
vara nog så tillräckligt.

Differentiell signal i RF kan uppnås genom att använda en RF-choke som
undertrycker den gemensamma spänningen i RF men inte i likspänning.

\section[Gemensam och diff]{Gemensam och differentiell spänning och ström}

När man har ett treledarsystem som vi har med differentiell matning eller
även om man bara har två ledare men mellan system som har gemensam jord
(gäller också om de bara har RF-koppling en annan väg) så kan man betrakta
de två signalledarna antingen som att de har sin individuella spänning och
ström, eller som att de har gemensam och differentiell spänning och ström.

\subsection{Gemensam och differentiell spänning}
\label{comdiffv}

Gemensam spänning och differentiell spänning är ett alternativt sätt att
betrakta spänning på de bägge ledarna, där man delar upp spänningen i det som
är gemensamt för de bägge spänningarna och det som skiljer dem åt. Man kan
alltså betrakta dem på detta alternativa och oberoende (ortogonala) sättet.

\begin{figure}
  \begin{center}
    \begin{circuitikz}[american voltages]
      % Ground reference
      \draw (3.5,1) node[ground]{};
      \draw (0,1) to (7,1);
      % Source ground
      \draw (0,4) to [american voltage source, l^=$U_{g}$] (0,1);
      \draw (0,4) to (2,4);
      % Source diff
      \draw (2,6) node[anchor=east] {$V_{ut+}$} to [american voltage source, l^=$U_{d}/2$] (2,4)
      to [american voltage source, l^=$U_{d}/2$] (2,2) node[anchor=east] {$V_{ut-}$};
      % Wires and load
      \draw (2,6) to [R, l^=$Z_{signal+}$] (5,6) node[anchor=west] {$V_{in+}$}
      to [R, l^=$Z_{d}/2$, v_=$U_{d+}$] (5,4)
      to [R, l^=$Z_{d}/2$, v_=$U_{d-}$] (5,2) node[anchor=west] {$V_{in-}$};
      \draw (2,2) to [R, l^=$Z_{signal-}$] (5,2);
      \draw (2,4) to [R, l^=$Z_{GND}$] (5,4);
      % Destination isolation
      \draw (5,4) to (7,4);
      % Destination ground
      \draw (7,4) to [R, l^=$Z_{g}$, v_=$U_{g}$] (7,1);
    \end{circuitikz}
  \end{center}
  \caption{Sammankopplat system med utjämningsledare och differentiell signal}
  \label{fig:kap4-7}
\end{figure}

I bild~\ssaref{fig:kap4-7} har man den gemensamma spänningskällan \(U_g\), som
från ersatt de förskjutna jordpunkterna i tidigare exempel.
Den differentiella spänningen \(U_d\), det vill säga den drivande spänningen
mellan \(V_{ut+}\) och  \(V_{ut-}\) är fördelad på två spänningskällor som
levererar halva spänningen var.
%%
\[V_{ut+} = U_g + \dfrac{U_d}{2}\]
\[V_{ut-} = U_g - \dfrac{U_d}{2}\]
%%
Omvänt kan man formulera uttrycken för gemensam spänningen \(U_g\) samt
den differentiella spänningen \(U_d\) som \(V_{ut+}\) och \(V_{ut-}\):
%%
\[U_g = \dfrac{V_{ut+}+V_{ut-}}{2}\]
\[U_d = {V_{ut+}-V_{ut-}}\]
%%
På motsvarande sätt på ingången kan man skriva uttrycken för den gemensamma
mottagna spänningen \(U_{g,in}\) och den mottagna differentiella spänningen
\(U_{d,in}\) baserat på inspänningarna \(V_{in+}\) och \(V_{in-}\), man får då
%%
\[V_{g,in} = \dfrac{V_{in+} + V_{in-}}{2}\]
\[V_{in+} = V_{in+}-V_{in-}\]
%%
Ett sätt att illustrera skillnaden är till exempel med en transformator.
En transformator med 1:1 lindning kopplas in mellan två balanserade signaler.
Transformatorns primärlindning kommer att omvandla den differentiella spänningen
\(V_d\) till en motsvarande spänning på utgången.
Däremot kommer den gemensamma spänningen inte att överföras.
Transformatorn blir då en isolator för den gemensamma spänningen precis som vi
förväntar oss av en galvanisk isolation.

Isolationen för den gemensamma spänningen i en transformator är dock främst ett
likströmsbeteende, så ju högre frekvens desto bättre koppling, det vill säga
sämre isolation.
Detta beror på den kapacitiva kopplingen mellan lindningarna som skapar en
ström, som sammankopplar sidorna och resulterar i att den gemensamma spänningen
ändå går igenom transformatorn.
För högre frekvenser är kopplingen väldigt god och transformatorn gör ingen
nytta för att undertrycka den gemensamma spänningen.

Eftersom nyttosignalen är differentiell kan man ibland medvetet använda den
gemensamma spänningen för att överföra matningsspänning till till exempel en
mikrofon.
Denna form av matningsspänning kallas för \emph{fantommatning}
(eng. \emph{phantom power}).
En vanligt förekommande spänning är \qty{48}{\volt}, som då symboliseras med P48.
Det förekommer även på modern Ethernet-utrustning och kallas då för
\emph{Power over Ethernet (PoE)}.

\subsection{Gemensam och differentiell ström}
\label{comdiffi}
\index{RF-choke}
\index{strömbalun}
\index{current balun}

Precis som för spänning kan man beskriva strömmarna i samma ledare som
gemensam och differentiell ström.
Vi kan därför återanvända formlerna och bara byta ut V mot I genomgående och
får då:
%%
\begin{eqnarray*}
I_+ = & I_g + I_d\\
I_- = & I_g - I_d\\
I_g = & \dfrac{I_+ + I_-}{2}\\
I_d = & \dfrac{I_+ - I_-}{2}
\end{eqnarray*}
%%
Om vi återgår till transformatorexemplet så kommer det vara den differentiella
strömmen på primärlindningen som ger upphov till magnetfältet i transformatorn
och som sedan inducerar en differentiell ström i sekundärlindningen.

Isolationen mellan lindningarna förhindrar att det går en ström mellan dem,
och därför förhindras den gemensamma strömmen vid låga frekvenser.
Vid högre frekvenser kommer dock den kapacitiva kopplingen mellan de två
sidorna att ske varvid en gemensam ström kommer uppstå för högre frekvenser,
det vill säga för högre frekvenser kommer isolationen att bli sämre.

Ett intressant specialfall är om vi sätter en ringkärna på vår kabel, lindar
kabeln flera varv genom den, eller bara lindar den runt luft.
Då kommer strömmen i den ena ledaren inducera en ström i den andra ledaren och
vice versa.
Denna koppling kan liknas vid att vrida en 1:1 transformator 90~grader fel.
Eftersom den inducerade strömmen har motsatt riktning så kommer den motverka
den gemensamma strömmen, men inte den differentiella strömmen.
Dessutom kommer denna koppling bli starkare för högre frekvenser (i den fina
teorin) och därmed skapa en högre isolation för gemensam ström.
Detta kallas för bland annat \emph{RF-choke} (eng. \emph{RF-choke}) och
\emph{strömbalun} (eng. \emph{current balun}).
Den kompletterar isolationen hos en transformator eller löser den nödvändiga
isolationen helt på egen hand.

RF-choke är ett oerhört användbart verktyg för att undertrycka RF-strålning
och det man ofta i EMC sammanhang kallar ledningsbunden strålning, som är en
gemensam ström ut på ledarna.
Att det är den gemensamma strömmen förstås lätt eftersom den differentiella
strömmen från de bägge ledarna kommer att motverka varandra i utstrålat
magnetfält medan den gemensamma strömmen samverkar och därför är det enbart
den som ger ett utstrålat magnetfält.

Det är därför man ofta hittar klumpar som sitter på kablar till till exempel skärmar.
Dessa klumpar är helt enkelt en ringkärna som förstärker kopplingen mellan
ledarna för att undertrycka den gemensamma strömmen för RF och därmed minska
störningen.

\subsection{Generell gemensam och differentiell analys}
\label{comdiffgeneric}
\index{mod!gemensam}
\index{gemensam strömöverföring}
\index{mod!differentiell}
\index{differentiell strömöverföring}
\index{Common Mode (CM)}
\index{CM}
\index{Differential Mode (DM)}
\index{DM}

Efter att ha studerat gemensam och differentiell spänning (kapitel
\ssaref{comdiffv}) och gemensam och differentiell ström (kapitel~\ssaref{comdiffi})
kan vi sammanfattningsvis konstatera att den grundläggande metoden att omvandla
de individuella spänningarna och strömmarna till \emph{gemensam överföring}
(eng. \emph{Common Mode, CM}) och \emph{differentiell överföring}
(eng. \emph{Differential Mode, DM}) är en kraftfull metod både för att
förstå och avhjälpa problem och uppnå isolation.
För spänning har vi ekvationerna
%%
\begin{eqnarray*}
V_+ = & V_{CM} + V_{DM}\\
V_- = & V_{CM} - V_{DM}\\
V_{CM} = & \dfrac{V_+ + V_-}{2}\\
V_{DM} = & \dfrac{V_+ - V_-}{2}
\end{eqnarray*}
%%
För ström har vi ekvationerna
%%
\begin{eqnarray*}
I_+ = & I_{CM} + I_{DM}\\
I_- = & I_{CM} - I_{DM}\\
I_{CM} = & \dfrac{I_+ + I_-}{2}\\
I_{DM} = & \dfrac{I_+ - I_-}{2}
\end{eqnarray*}

\subsection{Gemensam och differentiell impedans}

Precis som man har impedans på ingångar så har man det på ingångar i
treledarsystem.
Det som är den normala impedansen för en transmissionsledare till exempel är
egentligen den differentiella impedansen, det vill säga förhållande mellan den
differentiella spänningen och differentiella strömmen.
Den gemensamma impedansen är på samma sätt förhållandet mellan gemensam
spänning och gemensam ström
%%
\begin{eqnarray*}
Z_{DM} = & \dfrac{U_{DM}}{I_{DM}}\\
Z_{CM} = & \dfrac{U_{CM}}{I_{CM}}
\end{eqnarray*}
%%
Egentligen är det inte så konstigt, om man har en koaxialkabel i ett 50~ohm
system så har sändare och mottagare idealt 50~ohm som differentiell impedans.
I ett system som har isolerad jordning så kan den gemensamma impedansen vara
många megaohm eller högre, eftersom den är isolerad.

\subsection{Obalans}
\index{strömbalun}
\index{obalans}

Så här långt har huvudsakligen antagit att vi har balans, det vill säga att
transformatorer, induktorer med mera är ideala och ger lika bra koppling till
bägge sidor.
Givetvis finns inte detta i verkligheten, och man har en obalans.
Vid obalans får man en signal som är gemensam att läcka över till den som är
differentiell och omvänt att differentiell läcker över till den gemensamma.
Det resulterar dels i minskad isolation och dels i minskad signal.
I allmänhet är den minskade isolationen värre än förlusten av signal, som i
allmänhet är försumbar.

I en transformator ligger lindningarna ofta så att den kapacitiva kopplingen
från ena polen på en spole är starkare än från den andra polen.
Det ger därför en obalans i hur de kopplar kapacitivt.
Genom att lägga ett skärmlager mellan lindningarna kan den kapacitiva
kopplingen jämnas ut, då de kopplar kapacitivt till skärmlagret istället,
som kan lågresistivt hindra koppling.
En ännu bättre lösning är att ha dubbla lager med isolation, för då
kan de kopplas mot respektive sidas jord, och kvar blir bara den kapacitiva
kopplingen mellan jordarna, som oftast är ett mindre problem.
Med dessa metoder fås bättre isolation än vad en oskärmad transformator kan
erbjuda, på grund av just obalans.

Den kapacitiva kopplingen har väldigt hög impedans vid \qty{50}{\hertz}, så man
kan använda relativt höga motståndsvärden för att lasta ned den hårt.
Fördelen är att man kan undvika direkt koppling, vilket kan skapa andra
problem som när man vill ha relativ isolation galvaniskt.

I en strömbalun kan den ena ledaren ha något lite längre varv runt kärnan än
den andra.
Det ger inte en perfekt 1:1 relation i kopplingen och därmed en obalans.

I en transformator med mitt-tapp kan mitt-tappen sitta lite förskjuten från
riktiga mitten, så att anslutningen av mitt-tappen till jord skapar en
obalans.

Dessa exempel på brister i konstruktionen ska man vara medveten om, så att man
inte tillskriver en transformator eller strömbalun att ha egenskaperna av en
perfekt isolation.
Snarare ska man förvänta sig att den inte är perfekt och anpassa sin design
efter det.
Många gånger kan en kombination av åtgärder ge fullgott resultat utan att vara
särdeles dyrt eller klumpigt, men det kräver eftertanke och helhetssyn.

Ett enkelt fall i ljudsammanhang är \qty{50}{\hertz} \qty{230}{\volt} men man
vill hålla störningen mindre än säg \qty{1}{\milli\volt}.
Det kräver mer än \qty{106}{\decibel} isolation mellan \qty{230}{\volt}
differentiellt på primärlindningen och \qty{1}{\milli\volt} gemensamt på
sekundärlindningen.
Så god balans kan vara svår att finna i enskilda komponenter.
Principen återkommer oavsett spänning och frekvens, det är en brist
man behöver lära sig att förstå och hantera.

\subsection{Obalans i antennsystem}
\index{obalans!antennsystem}
\index{mantelström}
\index{obalans!mantelström}
\index{balun}
\index{strömbalun}
\label{obalans_antennsystem}

Obalans kan även förekomma i antennsystem, där en obalanserad antenn omvandlar
den utsända signalen, som är differentiell, till att delvis bli gemensam.
Detta gör att via reflektion från den obalanserade antennen går en ström i
matningsledningen som gör att den strålar.

Detta har traditionellt uttryckts som att strömmen vänder och går på utsidan av
skärmen, men det som hänt är att den differentiella strömmen, som ju motverkar
utstrålning plötsligt får en pålagd gemensam komponent som då kommer stråla.
Man kan uppleva det om man berör ledningen så kan man känna denna som en ström,
vilket man upplever går på utsidan.
Kabeln har då blivit en strålande del av antennen, något som för vissa
antenntyper är en medveten design.

Det är också denna ström som behöver motverkas för att operatören inte ska
skada sig.
Detta görs med en strömbalun, lämpligtvis en kvartsvåg ned från anslutningen
till antennen.
Strömbalunen motverkar den gemensamma strömmen utan att nämnvärt påverka
den differentiella, så det är ett fint exempel på en bra åtgärd.

De allra flesta antenner har en annan impedans i matningspunkten än vad dess
matarledning har.
Detta kräver en impedansanpassning för optimal energiöverföring.
En annan aspekt är att för en koaxial matning så överförs energin enkelsidigt
(single-ended) det vill säga att det är mittledaren i förhållande till
skärm/jord som överför energi.
När vi ansluter denna ledare till en dipolantenn vill vi se till att strömmen
går balanserat ut i de bägge ledarna, så att mittpunkten är nära noll, så att
det inte går en ström med gemensam mod ut i matarledningen.

Vi har alltså dels behovet att omvandla obalanserad signal till balanserad
samt undertrycka gemensam signal i ledaren, och därtill impedanskonvertera den.
Detta brukar man låta en balun (balanced-unbalanced) göra, vilket som namnet
anger bara ger indikation på konverteringen, men den gör alltså flera saker.
Eftersom ingen balun är perfekt designad så kommer den i sig själv ha en
obalans, varvid den ändå kommer ge viss gemensam ström.
För högre effekter kan därför en separat spärr komma att behövas.

Utöver balun finns även unun (unbalan\-ced-un\-balan\-ced) som gör
impedanskonvertering enbart.

Även om man har en bra balun riskerar man att få mantelströmmar, ty antennen
kan vara av en obalanserad typ, till exempel Off-Center-Feed (OCF)/Windom, eller
för att den kopplar olika med miljön som träd och torn med mera.

Att undvika att det går gemensam ström, även kallad \emph{mantelström} kan
krävas av många olika anledningar, och det är viktigt dels för att få ut
energin där den ska, det vill säga radierat ut i luften på ett korrekt sätt,
men även av säkerhetsskäl så att inte utrustnings- eller personskada uppstår.
