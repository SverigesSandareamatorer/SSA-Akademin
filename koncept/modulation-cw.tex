\section{Sändningsslaget A1A (CW)}
\harecsection{\harec{a}{1.8.1}{1.8.1}, \harec{a}{1.8.6a}{1.8.6a}, \harec{a}{1.8.7a}{1.8.7a}}
\index{A1A}
\index{CW}
\label{modulation_cw}

Bild~\ssaref{fig:BildII1-26} visar amplitudmodulation med morsetecken.
Man kan överföra meddelanden med morsetelegrafi på olika sätt.
Det enklaste sättet är att koppla in och ur sändarens bärvåg i takt med
teckendelarna i morsetecknen.
Man kan kalla det för bärvågstelegrafi.
Förfarandet kallas sedan mycket länge även för CW (continous waves), vilket
egentligen anger att bärvågen svänger med konstant amplitud, om man bortser
från att den nycklas.
Detta står i motsats till de dämpade bärvågssvängningar som var fallet i sedan
mycket länge förbjudna gnistsändare.

Fastän en sändare ''moduleras utan ton'', har den en viss bandbredd.
Det beror på att den takt, som sändaren nycklas med, egentligen är en ton --
låt vara med låg frekvens.
Antag att sändaren nycklas med en serie korta morsetecken.
Vid telegraferingshastigheten 60~tecken/minut alstrar bärvågspulserna en kantvåg
med frekvensen \qty{5}{\hertz}.
Som tidigare beskrivits, består en sådan kantvåg av summan av sinussignaler med
frekvenserna \qty{5}{\hertz}, \qty{15}{\hertz}, \qty{25}{\hertz},
\qty{35}{\hertz} och så vidare.

Det innebär att det uppstår sidofrekvenser över och under bärvågens frekvens och
med ett avstånd till bärvågen av \qty{5}{\hertz}, \qty{15}{\hertz},
\qty{25}{\hertz}, \qty{35}{\hertz} osv.
Telegrafisändaren har alltså liksom vid A3E en bandbredd, som dels står i
förhållande till nycklingshastigheten och dels till ''kantigheten'' på tecknen,
vilket bestämmer övertonshalten i bärvågen.
Vid så kallad mjuk nyckling kan den 9:e övertonen antas vara den högsta som
uppfattas av en motstation.
Med en nycklingsfrekvens av \qty{5}{\hertz} blir bandbredden inte större än
\(2 \cdot 10 \cdot 5 = \qty{100}{\hertz}\).

En hård (kantig) och snabb teckengivning ökar bandbredden och kan resultera i
att så kallade nycklingsknäppar kan uppfattas långt vid sidan om
sändningsfrekvensen.
Ju hårdare nycklingen är, desto längre bort från bärvågsfrekvensen hörs
nycklingsknäpparna.
Detta stör andra stationer.

Kännetecken för sändningsslaget A1A, telegrafi genom nycklad bärvåg:

Mycket liten bandbredd, extremt gott utnyttjande av sändareffekten, stor
överföringssäkerhet, lång räckvidd, enkla sändare.
