\section{Q-koden}
\label{q-koden}
\textbf{
HAREC b.\ref{HAREC.b.2.1}\label{myHAREC.b.2.1},
 b.\ref{HAREC.b.2.2}\label{myHAREC.b.2.2},
 b.\ref{HAREC.b.2.3}\label{myHAREC.b.2.3},
 b.\ref{HAREC.b.2.4}\label{myHAREC.b.2.4},
 b.\ref{HAREC.b.2.5}\label{myHAREC.b.2.5},
 b.\ref{HAREC.b.2.6}\label{myHAREC.b.2.6},
 b.\ref{HAREC.b.2.7}\label{myHAREC.b.2.7},
 b.\ref{HAREC.b.2.8}\label{myHAREC.b.2.8},
 b.\ref{HAREC.b.2.9}\label{myHAREC.b.2.9},
 b.\ref{HAREC.b.2.10}\label{myHAREC.b.2.10},
 b.\ref{HAREC.b.2.11}\label{myHAREC.b.2.11},
 b.\ref{HAREC.b.2.12}\label{myHAREC.b.2.12},
 b.\ref{HAREC.b.2.13}\label{myHAREC.b.2.13},
 b.\ref{HAREC.b.2.14}\label{myHAREC.b.2.14},
 b.\ref{HAREC.b.2.15}\label{myHAREC.b.2.15},
 b.\ref{HAREC.b.2.16}\label{myHAREC.b.2.16},
 b.\ref{HAREC.b.2.17}\label{myHAREC.b.2.17},
 b.\ref{HAREC.b.2.18}\label{myHAREC.b.2.18},
 b.\ref{HAREC.b.2.19}\label{myHAREC.b.2.19},
 b.\ref{HAREC.b.2.20}\label{myHAREC.b.2.20},
 b.\ref{HAREC.b.2.21}\label{myHAREC.b.2.21},
 b.\ref{HAREC.b.2.22}\label{myHAREC.b.2.22},
 b.\ref{HAREC.b.2.23}\label{myHAREC.b.2.23},
 b.\ref{HAREC.b.2.24}\label{myHAREC.b.2.24},
 b.\ref{HAREC.b.2.25}\label{myHAREC.b.2.25},
 b.\ref{HAREC.b.2.26}\label{myHAREC.b.2.26},
 b.\ref{HAREC.b.2.27}\label{myHAREC.b.2.27},
 b.\ref{HAREC.b.2.28}\label{myHAREC.b.2.28}
}
\index{Q-kod}
\index{GMT}
\index{UTC}
\index{QRK}
\index{QRM}
\index{QRN}
\index{QRO}
\index{QRP}
\index{QRS}
\index{QRT}
\index{QRU}
\index{QRV}
\index{QRX}
\index{QRZ}
\index{QSA}
\index{QSB}
\index{QSL}
\index{QSO}
\index{QSY}
\index{QTC}
\index{QTH}
\index{QTR}

\subsection{Bakgrund}

Vid sändning med morsetelegrafi används sedan år 1912 internationella
''trafikförkortningar'' enligt Q-koden, både för att minska risken för
mottagningsfel på grund av språksvårigheter, störningar m.m. och för
att minska sändningstiden.
En trafikförkortning i form av Q-kod har en entydig innebörd, men kan anpassas
något till aktuell situation.
Varje Q-kod består av tre bokstäver i bokstavsserien QAA--QZZ
\cite[M.1172]{ITU-RR}.

I reglementesprovet för amatörradiocertifikat ingår frågor om Q-förkortningar.

I CEPT-rekommendation T/R 61-02 \cite[Annex 6]{TR6102} nämns följande allmänna
Q-förkortningar som berör amatörradio.
Radioamatörerna använder emellertid i praktiken fler Q-förkortningar varav
några räknas upp sist i tabellen.

En uppsättning Q-koder handlar om signalkvalite och signalstyrka, dessa är
QRK, QRM, QRN, QRO och QRP.
En uppsättning Q-koder handlar om interaktion med den andra stationen,
dessa är QRT, QRZ, QRV och QSB.
En uppsättning Q-koder handlar om utbyte av kontakt, dessa är QSL, QSO,
QSY, QRX och QTH.

\begin{table}
  \label{tab:q-kod}
  \caption{Q-koderna}
  \begin{tabular}{lp{6cm}p{6cm}}
    Q-kod & Fråga & Svar eller meddelande \\
    \hline
    QRK &
    Vilken uppfattbarhet har mina (eller \dots:s) signaler?
    &
    Uppfattbarheten hos dina (eller \dots:s) signaler är
    \vspace{-\topsep}
    \begin{enumerate}[noitemsep]
    \item dålig
    \item bristfällig
    \item ganska god
    \item god
    \item utmärkt.
    \end{enumerate} \\
    QRM &
    Är min sändning störd?
    &
    Störningarna på din sändning är
    \vspace{-\topsep}
    \begin{enumerate}[noitemsep]
    \item obefintliga
    \item svaga
    \item måttliga
    \item starka
    \item mycket starka.
    \end{enumerate} \\
    QRN
    &
    Besväras du av atmosfäriska störningar?
    &
    Atmosfäriska störningar är
    \vspace{-\topsep}
    \begin{enumerate}[noitemsep]
    \item obefintliga
    \item svaga
    \item måttliga
    \item starka
    \item mycket starka.
    \end{enumerate} \\
    QRO
    &
    Ska jag öka sändningseffekten?
    &
    Öka sändningseffekten.
    \\
    QRP
    &
    Ska jag minska sändningseffekten?
    &
    Minska sändningseffekten.
    \\
    QRT
    &
    Ska jag avbryta sändningen?
    &
    Avbryt sändningen.
    \\
    QRV
    &
    Är du redo?
    &
    Jag är redo.
    \\
    QRX
    &
    När anropar du mig igen?
    &
    Jag anropar dig igen kl \dots på \dots kHz/MHz.
    \\
    QRZ
    &
    Vem anropar mig?
    &
    Du anropas av \dots * (på \dots kHz/MHz).
    \\
    QSB
    &
    Varierar min signalstyrka?
    &
    Din signalstyrka varierar.
    \\
    QSL
    &
    Kan du ge mig kvittens?
    &
    Jag kvitterar.
    \\
    QSO
    &
    Kan få förbindelse med \dots * direkt?
    &
    Jag kan få förbindelse med \dots * direkt.
    \\
    QSY
    &
    Ska jag gå över till annan frekvens?
    &
    Gå över till annan frekvens.
    \\
    QTH
    &
    Vilket är ditt geografiska läge?
    &
    Mitt geografiska läge är \dots
    \\
    (QRS)
    &
    Ska jag minska sändningshastigheten?
    &
    Minska sändningshastigheten
    (sänd \dots ord i minuten).
    \\
    (QRU)
    &
    Har ni något till mig?
    &
    Jag har inget till dig.
    \\
    (QSA)
    &
    Vilken styrka har mina
    (eller: \dots *:s) signaler?
    &
    Dina (eller: \dots *:s) signaler är
    \vspace{-\topsep}
    \begin{enumerate}[noitemsep]
    	\item knappast uppfattbara
    	\item svaga
    	\item ganska starka
    	\item starka
    	\item mycket starka.
    \end{enumerate}
    \\
    (QTC)
    &
    Hur många telegram har du att sända?
    &
    Jag har telegram till Dig.
    \\
    (QTR)
    &
    Kan du ge mig rätt tid?
    &
    Rätt tid är \dots
    \\
  \end{tabular}
* namn och /eller anropssignal
\end{table}

Användning an Q-koder för maritimt bruk enligt ITU-R M.1172
\begin{enumerate}
	\item Vissa Q-koder kan ges jakande betydelse genom att bokstaven C
	(vid telefoni uttalad som CHARLIE) sänds omedelbart efter
	förkortningen eller ges nekande betydelse med det engelska ordet NO
	omedelbart efter förkortningen.
	\item Q-koder kan kompletteras med andra lämpliga förkortningar,
	anropssignaler, frekvenser, tidsuppgifter, person- och ortsnamn,
	siffror, nummer osv. I den beskrivande texten för vissa Q-koder
	lämnas inom en parentes plats för ytterligare uppgifter. Dessa
	uppgifter ska då sändas i den ordning som anges i texten.
	\item Q-koderna antar formen av fråga, då de vid radiotelegrafering
	åtföljs av frågetecken liksom då de vid radiotelefonering åtföljs av
	bokstäverna RQ (ROMEO QUEBEC). När kompletterande uppgifter följer
	efter en uttalad fråga, ska ett frågetecken respektive RQ följa
	efter uppgifterna.
	\item Q-koder med numrerade alternativa betydelser ska åtföljas av
	motsvarande siffra. Siffran ska sändas omedelbart efter
	förkortningen.
	\item I internationell radiotrafik ska, då ej annat anges,
	tidpunkter anges i Coordinated Universal Time (UTC), tidigare GMT.
\end{enumerate}
