\section{Digital modulation}
\harecsection{\harec{a}{1.8.8}{1.8.8}}
\index{digital modulation}
\label{modulation_digital}

Utöver de klassiska analoga modulationsmetoderna finns ett antal digitala
modulationsformer.
De är anpassade för transmission av binära data.
I viss mån kan CW ses som digital modulation där 0 moduleras utan bärvåg och 1
moduleras med bärvåg.
Det finns dock flera andra modulationsmetoder som FSK, 2-PSK/BPSK, 4-PSK och
QAM vilka presenteras i följande delavsnitt.

\subsection{Frekvensskiftsmodulation -- FSK}
\harecsection{\harec{a}{1.8.8a}{1.8.8a}}
\index{frekvensskiftsmodulation}
\index{Frequency Shift Keying (FSK)}
\index{FSK}
\index{frekvensmodulation}
\index{GFSK}
\index{Gaussian Frequency Shift Keying (GFSK)}
\index{Gaussiskt filter}
\index{C4FM}
\index{JT65}
\index{JT9}

\emph{Frekvensskiftsmodulation} (eng. \emph{Frequency Shift Keying, FSK})
skiljer sig från CW-modulationen genom att den ändrar frekvensen, dvs. är en
variant av frekvensmodulation.
I den enklaste formen, binär FSK växlar man mellan två frekvenser, där en
frekvens får representera 0 och den andra får representera 1.
Denna metod har används för modem på telefonförbindelser, såsom Bell~103.

Eftersom varje växling mellan frekvenser ger avbrott i bägge signalerna, likt
nycklingen i CW, så kommer de att skapa sidband.
Av det skälet filtrerar man gärna signalen, och använder man ett Gaussiskt
filter får man \emph{Gaussian Frequency Shift Keying (GFSK)} som används av till
exempel GSM-telefoni.

Man kan använda fler än två frekvenser, till exempel används fyra frekvenser i
Continuous 4 level FM (C4FM), i Phase 1 radios, i Project~25 samt Fusion.

Frekvensskift används även för att sända långsamma meddelanden där JT65
använder 65 frekvenser som den skiftar mellan, medan JT9 använder 9~frekvenser.

\subsection{Binär fasskiftsmodulation -- 2-PSK \& BPSK}
\harecsection{\harec{a}{1.8.8b}{1.8.8b}}
\index{binär fasskift modulation}
\index{fasskift modulation!binär}
\index{2-PSK}
\index{fasskift modulation!2-PSK}
\index{BPSK}
\index{fasskift modulation!BPSK}
\index{Costas loop}

Istället för att modulera frekvensen kan man modulera polariteten eller fasen.
En sådan modulationsform är \emph{binär fasskift modulation} (eng.
\emph{Binary Phase Shift Keying (BPSK)} eller \emph{2-state Phase Shift Keying
(2-PSK)}.
Förenklat kan man säga att bärvågen moduleras med \num{+1} eller \num{-1}, ofta
med \num{+1} representerande \num{0} och \num{-1} representerande \num{1}.

En nackdel med BPSK är att om polariteten blir förväxlad kommer meddelandet att
bli inverterat, dvs. 0 blir 1 och 1 blir 0.
BPSK behöver därför också kompletteras med annan digital modulation för att
hantera polariteten, något som i allmänhet kan åstadkommas enkelt.

BPSK används av satellitnavigationssystem som GPS, GLONASS och Galileo.
För att återvinna BPSK behöver man ofta en speciell variant av PLL-loop känd
som \emph{Costas loop}, eftersom en normal PLL-loop inte klarar av
teckenändringarna på signalen.

\subsection{Fyrnivå fasskiftmodulation -- 4-PSK}
\harecsection{\harec{a}{1.8.8c}{1.8.8c}}
\index{4-PSK}
\index{fasskift modulation!4-PSK}
\index{kvadratur-modulering}
\index{quadrature modulation}
\index{In phase (I)}
\index{Quadrature (Q)}
\index{I/Q modulation}

Fasskiftmodulation kan även göras med flera nivåer.
När fyra olika faslägen används kallas det för \emph{fyrnivå fasskiftmodulation}
(eng. \emph{4-state Phase Shift Keying, 4-PSK}).

Istället för 180~graders fasskift (0 och 180~grader) som används vid 2-PSK/BPSK
så använder man \(360/4\) det vill säga 90~graders fasskift mellan symbolerna.
Ett effektivt sätt att avkoda det är att göra \emph{kvadraturmodulering} (eng.
\emph{quadrature modulation}) där man modulerar en signal till två komponenter,
i \emph{fas} (eng. \emph{In Phase, I}) och förskjuten 90~grader \emph{kvadratur}
(eng. \emph{Quadrature, Q}), ofta kallat I/Q modulering.

De fyra faslägena kan nu enkelt förklaras som amplituder i de olika faslägena
som anges av tabell~\ssaref{tab:4-PSK}.

\begin{table}[t]
\begin{center}
\begin{tabular}{|r|r|r|r|}
\hline
Symbol & Vinkel & I & Q \\ \hline
0 &   0 & +1       &  0 \\
1 &  90 &  0       & +1 \\
2 & 180 & \num{-1} &  0 \\
3 & 270 &  0       & \num{-1} \\ \hline
\end{tabular}
\end{center}
\caption{4-PSK i kvadratur-modulering}
\label{tab:4-PSK}
\end{table}

Amplituden är densamma för alla fyra symbolerna, men med olika vinkel.
I likhet med 2-PSK/BPSK behöver man återvinna fasen och sedan kunna avgöra
vad som är 0~grader, men givet att det görs i den övriga modulationen så
kan informationen avkodas korrekt.

\subsection{Kvadratur-amplitudmodulation -- QAM}
\harecsection{\harec{a}{1.8.8d}{1.8.8d}}
\label{QAM}
\index{kvadratur-amplitudmodulation}
\index{QAM}
\index{16QAM}
\index{DAB}
\index{DVB-T}
\index{DVB-T2}
\index{Wi-Fi}

Medan fasskiftning kan göras för fler fassteg har man funnit att det inte är
lika enkelt för högre upplösningar.
Redan vid åtta steg behöver man ha I- och Q-värden som är \(\sqrt{1/2}\), vilket
i och för sig går att approximera.
En smidigare modulationsform är istället att låta även amplituden variera, och
genom att låta några bitar modulera I och några bitar modulera Q kan man enkelt
få ett symbolmönster som är effektivt att implementera.
Denna modulationsform kallar man \emph{kvadratur-amplitudmodulation}
(eng. \emph{Quadrature Amplitude Modulation, QAM}).

Ofta benämner man olika varianter med antalet olika positioner, så att 16QAM
har 16 olika lägen i fas och amplitud tillsammans.
Ett exempel på hur 16QAM kan moduleras finns i tabell~\ssaref{tab:16QAM}.

\begin{table*}[ht]
\begin{center}
\begin{tabular}{|r|r|r|r|r|r|r|}
\hline
Symbol & Isym & Qsym & Amplitud      & Vinkel &  I &   Q \\ \hline
     0 &    0 &    0 & \(3\sqrt{2}\) &    +45 & +3 &  +3 \\
     1 &    0 &    1 & \(\sqrt{10}\) &    +72 & +3 &  +1 \\
     2 &    0 &    2 & \(\sqrt{10}\) &   +108 & +3 &  \num{-1} \\
     3 &    0 &    3 & \(3\sqrt{2}\) &   +135 & +3 &  \num{-3} \\
     4 &    1 &    0 & \(\sqrt{10}\) &    +18 & +1 &  +3 \\
     5 &    1 &    1 &  \(\sqrt{2}\) &    +45 & +1 &  +1 \\
     6 &    1 &    2 &  \(\sqrt{2}\) &   +135 & +1 &  \num{-1} \\
     7 &    1 &    3 & \(\sqrt{10}\) &   +162 & +1 &  \num{-3} \\
     8 &    2 &    0 & \(\sqrt{10}\) &   +342 & \num{-1} &  +3 \\
     9 &    2 &    1 &  \(\sqrt{2}\) &   +315 & \num{-1} &  +1 \\
    10 &    2 &    2 &  \(\sqrt{2}\) &   +225 & \num{-1} &  \num{-1} \\
    11 &    2 &    3 & \(\sqrt{10}\) &   +198 & \num{-1} &  \num{-3} \\
    12 &    3 &    0 & \(3\sqrt{2}\) &   +225 & \num{-3} &  +3 \\
    13 &    3 &    1 & \(\sqrt{10}\) &   +252 & \num{-3} &  +1 \\
    14 &    3 &    2 & \(\sqrt{10}\) &   +288 & \num{-3} &  \num{-1} \\
    15 &    3 &    3 & \(3\sqrt{2}\) &   +315 & \num{-3} &  \num{-3} \\ \hline
\end{tabular}
\end{center}
\caption{Exempel på 16QAM i kvadraturmodulering}
\label{tab:16QAM}
\end{table*}

Medan både amplituder och vinklar kan kännas udda, så är det enkelt att mappa
bitarna över till I- och Q-amplituder och faslägen via Isym- och Qsym-delarna av
symboler.

QAM-modulering används av DAB, DVB-T, DVB-T2, IEEE~802.11 (Wi-Fi),
mikrovågslänkar och många andra moderna system såsom EDGE
(efterföljaren till GSM med högre datatakt), UMTS när man kör
höghastighet (HSPA) liksom i LTE där man kör relativt långsamma
symboler men i stället väldigt många parallellt fördelat över ett
större frekvensband.
I mobiltelefonisystem använder man bland annat 64QAM och 256QAM.

Mikrovågslänkar använder upp till 2048QAM.
En fördel med QAM-moduleringen är att det är enkelt att få samma avstånd mellan
de olika symbolpositionerna, och därmed kan också modulationen anpassas till
störningen.
Detta nyttjas av många moderna modulationssystem så att QAM-modulationen
anpassas utifrån mottagarens rapportering om störning.
Denna dynamiska anpassning gör att kommunikationen kan upprätthållas även om
kapaciteten tillåts variera.
