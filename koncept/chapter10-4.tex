\section{Avstörningsmetoder}
\index{störning!avstörningsmetoder}
\label{avstörning}
\index{avstörning}

\mediumfig[0.5]{images/cropped_pdfs/bild_2_9-01.pdf}{Nätfilter}{fig:bildII9-1}
\mediumbotfig{images/cropped_pdfs/bild_2_9-02.pdf}{Lågpassfilter för sändare}{fig:bildII9-2}

\subsection{Allmänt}
För att prova ut ett filter, som bäst löser ett visst radiostörningsproblem,
kan man behöva tillgång till ett filtersortiment.
Som exempel nämns bland annat filter i SSA:s avstörningslådor.

\subsection{Nätfilter}
\harecsection{\harec{a}{9.3.1.1}{9.3.1.1}}
\index{nätfilter}
\index{avstörning!nätfilter}

Nätledningar kan fungera som antenn.
I sändarfallet kan HF-signaler komma ut i elnätet genom nätledningen och störa
andra apparater både genom direktanslutning och genom strålning.
I mottagarfallet kan HF-signaler uppfångas av nätledningen, ledas in i
apparaterna och LF-detekteras där.
För att förhindra sådana störningar behövs ett nätfilter.

Nätfiltret ska vara dimensionerat för den nätström, som apparaten är avsäkrad
för och bör anslutas så nära apparaten som möjligt.
Om filtret inte kan placeras där, kan det vara nödvändigt att även skärma
nätledningen mellan filtret och apparaten och jorda skärmen.

% \mediumfig[0.5]{images/cropped_pdfs/bild_2_9-01.pdf}{Nätfilter}{fig:bildII9-1}

Om ledningen förses med till exempel en serieinduktans -- en drossel -- så
dämpas HF-signalerna.
En drossel kan man göra till exempel genom att linda upp några varv av nätsladden
närmast apparaten på toroider eller en eller flera sammanlagda ferritstavar.
I svåra fall kan det behövas ett bredbandigt nätfilter, liknande
det på bild~\ssaref{fig:bildII9-1}.

Det kan förekomma kraftiga spänningstransienter (spänningsstötar) på elnätet.
Dessa transienter kan leda till felfunktioner i anslutna apparater.
För att förebygga sådana fel kan man koppla in ett överspänningsfilter, som kan
vara separat eller sammanbyggt med nätfiltret.

%% \newpage % layout
\subsection{Lågpassfilter}
\label{Lågpassfilter}
\index{lågpassfilter}
\index{avstörning!lågpassfilter}

% \mediumbotfig{images/cropped_pdfs/bild_2_9-02.pdf}{Lågpassfilter för sändare}{fig:bildII9-2}

Lågpassfilter släpper igenom signaler med frekvenser under filtrets
gränsfrekvens.

Ett lågpassfilter med lämpligt vald gränsfrekvens dämpar till exempel
övertonsutstrålningen från en sändare, vars sändarfrekvens ligger under filtrets
gränsfrekvens medan övertonerna ligger över dess gränsfrekvens.

Övertoner kan dämpas med lågpassfilter.
En överton är i detta sammanhang en multipel av sändningsfrekvensen
(grundtonen) exempelvis för
\qty{3,5}{\mega\hertz} grundtonen = (1:a harmoniska) \qty{3,5}{\mega\hertz},
1:a överton = (2:a harmoniska) \qty{7,0}{\mega\hertz},
2:a överton = (3:e harmoniska) \qty{10,5}{\mega\hertz} osv.

Viktigt för avsedd filterverkan är, att filtret ansluts med korrekt
impedansanpassning och med kortast möjliga ledningar.
Detta gäller för övrigt alla filter.

Utstrålning utanför sändningsslagets tillåtna bandbredd anses som
''icke önskad utstrålning''.
Vidare gäller att sådan utstrålning från amatörradiosändare ska hållas så låg
som dagens amatörradioteknik medger.
Bild~\ssaref{fig:bildII9-2} visar principen för lågpassfiltret TP~30 för kortvåg,
med gränsfrekvensen \qty{36}{\mega\hertz}, att kopplas mellan sändaren och
antennledningen.
Med denna gränsfrekvens dämpas övertoner från sändare så att risken för
TV-störningar minskar.

\mediumfig{images/cropped_pdfs/bild_2_9-03.pdf}{Högpassfilter för VHF/UHF-mottagare}{fig:bildII9-3}

\newpage
\subsection{Högpassfilter}
\label{Högpassfilter}
\index{högpassfilter}
\index{avstörning!högpassfilter}
\index{avstörning!antennförstärkare}

Högpassfilter släpper igenom signaler med frekvenser över filtrets
gränsfrekvens.

Bild~\ssaref{fig:bildII9-3} visar principen för högpassfiltret HP~40-S med
gränsfrekvensen \qty{47}{\mega\hertz}, att kopplas in mellan antennledningen och
en mottagare för VHF eller högre frekvenser.

Störningar kommer inte alltid ''utifrån''.
De kan till exempel alstras i bredbandiga antennförstärkare, vilka lätt
överstyrs av alla slags signaler från ett stort frekvensområde.
Man kan då koppla in ett högpassfilter före bredbandsförstärkaren, men en
bättre lösning är att byta till en väl skärmad passbands- eller ännu hellre
kanalförstärkare.

Koaxialkablar med täta skärmar och rätt monterade anslutningskontakter är också
viktigt för en lyckad avstörning.

\clearpage
\mediumplustopfig{images/cropped_pdfs/bild_2_9-05.pdf}{Spärrfilter för mottagare}{fig:bildII9-5}
\mediumplustopfig{images/cropped_pdfs/bild_2_9-04.pdf}{Ingångsimpedansen i resonanskretsar}{fig:bildII9-4}

\subsection{Spärrfilter och sugkretsar}
\label{spärrfilter}
\index{spärrfilter}
\index{avstörning!spärrfilter}
\label{Sugkretsar}
\index{sugkretsar}
\index{avstörning!sugkretsar}
\index{stub}
\index{avstörning!stub}

\mediumplustopfig{images/cropped_pdfs/bild_2_9-06.pdf}{Sugkretsar för mottagare}{fig:bildII9-6}

Om en störande signal råkar finnas inom passbandet för mottagaren kan
man undertrycka -- ''spärra'' -- den signalen med ett spärr- eller sugfilter.
Vilket man väljer är inte kritiskt.

Den störande signalen kan ''spärras'' med en parallellresonanskrets i
serie med mottagaringången, se bild~\ssaref{fig:bildII9-5}.
Kretsen består av en induktans och en kapacitans.

Om man använder en stub som resonanskrets -- till exempel en koaxialkabel -- så
ska den ha längden \(\lambda/4\) och vara ''kortsluten'' eller ha
längden \(\lambda/2\) och vara ''öppen''.
Bild~\ssaref{fig:bildII9-4} visar exempel på hur ingångsimpedansen kan användas.

Man kan även kortsluta -- ''suga bort'' den störande signalen med en
serieresonanskrets parallellt över mottagaringången, se bild~\ssaref{fig:bildII9-6}.
Om man då använder en stub, så ska den ha längden \(\lambda/4\) och
vara ''öppen'' eller ha längden \(\lambda/2\) och vara ''kortsluten''.

Den störande signalen kan undertryckas ytterligare med fler stubar,
som ordnas som i bild~\ssaref{fig:bildII9-6}.
Filtret består då av öppna \(\lambda/4\)-stubar, som utgör avgreningar från
antennkabeln med ett avstånd av \(\lambda/4\).

(Om stubarna i detta filter kortsluts, så bildas ett bandpassfilter i stället).

Exempel på kommersiella spärrfilter är SF~145-S för \qty{144}{\mega\hertz} och
SF~435-S, för \qty{435}{\mega\hertz} amatörband.
De är avsedda att kopplas in före mottagare som störs av amatörradiosändningar.

\begin{description}
\item[SF~145-S] spärrar amatörbandet \SIrange{144}{148}{\mega\hertz} och släpper
igenom banden 0--120 och \SIrange{174}{870}{\mega\hertz}.

\item[SF~435-S] spärrar amatörbandet \SIrange{430}{440}{\mega\hertz} och släpper
igenom 0--350 och \SIrange{470}{870}{\mega\hertz}.
\end{description}

\subsection{Nät- och skärmströmfilter för mottagning}
\index{gemensam överföring}
\index{gemensam strömöverföring}
\index{CM}
\index{Common Mode (CM)}
\index{common mode current}

\smallfig[.35]{images/cropped_pdfs/bild_2_9-07.pdf}{Nät- och skärmströmfilter}{fig:bildII9-7}

Bild~\ssaref{fig:bildII9-7} visar nät- och skärmströmfilter.
Det är vanligt att \emph{gemensam strömöverföring} (eng.
\emph{common mode current}) uppstår som läckage från utrustning och antenn.
Detta kallas ofta ledningsbunden störning.
Det gör att att antennkabeln kan också fungera som antenn.
Särskilt i skärmskarvar kan HF-strömmar läcka ut och in.
De kan då passera förbi eventuella antennförstärkare, filter etc. och orsaka
störningar.

I enkla fall kan gemensam ström stoppas med att linda upp kabeln några varv på
ferritstavar eller genom en stor ferritring som på bilden.
En nätkabel, så kallad sladdställ, får inte kapas och skarvas.

\newpage
\subsection{Phono-ingångsfilter (TBA~302)}
\index{avstörning!phonofilter}

\smallfigpad{images/cropped_pdfs/bild_2_9-08.pdf}{Phonoingångsfilter}{fig:bildII9-8}

\smallfig{images/cropped_pdfs/bild_2_9-09.pdf}{Högtalarledningsfilter}{fig:bildII9-9}

Bild~\ssaref{fig:bildII9-8} visar phonoingångsfilter.
Störande påverkan från radiosändningar kan uppstå om anslutningsledningarna
till phono-ingången i LF-förstärkare är dåligt skärmade och avkopplade.
Sådana störningar kan avhjälpas med ett filter.

\newpage
\subsection{Högtalarledningsfilter (EM 502-B)}
\index{avstörning!högtalarledning}

Bild~\ssaref{fig:bildII9-9} visar högtalarledningsfilter.
HF-instrålning på högtalarledningar kan ha en störande påverkan.
Detta kan undvikas genom koppla in HF-drosslar i ledningarna.
Dessa drosslar bör vara skärmade så att de inte verkar som antenner istället.

I enklare fall kan det räcka med att byta till skärmade högtalarkablar
eller att linda upp en sträcka av ledningarna på en ferritkärna.

\newpage
\subsection{Avkoppling av HF-signaler}
\harecsection{\harec{a}{9.3.1.2}{9.3.1.2}}
\index{avstörning!avkoppling}

\mediumminustopfig{images/cropped_pdfs/bild_2_9-10b.pdf}{HF-avkopplad bas på tre sätt}{fig:bildII9-10b}

\smallfig[.2]{images/cropped_pdfs/bild_2_9-10a.pdf}{HF-avkopplat styrgaller}{fig:bildII9-10a}

\smallfig[.3]{images/cropped_pdfs/bild_2_9-11.pdf}{Parasitfilter i HF-förstärkare}{fig:bildII9-11}

Med avkoppling av en signal menas att den avleds från en signalväg till en
annan.
Vid avstörning avkopplas vanligen den störande signalen till systemjord.

Störimmuniteten i mottagare kan alltså förbättras genom att LF-ingångarna
HF-avkopplas med kondensatorer och/eller HF-spärras med drosslar.

I svåra störningsfall kan det också bli nödvändigt med HF-avskärmning av
LF-ingångsstegen, liksom med ytterligare avstörningsfilter inne i förstärkaren.
Sådana åtgärder innebär emellertid att konstruktionsändringar har gjorts.
Apparatens elsäkerhetsmärkningar är då ogiltiga.

Bild~\ssaref{fig:bildII9-10b} och \ssaref{fig:bildII9-10a} visar några sätt att
avkoppla en oönskad signal från styrgallret i ett elektronrör respektive från
basen i en transistor.

\subsection{Parasitfilter}
\index{avstörning!parasitfilter}

Bild~\ssaref{fig:bildII9-11} visar parasitfilter i HF-förstärkare.
Förstärkarsteg kan råka i självsvängning, ofta på frekvenser i VHF/UHF-området.
Ett sätt att stoppa det är med så kallat parasitfilter.

\subsection{Nycklingsfilter}
\label{Nycklingsfilter}
\index{nycklingsfilter}
\index{avstörning!nycklingsfilter}

Bild~\ssaref{fig:bildII9-12} visar nycklingsfilter.
När en bärvåg nycklas, så bildas övertoner.
Blandningsprodukter av övertonerna och bärvågen hörs som knäppar på
omkringliggande frekvenser.
Märk att övertoner uppstår vid all bärvågsnyckling -- inte bara vid
morsetelegrafering!

När övergångstiden är kort (hård nyckling), så bildas fler övertoner
än när den är längre (mjuk nyckling).
Knäpparna kan till en del dämpas med ett nycklingsfilter där dels
insvängningsförloppet bromsas med en drossel i serie med nycklingskontakten och
dels ursvängningsförloppet med en seriekrets av en resistor och en kondensator,
kopplade parallellt över nycklingskontakten.

\smallfig{images/cropped_pdfs/bild_2_9-12.pdf}{Nycklingsfilter}{fig:bildII9-12}

%% k7per
%% \newpage % layout
\subsection{Förbättrad skärmning}
\harecsection{\harec{a}{9.3.1.3}{9.3.1.3}}
\index{avstörning!skärmning}

HF-energi kan i olyckliga fall även stråla ut genom sändarens hölje
och in genom andra apparaters hölje.
Det medför att apparaternas skärmningar och jordning måste förbättras.
Följ då elsäkerhetsbestämmelserna!
Se även kapitel~\ssaref{elektriskafält}, \ssaref{elektromagnetiskafält} och
\ssaref{jordning}.
