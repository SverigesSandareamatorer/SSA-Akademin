\chapter{Elektromagnetiska fält}
\label{ch:EMF}
\index{elektromagnetiska fält (EMF)}
\index{EMF}
En amatörradiostation genererar signaler, för att kommunicera trådlöst med hela
världen.
Radiovågorna kallas även elektromagnetiska fält (EMF)
(eng. \emph{Electromagnetic Field, EMF}).
Runt alla antenner som sänder ut radiovågor bildas elektromagnetiska fält av den
energi som skickas in i antennerna från radiosändaren.

\textbf{Radiovågorna från en amatörradiostation,} de elektro\-magnetiska fälten, klassas
som \emph{icke-joniserande strålning} och är som sådan strålning inte
tillräckligt energirik för att orsaka annat än uppvärmning av kroppens vävnad.

I allmänhet har studier visat att de nivåer av elektromagnetiska fält som
allmänheten kan utsättas för i närheten av en amatörradiostation ligger långt
under de värden där kroppstemperaturen skulle öka.

\index{icke-joniserande strålning}
\index{strålning!icke-joniserande}
\textbf{Icke-joniserande strålning,} som optisk strålning (infraröd strålning,
synligt~ljus och ultraviolett~strålning) och elektromagnetiska fält (radiovågor
och mikrovågor) är normalt inte lika energirik som joniserande strålning.
När elektromagnetisk strålning absorberas i biologisk vävnad eller material är
den dominerande effekten därför endast en temperaturhöjning i vävnaden eller
materialet.

\index{joniserande strålning}
\index{strålning!joniserande}
\textbf{Joniserande strålning,} partikelstrålning eller elektromagnetisk strålning, som
har tillräcklig energi för att rycka loss elektroner från de atomer som den
träffar och förvandla dem till positivt laddade joner, jonisering.
Exempel på joniserande strålning är röntgenstrålning och strålning från
radioaktiva ämnen.
Energin hos joniserande strålning kan vara så hög att den kan tränga in i
kroppen och påverka cellstruktur samt arvsmassa (DNA) i biologiskt material.

\index{WHO|see {World Health Organization (WHO)}}
\index{World Health Organization (WHO)}
\index{International Commission on Non-Ionizing Radiation Protection (ICNIRP)}
\index{ICNIRP}
Inom World Health Organization (WHO) finns ett program som kallas
''The International EMF Project'' och där samlas all vetenskaplig
information som finns om biologiska effekter orsakade av elektromagnetiska fält.
''International Commission on Non-Ionizing Radiation Protection'', (ICNIRP)
är en fristående organisation (erkänd av WHO) som bland annat använder denna
information för att utveckla riktlinjer för begränsning av exponeringsnivån för
elektromagnetiska fält.
Dessa riktlinjer används av många länder.

\index{Strålsäkerhetsmyndigheten (SSM)}
\index{SSM|see {Strålsäkerhetsmyndigheten (SSM)}}
\index{International Commission on Non-Ionizing Radiation Protection (ICNIRP)}
\index{ICNIRP}
Strålsäkerhetsmyndigheten (SSM) är den myndighet som har det formella ansvaret
för strålskydd i Sverige.
Myndigheten ska bland annat förebygga akuta skador och minska risken för sena
hälsoeffekter hos allmänheten till följd av exponering för elektromagnetiska
fält.

SSM har tagit fram allmänna råd SSMFS 2008:18~\cite{SSMFS2008:18} för
begränsning av allmänhetens exponering för elektromagnetiska fält.
De allmänna råden anger vilka referensvärden som gäller i Sverige.
Råden utgår från rekommendationer i EU-direktiv 1999/519/EG~\cite{1999/519/EG}.
EU-direktivet följer i sin tur de riktlinjer för begränsning av
elektromagnetiska fält som sammanställts av ICNIRP.

Eftersom grunden i amatörradioutövandet är att generera elektromagnetiska fält
för att kommunicera via radio så är kunskapen om EMF viktig.
Med de möjligheter radioamatörer har, måste de allmänna råden gällande EMF
följas.
Förståelsen för hur fält uppträder och hur de kan begränsas anses vara
fundamental kunskap för radioamatörer.
