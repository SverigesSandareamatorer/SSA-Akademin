\chapter{Elektromagnetiska fält}
\label{ch:emf}
\index{elektromagnetiska fält (EMF)}
\index{EMF|see {elektromagnetiska fält}}

I människans naturliga miljö har det alltid funnits strålning.
Den kommer bland annat från rymden, solen och från radioaktiva ämnen i marken
och i din egen kropp.

Strålning kan delas in i joniserande och icke-joniserande strålning.

Joniserande strålning är så energirik att den kan rycka loss elektroner från de
atomer som den träffar och förvandla dem till positivt laddade joner,
jonisering.
Exempel på joniserande strålning är röntgenstrålning och strålning från
radioaktiva ämnen.
Energin hos den joniserande strålningen kan vara så hög att den kan tränga in i
och påverka cellstrukturen i biologiskt material.

Energin hos icke-joniserande strålning, som optisk strålning och
elektromagnetiska fält, är normalt inte lika energirik som hos joniserande och
kan därför inte jonisera material. Icke-joniserande strålning är
elektromagnetiska fält (EMF) och optisk strålning.

Icke-joniserande elektromagnetisk strålning uppträder i många former.
Som radiovågor, mikrovågor, infraröd strålning, synligt ljus, ultraviolett
strålning.
Det som skiljer är våglängden.
Radiovågor har längst våglängd och ultraviolett kortast.
Elektromagnetiska fält som används för trådlös kommunikation kallas radiovågor
och mikrovågor.

Elektromagnetiska fält kan orsaka uppvärmning av kroppens vävnad.
I allmänhet har studier visat att de nivåer av EMF som allmänheten utsätts för
ligger långt under de värden där kroppstemperaturen skulle öka.

Inom World Health Organization (WHO) finns ett program som kallas ''The
International EMF Project'' och där samlas all vetenskaplig information som finns
om biologiska effekter av elektromagnetiska fält.
''International Commission on Non-Ionizing Radiation Protection'', (ICNIRP) är
en fristående organisation (erkänd av WHO) som bland annat utnyttjar denna
information för att utveckla riktlinjer för exponering av EMF som används av
många länder.

Strålsäkerhetsmyndigheten (SSM) är den myndighet som arbetar med ovanstående
frågor i Sverige och har tagit fram allmänna råd vilka är en tolkning av ett
EU-direktiv som bygger på ICNIRP:s riktlinjer för allmänhetens exponering för
elektromagnetiska fält.

Eftersom grunden i amatörradioutövandet är att generera elektromagnetiska fält
för att kommunicera via radio så är kunskapen om EMF viktig.
Med de befogenheter radioamatörer har, måste riktlinjerna för EMF följas och
förståelsen för hur fält uppstår och hur de kan begränsas är fundamentala.
