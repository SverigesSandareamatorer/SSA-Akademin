\section{Exempel på kontakt}
\harecsection{\harec{b}{7.2.1}{7.2.1}}

Det finns många sätt att genomföra en radiokontakt, men det finns några
grundregler för hur man uppträder och utväxlar samtal.
Ett trevligt och kamratligt uppträdande är en hederssak inom amatörradion.
Det behöver inte bli stelt för den skull!

Allmänt anrop är ett sätt att kalla på någon
-- vem som helst -- att kommunicera med.
På telegrafi låter det så här:

-- CQ CQ CQ de SM0XYZ SM0XYZ K

Det vill säga anropet först och därefter den egna anropssignalen.
På telefoni låter det så här:

-- Allmänt anrop, allmänt anrop, allmänt anrop från SM0XYZ Kom

Glöm inte Kom i slutet.
Riktat anrop gör man, när man vill tala med någon särskild station.
Då sänder man först anropssignalen på den station, som man vill tala med och
därefter sin egen anropssignal.
På telegrafi låter det så här:

-- SM0ZYX SM0ZYX de SM0XYZ SM0XYZ K

På telefoni låter det så här:

-- SM0ZYX SM0ZYX från SM0XYZ SM0XYZ Kom

Motstationen svarar förhoppningsvis på anropet, alltså

-- SM0XYZ från SM0ZYX Kom

\subsection{Upprättad förbindelse}

När en station svarat på anrop, lämnar man först sin signalrapport enligt
RST-koden och presenterar sig med sitt förnamn och berättar var man finns.
Motstationen kvitterar troligen med sina motsvarande uppgifter.

När man överlämnar ordet till motstationen avslutar man meningen med Kom och
lyssnar.
Om man har en telegrafiförbindelse och bara vill att den station man har
förbindelse med ska svara kan man sända KN (eng. \emph{come named station}).

Om förbindelsen varar länge, är det lämpligt att upprepa anropssignalerna
ungefär var tionde minut vid överlämning.

-- SM0ZYX från SM0XYZ Kom

\subsection{Avsluta förbindelse}

När man så småningom avslutar kontakten tackar man för sig på och utbyter
avskedshälsningar. Då kan det låta så här:

-- Tack för en trevlig förbindelse och på återhörande. SM0ZYX från
SM0XYZ. Klart Slut.

Träna med din instruktör på att klara olika slags trafiksituationer!

\subsection{Second operator}
\index{Second operator}
\label{secondoperator}

Den som självständigt använder en amatörradiosändare ska ha ett
amatörradiocertifikat.
Det finns ett undantag från kravet på amatörradiocertifikat då en person
tillfälligt använder en amatörradiosändare under uppsikt av någon som har ett
amatörradiocertifikat.
Detta kallas \emph{second operator} och innebär att en person som saknar
amatörradiocertifikat kan agera operatör jämte en person som har ett.

I Sverige är det reglerat i undantagsföreskriften PTSFS 2022:19 som tas upp i
avsnitt~\ssaref{PTSFS2022:19}.
Detta medger att man kan förevisas hobbyn och även träna under kontrollerade
förhållanden.
För att detta ska fungera krävs att den med amatörradiocertifikat instruerar
om hur man ska bete sig i etern, hur handhavandet går till och kan övervaka
att detta följs.

Självklart används anropssignalen för innehavaren av amatörradiocertifikatet.
Det är bra att det tydligt framgår att det är en second operator som är aktiv.
Antingen ropar amatören upp och sedan berättar att han lämnar över till second
operator Simon.
Alternativt kan en second operator göra anropen själv och då ropa till exempel
''SM5XYZ second operator Anna''.

Möjligheten att använda second operator ska användas med klokhet, och kan rätt
använd skapa en god förståelse för hobbyn och utgöra en morot för att få både
ungdomar och vuxna intresserade av amatörradio.

\subsection{CQ DX och split}
\label{cq dx och split}
\index{CQ DX}
\index{split}
\index{DX expedition}
\index{rar DX}
\index{pile-up}

Det förekommer att man hör någon ropa \emph{''CQ DX''}, vilket betyder att
stationen söker långväga kontakter, i allmänhet utanför sin egen världsdel.

-- ''CQ DX, CQ DX, CQ DX, SM0XYZ calling CQ DX and standing by''

I detta fallet är det SM0XYZ som söker att nå någon utanför Europa.
Är du själv inte ett DX, det vill säga om du befinner dig i samma världsdel så
ska du undvika att svara.

Ibland genomförs så kallade \emph{DX expeditioner} då man beger sig till en
plats som sällan aktiveras.
Man brukar tala om \emph{rara DX} (eng. \emph{rare DX}), då ett ovanligt
landområde aktiveras, som många vill ha i sin logg.

En station som ropar CQ kan få svar från många stationer samtidigt.
Då uppstår ett sammelsurium av signaler som kallas för \emph{pile-up}.

När stationen betar av en pile-up kan stationen även fråga \emph{''QRZ?''},
alltså ''vem där''?

Ett rart DX kan drabbas av enorma pile-ups, och det kan bli svårt för
motstationerna att höra DX-stationen bland alla andra som ropar samtidigt.
Det kan kan också vara svårt för DX-stationen att urskilja vilka motstationer
som svarar, om alla svarar på samma frekvens.
En strategi för att få detta att fungera effektivare är att köra split
\cite{LowBandDX}, det vill säga att DX-stationen sänder och lyssnar på olika
frekvenser (men fortfarande inom samma frekvensband).

Oftast väljer DX-stationen att lyssna på en frekvens som ligger några kilohertz
högre än den egna sändningsfrekvensen, och anger detta genom att sända
exempelvis \emph{''listening up''} eller \emph{''listening five up''}.

Genom att använda sig av split undviker DX-stationen att störas ut av sin egen
pile-up.
DX-stationen kan också välja sprida ut sin pile-up, genom att inte lyssna på
endast en frekvens, utan genom att svepa över ett lite större område.

\emph{''Listening five to ten up''} betyder då att DX-stationen lyssnar i ett
område mellan 5 och \qty{10}{\kilo\hertz} över den egna sändningsfrekvensen, och
motstationerna får försöka gissa var i detta frekvensområde som DX-stationen
lyssnar just för tillfället.

För trafik på morsetelegrafi använder man vanligtvis ett mindre avstånd mellan
sändar- och mottagarfrekvens än för trafik på SSB-telefoni, eftersom
telegrafisignalerna upptar mindre bandbredd.

Moderna transceivrar har nästan alltid möjlighet att ställa in split genom att
man använder ''VFO A/B'', ''RIT/XIT'' eller ''clarifier''.
Mer avancerade transceivrar kan ha möjlighet att separera de två frekvenserna i
hörlurarnas vänster- respektive högerkanal.

När en station ropar CQ och gör paus för anropande stationer, ange då din egen
signal kort och tydligt en gång i varje pass.
Istället för att ropa flera gånger varje pass, skrika eller på annat sätt
ta utrymme, ha tålamod och vänta ut bra tillfälle.
Ropa inte under den tid som DX-stationen sänder sitt CQ. Då hör han dig ju ändå inte.

Det kan vara nyttigt att lyssna in sig på operatörens stil.
Var medveten om att DX-stationen kan höra helt andra stationer starkare än vad
du hör, eftersom konditionerna kan vara helt annorlunda för DX-stationen.

Vid stora pile-ups kan operatören välja att bara lyssna efter vissa stationer,
och därför fråga efter ''only number five stations please'' eller efter ''only
European stations please''.
Detta syftar till att dela upp en stor pile-up för en chans att lättare
uppfatta vilka som anropar.

Vid uppdelning efter nummer kommer operatören avverka några stationer med ett
visst nummer i anropssignalen, för att sedan gå vidare till nästa och så
vidare, tills 0 till 9 är genomgångna.

Alternativet att gå efter regioner eller landsprefix kan vara att föredra om
operatören upplever att konditionerna dit snart försvinner och därför vill ge
dem en extra förtur innan de helt tappar chansen.

\paragraph{Lär dig ''DX:arens ordningsregler'':}

\begin{itemize}
\item Jag ska lyssna, och lyssna, och sedan lyssna lite till.
\item Jag ska ropa endast om jag kan läsa DX-sta\-tion\-en ordentligt.
\item Jag ska icke lita på cluster-information, utan vara helt klar över DX-stationens anropssignal innan jag ropar.
\item Jag ska icke störa DX-stationen, eller någon som anropar denne, och jag ska aldrig stämma av på DX:ets egen frekvens, eller i det segment där denne lyssnar.
\item Jag ska vänta tills DX:et avslutat föregående kontakt innan jag ropar själv.
\item Jag ska alltid ange min fullständiga anropssignal.
\item Jag ska ropa och lyssna med lämpliga intervaller.
\item Jag ska icke ropa kontinuerligt.
\item Jag ska icke ropa då DX:et svarar någon annan än mig.
\item Jag ska icke ropa då DX:et frågar efter en anropssignal som icke liknar min egen.
\item Jag ska icke ropa då DX:et söker efter ett annat geografiskt område än mitt eget.
\item När DX:et svarar mig så ska jag icke upprepa min anropssignal, annat än om jag tror att denne ej uppfattat den korrekt.
\item Jag ska vara tacksam om och när jag får kontakt.
\item Jag ska respektera mina amatörkamrater och uppträda så jag förtjänar deras respekt.
\end{itemize}

Försök inte agera polis och rätta andra stationer som du anser bryter mot reglerna!

\section{Innehåll i förbindelse}
\label{innehåll i förbindelse}
\harecsection{\harec{b}{7.2.3}{7.2.3}}

Tidigare har det i Sverige varit reglerat vad innehållet får vara i
förbindelser, eller snarare vad de inte får innehålla.
Den regleringen är numera borttagen.
Man ska vara medveten om att samma regler och förutsättningar inte gäller i
alla länder och för deras radioamatörer.
Därför uppmanas du att använda sunt förnuft, hålla god ton och respektera alla
amatörer.
Se även IARU etik och trafikmetoder.

\subsection{Tystnadsplikt}
\index{tystnadsplikt}
\index{LEK}

Innehållet i en radioförbindelse skyddas av
\emph{Lag om elektronisk kommunikation (LEK)}~\cite{SFS2022:482}.
I LEK regleras tystnadsplikt för radiobefordrade meddelanden i kapitel~6.

\begin{quote}
	Den som i annat fall än som avses i 31~\S{} första stycket och 32~\S{} i
	radiomottagare har avlyssnat eller på annat sätt med användande av sådan
	mottagare fått tillgång till ett radiobefordrat meddelande i ett
	elektroniskt kommunikationsnät som inte är avsett för honom eller henne
	själv eller för allmänheten får inte obehörigen föra det vidare.
	Lag (2022:482).\cite[kap 9, \S33]{SFS2022:482}
\end{quote}

Tystnadsplikten gäller alla radiomeddelanden som avlyssnats, oavsett ursprung.

Detta innebär att om du själv varit part i radiomeddelandet eller om
radiomeddelandet var en nyhetsbulletin avsett för många så får du föra det vidare.

En stor del av radioamatörhobbyn bygger dock på radiokommunikation med andra och
att andra kan höra dig när du sänder.
En radioamatör kan därför inte anses vara omedveten om att någon annan lyssnar
på det som sänds ut.
Därför är mycket accepterat inom amatörradio som annars skulle vara förbjudet.

Tips om rara DX, tips om någon som ropar CQ, QSL från lyssnaramatörer, att
berätta att du hörde någon ha förbindelse med någon annan anses därför normalt
inte vara ett brott mot tystnadsplikten.

Att koppla en radiomottagare till webben så att någon kan lyssna på radiotrafik
i realtid är tillåtet.

\emph{Observera även texten i andra punkten i 6~kap.~20~\S{} gällande den som i
	samband med tillhandahållande av ett elektronisk kommunikationstjänst har fått
	del av eller tillgång till innehållet i ett elektroniskt meddelande inte
	obehörigen får föra vidare eller utnyttja det han fått del av eller tillgång
	till.}

Detta kan vara aktuellt då någon som tillhandahåller en elektronisk
kommunikationstjänst från punkt A till punkt B får tillgång till innehållet i
ett elektroniskt meddelande när det har lämnat punkt A och innan det når fram
till punkt B.

\subsection{Inspelning av radiomeddelande}
\index{inspelning}
\index{GDPR}
\index{Dataskyddsförordningen}

Radiosamtal som du själv deltar i får spelas in utan att andra deltagare i
samtalet informeras om att du spelar in samtalet.

Grundregeln är att inspelning av radiomeddelanden är tillåten såvida inte
inspelningen är förbjuden för att skydda personers personliga integritet.

Uppspelning av de inspelade meddelandena får inte bryta mot bestämmelserna om
tystnadsplikt.
Det vill säga att meddelandet inte obehörigen får föras vidare.

Alla radiomeddelanden får inte spelas in.
Lagstiftningen skiljer även på analoga- och digitala inspelningar.
Dataskyddsförordningen~\cite{GDPR} samt 4~kap.~9a~\S{} i
Brottsbalken~\cite{SFS1962:700} är exempel på lagar som begränsar inspelning av
avlyssnade radiomeddelanden.

Av ovanstående följer att det inte är tillåtet att lagra inspelad radiotrafik
för senare lyssning via webbaserade medier då det kan anses kränka den
personliga integriteten.

\subsection{Kryptering av radiomeddelande}
\label{kryptering av radiomeddelande}
\index{kryptering}

Inom Sveriges gränser är kryptering av radiomeddelanden på amatörradiofrekvenser
tillåten under villkor att en anropssignal regelbundet sänds ut, anropssignalen
ska då kunna avläsas med kända tekniker.
Trots detta rekommenderas inte användning av kryptering för amatörradiotrafik.

Tekniken för kryptering av radiomeddelanden har blivit mera lättillgänglig i
samband med införandet av digitala radiosystem typ DMR (Digital Mobile Radio) på
amatörbanden.
Ett flertal av dessa radiosystem är dock ihopkopplade via internationella
nätverk och därigenom hörbara i flera länder där kryptering inte är tillåten.

Användningen av krypteringsteknik på amatörradiofrekvenser riskerar därför att
medföra begränsningar i de rättigheter vi har enligt PTSFS 2022:19.
