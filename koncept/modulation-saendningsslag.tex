\section{Sändningsslag}
\index{sändningsslag}
\label{sändningsslag}

Sätten att modulera kallas \emph{sändningsslag}.
Gemensamt för sändningsslagen är att en givare -- det kan vara en mikrofon, en
telegrafnyckel, en fjärrskriftsmaskin, en dator, en TV-kamera -- alstrar
en analog eller digital signal.
Denna styr underbärvågen så att huvudbärvågen moduleras med den avsedda
informationen och sänds ut.

Det enklaste sändningsslaget får anses vara morsetelegrafi med
''nycklad bärvåg''.
Då förekommer bara två tillstånd, nedtryckt och icke nedtryckt telegrafnyckel,
dvs. antingen bärvåg med någon varaktighet eller ingen bärvåg alls.
Kombinationer av bärvågselement med olika längd motsvarar skrivtecken.

För att återge tal, musik etc. behövs en noggrannare tillståndsstyrning av
bärvågen.
Det innebär att bärvågen måste moduleras av en underbärvåg och att denna
motsvarar lufttrycksvariationerna i ljudet.
