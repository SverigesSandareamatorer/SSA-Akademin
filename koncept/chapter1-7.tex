\section{Icke sinusformade signaler}
\textbf{HAREC a.\ref{HAREC.a.1.7}\label{myHAREC.a.1.7}}

\subsection{Grundton, övertoner- Kantvågor}
\textbf{HAREC a.\ref{HAREC.a.1.7.2}, a.\ref{HAREC.a.1.7.3}, a.\ref{HAREC.a.1.7.4b}\label{myHAREC.a.1.7.2}\label{myHAREC.a.1.7.3}\label{myHAREC.a.1.7.4b}}
\index{grundton}
\index{överton}
\index{kantvåg}

\begin{figure*}
\begin{center}
\includegraphics[width=\textwidth]{images/cropped_pdfs/bild_2_1-18.pdf}
\caption{Ren sinusvåg och övertonshaltig våg}
\label{fig:BildII1-18}
\end{center}
\end{figure*}

Bild \ref{fig:BildII1-18}.

Ett sinusformat förlopp med en enda frekvens -- en enda ton -- sägs vara
spektralt ren. En sådan svängning kallas för grundton.

Varje signal, som inte är sinusformad, är sammansatt av flera sinussvängningar.
Det är signalens grundton samt dess harmoniska övertoner, vilka kan ha 2, 3
o.s.v. gånger högre frekvens än grundtonen. Den inbördes styrkan på grundton
och övertoner avgör signalens form. Om signalen ligger inom det hörbara
området, kan man märka hur den ändrar karaktär beroende på övertonshalten. Man
kan säga att övertonerna modulerar grundtonen.

\begin{figure*}
\begin{center}
\includegraphics[width=\textwidth]{images/cropped_pdfs/bild_2_1-19.pdf}
\caption{Uppdelning av en signal i grundton och övertoner}
\label{fig:BildII1-19}
\end{center}
\end{figure*}

Bild \ref{fig:BildII1-19}.

Oscillatorsignalen i exemplet på bilden har 1 volts amplitud på grundtonen
\(f_0\) (1:a harmoniska), 0,7~volts amplitud på de 1:a övertonen
(2:a harmoniska) och 0,2~volts amplitud på den 2:a övertonen (3:e harmoniska).
Den totala amplituden blir emellertid inte summan av 1, 0,7 och 0,2~volt
eftersom de olika delspänningarnas toppvärden inte uppträder samtidigt.
I stället måste delspänningarna adderas vid varje tidpunkt för sig.

\begin{figure*}
\begin{center}
\includegraphics[width=\textwidth]{images/cropped_pdfs/bild_2_1-20.pdf}
\caption{Uppdelning av en fyrkantsvåg i grundton och övertoner}
\label{fig:BildII1-20}
\end{center}
\end{figure*}

Bild \ref{fig:BildII1-20}.

\index{Fourier, Joseph}
\index{Fourier!Fourier analys}
\index{Fourier!Fourier transform (FT)}
\index{Fourier!invers Fourier transform (IFT)}
\index{Fourier!Discrete Fourier Transform(DFT)}
\index{DFT}
\index{Fourier!inverse Discrete Fourier Transform (IDFT)}
\index{IDFT}
\index{Fourier!Fast Fourier Transform(FFT)}
\index{FFT}
\index{Fourier!inverse Fast Fourier Transform (IFFT)}
\index{IFFT}

\infobox{
Denna analys av vågor uppfanns av Jean-Baptiste Joseph Fourier (1768--1830)
vid analys av värmeutbredning och vibration som presenterades 1822. Denna metod
är kraftfull och har haft stort inflytande på vetenskapen och utvecklingen
både som matematiskt verktyg och som praktiskt analys med spektrum-analysatorer
och vid modern modulation och demodulation. Man pratar om \emph{Fourier analys}
(eng. Fourier analysis) och \emph{Fourier transform (FT)} för omvandling från
tid till frekvens och \emph{invers Fourier transform} för omvandling från
frekvens till tid. För tidsdiskret (samplad) form är termerna
\emph{Diskret Fourier Transform (DFT)} och
\emph{invers Diskret Fourier Transform (IDFT)} respektive. Senare optimeringar
av beräkningar har resulterat i \emph{Fast Fourier Transform (FFT)} och
\emph{Inverse Fast Fourier Transform (IFFT)}.
}

Det finns olika karaktärer av förlopp såsom sinusvåg, triangelvåg, sågtandsvåg,
fyrkantsvåg o.s.v.

Fyrkantsvågen är sammansatt av sinusvågor med grundfrekvensen och dess udda
övertoner, varvid amplituderna fördelar sig som \(1/1\), \(1/3\), \(1/5\),
\(1/7\), \(1/9\), \(1/11\) o.s.v. Teoretiskt når övertonsspektrum upp till
oändligt höga frekvenser, medan de motsvarande amplituderna minskar till
oändligt små värden.

En ideal fyrkantsvåg, vilken inte kan uppnås i praktiken, skulle bestå av ett
oändligt antal udda övertoner med fallande amplitud. Ju fler av de högre
övertonerna som filtreras bort, desto mer lutar fyrkantsvågens flanker, desto
rundare blir hörnen på vågen och desto vågigare blir kurvans topp.

\subsection{Överlagrade spänningar
(likspänningskomposant)}
\textbf{HAREC a.\ref{HAREC.a.1.7.4a}\label{myHAREC.a.1.7.4a}}

\begin{figure}[h]
\includegraphics[width=\textwidth]{images/cropped_pdfs/bild_2_1-21.pdf}
\caption{Överlagrade spänningar}
\label{fig:BildII1-21}
\end{figure}

Bild \ref{fig:BildII1-21}.

I signalkretsar förekommer det mycket ofta, att växelspänning överlagras på
likspänning eller omvänt. Likspänningen kallas då för likspänningskomposant.
Olika åtgärder behövs för att överlagra spänningar på varandra och att sedan
skilja dem åt.

Bilden visar ett avsnitt av en AM-mottagare. Från vänster hämtas en
AM-modulerad signal från MF-förstärkaren för att demoduleras, d.v.s. för att
återvinna den modulerande LF-signalen. MF-signalen halvvågslikriktas. Kvar blir
den positiva delen av MF-signalen och den modulerande LF-signalen,
sammanlagrade. LF-signalen ska nu skiljas ut och förstärkas. Alltså filtreras
MF-komposanten bort. Kvar blir LF-signalen, men överlagrad på en likspänning.
Likspänningen stoppas och kvar blir slutligen LF-signalen som förstärks.

\subsection{Brus}
\textbf{HAREC a.\ref{HAREC.a.1.7.5}\label{myHAREC.a.1.7.5}, a.\ref{HAREC.a.7.19}\label{myHAREC.a.7.19}}
\label{termisktbrus}

\subsubsection{Termiskt brus}
\index{brus}
\index{termiskt brus}
\index{brus!termiskt}

\begin{rev-nytt}[MAD]
Resistorer och resistans, i alla dess former, uppvisar en egenskap av
en varierande spänning även när ingen ström går genom motståndet. Denna extra
spänning innehåller ett brett spektra av toner, men är också ett tätt spektra,
sådan att ingen enskild ton kan särskiljas från någon annan. Istället för att
tänka sig en grundton och dess övertoner med ingen energi emellan dem så är det
istället ett kontinuerligt spektra med oändligt många toner. Detta spektra
begränsas dock av bandbredden.

\hilight{TODO: Illustrera resistor, brus en och in.}

\hilight{TODO: Illustrera brus över tid.}

\hilight{TODO: Illustrera brus över frekvens.}

\index{vitt brus}
\index{brus!vitt}
\index{white noise}
\index{Johnsson noise}
\index{brus!Johnsson}
Man kallar detta spektra i dagligtal för \emph{termiskt brus}
(eng. thermal noise), eftersom det beror på temperaturen hos motståndet, eller
\emph{Johnsson noise}, efter J. B. Johnsson som 1928 fann att detta brus fanns
i alla ledare \cite{ott1988}.
I dagligt tal pratar man dock bara om \emph{vitt brus} (eng. white noise) eller
\emph{brus}.

Effekten \(P_n\) av detta brus beror på Boltzmans konstant
\(k\ =\ 1,38 \cdot 10^{-23}\) J/K, den absoluta temperaturen \(T\) i
kelvin samt bandbredden \(B\) i Hertz och anges enligt formeln:

\(P_n = k T B\)

Den motsvarande spänningen \(e_n\) och strömmen \(i_n\) för resistansen \(R\) är

\(e_n = \sqrt{4kTBR}\)

\(i_n = \sqrt{\frac{4kTB}{R}}\)

\end{rev-nytt}

\subsubsection{Brusbandbredd}
\index{brus!brusbandbredd}
\begin{rev-nytt}[MAD]

Medans vi initialt antagit att brusets bandbredd är för frekvenser
från DC till övre gränsfrekvensen så är det inte nödvändigt. Formeln är även
relevant för bruset på ett band, och bandbredden för det bandpass filter vi har
för att enbart lyssna på detta band.

Exempelvis behöver tal på SSB hantera 300~Hz till 3~kHz, dvs. 2,7~kHz
bandbredd och därmed kommer även mottagarens bandbredd behöva vara så stort,
och därmed även brusbandbredd på 2,7~kHz. Vi kommer då att ta emot brus för
motsvarande bandbredd. Ett CW filter kan t.ex. vara 350~Hz och kommer därmed
också ha ett motsvarande förhållande lägre brus-effekt.

Detta är dock en förenkling, eftersom filtret inte filtrerar med branta kanter
och är helt plant. Filtrets egentliga brus-bandbredd beror på hur filtret
filtrerar över alla frekvenser och summan av dessa. Beroende på vilken typ av
filter så behövs därför en korrigeringsfaktor från den normala bandbredden
till brus-bandbredden. För ett normalt 12~dB/oktav lågpass filter är
korrigerings faktorn 1,22.

\end{rev-nytt}
