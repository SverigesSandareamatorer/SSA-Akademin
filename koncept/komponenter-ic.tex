\newpage
\section{Integrerade kretsar (IC)}
\label{integrerade kretsar}

\subsection{Allmänt om IC}
\index{integrerad krets}
\index{IC}

Att integrera betyder att samla till en enhet, det kan vara komponenter,
funktioner eller verksamheter.
Integration kan ske på olika nivåer och i många olika sammanhang.

Med integration avses här integration av komponenter för elektroniska
strömkretsar.
Särskilt halvledarelement av olika slag samt resistorer och kondensatorer med
små värden kan framställas med små dimensioner.
Många komponenter kan då samlas i samma hölje.

Komponenter inom ett hölje, avsedda för en viss funktion kallas
\emph{integrerad krets} (eng. \emph{Integrated Circuit -- IC}).

Komponenterna i en IC kan i sin tur vara del av komponenterna en hel strömkrets.
Redan inom höljet kan komponenter kopplas samman för en viss funktion eller som
en del av strömkretsen.
Skrymmande eller effektkrävande komponenter, såsom induktorer, transformatorer
och så vidare får emellertid inte plats, varför även yttre kopplingar behövs.
Det kan också behövas flera IC i en strömkrets -- kanske med innehåll för en
annan funktion.

En integrerad krets är uppbyggd på en basplatta av halvledarmaterial -- ett
chipp.
På plattan framställs, med fototeknik eller etsning, kompletta eller nästan
kompletta dioder, transistorer, resistorer och kondensatorer.
Metoden, som kallas planarteknik, medger att många komponenter kan få plats på
samma platta.

\subsection{Olika slags integrerade kretsar}

Det finns stora sortiment av både standardiserade och speciella IC, varav det
finns två huvudtyper:
\begin{itemize}
  \item digitala integrerade kretsar
  \item analoga integrerade kretsar.
\end{itemize}

\subsection{Digitala IC}

Digitala IC arbetar som framgår av namnet med digitala signalnivåer.
De enklaste typerna innehåller en eller flera digitala grindar (se avsnitt
\ssaref{digitala kretsar}).
Genom att koppla samman grindar kan man skapa kretsar för ett visst ändamål.
I början av 70-talet byggdes komplicerade system av grindar i SSI- och
MSI-teknik.
Ett sådant system är emellertid inte flexibelt eftersom eventuella ändringar
måste göras ''hårdvarumässigt''.
Det innebär att kopplingsledningar måste ändras, kanske hela kretsar bytas ut
och så vidare.

I dagens digitala system används IC i form av en mikroprocessor eller till och
med flera.
En mikroprocessor är en avancerad krets som kan programmeras (konfigureras)
mjukvarumässigt inte bara för ett ändamål utan för många olika.
I system med mikroprocessorer behövs också minnesfunktioner.
Mikroprocessorn är hjärtat i en dator.
Styrd av ett program (mjukvaran) kontrollerar den kringutrustningar med uppgift
att inhämta och avge information -- att kommunicera.

\subsection{Analoga IC}

Analoga IC arbetar med analoga signalnivåer, det vill säga spänningar och
strömmar med kontinuerligt varierande nivåer och frekvenser.
En analog IC kan även arbeta med digitala signaler.

Analoga IC innehåller en eller flera balanserade förstärkare samt olika slags
hjälpkretsar.
Med yttre komponenter kan en analog IC ges olika förstärkning och frekvensgång.
Med ett gemensamt namn kallas dessa förstärkare för operationsförstärkare
(OP-amp).
Operationsförstärkare utförs vanligen i SSI- eller möjligen MSI-teknik.

\subsection{Kombinerade och speciella IC}

Utöver standardiserade IC finns kombinerade och speciella IC.
Exempel på speciella digitala IC är sådana för telekommunikationsändamål.
Ett annat exempel på digitala IC är sådana för signalbehandling, såväl på HF
som LF-nivå.
Exempel på speciella analoga IC är sådana för radiokommunikationsändamål.

Bortsett från vissa skrymmande komponenter och manöverdon kan numera till
exempel en IC innehålla en komplett radiomottagare.
Ett annat exempel på speciella analoga IC är sådana för hörapparater.
Genom programmering anpassas de för det personliga behovet.

\subsection{Utvecklingen}

Det kan sägas hur ofta som helst.
Genom den fantastiska utvecklingen av mikroelektronik öppnas även för
radioamatören möjligheter som tidigare inte var tänkbara.

Denna utveckling har vidgat utrymmet för den experimentella verksamhet som
amatörradio i grunden innebär.
Hobbyn får sålunda med tiden en allt större teknisk spännvidd.

\subsection{Aktuell litteratur}

Ökat teknikomfång inom amatörradio ställer motsvarande krav på litteratur.
På senare tid inbegripes även digitalteknik.
Mest av utrymmesskäl behandlas i denna faktabok digitaltekniken mycket
kortfattat, men ändå så mycket som nämns i CEPT-rekommendationen T/R~61-02.
För djupare studium hänvisas till andra läromedel samt till
leverantörskataloger.
