\section{Fasmodulation (PM)}
\index{fasmodulation}
\index{PM}

Vid fasmodulation varierar bärvågens fasläge i förhållande till ett
referensvärde.
Vid PM är frekvensändringen -- deviationen -- direkt proportionell mot hur
snabbt fasläget på den modulerande frekvensen ändras och till den totala
fasändringen.
Hastigheten på fasändringen är direkt proportionell mot frekvensen på den
modulerande frekvensen och till den momentana amplituden på den modulerande
signalen.

Det betyder att deviationen i PM-system ökar både med den momentana amplituden
och frekvensen på den modulerande signalen.
Detta att jämföras med FM-system där deviationen är proportionell mot den
momentana amplituden på den modulerande signalen.

I PM-system uppfattar demodulatorn i mottagaren endast momentana ändringar i
bärvågsfrekvensen.
Till skillnad från vid FM, så kan därför ändringar i likspänningsnivåer
överföras endast om en fasreferens används.

Med konstant amplitud på insignalen till modulatorn är vid PM
modulationsindex konstant oavsett modulerande frekvens, medan vid FM
modulationsindex varierar med den modulerande frekvensen.
