\section{Innehåll i förbindelse}
\label{innehåll i förbindelse}
\harecsection{\harec{b}{7.2.3}{7.2.3}}

Tidigare har det i Sverige varit reglerat vad innehållet får vara i
förbindelser, eller snarare vad de inte får innehålla.
Den regleringen är numera borttagen.
Man ska vara medveten om att samma regler och förutsättningar inte gäller i
alla länder och för deras radioamatörer.
Därför uppmanas du att använda sunt förnuft, hålla god ton och respektera alla
amatörer.
Se även IARU etik och trafikmetoder.

\subsection{Tystnadsplikt}
\index{tystnadsplikt}
\index{LEK}

Innehållet i en radioförbindelse skyddas av
\emph{Lag om elektronisk kommunikation (LEK)}~\cite{SFS2022:482}.
I LEK regleras tystnadsplikt för radiobefordrade meddelanden i kapitel~6.

\begin{quote}
	Den som i annat fall än som avses i 31~\S{} första stycket och 32~\S{} i
	radiomottagare har avlyssnat eller på annat sätt med användande av sådan
	mottagare fått tillgång till ett radiobefordrat meddelande i ett
	elektroniskt kommunikationsnät som inte är avsett för honom eller henne
	själv eller för allmänheten får inte obehörigen föra det vidare.
	Lag (2022:482).\cite[kap 9, \S33]{SFS2022:482}
\end{quote}

Tystnadsplikten gäller alla radiomeddelanden som avlyssnats, oavsett ursprung.

Detta innebär att om du själv varit part i radiomeddelandet eller om
radiomeddelandet var en nyhetsbulletin avsett för många så får du föra det vidare.

En stor del av radioamatörhobbyn bygger dock på radiokommunikation med andra och
att andra kan höra dig när du sänder.
En radioamatör kan därför inte anses vara omedveten om att någon annan lyssnar
på det som sänds ut.
Därför är mycket accepterat inom amatörradio som annars skulle vara förbjudet.

Tips om rara DX, tips om någon som ropar CQ, QSL från lyssnaramatörer, att
berätta att du hörde någon ha förbindelse med någon annan anses därför normalt
inte vara ett brott mot tystnadsplikten.

Att koppla en radiomottagare till webben så att någon kan lyssna på radiotrafik
i realtid är tillåtet.

\emph{Observera även texten i andra punkten i 6~kap.~20~\S{} gällande den som i
	samband med tillhandahållande av ett elektronisk kommunikationstjänst har fått
	del av eller tillgång till innehållet i ett elektroniskt meddelande inte
	obehörigen får föra vidare eller utnyttja det han fått del av eller tillgång
	till.}

Detta kan vara aktuellt då någon som tillhandahåller en elektronisk
kommunikationstjänst från punkt A till punkt B får tillgång till innehållet i
ett elektroniskt meddelande när det har lämnat punkt A och innan det når fram
till punkt B.

\subsection{Inspelning av radiomeddelande}
\index{inspelning}
\index{GDPR}
\index{Dataskyddsförordningen}

Radiosamtal som du själv deltar i får spelas in utan att andra deltagare i
samtalet informeras om att du spelar in samtalet.

Grundregeln är att inspelning av radiomeddelanden är tillåten såvida inte
inspelningen är förbjuden för att skydda personers personliga integritet.

Uppspelning av de inspelade meddelandena får inte bryta mot bestämmelserna om
tystnadsplikt.
Det vill säga att meddelandet inte obehörigen får föras vidare.

Alla radiomeddelanden får inte spelas in.
Lagstiftningen skiljer även på analoga- och digitala inspelningar.
Dataskyddsförordningen~\cite{GDPR} samt 4~kap.~9a~\S{} i
Brottsbalken~\cite{SFS1962:700} är exempel på lagar som begränsar inspelning av
avlyssnade radiomeddelanden.

Av ovanstående följer att det inte är tillåtet att lagra inspelad radiotrafik
för senare lyssning via webbaserade medier då det kan anses kränka den
personliga integriteten.

\subsection{Kryptering av radiomeddelande}
\label{kryptering av radiomeddelande}
\index{kryptering}

Inom Sveriges gränser är kryptering av radiomeddelanden på amatörradiofrekvenser
tillåten under villkor att en anropssignal regelbundet sänds ut, anropssignalen
ska då kunna avläsas med kända tekniker.
Trots detta rekommenderas inte användning av kryptering för amatörradiotrafik.

Tekniken för kryptering av radiomeddelanden har blivit mera lättillgänglig i
samband med införandet av digitala radiosystem typ DMR (Digital Mobile Radio) på
amatörbanden.
Ett flertal av dessa radiosystem är dock ihopkopplade via internationella
nätverk och därigenom hörbara i flera länder där kryptering inte är tillåten.

Användningen av krypteringsteknik på amatörradiofrekvenser riskerar därför att
medföra begränsningar i de rättigheter vi har enligt PTSFS 2022:19.
