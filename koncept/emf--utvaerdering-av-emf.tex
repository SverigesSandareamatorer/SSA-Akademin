\section{Utvärdering av EMF}
\index{EMF!utvärdering}

Hur är det då med den egna stationen, överensstämmer fältstyrkorna som den kan
generera med referensvärdena?

För att kunna utvärdera detta måste man känna till vilka parametrar som är
avgörande vid en beräkning av fältstyrkan.
Mycket beror på vilken antenn man använder och hur den är placerad.
Man måste även förstå egenskaperna signalen från sändaren har.

\subsection{Antennen}
Antennen tar emot signalen från sändaren via en matningskabel och omvandlar
denna signal till en elektromagnetisk våg.
Hur denna omvandling går till är komplicerat men kan enklast förklaras med
begreppet förstärkning eller antennvinst.
Man måste alltså känna till vilken förstärkning antennen har.
Formeln som används för uträkning av fältstyrkan förutsätter att man benämner
antennens förstärkning relativt en isotrop antenn (dBi) och inte en dipolantenn
(dBd), samt att man räknar med linjära värden (gånger) och inte logaritmiska
(dB).

\begin{table*}[ht]
  \begin{center}
    \begin{tabular}{|l|ccccccccccc|}
    \hline
    dB     &  0  &  1  &  2 & 2,15 &  3  &  4  &  5  &  6  &  7  &  8  &  9  \\ \hline
    G & 1,0 & 1,3 & 1,6 & 1,64 & 2,0 & 2,5 & 3,2 & 4,0 & 5,0 & 6,3 & 7,9 \\ \hline\hline
    dB     &  10  &  11  &  12  &  13  &  14  &  15  &  16  &  17  &  18  &  19  &  20 \\ \hline
    G & 10,0 & 12,6 & 15,8 & 20,0 & 25,1 & 31,6 & 39,8 & 50,1 & 63,1 & 79,4 & 100,0 \\ \hline
    \end{tabular}
    \caption{G = Antennens förstärkning i linjära faktorer}
    \label{tab:forst}
  \end{center}
\end{table*}

För en antenn med förstärkningen 7\,dBi ska alltså värdet 5,0 användas.

\subsection{Sändarsignalen}
Då referensvärdet enligt de allmänna råden är definierade som medelvärden under
en sexminutersperiod är det medeleffekten som ska användas vid beräkningarna.

Effektens medelvärde under en sexminutersperiod beror på två olika saker.

\subsubsection{Modulationsfaktor}
\index{EMF!modulationsfaktor}
\index{modulationsfaktor}

Beroende på trafiksätt så blir medeleffekten olika.
Används FM så ger det modulationssättet att man använder maximal uteffekt
kontinuerligt jämfört med SSB där medeleffekten beror på hur man talar.

Nedanstående tabell är de värden som regelverket i USA~\cite{OETbul65b} använder
för att räkna ut medeleffekten på grund av moduleringen.

\begin{table}[H]
  \begin{center}
    \begin{tabular}{lc}
	\textbf{Trafiksätt} & \textbf{Modulationsfaktor} \\ 
	\hline
	\emph{SSB} & 0,2 \\
	\emph{CW} & 0,4 \\
	\emph{SSB med processing} & 0,5 \\
	\emph{FM} & 1,0 \\
	\emph{MGM (t.ex. RTTY, PSK)} & 1,0 \\
	\emph{Bärvåg} & 1,0 \\
    \end{tabular}
    \caption{Modulationsfaktor per trafiksätt}
    \label{tab:modfakt}
  \end{center}
\end{table}

\subsubsection{Intermittensfaktor}
\index{EMF!intermittensfaktor}
\index{intermittensfaktor}

Vid vanlig amatörradioanvändning sänder man inte kontinuerligt utan både lyssnar
och sänder växelvis.
Sänder man och tar emot lika mycket under en sexminutersperiod så blir faktorn
0,5 men om man lyssnar mycket mer och sänder sällan blir faktorn mindre.

\begin{table}[H]
  \begin{center}
    \begin{tabular}{|c|c|c|}
	\hline
	Sändning  & Mottagning & Intermittensfaktor \\
	(minuter) & (minuter)  & \\ \hline
	1 & 5 & 0,17 \\ \hline
	2 & 4 & 0,33 \\ \hline
	3 & 3 & 0,50 \\ \hline
	4 & 2 & 0,67 \\ \hline
	5 & 1 & 0,83 \\ \hline
	6 & 0 & 1,00 \\ \hline
    \end{tabular}
    \caption{Intermittensfaktor}
    \label{tab:intfakt}
  \end{center}
\end{table}

\subsubsection{Medeleffekt}
\index{EMF!medeleffekt}

För att räkna ut vilken medeleffekt som används ska man ta hänsyn
till både modulationsfaktor och intermittensfaktor enligt följande

\(\textit{Medeleffekt} = \textit{Maxeffekten} \cdot \textit{Modulationsfaktor} \cdot \textit{Intermittensfaktor}\)

\noindent\textbf{P = Medeleffekten under en sexminutersperiod}

\subsection{Kabeldämpning}
\index{EMF!kabeldämpning}

När uteffekten mäts vid sändaren och fältet genereras av effekten som når
antennen måste även den dämpning som matarledaren har vara känd.
Annars överskattas den genererade fältstyrkan.
 
Även här måste linjära enheter användas (gånger).

\begin{table*}[ht]
  \begin{center}
    \begin{tabular}{|l|c|c|c|c|c|c|c|c|c|c|c|}
	\hline
	dB & 0,0  & 0,5  & 1,0  & 1,5  & 2,0  & 2,5  & 3,0  & 3,5  & 4,0  & 4,5  & 5,0 \\ \hline
	k  & 1,00 & 0,89 & 0,79 & 0,71 & 0,63 & 0,56 & 0,50 & 0,45 & 0,40 & 0,35 & 0,32 \\ \hline
    \end{tabular}
    \caption{k = Matarkabels dämpning i linjära termer}
    \label{tab:feedannut}
  \end{center}
\end{table*}

För en kabel med dämpningen \qty{2,5}{\decibel} ska alltså värdet 0,56 användas.

\subsection{Avstånd}
\index{EMF!distans}

På vilket avstånd är det intressant att veta vilken fältstyrka som genereras?
De allmänna råden definierar detta på följande sätt

”1.3  Dessa allmänna råd omfattar områden där allmänheten kan vistas under sådana tider att begränsningarna är av betydelse.”

Ett bra utgångsläge är då att utvärdera området där antennen är placerad och
bedöma var allmänheten kan exponeras.
\\[1ex] % layout
\noindent\textbf{d = Avståndet från antennen till platsen där fältstyrkan ska bestämmas}

\subsection{Beräkning}
\index{EMF!beräkning}

Beräkning av det elektromagnetiska fältet kan med enkelhet bara genomföras i
fjärrfältet från en antenn.
I fjärrfältet vet vi sedan tidigare att man antigen kan utvärdera det elektriska
eller magnetiska fältet.
Av denna anledning beskrivs enbart beräkning av det elektriska fältets del av
det elektromagnetiska fältet. 

\noindent\textbf{E = Det elektromagnetiska fältets storlek i fjärrfältet}

Det elektromagnetiska fältets storlek (i fjärrfältet) räknas ut från
effekten (medelvärde), antennförstärkningen, matarledningens dämpning
och avståndet enligt följande förenklade formel:
%%
\[E=\dfrac{\sqrt{30 \cdot P \cdot G \cdot k}}{d}\]
%%
Genom enkel matematik kan man då använda samma formel för att räkna
ut på vilket avstånd man genererar en viss fältstyrka.
%%
\[d=\dfrac{\sqrt{30 \cdot P \cdot G \cdot k}}{E}\]
%%
Denna beräkning är enbart relevant för huvudloben.
Fältet under antennen beräknas inte, och därför kan resultatet inte användas
för att bedöma höjd på eller säkerhetsavstånd till antenntorn.
Använd datorprogram för att få bra bedömning på hur en antenn beter sig,
särskilt med avseende på antenner med riktverkan.

\begin{exempelbox}
En riktantenn för \qty{144}{\mega\hertz} med förstärkning enligt databladet på
14,92\,dBi (31 gånger).
Max uteffekt är \qty{1000}{\watt} och trafiksättet är MGM med 30~sekunders
intervaller.
Den valda matarledningen har en dämpning på \qty{2,5}{\decibel} (0,56~gånger).
Avståndet från antennen till beräkningspunkten är \qty{15}{\metre}.

Vilken medelfältstyrka genererar man på ett visst avstånd från antennen?
\tcblower
\noindent
\[P_{medel} = P_{pep} \cdot k_{mod} \cdot k_{if}
= 1000 \cdot 1 \cdot 0,5 = 500\]
\[k = 0,56\]
\[d = 15\]
\begin{align*}
  E &= \dfrac{\sqrt{30 \cdot P \cdot G \cdot k}}{d} =\\
&= \dfrac{\sqrt{30 \cdot 500 \cdot 31 \cdot 0,56}}{15}
= \qty{34,02}{\volt\per\metre}
\end{align*}

Då referensvärdet på denna frekvens är \qty{28}{\volt\per\metre}, överskrider
amatörradiosändningen referensvärdet på detta avstånd.
\end{exempelbox}

\begin{exempelbox}
En riktantenn för \qty{144}{\mega\hertz} med förstärkning enligt databladet på
14,92\,dBi (31 gånger).
Max uteffekt är \qty{1000}{\watt} och trafiksättet är MGM med 30~sekunders
intervaller.
Den valda matarledningen har en dämpning på \qty{2,5}{\decibel} (0,56~gånger).
Referensvärdet för \qty{144}{\mega\hertz} är \qty{28}{\volt\per\metre}.

På vilket avstånd från antennen når man referensvärdet?
\tcblower
\noindent
\[P_{medel} = P_{pep} \cdot k_{mod} \cdot k_{if}
= 1000 \cdot 1 \cdot 0,5 = 500\]
\[k = 0,56\]
\[E = 28\]
\begin{align*}
  d &= \dfrac{\sqrt{30 \cdot P \cdot G \cdot k}}{E} =\\
&= \dfrac{\sqrt{30 \cdot 500 \cdot 31 \cdot 0,56}}{28}
  = \qty{18,22}{\metre}
  \end{align*}

För att följa de allmänna råden bör allmänheten inte kunna vistas i huvudloben
framför antennen på ett avstånd mindre än \qty{19}{\metre} då sändning utförs
enligt exemplet.
\end{exempelbox}
