\section[Kraftfält antenner]{Kraftfälten omkring antenner}
\index{kraftfält}
\index{antenner!kraftfält}

För att sända ut och ta emot radiovågor behövs antenner.
Mycket förenklat är en antenn en elektrisk krets, som består av en induktor
och en kondensator som illustreras i bild~\ssaref{fig:BildII7-01}.

Med kondensatorns elektroder helt isärdragna och förminskade har
resonanskretsen fått ett mycket annorlunda mekaniskt utseende.
Sedan induktorn i LC-kretsen tagits bort, så återstår mekaniskt sett endast
en enkel ledare, men elektriskt sett finns kretsen ändå kvar.
Ledaren med sin utsträckning är fortfarande en induktor och ytorna på dess
motstående halvor är fortfarande elektroderna i kondensatorn med
omgivningen som dielektrikum.

En elektrisk ledare, en stång, tråd etc. är alltså en elektrisk
resonanskrets, vars resonansfrekvens mest bestäms av längden och
tjockleken. Ledaren (antennen) kan kallas dipol -- den har två poler,
detta är grunden för alla typer av antenner.

\mediumplusbotfig{images/cropped_pdfs/bild_2_7-01.pdf}{Från sluten LC-resonanskrets till antenn}{fig:BildII7-01}
\mediumfig{images/cropped_pdfs/bild_2_7-02.pdf}{Pendlingen mellan E-fält och H-fält}{fig:BildII7-02}

Det finns vissa likheter mellan en mekanisk pendel och en elektrisk
resonanskrets.
Energin i en mekanisk pendel växlar mellan två ytterlighetstillstånd.
Det ena är när pendeln just vänder i ett ytterläge.
Då innehåller den enbart lägesenergi och ingen rörelseenergi.
När pendeln rör sig mot mittläget, så omvandlas lägesenergin till rörelseenergi.
I mittläget, som är det andra ytterlighetstillståndet, innehåller pendeln enbart
rörelseenergi och ingen lägesenergi etc.

\subsubsection{Elektrisk resonanskrets}
\index{kraftfält!elektrisk resonanskrets}
\index{elektrisk resonanskrets}
\index{Maxwell}

Den elektriska resonanskretsen kan jämföras med den mekaniska pendeln där det
hela tiden pågår en pendling eller omvandling mellan lägesenergi och
rörelseenergi.
Se bild~\ssaref{fig:BildII7-02}.

När strömmen i den elektriska resonanskretsen just upphört för att vända så
innehåller kondensatorn mest laddning, det vill säga, det starkaste elektriska
fältet mellan elektroderna.
Detta fält kan jämföras med pendelns lägesenergi.
Den utjämningsström som följer från den ena elektroden över till den andra
omges av ett magnetiskt fält som kan jämföras med pendelns rörelseenergi.

Förloppet visas i bild~\ssaref{fig:BildII7-02}, där det framgår att dipolen omges
av det starkaste elektriska fältet vid tidpunkten \(t=0\) samt vid
\(t=1/2T\) med omvänd polaritet, där T är periodtiden.
Vidare att dipolen omges av det starkaste magnetiska fältet vid tidpunkten
\(t=1/4T\) samt vid \(t=3/4T\) med omvänd strömriktning och fältpolaritet.


\smallfig{images/cropped_pdfs/bild_2_7-03.pdf}{Elementär dipol}{fig:BildII7-03}

Med förklaringen av E- och H-fälten som bakgrund följer nu en enkel
framställning av hur radiovågor uppstår ur dessa fält.

Maxwell påvisade i sina ekvationer bland annat sambandet mellan elektroner
i rörelse i en ledare och elektromagnetiska vågor i rummet.
Vidare, att elektroner som rör sig med avtagande eller tilltagande hastighet
avger elektromagnetisk energi.

Hur energi strålar från en ledare kan förklaras med en (tänkt)
elementär dipol, som genomflyts av växelström (Bild~\ssaref{fig:BildII7-03}).

Dipolen består av två lika stora elektriska laddningar med motsatt polaritet.
När den matas med en växelström, så rör sig laddningarna ständigt,
omväxlande emot respektive ifrån varandra.
Tänk på två kulor i var sin ände av en spiralfjäder.
Avståndet mellan laddningarna ändras i takt med styrkan och riktningen på
strömmen.
Systemet är alltså under ständig hastighetsändring (ökning respektive
minskning), vilket är förutsättningen för att energi ska strålas ut.

Först är laddningarna nära varandra på grund av liten laddning.
Vid ökande ström ökar avståndet mellan laddningarna och det byggs upp ett
mer utbrett och energirikt E-fält.
Samtidigt byggs även ett H-fält upp omkring dipolen, vinkelrätt mot E-fältet
och så vidare.
Detta gäller både för en elementär dipol och en elektrisk ledare med många fria
elektroner (verklig antenn).

Formeln för det resulterande S-fältet är \(\overline{S} =
\overline{E}\times\overline{H}\), vilket visar att den lagrade energin
i dipolens närmaste omgivning ökar när avståndet (potentialen) mellan
dipolens laddningar ökar.

Bild~\ssaref{fig:BildII7-04} visar hur ett E-fält byggs upp omkring en dipol och
avskiljs från den.
De visade kraftlinjerna är E- fältet.
H-fältet visas inte, men ligger vinkelrätt mot E-fältet, i cirklar omkring
antennen. Se bild~\ssaref{fig:BildII7-05}.

\mediumfig{images/cropped_pdfs/bild_2_7-04.pdf}{Ett självständigt E-fält skapas}{fig:BildII7-04}

\mediumfig{images/cropped_pdfs/bild_2_7-05.pdf}{E-, H- och S-fälten omkring en antenn (förenklad framställning)}{fig:BildII7-05}

När dipolens laddningar ändrar riktning och åter börjar att röra sig
emot varandra, börjar det E-fält som byggts upp att också byta riktning.
Men det kommer inte att falla tillbaka till dipolens mitt
utan sluts till ett eget kretslopp -- Maxwells första ekvation.
Jämför med en såpbubbla som lämnat blåsröret.
Omkring dipolen har det nu bildats ett självständigt E-fält, som sin tur
alstrar ett eget H-fält.

En period av en elektromagnetisk våg (ett S-fält) har alstrats och
fortsätter att utvidga sig.
För varje följande period alstras ett nytt E-fält, som separeras från antennen
och bildar ett H-fält och så vidare.
Varje gång bildas alltså en ny ''fältbubbla'' inne i den föregående, vilken
håller på att utvidgas.
Resultatet är ett elektromagnetiskt fält, det vill säga en radiovåg.

Som nämnts består en radiovåg av ett högfrekvent elektromagnetiskt fält (S).
Det är i sin tur sammansatt av två andra fält, det elektriska E-
och det magnetiska H-fältet.
Energin i S-fältet fördelas lika mellan E-fältet och H-fälten,
vars krafter korsar varandra vinkelrätt.
S-fältet ligger i plan med både E- och H-fälten och breder ut sig vinkelrätt
mot dem.
S-fältets riktning beror av den inbördes riktningen på E- och H-fälten.

När E-fältet är vertikalt, sägs vågen vara vertikalt polariserad.
När samma fält är horisontellt sägs vågen vara horisontellt polariserad.
När E-fältet roterar i vågfrontens plan, och därmed även H-fältet, sägs vågen
vara cirkulärt polariserad.

Fälten framställs i text och bild som så kallade kraftlinjer med pilar som
föreställer kraftriktningen.
Linjernas längd föreställer fältets styrka.
Bild~\ssaref{fig:BildII7-06} visar ett avsnitt av en vågfront S med vertikal
polarisation.

\smallfig{images/cropped_pdfs/bild_2_7-06.pdf}{E-, H- och S-fält}{fig:BildII7-06}
