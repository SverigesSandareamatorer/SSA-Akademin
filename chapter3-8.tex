\subsection{Frekvensblandare}

Grundprinciper

En anordning som blandar signaler för att
skapa andra kallas som namnet säger för
blandare. Blandare används både i mottagare och sändare och funktionsprinciperna
är lika i båda fallen. Vad som skiljer i stort är
hur de används.
Det finns många blandarkopplingar varav de vanligaste beskrivs här. Enkla typer
med vissa nackdelar ställs mot sådana som
är mer komplicerade, men har fördelar.

Bild II 3-83
När en linjär förstärkare matas med två
signaler så sammanlagras de. Den resulterande signalen vid varje tidpunkt är den
förstärkta summan av de inmatade signalerna.
När en olinjärförstärkare matas med två
signaler så blandas de med varandra. Förutom ingångssignalerna uppträder genom
blandningen ytterligare signaler på förstärkarutgången, så kallade blandningsprodukter.

ADDITION AV TVÅ SIGNALER
l ngångssignaler:
Växetspänningar med frekvensen f1

Utsigna!:

u

v.

u

och frekvensen f2

t U1 + U2

Frekvensspektrum

f,

v

l

f2

Endast ingångsfrekvenserna
uppträder i utsigna!en

BLANDNING AV TVÅ SIGNALER

u

u1~1ngångssignaler
fl

Utgångssignal

--·- -----··--.. ·-········,..-t

.....-

Frekvensspektrum

Det uppträder ytterligare frekvenser
utöver ingångsfrekvenserna

symbol för en blandare

f1~ Utgångbia f1, f2, f1
f2

t- f2, f2- f1

fz

>f1

Bild II 3-83 Principer för frekvensblandning

113-63

KR
Det finns ingen förstärkare i kopplingen.
signalspänningarna adderas genom att de
två transformatorernas sekundärlindningar
är seriekopplade. Dioden "förvränger" kraftigt summaspänningens kurvform. Beroende
av hur dioden är polariserad (vänd i kopplingen) blir den negativa eller den positiva
halwågen bortskuren.
Signalen på blandarens utgång, alltså
efter dioden, innehåller bl.a. frekvenserna f 1 ,
f2 , f2 +f1 , f2 -f1 • Den lägsta frekvensen f1 kan
lättast påvisas genom att ansluta ett lågpassfilter till blandarens utgång.

Två av blandningsprodukterna är särskilt
intressanta, det är summan och skillnaden
av ingångssignalernas frekvenser. De oönskade, övriga blandningsprodukterna filtreras bort med en avstämd krets eller ett
bandpassfilter.

Entaktsblandaren
Bild 113-84a
Vi kan övertyga oss om, att de fyra blandningsprodukterna verkligen uppstår. Först
undersöker vi den enklaste blandaren, som
är ett olinjärt element i form av en diod.

u1

l ngångssignaler:
f1

u

:J u, I~T
Ut+ U2

~J~ . .

Utgångssig nal

Ua

t

~~---t
u

(utan LP-filter)

l

U

Framtagning av lägsta frekvens fl

Bild II 3-84a Entaktsblandaren

113-64

Frekvensspektrum

L

b

f1

1 Lt.~

1 11
'.


1

KRETSAR

:J:

t

.
[) l "
N-r-,l· ·- ·r llillnnwnlliLI
n nn "

l

U

l

T.I l"< lil~
lm~

=: - ---

-t

!

~~~~w~~

Frekvensspektrum

f2 tf 1

'---'-----'-....L--lll.-, f

(utan "'ngkmt.

med "ängk""

-t

-t

Framtagning av frekvenserna f2, f1 + f2, f2 - f1

u

k~.
Al . -~ /'--- .-- /t+,r·~\ -· / /' . ,.
···'\ ::.

,

u

l

u

l

Signal med f2 - fl
(skilinadsfrekvens)

t

Signal med f2 + f1
(summafrekvims)

--~-

-::y-'(-

l

~-\---1--+-4-·-.~~~<~ ~.-·.t···. ~-~---·-t ~~~na;;r :e11 f:d~r:~e
--+-- /

...

- 1T\--

UPtAJ\ f\ ~--t
u ,..--""

l

l
1

--,, .,

t--+---+-+-+--

'',..

-·

//

l
l·

-j:~·:"'-1---'

/

l

l

Signal med f2

/

:'.\, /

''< -..

-•·l

Alla tre signalerna adderade

l

Bild II 3-84b Entaktsblandaren

113-65

KRETSAR
Resultatet kan studeras med ett oscilloskop. Liksom på bilden ser man då att kondensatorn laddas upp till den positiva halvvågens toppvärde och med gott närmevärde
följer kurvformen på f 1 •
Bild II 3-84b
En svängningskrets med lämplig bandbredd, och som är avstämd till resonans
frekvensen f2 , ansluts nu till blandarens
gång. En signal med frekvensen f2 kan då
urskiljas och studeras i oscilloskopet. Svängningskretsen tillförs energi under de positiva
halvvågorna. Energin i svängningskretsen
kompletterar med den negativa halvvågen,
varvid en del av kretsens energi förbrukas.
Därför har de positiva och negativa halvvågorna inte samma amplitud (toppvärde).
Det syns i oscilloskopet hur amplituden
"svävar". Av detta dras slutsatsen att signalen består av fler frekvenser än f2 • Signalen
är sammansatt av f2 , f2 +f1 och f2 -f1 • Signalen
f1 ligger utanför svängningskretsens selektiva område och är därför bortfiltrerad (undertryckt). Blandningsprodukterna f2 +f 1 och f2 -f1
har båda en mindre amplitud än f2 •
Att det finns olika grundtoner och blandningsprodukter kan bevisas med en ännu
smalare svängningskrets med variabel frekvensavstämning, se bildens nedre del.
Vi har hittills utgått från entaktsblandaren.
Mer utvecklade blandartyper, såsom mottaktblandaren och ringblandaren, producerar färre blandningsprodukter.

Mottaktsblandaren
Bild II 3-85
Mottaktblandaren har två dioder, till skillnad
motentaktsblandarens enda diod ...................,,.,...
formatoremas ena lindning har mittuttag.
Ingången E1 ligger på den ena transformatorns primärlindning.lngången E2 1iggeröver
de båda mittuttagen. Utgången ligger på den
andra transformatorns sekundärlindning.
Ingången E1 matas med en signal med
en låg frekvens f. Eftersom en av de båda
dioderna alltid spärrar, så flyter det ingen
ström. De streckade pilarna visar endast i
vilken riktning strömmen kunde flyta, om de
spärrande dioderna vore öppna. Men så
länge som ingen signal ligger på ingång E2 ,
uppträder ingen signal på utgången.
113-66

Signalen på
avlägsnas och i stället
matas ingången
med en hög frekvens F.
Under den positiva halvvågen är de båda
dioderna öppna och genom båda flyter lika
stor ström. De båda transformatorernas
lindningshalvor genomflyts av lika ström i
motsatt riktning och då upphäver magnetfälten i lindningshalvorna varandraoch ingen
uppträder på utgången.
När signaler läggs
båda ingångarna
händer följande:
Dioderna öppnar och stänger i takt med
signalen på ingång E2 , med frekvensen F.
Den mycket svagare signalen på ingång E1 ,
med frekvensen f, kan alltefter polaritet passera diod D 1 eller D 2 • På återvägen överlagras signalen från E1 på signalen från E2 •
Strömmarna i lindningshalvorna är olika
stora. Då uppträder en signal på utgången.
Efter blandaren följer ett filter som endast
släpper igenom de önskade blandningsprodukterna F + f eller F - f.

Ringblandaren
Bild 113-86
Ringblandaren består av fyra dioder, som är
riktade åt samma håll i en "diodring".
ingången
matas med en signal med en
låg frekvens f. Till skillnad mot i mottaktsblandaren flyter en ström genom Di och D4
resp. D2 och 0 3 , men inte genom utgångstransformatorn. Ingen signal finns på utgången så länge som signalen F saknas.
Signalen på E 1 avlägsnas och i stället
matas ingången E2 med en hög frekvens F.
Till skillnad mot i mottaktsblandaren flyter en
ström genom dioderna Di och D2 resp. D3
och D4 och då upphäver magnetfälten i
transformatorernas lindningshalvor varandra. Ingen signal finns på utgången, så länge
som signalen f saknas.
När signaler läggs på båda ingångarna
händer följande:
De fyra dioderna kommer att öppna och
stänga parvis. Som i mottaktblandaren överlagras strömmen från ingång E1 på den
ström som dioderna öppnar för.
Här utnyttjas båda halvperioderna av F.
Strömmarna i lindningshalvorna blir olika
stora. På utgången uppträder då en signal.
Efter blandaren följer ett filter som släpper
igenom de önskade blandningsprodukterna.

KRETSAR
Bara signalen

e1

med frekvensen f ligger på
01

Bara signalen E2 med frekvensen f ligger på

båda dioderna öppnar

båda dioderna spärrar

Båda signalerna ligger på

Bild If 3-85 Mottaktsblandaren

113-67

R
Bara signalen E1 med frekvensen f ligger på

o,

Bara signalen E2 med frekvensen f ligger på

Båda signalerna ligger på

non no

lUlU

Bild II 3-86 Ringblandaren

113-68

KRETSAR

Signal utan svängkrets

Entaktsblandare
Frekvensspektrum

u

ut
nttöl\ 6tL.... n/\ nA~a

"

~t

~o~v~v~v~v~v~v~-~v~v\ v~v*v~v~o~v~~

li lL. . ~ ~~ ~ 11.
--
Signal med svängkrets

Signal utan svängkrets

Mottaktsblandare

u

.....

l

l..

+

LL

•

+

, 

f

Frekvensspektrum

u

eller

Signal utan svängkrets

Signal med svängkrets

;t;

il

M

Ringblandare

l

LL

M

-..
LL
M

ll

;;;

..

LL
M

... f

Frekvensspektrum

Bild 113-87 Jämförelse mellan olika blandare

113-69

KRETS R
Jämförelse av blandare

Bild 113-87
Bilden visar de tre beskrivna grundkopplingarna och de jämförs med avseende på frekvensspektrum på utgången.
Videntaktsblandaren uppträder summafrekvensen f+ F och skillnadsfrekvensen Ff, vidare ingångsfrekvenserna f och F, deras
övertoner 2f, 3f, 4f o s v, 2F, 3F, 4F o s v,
liksom deras blandningsprodukter F\(\pm\) 2f, F\(\pm\)
3f o.s.v., 2F \(\pm\)f, 2F \(\pm\) 2f, 2F \(\pm\) 3f o.s.v.
Vid mottaktblandaren saknas frekvensen F och dess övertoner. Vidare bortfaller
de jämna övertonerna av frekvensen f.
Vid ringblandaren bortfaller ännu fler icke
önskvärda signaler, nämligen ingångssignalerna f och F och alla deras övertoner.
Endast blandningsprodukter av udda övertoner uppträder.
På bilden visas det fallet att frekvensen f
är mycket låg och då ligger blandningsprodukterna mycket nära varandra i frekvens.
Videntaktsblandaren filtrerar svängningskretsen ut frekvenserna F+ f, F- f, och F. Vid
mottakt- och ringblandaren saknas däremot
frekvensen F, den filtrerade signalen innehåller endast blandningsprodukterna F + f
och F- f. Om dessa båda blandningsprodukter är väl åtskilda eller svängningskretsen
har en bättre selektionsförmåga, då blir enbart summafrekvensen F + f eller skillnadsfrekvensen F- f framfiltrerad.
Vi har visat tre typer av blandare med
passiva komponenter. Sådana innehåller
olinjära dioder (germanium- eller kiseldiade r).
Det finns även blandare med aktiva komponenter, d.v.s. elektronrör eller transistorer
(bipolära, FET, MOSFET), men det skulle
leda för långt att gå in på alla olika lösningar.
l kapitlen 4. Mottagare och 5. Sändare beskrivs hur frekvensblandning används för
modulering och demodulering.

Icke önskade övertoner och blandningsprodukter

Varje olinjärt arbetande funktionssteg alstrar förutom nyttafrekvenser även icke önskade signaler med andra frekvenser. Både
önskade och icke önskade signaler kan bestå av övertoner eller blandningsprodukter
(skillnads- och summatoner) eller bådadera.
113-70

Vissa av signalerna filtreras fram för att
utgöra nyttosignaler. Andra signaler filtreras
bort, så att t. ex. utsändning inte sker på fel
frekvenser.
Bild 113-88
l ett tidigare avsnitt har vi beskrivit en s.k.
super-VFO. Vi skall nu undersöka vilka blandningsprodukter som uppstår i en sådan. De
två mest uppenbara frekvenserna är blandningsprodukterna (summan) i området 144146 MHz och (skillnaden) i området 128-126

MHz.

Ut från blandaren finner vi ingångsfrekvensen 136 MHz och dess övertoner 272
MHz, 408 MHz o.s.v. såväl som VFO-signalen och dess övertoner. På bilden är VFOfrekvensen 8 MHz och dess övertoner inritade, d.v.s. 16 MHz, 24 MHz, 32 MHz o.s.v.
Tyvärr bildar också de båda ingångssignalernas övertoner blandningsprodukter vilket bilden visar.
Bandpassfiltret släpper igenom nyttafrekvensen, men dämpar alla övertoner och
blandningsprodukter. Detta är enklare ju
längre ifrån nyttasignalen de icke önskade
signalerna ligger. l vårt exempel faller VFOsignalens övertoner inom bandpassfiltrets
passband på följande sätt:
15 · 9.6 = 144 MHz till15 · 9.733 = 146 MHz
16 · 9.0 = 144 MHz till16 · 9.125 = 146 MHz
17 · 8.471 =144 MHz till17· 8.588 = 146 MHz
18 · 8.0 = 144 MHz till18 · 8.111 = 146 MHz
Eftersom det här handlar om 15 :e-18 :e
övertonerna, så blir amplituderna så små, att
vi kan bortse från dem.
Det är viktigt med goda filter i signalbehandlande funktionssteg. En god regel är att
på ett tidigt stadium filtrera bort oönskade
övertoner och blandningsprodukter-helst i
varje steg -så att onödigt komplexa signaler
undviks. Det är också viktigt med frekvensvalet, så att oönskade blandningsprodukter
kommer så långt bort från nyttafrekvensen
som möjligt, liksom att endast mycket höga
övertoner med motsvarande små amplituder faller inom det nyttiga frekvensområdet

ETSAR

Super-VFO för 144-146 MHz

BLANDARUTGÄNGENs FREKVENsSPEKTRUM

U

f 1 =8MHz
passbandkurva
sidfrekvens

nyttasignal

f,df.
144 MHz

f2 -· f1
128 MHz

a) Förenklad framställning med VFO:n avstämd till 8 MHz

u

9,5 MHz

136MHz

126,5 MHz·--·----1 ffi--·-,.145,5 MHz

........... . .L111.

b) Förenklad framställning med VFO:n avstämd på 9,5 MHz

u

8MHz
16 MHz
24 MHz

l / )2

MHz osv

136 MHz
144 MHz

128 MHz

\

ll

1--11--"--..--A--.li--.S..-"-.L..-JL.........L

.....

27 2 MHz o s v

c) Ingångssignalens övertoner med VFO:n avstämd på 8 MHz

u

136

BMHz

16MHz
/24 MHz

/32MHz

l

osv

(136-BH1flz
( 136-16 M~z\

(136-2'dM!1zL

~1Hz

/

(136•8lt11fz
( 136+ 16) MHz

/(136+24lMHz

l

osv

( 2'72--B l ~1H7
(272+16lMHz\
\

272MHz
( 272+ Bl MHz

j(272+16lMHz

l

osv

l.LL--L---   .~~ aI...Jl,I......JII.-.L..-A(..I...JLI-'''---- f

d Blandningsprodukter från blandning av övertoner

Bild II 3-88 Frekvensspektrum från en
113-71

KRETS R

113-72
