\section{Icke sinusformade signaler}
\textbf{HAREC a.\ref{HAREC.a.1.7}\label{myHAREC.a.1.7}}

\subsection{Grundton, övertoner- Kantvågor}

\begin{figure*}
\begin{center}
\includegraphics[width=14cm]{images/bild_2_1-18}
\caption{Ren sinusvåg och övertonshaltig våg}
\label{fig:BildII1-18}
\end{center}
\end{figure*}

Bild \ref{fig:BildII1-18}.

Ett sinusformat förlopp med en enda frekvens - en enda ton - sägs vara spektralt ren.
En sådan svängning kallas för grundton.

Varje signal, som inte är sinusformad, är sammansatt av flera sinussvängningar. Det är
signalens grundton samt dess harmoniska övertoner, vilka kan ha 2, 3 o.s.v. gånger högre
frekvens än grundtonen. Den inbördes styrkan på grundton och övertoner avgör signalens
form. Om signalen ligger inom det hörbara området, kan man märka hur den ändrar karaktär
beroende på övertonshalten. Man kan säga att övertonerna modulerar grundtonen.

\begin{figure*}
\begin{center}
\includegraphics[width=14cm]{images/bild_2_1-19}
\caption{Uppdelning av en signal i grundton och övertoner}
\label{fig:BildII1-19}
\end{center}
\end{figure*}

Bild \ref{fig:BildII1-19}.

Oscillatorsignalen i exemplet på bilden har 1 volts amplitud på grundtonen \(f_0\) (1:a
harmoniska), 0.7 volts amplitud på de  1:a övertonen (2:a harmoniska) och 0.2 volts
amplitud på den 2:a övertonen (3:e harmoniska). Den totala amplituden blir emellertid inte
summan av 1, 0.7 och 0.2 volt eftersom de olika delspänningarnas toppvärden inte uppträder
samtidigt. I stället måste delspänningarna adderas vid varje tidpunkt för sig.

\begin{figure*}
\begin{center}
\includegraphics[width=14cm]{images/bild_2_1-20}
\caption{Uppdelning av en fyrkantvåg i grundton och övertoner}
\label{fig:BildII1-20}
\end{center}
\end{figure*}

Bild \ref{fig:BildII1-20}.

Det finns olika karaktärer av förlopp såsom sinusvåg, triangelvåg, sågtandsvåg,
fyrkantvåg o.s.v.

Fyrkantvågen är sammansatt av sinusvågor med grundfrekvensen och dess udda övertoner,
varvid amplituderna fördelar sig som \(1/1\), \(1/3\), \(1/5\), \(1/7\), \(1/9\), \(1/11\) o.s.v. Teoretiskt når
övertonsspektrum upp till oändligt höga frekvenser, medan de motsvarande amplituderna
minskar till oändligt små värden.

En ideal fyrkantvåg, vilken inte kan uppnås i praktiken, skulle bestå av ett oändligt
antal udda övertoner med fallande amplitud. Ju fler av de högre övertonerna som filtreras
bort, desto mer lutar fyrkantvågens flanker, desto rundare blir hörnen på vågen och
desto vågigare blir kurvans topp.

\subsection{Överlagrade spänningar
(likspänningskomposant)}

\begin{figure*}
\begin{center}
\includegraphics[width=14cm]{images/bild_2_1-21}
\caption{Överlagrade spänningar}
\label{fig:BildII1-21}
\end{center}
\end{figure*}

Bild \ref{fig:BildII1-21}.

I signalkretsar förekommer det mycket ofta, att växelspänning överlagras på likspänning
eller omvänt. Likspänningen kallas då för likspänningskomposant. Olika åtgärder behövs för
att överlagra spänningar på varandra och att sedan skilja dem åt.

Bilden visar ett avsnitt av en AM-mottagare. Från vänster hämtas en AM-modulerad signal
från MF-förstärkaren för att demoduleras, d.v.s. för att återvinna den modulerande
LF-signalen. MF-signalen halvvågslikriktas. Kvar blir den positiva delen av MF-signalen
och den modulerande LF-signalen, sammanlagrade. LF-signalen skall nu skiljas ut och
förstärkas. Alltså filtreras MF-komposanten bort. Kvar blir LF-signalen, men överlagrad på
en likspänning. Likspänningen stoppas och kvar blir slutligen LF-signalen som förstärks.
