\section{Antenner - allmänt}

Aldrig så förnämliga radioapparater kommer inte till sin fulla rätt
utan ett effektivt antennsystem. Det är en huvudförutsättning för
framgångsrik radiokommunikation.

Antennen omsätter elektrisk energi från sändaren till
elektromagnetiska fält som strålas ut, d.v.s. radiovågor.

Vid mottagning fångar antennen upp radiovågorna och omsätter dem till
elektriska signaler som förs till mottagaren.

Antennsystemet består av den egentliga antennen och
transmissionsledningen mellan denna och sändaren respektive
mottagaren. I antennsystemet ingår även impedansanpassningar,
antennkopplare m.m.


\subsubsection{Våghastighet}

I vakuum breder elektromagnetiska vågor ut sig med hastigheten
\(c_0\), vilken mest kallas ljushastigheten.

\[c_0 \approx 300 \cdot 10^6 \text{ [m/s]}\]

I andra media än vakuum har samma vågor utbredningshastigheten
\(c\). Formeln är då

\[c = \frac{c_0}{\sqrt{µ_o \cdot \varepsilon_r}}\]

där \(µ_0\) är relativa permeabilitetskonstanten och \(\varepsilon_0\)
är relativa dielektricitetskonstanten för det medium som vågorna
passerar igenom.  För enkelhetens skull sätts här \(µ_0\) och
\(\varepsilon_0\) till 1, alltså \(c_0 = c\).

Sambandet mellan våghastigheten i vakuum, frekvensen och våglängden är förenklat

\[ c = \lambda \cdot f \quad c\text{ [m/s] }f\text{ [Hz] }\lambda\text{ [m]}\]

och våglängden således

\[ \lambda = \frac{c}{f} \quad \text{[m]} \]

\subsection{Antennlängd}

\subsubsection{Elektriska längden}

Längden för en resonant, ideal antenn som är en våglängd lång kan
beräknas med ovanstående formel. Vi kallar denna längd för den
\emph{elektriska} längden. Således \(l_e = \lambda\).

Elektriska längden (\(l_e\)) för en halvvågsantenn (\(\lambda/2\))
är hälften av den elektriska längden för en helvågsantenn
(\(\lambda\)):

\[l_e = \frac{c}{2f} \quad \text{[m]}\]

\subsubsection{Mekaniska längden}

Man skiljer på antennens elektriska och mekaniska längd. Av flera
orsaker blir den mekaniska antennlängden (\(l_m\)) för samma
frekvens kortare än den elektriska (\(l_e\)). Det beror bl.a. på
våghastighet och ledningsförmåga i de material som ingår samt övriga
elektriska egenskaper beroende på antennens mekaniska utförande,
påverkan från jordplan och omgivning m.m.

Ett förhållande mellan längd och tjocklek av 10000 ger t.ex. en c:a
2\% mekaniskt kortare antenn. Förhållandet 30 ger en c:a 5\% kortare
antenn. Det första värdet kan passa för en 2 mm tjock halvvågsantenn
för 7 MHz. Det andra värdet för en 3.5 mm tjock halvvågsantenn för 145
MHz. Diagram för den s.k. förkortningsfaktorn finns i de flesta
antennhandböcker.

I följande formel har den mekaniska längden (\(l_m\)) för en fritt upphängd
trådantenn valts 2\% kortare än den elektriska längden.

\[l_m = \frac{c}{2f} \cdot 0.98 = \frac{147\cdot 10^6}{f} \quad \text{[m]}\]

Exempel: Beräkna den elektriska och mekaniska längden på en halvvågsantenn med
resonansfrekvensen \(f = 7\) MHz.

\[
c = \lambda \cdot f
\quad c\ \text{[m/s]} \quad f\ \text{[Hz]} \quad \lambda \text{[m]}
\]

Elektriska våglängden för 7 MHz är

\[
\lambda = \frac{c}{f} = \frac{300 \cdot 10^6}{7 \cdot 10^6} = 42.86
\quad \text{[m]}
\]

Antennen är en halvvågsantenn, således är elektriska längden

\[
l_e = \frac{\lambda}{2} = \frac{42.86}{2} = 21.43 \quad \text{[m]}
\]

och mekaniska längden

\[
l_m = \frac{\lambda}{2f} \cdot 0.98 = \frac{42.86}{2}\cdot 0.98 = 21
\quad \text{[m]}
\]

Bild II 6-1 Spänning och ström i en halvvågsantenn

Bild II 6-2 Matningsimpedansen i en halvvågsantenn

\subsection{Ström och spänning i en halvvågsantenn}

%% TODO resten av den här filen

När en halvvågsantenn matas med HF-energi på grundfrekvensen, så uppstår en stående våg med ett typiskt utseende.
Bild II 6-1 visar att i vardera änden av
antennen uppnår spänningen U ett maximum (en spänningsbuk), l mitten uppnår
strömmen I ett maximum (en strömbuk).
Antennen strålar mest där strömbuken finns .
Tag t.ex. en 40 meter lång metalltråd
som antenn. Dess grundresonansfrekvens
är ca 3.5 MHz, men den är även i resonans
på de harmoniska övertonerna (7, 14, 21,
28 MHz o.s.v.).
Bild II 6-3 visar ström- och spänningsfördelningen på antennen vid de respektive
övertonerna.
80 m (3.5 MHz):
l matningpunkten står ett spänningsminimum (en spänningsnod) och ett strömmaximum (en strömbuk). Strömmen är hög
därför att matningspunkten har låg impedans.
Samma antenn på 40 m, 20 m, 15 m, 1O
m (7, 14, 21, 28 MHz) har ett spänningsmaximum (spänningsbuk) och ett strömminimum (strömnod) i matningspunkten, som
då har hög impedans.
Ur horisontaldiagrammet för antennen
kan utläsas att ytterligare strålningskäglor
(strålningslober) utvecklas för varje överton
i den påmatade frekvensen. Samtidigt blir
strålningen alltmertill riktad längs med antennen.

\subsection{Impedansen i antennens matningspunkt}

Impedansen Z för varje punkt på en antenn
kan beräknas med Ohms lag Z = U/1.

Bild II 6-2

I mitten av en halvvågsantenn på grundfrekvensen är impedansen Z låg eftersom
spänningen är låg där och strömmen hög.
Ute i ändarna är däremot impedansen hög
eftersom spänningen där är hög och strömmen låg.
Impedansen i mittpunkten är 73 Q på
grundfrekvensen, när antennen mätt i våglängder befinner sig mycket högt över jordytan, d.v.s. utan nämnvärd påverkan från
omgivningen. l praktiken kan impedansen
awika mycket från detta värde.

Bild II 6-3 Halvvågsdipol matad med harmoniska övertoner

116-3

Antenn och matningskabel måste vara
impedansanpassade till varandra för att det
inte skall skall uppstå vågreflexion i anslutningen.
Märk, att halvvågsantennen är i resonans inte bara på grundtonen utan även på
jämna övertoner, 2:a, 4:e etc. harmoniska,
varvid matningspunkten har hög impedans.
Vid matning med en lågohmig koaxialkabel
uppstår då en kraftig missanpassning i anslutningen mellan antenn och kabel, vilket
måste åtgärdas på något sätt. Se avsnittet
Transmissionsledningar i detta kapitel.
Matningsimpedansen i några antenner
Med W3DZZ-antennen (se nedan) löses
hjälpligt anpassningsproblemet med mittmatade partier på 2:a harmoniska övertonen, d.v.s. dubbla grundfrekvensen. På 80och 40 m-banden är antennens matningsimpedans c:a 60 .Q och på de högre banden
ca 120 n. En kompromiss är att mata denna
antenn med en 75 n-kabel för att inte få
alltför stor missanpassning på något band.
Den omvikta dipolen (folded dipole):
Matningsimpedansen är c:a 240 n. En
bandkabel med impedansen 300 n kan användas alternativt en koaxialkabel med
impedansen 50 eller 75 n över en transformator med impedansomsättningen 4:1.
Jordplanantennen (GP-antennen):
Matningsimpedansen är 30-60 n. När
jordplanets spröt inte riktas horisontellt, utan
snett nedåt, erhålls en matningsimpedans
av 50 n, vilket passar bra för en koaxialkabel med 50 .Q impedans.
Yagi- och Quad-antenner:
En anpassningsanordning för anslutning
av 50-60 n koaxialkabel ingår oftast i fabriksgjorda riktantenner. En 50-60 n koaxialkabel kan då användas direkt.

\subsection{Reaktansen i en icke resonant antenn}

Den elektriska svängningskretsen behandlas i kapitel 3. Där framställs svängningskretsens grundegenskaper resistans R, induktans L och kapacitans C som koncentrerade till komponenter kallade resistor, induktor respektive kondensator.
116-4

Även en enkel tråd har dessa egenskaper, men utfördelade över hela tråden. Denna
kan därför ses som ett stort antal komponenter, som tillsammans bildar en svängningskrets, vilken naturligtvis kan fungera som
antenn.
När antennen matas med växelström
med samma frekvens som antennens resonansfrekvens, så svänger antennen med de
minsta förlusterna. Resonansfallet kan i
korthet beskrivas så att den induktiva och
kapacitiva reaktansen i antennen tar ut varandra medan resistansen kvarstår.
Impedansen är vektorsumman av resistansen och de kapacitiva och induktiva reaktanserna. l resonans är antennens impedans lika med resistansen, vilket är ett
specialfall. Antennströmmen har alltid sändarens frekvens. Om sändningsfrekvensen
är en annan än antennens resonansfrekvens, så händer endera av följande:
När antennströmmen har lägre frekvens
än antennens resonansfrekvens, så blir den
resulterande reaktansen negativ (kapacitiv),
d.v.s. Xc är större än XL.
När antennströmmen har högre frekvens
än antennens resonansfrekvens, så blir den
resulterande reaktansen positiv (induktiv),
d.v.s. XL är större än Xc.

\subsection{Elektrisk ``förlängning'' och ``förkortning''}

Om sändarfrekvensen, awiker mycket från
antennens resonansfrekvens, så kan
reaktansen i antennen behöva elimineras
eller åtminstone minskas för en bättre
impedansanpassning mellan antenn och
matarledning. Den enklaste åtgärden är då
att försöka ändra antennlängden.
Bild II 6-4. Om detta inte låter sig göras,
så kan man i serie med en "för kort" antenn
sätta in en induktor - en s.k. elektrisk förlängning. Om i motsatt fall antennen är "för
lång", så kan man sätta in en kondensatoren s.k. elektrisk förkortning.
l amatörradio ändras sändarfrekvensen
mycket och ofta, varför antennsystemet bör
kunna stämmas av från marken/operatörsplatsen. Då kan en antennkopplare med
nödvändiga reaktiva komponenter behövas.
Se längre fram i kapitlet.

Dipol, elektriskt förlängd

Dipol, elektriskt förkortad

förkortning av antenner

\subsection{Anpassning till sändarens impedans}

Ett sändars lutsteg med elektronrör är vanligen utrustat med en avstämningsanordning
vid H F-utgången. Syftet är att kunna anpassa sändarens utgångsimpedans till impedansen i antennledningen. l moderna sändare består denna anordning mycket ofta av
ett s.k. n-filter, vars utgångsimpedans kan
variera mellan c:a 30-150 n.
Ett transistoriserat slutsteg är oftast utfört för en fast utgångsimpedans av 50 n
och är alltså i behov av en avstämningsanordning, om inte antennsystemet inom vissa
gränser håller samma impedans. Toleransgränsen för felanpassning brukar vara ett
SVF av storleksordningen 2:1 innan sändarens skyddskretsar styr ner uteffekten.
Vid lika impedans i sändarutgång, matarledning och antennanslutning uppträder
ingen stående våg på matarledningen och
mesta möjliga effekt överförs från sändaren
till antennen.

h ::::

\subsection{Antennens strålningsdiagram}

En antenns strålningsbild beskrivs bäst i tre
dimensioner. Bild II 6-3 visar bl. a. ett horisontaldiagram för en halvvågsantenn.
Bild II 6-5 visar strålningen i vertikalplanet som funktion av antennhöjden för samma
antenn. Vertikaldiagrammet kan ha mycket
olika utseende beroende på antennens utförande, dess elektriska höjd över mark och
omgivningens elektriska egenskaper. För
att överbrygga stora avstånd, måste antennen ha en flack utstrålning relativt markplanet. En horisontelit upphängd antenn med
en längd av ')J2 har övervägande flack utstrålning när den placeras på en höjd av AJ
2, A, 3 A/2, 2 A o.s.v. över mark. När en
horisontell antenn däremot placeras /J4, 3
A/4, 5 'A/4 o.s.v. över mark, är utstrålningen
övervägande vertikal, vilket inte skall förväxlas med polarisationen, som i detta fall är
horisontell.
Samma diagram gäller både för en
sändar-och mottagaranten n. styrkan på en
utstrålad signal motsvaras av styrkan på
mottagen signal.

\subsection{Antennvinst}

Med antennvinst G (eng. gain) menas förhållandet mellan effekten Pt i huvudstrålningsriktningen (framriktningen för en antenn med osymmetriskt utstrålad effekt) och
effekten från en definierad referensantenn.
En referensantenn som tänks vara oändligt liten och som strålar med exakt samma
effekt Pi i alla riktningar kallas isotropisk
antenn.
En isotropisk antenn är emellertid endast
teoretisk definierbar.

=A2

Bild II 6-5 Vertikaldiagram för halvvågsantenn

116-5

Med effekten Pi från den isotropiska antennen som referens blir antennvinsten

P.

G= 1O log t
[dBi]
~
En i praktiken definierbar referens är
halvvågsdipolen, vars huvudstrålning ärvinkelrätt ut från dipolen och runt omkring den.
Referenseffekten är då Pd och antennvinsten

P.

G=10 logL
pd

Di pol

[dBd]

Bild II 6-6

Bild II 6-6 Antennvinst dBd i effekt
Antennvinsten kan också definieras som
förhållandet mellan den elektriska fältstyrkan
uf i huvudstrålningsriktningen och referensfältstyrkan
(dipol).
Jämfört med A./2-dipol är antennvinsten

ud

u

ud

[dBd]

Ungefärlig antennvinst för olika antenner med en isotropantenn som referens
A./2-dipol
Isotrop
Isotrop antenn
-2.1 dBd O
dBi
GP, A/4
-1.8 dBd 0.3 dBi
Dipol, A/2
O
dBd 2.1
dBi
1.2 dBd 3.3 dBi
GP, 5/8 A
Dipol, 1/1 A
2-elements yagi
2-elements quad
3-elements yagi

Riktantenn

G= 20 log-'

6 dB antennvinst motsvarar en fördubblad fältstyrka [V/m], d.v.s. 1 S-en hets ökning
vid den mottagande stationen, liksom att
6 dB antennvinst motsvarar en 4-faldigad
sändareffekt [W/m 2 ).

Bild II 6-7

1.8

5

6
8

dBd
dBd
dBd
dBd

3.9
7.1
8
10.1

dBi
dBi
dBi
dBi

\subsection{Effektivt utstrålad effekt}

Effektivt utstrålad effekt (ERP - effective
radiated power) är den effekt som sändarantennen strålar ut i sin bästa strålningriktning. ERP beräknas som den effekt som
tillförs själva antennen, multiplicerat med
antennvinsten relativt en halvvågsdipol. Förlusterna på vägen från sändaren ut till antennen är alltså borträknad före beräkningen
av ERP.

\subsection{Fram-/backförhållande (antennvinst)}

Di pol

Med fram-/backförhållande (F/B) för en riktantenn menas förhållandet mellan den utstrålade effekten i framriktningen P1 och
effekten i backriktningen Pb

Riktantenn

Bild II 6-7 Antennvinst dBd i spänning
Man använder uttrycket d Bi när antennvinsten anges i förhållande till en isotrop
antenn och d Bd i förhållande till en halvvågsanten n.
Se Appendix C om decibelbegreppet
Exempel på beräkning av antennvinst
U1 = 40 J-tV Ud= 20 J-tV
G=?

u

40
G= 20 log-r = 20 log-=
ud
20
=20 log 2=20·0.3=6

116-6

[dBd]

P.
pb

Fl B= 10 logL

Di pol

[dB]

Bild 6-8

Riktantenn

Bild II 6-8 F/B-förhållande i effekt
Fram/backförhållandet kan också definieras som förhållandet mellan elektriska
fältstyrkan uf i framriktningen och referensfältstyrkan ub i backriktningen

u
ub

Fl B= 20 log-'

Di pol

[dB]

Bild 6-9

Riktantenn

Bild II 6-9 F/B-förhållande i spänning

Exempel1
Ut= 40 ~-LV

Ub = 4 ~-LV

F l B= 20 log

u, = 20
ub

F/B= ?
log

40

=
4
=20 log 10=20·1=20 [dB]

F/B = 20 dB betyder att fältstyrkan Ut i
huvudriktningen är 1O gånger så hög som
referensfältstyrkan Ub.
Exempel2
Ut= 15 ~-LV Ub = 15 ~-LV

F/B= ?
15
Fl B= 20 log-' = 20 log-=
15

u
ub

=20 log 1=20·0=0

F/B= O dB betyder att Ut= Ub, d.v.s. att
fältstyrkorna i fram- och backriktning är lika
stora, vilket inträffar för en dipol.

\subsection{Halvvärdesbredd}
studera diagrammet för den horisontella
strålningen från en riktantenn.
Antennen avger sin största utstrålade
effekt Pt i huvudriktningen. Effekten avtar
utanför huvudriktningen. Fältstyrkan Ut förhåller sig på liknande sätt.
Med effekthalvvärdesbredd menas den
vinkel inom vilken nyttaeffekten är minst
hälften så stor som i huvudriktningen.
Bild II 6- i O
p
Observera, att
motsvarar ~

[T

d

2 u,

( ~ 0,7 Ut motsvarande 3 dB).
Med spänningshalvvärdesbredd menas
den vinkel inom vilken spänningen (fältstyrkan) är minst hälften så stor som den
största nyttaspänningen Ut. Spänningshalvvärdesbredden på en di pol är ungefär 90°.
Bild 116-10

[dB]

..öppningsvinkel (X

Umax

u
Effekthalvvärde

,Öppningsvinkel

\

\

/

/

\c-6d8
0,5
J

Spänningshalvvärde

l

/j;

Umax

u

l

Bild II 6-1 O Halvvärdesbredder

116-7

