\subsection{Kristalloscillatorer}
Kvartskristaller i oscillatorkopplingar

En Le-oscillators frekvensstabilitet begränsas av de ingående komponenternas egenskaper. När mycket bättre stabiltet än så
krävs, speciellt inom stora temperaturområden, så är kvartskristallen en svängningskrets med bättre data. Kvartskristallens höga
Q-värde ger också en renare signal.
l en kristalloscillator (CO, Grystal Oscillator) är en kvartskristall det frekvensbestämmande elementet i stället för en LC-krets. l
övrigt kan samma kopplingsprinciper som
för en LC-VFO användas.
Kristallen kan utföras så att den svänger
antingen som en serie- eller parallellresonanskrets. Märk, att en kristall svänger på
något olika frekvens beroende på om den fås
att fungera som serie- eller parallellkrets.
Den högre frekvensen är den som vanligen
används.

Bild II 3-75
l parallellresonansalternativet kopplas
kristallen parallellt över oscillatorns återkopplingsled. Den minsta dämpningen av
den återkopplade signalen fås när signalens
frekvens är lika kristallens resonansfrekvens.
Kristallens reaktans är då som högst.
Parallellt över kristallens inre induktans
ligger dess inre seriekopplade kapacitanser
C och Cw Yttre kapacitanser (en trimbar och
två fasta kondensatorer i serie) är kopplade
parallellt över den inre anslutningskapacitansen Cw
Om den trimbara kapacitansen ändras,
så påverkas kristallens resonansfrekvens.
Man säger då att man "drar" kristallen inom
ett litet frekvensområde. Kristallens och oscillatorns egenskaper avgör hur stort området kan vara. Om kristaller dras för mycket,
så kan resonansfrekvensen bli ostabil.
Den relativa frekvensändringen uppgår
till högst 1o-4 = 0.01 °/o. Formel:

t, f

re a

IV

.. d .
absolut ändring
re vensan nng=
~ k
resonans1re vens

~ k

Övertonskristaller
Bild II 3-76
l serieresonansalternativet kopplas kristallen in i serie med oscillatorns återkopp!ingsled. Den minsta dämpningen av den
återkopplade utgångssignalen fås, när signalens frekvens är lika som kristallens resonanfrekvens. Kristallens reaktans är då som
lägst. S.k. övertonskristaller används för oscillatorfrekvenser över ca 20 MHz.

Bild II 3-75 Golpittsoscillator med kristall
i parallellresonansfallet

Bild II 3-76 Golpittsoscillator med kristall
i serieresonansfallet
113-55

KRETSAR

PT

Övertonskristallernas dimensioner är lika
grundtonskristallernas, men snittas ut annorlunda och slipas för att svänga på önskad
udda överton. En övertonskristall har övertonens frekvens instäm pi ad i höljet och kristallen förutsätts arbeta i oscillatorkopplingar
som seriekrets. Genom att låta kristaller
svänga på sin överton undviker man en svår
tillverkningsprocedur, nämligen att slipa
mycket tunna kristallskivor.
En övertonsoscillator måste alltid innehålla en svängningskrets som är avstämd till
den överton som anges på kristallen.
Modellförsök: En instrumentsträng sätts i
svängning på sin grundton genom en knäppning mitt på strängen. En knäppning på en
punkt bort från mitten får strängen att svänga
på en överton i stället.

\subsection{Superheterodyn-VFO}
Bild II 3-77
En enkel LC-VFO är inte tillräckligt frekvensstabil i ett högt frekvensläge, t.ex. 144-146
MHz. Man kan då använda en speciell koppling, som är en kombination av LC-VFO och
CO, kallad super-VFO.
l en super-VFO blandas en låg, variabel
frekvens från en VFO med en hög frekvens
från en CO. Ordet super kommer från
superheterodyne = överlagring, blandning.
En VFO arbetar stabilare på låg frekvens
medan en CO fortfarande arbetar stabilt
även på högre frekvenser, dock inte så högt
som vi behöver här. l vårt exempel arbetar
därför VFO i området 8-1 O MHz och CO på
17 MHz. VFO-signalen blandas med en fast
signalfrekvens, som är GO-signalen 17 MHz
multiplicerat med 8, d.v.s. 136 MHz.

VFO 8 -

10 MHz

co 17 MHz
Bild II 3-77 Superheterodyn- VFO

113-56

Ett bandpassfilter filtrerar fram den önskade blandningsprodukten, som ligger i
frekvensområdet 144-146 MHz. Resultatet
blir en hög frekvens, som är både variabel
och stabil.
Fördelar:
Frekvensstabiliteten hos en super-VFO
är mycket bättre än hos en enkel VFO, som
arbetardirekt i VHF-området. En super-VFO
är dessutom mycket brusfattigare än en
PLL-VFO, vilken beskrivs här nedan.
Nackdelar:
Vid frekvensblandning uppstår oönskade blandningsprodukter, vilka visserligen
dämpas av bandpassfilter, men som det är
omöjligt att undertryckta helt. Bl. a. alstras en
svag spegelfrekvens, som vandrar från 128
till 126 MHz, samtidigt som den önskade
blandningsprodukten vandrar från i 44 till
i 46 MHz. Risken för att spegelfrekvensen
förstärks och sänds ut måste elimineras,
vilket kan göras med effektiva bandpassfilter. Se vidare i avsnitt 3.8 om frekvensblandni ng.

3. 7

\subsection{Oscillatorer med faslåsning - PLL}

En kristalloscillator (CO) arbetar med god
frekvensstabilitet Frekvensen som är fast
bestäms av styrkristallen.
En LC-oscillator arbetar däremot inom
ett frekvensområde (VFO), som bestäms av
en LC-krets. Dennas frekvens är emellertid
mindre stabil än den med styrkristalL
l en PLL (Phase Locked Loop) kan god
frekvenstabilitet och stort frekvensområde
förenas. En PLL är en sluten krets för elektrisk styrning av en oscillator, så att dess
frekvens är både stabil och variabel.

Spänningsstyrd oscillator (VCO)
Bild 113-78
En VFO, vars frekvens kan styras med en
likspänning, kallas VCO (Voltage Controlied
Oscillator). l svängningskretsen i en VCO
fyller en kapacitansdiod (varicap, variable
capacitor) samma uppgift som den mekaniskt variabla kondensatorn i en VFO.
Bild 113-79
När en motriktad spänning läggs på dioden, så bildas ett spärrskikt i dioden, så att
zonerna med fria laddningsbärare isoleras
från varandra likt kondensatorplattor. Spärrskiktets tjocklek (c:a 1/1000 mm) beror av
spänningen över dioden. Vid hög spänning
är spärrskiktet tjockt, vilket motsvarar "stort
plattavstånd" och liten kapacitans. Vid låg
spänning är skiktet tunt, vilket motsvarar
"litet plattavstånd" och stor kapacitans.
Med en kapacitansdiod i svängningskretsen, i stället för en mekaniskt variabel kondensator, så behövs ytterligare två komponenter. D rossel n Dr hindrar högfrekvenssignalen att överlagras på styrkretsens likspänning. Då skulle bl.a. svängningskretsens godhetstal försämras (förlorad HF-energi innebär dämpning). Omvänt hindrar kondensatorn C att dioden och spärrspänningen kortsluts genom induktorn. Oscillatorfrekvensen ställs in med den variabla likspänningen U. Av en VFO har det blivit en VCO.

Bild II 3-78 VFO och VCO jämförs
Oscillator med PLL-styrning
Bild II 3-80
Människan jämför och reglerar förlopp utifrån givna fakta. Det kan liknas med PLLkretsens sätt att jämföra det inbördes fasläget mellan signalen från en VCO (är-värdet)
och signalen från en CO (bör-värdet).

......---:=::::;:::=! M .

. . Jiijo MHZJ/
Jamfora .
,
Låg spänning

stor kapacitans

Hög spänning
liten kapacitans

Varicap·
diod

Bild 113-79 Kapacitansdiod- Varicap

ata

:.~,:..c ~~go' O
Bild II 3-BOa Analogi

Människa-PLL

113-57

KRETSAR
Som resultat av jämförelsen justeras styrspänningen så att är- och bör-frekvenserna
hålls lika. En sådan reglerkrets består av
digitala komponenter.
Fasjämföraren levererar en cykliskt justerad styrspänning till kapacitansdioden i
VCO. Eftersom denna spänning ändras
språngvis, så avrundas förloppet så att frekvensändringarna blir mjuka. Avrundningen
sker med ett RC-filter där kondensatorn antar ett medelvärde av den pulserande utgångsspänningen frånjämföraren. Om VGOfrekvensen är för låg, så levererar jämföraren en positiv spänning. styrspänningen på
kapacitansdioden stiger då med en hastighet som bestäms av filtrets tidskonstant

Kapacitansen i kapacitansdioden minskar med ökande spänning, eftersom spärrskiktet blir tjockare och frekvensen på VCO
stiger.
När signalen från VCO åter är lik referenssignalen från CO, till fasläge och frekvens, så ökar utgångsresistansen i fasjämföraren. Lågpassfiltrets kondensator behåller
då sin laddning och styrspänningen till VCO
ändras inte t.v. Skulle frekvensen på VCO
vara för hög så blir jämförarens utgång
lågohmig och filtrets kondensator urladdas
med den hastighet som bestäms av tidskonstante n. Den sjunkande styrspänningen
medför att kapacitansdiodens spärrskikt blir
tunnare, kapacitansen tilltar och VCO-frekvensen sjunker tills en
ny fas- och frekvenslikhet uppnåtts.

~--------~--~----~

rv Utfrekvens

Lågpassfi !ter

fvco < fco

Referensfrekvens
(bör-värde)

fvco > fco

vco~~

co~~

VCO-si gnat och GO-signal, formad som kantvågor att jämföras i fasläge
Ua  

----.~

u
VCO-frekvensen för låg

VCO-frekvensen för hög

Bild 113-BOb Oscillator med PLL-styrning

113-58

PLL-oscillator i kombination med frekvensblandning
Bild II 3-81
Signalen f 1 från en VCO
alstrar en sändningsfrekvens i bandet 144146 MHz. Denna blandas med signalen f 2
(136 MHz), som är en
multiplicerad GO-frekvens. Blandningsprodukten f 1 - f2 filtreras
fram, d.v.s. en signal i
området 8-1 OMHz som
påförs en fasjämförare.
Utsignalen från en VFO,
som är variabel inom
samma frekvensområde 8-1 O MHz, påförs
också fasjämföraren.
Utsignalen från jämföraren är en likspänning, som beror av frekvensskillnaden mellan
blandningsprodukt och
VFO-signal. Jämförarens utsignal ändras
uppåt eller nedåt, beroende på frekvensfelets
riktning.

KRETSAR
V GO-frekvensen bestäms av en likspänningsnivå, som styrs av jämförarens utsignal. Vid varje frekvensändring i VCO, kommer systemet att sträva mot frekvensskillnaden noll i fasjämföraren vilket gör att sändningsfrekvensen hålls vid rätt värde.
Fördelar med en PLL-oscillator: Den har
samma frekvensstabilitet som en VFO eftersom denna även här arbetar på en låg frekvens. Till skillnad mot en super-VFO finns
inga sidafrekvenser i PLL-oscillatorn, eftersom VCO alstrar nyttafrekvensen direkt.
Nackdelar med en PLL-oscillator: Den har
högre brusnivå än en super-VFO. Frekvensstabiliteten är sämre än den för en PLLoscillator med CO och programmerbar
frekvensdelare.
PLL med programmerbar frekvensdelare
Bild II 3-82
Med PLL blir frekvensen på utsignalen från
en VCO låst till referensfrekvensen från en
CO. l princip fås en VCO med samma frekvensstabilitet som en CO, men också lika
svår att ändra frekvensen på. Med en frekvensdelare i fasregleringsslingan (PLL) kan
emellertid utfrekvensen ändras, medan CO
fortfarande avger samma referensfrekvens.
En frekvensdelare är en digital krets, som

räknar svängningar eller pulser upp till ett
valt tal för att återställas till i och börja om
igen. Vid varje återställning avges en utpuls.
Vid en delning med två avges en utpuls för
varannan inpuls. Vid delning med i 5 avges
en utpuls för var i 5:e in puls o.s.v.
Genom att välja delningstal i PLL kan
arbetsfrekvensen i VCO ställas in stegvis,
där varje steg är så stort som en referensfrekvens. signalfrekvensen från vco delas
med det valda delningstalet och resultatet
jämförs med referensfrekvensen från CO.
Varje avvikelse från likhet med referensfrekvensen kommer att medförajustering av
V GO-frekvensen.
Om man t. ex. vill täcka 2-metersbandet i
steg om 25 kHz, så väljer man den referensfrekvensen samt att delaren kan fås att dela
sändarens utfrekvens med vilket som helst
av talen 5760, 576i, 5762 o.s.v. upp till
5840. Om t.ex. delningstalet 5820 valts, så
kommer jämförarens styrspänning att styra
V GO-frekvensen till i 45500 kHz. Delarens
utfrekvens blir då i 45500/5820 = 25 kHz,
vilket motsvarar referensfrekvensen. l detta
exempel styrs alltså sändarens utfrekvens
så att den alltid blir i steg om 25kHz.

vco

144 - 146 MHz
.........
144 - 146 MHz
~~-------------------r--------------------Q

f avstämningsspänning

[Q

8 - 10 MHz
uppblandad VCO-frekvens

8- 10 MHz
referensfrekvens
LP-filter

fasjäm·
förare

d

~
VFO

Bild 113-81 PLL-oscillator kombinerad med frekvensblandning

113-59

KR
Men PLL-oscillatorn
brusar
förhållandevis
144 MHz till 146 MHz
starkt jämfört med en
VCO G !--------.---------·-------<>
VCO och speciellt jäm"""-'
rv 144 till 146 MHz
fört
med en CO. VCO-----<>
svängningskretsen har
nämligen ett relativt lågt
Programmerbar delare i med tumhjul
n ( 5760 till 5840)
f eller annat
Avstämnings
godhetstal eftersom en
spänning
kapacitansdiod belastar kretsen mer än en
mekanisktvariabel kondensator.
Med det lägre god25kHz
hetstalet blir svängLågpass f i !ter
Referensfrekvens
ningskretsen ett mindre
bra filter för dämpning
Arbetsfrekvens Delningstal n Referensfrekvens
av oscillatorbruset Kapacitansdioden tillför
144 000 kHz
: 5760
= 25 kHz
: 5761
= 25 kHz
144 025 kHz
dessutom ett elektron145 500 kHz
: 5820
= 25 kHz
brus. Därtill kommerdet
146 000 kHz
: 5840
= 25 kHz
s.k. fasbruset från frekvensdelaren och PLL
VCO-frekvens
Med svängningskretsens låga godhetsVCO-frekvens,
tal är frekvensstabilite4-kantformad
ten i en VCO inte så bra
som den i en kristallosVCO-frekvens,
j 2-delad
cillator, utan faktiskt
sämre än den i en VFO.
Trots det är långtidsBild II 3-82 PLL med frekvensdelare
stabiliteten god i en
VCO, närden ingår i en
PLL, eftersom att frekvensen hålls ständigt
För- och nackdelar med PLL -oscillatorn
efterjusterad. PLL kan däremot inte åstadPLL-oscillatorn har nästan samma frekvenskomma en lika bra korttidsstabilitet Ett fasstabilitet som en kristalloscillator och frekvensen är inställbar i steg. Till skillnad mot jämförelseförlopp omfattar ju redan tiden för
en period av referensfrekvensen. Och det
en VFO med mekaniskt inställbar frekvens,
kommer att förflyta en multipel av denna
så är den PLL-styrda VGO-oscillatorns frekvens elektroniskt inställbar. Detta underlät- kortaste tid innan styrspänningen kan återställa VCO-frekvensen igen. Detta beror på
tar utformning och placering av reglage etc.
att kondensatorn i regleringsslingans lågför frekvensinställning, frekvensminne och
passfilter först måste laddas upp under ett
automatisk frekvensavsökning.
antal perioder innan reglering sker.
Först när den PLL-styrda oscillatorn kom
Dessa kortvariga frekvensavvikelser är
till användning i handapparater och mobila
en typ av frekvensmodulation som leder till
apparater blev det möjligt med frekvenstäckning över ett helt band med bibehållet fasbrus från PLL-oscillatorn och som kan
störa. Det är dock endast i extrema fall som
krav på små dimensioner. Som jämförelse,
skulle en inbyggnad av säg 80 till800 stycken fasbruset verkar störande eftersom det i
kanalkristaller i en traditionell kristallstyrd
moderna apparater reduceras till en accepapparat vara en mycket platskrävande, dyr- tabel nivå genom noggrann skärmning och
filtrering.
bar och opraktisk lösning.

1\ J\ 1\ ~1\ ;

r-uLflIlj
LJL 

113-60

\subsection{Faktorer som påverkar frekvensstabilitet}

Sändarens frekvens skall hållas så stabil
som möjligt. En ostabil sändare är inte godtagbar och skapar svårigheter inte bara för
de radiostationer som deltar i förbindelsen
utan även för radiotrafiken på närliggande
frekvenser.
En frekvensstabil oscillator skall ha följande
egenskaper:
stabil mekanisk uppbyggnad
Skakningar från underlaget t.ex. vid mobilt
bruk, vibrationer från en transformatorkärna
etc. kan försämra oscillatorns frekvensstabilitet
Frekvensbestämmande komponenter såsom fasta och variabla kondensatorer, spolar etc skall vara stabilt monterade, trimkärnorna i spolarna fixerade o.s.v.
Förbindningarna får inte tillåtas att böja
sig eller vibrera. Apparatstommen måste
vara tillräckligt styv för att inte ändra formen
och därigenom medföra frekvensändringar
vid hantering o.s.v.
God elektrisk uppbyggnad och högt Q-värde

i svängningskretsarna

Alla elektriska förbindningar måste vara så
korta som möjligt och löd- och kopplingsställen fullgoda. Induktorer och kondensatorer i
svängningskretsarna måste vara förlustfattiga och högvärdiga i övrigt så att signalen
blir så ren som möjligt från oönskade sidafrekvenser.
Återkopplingen i oscillatorn skall vara så
fast (kraftig) att självsvängningen är stabil.
Men för att få en renast möjlig signal får
kopplingen inte vara så fast, att svängningskretsarna blir alltför belastade och deras
godhetstal för lågt.
Avskärmande kapslingar
Svängningskretsar skall skärmas från yttre
kapacitanstillskott (t.ex.från en hand) Det
görs med skiljeväggar och komponentkapslingar av metall. Skärmningarna förhindrar också oönskad koppling mellan oscillatorn och efterföljande förstärkare genom elektriska och magnetiska fält.

stabila drivspänningar
Ostabila drivspänningar medför frekvensändringar. l en oscillator med transistorförstärkare beror ostabiliteten på förändringar
mellan skikten i en transistors diodsträcka.
Skikten fungerar nämligen som "kondensatorplattor" och spärrskiktet där emellan som
dielektrikum. Tjockleken av spärrskiktet och
därmed "plattavståndet" står i förhållande till
den spänning som läggs över transistorn.
Den spänningsberoende kapacitansen i transistorn är ansluten till svängningskretsen via
kopplingskondensatorn.
Eftersom kapacitansen i transistorn är en
del av svängningskretsen, så påverkar den
resonansfrekvensen. Denna egenskap kan
vara till besvär, men kan även användas för
att på ett enkelt sätt ändra oscillatorns arbetsfrekvens.
Se Kapacitansdiod och PLL-oscillatorn.
Buffertsteg
En oscillator i en radiosändare kan bestå av
ett enda förstärkarsteg, som alstrar högfrekventa elektriska svängningar. Vanligen tas
endast små effekter ut från en så enkel
sändare, normalt mindre än en watt. Utan
särskilda åtgärder, som t. ex. att använda en
styrkristall, är nämligen frekvensen inte särskilt stabil och olämplig för kommunikationsändamål.
Särskilt varierande belastning över oscillatorns utgång medför frekvensändring. Oscillatorn bör därför ges en så låg och stabil
belastning som möjligt. Ett buffertsteg kopplas därför in efter oscillatorn. Det bör ha hög
ingångsimpedans för att belasta oscillatorn
så lite som möjligt. Det skall också kunna
lämna tillräcklig driveffekt till efterföljande
förstärkare och bör därför ha låg utgångs impedans. Det måste dessutom arbeta linjärt
(se klass A-drift, bild 113-44) för att inte alstra
övertoner och därmed förvränga oscillatorsignalen. Bild II 3-42 visar ett buffertsteg i
kollektorkoppling, vilken har dessa egenskaper.

113-61

KRETSAR
Temperaturkompensation och termostater
Det alstras alltid förlustvärme i elektriska
apparater och även i en oscillator. Vid uppvärmningen utvidgas spolar och kondensatorer i svängningskretsarna, vilket leder till
frekvensändringar. Även spärrskiktskapacitansen i transistorerna är temperaturberoende. Det totala temperaturberoendet kan
kompenseras genom ett antal åtgärder.
Oscillatorn bör monteras så långt bort
som möjligtfrån övriga värmealstrande komponenter. Den avskärmande kapslingen omkring oscillatorn skall vara så tjockväggig
och värmeisolerande som möjligt. Inbyggnad i en termostatreglerad kapsling är ett
ännu bättre alternativ.
Komponenterna bör ha uppnått drifttemperaturför användningen. Oscillatorn bör
därför värmas upp under åtminstone 15 minuter.

\subsection{Frekvensstabilitet och oscillatorbrus}

Frekvensstabiliteten i kristalloscillatorer är
ca 100 gånger bättre än den är i LCoscillatorer. Likaså är utgångssignalen från
kristalloscillatorer renare från s.k. fasbrus
Gitter). Varje oscillator avger nämligen även
oönskade signaler med frekvenser som ligger omkring utgångssignalens nominella
frekvens.
Oscillatorn är ju en förstärkare, vars utgångsspänning delvis återkopplas till ingången i medtas. Detta innebär att utgångssignalen förstärks lavinartat till ett maximum, omväxlande med att den dämpas lavinartat till
ett minimum. Utan yttre påverkan befinner
sig alltså förstärkaren i ett självsvängningstillstånd mellan två yttervärde n. l återkopplingsvägen placeras ett filter som frekvensbestämmande element, t.ex. en LC-krets
eller en kvartskristall.
Återkopplingen blir starkast på filtrets
resonansfrekvens, vilket medför att oscillatorn svänger bäst där. Eftersom filtret oundvikligen har en viss bandbredd, så kommer
även ett spektrum av andra frekvenser tätt
omkring resonansfrekvensen att släppas
igenom. De oönskade frekvenserna omkring
den nominella kallas för brus.

113-62

l moderna konstruktioner används oftast
PLL-oscillatorer. På grund av sin funktion
pendlar deras frekvens alltid något. Hur
mycket beror bl.a. på loop-filtret. Alltså är
frekvensen egentligen ett mycket litet band
av flera frekvenser varav en framträder mest.
Försök:
Volymkontrollen i en lågfrekvensförstärkare
utan insignal vrids till maximum. Det kommer att höras ett brus i högtalaren, som
huvudsakligen kommer från ingångsstegets
transistorer. När en mikrofon ansluts måste
volymkontrollen vridas ner och då hörs bruset
mindre. Men bruset finns ändå där på en
lägre nivå och överlagras på insignalen från
mikrofonen.
Även i en högfrekvensoscillator överlagras bruset på insignalen. Men ju högre godhetstalet är i svängningskretsen, t.ex. en
kristall, desto smalare är filtrets bandbredd,
desto kraftigare blir brusundertryckningen
och desto merframhävs den önskade signalen. P.g.a. det större godhetstalet i svängningskretsen, och därmed den mindre bandbredden, så brusar alltså en kristalloscillator
mindre än en LC-oscillator.
En nackdel med kristalloscillatorn är att
dess frekvens inte kan ändras inom ettstörre
område. Önskas flera valbara frekvenser
från en kristalloscillator måste flera kristaller
användas tillsammans med något slags
omkopplingsanordning (kanalväljare).
Komponentmängden i en kristalloscillator
är mindre än i en VFO, men i apparater för
flera frekvenser uppvägs denna fördel av
merkostnaden för flera kristaller och kanalväljaren.
Kristalloscillatorn har många användningsområden, där en frekvensstabil och
brusfattig signal önskas och där platsbrist,
skakningar m.m. utesluter användning av en
LC-VFO.

ETSAR
