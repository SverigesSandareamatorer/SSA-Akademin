\section{Oscillatorer}

Alstring av svängningar
Ordet aseiiiare (lat.) har betydelsen svänga
och den företeelse eller anordning som skapar en svängning kallas oscillator. Vid alla
slags svängningar sker växelverkan mellan
olika energiformer. Svängningar förekommer i olika former. Det kan vara vibrationer i
en kropp, en pendel som svänger, rörelser i
gaser och vätskor, elektriska laddningar i en
strömkrets o.s.v.
Mekanisk pendel
Bild II 3-63
Energiinnehållet i en pendel växlar mellan
lägesenergi (potentiell energi) och rörelseenergi (kinetisk energi). Lägesenergin är
störst i pendelns ytterlägen och minst i mittläget. Omvänt är rörelseenergin störst i mittläget och minst i ytterlägena.

Stöt till en pendel bara en gång så att den
börjar pendla, men utslagen blir allt mindre.
Pendeln utför en dämpad svängning därför
att det förbrukas energi under pendlingen.
Dämpningen kommer av att energi förloras av friktionen i upphängningspunkten och
av luftmotståndet.
Stöt nu till pendeln varje gång, som den
pendlar tillbaka. För lika stora utslag varje
gång- en odämpad svängning- fordras det
upprepade energitillskott som precis kompenserar förlusterna.
Villkoren för att en svängning skall fortgå
(vara odämpad) är att energitillskotten
• kommer vid rätt tidpunkt,
• har rätt riktning- polaritet,
• kompenserar förlusterna.

PENDEL

\

l
Epot

A

t
DAMPAD SVÄNGNING

ODÄMPAD SVÄNGNING

A

=

utslag från viloläge

Bild II 3-63 Svängningar

113-49

KRETSAR
Alstring av mekaniska svängningar
Bild II 3-64
En elektrisk ringklocka med självbrytande
kontakt är ett exempel på en enkel elektromekanisk oscillator. Denna bild visar hur
kontakten ersatts med ett elektronrör. En
mer "tidsenlig" lösning med transistor hade
naturligtvis också kunnat användas.
När anodspänningen kopplas till, så börjar anodström att flyta genom trioden från
katod till anod och genom elektromagneten.
Magneten drar då till sig bladfjädern, som
kopplar styrgallret till en negativ förspänning.
Den negativa förspänningen stryper anodströmmen och magnetfältet upphör. Bladfjädern släpper då från magneten och kopplar
bort styrgallret från förspänningen. Anodström börjat att flyta igen, varvid elektromagneten drartill sig bladfjädern o.s.v. Förloppet
kallas självsvängning.
Anordningen som alstrar svängningarna
kallas generator eller oscillator. Frekvensen
är det antal svängningar per sekund som
oscillatorn alstrar, i detta exempel bladfjäderns svängningshastighet.

Amperemetern
visar den pulserande
anodströmmen, som är en likström.
Amperemetern A 2 visar växelströmmen
svängningskretsen.
1 Halvvågen

Ua
POSITfVT GALLER:
- i anodkretsen flyter en ström
{A1 visar mot höger)
- i svängningskretsen flyter en ström
(A 2 visar mot höger)

2 Halvvågen

....--~

~--------~+

U9

,A1

,

-F-------~

V
~----~-

Bild II 3-64 Elektromekanisk oscillator

Alstring av elektriska svängningar
Bild II 3-65
En elektrisk svängningskrets är motsvarigheten till ett mekaniskt föremål i svängning.
Bilden visar en elektronisk oscillator med en
LC-krets i anodkretsen till en triod och som
är induktivt återkopplad till styrgallret

113-50

t

+~--~

Ua
NEGATIVT GALLER:
- ingen ström flyter i anodkretsen
(A 1 visar noll)
- i svängningskretsen flyter en ström
i motsatt riktning
(A2 visar mot vänster)

Bild 113-65 Elektronisk oscillator (Meissner)

KRETSAR
En elektronisk oscillator är en förstärkare, vars utsignal återförs till ingången så
att förstärkaren råkar i självsvängning -det
blir en s.k. positiv återkoppling.
Elektroniska oscillatorer används både i
mottagare och sändare. Funktionsprincipen
är lika i båda fallen. Vad som möjligen skiljer
är användningssättet Det finns många oscillatorkopplingar varav några beskrivs här.
Självsvängning kan demonstreras akustiskt med en mikrofon, en förstärkare och en
högtalare. Resultatet blir att det genereras
en ton. Ljudet från högtalaren är tryckvågor
som påverkar mikrofonen och omvandlas
där till elektriska signaler. Dessa går genom
förstärkaren tillbaka till högtalaren och omvandlas åter till tryckvågor som mikrofonen
uppfattar o.s.v. Det uppstår självsvängning,
ett tjut, som beror på akustisk återkoppling.
Om förstärkaren kompletteras med ett frekvensfilter i form av en svängningskrets, så
blir "tjutet" i stället en ton med samma frekvens som filtrets resonansfrekvens.

Bild II 3-66 Oscillator enligt Meissner
Bild II 3-67
Förstärkaren kan t.ex. vara en emitterkopplad transistorförstärkare enligt bilden.
Kopplingskondensatorerna Ck är nödvändiga för att förhindra kortslutning av de likspänningar som pestämmer arbetspunkten
för transistorn. A andra sidan kan växelspänningssignalerna passera till och från
transistorn.

:r
ck

r-

.,.--1-----o.
utgång

LC-oscillatorer
Variabel frekvens oscillator- VFO
En oscillator med inställbar frekvens kallas
för VFO (variabel frekvensoscillator). Förutom frekvensstabilitet fordras också, att
noggrann inställning och avläsning av frekvensen skall kunna göras.
En Le-oscillator är urtypen för en oscillator med variabel frekvens. Meissner-kopplingen är lätt att urskilja och används här för
att beskriva grundprincipen för en oscillator
i stort. Bl.a. Colpitts- och Glapp-kopplingarna har emellertid bättre stabilitet och
inställbarhet i återkopplingsledet
Meissner-koppling
Bild 113-66
Bilden visar en Meissner-oscillator, som består av en LC-svängningskrets med återkopplingsspole och en förstärkare. Magnetfältet mellan induktansen i svängningskretsen och återkopplingsspolen är polariserat
så att en förändring i utsignalen medverkar
till självsvängning. (Motsatsen är motkoppling.)

f
Bild II 3-67 Emitterkopplad förstärkare
Bild II 3-68
Återkopplingsvägen görs i detta fall så, att
svängningskretsen kopplas parallellt över
förstärkaringången. Aterkopplingsspolen
fungerarsom förstärkarens kollektorresistor.

r

Bild II 3-68 Komplett Meissneroscillator

113-51

KRETSAR
Självsvängningsvillkoret

Bild II 3-69
Självsvängning i en förstärkare uppstår genom återkoppling. signalspänningen ain över
ingången blir förstärkt med faktorn A. När
som i bild 113-68 förstärkaren är emitterkopplad, blirutsignalen fasvriden 180° i förhållande till insignalen. Fasvridningen a=180° betecknas här med minustecken, alltså blir
förstärkningen -A.
På förstärkarens utgång fås en signalspä~ning out ~ed sambandet
Uut

=-A.

uin

En del av utsignalen återförs (återkopplas) till ingången. l en Meissner-oscillator
sker återkopplingen med en induktor, som är
induktivt kopplad till svängningskretsens induktor.
Kvoten k mellan den återkopplade signalspänningen ok och signalspänningen out
på förstärkarens utgång kallas återkopplingsfaktor. Den återkopplade spänningen
ok fasvrids så att den kommer "i fas med "
med insignalen. För den återkopplade signalen fås då sambandet

ok= -k. out

Tillräcklig signalspänning från utgången
måste återföras till ingången för att det skall
uppstå självsvängning. Det sker när den
återkopplade signalspänningen ak är minst
lika stor som ingångsspänningen Din och är i
rätt fasläge, d.v.s. i detta exempel
Ok~ Din
eller -k· Du!A eller k~ 1/A
Självsvängningsvillkoret blir
k~ 1l A eller k· A ~ 1
Ett k· A ::::: 3 är önskvärt för att oscillatorn
skall svänga igång snabbt.

Bild II 3-69 Svängningsvillkoret
113-52

Hartiey-koppling

Bild II 3-70
Återkopplingen sker galvaniskt över ett uttag
på induktorn i oscillatorns LC-krets.

Bild II 3-70 Hartiey-koppling
Huth-KO hn- eller TGTP-koppling (tuned grid
- tuned plate)

Bild II 3-71
Kopplingen är en förstärkare med LC-kretsar både på in- och utgång. Båda ~retsarna
är avstämda till samma frekvens. Aterkopplingen sker över de inre kapacitanserna
mellan elektronrörets elektroder resp. mellan transistorns materialskikt Denna koppling är av flera skäl inte särskilt vanlig.

{I];

----+---

Y. C>
Bild /13-71 TPTG-koppling
Golpitts-koppling

Bild 113-72
Återkopplingen sker över en kapacitiv spänningsdelare, som ingår som en del av oscillatorns LC-krets.

Bild II 3-72 Golpitts-koppling

Glapp-koppling

Bild 113-73
Denna koppling är en variant av Colpittskopplingen. Vridkondensatorn för frekvensinställningen är seriekopplad med spänningsdelarens kondensatorer. Glapp-oscillatorns
frekvensstabilitet är god.

Bild II 3-73a Glapp-koppling

Vi utvecklar denna beskrivning vidare.
Vridkondensatorn samt en fast och en trimningsbar kondensator är kopplade parallellt
med varandra. Alla tre kondensatorerna är i
sin tur seriekopplade med den kapacitiva
spänningsdelaren C3 C4 • Förstärkarens ingång är kopplad till den övre anslutningen av
C3 • Utgången från oscillatorns förstärkare
återkopplas över dämpresistorn Rct till mitten
av spänningsdelaren c3c4 (återkopplingskretsen).

..---+----+e v

Bild II 3-73b Förstärkare i Glappkoppling

Förstärkarens arbetspunkt bestäms av
spänningsdelaren R1 R2 • Ingen kopplingskondensator behövs eftersom det enbart
finns kondensatorer mellan förstärkaringång
och jord.
Kondensatorn C6 avkopplar kollektorn på
transistor T1 HF-mässigt till jord. Förstärkaren är alltså kollektorkopplad.

Kondensatorn C7 kopplar oscillatorns utsignal till buffertsteget För frekvensstabilitetens skull stabiliseras spänningen 8 V med
en lC-krets som avkopplas HF-mässigt med
en kondensator.

Frekvensinställning och bandspridning
Bild II 3-74
Att ställa in frekvensen i en Le-oscillator
gjordes förr oftast med en vridkondensator.
l moderna mottagare och sändare används
i stället en s.k. varicap, som styrs med en
likspänning.
Med en svängningskrets med endast en
induktor och en vridkondensator, skulle alla
amatörradiobanden endast vara smala områden utspridda på en mekanisk skala, d.v.s.
över vridkondensatorns hela kapacitansområde, varvid kapacitansen kan varieras med
förhållandet 1:5 a 1:1 O, tex. 10-50 pF a 10100 pF.
För att i stället få vart och ett av amatörradiobanden utspridda över större delen av
skalan kan man ordna med bandomkoppling
och s.k. bandspridning. Man parallellkopplar
då en relativt stor fast kapacitans med vridkondensatorns relativt lilla kapacitans. Den
totala kapacitansvariationen i LC-kretsen blir
då liten, trots att kondensatorns hela kapacitansområde utnyttjas. Resultatet blir en frekvensskala med större upplösning, d.v.s. bättre avläsningsnoggrannhet
Bandspridning kan också ordnas med
två seriekopplade kondensatorer, varav den
större görs variabel. Typiskt värde på vridkondensatorn i en kortvågsutrustning är då
100-500 pF och den fasta kondensatorn
mycket mindre än så.

nrno
Bild II 3-74 Bandspridning

113-53

113-54

ETSA
