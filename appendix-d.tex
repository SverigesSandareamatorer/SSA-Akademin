\chapter{S-ENHETER OCH dB}

I kommunikationsradiomottagare brukar det
nästan alltid finnas en anordning som mäter
och visar styrkan av mottagna signaler.
Eftersom spänningen från antennen in i
mottagaren kan variera mycket, är det praktiskt att uttrycka styrkevärdena i en
misk måttenhet, s.k. s-enhet.
signalspänningen mäts över en impedans av 50 n.
Eftersom s-enheten är logaritmisk, så
motsvarar t. ex. signalstyrkan S8 halva signalspänningen, d.v.s. 25 ~V eller -6
jämfört
med S9. Om halveringen fortsätts, fås att so
(noll) motsvarar en kvarvarande signalstyrka
av0.1J.LV.
l en kortvågsmottagare alstras det ett
interntbrus meden nivå av åtminstone 0.1 J.LV.
Detta brus blandas med den inkommande
signalen. En insignal med en styrka under
under brusnivån kommer alltså inte att kunna
höras, alltså SO. Vid högre signalstyrkor än
S9 anges styrkan som S9 +ett antal dB. Det
är då frågan om mycket starka signaler.
Följande tabell gäller för det ideala sambandet mellan s-enheter och signalstyrkor
över två alternativa brusnivåer.

signalstyrkan mäts vid mottagarens antenningång, varför skillnaden i signalstyrkan
olika antenner och mottagningsriktningar samt dämpningen i antenn och nedledning kan behöva bedömas.
l kortvågsområdet (under 30 MHz) uppträder ett atmosfäriskt bredbandigt brus tillsammans med bruset från den stora mängden rundradio- m.fl. andra starka sändare.
Detta brus är mer dominerande än mottagarens interna brus. l praktiken har de flesta
KV-mottagare en högre brusnivå än 0.1 J.LV.
Över 30 MHz däremot, är det mest mottagarens interna brus som sätter gränsen för
hörbarheten av svaga signaler. Med samma
s-skala som för kortvågsom rådet, börjar man
uppfatta signaler i bruset utan att s-metern
ger utslag.
Vid IARU Region i-konferensen 1978 i
Miskolcz föreslog de nationella föreningarna
VERON (Nederländerna) och RSGB (Storbritannien) en annan s-skala över 30 MHz.
Vid konferensen 1981 i Brighton antogs förslaget som rekommendation.
Mätningar skall i båda fallen göras med
en kvasi-toppvärdesdetektor med en stigtid
av 1O ms $\pm$0.2 ms och en falltid av 500 ms.

s-enheter, rekommenderade normvärden inom IARU Region 1

S-meter
värde
s

dB m

( vid 50 .Q)

9+ 40 dB
9+ 30 dB
9+ 20 dB
9+ 10 dB
9
8
7
6
5
4
3
2
1

-33
-43
-53
-63
-73
-79
-85
-91
-97
-103
-109
-115
-121

5.0 mV
1.6 mV
500
~v
160
J.LV
50
J.LV
25
~v
12.6 ~v
6.3 J.LV
3.2 JlV
1.6 J.LV
0.8 J.LV
0.4 JlV
0.21 JlV

Över 30 MHz

Under 30 MHz
dB~ V

dB m

(U vid 50 .Q)

dBJ.LV

74
64
54
44
34
28
22
16
10
+4
-2

-53
-63
-73
-83
-93
-99
-105
-111
-i 17
-123
-129
-135
-141

500
160
50
16
5
2.5
1.26
o.63
o.32
0.16
0.08
0.04
0.02

54
44
34
24
14
8
+2
-4
-1 o
-16
-22
-28
-34

-8

-14

J.LV
J.LV
J.LV
J.LV
J.LV
J.LV
J.LV

~v
~v

J.LV
J.LV
JlV
J.LV

D-1

APPENDIX

D-2

D
