\section{Transvartern}

Bild II 4-19

En transverter (transmitter-converter), är
en kombinerad frekvensomvandlare för både
sändning och mottagning. Den förflyttarbåde
mottagnings- och sändningssignaler mellan två frekvensområden.
Transvertern är ett bra exempel på hur
samma teknik kan användas både i mottagare och sändare. Om t.ex. en KV-transceiver redan finns, kan både mottagning
och sändning ordnas även på andra band
med en transverter som tillsats.

efter kristalloscillatorn CO kan användas för
sändning och mottagning.
Fördelen med en transverter är att kostnaden för en sådan är låg jämfört med den
för en komplett transeeiver även för det
tillkommande bandet. Förutsättningen är att
en transeeiver för något band redan finns.
Nackdelen är att den befintliga transeeivern inte samtidigt kan användas på några
andra frekvenser än de som används för
tillfället.

Exempel
En konverterförflyttar de mottagna UH Fsignalerna till kortvågsområdet Som huvudmottagare används en KV -transceiver i
mottagningsläge. Konvertern kan utökas till
att även fungera vid sändning och kallas då
transverter. Med KV -transceivern i sändningsläge flyttas dess signaler till UH F-området genom blandning i transvertern av
KV-signalen och en multiplicerad signal från
en lokaloscillator (LO). Den önskade blandningsprodukten i UHF-områdetfiltreras fram
och förstärks i efterföljande driv- och slutsteg. Samma frekvensmultipliceringskedja
UH F-antenn

Sändarblandare

s

Bandfilter

Drivsteg

-

f1 28-30 MHz

UH F-försteg

t
T 44,889MHz
D

CO

3

t2=

Bild 114-19 Transverter mellan UHF och KV
114-14

PA

404MHz

ARE

M T
