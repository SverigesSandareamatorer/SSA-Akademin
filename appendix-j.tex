\chapter{RAPPORTKODER}
Det finns olika sätt och system att rapportera hur en radiostation hörs.

Amatörradiotrafik
l amatörradiotrafik används RST -koden vid
rapportering av telegrafisignaler och RSMkoden för telefonisignaler. Namnet kommer
av begynnelsebokstäverna i de engelska
orden
Readability
(läsbarhet),
Signal strenght
(signalstyrka),
Tone
(ton)
Modulation
(modulation).
R-skala (läsbarhet)

1 Oläsbar

2
3
4
5

Knappt läsbar, enstaka ord tydbara
Läsbar med stor svårighet
Läsbar med obetydlig svårighet
Helt läsbar

s-skala (signalstyrka)
1 Sigalerna knappt uppfattbara
2 Mycket svaga signaler
3 Svaga signaler
4 Något svaga signaler
5 Ganska goda signaler
6 Goda signaler
7 Mycket goda signaler
8 starka signaler
9 Mycket starka signaler

Kommersiell sjö- och luftradiotrafik

1kommersiell sjö- och luftradiotrafik används
t.ex. Q-förkortningarna QSA (signalstyrka),
ORM (störningar från annan station), QRN
(atmosfäriska störningar), QSB (fädning) och
ORK (uppfattbarhet) åtföljda av en siffra för
graden i skala 1-5. Jämför med SINPOkoden.
Exempel:
"QSA 5, ORK 3, QRN 1", vilket betyder
"ljudstyrka mycket god, uppfattbarhet ganska god, störningar från andra stationer
måttliga, atmosfäriska störningar obefintliga".
Se vidare i avsnitt 111.1.2

M-skala (modulation)
1 Moduleringen oförståelig
2 Mycket dålig modulering, p.g.a.
parasitsvängningar eller annan orsak
3 Dålig modulering, p.g.a. obehöriga
frekvensändringar hos bärvågen i takt
med moduleringen
4 Ganska god modulering, låter klippt
p.g.a. övermodulering, överstyrning
etc
5 God modulering, helt felfri.

T-skala (ton)
1 Mycket rå växelströmston, ostabil och omusikalisk
2 Mycket rå växelströmston, stabil men musikalisk
3 Rå växelströmston, ostabil och omusikalisk
4 Rå växelströmston. stabil och någorlunda musikalisk
5 Tydligt växelströmsmodulerad ton, ostabil men musikalisk
6 Tydligt växelströmsmodulerad ton, stabil och musikalisk
7 Nästan ren likströmston, ostabil och med tydligt brum
8 Nästan ren likströmston, med spår av brum eller ojämnheter
9 Abalut ren likströmston, stabil

Dessutom kan följande tillägg till T-skalan förekomma:
x Absolut ren likströmston, mycket stabil, kristallklar, mjuka tecken utan knäppar
c Absolut ren, men ostabil likströmston vid nycklingen
k Knäppar alstras vid nycklingen

J-1

APPENDIX
Rundradiosändningar m.m.

För rapportering till rundradiostationer m.m.
förekommer ett system som kallas SINPO
eller SINPFEMO-koden.
Förr användes SINPO för radiotelegrafi
och SINPFEMO för radiotelefoni. Numera
används enbart SINPO-koden.
Namnet på koden kommer av begynnelsebokstäverna i orden
Signal strength (signalstyrka),
lnterference (störningar från annan radiosändning),
Noice (atmosfäriska störningar),
Propagation disturbance (vågutbredningsstörningar),
Frequency of fading (fädningsfrekvens),
Emission quality (modulationskvalitet)
Modulation depth (modulationsgrad),
Over all merit (sammanfattande omdöme).

Rapporten inleds med koden SINPO eller
SINPFEMO följd av fem resp. åtta siffror,
vilka var och en i tur och ordning graderar
egenskaperna i skala 1-5. För icke bedömda
egenskaper skall bokstaven x sättas i stället
för en siffra.
Kod

Grad

s

1
2
3
4
5

Koder Grad

l,

p

1
2
3
4
5

Kod
F

Grad

1
2
3
4
5

Kod

E

1
2
3
4
5

Kod
M

Grad

Kod

Grad
1

o

J-2

Grad

1
2
3
4
5

2
3
4
5

Bedömning
Knappt uppfattbar
Dålig
Tillfredsställande
God
Utmärkt
Bedömning
Mycket stark
stark
Måttlig
Svag
Ingen
Bedömning
Mycket snabb
Snabb
Måttlig
Långsam
Ingen
Bedömning
Mycket dålig
Dålig
Tillfredsställande
God
Utmärkt
Bedömning
ständig övermodulering
Dålig eller ingen
T ilitredsställande
God
Maximal
Bedömning
Oanvändbar
Dålig
Tillfredsställande
God
Utmärkt

