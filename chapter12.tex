\chapter{Nationella och internationella bestämmelser för Amatör- och Amatörsatellittjänsterna}
Tekniskt sett kan radioamatörerna världen
över, med hjälp av sina radiostationer, tämligen lätt skapa kontakt med varandra. Därvid krävs att reglerna i de länder som berörs
vid kontakten respekteras.
En hel serie både internationella och nationella regler styr radiokommunikationerna
i en nation. Varje radioamatör skall känna till
och följa dessa regler så långt de har anslutning till amatörradio. Vissa länder - t.ex.
CEPT-länderna - har i någon utsträckning
harmoniserat sina bestämmelser inbördes.
Nationella avvikelser förekommer likväl och
reglerna i det land, som man gör radiosändningar ifrån, skall alltid följas.

\section{ITU Radioreglemente (RR)}

Amatör- och Amatörsatellittjänsterna är radiokommunikationstjänster med syfte att tillhandahålla nödvändig kommunikation i händelse av naturkatastrofer, träna operatörer
och tekniker i radio- och telekommunikationsteknik till ingen kostnad för stat och
samhälle, bidra till att tidsenlig radiokommunikation främjas och att förbättra internationell förståelse och välvilja.

Artikel1 (RR) Termer och definitioner

Si .56 (RR) Amatörtjänst

En radiokommunikationstjänst avsedd för
självutbildning, inbördes kommunikation och
tekniska undersökningar bedrivet av amatörer, det vill säga av behörigen godkända
personer intresserade av radioteknik, endast av personligt intresse och utan ekonomiskt syfte.
S 1.57 (RR) Amatörsatellittjänst
En radiokommunikationstjänst som använder rymdstationer på jordsatelliter för samma
ändamål som för Amatörradiotjänsten.
81.96 (RR) Amatörradiostation
Radiostation inom amatörradiotjänst

Artikel S25 (RR) (f.d. Artikel 32)
Sektion l. Amatörtjänst
825. i \  1. Radiokommunikation mellan
amatörstationer i olika länder skall vara förbjuden, om administrationen i en av de berörda nationerna har meddelat att den är
emot sådan radiokommunikation.
825.2 \  2. (1) Närsändningarmellan amatörstationer i olika !änder är tillåtna, skall det
ske på klart språk och begränsas till meddelanden av teknisk natur i samband med prov
och till personliga kommentarer, som på
grund av sin oviktighet inte är skäl nog för att
ta den allmänna telekommunikationstjänsten i anspråk.
(2) Det är absolut förbjudet att
825.3
använda amatörradiostationer för internationell radiokommunikation för tredje parts
räkning.
825.4
(3) De föregående bestämmelserna får ändras genom särskilda överenskommelser mellan administrationerna i berörda länder.
825.5 \ 3
(1) Varje person som söker en
licens för att använda apparaterna i en
amatörradiostation skall bevisa sin förmåga
att för hand sända rätt och med hörseln rätt
ta emot texter i form av morsesignaler. Berörda administrationer får emellertid bortse
från detta krav för stationer som endast
används på frekvenser över 30 MHz.
825.6
(2) Administrationerna skall vidta sådana åtgärder som de finner nödvändiga för att kontrollera de handhavandemässiga och tekniska kvalifikationerna hos varje
person som önskar använda apparaterna i
en amatörradiostation.
825.7 \ 4
Den högsta effekten från en
amatörstation skall fastställas av berörda
administrationer, med hänsyn till operatörernas tekniska kvalifikationer och under vilka förhållanden dessa stationer skall användas.

1112- 1

LER

C TRAFIKMET DER

S25.8 \ 5
(1) Alla allmänna regler i överenskommelsen och de i denna artikel skall
tillämpas på amatörradiostationer. Särskilt
den utsända frekvensen skall vara så stabil
och så fri från sidafrekvenser som den tekniska utvecklingen för sådana stationer medger.
S25.9
(2) Under loppet av sändningarna skall amatörstationer sända sina anropssignaler med korta mellanrum.

Sektion Il. Amatörsatellittjänst
S25.1 O \ 6 Bestämmelserna i Sektion 1 i
denna artikel skall gälla i all tillämplig omfattning även för amatörsatellittjänst
S25.1 O \ 7. Rymdstationer i amatörsatellittjänst, som arbetar i band som delas med
andra tjänster, skall förses med lämplig utrustning för att kontrollera utstrålningen om
skadlig störning rapporteras, allt i överensstämmelse med den procedur som föreskrivs i Artikel S15 *.Administrationer som
godkänner sådana rymdstationer skall informera RRB (Radio Registrations Board)
och skall tillse att tillfredställande jordkontrollstationer upprättas före uppskjutningen för
att säkerställa att varje rapporterad skadlig
störning skall kunna avbrytas omedelbart av
den bemyndigande administrationen. Se
S22.1 **.
* S15 behandlar "lnterference"
** 822 behandlar "Space Services"

Bild III 2-1 ITU Regionkarta (ur RRB-2)

1112-2

~©

Artikels (RR 8..1) Frekvenstilldelning
Inledning
391 § 1. l Unionens alla dokument där
termerna allocation, alfatment och assignment används skall de ha den betydelse
som ges i nummer 17 till19, varvid termerna
på de tre arbetsspråken skall vara som följer
(franska, engelska och spanska):
Frekvensfördelning till:
Tjänster
Allocation (tilldelning)
Allotment
(fördelning)
Områden
stationer
Assignment (anvisning) .... etc.
(För enkelhetens skull återges här endast
betydelserna på engelska språket).
sektion l. Regioner och områden
392 § 2. Förtilldelning av frekvenserhar
världen delats in i tre Regioner så som visas
på följande karta och som beskrivs i 393 till
399 .... etc.
Det innebär att tilldelning, fördelning och
anvisning av frekvenser mycket väl kan skilja
mellan ITU-regionerna. Skillnaderna förklaras t.ex. av regionalt olika behovsstruktur,
befolkning etc.
Det förekommer också likheter. På nedanstående karta har markerats en tropisk
zon, vilket förklaras av den annorlunda vågutbredningen där. T.ex. behöver särskild
hänsyn tas vid frekvenstilldelning (allokering) till rundradiotjänsten i zonen.

\section{CEPT}
Begreppet CEPT
Vid sidan av folkrättsligt bindande avtal såsom den internationella telekonventionen
(ITC) - har det internationella samarbetet
lett till överenskommelser som inte är tvingande. Sådana avtal görs bl.a. inom CEPT.
CEPTbetyder Conference Europeanne des
Administrations des Poste s et des Telecommunications, d.v.s. Europeiska konferensen
förpost- och teleadministrationerna. "Konferens" är att förstå som ett ständigt arbetande
samarbetsorgan.
Arbetet inom CEPT har huvudsakligen
karaktär av ömsesidiga programförklaringar
mellan länder. Trots att dessa viljeförklaringar
eller rekommendationer inte är bindande har
de visat sig värdefulla för utvecklingen av det
internationella samarbetet.
Länder anslutna till CEPT förenklar handläggningen av ärenden bl.a. rörande amatörradio genom att ömsesidigt bekräfta
rekommendationer inom området.

CEPT-rekommendationerna
Länder anslutna till CEPT förenklar numera
handläggningen av tillståndsärenden om
amatörradio genom att ömsesidigt bekräfta
och inom sitt land tillämpa rekommendationer som länderna utformat i samråd. Det
innebär att svenska amatörradiobestämmelser kan "harmoniseras" till andra länders.
För kompetenskrav vid examinering av radioamatörer finns CEPT-rekommendationerna TIR 61-01 och TIR 61-02.

CEPT-rekommendation TIR 61-02
Rekommendationen T/R 61-02 innebär att
administrationerna i CEPT-Iänder utger ömsesidigt erkända Harmoniserade Amatörradio Examinerings Certifikat (HAREC) till de
personer som vid nationella prov uppfyller
rekommendationens kunskapskrav motsvarande nivå A respektive B. Dessa HARECnivåer motsvarar kraven för de svenska certifikatsklasserna CEPT i respektive CEPT
2. Radioamatörer med ett sådant certifikat
får utöva amatörradio i annat CEPT-Iand,
som godkänt T/R 61-02 och får tilldelas ett
CEPT-certifikat av det landet utan att behöva genomgå ytterligare kunskapsprov.
Det medger också att en person som
uppvisar ett CEPT-certifkat (HAREC), utfärdat av ett annat CEPT-Iand, tilldelas ett
motsvarande tillstånd vid återkomsten till
hemlandet utan att behöva genomgå ytterligare kunskapsprov.
Rekommendationen godkändes år 1990
och reviderades år 1994 med målsättning
att möjliggöra för icke CEPT-Iänder att delta
i systemet.
Sverige tillämpar T/R 61-02.

CEPT-rekommendation TIR 61-01
Rekommendationen T/R 61-01 möjliggörför
radioamatörer från CEPT-länderna att utöva
amatörradio under korta besök i andra CEPTländer, utan att behöva ett tillfälligt tillstånd
från det besökta CEPT-Iandet. Den godkändes år 1985. Erfarenheterna med detta system är goda. År 1992 reviderades rekommendationen med målsättning att möjliggöra för icke CEPT-Iänder att delta i systemet.

\section{Svenska lagar, bestämmelser}

Lagar, föreskrifter och anvisningar
tillämpas för amatörradioanvändning.
Märk, att ändringar kan tor'eKt'JmJma.
Använd
lagen om radiokommunikation m.fl.
Denna lag reglerar all
i
Sverige. Tillstånd behövs i princip för all
slags radiosändning.
Post- och telestyrelsen - PTS - är enligt
Förordning om radiokommunikation den
svenska myndigheten (administrationen) för
telekommunikation. PTS skall bland annat
svara för att möjligheterna till radiokommunikationer utnyttjas effektivt och har därvid att
beakta den internationella regleringen inom
området. Regleringen av amatörradioanvändningen begränsas nu till den minsta
omfattning som följer av internationella avtal
och europeiska rekommendationer,
CEPT-rekommendationer.

Post- och telestyrelsens föreskrifter om
innehav och användning av amatörrad i om
anläggningar m.m.
Post- och telestyrelsens styrelse beslutar
om Post- och telestyrelsens föreskrifter om
innehav och användning av amatörradioanläggningar. Dessa föreskrifter är anpassade tilllagen om radiokommunikation.
Enligt radiolagen kan ett tillstånd att
inneha och använda radiosändare +r. ..."'"''"'"
med villkor angående kompetenskrav för
den som skall handha radioanläggningen.
För att få ett amatörradiotillstånd måste
man ha ett radioamatörcertifikat, som är ett
kompetensbevis från Post- och telestyrelsen. Med stöd av lagen ställer PTS
tenskrav, att jämföras med artikel S25 i
internationella radioreglementet
åberopar därvid CEPT-rekommendationen
T/R 61-02 som kunskapsnorm för de svenska klasserna CEPT 1 och CEPT 2.
Det innebär att PTS numera tillhandahåller endast dessa två certifikatsklasser.

tillståndsvillkor
Föreningen Sveriges Sändareamatörers
1995: 1 om innehav och
av amatörradioanläggningar
CEPT-rekommendationer för nybörjarcertifikat däremot, finns inte f.n. (år 1997).
CEPT-Iänderna är nämligen meningarna delade om de krav som skulle
rekommenderas. Eftersom CEPT-rekommendationer för nybörjarcertifikat ej finns,
tillgodoses behovet av svenska sådana certifikat på annat sätt.
Med stöd av Post- och telestyrelsens
föreskrifter har Föreningen Sveriges Sändareamatörer- SSA - tillstånd att inneha
och använda amatörradiosändare för Föreningens utbildningsverksamhet inom amatörradioområdet Mot denna bakgrund beslutar SSA:s styrelse en serie anvisningar
där villkoren för SSA-certifikat och SSAtillstånd specifiseras.
.,,.., ..,,.,l!"'',fti!"!!U"'\ 11"1!

litteraturhänvisning om lagar och föreskrifter
Följande kan lånas på de flesta större bibliotek eller kan beställas från Fritzes Förlag
eller Föreningen Sveriges Sändareamatörer:
e Telelag,
• Lag om radiokommunikation,
e Förordning om radiokommunikation,
• Post-och telestyrelsens föreskitter om godkännande av provförrättare m.m.,
• Post- och telestyrelsens föreskifterom innehav och användning av amatörradioanläggningar m. m.,
• Föreskrifter om ändring i Post- och telestyrelsens föreskitter om innehav och användning av amatörradioanläggningar
m.m.,
e Post- och telestyrelsens föreskrifter om
avgifter för certifikat m.m. inom radioområdet
Följande kan beställas från
~o-r.v·on1nncln Sveriges Sändareamatörer:
• SSA:s anvisningar om amatörradioanläggningar vid SSA-tillstånd,
e SSA:s anvisningar om kunskapskrav för
SSA-certifikat,
• SSA:s anvisningar om provförrättning för
SSA-certifikat.

1112-5


1112-6

