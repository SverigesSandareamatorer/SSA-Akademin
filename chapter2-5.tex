\section{Halvledardioden}
\textbf{HAREC a.\ref{HAREC.a.2.5}\label{myHAREC.a.2.5}}

Allmänt
l en strömkrets kan av olika anledningar
ström tillåtas att flyta i en riktning men kanske inte i den motsatta. En anordning med
en sådan funktion kallas för diod.

Bild II 2-12 överst
En halvledardiod består av ett P-ledande
och ett N-ledande materialskikt som fogats
samman.

Först bestod en diod av två elektroder i
vakuum (se avsnitt 2.7). Därav namnet
vakuumdiod.
Numera består en diod oftast av någon
halvledare. Därav namnet halvledardiod.

Mellan de båda skikten utbildas ett tunt
gränsskikt, som inte innehåller laddningsbärare. Detta skikt kan vara ledande eller
icke ledande - ett spärrskikt- beroende på
polariseringen.

n

p

l

+
+

+
+ ·.·':'.

+
+

=l

pn - skikt utan
pålagd spänning

[>l

r-~

l
l

f

1.. - - ·~

,. .... - . .

\

p

+
+

c=::>

n

+
+

+
+

PASSRIKTNING

-

c:::::t> hålledning
....,.,.. elektronledning

+

+
+

p
+
+

+
+

pn- skiktet uppl6ses

n
SPÄRRIKTNING
pn -skiktet byggs upp

---,
l
l
l

,....--'

't

, .... l

l

:--

J

-~

\.. ... .,.J

o
o
-

+

Bild II 2-12 Spärrskiktet i en halvledardiod
112-19

K MP
Halvledardiodens karaktär
Framström, temperatur, förlusteffekt,
passriktning
Bild II 2-12 mitten
Förbinder man den positiva polen på en
spänningskälla med P-skiktet i en diod och
den negativa polen med N-skiktet så är
dioden polariserad i passriktningen. Spärrskiktet upplöses då och ström flyter genom
dioden. Elektronerna flyter till den positiva
polen och hålen till den negativa polen.
Backspänning, backström, läckström, spärrriktning
Bild II 2-12 underst
Förbinder man i stället den negativa polen
på en spänningskälla med P-skiktet i en diod
och den positiva polen med N-skiktet så är
dioden polariserad i spärrriktningen. Spärrskiktet blir då ännu kraftigare.
Endast en obetydlig ström l flyter genom dioden i den s.k. spärriktnfhgen även
vid ökande spänning U . Men över en viss
spänning ökar strömm~h snabbt- den s.k.
zenereffekten uppstår. Dioden kan då lätt
förstöras av en alltför hög ström.

Bild II 2-13
Strömmen 10 börjar att flyta när spänningen U0 har nått ett tröskelvärde (vid kiseldioder 0.6 V). När spänningen ökar ytterligare däröver, så ökar även strömmen.
Produkten av spänningsfallet överdioden
och strömmen genom den kallas förlusteffekt. Denna värmer upp dioden. Vid för hög
temperatur förstörs kristallstrukturen. En kiselkristall kan klara upp till 200
medan en
germaniumkristall klarar bara 75 °C.

oc

*
1
2
3

f

2

3

Allmän symbol
Zenerdiod
Kapacitansdiod

Bild 112-14 Schemasymboler för dioder

lo
50 mA

1v

l
l

l

l

l
l

Bild II 2-13 Halvledardiodens karaktäristik

112-20

20 nA

Uo

MP NENTER
Diodtillämpningar

Likriktning är det vanligaste tillämpningen
(se kapitel3). Halvledardioder görs även för
en rad andra ändamål och finns i en mängd
utföranden, såsom
• Dioder för spänningsstabilisering (zenerdiod).
Inom ett visst område är spänningsfallet
över en zenerdiod i en strömkrets i det
närmaste konstant medan strömmen varierar. Denna egenskap kallas zenereffekt
och används för konstanthållning av spänning.
Det finns zenerdioder
många olika
spänningar och effekter.
•

Dioder som variabel kondensator, s.k.
kapacitansdiod (VariCap).
När en diod är polariserad i spärriktningen så bildas det ett spärrskikt Olika polariseringsspänning alstrar olika tjocka
spärrskikt En spärrad diod har på så sätt
egenskaper som liknar dem i en variabel
kondensator. Det finns därför dioder där
reglerbarheten av kapacitansen är speciellt utvecklad.

•

Lysdioder (LED).
Energi frigörs i spärrzonen i en diod som
är polariserad i passriktningen. Det sker
genom rekombination av par av laddningsbärare, varvid det normalt avgår
energi i form av värme.
Vid en viss inblandning av främmande
atomer avgår istället ljus. Spänningfallet
över en lysdiod är ungefär dubbelt så
stort som över en kiseldiod, d.v.s. ungefär 1.5 volt. Strömmen är i proportion med
önskad ljusstyrka och mellan 1O och 50

mA.

•

o.s.v ..

Vakuumdioden jämfört halvledardioden

Bild II 2-15
Bilden visar principen för hur de båda diodtyperna ingår i en strömkrets. Den stora
skillnaden är att arbetsspänningen för en
vakuumdiod är mångfalt högre än den för en
halvledardiod samt att vakuumdiodens en a
elektrod (katoden) behöver hettas upp för att
avge elektroner.

R

PASSRIKTNING

R

R

SPÄRRIKTNING

R

Bild II 2-15 Dioders polarisering i kretsen
