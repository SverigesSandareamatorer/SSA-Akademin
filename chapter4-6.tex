\section{Mottagningskonvertern}

Bild II 4-18
Konverter betyder i detta sammanhang frekvensomvandlare. När det är önskvärt att
flytta över alla signalerna inom ett helt frekvensområde till ett annat, så används en
mottag ningskonverter där frekvensblandning och frekvensfilter används.
Konvertern fungerar som tillsats före en
mottagare för att denna även skall kunna
användas inom ett annat frekvensområde. l
en konverter är oscillatorfrekvensen fast,
medan avsökningen av frekvensområdet
görs med VFO i mottagaren. Mellanfrekvensfiltret i mottagaren är så brett som hela
det frekvensområde som tas emot av konvertern och avsöks med mottagaren.
Exempel: l en KV -mottagare för området
28-30 MHz vill man även kunna lyssna i
området 432-434 MHz (UHF). Den i konvertern mottagna UHF-signalen förstärks för
att sedan blandas med 404 MHz, en frekvens som uppmultiplicerats från en kristalloscillator (CO) i konvertern. De blandningsprodukter som filtreras fram kommer att
ligga inom området 28-30 MHz oc.~ kan
alltså avlyssnas i KV-mottagaren. Ovriga
blandningsprodukter blir undertryckta i KVmottagarens ingångskretsar.
Blandningsfrekvensen 404 MHz i konvertern är beräknad på följande sätt:

Mittfrekvensen i UHF-bandet är
(432 + 434)/2 = 433 MHz = f 1 •
Mittfrekvensen i KV-mottagarens frekvensband är (28 + 30)/2 = 29 MHz.
Med vilken frekvens f2 måste 433 MHz
blandas för att erhålla en blandningsprodukt
av 29 MHz? 29 MHz är mindre än f 1 , alltså
kan endast skillnadsfrekvensen komma i
fråga. (Vid summafrekvens skulle blandningsfrekvensen bli högre än 433 MHz).
Vid användning av skillnadsfrekvensen
ges två möjligheter:
för f2 - f1 = f2 - 433 = 29 MHz är f2 = 462 MHz
för f 1 - f2 = 433 -f2 = 29 MHz är f2 = 404 MHz
Vi bestämmer oss för alternativet 404
MHz av ett speciellt skäl. Här motsvaras den
högsta UH F-frekvensen 434 MHz av 434 404 = 30 MHz och den lägsta UH F-frekvensen 432 MHz av 432-404 = 28 MHz. På så
sätt kan kHz-graderingen på KV-mottagarens skala användas direkt utan omräkning.
Fördelen med en konverter är att kostnaden för en sådan är låg jämfört med den
för en komplett mottagare för ett tilkommande
band. Förutsättningen är att en mottagare
redan finns.
Nackdelen är att mottagaren inte samtidigt kan användas för sin ordinarie funktion.

UH F-antenn

f1 432 - 434 MHz
432 MHz

eller f1 430 L------l

(> t - - - - - -

UH F-försteg

CJ

~

eller

f2 = 404MHz
f2 =
~1Hz

CJ

/I I~

44,889 MHz

4 .•... MHz

Bild II 4-18 Mottagningskonverter UHF till KV

114- 13

M TTA

