\chapter{Sätt att uttrycka måttenheter}

Inom fysiken förekommer allt mellan
mycket höga och mycket låga värden på
frekvens, spänning, ström, resistans etc.
l en radiomottagares antenningång
är signalspänningen ofta mindre än 11.1V.
l slutsteget i en amatörradiosändare
kan anodspänningen vara mer än 2 kV
och uteffekten upp till1 kW.
l spektrum för elektromagnetiska vågor finns mycket höga frekvenser.
1 000
1 000
1 000
1 000
1 000
1 000
1 00

000
000
000
000
000
W

För att ange storheten på måttenheter används ofta ett prefix före måttenheten (av latinets pre, före och fixare,
att tillägga). Med prefixet anges från fall
till fall vilken multiplikations- ellerdivisionsfaktor (talfaktor) som används. Det finns
ett antal prefix.
Märk, att enhetens sort inte har något
att göra med själva prefixet. Nedan ges
sorterna Hz, W, V, F etc. som exempel.

000
000
000
000
W

000 000 000 Hz = 1 EHz = 1 · 10 18 Hz (E är Exa)
000 000 Hz
= 1 PHz = 1 · 1015 Hz (P är Peta)
000 Hz
= 1 THz = 1 · 10 12 Hz (T är Tera)
W
= i GW = i · i 09 W (G är Giga)
= 1 MW = 1 · 1 06 W (M är Mega)
= 1 kW = 1 · 103 W (k är kilo}
= 1 · 102 W (h är hecto)
1O
= i · i 0 1 W (da är deca)
i V
= 1 V = 1 · 10\(\circ\) V (1 = 10\(\circ\) är grundenhet)
= 1 . 1o- 1
(d är deci)
1:1o
1 : 1 oo
= 1 · 1o-2
(c är cent i)
1 : 1 000 V
= i mV = 1 · 1o-3 V (m är milli)
i : 1 000 000 V
= 1 )l V = 1 · i o-e V (!l är mikro)
1 : 1 000 000 000 F
= 1 nF = 1 · 1o-9 F (n är nano)
1 : 1 000 000 000 000 F
= i pF = 1 · 1o- 12 F (p är pico)
1 : 1 000 000 000 000 000
=1 f
= 1 · 1Q- 15
(f är femto)
1 : 1 ooo 000 ooo ooo 000 000 = 1 a
= 1 · 1o- 18
(a är atto)
Exponenter, t. ex. siffran 6 i uttrycket 106 , förklaras i appendix B.

Mer om att uttrycka måttenheter
En decimal talstorhet uttrycks ofta med ett s.k. tekniskt flyttaL Decimaltecknet placeras då så att den
visade tio-exponenten i talet blir en multipel av 3. Se
exempel i ovanstående uppställning.
Decimaltecknet kan även placeras så att tioexponenten är något annat än en multipel av 3.Talstorheten uttrycks då med ett s.k. tekniskt flyttaL
l miniräknare m.m. visas ofta exponenten som
bokstaven E, åtföljt av ett värde. Ibland utelämnas
själva bokstaven medan exponentvärdet står kvar.
Ex. 1000 visas som 1 · 103
eller 1 E+03
125 visas som 1.25 · 102 eller 1.25 E+02
1O
visas som 1 · 10 1
eller 1 E+01
1
visas som 1 · 10\(\circ\)
eller 1 E+OO
0.1
visas som 1 · 1o-1
eller 1 E -01
0.01 visas som 1 · 1o-2
eller i E -02
0.012 visas som 12 · 1o-3 eller 12 E-03

Metallers resistivitet
Ämne
Resistivitet
vid 20\(\circ\) C
Q· mm 2

m
Aluminium
Bly
Guld
Järn
Koppar
Kvicksilver
Nickel
Platina
Silver
Tenn
Volfram
Zink

0.028
0.22
0.024
0.105
0.018
0.958
0.078
0.1 08
0.016
0.115
0.056
0.058

A-1

Bokstäver ur bl. a. grekiska alfabetet används som symboler för tekniska begrepp.
Märk, att samma symboler används olika inom olika teknikområden.
Här anges några användningar inom elektroniken.
Versaler
"stora"
bokstäver

A

B

a
~

r

y

E

z

E

e

11

A

'A

il

I

M
N

ö

s

v

t
K

Jl

v

.!::..

~
o

p

1t

o
L
T

p

()

't

y
cil

u

x
n

<p

'P

A-2

Gemena
"små"

x

'V
ro

Uttal

Alpha
Beta
Gamma
Delta
Delta
Epsilon
Z eta
tE ta
Teta
Jota
Kappa
lambda
My
Ny
Xi
Om ikron
Pi
Rh o
Sigma
T au
Ypsilon
Fi
Fi
Chi
Ps i
Omega
Omega

Användningsexempel

ledningsförmåga
Del av .. storhet
Förlustvinkel etc.
Dielektricitetskonstant
Verkningsgrad
Vinklar
Kopplingskoefficient
Våglängd
Permeabilitet
Frekvens

3.14159 ...
Resistivitet
Summa
Tidskonstant
Magnetiskt flöde
Fasvinkel
Resistans
Vinkelfrekvens
