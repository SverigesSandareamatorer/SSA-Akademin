\documentclass[a4paper,twoside,twocolumn,openright]{book}

% Prepare for abstract
\usepackage{fancyhdr}
\pagestyle{empty}
\newenvironment{abstract}%
{\cleardoublepage\null \vfill\begin{center}\bfseries Abstract \end{center}}%
%{\cleardoublepage\null \vfill\begin{center}%
%\bfseries \abstractname \end{center}}%
     {\vfill\null}

% Prepare for index
\usepackage{makeidx}
\makeindex

% Prepare for svenska tecken
\usepackage[T1]{fontenc}
\usepackage[utf8]{inputenc}
\usepackage[swedish]{babel}

% Prepare for tables
\usepackage{multirow}

% Prepare for lists
\usepackage{enumitem}

% Prepare for graphics
\usepackage{xspace,graphicx}

\begin{document}

\frontmatter
\title{KONCEPT FÖR RADIOAMATÖRCERTIFIKAT}
\author{Lennart Wiberg}
\maketitle

Lennart Wiberg

KONCEPT FÖR RADIOAMATÖRCERTIFIKAT
Första upplagan
ISBN 91-86368-08-7

Att mångfaldiga innehållet i denna bok, helt eller delvis, utan medgivande av
rättighetshavarna, är förbjudet enligt lagen (1960:729) om upphovsrätt till
litterära och konstnärliga verk. Förbudet gäller varje form av mångfaldigande
såsom tryckning, kopiering, överföring till annat medium etc.
Copyright © Föreningen Sveriges Sändareamatörer
Denna elektroniska faksimilutgåva är baserad på den första tryckta
upplagan. Notera att information om certifikatsklasser, lagar och
förordningar har ändrats sedan tryckningen.
Tryckt i Sverige / Printed in Sweden 1997
Smegraf, Smedjebacken


Förlag
Föreningen Sveriges Sändareamatörer
Box 45, S-191 21 Sollentuna
Telefon +46 8 585 702 73
E-post hq@ssa.se


\tableofcontents

\mainmatter


c
Rättelser
Sidorna ii-ix.

Innehållssidorna
Här har skett en omflödning med en rad per spalt.
Således skall respektive spalts sista rad komma
överst i nästföljande spalt.

Sida A -1.

Rutan "Mer om att uttrycka måttenheter"
Texten "tekniskt flyttal'' på rad 8 skall vara: allmänt flyttal

Sida 113- 12.

Vänstra spalten
Texten "bild 3-000" på rad 14 skall vara: bild 3-17

Sida 113- 13.

Bild II 3-16 Seriekopplad LC-krets
Bilden skall vara den nedanstående

XL

Xc
~~YY~-i )1:------,,-----<o
-ca

.....

spänning
över XL

-ca

spänning över XL

.....

;spänning
; över Xc
ström

spänning över Xc

Bild II 3-16 Seriekopplad LC-krets

IN
Denna faktabok omfattar det av Post- och telestyrelsen anvisade kompetensområdet för
radioamatörcertifikat.
Innehållet är delat i tre ämnesgrupper; inlärningsanvisningar för morsesignalering,
grundläggande radioteknik samt regler och trafikmetoder. l appendix finns bland annat
grundläggande matematik och frekvensplaner för amatörradiotrafik.
Separata SSA-kurser i praktisk morsesignalering finns på ljudband och datadisketter.

\chapter*{Förord}

\section*{Amatörradio}

Amatörradio är en teknisk hobby med inriktning på kommunikation och experiment med
radioanläggningar samt radiovågors utbredning. Det är en verksamhet som utövas över
hela världen av licensierade radioamatörer, även kallade sändaramatörer.

Syftet med amatörradio är att främja personlig utveckling och internationell förståelse
samt teknisk färdighet och erfarenhetsutbyte inom området. Amatörradio kan därtill vara
en tillgång då samhällets normala resurser för radiokommunikation behöver förstärkas.

\section*{En hobby med krav}

För att inneha och använda en radioläggning i ett land, krävs tillstånd (licens) från
dess teleadministration. För ett amatörradiotillstånd föreskrivs i det internationella
radioreglementet bland annat handhavandemässiga och tekniska kvalifikationer hos
varje person som önskar använda en amatörradiostation. De nationella teleadministrationerna tillser detta genom kompetensprov.

\section*{Utbildningsställen}

Amatörradioklubbarna bedriver huvuddelen
av utbildningen med radioamatörcertifikat
som mål. Även vissa skolor, militära förband
m.fl. har amatörradio på programmet. l någon utsträckning förekommer även självstudium. Rekrytering av handledare för terminslånga kurser är en nyckelfråga för kursarrangören, liksom målinriktade, anpassade läromedel.

\section*{Andra förutsättningar}

Den svenska teleadministrationen har omdanats på senare tid. Därvid har även amatörradioanvändningen berörts, främstgenom
att provförrättningarna för amatörradiocertifikat delegerats till av myndigheten utsedda,
ideellt arbetande förrättare. Vidare genom
att teleadministrationerna inom CEPT har
infört harmoniserade certifikats- och tillståndsklasser för amatörradio.
Främst av dessa anledningar har det
uppstått behov av samordnade hjälpmedel
för utbildning och examinering, vilket amatörradiorörelsen själv har att tillgodose.

\section*{Föreningen Sveriges Sändareamatörer - SSA}

SSA är en ideell förening för personer med
intresse för amatörradio. Verksamheten är
religiöst och politiskt obunden. Ett av syftena
är att bland medlemmarna verka för ökade
tekniska kunskaper och god radiotrafikkultur
för att därigenom skapa en kår av kunniga
radioamatörer. SSA representerar Sverige
som nationell förening i The International
Amateur Radio Union (IARU), Region 1.

\section*{Internationell samverkan}

De nationella föreningarna inom IARU samarbetar över nationsgränserna. Ett exempel
är när DARC (Deutscher Amateur-RadioCiub e. V.) för några år sedan ställde sina
Ausbildungsunterlagen till SSAs förfogande
som källmaterial för boken El-lära och Radioteknik.
Detta material har till stor del kunnat
utnyttjas även i här föreliggande bok.

xi

\pagebreak

TACK!
Ett stort tack till alla dem, som på olika sätt
bidragit till att förverkliga boken.
Ett särskilt tack till Bengt Falkenberg
SM7EQL och Bertil Nordahl SM7CZL, vilka
har varit rådgivare och sakgranskare.
Tack också till Ulf Sjöden SM6CVE för
svenska texter för bilderna från Ausbildungsunterlagen.

Författaren

xii

\pagebreak

\chapter*{INLEDNING}

VAD HUR VAR

\section*{VAD behöver en radioamatör kunna?}

CEPT är ett samarbetsorgan mellan europeiska länders teleadministrationer (myndigheter). En av dem är svenska Post- och telestyrelsen - PTS.

Dessa administrationer har antagit rekommendationer om sinsemellan harmoniserade krav på radioamatörers kompetens.


Sverige har antagit CEPT-rekommendationen T/R 61-02. Vid genomförandet av
kompetensprov skall de i den rekommendationen angivna kraven särskilt beaktas.


För den som godkänts i ett sådant prov utfärdas ett harmoniserat radioamatörcertifikat
(HAREC). Rekommendationen anger kompetensnivåerna HAREC A och HAREC B. De svenska
certifikatsklasserna CEPT 1 respektive CEPT 2 motsvarar dessa nivåer.

\begin{itemize}
\item Kompetenskraven för klassen CEPT 1
(HAREC nivå A) omfattar

Ämnesgrupp I - Färdighet i morsesignalering,

Ämnesgrupp II - Kunskaper i radioteknik,

Ämnesgrupp III - Kännedom om reglemente och trafikmetoder.

\item Kompetenskraven för klassen CEPT 2
(HAREC nivå B) omfattar endast

Ämnesgrupp II- Kunskaper i radioteknik,

Amnesgrupp III- Kännedom om reglemente och trafikmetoder.

Eftersom certifikat av denna klass ej dokumenterar föreskriven färdighet i
morsesignalering, är de lägre frekvensbanden för amatörradio ej tillgängliga för klassen.
\end{itemize}

\subsection*{HUR blir man radioamatör?}

För att få inneha och använda amatörradiosändare måste man ha Post- och telestyrelsens
amatörradiotil/stånd (licens). Det får man ansöka om efter att ha avlagt certifikatsprov
i avsedd klass med godkännande.

Till tillståndet knyts en internationellt unik anropssignal.

\subsection*{VAR hålls det certifikatskurser?}

Vissa amatörradioklubbar, militära förband,
FAO-förbund och andra sammanslutningar
håller certifikatskurser. Det går också att
studera på egen hand.

\subsection*{VILKA läromedel behöver man?}

Denna bok omfattar teorin för certifikatsklasserna CEPT 1 och CEPT 2.

För ämnesgrupp l behövs även någon separat kurs i praktisk morsesignalering.
Sådana finns på ljudband eller datadiskett

För ämnesgrupp III behövs även gällande lagar, föreskrifter och anvisningar inom
området.

Alla dessa läromedel kan köpas från SSA.

\part{MORSESIGNALERING}

Många slags signaler har genom tiderna använts för att sända budskap. Till en början
användes akustiska och optiska signaler, det var rop, hornstötar, rökpuffar, ljusblinkar,
signalflaggoro.s.v. Undertidigt 1800tal började man sända meddelanden med hjälp av
elektriska impulser genom ledningar. År 1837 presenterade amerikanen Samuel
F B Morse en elektromagnetisk skrivtelegraf. Redan i början på 1840-talet hade han
förbättrat apparaten och utvecklat ett system, som i stort bibehållits in i våra dagar.
Flera andra personer har med tiden vidareutvecklat den teckenkod som Morse först
formulerade och kompletterat den med skiljetecken och ytterligare andra tecken. Koden
kallas fortfarande MORSE-koden. Kommunikationssättet kallas telegrafi och betyder
fjärrskrift (av grekiskans tele= fjärr och grafein = skriva).

Grundprincipen för telegrafi är densamma än i dag, men nu används mest maskinella
hjälpmedel, både vid sändning och mottagning. Jämsides med morsekoden, som utformades för
manuell signalering, har det utvecklats signalkoder som är speciellt avsedda för
signalering med t.ex. teleprintrar, telefaxmaskiner och datorer. Men trots den snabba
tekniska utvecklingen överförs fortfarande meddelanden manuellt med morsesignalering.
Metoden hävdar sig nämligen speciellt bra under svåra atmosfäriska och trafikmässiga
förhållanden samtidigt som den tekniska utrustningen kan vara förhållandevis enkel.
Därför lever den 160-åriga morsesignaleringen vidare.

\subsection{Morsesignalering inom amatörradion}

Med amatörradio har människor av många nationaliteter och med många olika yrken
och bakgrunder mycket goda kontaktmöjligheter. Ett roligt sätt att ha kontakter över
radio är då att morsesignalera. Det är ett levande sätt att uttrycka sig. Radioamatörer
tar sig gärna en pratstund eller deltar i tävlingar på det sättet. Det hindrar dock inte
att många andra sändningsslag också kommer till användning.

För tillträde till amatörradiofrekvenserna på kortvåg (under 30 MHz), därtrafiken är tät
och räckvidden lång, föreskrivs fortfarande i det internationella radioreglementet att
radioamatörerna skall ha färdighet i morsesignalering.

\subsection{Morsetecknen}
Bild I.1

Morsetecknen består av korta och långa teckendelar samt mellanrum. Man utgårfrån
den korta teckendelen vars längd sätts till en enhet. En lång teckendel skall vara tre
enheter lång, d.v.s. tre gånger längden av den korta teckendelen. Mellan teckendelarna
inom tecknet skall mellanrummet vara en enhet långt. Mellan hela tecknen inom ord
eller teckengrupp skall mellanrummet vara tre enheter långt och mellan hela ord eller
teckengrupper sju enheter långt.


Bild I.1 Morsetecknens uppbyggnad

\subsection{Planlagd övning}

Att delta i en organiserad kurs i den lokala radioklubben, FAO-avdelningen etc. är bra,
eftersom man då kan få en handledare och tillgång till övningsmateriaL Inte minst viktigt
är stödet av studiekamraterna. Det går också att på egen hand lära sig att signalera, men
det är ensamt och därför kanske lite svårare. De kassettband och datadisketter som kan
köpas från SSA kan användas i båda fallen. För att lära morsesignalering måste man
vara motiverad. Det krävs nämligen tålamod och regelbunden träning. Helstbörträningen
ingå i den personliga dagliga rutinen, även om det bara blir under några minuter.
Det går att hoppa över 1-2 dagar i veckan, men det bör då ingå i övningsplanen. Att
hoppa över ännu fler blir lätt en ovana. Att träna lite då och då ger inget bra resultat.

\subsection{Ordning för teckeninlärning}

Bild 1.2
Inlärningsordningen enligt bilden rekommenderas. Man börjar med tecken som låter så
olika som möjligt. Detta för att undvika förväxling längre fram, när tecknen blir fler och
hastigheten högre. Under inlärningen blandas nya tecken med de redan inlärda. Följ
kursens ordning och hoppa inte över något!
Öva utan avbrott, så att hjärnan blir ordentligt "programmerad". Det är lämpligt att lära
in 2 till 4 nya tecken varje vecka.

SSA:s telegraferingskurser på band och datadiskett har denna inlärningsordning. Det
finns dock kurser med annan uppläggning.

\begin{center}
\begin{tabular}{|r|l|}
\hline
Lektion & Nya tecken \\ \hline
1 & = + L N E O \\
2 & I X \\
3 & V T \\
4 & / ? + (vänta) \\
5 & - x (repetition) \\
6 & A Z \\
7 & . (punkt) \\
8 & H Ö \\
9 & 7 4 9 5 \\
10 & 8 1 \\
11 & 3 6 \\
12 & R D \\
13 & 2 0 \\
14 & F Y \\
15 & Ä B \\
16 & P S \\
17 & U Q \\
18 & W K \\
19 & Å M \\
20 & C G J \\
21 & ~ (lystring) \\
22 & @ (avslutning) \\
23 & f (förstått eller felslagning) \\
24 & ........ (felslagning) \\
\hline
\end{tabular}
\end{center}

Bild 1.2 Inlärningsordning för morsetecken

\subsection{Inlärningstid}

Behövlig inlärningstid är mycket individuell. För ett UC-certifikat (40 tecken/minut),
vilket är ett utbildningscertifikat från SSA, bör man räkna med åtminstone 100 effektiva
timmar för mottagning och 25 timmar för sändning.

Att klara CEPT i-certifikatet innebär en taktökning till 60 tecken/minut och högre
krav på säkerhet i både mottagning och sändning, För det bör man räkna med ytterligare
25 timmar eller mer.

\subsection{Inlärningsmetodik}

Morsetelegrafi bör läras med beprövad metodik. Bästa sättet är att man också skriver
ner tecknen när man hör dem. Metoden är s.k. "nervbaning" med målet att handen
reflexartat skriver ett visst tecken då en viss rytm hörs. Att träna bara genom att höra
tecknen är nästan verkningslöst. Först när Du lärt dej alla morsetecken grundligt genom
mottagning är det dags med sändningsträning.

\subsection{Mottagningsövningar}

Morsetecken är ju långa och korta teckendelar i form av ljud, ljus etc. De kan även
illustreras som långa och korta streck. För att tecknen skall uppfattas som en melodi
eller ljudföljd och för att man inte skall frestas att räkna korta eller långa
teckendelar är lämpligt att morsetecknen lärs in i hög hastighet, men med förlängt
mellanrum, s.k. spärrad stil. Det är själva ljudbilden som skallläras in.
I början kan det ändå vara svårt att låta bli att räkna teckendelar, men efterhand
uppfattar man trots allt tecknen som ljudbilder.

När man skall skriva mycket under en längre tid är sittställningen viktig. För att inte
bli trött skall man försöka inta en avslappnad sittställning och låta hela underarmen
vila mot bordet.

Använd papper med stora rutor och en bra kulspetspenna. Skriv gärna på varannan
rad så att det finns plats under att rätta texten.

För att spara tid bör man använda små handrörelser och inte lyfta pennan mer än
nödvändigt. studera skrivanvisningarna i slutet av detta kapitel. Använd för tydlighetens
skull textad stil, men tydlig skrivstil går också bra. Lyssna på hela tecknet innan du
skriver ner det. Skriv lugnt. Hoppa över tecken som du missar! Försök inte att minnas
tecken som du just missat. Då kommer du nog att missa efterföljande tecken också.
Koncentrera dig i stället på tecknen som kommer.

Vissa morsetecken är så korta att det är svårt att hinna skriva ner dem. För att spara
tid måste vissa tecken skrivas i ett penndrag, t.ex. bokstäverna M, N m. fl.. Bokstaven E
som är det kortaste tecknet skriver man som en bakvänd trea ( $\epsilon$ ). Bokstaven U
bör formas fyrkantig och bokstaven V spetsig, annars förväxlas de lätt. En nolla skrivs
som Ø, med genomstrykning och en etta som 1. En nolla utan streck kan lätt förväxlas med
bokstaven O och en etta utan fot med bokstaven I.

Var noga med handstilen från början och jobba hela tiden med att förbättra den. En
olämpligt inlärd handstil är mycket svår att arbeta bort och då får man problem vid högre
hastigheter. Du skall ju själv kunna tyda din text i efterhand, men viktigast är att
provförrättaren också skall kunna läsa den.

Inlärningstexter är ofta uppdelade i grupper med 5 eller 4 tecken. Dessa skall simulera
ord. Var noga med att du får tydliga ordmellanrum även på papperet.

\subsection{Eftersläpning vid mottagning}

Tiden för vart och ett morsetecken varierar kraftigt. För att få en lugnare nedskrivning
bör man försöka hålla några tecken i minnet och släpa efter med nedskrivningen. Detta
är nödvändigt i högre hastigheter och särskilt vid vissa teckenkombinationer.

Läs inte! Det är frestande att försöka bilda ord av de bokstäver som man just
skrivit ner. Läsningen tar bort uppmärksamhet från mottagningen och det blir lätt felgissningar. Man tappar lätt den text som man just då tar emot. Läs alltså inte och
gissa inte på orden. Täck över det skrivna med den lediga handen!

\subsection{Sändningsövningar}

Att telegrafera är att uttrycka sig. De handsända morsetecknen skall vara tydliga, på
samma sätt som att tal och vanlig handskrift skall vara det. Det är därför mycket viktigt
att teckengivningen lärs in på rätt sätt. Speciellt de första sändningsövningarna bör ske
tillsammans med en kunnig instruktör.
Om instruktör saknas --- följ då noga anvisningarna och var självkritisk!

\subsection{Hjälpmedel vid sändningsövning}

För sändningsövningarna behövs en kassettbandspelare eller en dator med SSA:s
dataprogram "Träna Morse". Vidare behövsen summer ansluten till en telegrafnyckel
och en stereohörtelefon. Eventuellt kan man ha en andra summer som nycklas av datorn
och vars ljud matas i en av hörlurarna.

Lär in sändning med en manuell telegrafnyckel och inte med en s.k. bug. Vid provtagning blir man nämligen ofta nervös och då är det lätt att sända fel med en bug. Särskilt
med en el-bug är risken stor för nya fel "bara för att man råkat snudda vid fel paddel".
Vid rättning med bug kommer därför sällan ett fel ensamt.

\subsection{Arbetsställning vid sändning}

Bilder 1.3 och 1.4

Det är viktigt att ha rätt arbetsställning redan från början. Vid hög sändningstakt och
långasändningspass blir man annars lätt trött och får dålig teckengivning.
Över 60-takt börjar rätt arbetsställning att få stor betydelse.

Vid trötthet under sändning höjer man ofta axeln varvid armbågen åker ut. Det blir
då arbetsamt och man får "bryta sig" genom slutet på texten under dålig teckengivning.

Sitthöjden bör vara så att båda fötterna kan vila på golvet eller på en fotpall.
Telegrafnyckeln bör placeras så, att underarmen är vågrät när handen vilar på
nyckelknoppen. Överarmen kan då hänga avslappnad rakt nedåt och över- och underarmen kan
bilda en rät vinkel.

Bild 1.3 Rätt sittställning sett framifrån


Bild 1.4 Rätt sittställning sett från sidan

Nyckeln bör vara fastsatt. Det är tyvärr vanligt, att nyckeln ställs löst på ett olämpligt
högt bord. Detta medför en olämplig och tröttande arbetsställ ning.

\subsection{Nyckelfattning och handrörelser}

Bild 1.5

Håll löst omkring nyckelknoppen med tummen och långfingret. Pekfingrets undersida
skall vila lätt ovanpå knoppen. Använd alltid denna trefingerfattning. Håll nyckelknoppen
ganska långt in på fingrarna. När man senare vill öka takten kan man flytta ut fattningen mot fingerspetsarna.

Morsetecknen skapas med rytmiska handledssvängningar uppåt/nedåt. Håll inte
hårt om knoppen- men släpp den inte heller - och spänn inte handleden. Handleden
skall svänga mellan ett något upplyft och ett vågrätt läge. I det vågräta läget når
nyckeln sitt s.k. kontaktläge. För att nästa tecken skall hinnas med i tid, får handleden
inte svänga djupare än till det vågräta läget.

Bild 1.5 Rätta handledsrörelser

\subsection{Styrd sändning}

Sändningsövningarna börjar med styrd sändning, men först sedan morsetecknen lärts in
grundligt genom lyssning. Vid styrd sändning använder man stereohörlurar så att
datorns eller bandspelarens sändning hörs i den ena luren och den egna sändningen i
den andra. Den ton som nycklas hämtas från en generator som avger en konstant ton.

En textutskrift används som förlaga för den egna sändningen. Det gäller att lyssna
på morsetecknen från datorn eller bandet, samtidigt läsa samma tecken från utskriften
och själv sända dessa med nyckeln. Ljudbilden från en egna sändningen skall då
sammanfalla med den från förebilden. På så sätt samövas hand- och armmusklerna, synen
och hörseln för rätt teckengivning.

I SSA:s kurser på ljudband och data finns rytmiska ramsor för övning av styrd sändning.
Börja med att öva ramsorna i nummerordning. När man blir säkrare behöver man
inte alltid träna alla ramsor. Man känner själv vilka ramsor som man behöver öva mera.

Styrd sändning övas utan spärrning.
Teckenhastigheten och trafikhastigheten ska då vara lika. Trafikhastigheten bör
åtminstone vara 35 till45 tecken/minut för attteckenrytmen skall bli bra. Träna mycket på
siffror i den styrda sändningen. Det ger färdighet vid övergångarna mellal korta och långa
teckendelar i tecknen. Även mellan vissa morsetecken kan övergångarna vara svåra.

\subsection{Fri sändning}

Först skall styrd sändning av ramsor och tecken klaras utan problem. Börja först därefter
med fri sändning utan ljudförebild. Normalt skall sändning göras utan spärrning.

Försök komma ihåg teckenrytmen från den styrda sändningen. Siffror och skiljetecken är
svårast att sända. Öva därför dessa tecken extra mycket. Då blir också bokstäverna
lättare att sända!

Sänd inte fortare än att handleden fortfarande arbetar mjukt vid kontaktläget, men
ändå distinkt. Sändningen är ditt visitkort och därför krävs att den har kvalitet.

Bild 1.6 Telegrafnyckel

\subsection{Kontroll av teckengivningen}

Läsbarheten av den sändningsstil, som uppvisas vid certifikatsprovet, bedöms. Ett i
övrigt godkänt prov kan alltså bli underkänt p.g.a. dålig teckengivning. Återkalla därför
ett tveksamt utformat morsetecken och sänd om det, men var då klar över att provtexten
ökar med det antal tecken man sänder om. Det innebär tidsförlust.

För kontroll av teckengivningen under sändningsprovet, och registreringen av det,
användes förr en teckenskrivare med pappersremsa. Emellertid är en sådan skrivare
numera ett svåråtkomligt hjälpmedel.

Det hjälpmedel, som i stället stårtill buds, är en ljudbandspelare, men tyvärr har ju en
sådan inte grafisk visning. En telegraferingskunnig person bör därför anlitas för
bedömning av teckengivningen.

\subsection{Exempel på provtext}

.... ASCUNCION ÄR HUVUDSTAD I PARAGUAY, SOM LIGGER I SYDAMERIKA.
VID RADIOTELEFONERING UTGÖRES NÖDSIGNALEN AV ORDET MAYDAY
OCH SKALL OM MÖJLIGT UTSÄNDAS PÅ FREKVENSEN 2182 KHZ. QRV? MEANS
ARE YOU READY? THE RECEIVER CONTROL SETTINGS SHOULD BE
ADJUSTED AS INDICATED ON PAGE 3-4,
DATED 7/10 1994. QRT + @
(summa 277 teckenvärden)

\subsection{Beräkning av antalet teckenvärden}

Vid beräkning av antalet teckenvärden i en telegramtext skall bokstäver (utom Å) räknas
som ett (1) teckenvärde. Siffror, skiljetecken, felsändningstecken samt bokstaven
Å skall räknas som två (2) teckenvärden.

I ovanstående exempel på provtext är
fördelningen av teckenvärdena följande

$\begin{array}{lccrcr}
Bokstäver & 1 & \cdot & 227 & = & 227 \\
Siffror & 2 & \cdot & 13 & = & 26 \\
Skiljetecken & 2 & \cdot & 12 & = & 24 \\
summa teckenvärden & & & & = & 277
\end{array}
$

Observera, att lystrings-, slut- och avslutningstecken samt felsända avsnitt med
respektive felsändningstecken också skall ingå i summan av teckenvärden.

Den därefter beräknade takten är den s. k. telegram- eller trafikhastigheten.

\subsection{Beräkning av takten}

Formel:

$
\frac{summa teckenvärde \cdot 60}{tid [sekunder]} = tecken / min
$

Exempel: Felfri sändning av ovanstående exempel på provtext med summa teckenvärde 277 tar exakt 4 minuter och 20 sekunder (260 sekunder). Takten blir då

$
\frac{277 \cdot 60}{260} = 63.9 tecken/min
$

MORSEALFABETET
A
B

c
D

1

----

----

2

3
4

5
6

E
F

Bild 1.7

lim---

G

7

H

8

l

J
K
l

M
N

o

p
Q

IIIICiBIIIIIIIII!I!!I!ll!lil!lt'.!!llll!l!ll:

T

u
v

x
y

z
A
Ä

ö
sos

E

o

N

CH

11-6

-1111--

----------

------

1!11



.

----Dl!li

1111

D

ll!ll

1!111

----

- -

IBil

l!ilill

•

mm

-----

o

ibland förkortat

------

Punkt

=
~
f

s

11!1!

---II!BilEIIS-

.......... """'

R

---l!!!!ilmlllt

----\&!lilllml'!

?

lilll'!lmBEi211!11!1i1UIBIII!IIal!l

a:
ID

9

-------

-----

+
@

(

)

"

lillllllimli.BII!IIIBli8Bii!fi!IIBII2IIallllli5ll1llill!l

Kommatecken

ll!lllimliRDIEII111t4C11111MIII1

Frågetecken

------

l!lillll'lS!ll

---llillliiZIIIII11'i!21!11:DIBI!II!Imll!l!lllllR\&IEB

Bråkstreck

lli'illS'!IIS----

Åtskillnadtecken

-----

lystringstecken

----

lmf

Väntatecken

•

IIIIii

Förstått-tecken

---

-----------------

Ell!lill!lll!II!!Bii!Uili!D!IIlli!SIII!lll!!

------



............

Avslutningstecken
Vänster parentes
Höger parentes

Bl!lli\&f2llliiiii!IEIIIIIillliBm

Utropstecken

l!llll

li!ll!

-----~~~~

x

sluttecken eller plustecken

Apostroftecken

liiB!IIIlllll!l!lim!;IBiill!l:llll--

--------

Felsändningstecken

-----------

x

IIIIIt----

Bindestreck eller minustecken

- - - ll!liiD!III-

III'IIIE1!11111Et11ll!:l'l!lll!llili!IIIIIIIISUIII!DIB!II!il

- - -

tm

----

Anföringstecken
Understrykningstecken
Repetitionstecken
Kolon eller divisionstecken
Semikolon
Multiplikationstecken

SKRIVSTIL

TEXTADSTIL

~

.........a

©

~

3

1

a

a

.,
a

~l

- -s

-a

.....,..a

-

a

detta skrivsätt behöver man sällan lyfta pennan.
Bokstäverna är formade så att de inte lätt förväxlas.

-a

l
""-.!

Bild 1.8 Handstilar

\part{RADIOTEKNIK}

\chapter{El-lära}

\section{Elektriska grundbegrepp}

Elektrisk laddning, spänning och ström hänger samman med hur materian är uppbyggd.
Den förmåga ett material har att leda laddningar, d.v.s. ström, kallas konduktivitet.

\subsection{Grundämnen}

Det finns många former av materia. Ofta är en form av materia sammansatt av andra
former med enklare uppbyggnad.
Sammansatt materia kan sönderdelas på kemisk väg, men
däremot inte de enklaste formerna. All materia är uppbyggd av atomer. De enklaste
materieformerna, som kallas grundämnen, innehåller endast ett slags atomer. Över
100 grundämnen är kända.
Vart och ett av grundämnena har sin speciella atomuppbyggnad och därmed en
materialstruktur, som skiljer sig från varje annat grundämne.
Tre fjärdedelar av alla grundämnen är metaller (elektriska ledare) medan de flesta
övriga är icke-metaHer (isolatorer). Det finns även en liten mellangrupp som kallas för
halvledare.

\subsection{Atomernas uppbyggnad}

Länge ansågs atomerna vara de minsta
beståndsdelarna i materian. Men omkring
sekelskiftet upptäcktes att atomerna i sin tur
består av ännu mindre beståndsdelar, s.k.
elementarpartiklar såsom protoner, neutroner, elektroner m .fl. Det gemensamma namnet för alla dessa partiklar är nukleoner.
En atom består dels av en kärna, som är
sammansatt av protoner och neutroner, dels
av elektroner, som kretsar omkring kärnan.

Protonerna är positivt (+)laddade.
Neutronerna är neutrala, ej laddade.
Elektronerna är negativt (-) laddade
Bild II i -1 Elektronerna kretsar i banor
omkring atomkärnorna, liksom planeterna
kretsar i banor omkring sina solar.

Atomkärna
Proton
'--\-~~- Neutron
-

Elektronskal

Bild II 1-1 Atomernas uppbyggnad
Banor med samma avstånd till atomkärnan är på samma energinivå och sägs bilda
ett elektronskal.
Det kan finnas flera elektronskal. Ju fler
elektroner som finns i ett elektronskal, desto
starkare är elektronerna i skalet bundna till
atomen. Det yttersta skalet kan emellertid
aldrig innehålla fler än 8 elektroner.
Elektronerna i det yttersta skalet kallas
för valenselektroner, vilka används även av
angränsande atomer vid den kemiska bindningen till atomstrukturer, molekyler och
ämnen. För bindningen behövs ett visst
antal valenselektroner.

De valenselektroner som ej behövs för
bindningen kan röra sig fritt genom materia/strukturen. De kallas fria elektroner och är vad vi kallar elektrisk ström.
Valenselektronerna är alltså inte bara av
betydelse för materialets kemiska struktur
utan också för dess elektriska egenskaper.
Atomernas massa och volym är ytterst
liten. Tag som exempel en kopparkub med
volymen 1 cm 3 och vikten 8.9 gram. Den
består av c:a 8.5 · 1025 kopparatomer, d.v.s.
85 000 000 000 000 000 000 000 000
stycken.

IIi- i

ElVarje elementarpartikel har en massa och
en atoms totala massa är summan av
elementarpartiklarnas massor. Den enklaste
atomen är väteatomen med en proton och
en elektron. Väteatomens totala massa har
kunnat beräknas till 1.66 . 1o- 24 gram.
Nästan hela massan i atomen är samlad
till kärnans protoner och neutroner. Var och
en av dem har en massa som är ungefär
2000 gånger större än massan i en elektron.
1 cm 3 av koppar innehåller t. ex. 10 23 stycken
fria elektroner.

\subsection{Elektrisk laddning och kraftverkan}

Enligt sägnen upptäckte Thales från Milteus
redan för 2500 år sedan, att en bit bärnsten
drog till sig små grässtrån, sedan stenen
gnidits mot en bit ylle. Det grekiska ordet för
bärnsten är ELEKTRON och de krafter som
uppstod kom att kallas "elektriska". Av den
elektriska spänningen mellan kroppar med
olika laddning, verkar krafter mellan dem
och deras omgivning. Krafterna kallas för
elektriska fält och är det som gör att elektriskt
laddade kroppar kan komma i rörelse.
Ett exempel får man varje gång man
kammar sig med en kam av isolerande material. Då kommer håret att dras mot kammen därför att håret och kammen har fått
olika slags elektriska laddningar. Samtidigt
har hårstråna sinsemellan samma slags laddning och stöter bort varandra- håret "reser
sig".
Lika laddningar stöter bort varandra.
Olika laddningar drar varandra till sig.

\subsection{Konduktivitet - Ledare, halvledare och isolator}

En elektrisk ström sägs flyta, när de fria
laddningsbärarna i ett material -en strömledare - fås att röra sig samtidigt i samma
riktning. Hur många som rör sig beror på
strömledarens egenskaper och spänningen
mellan ledarens ändar.
Alla material har någon grad av elektrisk
ledningsförmåga som beror på materialets
atomstruktur, dimensioner och temperatur.
Vissa material (t.ex. metaller, kol, halvledare) leder elektrisk ström bättre än andra
(t.ex. glas, gummi, plast). Mängden av fria
laddningsbärare i materialet begränsar hur
stor strömmen kan bli.

111-2

EPT
\subsubsection{Ledare}
Metaller har god elektrisk ledningsförmåga
och kallas ledare. Bäst ledande är de metaller, vars atomer har det minsta antalet valenselektroner i det yttersta elektronskalet. Koppar-, silver- och guldatomerna har en enda
valenselektron och därmed mycket god ledningsförmåga. Järn, zink och magnesium
har två valenselektroner och därmed något
sämre ledningsförmåga. Ännu sämre ledare
är de s.k. halvledarna med 3 till 5 valenselektroner.

\subsubsection{Isolatorer}
Glas, plast, porslin, pertinax, vissa mineraler
etc. har mycket dålig ledningsförmåga och
kallas isolatorer. Isolatorerna är dåliga ledare därför att de har så många valenselektroner. Det största möjliga antalet är 8
stycken.
l icke ledande material är elektronerna
mycket hårt bundna till sitt valensskal och
därför svåra att flytta. l fasta material är
också positiva laddningar svåra att flytta,
eftersom de är bundna i atomkärnorna. Atomerna är i sin tur bundna i en struktur som
kännetecknar vart och ett material.

\subsubsection{Halvledare}
Till exempel en ren kristall av mineralen
germanium [Ge] eller kisel [Si] leder inte
elektrisk ström. Båda dessa mineral är därför isolatorer. Men om några atomer av ett
främmande material blandas in i deras kristaller, så blir de i någon mån elektriskt ledande- de blir halvledare. Inblandningen är
1 eller 2 främmande atomer per 100 millioner
germanium- eller kiselatomer. Liknande resultat fås med andra material. Beroende på
materialen och i vilka proportioner de blandas fås olika egenskaper.

\subsubsection{N-ledning}
Man talar om N-ledande material respektive
N-ledning- "elektronledning".
Germanium, kisel m.fl. halvledare har
fyra elektroner med "fasta platser" i valensskalet - förutsatt att materialet är helt
rent. Då finns det inga fria elektroner för
laddningstransport.
För att skapa fria elektroner kan det rena
materialet förorenas- dopas - med atomer
av t. ex. arsenik [As] eller antimon [Sb]. Båda
dessa material är 5-värdiga. De har 5 elektroner i valensskalet
4 elektroner är fast bundna medan den 5:e är
löst bunden till atomen. Den 5:e elektronen
kan lossgöras från atomen med yttre
kan, t. ex. värme eller elektrisk spänning och
då skapas en fri elektron. När en spänning
läggs på materialet kommer den fria elektronen att vandra mot den positiva polen. Materialet är N-ledande.

Bild II 1-2 Tankeförsök med kulor i ett rör

\subsubsection{P-ledning - "hålledning"}
När germanium eller kisel dopas med indium
[In] eller gallium [Ga] så blir de P-ledande.
Indium och gallium är 3-värdiga - deras
valensskal innehåller 3 elektroner. Men för
en fast bindning med germanium eller kisel
saknas det en elektron och det uppstår då ett
"hål" - en "bristelektron". Hålet kan fyllas ut
av en elektron från en annan atom. l den
atom som elektronen lämnar bildas det i sin
tur ett hål o.s.v. När en spänning läggs på,
kommer "hålet" att vandra mot den negativa
polen. Materialet är då P-ledande.

\subsection{Elektrisk spänning - Enheten Volt}
Bild II 1-2

I ett tankeförsök med ett rör med kulor i, tänks materialet i röret motsvara
atomstrukturen i en strömledare och kulorna de fria elektronerna. Tänker man sig ett slag
mot en ände av röret så flyttar det sig av den energi som tillförs. På grund av
obundenheten till röret så följer av masströgheten kulorna inte med röret, utan hamnar i dess ena ände.

Att kulorna samlas i ena änden av röret tänks motsvara ett elektronöverskott i ena
änden av en ledare och ett motsvarande underskott i den andra änden.

Man kallar änden med elektronöverskott för minuspol och änden med elektronunderskott för
pluspol. Olika stora elektriska laddningar vid polerna innebär att de sinsemellan har
olika potential. Potentialskillnaden kallas spänning.

Likspänning innebär ett överskott av elektroner och alltid vid samma
anslutningspol.

Växelspänning innebär ett överskott av elektroner, omväxlande vid den ena
anslutningspolen och den andra.

Måttenheten för spänning är Volt [V].
I formler betecknas spänning med
U för effektivvärdet
u för momentanvärdet (ögonblicks-)
û för toppvärdet (amplitud-)

Spänningen över ändpunkterna på en
strömledare är 1 Volt [V], då ledaren
genomflyts av en likström av 1 Ampere
[A] under effektutvecklingen 1 Watt (W].

\subsection{Symboler}

När man ritar scheman för elektriska kretsar, används symboler. Följande symbol visar
ett elektriskt batteri med en enda cell.

Förtydligande kommentarer och skrivtecknen invid symbolen förekommer. Ofta
referar dessa till en komponentlista. Se f.ö. i kapitel 2. Komponenter.

\subsection{Elektrisk ström - Enheten Ampere}

När en sluten strömkrets innehåller en spänningskälla, så kan en laddningsutjämning en
ström- ske genom kretsen. Det innebär att fria elektroner förflyttar sig genom kretsen i
riktning från spänningskällans minuspol till dess pluspol. Vid pluspolen är det nämligen
brist på negativa laddningar och naturen söker alltid en utjämning. Under
utjämningsförloppet är spänningskällan även en strömkälla.

I gaser och elektrolyter (elektriskt ledande vätskor och geler) samt i halvledare består
strömmen av joner (positiva eller negativa laddningar); i metaller däremot av elektroner
(negativa laddningar).

Av tradition anses strömriktningen vara positiv i jonströmmens riktning - den s.k.
tekniska strömriktningen - medan elektronströmmens riktning är den motsatta - den
s.k. fysikaliska strömriktningen.

Måttenheten för ström är Ampere [Aj.
I formler betecknas ström med
I för effektivvärdet,
i för momentanvärdet (ögonblicks-),
i för toppvärdet (amplitud-).

Bild II 1-3 Potential och spänning i en strömkrets

Strömmen är $1 A$, när $6.25 \cdot 10^{18}$ elektroner per sekund flyter genom ett givet
ledartvärsnitt, vilket motsvarar laddningen 1 Coulomb.

\subsection{Strömkrets}

Bild II 1-3
En elektrisk strömkrets består av en eller
flera energikällor och energiförbrukare. Källor kan vara batterier, nätaggregat etc. Förbrukare kan vara lampor, ledningar etc.
Varje energiförbrukare har en resistans
och de elektriska laddningarna "köar" före
förbrukaren. strax efter förbrukaren finns
ingen kö.
Det uppstår en skillnad i laddningsmängd
(en potentialskillnad) mellan varje punkt i en
strömkrets, när det flyter ström. Man talar om
spänningsfall.

\subsection{Strömförlopp}

Likströms- och växelströmsförloppen kan vara sammansatta av ett huvudförlopp och
underordnade förlopp.

Likström kan ha konstant styrka eller den kan variera enligt något förlopp,
men växlar aldrig riktning. Växelström kan variera enligt något
visst förlopp, t.ex. sinusvåg, fyrkantvåg, och växlar ständigt riktning.

$R_1 \cdot l_1 + R_2 \cdot I_2 + \cdots R_n \cdot I_n = U_1 + U_2 + \cdots U_n$

\subsection{Resistans - Enheten Ohm}

Närfria elektronertvingas fram genom atomstrukturen i en ledare, t.ex. glödtråden i en
lampa, så avgår energi i form av värme.
Detta fenomen kallas för resistans (av latinets resistere som betyder att motstå).
Resistansen och därmed förlusterna i en
strömkrets fördelas i förhållande till de ingående materialen och deras dimensionering.

Resistans uttrycks i enheten Ohm och betecknas med den grekiska bokstaven
omega ($\Omega$).
I formlerbetecknas resistansen i en elektrisk krets eller en del av den med R.

Resistansen i en resistor är 1 Q, när en
spänning av 1 V driver en ström av 1 A
genom den resistorn.

\subsection{Ohms lag}
Ohms lag beskriversambandet mellan grundbegreppen ström I [ampere], spänning U
[volt] och resistans R [ohm].
Sambandet gäller både för likspänning och effektiwärdet för växelspänning och
växelström.

I en ledare med resistansen R är strömstyrkan l genom resistansenproportionell
mot den pålagda spänningen U.

$\begin{array}{ccc}U=I \cdot R & I=\frac{U}{R} & R=\frac{U}{I}\end{array}$

\subsection{Kirchhoffs lagar}

Den tyske fysikern G R Kirchhoff (1824-1887) formulerade sina välkända lagar, först
1845 och sedan 1847.

\subsubsection{Kirchhoffs strömlag}

Den algebraiska summan av alla strömmar, som flyter till eller från varje punkt i
en elektrisk krets, är lika med noll.

$I_1 + I_2 + I_3 + \cdots + I_n = 0$

\subsubsection{Kirchhoffs spänningslag}

I varje sluten strömkrets är den algebraiska summan av alla spänningskällor lika
med det totala spänningsfallet i alla resistorer.

Uttryckt på ett annat sätt är algebraiska summan av spänningarna i en strömkrets lika med
noll.

\subsection{Elektrisk effekt - Enheten Watt}

När en ström flyter genom en resistans utvecklas värme. Värme är en form av effekt,
som är högre ju starkare strömmen och högre spänningen är.
Måttenheten voltampere [VA] för elektrisk effekt härleds ur produkten av volt [V]
och ampere [A].
För effekt som alstras av likström används enheten Watt [W] i stället för voltampere [VA]. Vid sidan om grundenheten 1 W används delar och multipler av denna.

$1 volt [U] \cdot 1 ampere [l]= 1 watt [P]$

Effektformeln $P = U \cdot I$ kan skrivas om på
flera sätt. Den gäller i första hand för likström,
men även för växelström, om ström och
spänning inte är fasförskjutna, vilket är fallet
när belastningen är resistiv

$
\begin{array}{lll}
U = R \cdot I & I = \frac{U}{R} & R = \frac{U}{I} \\
P = U \cdot I & P = \frac{U \cdot U }{R} & P = \frac{U^2}{R} \\
&U = \sqrt{P \cdot R} & \\
P = R \cdot I \cdot I & P = R \cdot I^2 & P = R \cdot I^2
\end{array}
$

Med hjälp av dessa formler kan effekten beräknas ur resistans- och strömvärdena
respektive ur resistans- och spänningsvärdena.

\subsection{Elektrisk arbete - Enheten Joule}

Energi finns i olika former, alltid och överallt.
Energi kan varken skapas eller förstöras,
bara omvandlas från en form till en annan.
Formen kan vara mekanisk, kemisk, elektrisk etc.
Arbete är omvandlingsprocessen från
en energiform till en annan.
Arbetsmängden i alla energiformer kan
mätas med samma enhet- Joule [J].

Bild II 1-4 "Formelsnurra" för
Ohms och Joules lagar

1 Joule motsvarar det arbete som utvecklas när ett föremål förflyttas 1 meter
med kraften 1 Newton [N], d. v. s. 1 Newtonmeter [Nm].
Arbetet [W=Work] är mer ju längre tid [s]
en viss effekt [P=Power] utvecklas.

\subsection{Joules lag}

$Arbete = Effekt \cdot tid$

$[W] = [P] \cdot [s]$

Eftersom effekten uttrycks som $P = U \cdot I$
så kan det elektriska arbetet uttryckas som
$W = U \cdot I \cdot t$, vilket också är Joules lag.
Om grundenheterna för volt [U], ampere
[l] och sekund [s] sätts in i formeln fås en
måttenhet, uttryckt som voltamperesekunder
[VAs] eller wattsekunder [W s] eller joule [J].
Måttenheten för elektriskt arbete är 1
Joule per sekund, som vanligen kallas 1
wattsekund [1 Ws] eller helt enkelt watt [W].
Vid sidan avgrundenheten används multipler
av denna.
Exempel:
$
\begin{array}{lll}
1 kilowattsekund & = 1 kWs & = 1 000 Ws \\
1 wattimme & = 1 Wh & = 3600000 Ws \\
 & & = 3.6 · 10^6 Ws \\
1 kilowattimme & = 1 kWh & = 1 000 Wh \\
 & & = 3.6 · 10^9 Ws
\end{array}
$

\subsection{Formelsnurran}

Bild 111-4

Så här finner man rätt formel i "snurran":
Välj ett segment med önskad storhet I, U, R
eller P som det första ledet i formeln. Inom
valt segment finns tre alternativ för det andra
ledet i formeln. Välj det alternativ som innehåller två kända storheter.

Exempel:

\subsubsection{Ohm's lag}

R söks, U och I är kända;
Om $U = 230 V$ och $I = 2 A$, så blir

$R=\frac{U}{I}=\frac{230}{2}=115 \Omega$

\subsubsection{Joule's lag}

P söks, U och I är kända;

Om $U = 230 V$ och $I = 2 A$, så blir

$P = U \cdot I = 230 \cdot 2 = 460 W$

\subsection{Amperetimmar (Ah) och batterikapacitet}

Det finns flera sätt att lagra energi. Ett sätt är att göra det i kemisk form i speciella
celler, där man kan ta ut energin i elektrisk form.

Det finns celler som kan laddas upp och laddas ur upprepade gånger, s. k. ackumulatorer.
Det finns också sådana celler som endast kan användas en gång och som inte
kan laddas upp igen, s. k. primärceller.

Energi i form av en elektrisk laddning kan även lagras i en kondensator. Energin kan
då lagras och tas ut utan omvandling.


Kapaciteten i en elektrisk cell uttrycks som produkten av den ström [A] som
cellen avger och under den tid [s, h] detta kan ske.
Uttryckt med tidsenheten timmar blir då kapaciteten Ah.


Den kapacitet som anges i en cells produktdata är den nominella. Denna kapacitet
gäller endast under vissa normerade förhållanden såsom celltemperatur, strömstyrka
och urladdningstid.

Den praktiska kapaciteten i en cell begränsas av användningen. En elektrisk cell
avger sålunda regelmässigt mindre energimängd, desto högre urladdningsströmmen
är. Kapaciteten i en elektrisk cell skiljer sig i det avseendet från den i t.ex. en
oljetank, där man kan ta ut lika mycket energimänd som man häller i och oberoende av hur
fort man gör det.


Elektriska celler kan samlas till s.k. batterier, varvid cellerna oftast seriekopplas.
Batteriets polspänning är då summan av cellernas polspänningar.

Hur stort arbete ett batteri avger, beror då såväl på hela batteriets polspänning som på
de enskilda cellernas kapacitet.
Exempel.
Ett batteri med polspänningen $12 V$ och cellkapaciteten $100 Ah$ kan nominellt avge
$P = U \cdot I = 12 \cdot 100 = 1200 VAh = 1.2kWh$.

Hur länge batteriet "räcker" per laddning beror som sagt bl.a. på vilken strömstyrka
man tar ut. Tar man ut $1 A$ ur $100 Ah$-cellen här ovan, så blir urladdningstiden
nominellt $t = 100 Ah/1 A = 100 h$.

\cleardoublepage

\section{Elektriska kraftkällor}

\subsection{Elektromotorisk kraft - EMK}

Det som driver ström genom en elektrisk strömkrets är kretsens elektromotoriska kraft
(EMK).
Måttenheten för EMK är Volt [V]. EMK är summan av de potentialökningar som uppstår i
kretsen.

De vanligaste slagen av emk är
\begin{itemize}
\item elektromagnetisk emk som uppkommer i
strömledare i magnetfält som varierar
(ex. lindningarna i en roterande generator),
\item elektrokemisk emk som uppkommer i
beröringsytan mellan en metallisk ledare
och en elektrolyt (ex. battericell),
\item elektrostatisk emk, t. ex. i kondensatorer,
\item kontaktemk i beröringsytan mellan metaller med olika termoelektrisk potential
eller mellan metall och luftens syre (ex.
korrosion mellan metaller),
\item termoemk som uppkommer i en strömkrets där två sammanlödda metaller med
olika temperatur ingår (ex. termokors för
strömmätning).
\end{itemize}

\subsection{Polspänning}

Den spänning, som kan mätas mellan kretsens anslutningspoler då kretsen är öppen.

\subsection{Inre resistans}

Liksom att komponenterna i strömkretsen har en viss resistans, så har också en strömkälla
en inre resistans. Den inre resistansen i en strömkälla ingår i kretsens totala resistans.

\subsection{Kortslutningsström}

Om man på kortaste väg förbinder strömkällans anslutningspoler så blir kretsen totala
resistans lika med källans inre resistans.

Den kortslutningsström som då uppstår, begränsas enbart av strömkällans polspänning och
inre resistans.

Eftersom den inre resistansen oftast är mycket liten blir kortslutningsströmmen
motsvarande hög.

\subsection{Serie- och parallellkopplade kraftkällor}

\subsubsection{Seriekopplade kraftkällor}

För att uppnå en högre total spänning (emk)
kan flera kraftkällor (delspänningar) kopplas
i en slinga efter varandra. Detta kallas seriekoppling.

Seriekopplade delspänningarverkar med
eller mot varandra, beroende på deras
inbördes polariteter.

Den totala spänningen över kopplingen
är summan av de ingående de/spänningarna, med hänsyn taget till deras
polariteter.

\subsubsection{Parallellkopplade kraftkällor}

För att erhålla högre ström, kan flera svagare kraftkällor parallellkopplas. Vid
parallellkoppling erhålls däremot inte högre spänning.

Vid parallellkoppling av kraftkällor måste
deras polaritet vara lika.

För minsta utjämningsström mellan parallellkopplade kraftkällor bör även deras
polspänning och inre resistans vara så lika
som möjligt.

\cleardoublepage

\section{Elektriskt fält}

\subsection{Potential}

Potentialskillnaden - spänningen - mellan olika laddade kroppar, skapar krafter mellan
varandra samt mellan dem och deras omgivning. Detta fenomen kallas elektriskt kraftfält och är orsaken till att elektriskt laddade kroppar kan komma i rörelse.

\subsection{Elektrisk laddning}

Elektriska laddningar är grunden för elektricitetsläran. Varje proton i atomkärnan är
bärare av en positiv laddning. Neutronerna i atomkärnan är elektriskt neutrala. Antalet
protoner i kärnan bestämmer därför ensamt kärnans totala positiva laddning, kallat för
kärnladdningstalet Elektronerna som kretsar omkring atomkärnan är bärare av var sin
negativa laddning.

Elementarladdningen [ e ] är den laddning som finns i en elektron och har länge
ansetts vara den minsta möjliga laddningen. Nutida elektronfysik konstaterar ännu
mindre enheter, men detgår vi inte in på här.

Antalet protoner och elektroner i en atom är lika och elektronernas samlade negativa
laddning blir då lika stor som protonernas samlade positiva laddning. När laddningar med
olika polaritet är lika stora väger de ut varandra och blir elektriskt neutrala till sin
omgivning.

Måttenheten för elektrisk laddning är Coulomb [C].

Laddningsmängden 1 Coulomb motsvarar 6.25 trillioner ($6.25\cdot10^{18} $) elementarladdningar.

Sambandet mellan laddning och ström är

$Q = I \cdot t$

Laddning [Q] är ström [l] under tiden [t]

$1 C= 1 A ·1 s= 1 amperesekund [1 As]$

$1 Coulomb = 1 Ampere·1 sekund$

\subsection{Kraftfält omkring elektriska laddningar}

Bild II 1-5
Mellan elektriska laddningar bildas krafter.

\begin{itemize}
\item Varje laddning är omgiven av ett elektriskt kraftfält.
\item Mellan positiva (+) elektriska laddningar
och (-) negativa laddningar bildas krafter.
\item Fältkrafternas styrka och riktning symboliseras som linjer mellan positiva och
negativa laddningar, där styrkan är densamma utmed respektive linje.
\end{itemize}

(även 1.1)

Kroppar med olika slags laddningar dras
till varandra

Kroppar med lika slags laddningar stöter bort varandra

Oladdade kroppar påverkas inte och ger ingen kraftverkan.

\subsection{Elektrisk fältstyrka}

I en trådformad ledare, som det flyter likström igenom, fördelas strömmen lika över
tvärsnittet. Om ledaren i stället är ett tunt plan, så blir strömfördelningen annorlunda.
Bilden visar ett plan med två elektroder, som anslutits till en spänningskälla. Utmed
sträckan mellan elektroderna fördelas strömmen över planet så som strömlinjerna på bilden.
Fördelningen beror på elektrodernas utformning och polaritet. Strömtätheten är inte lika
över hela planet, eftersom planet kan ses som många parallellkopplade resistorervars
resistanser ökar med tilltagande strömlinjelängd.

Strömtätheten i planet är större där resistansen mellan elektroderna är liten. Närmast
elektroderna där alla strömlinjer samlas är strömtätheten extremt hög. Där strömtätheten
är som störst finns den största potentialskillnaden (spänningen) per längdenhet
strömlinje. Man kan mäta potentialerna i planet. Spänningen mellan två punkter utmed en
tänkt strömlinje är därvid proportionell med linjens längd mellan punkterna. Halva
spänningen finner man mitt emellan punkterna.

Bild II 1-5 Elektriska kraftfält

Elektriska fält är upplagrad energi. Fältstyrkan kan bli så hög, att det blir en
urladdning mellan polerna. Korona från ändarna av en antenn är ett annat tecken på hög
fältstyrka. För att försvåra urladdning kan man öka elektrodytan, t. ex. göra den
klotformad. Omvänt kan man medverka till urladdning genom att minska elektrodytan.
Ett exempel är åskledarens spets.

Bild II 1-6
I diagrammet U = f (l) visas spänningarna utmed "mittströmslinjen" l genom plus- och
minuspolerna. Kurvutseendet är typisk även för omkring liggande linjer, oavsett längd.

Bilden framställer en ledare som ett idealt plan, medan den i praktiken är en volym.
För att efterlikna en volym föreställer vi oss att bilden roterar omkring
mittströmslinjen, med fältlinjerna oförändrade. Även om resistansen i den rotationskropp
som uppstår är så hög att ingen ström flyter, så är spänningsbilden fortfarande densamma.

Spänningsbilden gäller även för isolerande fasta material, gaser och vakuum.
Det finns alltså spänning mellan olika punkter även i "friska luften". Denna
spänningfältstyrka- kan mätas med särskilda instrument, s. k. fältstyrkemätare.

Av brantheten på spänningskurvan i bilden framgår vilken delspänningen är per dellängd av
en spänningslinje. Kvoten av delspänning och avståndet mellan mätpunkterna kallar man för
elektrisk fältstyrka.

I formler betecknas elektrisk fältstyrka med bokstaven E.
Elektrisk fältstyrka mäts i volt per meter.

$
\begin{array}{cc}
E=\frac{\Delta U}{\Delta l} & \frac{[volt]}{[meter]}
\end{array}
$

Bild II 1-6 Elektrisk fältstyrka

\subsection{Skärmning av elektriska fält}

I grunden finns det två slags fält, det elektriska och det magnetiska. Dessutom finns det
även elektromagnetiska fält, som är sammansatt av båda dessa. Fält kan vara statiska
eller dynamiska, varav här avses dynamiska. Ett dynamiskt elektriskt fält genererar ett
magnetiskt fält. Omvänt generar ett dynamiskt magnetiskt fält ett dynamiskt elektriskt
fält. Denna växelverkan gör att fälten kan hållas igång av varandra med tillskott av
yttre energi.

Fält i rörelse alstrar elektromagnetisk strålning, som påverkar omgivningen. När
påverkan inte är önskvärd måste fältet skärmas av. Ett sätt att skärma av ett elektriskt
fält är en metallisk kapsling som anslutits till apparatens jordreferens. Skärmen behöver
inte vara tät, men utförd så att all magnetiskt inducerad ström i den bryts. (Jfr 1.4)

\cleardoublepage

\section{Magnetiskt fält}

\subsection{Magnetism}

Enligt den romerske författaren Plinius lär, vid tiden ungefär 160 år f. K. herden Magnes
en dag ha känt hur järnstiften i sandalerna häftade vid en viss sorts sten. Det kunde ha
varit svart järnmalm, som grekerna i äldsta tider benämnde Lithos herakleia efter staden
Herakleia i Lydien, där sådan malm förekommer. Staden fick sedermera namnet Magnesia och
man kan tänka sig att stenen kom att kallas Magnetes. En hel mineralgrupp med liknande
egenskaper, såsom järn, nickel m. fl. kallas magnetiska.

Magnetism uppstår av elektriska laddningar i rörelse. Elektronernas rörelser i en atom
skapar nämligen magnetfält. Det gör att atomerna var för sig fungerar som en magnetisk
dipol - en magnet. I de flesta material är atomerna orienterade så att deras magnetiska
kraftertar ut varandra. Materialet som helhet är då omagnetiskt och utövar inga yttre
krafter. Men vid påverkan från ett yttre magnetfält kan dipolerna (atomerna) i ett
material orienteras i samma riktning och deras magnetfält kommer då att
samverka. Hela materialet blir då magnetiskt. När det yttre magnetfältet avlägsnas,
kvarstår orienteringen endast delvis- magnetisk remanens. l terrornagnetiska legeringar
kvarstår en större del av orienteringen, även om påverkan från det yttre magnetfältet har
upphört. Materialet är då permanentmagnetiskt

\subsection{Kraftfält i och omkring magneter}

Bild 111-7

Varje magnet omges av ett magnetiskt kraftfält Magnetfältets fördelning, styrka och
riktningar beskrivs som kraftlinjer med slutna kretslopp.

Utanför magneten går kraftlinjerna från nord- till sydpol och inne i magneten i motsatt
riktning. Kraftriktningen i varje punkt av fältet är den som nordändan på en kompassnål
skulle peka åt. Om man hänger upp en magnet i en tråd, så kommer den att inta
samma riktning som jordens magnetfält.

Poler med samma polaritet stöter bort varandra (repellerar).

Poler med olika polaritet dras till varandra (attraherar).

\subsection{Magnetiska fält omkring strömbanor}

Bild II 1-8

Omkring varje ledare, som det flyter en elektrisk ström igenom, alstras det ett
magnetiskt kraftfält.

Magnetiska kraftlinjerna fördelar sig koncentriskt omkring en rak ledare och vinkelrätt
mot denna.

Mellan ändarna av en ledare med bågformad utsträckning bildas kraftlinjer som verkar med
varandra.

En strömgenomfluten cylindrisk spole induktor- uppvisar samma magnetiska fältbild som en stavformad permanentmagnet

\subsection{Bestämma magnetiska fältriktningen}

Magnetfältets riktning omkring en ledare kan enkelt bestämmas med vänsterhandsregeln

När en ledare fattas med vänster hand och med tummen i strömmens riktning, så
kommer fingrarna att peka i fältriktningen.

När en ledare formas som en spole och en elektrisk ström flyter genom den, kommer
magnetfältet att ha ett utseende som liknar det omkring en permanentmagnet

En sådan spole kallas elektromagnet.

Magnetfältets riktning i en spole kan också bestämmas med vänsterhandsregeln.
När en spole fattas med vänster hand och med fingrarna i strömmens riktning, så
kommer den utsträckta tummen att peka mot spolens nordpol.

Fälten omkring alla slags magneter, såväl permanentmagnetiska som e!~ktromag­
netiska, återverkar på varandra. Aven enkla
elektriska ledare är elektromagneter.

Bild II 1-7 Kraftfält omkring magneter

Bild II 1-8 Magnetiska fält omkring strömledare

\subsection{Exempel på elektromagneter}

Bild II 1-9

\subsubsection{Elektromagnet}
Det bildas ett magnetfält genom en spole så länge som det flyter ström genom den. En
järnkärna i spolen koncentrerar fältet p.g.a. den större magnetiska ledningsförmågan.

Elektromagneter används för att sätta magnetiska material i rörelse eller hålla fast
dem.

\subsubsection{Elektrisk ringklocka}
Anordningen består av en elektromagnet och en järnplatta på en fjäder. På plattan
sitter en självbrytande kontakt samt en kläpp som kan slå på en klocka.

Kontakten åstadkommer en växelvis brytning och slutning av strömmen genom
elektromagneten. Armaturen med kläppen kommer då i svängning och slår på klockan.

\subsubsection{Telefon}
I en enkel telefon finns bl.a. en mikrofon, ett batteri och en hörtelefon.

Särskilt i äldre telefoner består mikrofonen av en kolkornskammare med ett membran.
Trycksvariationer (ljud) får membranet att vibrera, varvid resistansen genom kolkornen
varierar i motsvarande grad. Därmed varierar talströmmen genom mikrofonen.

Hörtelefonen består av en elektromagnet och ett membran av mjukjärn. Variationer i
talströmmen genom mikrofonen passerar även hörtelefonen får dess magnetfält att variera.
Hörtelefonens membran alstrar då trycksvariationer, d.v.s. ljud.

\subsubsection{Elektromagnetiskt relä}
Reläet består av en elektromagnet, en järnplatta (ankare) på en fjäder och en elektrisk
kontakt. Med en svag ström/låg spänning genom spolen i manöverkretsen, så kan
man med reläets arbetskontakt styra starkare ström/högre spänning i huvudkretsen.

Bild II 1-9 Tillämpade elektromagneter

\subsection{Magnetisk fältstyrka}

Som magnetisk fältstyrka Henry $[H]$ förstår man flödet per meter fältlinje, d.v.s.

$H=\frac{\Phi}{l} = \frac{I \cdot N}{l}$

$H [A/m]$ $I [A]$ $N [varvtal]$ $l [fältlinjelängd]$

Magnetisk fältstyrka uttrycks således som Ampere per meter flödesväg.

\subsection{Magnetisk flödestäthet}

Den magnetiska flödestätheten mäts i enheten Tesla $[T]$ (förut Gauss).

Formeltecknet är $B$.
Formeln är $B = \mu_0 \cdot HH$

Flödestäthet $B [Vs/m^2]$ Fältstyrka $H [A/m]$

$\mu_0$ är permeabilitetstalet (fältkonstanten) för den magnetiska ledningsförmåga för
luft och omagnetiska material.

För järn eller annat magnetiskt ledande material tillkommer permeabilitetstalet $\mu_r$.
Det anger hur många gånger bättre än luft etc., som materialet det leder ett magnetisk
flöde.

Formeln är $B = \mu_0 \cdot \mu_r \cdot H$

\subsection{Magnetiskt flöde}

Det magnetiska flödet är produkten avflödestätheten $B$ och tvärsnittsytan $A$ av flödesvägen, således

$\Phi = B \cdot A$
$\Phi [Weber eller Vs]$ $B [T eller Tesla]$ $A [m^2]$

\subsection{Skärmning av magnetiska fält}

I grunden finns det två slags fält, det elektriska och det magnetiska. Det finns även
elektromagnetiska fält, som är sammansatt av båda dessa. Fält kan vara permanenta eller
rörliga, varav här avses de rörliga. Ett rörligt magnetiskt fält genererar ett elektriskt
fält.
Omvänt generar ett rörligt elektriskt fält ett rörligt magnetiskt fält. Denna växelverkan
gör att fälten kan hållas igång med tillförsel av yttre energi.

Fält i rörelse alstrar elektromagnetisk strålning, som påverkarfunktioner i omgivningen.
När påverkan inte är önskvärd, måste fältet skärmas av. Ett sätt att skärma magnetiska
fält är en metallisk kapsling. Kapslingenskall vara tät och bilda en sluten magnetisk
krets. Kapslingen skall vara utförd i ett material som är en god ledare av magnetiskt
flöde.
(Jämför 1.3)

\cleardoublepage

\section{Elektromagnetiskt fält}

\subsection{Vågutbredning}

En tillståndsändring i ett medium innebär att energi tillförs eller tas bort. Om detta
sker växelvis, så uppstår förlopp såsom pendling, svängning, vågbildning etc.
Eftersom naturen söker jämvikt, så brederförloppet ut sig genom mediet efter någon modell.

Energi kan inta olika tillstånd. I en pendel växlar energin mellan lägesenergi och
rörelseenergi. Vågor på en vätskeyta liksom fjädring i fasta material är exempel på
detta. Det kan även innebära trycksvängningar i gaser o.s.v.

I detta avsnitt behandlas elektromagnetiska fält. Sådana uppstår av svängningar i
elektriska och magnetiska fält. För att förklara pendling och utbredning används här
modeller.

\subsection{Utbredningsmodeller}

\subsubsection{Vågutbredning längs en linje}

Bild II i-10

När änden av en tråd sätts i pendling med en frekvens f, så kommer till sist hela tråden i
svängning med den frekvensen. Den pendling, som först skapades, vandrar längs tråden med
utbredningshastigheten v. Våglängden är Å (lambda), som är avståndet mellan två
närliggande punkter med samma svängningsläge och svängningsriktning.

Bild II 1-10 Vågor längs en linje

Bild II 1-11 Vågutbredning på en yta

\subsubsection{Vågutbredning på en yta}

Bild II 1-11

När ett föremål släpps genom en vätskeyta, så bildas vågor som breder ut sig som cirklar
i varandra (koncentriska).

De punkter på vågen, som för ögonblicket har samma svängningsläge, och är lika långt från
energikällan, kallas för vågfront

Sambandet mellan utbredningshastighet $v$, våglängd $\lambda$ och frekvens $f$ är

$v = \lambda \cdot f$ $v [m/s]$ $\lambda [m]$ $f [Hz=1/s]$

Bild If 1-12 Vågutbredning i rummet

Exempel: När våglängden $\lambda = 2 m$ och antalet svängningar per sekund $f = 10 Hz$,
så breder vågen ut sig med hastigheten $v = 20 m/s$.

\subsubsection{Vågutbredning i rummet}

Bildll1-12

Ljud är energi i form av tryckvågor i luften. När en mekanisk kropp sätts i svängning
(stämgaffel, dricksglas etc), överförs svängningarna till den omgivande luftmassan som
börjar att svänga med. I luftmassan bildas det omväxlande över- och undertryckszoner, som
breder ut sig åt alla håll. De mekaniska svängningarna i ljudkällan omvandlas alltså till
tryckvågor.

Det mänskliga örat uppfattar tryckvågor inom frekvensområdet c:a $15-18000 Hz$ som ljud.
Dessa vågor kallas ljudvågor. Utbredningshastigheten för ljudvågor är $v = c:a 340 m/s$ vid 
15$\circ$C och normalt lufttryck.

\subsection{Elektromagnetiska fält}

Bild 111-13

I detta avsnitt görs i huvudsak endast jämförelse mellan ljusvågor och radiovågor, vilka
båda är elektromagnetisk strålning. Hur ett elektromagntiskt fält frigörs från en ledare,
framgår av kapitel 7 Vågutbredning.

Elektromagnetiska fält är energi, som är sammansatt av mycket snabbt svängande elektriska
och magnetiska fält. När elektrisk ström genom en ledare ändras i styrka, så bildas ett
magnetfält omkring ledaren. Detta magnetfält alstrar en elektromotorisk kraft (EMK), som
är motriktad den som driver fram strömmen. Magnetfältet motverkar således strömändringen.
På liknande sätt alstrar en ändring av magnetfältet omkring ledaren en EMK i form av ett
elektriskt fält. Detta driver en motriktad ström och därmed ett motverkande magnetiskt
fält.

Både det elektriska och det magnetiska fältet har således alstrats av ändringar i det
andra och existerar därför bara tillsammans.

De båda fälten kombineras till ett elektromagnetiskt fält, som har egenskapen att kunna
stråla (breda ut sig) i alla tre dimensioner. Beroende på frekvensen har
elektromagnetiska fält olika egenskaper och användning, vilket framgår av bilden.


\begin{center}
\begin{tabular}{|rl|rl|l|}
\hline
\multicolumn{2}{|c|}{\multirow{2}{*}{Frekvens}} & \multicolumn{2}{|c|}{\multirow{2}{*}{Våglängd}} & \multicolumn{1}{|c|}{Egenskaper/} \\
 & & & & \multicolumn{1}{|c|}{användning} \\ \hline
300 & Hz  & 100 & mil & \\
  1 & kHz & 300 & km & ULF \\ \cline{5-5}
  3 & kHz & 100 & km & \\
 10 & kHz &  30 & km & VLF \\ \cline{5-5}
 30 & kHz &  10 & km & \\
100 & kHz &   3 & km & LF \\ \cline{5-5}
300 & kHz &   1 & km & \\
  1 & MHz & 300 & m & MF \\ \cline{5-5}
  3 & MHz & 100 & m & \\
 10 & MHz &  30 & m & HF \\ \cline{5-5}
 30 & MHz &  10 & m & \\
100 & MHz &   3 & m & VHF \\ \cline{5-5}
300 & MHz &   1 & m & \\
  1 & GHz & 300 & mm & UHF \\ \cline{5-5}
  3 & GHz & 100 & mm & \\
 10 & GHz &  30 & mm & SHF \\ \cline{5-5}
 30 & GHz &  10 & mm & \\
100 & GHz &   3 & mm & EHF\\ \cline{5-5}
300 & GHz &   1 & mm & \\\
  1 & THz & 300 & $\mu$m & Infrarött \\
  3 & Thz & 100 & $\mu$m & ljus \\
 10 & THz &  30 & $\mu$m & (värme- \\
 30 & THz &  10 & $\mu$m & strålning) \\
100 & THz &   3 & $\mu$m & \\ \cline{5-5}
300 & THz &   1 & $\mu$m & Synligt ljus \\ \cline{5-5}
  1 & PHz & 300 & nm & \\
  3 & PHz & 100 & nm & Ultraviolett \\
 10 & PHz &  30 & nm & ljus \\ \cline{5-5}
 30 & PHz &  10 & nm & \\
100 & PHz &   3 & nm & Rönt-\\
300 & PHz &   1 & nm & gen-\\
  1 & EHz & 300 & pm & strålning\\ \cline{5-5}
  3 & EHz & 100 & pm & \\
 10 & EHz &  30 & pm & Gamma-\\
 30 & EHz &  10 & pm & strål-\\
100 & EHz &   3 & pm & ning\\
300 & EHz &   1 & pm & \\
\hline
\end{tabular}
\end{center}

Bild II 1-13 Elektromagnetiskt spektrum

\subsubsection{Ljusvågor}

Ögat uppfattar elektromagnetisk strålning bara inom ett visst frekvensområde som ljus.
Ljusets utbredningshastig het beror av vilket material, som det passerar igenom. I
vakuum är hastigheten störst, c= 299793077 m/s (= ca $3 \cdot 10^8 m/s$). I tätare ämnen är
hastigheten lägre, t. ex. i glas ca 200 000 000 m/s. Det för människan synliga ljuset har
våglängder mellan $7.7 \cdot 10^{-7}$ och $3.9 \cdot 10^{-7} m$, motsvarande 7.7 till 3.9
tiotusendels mm.

Sambandet mellan ljusets utbredningshastighet c i vakuum, frekvensen f och våglängden A är

$c = \lambda \cdot f = 3 \cdot 10^3$

$c [m/s]$ $f [Hz  i/s]$ $A [m]$

\subsubsection{Radiovågor}

Även radiovågor är elektromagnetisk strålning, men inom ett lägre frekvensområde än
det för ljus. Men utbredningshastigheten för radiovågor genom olika material följer ändå
samma lagar som de för t. ex. ljusets utbredning.

Radiovågor anses omfatta ett frekvensområde från ca 1O kHz ($\lambda = 30 km$) till 300
GHz ($\lambda = 1 mm$).

Rundradio tilldelas frekvenser i intervallet 100 kHz till 1 000 MHz. Amatörradio tilldelas
ett antal frekvensområden i intervallet 1.8 MHz till 250 GHz.

Att märka är att elektromagnetiska fält, som sagts ovan, förekommer så långt ner i
frekvens som ett fåtal kHz. Detta skall självklart inte förväxlas med ljudtryck med samma
frekvens.

\subsubsection{Egenskaper hos elektromagnetiska vågor}

Elektromagnetiska vågor med högre frekvens än radiovågor uppfattas som värmestrålning,
vågor med ännu högre frekven som ljus etc., men fortfarande är huvudegenskaperna samma.
Som exempel kan nämnas polariserade vågor. Dessutom kan
man finna motsvarigheten till sådana egenskaper som interferens, överlagring, även i
andra vågtyper, t.ex. i ljud.

\subsection{Vågpolarisation}

Bild II 1-14

\subsubsection{Vågor längs en linje (tråd el. dyl.)}
En vågrörelse i ett plan kallas linjärt polariserad. Om änden på en horisontell tråd sätts
i rörelse uppåt-nedåt, uppstår på tråden en linjärt polariserad vågrörelse i vertikalplanet
-vertikal polarisering.
Om tråden sätts i rörelse höger-vänster kommer dess svängning att vara horisontellt
polariserad.
Om tråden sätts i svängning i ett plan och detta plan ständigt vrider sig, kommer även
vågrörelsen utmed tråden att vrida sig. En vågrörelse, vars polarisering vrider sig
roterar - kallas för cirkulärt polariserad. Vridning mot- respektive medurs kallas för
vänster- respektive högervriden polarisering.

Bild II 1-14 Polarisation av elektromagnetiska vågor

\subsubsection{Elektromagnetiska vågor}

De magnetiska och elektriska fälten omkring en ledare är vinkelrätt orienterade mot
varandra. Det elektromagnetiska fält, som de bildar tillsammans, bildar en vågfront som är
vinkelrätt orienterad mot dem.

Polariseringsriktningen för en elektromagnetisk våg definieras som den riktning dess
elektriska fält har.
Vertikalt elektriskt fält- vertikal polarisering.
Horisontellt elektriskt fält - horisontell polarisering.

\subsubsection{Ljusvågor}

Ljus är elektromagnetiska vågor. När dagsljus, som f.ö. är opolariserat, belyser ett
polariseringsfilter, så passerar endast de vågkomposanter genom filtret, som har samma
polarisering som filtret.

När det polariserade ljuset därefter sänds mot ett efterföljande filter, så passerar ljuset
genom det filtret endast när det har samma polarisering som ljuset. När de båda filtren är
vridna 90$\circ$ i förhållande till varandra, passerar inget ljus alls.

\subsubsection{Radiovågor}
Radiovågor är elektromagnetiska vågor inom
det frekvensområde som lämpar sig för radiokommunikation.

Beroende på sändarantennens utformning så avger den vågor med en polarisation. På samma
sätt är en mottagarantenn mest mottaglig för vågor med en viss polarisation.
Överföringsförlusterna blir lägst mellan antenner med samma polarisation.

I det högre frekvensområdet för radio (VHF, UHF, SHF) är polariseringsvridning under
överföringen mindre vanlig. Genom att utforma antennerna med horisontell, vertikal eller
cirkulär (höger- alternativt vänstervriden) polarisation, så fås överföringsegenskaper för
olika syften.

Cirkulärt polariserade antenner ger lägst överföringsförluster när polariseringsriktningen
är lika i sändar-och mottagarantennen.

I det lägre frekvensområdet för radio (HF och lägre) utnyttjas oftast rymdvågsutbredning.
Eftersom de utsända vågorna då reflekteras mot jonosfärskikt, uppstår
polariseringsvridningar som inte kan förutses. Då är det en fördel att kunna växla mellan
antenner med olika polarisation.

\subsection{Väg interferens}

Bild II 1-15

När vågor från olika energikällor blandas med varandra (överlagras), så kommer de att
antingen samverka eller motverka. Beroende av det tidsmässiga läget mellan vågorna och
deras amplituder, så blir resultatet en förstärkning eller en försvagning. Om har samma
frekvens och lika stora, motriktade amplituder, så uppstår en utsläckning, vilket kallas
fädning (eng. fading).

Denna vågmekanism är liknande i gaser (luft), vätskor, elektromagnetiska fält etc. Ett
försök kan göras med en stämgaffel som man slår an och håller intill örat. När man
vrider stämgaffeln runt sin längdaxel, så kommer avståndet mellan vart och ett av
gaffelbenen och örat att variera. Då uppstår en växelvis med- och motverkan mellan tonerna
från gaffelbenen och därmed varierande tonstyrka.

Detta fenomen utnyttjas bl.a. i antenner för riktad sändning respektive mottagning av
radiovågor.

Bild II 1-15 Våginterferens

\cleardoublepage

\section{Sinusformade signaler}

Bild II 1-16 Alstring av en sinusformad signal

I detta avsnitt behandlas några grundbegrepp inom växelströms/äran. Förloppen
framställs med vektor- och linjediagram.

För närmare beskrivning används sådana begrepp som momentanvärde, toppvärde, topp- till
toppvärde, effektiwärde, fasläge, fasförskjutning och båghastighet.

\subsection{Momentanvärde}

Momentanvärdet är storheten på en spänning $u$, en ström i etc. vid en viss tidpunkt $t$.
(Storheter som ändrar sig som en funktion av tiden kännetecknas ofta med gemena bokstäver.)

Bilden visar en sinusformad växelspänning med frekvensen $50 Hz$. Spänningen $u$ är $+230 V$ vid tidpunkten 2.5 millisekunder efter en positiv nollgenomgång. Efter totalt $5 ms$
uppnås toppvärdet $u$ d.v.s. $+325 V$. Efter totalt $1O ms$ sker en neagativ
nollgenomgång. Efter totalt $12.5 ms$ är spänningen $-u$, d.v.s. $-230 V$ o.s.v.

\subsection{Toppvärde eller amplitud}

Toppvärdet u är det högsta värdet över eller under noll. På bilden är de högsta
värdena $+325 V$ och $-325 V$.

\subsection{Topp-till-toppvärde}

Topp-till-toppvärdetuss är summan av toppvärdena över och under noll. På bilden är
detta värde 650 V.

\subsection{Effektivvärde}

Effektivvärdet av en växelspänning $u$ är det värde, som medför samma effektutveckling
som en likspänning $U$.

För ett sinusformat förlopp gäller följande samband mellan toppvärdet och effektivvärdet
(det s.k. kvadratiska medelvärdet), vilket motsvarar amplituden vid vinklarna 45,
135, 225 och 270$\circ$.

$U=\frac{\hat{u}}{\sqrt{2}}$ $I=\frac{\hat{i}}{\sqrt{2}}$ ($\sqrt{2} = 1.414$)

\subsection{Fasläge}

Fasläget är när inom en period, som ett givet momentanvärde uppträder. Tidpunkten för
varje momentanvärde motsvarar en andel av 360$\circ$ elektriska grader. T.ex. uppnås
värdet volt vid 0$\circ$, 180$\circ$ och 360$\circ$ (= 0$\circ$).

\subsection{Bågmått}

I beräkningar av växelströmskretsar används ofta inte vinkelmått för fasläget (gradtal)
utan i stället begreppet bågmått.

I en s.k. enhetskrets med radien $r = 1$ motsvaras vinkeln 360$\circ$ av en båge med
längden $2 \cdot \pi \cdot r= 2 \cdot \pi \cdot 1 = 2 \pi =$ omkretsen
Vid $f$ perioder per sekund blir båglängden $= 2\pi f$. Denna storhet kallas båghastigheten
och betecknas med ro (uttalas omega).
$\omega= 2\pi f$ $[1/s]$

\subsection{Period}

En period har passerat, när en storhet (spänning, ström o.s.v.) återtagit samma tillstånd
eller värde efter att ha gjort en fullständig växling, t.ex. en hel pendelrörelse eller ett
helt varv vid rotation.

\subsection{Periodtid T}

Periotid T är den tid som åtgår för att strömmen ellerspänningen skall genomlöpa
en period. Periodtiden är det inverterade värdet av frekvensen.

Måttenheten för periodtid är sekund [sj

Periodtid

$(T) = \frac{1}{f}$

T [s] f[Hz] eller
T [ms] f [kHz] eller
T [ms] f [MHz]

Exempel:

\begin{center}
\begin{tabular}{lll}
$T_1=\frac{1}{10}$ s & = 0.100 s & = 100 ms (f = 10 Hz)\\
$T_2=\frac{1}{50}$ s & = 0.020 s & = 20 ms (f = 50 Hz)\\
$T_3=\frac{1}{1000}$ s & = 0.001 s & = 1 ms (f = 1 kHz)\\
$T_4=\frac{1}{1000000}$ s & = 0.000001 s & = 1 $\mu$s (f = 1 MHz)\\
\end{tabular}
\end{center}

\subsection{Frekvens}

Frekvens är antalet perioder per tidsenhet.

Följande begrepp demonstreras med hjälp av pendeln:

Period = en fullständig fram- och tillbakasvängning i ett system, t.ex. pendelns väg
mellan punkterna 2- 3- 2- 1 - 2- 3- o.s.v.

Periodtid T = tidsåtgången för en fullständig svängning.

Amplitud A = den största avvikelsen från viloläget.

Frekvens f = antal svängningar/tidsenhet.

Sambandet mellan frekvensen f och periodtiden T är

$f=\frac{1}{T}$ t. ex.

$5 [H z] = \frac{1}{5} [sekunder]$

\subsection{Enheten Hertz}

Måttenheten för frekvens är Hertz [Hz].
l formler betecknas frekvensen med f.

\begin{center}
\begin{tabular}{ll}
1 Hz      & = 1 period per sekund (p/s) \\
1O Hz     & = 1O perioder per sekund \\
50 Hz     & = 50 perioder per sekund \\
1 000 Hz  & = $10^3$ Hz = 1 kHz (kilohertz) \\
1 000 kHz & = $10^6$ Hz = 1 MHz (megahertz) \\
1 000 MHz & = $10^9$ Hz = 1 GHz (gigahertz) \\
\end{tabular}
\end{center}

Nätfrekvensen för elkraft är i Europa 50 Hz.

Andra nätfrekvenser förekommer, t.ex. 60Hz i USA.

Frekvensområdet vid överföring av kvalitativt tal och musik, lågfrekvens LF, är mellan ca 16Hz och 16kHz.

Frekvensområdet för talöverföring, t.ex. över telefonlinjer eller kommunikationsradio,
är c:a 300 till 3000 Hz.

Frekvensområdet för radioöverföring, högfrekvens HF, är i huvudsak mellan 50 kHz, s.k.
långvåg, och 100-tals GHz, s.k. mikrovåg.

\subsection{Fasförskjutning}

Bild II 1-17

Med fasförskjutning menas tidsskillnaden mellan förlopp, t.ex. spänningar och/eller
strömmar. Fasförskjutningen mellan vektorerna kallas även fasvinkel och uttrycks
som ett gradtal mellan O och 360$\circ$.

\subsection{Vektorer}

En spänning, ström, kraft o.s.v. kan beskrivas som en vektor med en storhet och riktning.
På bilden har vektorerna $X_L$, $R$ och $X_C$ en inbördes fasförskjutning av 90$\circ$.
De motsvarar spänningsfallen i en krets med en induktor, en resister och en kondensator
kopplade i serie.

Antag att vektorerna roterar i ett oförändrat inbördes läge och med en vinkelhastighet
av $\omega= 2\pi f$. Systemet roterar då $360\circ = 2\pi radianer = 1 varv/period$.

Vid varje tidpunkt har vektorsystemet uppnått en viss vridningsvinkeL Momentanvärdet på
vektorernas spänningar avsätts till höger i bilden. Avståndet mellan en vektorspets och
noll-linjen är vektorns momentana värde, som kan vara positivt eller negativt.

Bild II 1-17 Vektorer och fasförskjutning

\cleardoublepage

\section{Icke sinusformade signaler}

\subsection{Grundton, övertoner- Kantvågor}

Bild II 1-18

Ett sinusformat förlopp med en enda frekvens- en enda ton- sägs vara spektralt ren.
En sådan svängning kallas för grundton.

Varje signal, som inte är sinusformad, är sammansatt av flera sinussvängningar. Det är
signalens grundton samt dess harmoniska övertoner, vilka kan ha 2, 3 o.s. v. gånger högre
frekvens än grundtonen. Den inbördes styrkan på grundton och övertoner avgör signalens
form. Om signalen ligger inom det hörbara området, kan man märka hur den ändrar karaktär
beroende på övertonshalten. Man kan säga att övertonerna modulerar grundtonen.

Bild II 1-19

Oscillatorsignalen i exemplet på bilden har 1 volts amplitud på grundtonen f0 (1:a
harmoniska), 0.7 volts amplitud på de n 1 :a övertonen (2:a harmoniska) och 0.2 volts
amplitud på den 2:a övertonen (3:e harmoniska). Den totala amplituden blir emellertid inte
summan av 1, 0.7 och 0.2 volt eftersom de olika delspänningarnas toppvärden inte uppträder
samtidigt. I stället måste delspänningarna adderas vid varje tidpunkt för sig.

Bild II 1-20

Det finns olika karaktärer av förlopp såsom sinusvåg, triangelvåg, sågtandsvåg,
fyrkantvåg o.s.v.

Fyrkantvågen är sammansatt av sinusvågor med grundfrekvensen och dess udda övertoner,
varvid amplituderna fördelar sig som 1/1, 1/3, 1/5, 1/7, 1/9, 1/11 o.s.v. Teoretiskt når
övertonsspektrum upp till oändligt höga frekvenser, medan de motsvarande amplituderna
minskar till oändligt små värden.

En ideal fyrkantvåg, vilken inte kan uppnås i praktiken, skulle bestå av ett oändligt
antal udda övertoner med fallande amplitud. Ju fler av de högre övertonerna som filtreras
bort, desto mer lutar fyrkantvågens flanker, desto rundare blir hörnen på vågen och
desto vågigare blir kurvans topp.

Bild II 1-18 Ren sinusvåg och övertonshaltig våg

Bild II 1-19 Uppdelning av en signal i grundton och övertoner

Bild II 1-20 Uppdelning av en fyrkantvåg i grundton och övertoner

\subsection{Överlagrade spänningar
(likspänningskomposant)}

Bild ll 1-21

I signalkretsar förekommer det mycket ofta, att växelspänning överlagras på likspänning
eller omvänt. Likspänningen kallas då för likspänningskomposant Olika åtgärder behövs för
att överlagra spänningar på varandra och att sedan skilja dem åt.

Bilden visar ett avsnitt av en AM-mottagare. Från vänster hämtas en AM-modulerad signal
från MF-förstärkaren för att demoduleras, d.v.s. för att återvinna den modulerande
LF-signalen. MF-signalen halvvågslikriktas. Kvar blir den positiva delen av MF-signalen
och den modulerande LF-signalen, sammanlagrade. LF-signalen skall nu skiljas ut och
förstärkas. Alltså filtreras MF-komposanten bort. Kvar blir LF-signalen, men överlagrad på
en likspänning. Likspänningen stoppas och kvar blir slutligen LF-signaien som förstärks.

Bild II 1-21 Överlagrade spänningar

\cleardoublepage

\section{Modulation}

\subsection{Allmänt}

Modulera (lat. modulari, rytmiskt avmäta) är att med hjälp av en oftast högfrekvent
elektrisk signal (bärvågen) överföra informationen i en lågfrekvent signal. På så sätt kan
lågfrekvens, t.ex. tal och musik, först omvandlas till en elektrisk signal, som får 
påverka (modulera) en högfrekvent elektrisk signal. Denna modulerade signal strålas ut från
antennen som ett elektromagnetiskt fält.

Den signal som innehåller informationen kallas modulerande signal eller basband eller
underbärvåg.

Den signal som informationen överförts till kallas modulerad signaleller huvudbärvåg.

\subsection{Modulationssystem}

Den största gruppen av modulationssystem är definierad med avseende på hur huvudbäNågen är
modulerad. Vanligast är då amplitud- och vinkel modulation. Av vinkelmodulation finns
främst två slag, frekvensmodulation och fasmodulation. Därutöver finns system för
pulsmodulation

\subsection{Sändningsslag}

Sätten att modulera kallas sändningsslag. Gemensamt för sändningsslagen är att en
givare-det kan vara en mikrofon, en telegrafnyckel, en fjärrskriftmaskin, en dator, en
TV-kamera o.s.v.- alstrar en analog eller digital signal. Denna styr underbärvågen så att
huvudbärvågen moduleras med den avsedda informationen och sänds ut.

Det enklaste sändningsslaget får anses vara morsetelegrafi med "nycklad bärvåg".
Då förekommer bara två tillstånd, nedtryckt och icke nedtryckt telegrafnyckel, d.v.s.
antingen bärvåg med någon varaktighet eller ingen bärvåg alls. Kombinationer av
bärvågselement med olika längd motsvarar skrivtecken.

För att återge tal, musik etc. behövs en noggrannare tillståndsstyrning av bärvågen.
Det innebär att bärvågen måste moduleras av en underbärvåg och att denna motsvarar
lufttrycksvariationerna i ljudet.

\subsection{Kännetecken för modulerade signaler}

Bild 111-22

En modulerad signal kännetecknas av dess amplitud, frekvens och fasläge.

Vid amplitudmodulation påverkas huvudbärvågens amplitud, så att den i varje tidpunkt
motsvarar den modulerande signalens variation.

Vid frekvensmodulation påverkas huvudbärvågens frekvens, så att den i varje tidpunkt
motsvarar den modulerande signalens variation.

Vid fasmodulation, som är besläktad med frekensmodulation, påverkas i ställettörfrekvensen
huvudbärvågens fasläge i förhållande till en referenssignal, så att fasläget i varje
tidpunkt motsvarar den modulerande signalens variation.

Frekvens- och fasmodulation liknar varandra och kan sammanfattas som vinkelmodulation,
eftersom fasvinkeln mellan bärvågens spänning och ström varierar i båda fallen.

Vid pulsmodulation används pulståg (korta upprepade bärvågspaket); t. ex. pulsamplitud-,
pulslängds-, pulsläges- och pulskodmodulation. Pulskodmodulation används t.ex. vid
samtidig överföring av flera telesamtal på samma linje, bärvåg etc.

\subsection{Bandbredd vid olika sändningsslag}

Varje radiosändning tar upp plats omkring den nominella bärvågsfrekvensen- tillsammans
bandbredden.

Radioamatören måste veta detta "platsbehov", främst för att inte sända utanför de
frekvensband som är tilldelade för amatörradioanvändning, men även för att kunna
umgås med annan trafik inom banden.

I alla sändningsslag ökar den använda bandbredden med ökad modulation. Eftersom största
frekvenseffektivitet alltid skall eftersträvas så upptar en sändare med kraftigare
modulation än vad som behövs för en överföring, alltid onödigt frekvensutrymme.

Bild II 1-22 Modulerade signaler

\subsection{Beskrivningskod för sändningsslagen}

Vid 1979 års radioförvaltningskonferens (WARC 79) i Geneve reviderades det internationella
radioreglementet (RR), som i huvudsak trädde i kraft 1982. Däri ingår bl. a. ett nytt
system för klassindelning och beteckning av sätten att utsända information över
radio m.m. Reglementet har reviderats senare, men i detta stycke gäller det ännu.

Indelningen i sändningsslag behövs för att känneteckna utsändningarna, t. ex. i
frekvenslistor, författningar och föreskrifter. Indelningen är också av stort värde vid
teknisk beskrivning av apparater och system för radiokommunikation.

Emellertid används av många även äldre benämningar, vilka lever kvar i litteraturen, i
märkning av manöverdonen på sändare och mottagare o.s.v ..

Dessa äldre benämningar är dock inte entydiga och skapar lätt missförstånd, varför
beskrivningskoden enligt WARC 79 bör användas för tydlighetens skull.

Här följer avkortade koder enligt WARC 79 för några av de sändningsslag, som amatörer
använder mest, samt för jämförelse även de benämningar som fortfarande används jämsides
(se vidare i Appendix E).

\begin{description}
\item[NON] Bärvåg utan modulerande signal. Ingen information.

\item[A1A] Bärvåg med dubbla sid band. En enda kanal med kvantiserad bärvåg. Ingen
modulerande underbärvåg. Telegrafi. Även kallat nycklad bärvåg (CW).

\item[A3E] Linjärt modulerad huvudbärvåg. Dubbla sidband. En enda kanal med
analog information. Telefoni.

Även kallat amplitudmodulation (AM).

\item[J3E] Linjärt modulerad huvudbärvåg. Ett sidband med undertryckt bärvåg. En
enda kanal med analog information. Telefoni.

Även kallat enkelt sidband (Single Side Band-SSB).

\item[F3E] Vinkelmodulerad bärvåg. Frekvensmodulering. En enda kanal med analog
information. Telefoni.

Även kallat frekvensmodulering (FM)

\item[G3E] Vinkelmodulerad bärvåg. Fasmodulering. En enda kanal med analog information.
Telefoni.

Även kallat fasmodulering (PM)
\end{description}

Såväl A1A, A3E som J3E är sändningsslag där amplituden moduleras. Därför är
termen amplitudmodulation inte tillräcklig för att beskriva flera likartade sändningsslag.

\subsection{Modulerande signaler}

\subsubsection{Basband}

Basband är ett frekvensområde för en modulerande signal. Det finns ett basband för
alla slags modulerande signaler, vare sig de är analoga eller digitala. Det kan finnas mer
än ett basband i en komplett modulationsprocess. Till exempel är en nycklad ton, som
går till sändaren genom mikrofoningången, dess analoga basband medan nycklingspulserna
till tongeneratorn är dess digitala basband.

Bild 111-23

Ett vanligt sätt att överföra information över radio är med telefoni, d.v.s. tal.

Frekvensområdet 300-3000 Hz räcker för god förståelighet av tal. Dels är örat känsligast
inom det området och dels finns där den mesta energin i talet.

Mikrofonen tar upp de lufttrycksvariationer, som uppstår när man talar, och omvandlar dem
till elektriska svängningar. Svängningarna varierar mellan positiva och negativa
spänningsvärden.

\subsubsection{Försök}

\begin{enumerate}
\item Anslut en mikrofon till ett oscilloskop och studera spänningsförloppen för olika slags
ljud, toner, tal o.s.v. som funktion av tiden. På bilden är dessa svängningar mycket
förenklade, t.ex. sinusformade.

\item Anslut en högtalare och ett oscilloskop till en LF-generator, vars frekvens och amplitud
kan ändras. Lyssna på ljud med låg och hög frekvens samt på svaga och starka ljud. En
baston har låg frekvens och en diskantton har hög frekvens. En svag ton har liten
amplitud och en stark ton har stor amplitud.
\end{enumerate}

\subsection{Sändningsslaget A3E (även kallat AM)}

Bild 111-24

Bilden visar frekvensspektrum av en signal vid amplitudmodulation med

\begin{enumerate}[label=\alph*.,noitemsep]
\item en sinuston,
\item en blandning av tre sinustoner,
\item ett frekvensspektrum.
\end{enumerate}

\subsubsection{Försök}

Modulera en A3E-sändare med en 3 kHzsignal. Med en mottagare utrustad med ett
smalt filter för telegrafi, kan man urskilja och påvisa bärvågen och de båda sidbanden.

\subsubsection{A3E-modulation med en ton}

Bild 111-25

En omodulerad bärvåg har konstant amplitud. En amplitudmodulerad signal är i grunden
resultatet av svävning mellan frekvenser eller av icke linjär blandning av frekvenser. När 
bärvåg och basband blandas, så är särskilt tre blandningsprodukter av intresse.

Dessa är

\begin{enumerate}[label=-,noitemsep]
\item bärvågen,
\item det lägre sidbandet (förkortat LSB) och
\item det övre sidbandet (förkortat USB).
\end{enumerate}

AM-signalen består således inte bara av bärvågsfrekvensen fHF utan även av övre
och nedre sidofrekvenser, vilka är summan och skillnaden av bärvågsfrekvensen $f_{HF}$ och
den modulerande frekvensen $f_{LF}$. Alltså $f_{HF} + f_{LF}$ (övre sidfrekvens) och
skillnadsfrekvensen $f_{HF} - f_{LF}$ (undre sidfrekvens).

Bild II 1-23 Modulerande signaler

Bild II 1-24 Sidband vid A3E-modulation

Eftersom tal inte bara omfattar en enda frekvens utan ett helt frekvensspektrum (c:a
0.3 - 3 kHz), så uppstår inte bara två sidfrekvenser utan två sidband, det lägre sidbandet
(LSB, Lower Side Band) och det övre (USB, Upper Side Band).

LF-signalens frekvens bestämmer sidfrekvensens avstånd från bärvågen. Bandbredden på en
amplitudmodulerad signal med full bärvåg och två sidband är dubbelt så stor som den högsta
modulerande LFfrekvensen:

$b= 2 \cdot f_{LFmax}$

Om de modulerande LF-frekvenserna är mellan 0.3 och 3 kHz, så blir sändningens
totala bandbredd 6 kHz.

LF-signalernas amplitud påverkar sidbandens och sidfrekvensernas amplitud. Vid
maximal modulation (100 \% modulationsgrad) varierar signalamplituden mellan noll
och dubbla värdet av det för en omodulerad bärvåg.

Som mest kan vardera sidbandet överföra en fjärdedel så mycket effekt som bärvågen, d.v.s.
en sjättedel av den totalt utsända effekten. Då avger sändaren dubbelt så stor medeleffekt
som utan modulation. Toppeffekten (PEP, Peak Envelope Power) är till och med fyra gånger
så stor.

slutförstärkaren och kraftförsörjningen måste dimensioneras för toppeffekten vid
full modulation eller att modulationsgraden anpassas så att överbelastning inte sker.

Bild II 1-25 A3E-modulation med toner med olika styrka och frekvens

\subsubsection{Fördelar med A3E-modulation}

En A3E-sändare är enkel jämfört med en J3E-sändare, vilken har en mer komplicerad
signalbehandling.

\subsubsection{Nackdelar med A3E-modulation}

Eftersom samma information finns i båda sidbanden och ingen finns i bärvågen, så sänds
effekten i bärvågen och ett av sidbanden ut till ingen nytta. I talpauser sänds endast
bärvågseffekten och till ingen nytta. Aven frekvensutrymme slösas bort. Då en annan,
alltför närliggande sändares bärvåg blandas med den egna, så alstras renstoner i
mottagarna.

\subsection{Sändningsslaget A1A (även kallat CW)}

Bild 111-26

Man kan överföra meddelanden med morsetelegrafi på olika sätt. Det enklaste sättet är att
koppla in och ur sändarens bärvåg i takt med teckendelarna i morsetecknen. Man kan kalla
det för bärvågstelegrafi. Förfarandet kallas sedan mycket länge även för CW (continous
waves), vilket egentligen anger att bärvågen svänger med konstant amplitud, om man bortser
från att den nycklas. Detta i motsats till de dämpade bärvågssvängningar som var fallet i
sedan mycket länge förbjudna s.k. gnistsändare.

Fastän en sändare "moduleras utan ton", har den en viss bandbredd. Det beror på att den
takt, som sändaren nycklas med, egentligen är en ton - låt vara med låg frekvens. Antag
att sändaren nycklas med en serie korta morsetecken. Vid telegraferingshastigheten
60 tecken/minut alstrar bärvågspulserna en kantvåg med frekvensen 5 Hz. Som tidigare
beskrivits, består en sådan kantvåg av summan av sinussignaler med frekvenserna 5 Hz,
15 Hz, 25 Hz, 35 Hz o.s.v.

Det innebär att det uppstår sidfrekvenser över och under bärvågens frekvens och med
ett avstånd till bärvågen av 5 Hz, 15 Hz, 25, 35Hz o.s.v .. Telegrafisändaren har alltså
liksom vid A3E en bandbredd, som dels står i förhållande till nycklingshastigheten och
dels till "kantigheten" på tecknen, vilket bestämmer övertonshalten i bärvågen. Vid s.k.
mjuk nyekling kan den 9:e övertonen antas vara den högsta som uppfattas av en motstation.
Med en nycklingsfrekvens av 5 Hz blir bandbredden inte större än
$2 \cdot 10 \cdot 5 = 100Hz$.

En hård (kantig) och snabb teckengivning ökar bandbredden och kan resultera i att s.k.
nycklingsknäppar kan uppfattas långt vid sidan om sändningsfrekvensen. Ju hårdare
nycklingen är, desto längre bort från bärvågsfrekvensen hörs nycklingsknäpparna. Detta
stör andra stationer.

Kännetecken för sändningsslaget A1A, telegrafi genom nycklad bärvåg:

Mycket liten bandbredd, extremt gott utnyttjande av såndareffekten, stor
överföringssäkerhet, lång räckvidd, enkla sändare.

Bild II 1-26 Amplitudmodulation med morsetecken

\subsection{Sändningsslaget J3E (även kallat SSB)}

\subsubsection{Princip}

Som sagts är det onödigt sända ut två sidband, eftersom båda innehåller samma information.

Signaler med endast ett sidband och undertryckt bärvåg kan alstras på flera sätt.
Numera är den s.k. filtermetoden i särklass vanligast och den enda som behandlas här.

Bild II i-27

Med filtermetoden blandas HF- och LFsignalerna i en speciell blandare. Där undertrycks
båda dessa signaler medan blandningsprodukterna med deras summa- och skillnadsfrekvenser
blir kvar, d.v.s. det övre och nedre sidbandet.

Utsignalen från blandaren benämns DSBsignal (Double Side Band). Till skillnad från
i A3E-signalen saknas dock bärvågen i DSBsignalen. För att även undertrycka det ena
sidbandet före sändningen, så följs blandaren av ett bandpassfilter med bandbredd
och frekvensläge för avsett sidband.

Den signal som sänds ut innehåller därför endast ett sidband (Single Side Band).

\paragraph{Exempel}

Bild II 1-28

Ett SSB-filter har ett passband av 9000.39003 kHz. Vid bärvågsfrekvensen 9000kHz
sträcker sig det övre sidbandet från 9003.39003 kHz och släpps igenom. Däremot blir
bärvågsfrekvensen undertryckt.

Det undre sidbandet 8997-8999.7 kHz faller utanför filtrets passband och blir också
undertryckt.

Skall däremot det undre sidbandet kunna passera igenom samma filter, så måste
bärvågsfrekvensen höjas med 3 kHz, alltså till 9003 kHz. Då faller det undre sidbandet,
9002.7-9000.0 kHz inom filtrets passband.

Det övre sidbandet 9003.3-9006.0 kHz faller nu utanför passbandet och blir undertryckt.

Bild II i -29

LF-signalens amplitud bestämmer amplituden på sidfrekvensen.

LF-signalens frekvens bestämmer sidfrekvensens avstånd från bärvågsfrekvensen (bärvågen
undertryckt).

Bandbredden på den utsända signalen är skillnaden mellan högsta och lägsta
modulerande frekvens i signalen:

t.ex. $b = 3kHz - 0.3 kHz = 2.7 kHz$

\subsubsection{Fördelar med J3E-modulation}
Bra verkningsgrad vid J3E-modulation jämfört med vid A3E-modulation (traditionell AM).
Effekten i det utsända sidbandet motsvarar den i ett av sidbanden vid A3E. Hela den
utsända effekten finns alltså i ett enda sidband, som överför hela informationen.

I sändningspauserna sänds ingen effekt ut. Bandbredden är mindre än hälften av den
vid A3E. Vid mottagning av en J3E-sändning (SSB) är det mindre besvär med interferenstoner
från J3E-sändningar på närliggande frekvenser, eftersom ingen bärvåg och endast ett
sidband sänds ut.

\subsubsection{Nackdelar med J3E-modulation}
J3E-modulation medför mera komplicerade apparater, både för mottagning och sändning.
En J3E-signal blir förvrängd och hörs i fel tonläge, om mottagaren inte är inställd på
exakt rätt frekvens.

Bild II 1-27 Sidband vid DSB

Bild II 1-28 Sidbandsval vid SSB

Bild II 1-29 Sidbandlägen vid SSB

\subsection{Vinkelmodulation}
Termen vinkelmodulation är samlingsnamnet för frekvensmodulation (FM) och fasmodulation
(PM). Ofta sägs utrustningar vara för frekvensmodulation, när de antingen är för frekvens-
eller fasmodulation. Det finns alltså skillnader och likheter mellan dessa system, vilka
emellertid inte är oberoende av varandra, eftersom frekvensen i en signal inte kan
varieras utan att fasen också varieras, och vice versa.

Hur effektiv kommunikationen då är, beror mest på mottagningsmetoderna. I båda fallen
uppfattas ändringar i den mottagna signalens frekvens och fasläge. Amplitudändringar
uppfattas däremot inte. De flesta störningar - särskilt pulserande sådana som från
tändningssystem - kommer att därför att skiljas bort.

För att effektivt utnyttja fördelarna med vinkelmodulation, antingen det är frekvenseller
fasmodulation, behövs tillräckligt frekvensutrymme. Det innebär att främst högre
frekvensband kommer i fråga.

\subsection{Frekvensmodulation (även kallat FM)}

Bild II 1-30 (överst och i mitten)

Vid frekvensmodulation varierar bärvågens frekvens i takt med den modulerande signalens
amplitud och polaritet. På bilden ökar bärvågens frekvens när den modulerande signalen är
positiv (första halvperioden) och minskar när den modulerande signalen är negativ (andra
halvperioden). Bilden visar att perioderna i den modulerade bärvågen tar kortare tid (har
högre frekvens), när den modulerande signalen är positiv, och mertid (har lägre frekvens)
när den modulerande signalen är negativ. Bärvågen kommer alltså att pendla omkring ett
medelvärde, d.v.s. vara frekvensmodulerad.

Frekvensawikelsen L1f (deviationen) från bärvågens vilafrekvens är vid varje tillfälle
proportionell till den modulerande signalens amplitud. Sålunda är deviationen liten när
den modulerande signalens amplitud är liten och störst när amplituden når sitt toppvärde,
antingen amplituden är positiv eller negativ. Vid en modulationsfrekvens av 300 Hz
varierar bärvågsfrekvensen 300 gånger per sekund, vid 3kHz - 3000 gånger per sekund.

Likspänningsnivåer kan överföras med FM, eftersom en motsvarande frekvensavikelse kan
framställas.

Bilden visar också vad som oftast sägs, att bärvågsamplituden inte ändras av modulationen.
Detta är emellertid bara delvis sant, eftersom såväl bärvågsamplitud som sidbandsamplitud
varierar med modulationsindex, vilket förklaras nedan.

\subsubsection{Sidbanden vid vinkelmodulation}

Vid AM produceras endast ett sidbandspar med samma inne hål!, ett över och ett under
bärvågsfrekvensen. Vid vinkelmodulation, både vid FM och PM, produceras däremot flera
sidbandspar över och under bärvågsfrekvensen. Dessa sidband uppträder på multiplerna av
varje modulerande frekvens. Vid basband med samma frekvensomfång har därför en
vinkelmodulerad signal större bandbredd än en AM-signal.

Vid vinkelmodulation beror antalet sidband på sambandet mellan den modulerande frekvensen,
frekvensdeviationen och modulationsindex.

\subsubsection{Bandbredden vid vinkelmodulation}

Bild II 1-30 (nederst)

Vi gör tankeexperimentet att en FM-sändare moduleras med en fyrkantvåg. Frekvensen
kommer då att hoppa växelvis mellan frekvenserna $f$ och $f + \Delta f$. Sättet kallas FSK
(frekvensskiftnyckling) och används t. ex. vid sändning av radiofjärrskrift (RTTY, AMTOR,
Paketradio etc.).

Vi föreställer oss två sändare, som sänder varannan gång, varav den ena sänder frekvensen
$f$ och den andra sänder $f + \Delta f$. Båda sändarnas HF-signaler kommer då att bilda
ett frekvensspektrum, som förutom $f$ och $f + \Delta f$ även innehåller sidfrekvenser.

Bredden på detta spektrum beror bl. a. på nycklingsfrekvensen. Eftersom en fyrkantvåg
innehåller summan av dess grundfrekvens och övertoner, kommer alla dessa toner att
modulera vardera sändaren. De högsta modulerande LF-frekvenserna alstrar sidfrekvenserna
längst ut från vilofrekvensen. LF-signalens frekvensspektrum påverkar alltså
HF-signalens bandbredd.

Spektrum nederst i bilden är en förenklad framställning av frekvensskiftnyckling.

Bild II 1-30 Frekvensmodulation

Vid modulation med en sinussignal istället för med en fyrkantsignal, uppstår ett
frekvensspektrum som på överst i bilden.

\paragraph{Frekvensdeviation och modulationsindex}

Bild II 1-31

Vid vinkelmodulation uppstår talrika sidefrekvenser, som beror av den modulerande
frekvensen $f_{LF}$. Amplitudfördelningen mellan sidfrekvenserna står i förhållande till
deviationen, varvid deras amplitud blir mindre desto längre bort från bärvågen de är.

I praktiken anses en sidfrekvens försumbar när dess amplitud är mindre än 1 \% av
amplituden för omodulerad bärvåg.

För beräkning av bandbredden används begreppet modulationsindex m, vilket är kvoten av
maximal deviation $\Delta f$ och högsta frekvensen $f_{LF}$.

$m = \frac{\Delta f_{max}}{f_{LFmax}}$

Inom amatörradion är det vanligt att arbeta med $\Delta f_{max} = 3 kHz$ och
$f_{LFmax} = 3 kHz$, d.v.s. $m = 1$.

Vid modulationsindex $m = 1$, gäller följande
formel för bandbredden $b$

$b = 2 \cdot ( \Delta f_{max} + f_{LFmax}) = 2 \cdot \Delta f_{max} + 2 \cdot f_{LFmax}$

Med ovan nämnda värden blir bandbredden $b = 2 \cdot (3 kHz + 3 kHz) = 12 kHz$

Bandbredden ökar således både med ökande deviation och ökande modulerande frekvens. För
att inte interferera med trafik på grannkanalerna måste såväl deviation som frekvensen på
den modulerande signalen begränsas. En deviationsbegränsare begränsar amplituden på denna
signal. Ett lågpassfilter reducerar den distorsion, som uppstår av begränsningen. Vidare
undertrycks modulerande frekvenser högre än 3 kHz, vilket är tillräckligt för överföring
av tal.

\paragraph{Jämförelse}

En VHF-rundradiosändare är tilldelad ett större frekvensutrymme och kan därför använda
mycket större bandbredd

Där är $\Delta f_{max} = 75 kHz$ och $f_{LFmax} =15 kHz$, därmed är $m = \frac{75}{15} = 5$
och $b = 2 \cdot (75 + 15) = 180 kHz$.

Som framgår av tabellen på nästa uppslag varierar bärvågens liksom sidfrekvensernas
inbördes amplitud med modulationsindex. Detta skall jämföras med AM där bärvågens amplitud
är konstant och endast sidbandens amplitud varierar.

Vid vinkelmodulation utsläcks bärvågen $A_0$ vid modulationsindex 2.404. Den blir sedan
"negativ" vid högre index, vilket betyder att den återkommer, men att dess fasläge blir
omvänt. I vinkelmodulation tas energin i sidbanden från bärvågen, vilket innebär att
den totala effekten förblir densamma oavsett modulationsindex.

\paragraph{Kännetecken för sändningsslaget F3E (FM)}

Fördelar: F3E-sändaren är enkel till sin uppbyggnad och hög överföringskvalitet
uppnås vid stor bandbredd, störningar från amplitudmodulerade signaler t. ex. tändgnistor
undertrycks i mottagaren.

Nackdelar: En relativt stor bandbredd behövs för överföring av ett basband med
stort frekvensomfång. Sändaren måste avge full effekt, även när modulation inte sker.

Bild 111-31 sidbandsspektrum vid FM-modulering med 1 sinuston

\begin{table*}[h]
\begin{center}
\begin{tabular}{ll|l|l|l|l|l|l|l|l|}
\cline{3-9}
&\multicolumn{1}{l}{}  & \multicolumn{7}{|c|}{Modulationsindex} \\ \cline{3-9}
&\multicolumn{1}{l|}{}  &   1   &   2   &    3   &    4   &    5   &    6   &    7   \\ \hline
\multicolumn{1}{|c|}{\multirow{11}{*}{\rotatebox[origin=c]{90}{Relativ amplitud på}}}&$A_0$ & 0.765 & 0.224 & -0.260 & -0.397 & -0.178 &  0.151 &  0.300 \\
\multicolumn{1}{|c|}{}&$A_1$ & 0.440 & 0.577 &  0.334 & -0.066 & -0.328 & -0.277 & -0.005 \\
\multicolumn{1}{|c|}{}&$A_2$ & 0.115 & 0.353 &  0.486 &  0.364 &  0.047 & -0.243 & -0.301 \\
\multicolumn{1}{|c|}{}&$A_3$ & 0.020 & 0.129 &  0.309 &  0.430 &  0.365 &  0.115 & -0.168 \\
\multicolumn{1}{|c|}{}&$A_4$ &       & 0.034 &  0.132 &  0.281 &  0.391 &  0.358 &  0.158 \\
\multicolumn{1}{|c|}{}&$A_5$ &       & 0.016 &  0.043 &  0.132 &  0.261 &  0.362 &  0.348 \\
\multicolumn{1}{|c|}{}&$A_6$ & \multicolumn{2}{c|}{} &  0.011 &  0.049 &  0.131 &  0.246 &  0.339 \\
\multicolumn{1}{|c|}{}&$A_7$ & \multicolumn{3}{c|}{} &  0.015 &  0.053 &  0.130 &  0.234 \\
\multicolumn{1}{|c|}{}&$A_8$ & \multicolumn{4}{c|}{}           &  0.018 &  0.057 &  0.128 \\
\multicolumn{1}{|c|}{}&$A_9$ & \multicolumn{4}{c}{Tomma fält för $A_n$ under 0.01 (1 \%)} &        &  0.021 &  0.059 \\
\multicolumn{1}{|c|}{}&$A_{10}$ & \multicolumn{5}{c}{} &  &  0.024 \\ \hline
\end{tabular}
\end{center}
\caption{Relativa amplituden på bärvåg $A_0$ och sidfrekvenser $A_1$-$A_{10}$ vid
modulationsindex 1-7 (Vid omodulerad bärvåg är modulationsindex 0. Då är
bärvågens relativa amplitud 1.0)}
\end{table*}

\subsection{Fasmodulation (även kallat PM)}

Vid fasmodulation varierar bärvågens fasläge i förhållande till ett
referensvärde. Vid PM är frekvensändringen - deviationen - direkt proportionell
till hur snabbt fasläget på den modulerande frekvensen ändras och till den
totala fasändringen. Hastigheten på fasändringen är direkt proportionell till
frekvensen på den modulerande frekvensen och till den momentana amplituden på
den modulerande signalen.

Det betyder att deviationen i PM-system ökar både med den momentana amplituden
och frekvensen på den modulerande signalen. Detta att jämföras med FM-system där
deviationen är proportionell till den momentana amplituden på den modulerande
signalen.

I PM-system uppfattar demodulator i mottagaren endast momentana ändringar i
bärvågsfrekvensen. Till skillnad från vid FM, så kan därför ändringar i
likspänningsnivåer överföras endast om en fasreferens används.

Med konstant amplitud på insignalen till modulatorn, så är vid PM
modulationsindex konstant oavsett modulerande frekvens, medan vid FM
modulationsindex varierar med den modulerande frekvensen.

\subsection{Frekvens- och fasmodulation jämförs}

\begin{itemize}

\item Frekvensmodulation (FM) alstras genom att sändarens oscillatorfrekvens
varieras (devieras) i takt med den modulerande signalen (t.ex. tal). Det gör man
genom att variera resonansfrekvensen i den svängningskrets som styr
oscillatorfrekvensen.

\item Fasmodulation (PM) alstras vanligen genom att efter sändarescillatorn
variera den modulerande signalens fasläge i förhållande till en omodulerad
bärvåg - s.k. fasmodulering. Det gör man genom att variera resonansfrekvensen i
en svängningskrets efter oscillatorn- d.v.s. utan att påverka
oscillatorfrekvensen.

\item I båda fallen ändrar man alltså resonansfrekvensen i en svängningskrets i
takt med frekvensen i den modulerande spänningen, men att denna krets har olika
placering i FM-sändare respektive PM-sändare.

\item I sändaren alstras det i båda fallen utfrekvensersom devierar från
oscillatorns vilofrekvens. Graden av deviation skiljer emellertid vid FM och PM.
Vid FM är deviationen proportionell mot amplituden på den modulerande
underbärvågen medan deviationen vid PM är proportionell mot produkten av den
modulerande underbärvågens amplitud och frekvens.

\item Den hörbara skillnaden mellan FM och PM är därför en annorlunda
frekvensgång. Vid samtidig användning av PM-sändare och FM-mottagare är det
alltså lämpligt att justera frekvensgången i PM-sändarens modulator, lämpligen
med 6 dB dämpning per oktav ökad frekvens.

\end{itemize}

\subsection{Pulsmodulation}

Pulsmodulation används mest i mikrovågområdet. Pulsmodulerade signaler sänds
vanligen som en serie korta pulser åtskilda av relativt långa pauser utan
modulering.

En typisk sändning kan bestå av pulser med en längd av 1 $\mu$s och en frekvens
av 1000 Hz. Toppeffekten på en pulssändning är därför mycket högre än dess
medeleffekt.

Före WARC 79 var symbolen för all pulssändning P. Därefter används P endast för
omodulerade pulståg. Annan pulsmodulation har följande symboler

\begin{description}
\item[K] - puls-/amplitudmodulation (PAM)
\item[L] - pulsviddmodulation (PWM)
\item[M] - pulsposition/fasmodulation (PPM)
\item[Q] - vinkelmodulation under pulsen
\item[V] - kombination av dessa eller annat sätt.
\end{description}

\begin{table*}[h]
\begin{center}
\begin{tabular}{|l|l|l|l|l|}
\hline
Sändningsslag & Amplituden på & Tonhöjden på & Bandbredden b      & För stor amplitud \\
              & LF-signa!en   & LF-signalen  & förhåller sig till & på LF-signalen \\
              & påverkar      & påverkar     &                    & medför \\ \hline
A3E (AM) & amplituden i   & sidfrekvenser- & LF-signalens    & övermodulering \\
         & båda sidbanden & nas avstånd    & högsta frekvens & och för stor bandbredd \\
         &                & från bärvågen  & & \\
J3E (SSB)& amplituden på  & sidfrekvenser- & skillnaden mellan & för stor bandbredd,\\
         & utsänt sidband & nas avstånd    & LF-signalens      & överstyrning av\\
         &                & från bärvågen  & högsta och lägsta & förstärkarsteg\\
         &                &                & frekvens          & \\
F3E (FM) & deviationen    & hastigheten på & dubbla summan     & för stor deviation,\\
         &                & bärvågens      & av största devia- & för stor bandbredd\\
         &                & frekvens-      & tion och högsta   & \\
         &                & ändring        & LF-frekvens       & \\ \hline
\end{tabular}
\end{center}
\caption{Jämförelse mellan några. vanliga. sändningsslag inom amatörradio}
\end{table*}

\cleardoublepage

\section{Effekt och energi}

\subsection{Effekt i en sinusformad signal}
För beräkning av effekten av en sinusformad signal använder man effektiwärdet
av spänning och ström.

$U_{eff} = \frac{U_{max}}{\sqrt{2}}$ och $I_{eff} = \frac{I_{max}}{\sqrt{2}}$

$P = U_{eff} \cdot I_{eff}$

Bild II 1-32 Effektförhållande

\subsection{Effektändring uttryckt i dB}
Måtten i det metriska systemet är alldagliga och ingen finner det märkligt att
det t. ex. går tio decimeter på en meter. Däremot är begreppet decibel ovant för
många.

I detta avsnitt förklaras det mycket användbara begreppet decibel. Decibel (dB)
är en tiondedel av grundenheten Bel (B).

Räkning med decibel grundas på logaritmer, som är ett bekvämt sätt att uttrycka
och behandla talvärden.

Decibel är ett dimensionslöst uttryck för graden av dämpning alternativt
förstärkning.

Effektdämpning är följden av att vissa komponenter bromsar elektrisk ström. Den
bromsande faktorn kan vara en resistans R, induktans L, kapacitans C eller
sammansatta nätverk av R, L och C.

Effektförstärkning innebär att en transistor, ett elektronrör eller annan s.k.
aktiv komponent kan styra en större elektriskström och därmed större effekt än
den själv styrs med. Vad som förorsakar effektförändringarna går vi inte in på i
detta sammanhang, utan byggdelarna betraktas som "svarta lådor" med
anslutningsklämmor.

En byggdel med två ingångs- och två utgångsklämmor kallas för "fyrpol".

Antag attden inmatade effekten P är 1 W. Om effekten inte ändras vid passagen
genom fyrpolen, så är även den uttagna effekten 1 W.

Effektförhållandet mellan in- och utgångarna är då

$\frac{P_{in}}{P_{ut}} = \frac{1 watt}{1 watt} = 1 (kvoten = 1)$

Oförändrad effekt varken dämpas eller förstärks, varför både dämpningen och
förstärkningen har talvärdet 0. Enheten på talvärdet är Bel, dämpningen eller
förstärkningen är således 0 Bel. En tiondel därav är 0 decibel (0 dB).

Omräkning av kvoten av en effektändring till dB görs så, att 1O-logaritmen för
kvoten söks och resultatet blir effektändringen uttryckt i Bel (B). Om
resultatet uttrycks i dB, skall Bel-värdet multipliceras med 10.

Logaritmer förklaras i Appendix B.

För att förenkla beräkningen av dB-talet divideras det högre effekttalet med det
lägre. Bokstaven a i följande formler betyder antingen förstärkning (+ a) eller
dämpning (-a) beroendet på vilket förtecken som sätts.


$a[B] = \log \frac{P_{hög}}{P_{låg}}$

$a[dB] = 10\log \frac{P_{hög}}{P_{låg}}$

Att addera eller subtrahera värden på en logaritmisk skala, motsvarar att
multiplicera resp. dividera värden på en linjär skala. Huvudskalorna på en
räknesticka är logaritmiska. (Räknestickan är ett enkelt, förut mycket använt
hjälpmedel).

Med hjälp av följande nomogram kan en effektändring, uttryckt som kvot
(effekterna dividerade med varandra), omvandlas till decibel och omvänt.



Följande avrundade värden kan utläsas:
O dB = 1
1 dB = 1.25 2 dB = 1.6
3 dB = 2
4 dB = 2.5
5 dB = 3.2
6 dB = 4
7 dB = 5
8 dB= 6.3
9 dB = 8
1O dB = 1O 11 dB = 12.5
d. v. s. vid ökning fördubblas effekten för var
3:e dB och vid minskning halveras effekten
för var 3:e dB.

Om kvoten är en eller flera 1O-potenser
högre än 1O, så kan nomogrammet utökas
enligt följande tabell.
Kvot av Analys
PhögiPiåg
1
1 har O nollor
1O
1O har 1 nolla
100
100 har 2 nollor
1 000
1 000 har 3 nollor
1O 000 1O 000 har 4 nollor

Skriv

dB

o. 10 = o
1·10=10

2. 10

= 20

3. 10 = 30
4. 10 = 40

\subsection{Strömändring uttryckt i dB}
Förhållandet mellan strömmar liksom mellan spänningar kan även uttryckas i dB, men
annorlunda än mellan effekter. En fyrpol
med inbördes lika ingångs- och utgångsimpedans är förutsättningen för jämförelse.
Enligt Joules lag är P= l

2

•R

(P= U·~

En jämförelse uttryckt i dB kan endast göras
under samma förutsättningar; här att impedanserna (resistanserna) är lika,

111 -54

a[ dB]= 1Olog 'h'og22
flåg

Eftersom log

x2

=

2 ·log x, fås slutligen

a[ dB] = 20 log /hög
flåg

\subsection{Spänningsändring uttryckt i dB}
Förhållandet mellan spänningar kan uttryckas i dB på ett liknande sätt som med
strömmar.

~

2

Enligt Joules lag är P=

(P=

U·~

Två effekter kan ställas i förhållande till varandra på följande sätt:

~ög

-=
~åg

uhög2:R
2
ulåg

:R

R avkortas och efter omskrivning fås en

formel som liknar den för strömmar

~ög

2
uhög

~åg

ulåg

-=--2

log Uhög
qåg

R kan avkortas om in- och utgångsimpedanserna (resistanserna) är lika.

således

Effektförhållandet eller kvadratvärdet på
strömförhållandet kan uttryckas logaritmiskt
i B eller dB

a[ dB] = 20

således

Effekt

~ög

lhög

~åg

flåg

2

-=-2

Med följande nomogram kan kvoten av en
ström- eller spänningsändring omvandlas till
decibel och tvärt om.
Följande avrundade värden kan utläsas:
O dB = 1
1 dB ~ 1.12 2 dB ~ 1.25
3 dB ~ 1.4
4 dB ~ 1.6
5 dB ~ 1.8
6 dB ~ 2
7 dB ~ 2.24 8 dB ~ 2.5
9 dB ~ 2.8
1O dB ~ 3.2 11 dB ~ 3.6
d. v. s. vid ökning fördubblas strömmen resp.
spänningen för var 6:e dB och att vid minskning halveras strömmen resp. spänningen
för var 6:e dB.

L-LÄRA
1

1.2

1.1

l

l

o

i
1

2

lijl l i l i

l
o2 4 6 8
l

1.3
!

l

3
l
i

l
10

l

l
2

l

l

4
l
l
12

l

J

l
3

5
l
l
14

1.6

1.5

1.4
l

l

i

6
l

l

l
4

7
li

l

l
16

Om kvoten är en eller flera 1O-potenser
högre än 1O, så kan nomogrammet utökas
enligt följande tabell.
Skriv
Kvot*
Analys
20
1 har O nollor
1
1 . 20
10
1O har 1 nolla
2. 20
100
100 har 2 nollor
1 000
1 000 har 3 nollor 3. 20
10 000 1 O 000 har 4 nollor 4. 20
*kvot av Uhö/U 1å9 resp. lhö/l,å9

o.

=

dB

o

= 20

= 40

= 60
= 80

Se Appendix C för beräkning med tabeller.

\subsection{Ändring uttryckt i dB vid förstärkande eller
dämpande anordningar kopplade i serie}

Ett räkneexempel på effektändringar:
Fråga:
Vi har en enkel sändaranläggning med
ett drivsteg med en in effekt av 1O W. Drivsteget förstärker med 6 dB. Vidare har vi ett
effektslutsteg som förstärker med 1O dB.
Antennkabeln dämpar med 1 dB.
Med vilken effekt matas själva antennen?
Svar: (två sätt att lösa uppgiften)
1) Drivsteget förstärker fyra gånger, slutsteget förstärker tio gånger och kabeln
dämpar 1/1.25 = 0.8 gånger. Antennen
matas då med 10 · 4 · 10 · 0.8 = 320 W.
2) Drivstegets 6 dB plus slutstegets 1O dB
minus antennkabelns 1 dB = 15 dB.
15 dB är 1O+ 5 dB d.v.s. 1O· 3.2 = 32 ggr.
Antennen matas med 1OW· 32 = 320 W.

1.8

1.7

l

l

l
5

8
l
l
18

2.0 g gr

1.9
l

l

l

l

6

Spänning
dB

10 ggr

9
l
i

Spänning
20

dB

\subsection{Impedansanpassning}
Impedansanpassning är av stor betydelse
inom kommunikationsstekniken. Normaltvill
man nämligen överföra mesta möjliga effekt
från energikällan (t. ex. sändaren) till förbrukaren (t.ex. antennen).
Varje spänningskälla har en inre resistans Ri. Det innebär som först att källan inte
kan avge oändligt stor ström. För att förenkla
det hela antar vi nu att en sändare med den
inre resistansen Ri ansluts direkt till en antenn med resistansen Ra.
Målet med anpassningen är att finna det
optimala förhållandet mellan sändarresistansen och antennresistansen för att kunna
överföra maximal effekt. Vi har de två ytterlighetsfallen obelastad sändare respektive
kortsluten sändare. Sändarens elektromotoriska kraft (EMK) betecknas som E [V]
och sändarens utspänning som U [V].
Fall1.
En obelastad sändare avger ingen ström när
ingen antenn eller en med oändligt stor resistans har anslutits.
Alltså vid obelastad sändare:

Ra =oo

l= O U= E

Fall2.
När sändarutgången är kortsluten, d.v.s.
belastningen (antennresistansen) är noll
ohm, avger sändaren en ström som beror av
EMK och inre resistans. Eftersom såndarutgången är kortsluten är utspänningen Unoll.
Alltså vid kortsluten sändare:

Ra=O !=E
Ri

U=O

111 -55

EL-LÄRA
l båda ytterlighetsfallen är den effekt som
omsätts i Ra lika med noll. För att få ut någon
effekt måste man alltså söka ett värde på Ra
som ligger mellan ytterlighetsvärdena.
Enligt formeln för spänningsdelare är
utspänningen

U=E·

Ra
Ra+Ri

Formeln för uteffektens effektivvärde är

u2
p=ut

R

a

Efter insättning får man
p = p ·Ra

ut

(Ra +Ri)2

För att finna det optimala förhållandet
mellan Ri och Ra, d.v.s. när Ra tar upp maximal effekt, måste man differentiera formeln
med d Pa/dRa, men vi hoppar över denna
utflykt i matematiken.
l stället konstaterar vi helt enkelt att
maximal effektöverföring sker när Ri= Ra.

\subsection{Förhållandet mellan in- och uteffekt uttryckt som \% verkningsgrad}

Antag att en antennkabel har en effektförlust
av 1 dB. Det innebär en effektdämpning av
1.25 gånger, d.v.s. 0.8. Nu matar vi in 1O W
i kabeln och får alltså ut 8 W. Hur stor
verkningsgrad har kabeln uttryckt i o/o ?
Lösning:

8

1J = 1Q ·1 00 = 80o/o

\subsection{Toppvärdeseffekt P.E.P.}

Uteffekten från en sändare kan mätas över
en konstlast (dummy load). En konstlast är
en res istor som kan omsätta sändarens hela
effekt till värme. Med HF-mätprob och en
detektordiod eller en HF-voltmeter kan man
mäta effektivvärdet på spänningen över
konstlasten och beräkna uteffekten med
formeln

u2

p
ut =RU = HF-spänningens effektiwärde
R = resistansen i konstlasten

111-56

På grund av utsignalens karaktär kan
man inte mäta effektiwärdet av uteffekten
från SSB-sändare.
Med oscilloskop kan man emellertid mäta
utspänningen på den största modulationstoppen.
Med detta toppvärde kan man beräkna
spänningen över konstlasten.
Uteffekten definierad som P.E.P. (Peak
Envelope Power) är "den medeleffekt som
matas in i en antennmatarledning under det
högsta effektvärdet inom en frekvenscykel
och mätt under normal drift".

f12

P.E.P.=R
där Oär momentanspänningen på den största
modulationstoppen.


\chapter{KOMPONENTER}

\section{Resistorn}

Allmänt

Strömkretsar består av komponenter med
olika egenskaper. Den vanligaste egenskapen, åtminstone i likströmskretsar, är resistansen. För att få avsedd funktion, så anpassar man resistansen i komponenterna.
Exempel
En krets med strömkälla, lampa, kopplingsledningar och smältsäkring. Kopplingsledningarna mellan komponenterna bör ha
låg resistans och därför lågt spänningsfall
(små förluster). Lampan skall däremot ha
hög resistans och därmed höga förluster för
att kunna bli het och lysa. Smältsäkringen
skall skydda ledningarna från för hög ström.
Säkringen ges därför en resistans, som gör
att den smälter när strömmen överstiger ett
tillåtet värde.
Som hjälpmedel för att fördela spänningar
och strömmar i en krets, så används en
komponenttyp kallad resistor. Dess utmärkande egenskap är resistans- även kallad
ohmskt motstånd.

Enheten Ohm

(Se även kapitel 1)
Resistansen mellan två punkter i en strömkrets är 1 n (uttalas "en åm"), när spänningen mellan punkterna gör att en ström av
1 A (en ampere) flyter i kretsen.
Inom elektroniken används höga resistansvärden och därför även följande multipler av enheten
3
(1 kn) == 106 Ohm
1 kiloohm
1 mego hm
(1 Mn) == 10 Ohm

Resistans i strömledare

För att bestämma resistansen, t. ex. i en tråd,
behöver man veta dess resistivitet, tvärsnittsyta, längd och temperatur.

Resistivitet
Resistivitet är ett materials strömledningsegenskaper. Ett annat namn för resistivitet
är specifik resistans. Symbolen för resistivitet
är p (uttalas rå).

..
. . . ..
ohm ·mm 2
Form e ln for res1St1v1tet ar p =- - - m
Följande formel gäller för beräkning av
resistansen i en strömledare med linjär ström/spänningskaraktär.

R= p!

A

![meter] A [mm

2

]

[p= n·A]
m

Exempel
l = 4 m koppartråd
A== 2 mm 2
p (koppar) == 0.017

l
A

R=p-

R=0.017i=o.034 n
2

Not. Förväxla inte A [tvärsnittsytan] i denna
formel med beteckningen A t. ex. i Ohms lag då A
betecknar strömstyrkan.

Resistiva material

Resistorer kan utföras med olika typer av
resistiva material, vilket bestämmer användningsområdet.
En resistor, vars resistans är oberoende
av ström, spänning och annan yttre påverkan t.ex. temperatur och ljus, sägs ha linjär
karaktär. Om resistansen däremot beror av
yttre påverkan, så sägs resistorn ha olinjär
karaktär.
Man skiljer mellan tre huvudgrupper av
resistiva material. Det kan vara en kropp av
pressat kol eller ett ledande ytskikt på ett
isolerande underlag eller metalltråd på en
isolerande stomme. På senare tid har tillkommit integrerade resistorer, d.v.s. flera
resistorer av resistiva skikt på ett gemensamt isolerande underlag. Här beskrivs i
korthet
resistortyper. Se f.ö. leverantörskataloger.

Utförandeformer

Resistorer kan utföras med fast eller ställbart resistansvärde. Här följer först en översikt över resistorer med olika resistiva material och fast resistansvärde.

112-1

K MP NENTER
Fasta resistorer med linjär karaktär
Massaresistar
Det resistiva materialet består av kolmassa
med bindemedel (kolkomposit). Massan är
bakad till en stav eller ett rör. Anslutningsledningarna är inbakade i materialet.
Massaresistorer är lämpliga för lik- och
växelströmskretsar med låga krav på temperaturberoende och egenbrus. Den homogena kroppen gör att egeninduktansen är
låg. Å andra sidan uppstår vid höga frekvenser en skineffekt, d.v.s. strömkoncentration
vid ytan, som medför viss resistansökning.
Kolfilmsresistor
Det resistiva materialet består av ett kolskikt,
som genom förångning överförts till ett keramiskt rör. Resistansen bestäms av tjockleken på skiktet samt av spiralformade spår i
detta. Genom spiraliseringen tillförs en induktans, men som i någon mån uppvägs av
egenkapacitansen.
Metallfilmresistor
l denna typ är kolfilmen ersatt av ett metallskikt. Eftersom egenkapacitansen är liten,
så är typen lämpad för höga frekvenser.
Tjockfilmsresistor
Det resistiva materialet består en film av bl. a.
metalloxid, som screentrycks på ett keramiskt underlag. Typen har god tålighet mot
pulser och höga temperaturer, men har relativt högt egenbrus. Ytmonterade resistorer
är oftast tillverkade av tjockfilm.
Tunnfilmsresistor
Det resistiva materialet består av en tunn
metallfilm, som genom förångning överförts
till ett underlag av glas eller keramik.
Denna resistertyp har över lag god stabilitet och används ofta i apparater med hög
precision. Egenskaperna vid höga frekvenser är dock inte så bra.
Metalloxidresistor
Denna resistertyp har ett spiralformat skikt
av metalloxid. Temperatur- och spänningsberoendet är måttligt. Tåligheten mot pulser
och höga temperaturer är stor. Typen kan i
någon mån ersätta trådlindade resistorer.

112-2

Resistarnät
Resisternät (integrerade resistorer) består
av flera resistiva skikt på ett gemensamt
isolerande underlag, d.v.s. liknande teknik
som för tjock-och tunnfilmsresistorer.
Trådlindad resistor
Det resistiva materialet är en metalltråd,
som är lindad på en stomme som tål hög
temperatur; det kan var keramik, glas etc.
Tåligheten mot pulser och höga temperaturer är stor.

Fasta resistm·er med olinjär karaktär
Vanligast är att materialet i resistorer har
linjär ström-/spänningskaraktär, men det
finns även sådana med olinjär karaktär. l
resistorer med olinjär karaktär är det ingående materialet av halvledartyp.
Spänningsberoende resistor
- Voltage Dependent Resistar (VOR)
Linjära resistorer påverkas knappast av den
pålagda spänningen. Resistorer av kiselkarbid har däremot en hög resistans vid låg
spänning och omvänt en låg resistans vid
hög spänning. VOR används t.ex. för begränsning av spänningstoppar.
Ljusberoende resistor, fotoresistor
- Light Dependent Resistar (LDR)
Ledningsförmågan i halvledare påverkas inte
bara av värme utan även av ljus. Halvledare
av germanium och särskilt sammansatta
halvledare av kadmiumoxid, blysulfid och
indiumantimonid har särskilt stor ljuskänslighet. Kadmiumsulfid är känsligast för synligt ljus medan andra material är känsligast i
det infraröda området.
Magnetfältberoende resistor (fältplatta)
Resistansen ökar med längden på strömledaren. Denna egenskap används i magnetfältberoende fältplattor. En sådan består
av en keramisk bärarplatta med en yta av
indiumantimon id. l ytan är ytterst smala parallella metallbanor inlagda på ett avstånd av
någon J.lm. Normalt går strömmen kortaste
vägen tvärs över banorna, men när ett magnetfält träffar vinkelrätt mot plattans yta, så

K MP NENTER
avlänkas elektronerna. De får då längre väg
över till nästa metallbana och den totala
resistansen ökar.

Temperaturberoende resistor

Se nedan om NTC och PTC i resistorer.

Temperaturkoefficienten för resistorer
Resistansen i ingående material påverkas
av temperaturen, vaNid det skiljer mellan
materialen.
Amorft kol och de flesta halvledande
materialleder bättre när de är varma- de har
en negativ temperaturkoefficient (NTC). Sådana material finns t. ex. i dioder och transistorer.
Däremot leder metaller och speciella
halvledarmaterial bättre när de är kalla- de
har en positiv temperaturkoefficient (PTC).
Glödtråden i glödlampor och elektronrör är
resistorer med positiv temperaturkoefficient
(PTC). l vissa metallegeringar, som t.ex. i
konstantan, kan resistansen till och med
vara nästan konstant vid varierande temperatur.
Alla material har en temperaturkoefficient,
som anger hur mycket resistansen ändras
per grad. Resistansen vid någon annan temperatur kan därför beräknas med följande
formel, där man sätter in begynnelsetemperaturen [ o] (o c), temperaturändringen [ .1.0]
och temperaturkoefficienten [a].

Rvarm= Rkall$\pm$ a· il ?J· Rkatt
Resistansändringen är ledet
AR= $\pm$a· .1.19- ·Rkatt
Temperaturkoefficienten kan vara positiv (NTC) eller negativ (NTC).
l principscheman har PTC- respektive NTCresistorer symboler som på bilden.
Bild nr 112-1

Variabla resistorer
En resister kan även utföras med variabelt
resistansvärde. Då används endast den andel av det resistiva materialet, som finns
mellan en resistors ena ände och ett uttag
någonstans mellan ändarna. En sådan anordning kallas för reostat. Om en variabel
resister används som spänningsdelare, så
kallas den för potentiometer.

l en potentiometer används dels hela
resistansen mellan ändpunkterna och dels
andelen mellan uttaget och någon av
ändpunkterna.
Uttagets mekaniska utförande beror oftast av hur bekvämt inställningen skall kunna
ske. En potentiometer, där det resistiva
materialet är lagt på en cirkulär bana och
uttaget är fäst vid en axel i banans centrum,
medger enkel inställning med mejsel, ratt
etc. Ett enklare slags uttag är en släpkontakt
eller ett spännband som kan flyttas utmed en
stavformad resistor.

Resistiva material i variabla resistorer

Banan i en variabel resister består i princip
av liknande resistiva material som i en fast
resistor.
Billigast och enklast är en bana av kol,
som är tryck på ett enkelt underlag. Nackdelar är låg efekttålighet, dålig upplösning och
linjäritet, högt brus och kort livslängd. Fördelen är lågt pris.
Bättre än en kolbana är en bana av kolkomposit, d.v.s. kolpulver med bindemedel,
som är tryckt på ett underlag. Nackdel är
högre pris och låg effekttålighet, medan fördelarna är god upplösning, lågt brus och
lång livslängd.
Vill man ha god effekttålighet och temperaturstabilitet, utöver kolkompositens egenskaper, så erbjuder en bana av cermetsådana fördelar. En cermetbana består av en
blandning av metaller och keramik, som
trycks på ett underlag.
Trådlindad bana har främst god tålighet
mot hög effekt. Tålighet vid hög ström genom uttaget är en nannan fördel.

Linjära och olinjära patentiometrar

En potentiometer har en kuNform, varvid
avses resistansändringen som funktion av
uttagets rörelseväg utmed resistansbanan.
KuNformen kan utföras linjär, logaritmisk
etc .. Olinjära kuNor består då oftast av en
följd av linjära segment, som tillsammans
någorlunda motsvarar den önskade olinjära
formen. Ett exempel på det är när en kurvform
anges som linjär/logaritmisk.

112-3

K MP N
Effektutveckling i resistorer
l resistorer utvecklas värme av den ström
som flyter igenom. Värmeutvecklingen sker
enligt Joule's lag, som återges i kapitel1.
Hur mycket effekt i form av värme som
strålas ut från resistorn beror på storleken på
dess yta och egentemperatur samt på omgivningens temperatur. Det finns en övre
gräns för hur stor värme det ingående materialet tål innan det förstörs och eventuellt
fattar eld.
En resistors effekttålighet framgår i vissa
fall av påstämplade värden. l övriga fall är
man hänvisad till kataloguppgifter eller en
bedömning, som ev. kan grundas på höljets
utseende och dimensioner.
standardiserade komponentvärden
Resistorer tillverkas vanligen med standardiserade värden ur någon talserie.
Märkning av resistorer
Resistorer märkas med huvuddata enligt
något system av siffror och bokstäver eller
med en färgkod. Flera olika system tillämpas.
(Se f.ö. leverantörskatalogerför information
om komponentdata, märkning o.s.v.)

2

1
2
3
4

3

4

Allmän symbol
ställbar resistor, potentiometer
Trimbar resister (trimmer)
Automatiskt ställbar resister

Bild 112-1 Schemasymboler för resistorer

112-4

5

5
6
7
8

6

7

Temperaturberoende resistor,
Temperaturberoende resistor,
Spänningsberoende resistor,
Ljusberoende resistor,

8

NTC
PTC
VOR
LDR

K MP NENTER

2.2 Kondensatorn
Allmänt

Följande formler gäller för kapacitansen i
en enkel kondensator med två plattor. När
en kondensator är uppbyggd av n stycken
plattor, ökar kapacitansen med faktorn (n-1 ).
Med vakuum som dielektrikum gäller

Så snart det finns en elektrisk potentialskillnad .,..-en spänning - mellan två kroppar, så
uppstår ett elektriskt kraftfält mellan dem.
Ett sådant fält är lagrad elektrisk energi.
Kropparna måste då isoleras från varandra.
Elektrisk energi lagras mellan olika delar
av en strömkrets, även om de inte är direkt
avsedda för det. Särskilt vid mycket höga
frekvenser har detta stor betydelse för utformningen av en strömkrets. Vid låga frekvenser och likström däremot, har kretsens
utformning mindre inverkan. Då behövs i
stället särskilda anordningar för ta upp eller
avge energi på önskade ställen i strömkretsen.
En sådan anordning kallas kondensator.
Den består i princip av två band eller plattor
med anslutningsledningar samt ett isolerande skikt- dielektricum- däremellan.
Kapacitansen är näst efter resistansen
den vanligaste egenskapen i en strömkrets.

Kapacitans är elektricitetsmängden per volt
där måttenheten är Farad [F]. Eftersom denna
enhet är mycket stor, används inom elektroniken oftast bråkdelar av den.
1 mikrofarad (1 !-!F) = 1o-6 F
1 nanofarad (1 n F) = 1 9 F
1 pikofarad (1 p F) = 1o- 12 F

Kapacitans

Kondensatorn i likströmskretsen

Förmågan att lagra elektrisk energi (elektrisk laddning) kallas kapacitans. Ordet kommer från latinets capax, som betyder rymlig,
duglig.
Kapacitansen betecknas i formler med
bokstaven C
•
•
•
•

En kondensators kapacitans bestäms av
ytan på kondensatorns plattor,
avståndet mellan dessa ytor,
den absolutadielektricitetskonstanten c0
den relativa dielektricitetskonstanten
som är den faktor kapacitansen ökar med
när dielektrikum är annat än vakuum.

c,;

Kapacitans, dimension och dielektrikum

Kapacitansen är proportionell med den yta,
som kondensatorplattorna skuggar varandra, och är omvänt proportionell med plattavståndet.

1•

A

C=cod

Med ett godtyckligt dielektrikum gäller

2.
C [Farad]

d [mm]

E [F/m]

Enheten Farad

o-

En kondensator i en likströmskrets har alltid
samma polaritet. Därvid förhåller sig kondensatorns polspänning till dess laddningsmängd.
En ström flyter till kondensatorn och laddar upp den, när den anslutna spänningskällan har högre spänning än kondensatorn. Ju
högre spänningen är, desto större är laddningen. Ju kortare uppladdningstiden är, desto högre effekt utvecklas under den tiden.
När en uppladdad kondensator ansluts
till en krets med lägre spänning, så urladdas
kondensatorn till kretsen. Ju kortare urladdningstiden är, desto högre effekt utvecklas
under den tiden.
Laddningen i en kondensator kan innebära hög polspänning. Om kondensatorns
kapacitet är stor, kan laddningsmängden bli
betydande. Varning för elektriska stötar och
brännskador!

112-5

K
Kondensatorn i växelströmskretsen
l en likströmskrets förhåller sig kondensatorns polspänning till laddningsmängden. l
en växelströmskrets växlar emelltid spänningen och polariteten ständigt och därmed
kondensatorns laddning och polaritet.
Not: Vissa kondensatortyper kan inte användas i rena växelströmskretsar.
Försök
En glödlampa och en kondensator kopplas i
serie med varandra och ansluts till en
växelströmskrets. Med lämpligt valda värden på komponenterna kommer lampan att
lysa upp.
Detta visar att en kondensator inte hindrar elektronflödet i en växelström krets. Man
brukar säga att kondensatorn "släpper igenom växelström", men i stället är det så att
laddningar pendlar mellan kondensatorns
plattor genom den strömkrets som kondensatorn är ansluten till.
Använd för säkerhets skulllåg spänning,
t.ex. den från en ringledningstransformator!

Kapacitiv reaktans
Strömstyrkan i en växelströmskrets beror
bl.a. på hur stor kondensatorns kapacitans
är, d.v.s. på dess kapacitiva reaklans Xc.
Ordet reaktans kommer från latinets re
(åter) agere (verka).
Större kapacitans innebär större förmåga att ta upp elektrisk laddning och ger
därmed en lägre reaktans. Resultatet blir ett
kraftigare elektronflöde. En mindre kapacitans innebär ett svagare elektronflöde.

1
2n:fC

Xc=-- eller

1
mC

Xc=-

[O]
[Hz]
[F]
eller
[MO]
[MHz]
[uF]
(samma sortenheter i alla led)
Exempel:

f = 50 Hz
1
1
xc -- 2 n:fC - 2 n 50 ·1 O·1

1.

2.

112-6

C = 1O J.tF

c = 1oJ.tF

Xc = ?
=318.3 Q

XC = ?

En kondensators reaktans är således
omvänt proportionell med dess kapacitans
och frekvensen i kretsen.
Jämför detta med en induktor där reaktansen är proportionell med frekvensen.
När en ström flyter genom en res istor, så
uppstår det värmeförluster. När ström flyter
genom en ideal reaktans- en induktor eller
en kondensator - uppstår däremot inga
värmeförluster.

Fasförskjutning i en kondensator
Med fasförskjutning menas här den tidsmässiga förskjutningen mellan ström- och
spänningsförloppen. l en kondensator når
nämligen strömmen inte sitt toppvärde samtidigt som spänningen. l en ideal kondensator är spänningen fasförskjuten 90$\circ$ efter
strömmen.
Förlustvinkel
l praktiken är fasförskjutningen i en kondensator något mindre än 90$\circ$ på grund av att
laddning läcker igenom dielektrikum. Man
talar om en förlustvinkeL Läckningen kan
ses som en resister som är kopplad parallellt
över kondensatorn.
läckström m.m.
Med det extremt tunna. dielektrikum har
elektolytkondensatorn en mycket högre kapacitet än andra former, men har också
några nackdelar, bl.a. att
• den normalt endast kan användas med
likspänning,
• den har hög förlustfaktor p.g.a.läckström,
• det utvecklas värme av läckströmmen,
vilket skapar övertryck p.g.a. gasbildning.
Utförandeformer
Kondensatorer kan utföras med fast kapacitansvärde. Dielektrikum består då av ett
skikt av glimmer, impregnerat papper o.s.v.
Kondensatorer kan även utföras med variabelt kapacitansvärde. Dielektrikum består
då oftast av luft, men kan även vara ett fast
material.

Fasta kondensatorer
Kondensatorer har oftast namn efter utförande och materialet i dielektrikum
Pappers- och plastkondensatorer
'Plattorna" i dessa typer består av aluminiumremsor med anslutningstrådaL Däremellan finns en pappers- respektive plastremsa
som dielektrikum. För att spara plats, så
rullas det hela ihop och skyddas med en
plastingjutning.
Keramiska kondensatorer
l keramiska kondensatorer består dielektrikum av något keramiskt material. På ömse
sidor om detta sätts en metallbeläggning
med anslutningstrådaL
Glimmerkondensatorer
l denna kondensatortyp består dielektrikum
av tunna glimmerskivor
Elektrolytkondensatorer
Elektrolytkondensatorer har elektroder av
aluminium eller tantal, därpluspolen (anoden)
ges ett mycket tunt oxidskikt Detta är inte
ledande och fungerarsom dielektrikum. Mellan oxidskiktet och minuspolen (katoden)
läggs en elektrolyt med låg resistivitet.
Elekrolytkondensatorer har särskilt högt
kapacitansvärde. Till skillnad från andra kondensatortyper, så är elektolytkondensatorer
polariserade. Utom i ett specialfall innebär
det, att polariteten på den pålagda spänningen inte får kastas om. Flera olika slags
elektrolytkondensatorer finns, såsom våta
och torra aluminiumelektrolytkondensatorer, tantalelektrolytkondensatorer m. fl.

Variabla kondensatorer
Variabla kondensatorer har oftast sitt namn
efter utförandeformen, såsom vridkondensator och trimbar kondensator (trimmer).

Temperaturkoefficient
På liknande sätt som med resistorer, så
påverkas kapaciteten i kondensatorer av
temperaturen. Att sambandet mellan kapacitet och temperatur är viktigt, förstås av att
temperaturkoefficienten i den frekvensbestämmande kapacitansen i en oscillatorkrets
är en av faktorerna för stabil frekvens.
Temperaturkoefficienten a anger kapacitetsändringen pergrad temperaturändring.
Kapacitetsändringen blir då

AG= $\pm$ac ·Ck· AiJ
varvid Ck är kapacitetsvärdet vid den lägre
temperaturen (oftast 20$\circ$C) och ~t) är
temperaturändringen i grader Kelvin.
Kelvin [K] är den normerade måttenheten
för absolut temperatur.
En ändring med 1 K motsvarar en ändring med 1
Är ac positivt betyder det att kapaciteten
ökar med ökande temperatur.
Är ac negativt betyder det att kapaciteten
minskar med ökande temperatur.
En kondensator som är märkt med N 100
betyder a c = -1 00 ·i 6 l K

oc.

o-

standardiserade komponentvärden
Kondensatorertillverkasvanligen med standardiserade värden ur någon talserie.
Märkning av kondensatorer
Kondensatorer märkas med huvuddata enligt något system av siffror och bokstäver
eller med en färgkod. Flera olika system
tillämpas.
(Se f.ö. leverantörskatalogerför information
om komponentdata, märkning o.s.v.

1
2
3
4

j

T
1

2

3

4

Allmän symbol
Trimbar kondensator (trimmer)
Vridkondensator
Polariserad kondensator,
elektrolytkondensator

Bild fl 2-2 Schemasymboler för kondensatorer

112-7

K MP NENTE

112-8

NENTER
2.3 Induktorn
Allmänt
När elektrisk ström flyter genom en ledare,
så alstras ett magnetfält omkring den. Så
snart strömmens styrka eller riktning ändras,
uppstår en motsvarande s.k. elektromotorisk kraft (EMK), som motverkar ändringen.
Kraften finns i magnetfältet, som är lagrad
magnetisk energi.
Självinduktion - induktans
Magnetfältets förmåga att alstra en motverkande EMK kallas självinduktion eller induktans. Ordet induktans kommer från latinets
inducere, som betyder införa.
När en ledare, som ingår i en sluten krets,
rör sig i ett magnetfält, så kommer en ström
att flyta genom ledaren på grund av den
EMK (spänning) som alstras. Varje ändring
av strömmen motverkas av det magnetfält
som strömmen själv alstrar.
När det uppstår självinduktion i en ledare,
så kallas ledaren induktor. Självinduktionen
är jämnt utbredd över ledarens hela längd.
När ett större induktansvärde behövs på
något särskilt ställe i strömkretsen, så kan
ledarens längd ökas just där och lindas upp
till en spole med lämplig form. Hela spolen
kallas då för induktor.
Det att ett motverkande magnetiskt fält
alstras omkring en ledare när strömmen i
den ändras, påverkar kretsens egenskaper
och därmed utformning på olika sätt. Vid
snabba strömändringar, t.ex. vid hög frekvens, är motverkan större än vid långsamma
ändringar. Vid konstant likström uppstår
däremot ingen motverkan- självinduktion.
Induktansen är efter resistansen och
kapacitansen den vanligaste egenskapen i
en strömkrets.
Försök med induktion
Försök 1
Bild II 2-3 överst
Ett känsligt vridspoleinstrument kopplas till
en induktor. Instrumentet bör ha noll på
skalans mitt, så att strömriktningen syns. En
permanentmagnet används för att visa att

självinduktion uppstår när magneten förs
fram och tillbaka genom induktorn.
Instrumentet ger utslag när magneten är
i rörelse. Utslaget blir större vid snabbare
hastighetsändring. Utslagsriktningen växlar,
när magneten förs in i respektive dras ut ur
induktorn - det uppstår en växelström.
En växelspänning uppstår över induktorn,
även när den ingår i en strömkrets som sluts
och bryts-alltså utan en magnetsom rör sig.
Försök 2
Bild II 2-3 mitten
Permanentmagneten byts nu mot ännu en
induktor. Utöver den första induktorn, som vi
nu kallar sekundärlindning, kallar vi den nya
induktorn för primärlindning.
När vi släpper ström genom primärlindningen så alstrar den ett magnetfält. Först är
strömmen noll för att sedan ändras till ett
högt värde och därefter återgå till noll. Det
blir en strömstöt
Varje ändring alstrar en mot-emk, som
bygger upp ett magnetfält, först i en riktning
och sedan i den andra. l båda fallen passerar
fältet genom båda lindningarna. Fältet från
primärlindningen inducerar en spänningsstöt i sekundärlindningen. Stöten har en riktning, när primärlindningens strömkrets sluts
och motsatt riktning när den bryts, d.v.s. det
blir en växelspänning. När sekundärlindningen ingår i en sluten krets, uppstår en
växelström genom sekundärlindningen.
Försök 3
Bild II 2-3 nederst
Vad händer när primärlindningen i försök 2
ansluts till en växelspänning, t.ex. med nätfrekvensen 50 Hz? Använd för säkerhets
skull en skyddstransformator mellan nätet
och lindningen!
l sekundärlindningen uppstår då spänningsstötar, vars polaritet i detta fall växlar
100 gånger per sekund. Det uppstår alltså en
växelspänning över sekundärlindningen och
om denna ingår i en sluten strömkrets uppstår det en motsvarande växelström.

112-9

N

R

E

VÄXELSTRÖM

STRÖMSTÖT

STRÖMSTÖT

Fältspole

Primärkrets

Induktionsspole

sekundärkrets
STRÖMSTÖT

l ndu kti ensspole

Pr i märkrets

sekundärkrets
VÄXELSTRÖM

Bild II 2-3 Försök med induktion
112- 1o

K MP NENTER

2

Allmän symbol,
induktor utan kärna
2 Induktor med kärna
3 Trimbar induktor
4 ställbar induktor

4

3

Bild II 2-4 Schemasymboler för induktorer

Olika utföranden
Elektromagneter, drosslar, induktorer för
svängningskretsar, ramantenner o.s.v.
Enheten Henry
Måttenheten för självinduktion är Henry (H)
1 Henry (1 H) är självinduktionen i en induktor, som alstrar en motspänning av 1 volt vid
en strömändring av 1 ampere under 1 sekund.
l formler betecknas induktans med L
Sambandet är
Volt= Henry · Ampere/sekund
1 H är en stor måttenhet. För elektroniktillämpningar används därför ett mer hanterligt format.
Exempel:
1 H= 1000 mH
1 mH = 1 · 1o-3 H
1 mH = 1000 J..LH
1J..LH = 1 · 1o-3 m H= 1 · 1o-s H

Hur induktansen påverkas
Induktansen beror på induktorns mekaniska
dimensioner, antalet lindningsvarv och materialet i kärnan.
Induktansen i en cylindrisk induktor är
proportionell mot tvärsnittsytan, omvänt proportionell mot längden och proportionell mot
kvadraten på lindningsvarvtalet
Induktansen ökar, om induktorn förses
med en kärna av järn och minskar med en
kärna av omagnetisk, ledande metall, t.ex.
koppar, mässing eller aluminium.

Induktiv reaktans
Till skillnad från när en resistor ansluts till en
spänning, så blir strömökningen i en induktor fördröjd. Orsaken är att en induktor inte
bara har en resistans, vilken ju inte påverkas
av strömvariationer, utan har även en induktiv reaktans XL. Ordet reaktans kommer från
latinets re (åter) agere (verka).
Reaktans - växelströmsmotstånd eller
skenbart motstånd - uppträder så länge
som strömmen genom induktorn ändras. En
induktor gör således också motstånd mot
varje strömändring och detta motstånd ökar
med ökande ändringshastighet
En fullbordad pendling i en växelström
kan ses som ett varv i en cirkel - 360$\circ$ -och
en fullbordad pendling kallas en period.
En period motsvarar omkretsen i en cirkel med radien r, där omkretsen är 2 · n · r
(n = 3.141593 .. ). När strömmen växlar 1
gång/sekund har pendlingen en frekvens [f]
av 1 Hertz [Hz]. Vid 50 växlingar/sekund har
pendlingen en frekvens av 50 Hz o.s.v.
Induktiva reaktansen XL - växelströmsmotståndet i en induktor- är en funktion av
strömmens s.k. vinkelhastighet m= 2 · rc • f
och av storheten av induktansen L.
Den induktiva reaktansen är proportionell mot strömmens frekvens och mot
induktorns induktansvärde. Inga förluster
uppstår i en ideal induktor, d.v.s. en som
teoretiskt saknar resistans.
Sambandet är
X L = 2 · rc · f· L = mL
eller

XL [Q]

f [Hz] L [H]
[MQ]
[MHz] [mH]
(exempel på prefix)

112- 11

K M N
Exempel:
1.
L= 1H
f = 50 Hz
XL = 2 . Jr. 50 . 1= 3 i 4 .Q
2.

L= 1H
f = 5 kHz
XL = 2 . Jr . 5000 . 1= 31400 .Q

Fasförskjutning mellan
och
ström i en induktor
Bild II 3-000 (i kapitel 3)
Med fasförskjutning menas den tidsmässiga förskjutningen mellan ström- och
spänningsförlopp. Strömmen genom en induktor, når inte sitt toppvärde samtidigt som
spänningen över den. Orsaken är växlingarna mellan elektrisk och magnetisk energi i
induktorn.
l en ideal induktor är spänningen fasförskjuten 90$\circ$ före strömmen. l praktiken är
dock förskjutningen något mindre än 90$\circ$ på
grund av resistansen i induktorn.
Q-faktor- godhetstal
Q-faktorn kan avse två olika saker, som inte
skall förväxlas. Det är Q-faktorn för en komponent respektive den för en hel strömkrets.
Q-faktorn för en induktor är kvoten av
dess reaktans och serieresistans.
Q

komponent -

xkomponent

R

komponent

Q-faktorn för en hel svängningskrets beror däremot på bredden på det frekvensband som en viss komponentkombination
ger. Q-faktorn för en resonant svängningskrets är därför ett mått på dess selektivitet
(se kapitel 3).
Medan Q-faktorn för en ingående komponent påverkar Q-faktorn för en hel krets,
så gäller inte det omvända.

Yteffekt {skin-effect)
l en ledare av homogent material fördelar sig
en likström likaöverhela tvärsnittet. Strömtätheten för en växelström däremot, minskar i
ledarens mitt och ökar i stället vid ytan. Ju
högre frekvensen är, desto större är strömtätheten vid ytan. Fenomenet kallas yteffekt
(på engelska skin effect) och uppträder i alla
ledare.

112- 12

Det djup i ledarmaterialet där laddningstätheten sjunkit till 37% av värdet vid ytan
kallas skin depth. För koppar är detta djup
c:a 70 mm vid 100 Hz. Vid 1 MHz har djupet
minskat till 0.07 mm och vid 100 MHz till
0.0067 mm. På grund av yteffekten är alltså
materialet i mitten av homogena ledare elektriskt mindre verksamt vid höga frekvenser.
Resistansen blir alltså större för växelström
än för likström, om ledaren är samma
Utöver frekvensen påverkas yteffekten
av ledarmaterialets elektriska och magnetiska ledningsförmåga. För att få låg resistans
i ledare för högfrekvent ström är det viktigt att
omkretsen är stor och att materialskiktet vid
ytan har hög ledningsförmåga. Det är därför
som induktorerna i sändarslutsteg ofta är
försilvrade och består av rör med stor diameter eller av breda band.

Temperaturkoefficient
Liksom med resistorer, så påverkas induktansen av temperaturen. Att sambandet
mellan induktans och temperatur är viktigt,
förstås av att temperaturkoefficienten i den
frekvensbestämmande induktorn i en oscillatorkrets påverkar frekvensstabiliteten.
Eftersom metallen koppar utvidgar sig
vid temperaturökning och induktorns tvärsnittsyta då blir större, så är temperaturkoefficienten vanligen positiv.
Temperaturkoefficienten aL anger induktansändringen per grad temperaturändring.
Induktansändringen blir då
!lL =$\pm$aL· Lk ·llfJ

varvid Lkär induktansvärdet vid den lägre
temperaturen (oftast 20 $\circ$C) och fl{} är
temperaturändringen i oKelvin.
Kelvin [K] är den normerade måttenheten
för absolut temperatur. En ändring med 1 K
motsvarar en ändring med 1 oc.
Induktorer kan innehålla kärnor av någon
metallegering, vars egenskaper också är
temperaturberoende.
l praktiken kan man knappast påverka
temperaturkoefficienten i en induktor. Eftersom en svängningskrets för det mesta även
innehåller kondensatorer, så kan man t.ex.
kompensera en positiv temperaturkoefficient
i induktorn med en negativ i en kondensator.

PT

NENTER

Förluster i kärnmaterial
När ett magnetiskt växelfält passerar ett
kärnmaterial så kommer atomerna (som är
permanentmagneter) att ständigt inta nya
lägen i materialet i takt med fältets frekvens.
Då uppstårvirvelström mar, s.k. järnförluster,
som dels påverkar materialets ledningsförmåga och dels höjer temperaturen i kärnan och därmed i hela induktorn.

112-13

K M

112-14

NENTER

K

p

R

2.4 Transformatorn
Allmänt

En transformator består av en eller flera
lindningar eller spolar av elektriska ledare.
Lindningarna är magnetiskt kopplade till varandra. Det innebär att de är anordnade så,
att ett magnetfält, som alstrats i någon av
lindningarna, även passerar genom övriga
lindningar.
När en växelspänning läggs över en lindning, kallas den primärlindning. l och omkring primärlindningen alstras då ett magnetiskt fält som växlar i takt med spänningen.
Primärfältet passerar även genom övriga
lindningar- sekundärlindningarna-och alstrar där spänningar och strömmar.
Den s.k. kopplingsfaktorn mellan lindningarna varierarförolika frekvenser, sämre
vid låga frekvenser (hundratals Hz) och bättre vid höga frekvenser (tusentals Hz). Speciellt vid låga frekvenser behövs en bättre
koppling för att avsedd effekt skall kunna
överföras mellan lindningarna. Då kan ledningsförmågan i den magnetiska flödesvägen ökas med hjälp av en järnkärna.

Terminologi
primärkrets
sekundärkrets
primärlindning
sekundärlindning
primärspänning u 1 sekundärspänning u2
primärström i1
sekundärström i2
lindningsvarvtal n primärt n1 sekundärt n2
varvtalsomsättning = !i eller n2

n2

impedansomsättning

ni

z rt

= z = ,i
1

2

2

Den ideala (förlustfria) transformatorn
Transformering av spänning och ström
Transformatorn är obelastad när sekundär-

kretsen är bruten.
Bild 112-6
När primärlindningen ansluts till en växelspänning, induceras det växelspänningar
både över primär- och sekundärlindningarna. Det uppstår även en ström i primärlindningen, men däremot inte i sekundärlindningen när sekundärkretsen är bruten. För
den obelastade transformatorn gäller sambandet

!!J..=!i
u2

n2

d.v.s. spänningen över lindningarna är proportionell med lindningsvarvtalet

2
1 Allmänna symboler
2 Transformator med järnkärna

Bild II 2-5 Schemasymboler för
transformatorer

Utföranden

Transformatorn kan utföras för olika ändamål, t.ex. som
spä n n ingstransformator,
strömtransformator eller
impedanstransformator
Utförandet påverkas även av överförd effekt
och frekvens.

Transformatorn är belastad när sekundärkretsen är sluten.
Bild II 2-7
När någon av transformatorns sekundärlindningar ingår i en sluten strömkrets, uppstår en sekundärström där.
sekundärströmmen alstrar ett magnetfält, som motverkar primärströmmens fält,
hindrar dess växlingar och tar ut energi från
primärkretsen.
Strömförbrukningen på primärsidan ökar
således i proportion till strömförbrukningen
på sekundärsidan. Transformatorn reglerar
själv hur mycket energi som den tar från
strömkällan och lagrar i fältet för att föra över
till sekundärkretsen.

112-15

K MP NENTER
För den belastade transformatorn gäller, att
strömmen genom lindningarna är omvänt
proportionell med lindningsvarvtalet

i1 n2
-=-

i2

n1

Av föregående formler följer att:

Bild II 2-6 Obelastad transformator

Bild II 2-7 Belastad transformator

112-16

Av~=

u1 ·i1 och ~ = u1 ·i1 följer att~=~

Om man bortser från förlusterna i transformatorn, så är den effekt som den tar från
kraftkällan lika med den effekt den avger.

Transformatortillämpningar
Sparkopplade transformatorer

Bild II 2-8
Här ovan har transformatorn beskrivits så att
primär- och sekundärlindningarnas enda
förbindelse med varandra är över ett magnetfält, alltså utan galvanisk förbindelse.
Varje lindning kan emellertid förses med
godtyckliga uttag. Över
av
finns då en spänning i proportion till det
lindningsvarvtal som finns mellan uttagen.
Detta är en metod att spara in på antalet
lindningar. För att t.ex. omsätta
ningen 230 V till 115 V används
spartransformator.

Med en spartransformator kommer olika
strömkretsar i galvanisk förbindelse med
varandra och särskild försiktighet skall därför iakttas vid användning av sparkopplade
transformatorer, p.g.a. risken för elolycksfall. Spartransformatorer bör därför inte användas i amatörradiosamman hang. Säkrast
ärtransformatorer med galvaniskt skilda ledningar och dessutom med speciellt bra isolering och kapsling- s.k. skyddstransformatorer.

Bild II 2-8 Sparkopplad transformator

Strömtransformatorer

Hög sekundärström under låg sekundärspänning kännetecknar en strömtransformator.
Strömtransformatorer används i elektriska svetsningsutrustningar, induktionsugnar
o.s.v. Strömtransformatorer används även
för mätning av höga växelströmmar.
Bild 112-9
Bilden visar principen för en induktionsugn, som är en transformator med en sekundärlindning med endast ett fåtal varv omkring en smältdegel.

Högspänningstransformatorer

Hög sekundärspänning under förhållandevis låg sekundärström kännetecknar en
spänningstransformator.

Högspänningstransformatorer används i
distributionsnät, neonskyltar, tändsystem för
förbränningsmotorer, anodspänningsaggregat för sändare o.s.v.
Bild II 2-i O
Bilden visar en transformator med ett gnistgap i sekundärkretsen för tändning av gas.

och klenspänningstransformatorer

2-1 i
Lågspänningstransformatorer används i lokala distributionsnät, vanligen med spänningen 400/230 V. För ökad säkerhet mot
elektrisk chock krävs dock att vissa apparater drivs med en s.k. klenspänning av högst
50 V över en s.k. skyddstransformator med
förstärkt isolering.

112-17

K MP NENTER

PT
h

12

nz

= n1

= 500

220 Vrv

Bild II 2-9 Strömtransformator

Uz
U1 : : 220 V
n1 =500 ~.....-   

-.J

~

Uz:::: 4 400 V

nz =10 000

Bild II 2-1 O Högspänningstransformator

u,

n1

u2

nz

- : : : : - ::::

Uz

n1 ::: 1000 ,   

--.J

n2

u, : : 220 v

1000
so
--20

~U2::::

1

4,4 v

= 20

Bild II 2-11 Klenspänningstransformator

Sambandet mellan varvtal och impedans

Transformatorn kan även användas för anpassning av impedanser. Impedansen Z i en
lindning är proportionell mot kvadraten av
dess lindningsvarvtal n.

112- 18

Om effekten i sekundärlindningen är lika
stor som i primärlindningen, gäller formeln

zp

n/

zs

ns

-=-2

PT
2.5

R

N

K

Halvledardioden

Allmänt
l en strömkrets kan av olika anledningar
ström tillåtas att flyta i en riktning men kanske inte i den motsatta. En anordning med
en sådan funktion kallas för diod.

Bild II 2-12 överst
En halvledardiod består av ett P-ledande
och ett N-ledande materialskikt som fogats
samman.

Först bestod en diod av två elektroder i
vakuum (se avsnitt 2.7). Därav namnet
vakuumdiod.
Numera består en diod oftast av någon
halvledare. Därav namnet halvledardiod.

Mellan de båda skikten utbildas ett tunt
gränsskikt, som inte innehåller laddningsbärare. Detta skikt kan vara ledande eller
icke ledande - ett spärrskikt- beroende på
polariseringen.

n

p

l

+
+

+
+ ·.·':'.

+
+

=l

pn - skikt utan
pålagd spänning

[>l

r-~

l
l

f

1.. - - ·~

,. .... - . .

\

p

+
+

c=::>

n

+
+

+
+

PASSRIKTNING

-

c:::::t> hålledning
....,.,.. elektronledning

+

+
+

p
+
+

+
+

pn- skiktet uppl6ses

n
SPÄRRIKTNING
pn -skiktet byggs upp

---,
l
l
l

,....--'

't

, .... l

l

:--

J

-~

\.. ... .,.J

o
o
-

+

Bild II 2-12 Spärrskiktet i en halvledardiod
112-19

K MP
Halvledardiodens karaktär
Framström, temperatur, förlusteffekt,
passriktning
Bild II 2-12 mitten
Förbinder man den positiva polen på en
spänningskälla med P-skiktet i en diod och
den negativa polen med N-skiktet så är
dioden polariserad i passriktningen. Spärrskiktet upplöses då och ström flyter genom
dioden. Elektronerna flyter till den positiva
polen och hålen till den negativa polen.
Backspänning, backström, läckström, spärrriktning
Bild II 2-12 underst
Förbinder man i stället den negativa polen
på en spänningskälla med P-skiktet i en diod
och den positiva polen med N-skiktet så är
dioden polariserad i spärrriktningen. Spärrskiktet blir då ännu kraftigare.
Endast en obetydlig ström l flyter genom dioden i den s.k. spärriktnfhgen även
vid ökande spänning U . Men över en viss
spänning ökar strömm~h snabbt- den s.k.
zenereffekten uppstår. Dioden kan då lätt
förstöras av en alltför hög ström.

Bild II 2-13
Strömmen 10 börjar att flyta när spänningen U0 har nått ett tröskelvärde (vid kiseldioder 0.6 V). När spänningen ökar ytterligare däröver, så ökar även strömmen.
Produkten av spänningsfallet överdioden
och strömmen genom den kallas förlusteffekt. Denna värmer upp dioden. Vid för hög
temperatur förstörs kristallstrukturen. En kiselkristall kan klara upp till 200
medan en
germaniumkristall klarar bara 75 $\circ$C.

oc

*
1
2
3

f

2

3

Allmän symbol
Zenerdiod
Kapacitansdiod

Bild 112-14 Schemasymboler för dioder

lo
50 mA

1v

l
l

l

l

l
l

Bild II 2-13 Halvledardiodens karaktäristik

112-20

20 nA

Uo

MP NENTER
Diodtillämpningar

Likriktning är det vanligaste tillämpningen
(se kapitel3). Halvledardioder görs även för
en rad andra ändamål och finns i en mängd
utföranden, såsom
• Dioder för spänningsstabilisering (zenerdiod).
Inom ett visst område är spänningsfallet
över en zenerdiod i en strömkrets i det
närmaste konstant medan strömmen varierar. Denna egenskap kallas zenereffekt
och används för konstanthållning av spänning.
Det finns zenerdioder
många olika
spänningar och effekter.
•

Dioder som variabel kondensator, s.k.
kapacitansdiod (VariCap).
När en diod är polariserad i spärriktningen så bildas det ett spärrskikt Olika polariseringsspänning alstrar olika tjocka
spärrskikt En spärrad diod har på så sätt
egenskaper som liknar dem i en variabel
kondensator. Det finns därför dioder där
reglerbarheten av kapacitansen är speciellt utvecklad.

•

Lysdioder (LED).
Energi frigörs i spärrzonen i en diod som
är polariserad i passriktningen. Det sker
genom rekombination av par av laddningsbärare, varvid det normalt avgår
energi i form av värme.
Vid en viss inblandning av främmande
atomer avgår istället ljus. Spänningfallet
över en lysdiod är ungefär dubbelt så
stort som över en kiseldiod, d.v.s. ungefär 1.5 volt. Strömmen är i proportion med
önskad ljusstyrka och mellan 1O och 50

mA.

•

o.s.v ..

Vakuumdioden jämfört halvledardioden

Bild II 2-15
Bilden visar principen för hur de båda diodtyperna ingår i en strömkrets. Den stora
skillnaden är att arbetsspänningen för en
vakuumdiod är mångfalt högre än den för en
halvledardiod samt att vakuumdiodens en a
elektrod (katoden) behöver hettas upp för att
avge elektroner.

R

PASSRIKTNING

R

R

SPÄRRIKTNING

R

Bild II 2-15 Dioders polarisering i kretsen
112-21

K M

112-22

Transistorn
Allmänt

En transistor består av skikt av halvledarelement som sammanfogats. Vanligt är två Nskikt och ett mellanliggande P-skikt (NPNtransistor) eller två P-skikt och ett mellanliggande N-skikt (PNP-transistor). Skikten är
försedda med anslutningar.

B~

...-----Kollektor
(C)
r

n
p

~

+ + ~-------Bas (B)

+ +
~

n
G

l... Emitter (E)

E

NPN

PNP

FET

Bild II 2-16 Schemasymboler

n

c

n

C

Vanliga transistortyper
N PN-transistorer (bipolära)
PNP-transistorer (bipolära)
FET-transistorer (fälteffekt-)

NPN-transistorer

Halvledarskikten kallas
E emitter
B bas
C kollektor

E BC

Spärrzonerna
Bild II 2-17 överst
Mellan skikten B och E respektive mellan B
och C bildas zoner, vars ledningsförmåga
kan styras elektriskt över anslutningarna.
Bild II 2-17 mitten
Spänningskällan U8 E
Mellan bas och e mitter finns en diodsträcka.
När en positiv spänning läggs på basen och
en negativ spänning på emittern, så polariseras diodsträckans spärrzon i passriktningen. Spärrzonen upplöses då och det flyter en
s.k. basström 16 •

p~B
+ +
n

+

E

Bild II 2-17 Skikten i en bipolär transistor

112-23

K MP NENTER
Bild II 2-17 nederst
Spänningskällan UeE

När en positiv spänning läggs på kollektorn
och en negativ spänning läggs på emittern,
så polariseras diodsträckan i spärriktningen.
Spärrzonen förstärks då och det flyter ingen
ström.
Bild 112-18
Inverkan av både U8 Eoch UeE

Två spänningskällor U8 Eoch UcE ansluts till
en emitterkopplad NPN-transistor.
Ur den starkt dopade emitterzonen strömmar elektronerna in i den svagt dopade
baszonen (spänning: U8 E). De flesta elektronerna blir emellertid inte kvar i basen. De
stöter igenom det tunna basskiktet och når
fram till kollektorskiktet med spänningen U E.
Det flyter en kollektorström.
c
För strömmer l (emitterström), 18 (basström) och le (kollektorström) gäller:
lE= Is+ le där 18 <<le

(<<mycket mindre än)

Kollektorströmmen le kan styras med basspänningen U8 E.
En liten ändring i basspänningen ger stor
förstärkande verkan i kollektorströmmen.

n

c

p + + B
+ +

n

h
FE

hFE
IIie

!ll 8

=IIie
/:lf
B

strömförstärkningsfaktorn
ändringen i kollektorströmmen
ändringen i basströmmen

PNP-transistorer
Ersätter man de två N-skikten i en NPNtransistor med P-skikt och P-skiktet med ett
N-skikt så erhåller man en PNP-transistor.
Uppbyggnad, koppling och användning
av en PNP-transistor motsvarar i övrigt den
för en NPN-transistor. Spänningskällorna
måste emellertid ha motsatt polaritet.

R

Is

R

UcE

E
..,..
lE

la<< lE

lE

=Is+ le

Bild II 2-18 Emitterkopplad transistor

112-24

Förstärkningsfaktor
Om strömmen i ingångskretsen för en transistor ändras, så kan strömmen i utgångskretsen ändras mer. Det blir då en förstärkning.
Av sambandet le= f(/8 ) framgår strömförstärkningsfaktorn ~ eller hFE' som är kvoten av ändringen i utgångsströmmen och i
ingångsströmmen i transistorns aktiva (linjära) område.
Bild II 2-19
För emitterkoppling gäller:

UsE

UcE

l c (mA)

100

5

10

UcE ::0 V

UcE = 5 V

UsE {mV)

Bild II 2-19 Karaktäristika för transistor BC 107

112-25

K
Fälteffekttransistorer

Allmänt
Fälteffekttransistorer (förkortat FET) har
mycket hög ingångsimpedans och styrströmmen blir därför mycket svag. Man säger
därför att en FET är spänningsstyrd.
Även NPN- och PNP-transistorer- kallade bipolära transistorer - styrs med spänning, men dessa typer har en relativt lågt
ingångsimpedans och därför högre styrströ m.
Man säger därför att de är strömstyrda.

D+

sBild II 2-20 Schemasymbol för en FET

Bild 112-21 Skikten i en N-kanal FET

Fälteffekttransistorn har tre anslutningar
(elektroder)
S source (katod)
D drain (anod)
G gate (grind, styre)
Fälteffekttransistorns uppbyggnad
Bild II 2-21
Bilden visar ett N-ledande skikt (även kallat
N-kanal) med elektroderna S och D anslutna
till respektive ändar av skiktet. N-kanalen
passerar mellan två P-ledande skikt förbundna med styrelektraden G.
När en spärrspänning läggs mellan G
och S, så breder spärrskikten ut sig och Nkanalen blir trängre. Läggs en negativ spänning på S och en positiv spänning på D, så
kommer det att flyta en ström i N-kanalen.
Strömmens styrka kan påverkas med spänningen på G.
En liten spänningsändring llU medför
stor ändring av strömmen lll i tf-kanalen.
Detta innebär förstärkning. os

Bild II 2-22
l en MOS-FET är G-elektroden isolerad med
ett kiseloxidskikt Funktionssättet är samma
som för en FET. Drain-strömmen kan ökas
eller minskas med hjälp av en positiv respektive negativ spänning på G.

112-26

Bild 112-22 Skikten i en N-kanal MOS-FET
Resistansen mellan gate och source
För att erhålla en förstärkning med en FETtransistor, sätter man in en resister R0 i
drain-strömkretsen. Över resistorn uppstår
då spänningsändringar i proportion med
strömändringarna.
För att fastställa vilaströmmen och därmed arbetspunkten för samma transistor
sätter man in en resister R i source-strömkretsen. storleken på soufce-resistorn ger
sig av önskad gate-förspänning -U 88 .
 -UGs
R s-

lo

K MP NENTER
Sambandet drain-ström och spänning
Bild 112-23
För att beskriva en FET använder man sig
av karaktäristiska kurvor. Vi har redan presenterat bipolära transistorers in- och utgångsegenskaper i kuNform. Eftersom ingångsströmmen (gateströmmen) i en FET
är praktiskt taget noll, så är en sådan kuNa
utan praktisk mening. l stället framställer
man grafiskt sammanhanget mellan styrspänningen UGs och utgångsströmmen (drainströmmen 10 ). Eftersom det finns N-kanal
FET och P-kanal FET så skiljer polariteten
på UGs för dessa båda typer.

Bild II 2-23 Karaktäristikför N-kanal FET

112-27

p

112-28

K

P NENTER

2. 7 Elektronrör
Allmänt
Ett elektronrör består av två eller flera elektroder i en lufttom glaskolv.

Direktupphettad
katod

Indirekt upphettad katod

Allmän
symbol

Bild II 2-24 Schemasymboler för dioder

Vakuumdioden (två.elektrodröret)
Bild 112-24
Dioden innehåller två elektroder
a anod
k katod, med f f= glödtråd (filament)
Anoden skall dra elektronerna från katoden.
Katoden skall avge elektronerna och måste
därför hettas upp.
Upphettningen av katoden görs på något
av följande sätt:
Direkt upphettning, d.v.s. katoden är i sig
själv en glödtråd. En 4- till 6-volts strömkälla
är vanligt.
Indirekt upphettning, d.v.s. en glödtråd
omsluter och hettar upp ett speciellt katodmaterial. En 1.5 till12.6 volts glödströmkälla
är vanligt.
Ed i soneffekten
Bild 112-25
När katoden upphettas lossnar fria elektroner från den och bildar ett moln. Med en
spänning mellan anod och katod, med
anoden positiv, så kommer elektronerna att
dras mot anoden. En s.k. anodström börjar
att flyta.

uh

la/Ua .. karaktäristikan för en vakuumdiod
Bild II 2-26
Bild II 2-25 Edisoneffekten
När anoden ges positiv potential (anodspänning), flyter en elektronström från katod
la l Ua- karaktäristik
till anod (anodström).
la
Om anodspänningen
ua ökar' så ökar anodströmmen la. Varje
par av talvärden representerar en punkt
i ett diagram, som det
på bilden. När anodspänningen ökattill ett
Ua
visst värde, så ökar
(
Al B
inte anodströmmen
l
ytterligare.
l ett melA: Initialströmsområde
lanområde är kurvan
B: Den linjära delen
i det närmaste rak.
C: Mättnadsområde
Bild II 2-26 Diodens karaktäristik

112-29

P NENTER
likriktarverkan

När anoden i en vakuumdiod ges positiv
potential i förhållande till katoden, flyter en
s.k. anodström förutsatt att katoden upphettas så att den avger fria elektroner.
När anoden ges en negativ potential i
förhållande till katoden flyter däremot ingen
anodströ m.
Vakuumdioden kan därför användas för
likriktning av växelströmmar. Den har en
likriktande funktion.

En anodström flyter

Halwågslikriktning.
Bild 112-27
När anoden ges en omväxlande positiv och
negativ potential, en växelspänning, så flyter
anodström under varje positiv halvperiod av
växelspänningen. En likströmspuls uppstår
under varannan halvperiod.
Helvågslikriktning.
Bild 112-28
Med ett elektronrör med dubbla anoder och
en transformator med mittuttag på sekundärlindningen, kan växelspänningens båda
halvperioder utnyttjas så, att anodström flyter i samma riktning under alla halvperioder.

Det flyter ingen ström

En pulserande likström flyter

Bild II 2-27 Halwågslikriktning

1 :a halvvågen

Bild II 2-28 Helvågslikriktning

112-30

2 :a halvvågen

K MP NENTER
Ua
VÄXELSPÄNNING
PAANODERNA

HALVVAGsLIKRIKTNING

laf

!@.

f@.

~

HELVAGsLIKRIKTNING

r
F

Bild II 2-29 Likriktande funktion

Vakuumtrioden (treelektrodröret)

Triodens funktion

g2

TRIOD

PENTOD

Bild II 2-30 Symboler för triod och pentod
Trioden innehåller tre elektroder
a anod
g 1 styrgaller
k katod, med f f= glödtråd (filament)

Det flyter både anod- och gallerström

Bild II 2-31
En triod fungerar som en diod, när styrgallret
ges samma potential som katoden. Valet av
förspänning avgör triodens arbetssätt. styrgallret kan ges positiv, neutral eller negativ
potential (förspänning) i förhållande till katoden. Med styrgallret positivt ökar anodströmmen. Med gallret negativt minskar den.
Trioden har en förstärkande funktion eftersom anodströmmen kan styras med styrgallret. En liten ändring av gallerspänningen
medför stor ändring av anodströmmen. Vid
positiv förspänning flyter det gallerström,
som inte får bli för hög. Vanligen väljs en
negativ förspänning.

Det flyter anodström men ingen gallerström

Bild II 2-31 Elektronstömmen i en triod

112- 31

K MP NENTE
Triodens strömkretsar och strömkällor

Glödströmskrets Anodkrets Gallerkrets
Glödbatteri
Anodbatteri Gallerbatteri
Glödspänning Ut Anodsp. Ua Gallersp. U91
Glödström lt
Anodstr. la Gallerstr. 191
Vanligen används nätdrivna strömkällor i
stället för batterier.
Valet av gallerförspänning är avgörande för
triodens arbetssätt.

Tetroden (fyraelektrodröret)

Denna rörtyp innehållerfyra elektroder. Uppbyggnaden är densamma som pentodens,
men bromsgallret saknas.

Pentoden (femelektrodröret)

Pentaden innehåller fem elektroder.
a
anod
g3
bromsgaller
g2
skärmgaller
styrgaller
g1
k
katod, med f f = glödtråd (filament)
Bromsgallret förbinds med katoden. Skärmgallret ges en potential som är något lägre än
anodspänningen. Broms- och skärmgallren
förhindrar elektronerna att studsa tillbaka till
styrgallret efter anslaget mot anoden.

- - - - - - · - - - Ua.
Bild II 2-32 Karaktäristika för elektronrör

112-32

Karaktäristika för elektronrör
Bild II 2-32

1iU91 -diagram för en triod eller pentod, vid Ua
=konstant
liUa- diagram för en triod, vid U~ 1 =konstant
liUa - diagram för en pentoa, vid U91 =
konstant
Tre kurvor visas i VUa- diagrammen, med
olika värden på U91 = konstant (U 91 är s.k.
parameter).

la

l ug1 - karaktäristik för en triod eller pentod

la

l

18

l U8 - karaktäristik för en pentod

ua- karaktäristik för en triod

Branthet S och inre resistans Ri
Bild 112-33

•

Om man (vid konstant anodspänning)
ändrar gallerförspänningen med värdet
ilU 91 så ändrar sig anodströmmen med
värdet illa.

la

il/
Branthet S = a
ilU91
S [mA/V]

illa [mA]

ilU 91 [V]

Bild 112-34
• Om man (vid konstantgallerförspänning)
ändraranodspänningen medilUasåändras anodströmmen med värdet illa
Inre resistans
Ri [kQ]
•

BRANTHET

R.= ilUa
' illa
ilUa [V]

3mA

Om man vill ändra anodströmmen med
ett yärde illa , så ges det två möjligheter:
- Andra gallerförspänningen med värqet ilU 91
- Andra anodspänningen med värdet
il Ua
Med ändring av gallerförspänningen
med värdet ilU 1 kan man åstadkommasamma anodströmsändring illa som
med en ändring av anodspänningen
med värdet ilUa.

:::;. ... ,...J....J..---Ug1
-4 V
-2V

Bild II 2-33 Branthet

9mA

INRE MOTSTAND Rl
8m A

R;

=

v

10
1 mA

v

10
0,001 A

: : - = - - - = 10000

A

:: 10 000 1l

Bild II 2-34 Inre resistans

112-33

K MP NENTER

PT

Barkhausen's elektronrörsformler
Förstärkningsfaktorn J.l
Följande samband gäller mellan de s.k. rörkonstanterna

J1

= S·Ri

Exempel:
Beräkna~,

om S = 2 mA/V
Svar:~=

20

R = 1o kQ
(~är

~

=?

dimensionslös)

Transistorer jämfört med elektronrör
Transistorer
Fördelar:
Lågt pris- små dimensioner -lång livslängd
-enkel strömförsörjning (g lödström behövs
inte) -låg driftspänning (6V, 12V ...... ).
Nackdelar:
Känslighet för överbelastning och höga
temperaturer.
Elektronrör
Fördelar:
Tålighet mot överbelastning
Nackdelar:
Behöver hög anodspänning
Behöver glödström
Utrymmeskrävande

Ett användningsområde, där elektronrör
ännu är vanliga, är i större sändarslutsteg.
Transistorer ersätter numera nästan helt
elektronrören, men man bör ändå känna
elektronrörens egenskaper och arbetssätt.

112-34

~©~

EPT

2.8 Digitala kretsar
Särskilt under det senaste decenniet har
digital elektronik blivit vanlig i utrustningar
för radio- och telekommunikation. Även inom
amatörradio används nu denna teknik. Ämnet är mycket omfattande. Utvecklingen är
att enkla digitala funktioner snabbt ersätts av
komplexa datasystem. Här redogörs endast
för några grundläggande digitala funktioner.
Amatörradions kärna finns ännu i analogtekniken, där det under ett förlopp kan förekomma många olika storheter, t.ex. spänningar mellan noll och ett högsta värde.
l digitaltekniken förekommer bara ett bestämt antal tillstånd. l det enklaste digitala
systemet finns två tillstånd, t. ex. Ooch 1 eller
Till och Från eller Hög och Låg eller Fel och
Rätt o.s.v. Ett system med två tillstånd kallas
binärt. En lampa som tänds eller släcks med
en enkel strömställare är ett binärt system.
Strömställaren kan ha olika utföranden. Det
kan vara en mekanisk kontakt som är styrd
för hand eller av en reläspol e. Det kan också
vara en transistor eller annan anordning.

Transistorn som strömställare
Bild II 2-3S

Bilden visar två transistorkopplingar. Den till
vänster är en analog förstärkare för växelspänning. Om det på grund av en viss basspänning flyter en kollektorström av 1 mA
och kollektorresistorn är S n, så blir spänningsfallet över den resistorn SV. Eftersom
matningsspänningen är 12 V, så blir då
spänningen 7V mellan kollektorn och minus.

Kopplingen till höger fungerar som en
binär strömställare. Antag att insignalen intar ett avtvå spänningstillstånd, antingen OV
(låg) eller SV (hög). När inspänningen är
t. ex. SV, så flyter så mycket basström genom
basresistorns 1O k.Q, att transistorn blir fullt
utstyrd.
Därmed är spänningen mellan kollektor
och emitter, d.v.s. utspänningen, nära O V
(0.1 till 0.2V beroende på transistortyp). Man
säger då att utgången är låg (L) eller O(noll).
Om däremot inspänningen ärO V, såspärras kollektorströmmen och utspänningen blir
nära SV. Man säger då att utgången är hög
(H) eller 1.
För NPN-transistorn i bilden, gäller att
• hög inspänning ger låg utspänning,
• låg inspänning ger hög utspänning.
Denna logiska funktion kallas inverterande.

NOT-gate eller inverterande grind
Bild 112-36

Logiska funktioner beskrivs med internationella symboler. En ring vid utgången betyder att utspänningens nivå är motsatt inspänningens. Sambandet mellan in- och utnivåerna beskrivs med en sanningstabe/1.

m
1

o

Bild 1/2-36 NOT-gate

+12V

r
l

Bild II 2-35 Transistorn som analog förstärkare respektive digital strömställare

112- 3S

p

K

Villkorskretsar- s. k. grindar

Det finns olika sätt att bygga grindar. idag är
de flesta grindarna elektroniska lösningar.
Därutöver finns elektromekaniska grindar i
form av strömbrytare och reläkontakter.
Föregångarna till de elektroniska televäxlarna (AXE m.fl.) var stora system av
mestadels elektromekaniska reläer.
För att överskådligt förklara arbetssättet
i de vanligaste grindarna, görs det enklast
med reläsymboler. En reläkontakt kan då
motsvara en transistor eller diod. Reläspolar
kan motsvara logiska nivåer i insignaler.
Elektriska kontakter kan vara normalt
öppna och sluter vid påverkan (s.k. slutande
kontakt). Alternativt kan de vara normalt
slutna och öppnar vid påverkan (s.k. brytande kontakt). l kretsscheman visas kontaktlägena vid systemet i vila.
Bild 112-37
Av bilden framgår att samma villkor kan
skapas med slutande alternativt brytande
kontakter. Observera då placeringen av resistorn på kretsens utgångssida i respektive
fall. När resistorn ligger närmast pluspolen
kallas den pull-up. När den ligger närmast
minuspolen kallas den pull-down. l båda
fallen definierar resistorn den logiska nivån

Bild
Sanningstabellen i bilden säger, att när alla
insignaler är 1 så är utsignalen också 1.

ELLER-grind eller OR-gate
Bild 112-38

Sanningstabellen säger, att när en ellerflera
av insignalerna är 1 så är utsignalen också
1. När alla insignaler är O, så är utsignalen O.

OCH INTE-grind
Bild 112-39

NANO-gate

Sanningstabellen säger, att när ingen eller
någon insignal är 1, men inte alla, så är
utsignalen 1 . När alla insignaler är 1, så är
utsignalen O.

INTE ELLER-grind eller NOR-gate
Bild 112-40

Sanningstabellen säger, att när någon eller
alla insignaler är 1, så är utsignalen O. När
alla insignaler är O, så är utsignalen 1.

c
c

A

l
l
H
H

B

c

A

l

l
l
l
H

o o o
o 1 o
1
o o

H

l
H

1

B

1

c

1

Bild II 2-37 OCH-grind (AND-gate)

112-36

A
B

c

MP NENTER
+

ATBT-

R

c
A

B

A

B

c

L
L
H

L

H

L
H

H

H

H
H

L

c
o o o
o 1 1
1 o 1
A

B

1

1

1

c

~fic

Bild 112-38 ELLER-grind (OR-gate)

c
A

B

c

A

B

c

L
L
H

L

H

H

H

1
1
1

H

H

L

o o
o 1
1
o

L

H

1

1

o

~fic
A~
B~c

Bild 112-39 OCH INTE-grind (NAND-gate)

112-37

PT

K MP

A

c
B

A

c

R

B

A

B

c

A

L
L

H

L

H

L

L
L
L

o o 1
o 1 o
1 o o
1
1 o

H
H

H

B

c

Bild II 2-40 INTE ELLER-grind (NOR-gate)

Inverterad ingång
En ingång kan behöva ha en inverterad
funktion i förhållande till de övriga (s.k. low
active). Man kan då göra på följande sätt
med en OCH-grind som exempel.

A

c

A
B

c
c

A

B

L
L

H

H

L

L
L

H
H

L

H

L

c
o o o
o 1 1
1 o o
1 1 o

A

B

Bild II 2-41 Inverterad ingång

112-38

A
B
A
B

c

R

A

B

c

c

A

L

B

L

C

A

L

o o o

L

H

H

H
H

L
H

H
L

o
1
1

B

1

o
1

C

1
1

o

A

B

A

B

C

L
H

H
L

L
L

H

o o

1

H

H

H

1
1

1

L

L

A
B

Bild II 2-42 Exklusiv ELLER-grind

C

o

o
o o
1

1

c

(EXOR-gate)

Bild II 2-43 Exklusiv INTE ELLER-grind
(EXNOR-gate)

Exklusiv ELLER-grind (EXOR-gate)
Bild II 2-42

Exklusiv INTE ELLER-grind (EXNOR·gate)
Bild 112-43

Sanningstabellen säger, att när alla insignaler antingen är 1 eller O, så är utsignalen O.
När en, men inte alla insignaler är i, så är
utsignalen 1.

Sanningstabellen säger, attnär alla insignaler
antingen är 1 eller O, så är utsignalen 1. När
en, men inte alla insignaler är 1, så är utsignalen O.

112-39

K MP N
Grindar med dioder och transistorer
l stället för reläer i grindar använder man nu

ytterst sällan något annat än kombinationer
av dioder, transistorer och resistorer.
Bild II 2-44
Bilden visar en NANO-grind. Den egentliga
grinden består av tre dioder och en resistor.
Två av dioderna är ingångar och den tredje
är utgång. Grinden styr en digitalt arbetande
transistor liksom den i bild Il 2-35. Resultatet
är en s.k. DTL-Iogik.

Bild II 2-45
Även denna bild visar en NAND-grind. Här
består den egentliga grinden av en ingångstransistor med två emittrar, vilka motsvarar
dioderna vid A och B i föregående bild.
Kollektorn i denna transistor motsvarar ingångsdioden till transistorn i bild Il 2-44.
De övriga tre transisitorerna i bild Il 2-45
bildar en s.k. switch, som ger snabb övergång mellan väl definierade logiska nivåer.
Resultatet är en s.k. TTL-Iogik

A

B

A
B

Bild 112-44 DTL-Iogik

Bild 112-45 TTL-Iogik

112-40

ENTE
2.9 Integrerade
Allmänt om IC
Att integrera betyder att samla till en enhet,
det kan vara komponenter, funktioner, verksamheter etc. Integration kan ske på olika
nivåer och i många olika sammanhang.
Med integration avses här komponenter
för elektroniska strömkretsar. Särskilt halvledarelement av olika slag samt resistorer
och kondensatorer med små värden kan
framställas med små dimensioner. Många
komponenter kan då samlas i samma hölje.
Komponenter inom ett hölje, avsedda för
en viss funktion kallas integrerad krets (eng.
lntegrated Circuit -/C).
Komponenterna i en IC kan i sin tur vara
del av komponenterna en hel strömkrets.
Redan inom höljet kan komponenter kopplas samman för en viss funktion eller som en
del av strömkretsen. Skrymmande eller effektkrävande komponenter, såsom induktorer, transformatorer o.s.v. får emellertid inte
plats, varför även yttre kopplingar behövs.
Det kan också behövas flera IC i en strömkrets - kanske med innehåll för en annan
funktion.
Integrationsgrad
En integrerad krets är uppbyggd på en basplatta av halvledarmaterial - ett chip. På
plattan framställs, med fototeknik eller etsning, kompletta eller nästan kompletta dioder, transistorer, resistorer och kondensatorer. Metoden, som kallas planarteknik, medger att många komponenter kan få plats på
samma platta.
Den snabba utvecklingen av produktionsmetoder för integrerade kretsar gör alltmer
avancerade system möjliga och dessutom
på allt mindre utrymme. Med avseende på
integrationsgrad används följande begrepp.
SSI

Small Scale Integration innebär något
1O-tal halvledare på samma ch ip.
MSI Medium Scale Integration innebär
något 100-tal halvledare på ett ch ip.
LSI Large Scale integration innebär något
10000-tal halvledare på ett ch ip.
VLSI Very Large Scale Integration innebär
100000 eller fler halvledare.

Olika slags integrerade kretsar
Det finns stora sortiment av både standardiserade och speciella IC, varav det finns två
huvudtyper:
e digitala integrerade kretsar,
"'analoga integrerade kretsar.

Digitala IC

Digitala
arbetar som framgår av namnet
med digitala signalnivåer. De enklaste typerna innehåller en eller flera digitala grindar
(se avsnitt 2.8). Genom att koppla samman
grindar kan man skapa kretsar för ett visst
ändamål. i början av 70-talet byggdes komplicerade system av grindar i SSI- och MSIteknik. Ett sådant system är emellertid inte
flexibelt eftersom eventuella ändringar måste göras "hårdvarumässigt". Det innebär att
kopplingsledningar måste ändras om, kanske hela kretsar bytas ut o.s.v ..
l dagens digitala system används IC i
form av en mikroprocessoreller t.o.m. flera.
En mikroprocessor är en avancerad LSIkrets, som kan programmeras (kopplas upp)
"mjukvarumässigt" inte bara för ett ändamål
utan för många olika. l system med mikroprocessorer behövs också minnesfunktioner. Sådana kan också samlas i LSI-kretsar.
Mikroprocessorn är hjärtat i en dator. Styrd
av ett program (mjukvaran) styr den kringutrustningar med uppgift att inhämta och avge
information - att kommunicera.

Analoga IC

Analoga IC arbetar med analoga signalnivåer, d.v.s. spänningar och strömmar med
många olika nivåer och frekvenser. En analog IC kan därför även arbeta med digitala
signaler.
Analoga IC innehåller en balanserad förstärkare eller flera samt olika slags hjälpkretsar. Med yttre komponenter kan en analog IC ges olika förstärkning och frekvensgång. Gemensamt namn för dessa förstärkare är operationsförstärkare (OP-amp).
CP-förstärkare utförs vanligen i SSI- eller
möjligen MSI-teknik.

112-41

K

N

Kombinerade och speciella IC

Utöver standardiserade IC finns kombinerade och speciella IC.
Exmpel på speciella digitala IC är sådana
för telekommunikationsändamåL
Ett annat exempel på digitala IC är sådana för signalbehandling, såväl på HFsom LF-nivå
Exempel på speciella analoga !C är sådana för radiokommunikationsändamåL
Bortsett från vissa skrymmande komponenter och manöverdonen kan numera t. ex. en
IC innehålla en komplett radiomottagare.
Ett annat exempel på speciella analoga
IC är sådana för hörapparater. Genom programmering anpassas de för det personliga
behovet.

Utvecklingen
Det kan sägas hur ofta som helst. Genom
den fantastiska utvecklingen av mikroelektronik öppnas även för radioamatören möjligheter, som bara för ett par decennier inte
var tänkbart.
Denna utveckling harvidgat utrymmet för
den experimentella verksamhetsom amatörradio i grunden innebär. Hobbyn får sålunda
med tiden en allt större teknisk spännvidd.
Aktuell litteratur
Ökat teknikomfång inom amatörradio ställer
motsvarande krav på litteratur. På senare tid
inbegripes även digitalteknik.
Mest av utrymmesskäl behandlas i denna
faktabok digitaltekniken mycket kortfattat,
men ändå så mycket som nämns i CEPTrekommendationen T/R 61-02. För djupare
studium hänvisas till andra läromedel samt
tillleverantörskataloger.

112-42


\chapter{KRETSAR}

\section{Komponenter i serie}

parallellt

Seriekopplade resistorer
Bild II 3-1
Den totala resistansen av seriekopplade
resistorer är summan av resistanserna

1

1

1

1

R=R-1 +R-2 +R-3 ... Rn
För två resistorer gäller

1
1
1
eller
R R1 R2
För tre resistorer gäller

-=-+-

Strömmen är lika stor genom alla seriekopplade resistorer i strömvägen (ingen avgrening)

l= ~ + /2 + /3 + .....
Det totala spänningen över seriekopplade
resistorer är summan av spänningen över
var och en av dem

U=U1 +U2 +U3 + .....
Spänningen över var och en av seriekopplade
resistorerförhåller sig som deras resistanse r.
För två resistorer gäller

R1

u1
u2

1

1

1

R=------R1~·-R~2·R~3~---R1·R2 + R1·R3 + R2 ·R3
Strömmen förgrenar sig mellan parallellkopplade resistare r. Den totala strömmen är
summan av grenströmmarna
(Kirchhoff's 1 :a lag)
Spänningen är lika stor över var och en av
parallellkopplade resistorer

u= u1 = u2 = u3 =

-=-

R2

.. .

un

(Kirchhoff's 2:a lag)

Parallellkopplade resistorer
Bild II 3-2
Den totala resistansen av parallellkopplade
resistorer är lägre än den lägsta enstaka
resistansen

Grenströmmarna genom parallellkopplade
resistorer fördelar sig omvänt proportionellt
till deras respektive resistanser.
För två resistorer gäller

l

12

R2
R1

,

(r

1

- = - + - + - eller
R R1 R2 R3

-l

11

+

l '
lz

l,....

R1

u1

:r2
l

Bild 1/3-1 Seriekopplade resistorer

u

Gu

~

,,

u1[ R,

,.....

l

1

z

RZ

J11

JUz

J 'z

Bild II 3-2 Parallellkopplade resistorer
113-1

KRETSAR
Spänningsdelare
Bild II 3-3
Spänningsdelare förekommer i flera former.
Bilden visar en spänningsdelare med
resistorer där spänningen U delas upp i
spänningen U1 över resistorn R1 respektive
U2 över R2 • Man kan då t. ex. använda spänningen
för något ändamål.
Ett alternativ till spänningsdelning med
resistorer med fasta värden är patentiometern Det är en variabel spänningsdelare i
form av en resister med ett uttag som kan
flyttas mellan ändanslutningarna.
Om man nu ansluter en apparat parallellt
över R2 , t. ex. ett instrument vars inre resistans motsvaras av Ry, så kommer spänningarna över R1 och R2 att påverkas.
Om Ry är mycket större än R2 , så kan
man bortse från påverkan. För att beräkna
U2 kan man då använda följande formel för
en obelastad resistiv spänningsdelare.

.....l

u1[ R1

~l

u2

U2 =
U
eller
U2 =U·----"--R2 R1 +R2
+R2
Om Ry däremot är av samma storleksordning eller lägre än R2 , så måste man för
att beräkna U2 använda en formel för en
belastad resistiv spänningsdelare, t. ex.
R2 ·Rr
U2 =U·

u
~ lz

21

U

1.,...

~ly
Ry

R2

~ lz

+ly

Bild II 3-3 Resistiv spänningsdelare
Det
ström mellan X och Y när det
finns en potentialskillnad- spänning- däremellan. Bryggan är då i obalans.
Det flyter däremot ingen ström där när
det inte finns en potentialskillnad, d.v.s. när
bryggan är i balans. Balans (mätvärdet) får
man genom justering av den graderade
potentlometern till noll ström. Då gäller sambandet

R2 +Rr
R2 ·Rr
R1+--=-R2+Rr

Härav förstås att t. ex. en spänningsmätning ger olika resultat beroende på den inre
resistansen i voltmetern.

Wheatstone's brygga
Bild 113-4
En speciell tillämpning av spänningsdelare
är en s.k. brygga (Wheatstone's brygga),
som används för att jämföra spänningar.
Bryggan kan ses som två parallellkopplade spänningsdelare varav den ena är en
potentiometer med en skala graderad t. ex.
i n. Den andra spänningsdelaren består av
en resister med känd resistans och en resisto r med okänd resistans, d.v.s. mätobjektet
l ledningen som förbinder de respektive mittuttagen X och Y, finns en amperemeter som
nollströmsindikator.

113-2

JUz

u

Bild 113-4 Wheatstone's brygga

KRETSAR
Spänningsdelare och bryggor har tagits
med för att påvisa att apparater påverkar
varandra när de kopplas samman, vilket är
fallet även vid mätningar.
Spänningsdelning kan även utföras med
kondensatorer och induktorer förutsatt att
det är fråga om en växelströmskrets.
Parallellkopplade kondensatorer
Bild II 3-5
l stället för en enda kondensator kan man
parallellkoppla flera kondensatorer för att
uppnå önskad total kapacitans.
Den totala kapacitansen för parallellkopplade kondensatorer är summan av de
enskilda kapacitanserna.
C=C1 +C2 +C3 ... Cn

Seriekopplade kondensatorer
Bild II 3-6
Den totala kapacitansen för seriekopplade
kondensatorer är lägre än kapacitansen för
kondensatorn med det minsta värdet.

1

C1 = 5 11 F C2 = 1o 11F
C = C1 + C2 = 5 + 1O= 15 J..LF

2.

C1 = 1 nF

eller
C= C1. C2
C1 c2
C1+C2
För tre kapacitanser gäller

c
1

1

C=

1

c1c2c3

C1C2 + C1C3 + C2C3

C1 = 5 J.tF

C2 = 5 pF

C2 = 1o11 F

1
1
1
-=-+c1 c2

C= 5·10 =3! F
5+10
3 Jl

c

+
+

+
+

+

+
+

+

+

+

+
l+

1

-=-+-+-eller
C C1 C2 C3

+

+

1

Räkneexempel:

C= C1 + C2 = 1+ 0.005 = 1.005 nF

+

1

! = 1 +-1-

Räkneexempel:

1.

1

C = C + C + C + . . . . . .. . . ;
2
3
1
För två kapacitanser gäller

+ +l

(1

l+ +

+l

(z

Bild II 3-5 Parallellkopplade kondensatare

+

-· (1

+

Bild II 3-6 Seriekopplade kondensatorer

113-3

KRETSAR
Sammankopplade induktorer
Galvaniskt kopplade induktorer

Induktansvärdetför galvanisktsammankopplade induktorer kan i princip beräknas på
samma sätt som för motsvarande sammankoppling av resistorer.

Galvaniskt seriekopplade induktorer
Förutsatt att magnetfälten från de respektive
induktorerna inte återverkar på varandra d.v.s. inte "kopplar magnetiskt till varandra"
-så gäller:

L= L1 + L2 + L3 + . . . Ln
Räkneexempel:
L1 = 20 mH L2 =50 mH L= ?
L = L 1 + L2 = 20 + 50 = 70 mH

Galvaniskt parallellkopplade induktorer
Förutsatt att magnetfälten från de respektive
induktorerna inte återverkar på varandrad.v.s. inte "kopplar magnetiskt till varandra"
-så gäller:
1
1
1
1
- = - + - + ... -

L

L1

L2

Ln

För två induktorer gäller:
1
:! = ! + eller
L = L; · L2
L L1 L2
L1 +L2
Räkneexempel:
L1 = 50 mH L2 = 60 mH L = ?

Medverkande magnetfält

L1

-

L2

~:~-'Olflf;Motverkande magnetfält

L1

L2

~~M~

-

Bild 113-7 Magnetiskt kopplade induktorer
Bild 113-7
Bilden visar seriekopplade induktorer, vars
magnetfält kopplar till varandra på olika sätt.
"Pricken" vid änden av induktorerna på
bilden markeraratt magnetfälten där har inbördes polarisering.

Magnetiskt kopplade induktorer i serie
Formel:
L=L 1 +L2 $\pm$2M
Räkneexempel:
Två induktorer har en impedans av 20 resp.
1O J..LH och en ömsesidig induktans av 2 JlH.
Induktorerna är kopplade och placerade så
att deras magnetfält verkar med varandra.
Vardera induktansen ökas därför med
M=2J..LH

L=L1 +M+L 2 +M

L= L1 • L2 = 50 . 60 = 3000 ~ 27 m H
L1 +L2
50+60
110

L = 20 + 2 + 1O + 2 J..LH = 34 J..LH

Magnetiskt kopplade induktorer

Räkneexempel:
Två induktorer har en impedans av 20 resp
1O !lH och en ömsesidig induktans av 2!-LH.
Induktorerna är kopplade och placerade så
att deras magnetfält verkar mot varandra.
Vardera induktansen minskas därför med
M=2J..LH

l praktiken anordnas ofta induktorer så, att
deras respektive magnetfält kan återverka
på varandra- s.k. magnetisk koppling.
En ömsesidig induktans M uppstår i
induktorerna på grund av denna koppling.
Den ömsesidiga induktansen ökar eller minskar det resulterande induktansvärdet beroende på om induktorernas magnetfältverkar
med eller mot varandra.
Beräkningen av värdet på "M" är emellertid relativt komplicerad och behandlas ej här.
l stället görs en förenklad framställning.
113-4

L= L1 -M+L 2 -M

L = 20 - 2 + 1O - 2

= 26 JlH

Magnetiskt kopplade induktorer i parallell
Formel:
L= L; ·L2 ·M 2
L1 +L 2 $\pm$2M

~©rNJ

EPT

KRETSAR

Upp- och urladdning av en kondensator

Uppladdning

Bild 113-8
En kondensator C seriekopplas med en resistans R och kopplas in över spänningen U.
Spänningen över kondensatorn stigerfrån
volt till umax
Laddningsströmmen sjunker från lmax till
noll ampere.

u

Uc

o

Spänningen över kondensatorn ökar
exponentiellt uppladdningen.

(1- e-~ J

uC = Umax ·

e

spänningen över kondensatorn efter
en given inkopplingstid
slutspänningen efter minst t= 5-r
inkopplingstiden
2. 718 (e = basen för den naturliga
logaritmen)

l förloppet ingår storleken av resistans
och kapacitans enligt följande samband, som
kallas tidskonstant
r= R· C
C [F] R [Q] s [sek] -r [tidskonstant i sek]
Efter tiden t= 1r från inkopplingsögonblicket har spänningen över kondensatorn
ökat från noll till 63$\circ$/o av maxvärdet
Efter tiden t= 5-r är kondensatorn uppladdad till 99 o/o.

Strömmen från kondensatorn minskar

exponentiellt under uppladdningen.

ic
lmax

strömmen från kondensatorn efter en
given inkopplingstid
begynnelseströmmen

Efter tiden t= 1r från inkopplingsögonblicket har strömmen till kondensatorn
minskat till 37$\circ$/o av maxvärdet
Efter tiden t= 5 r återstår 1 $\circ$/o av strömmens maxvärde.

Uc
100%---

le

U max

-~------:-:::::;..:;;:;;;-~---

o

2T

l max
100% - - - - - - - - - - - - - - - -

o
Bild If 3-8 Uppladdning av en kandensa to
Urladdning

Bild 113-9
En kondensator C urladdas över resistor R.

Spänningen över kondensatorn minskar
exponentiellt under urladdningen.
t

uC =Umax ·e--:r
Strömmen från kondensatorn minskar

exponentiellt under urladdningen. Strömriktningen är motsatt den vid uppladdningen.
t

iC =-lmax · e--:r
Efter tiden t= 1r är kondensatorn urladdad Så, att 37 $\circ$/o av lmax respektive umax
återstår.
Efter tiden t= 5 r är kondensatorn urladdad så, att mindre än i o/o av lmax respektive
umax återstår.

113-5

KRETSAR
Exempel på beräkning av tidskonstanten:
R = 1 kQ
1. C = 1O J.lF

r= R. c = 1·1 0 3 ·1 o·1 o-e = 1o·1 o-3

d.v.s. 1/100 sekund

2. C

= 1000 J.lF

1: =R·

In- och urkoppling av en induktor
Inkoppling
Bild II 3-1 O
En induktor L i serie med en resistans R
kopplas in över en likspänning U.
Spänningen över induktorn ökar från O till

u max·

R = 1 kQ

C= 10 3 ·1 0 3 ·1 o-e = 1 sekund

(Egentligen, induktorns motspänning
minskar så att. .. )
Strömmen genom induktorn ökar från O
till umax·

Strömmen genom induktorn ökar exponentiellt efter inkopplingen

iL

=lmax ·

(1- e-~ J

iL strömmen efter en given inkopplingstid
lmax slutströmmen efter minst t= 5-r

t

e

inkopplingstiden
2.718 (e = basen för den naturliga
logaritmen)

l förloppet ingår storleken av resistans
och induktans enligt följande samband, som
kallas tidskonstant

u

L
R

1:=-

L [H]

Uc

U max

le
Bild II 3-9 Urladdning av en kondensator

113-6

R [O]

s [sek]

1:

[tidskonstant]

Efter en tid av t= 11: från inkopplingsögonblicket har strömmen genom induktorn
ökat från noll till 63$\circ$/o av lmax och motspänningen över induktorn minskat till 37o/o
av maxvärdet

Urkoppling
Spänningskällan kopplas bort från samma
induktor som ovan. En resister är inkopplad
över induktorn. Energin i induktorn avleds
genom resistorn som en ström med motsatt
riktning än vid inkopplingen. Strömmen är
vid urkopplingstillfället lmax = iL och minskar
därefter exponentiellt.

ETSAR
iL
lmax

e

t

strömmen genom induktorn efter en
given urkopplingstid
strömmen i urkopplingsögonblicket

2.718

tiden efter urkopplingsögonblicket

Efter en tid av t= 1r från urkopplingsögonblicket har strömmen genom induktorn
minskat till 37$\circ$/o av maxvärdet
Teoretiskt kan spänningarna och strömmarna aldrig nå ett noll- eller maxvärd e, men
för praktiskt bruk anses detta inträffa efter en
tid av minst 61:.
All den energi som lagras i en induktor
finns i dess magnetfält. När strömmen bryts
eller minskas så återgår energin omedelbart
till kretsen. i en induktor kan det således inte
finnas någon kvarstående energi, vilket det
däremot kan göra i en kondensator.

Under den tid som magnetfältet i en induktor avvecklas eller byggs upp, så induceras en motspänning i den. Denna spänning
är högre än den som finns över induktorn
innan strömmen bryts eller ändras och är
proportionell till den hastighet som ändringen har. När en en strömkrets med induktor
bryts är det vanligt att det i brytögonblicket
bildas en gnista eller ljusbåge över brytarens
kontakter.
Om induktansen är stor och kretsströmmen hög, så skall en stor mängd energi
frigöras på mycket kort tid. Det är därför inte
ovanligt att brytarkontakter bränns eller smälter. l likströmskretsar kan gnistan eller ljusbågen minskas eller undertryckas genom att
en kondensator i serie med en resistor kopplas över kontaktstället Kondensatorn fångar
upp en del av energin i induktorn och resistorn minskar hastighetsändringen.

l~
L

u
'------{ /

1-----4-----l

lL

l Max

37:'

47:'

10~0--------------~~.

'(

Bild II 3-1 O Inkoppling av en induktor

113-7

KRETSAR

PT

Växelströmskretsar
Komponentegenskaper vid växelström
Inom radiotekniken används mycket ofta
svängningskretsar bestående av kondensatorer och induktorer, som är kopplade i serie
eller parallellt med varandra. När svängningskretsens egenfrekvens sätts lika med
frekvensen på den signal som tillförs kretsen, så får kretsen särskilda egenskaper
som används på olika sätt.
För att förstå hur "LC-kretsar" fungerar,
beskrivs först hur de ingående komponenternas resistans, induktans och kapacitans
förhåller sig till varandra, när de kombineras
och kopplas til en växelströmkälla.
Bild II 3-11
Bilden visar amplituden av spänning och
ström vid ett sinusformat förlopp samt den
effekt som då utvecklas. Tidsaxeln är graderad O - 360$\circ$ per period.

Fall a: Förloppen med en resister R.

Med en resister följer ström- och spänningskurvorna varandra tidsmässigt, även
vid riktningsändring. När kurvorna följs åt på
det sättet, sägs de vara i fas med varandra.
Effekt överförs från strömkällan till resistorn. Den effekt som utvecklas i resistorn är,
vid varje tidpunkt av perioden, produkten av
strömmmen och spänningen just då. Eftersom storheterna av spänning och ström är
antingen positiva eller negativa samtidigt, så
blir produkten alltid positiv. Det betyder att
den effekt som utvecklas pulserar två gånger per period mellan ett noll- och maxvärde.
Fall b: Förloppen med en induktor L.

Med en induktor är utvecklingen av ström
och spänning inte samtidig. Vid inkopplingen stiger spänningen genast till maxvärdet
medan strömmen stiger långsammare och
bygger under tiden upp ett magnetfält i induktorn och omkring övriga ledare i kretsen.

u~
a

u~

p

b

u~
c
Bild II 3-11 Faslägen och effekter i L C-kretsar

113-8

p

ETSAR
Strömmen fördröjs alltså i förhållande till
spänningen. Eftersom kurvornas max- och
nollvärden inträffar vid olika tidpunkter, så
heter det att de är ur fas eller fas förskjutna.
En växelström genom en ideal induktor
ärtasförskjuten 90$\circ$ efterspänn ingen. Strömmen når toppvärdet vid tidpunkten 90$\circ$ av
perioden, när spänningen nått ner till noll.
När spänningen minskar, så sjunker strömmen och tar med sig energin i magnetfältet.
Först vid 180$\circ$, när spänningen har nått maxvärdet åt andra hållet, ändrar också strömmen riktning och bygger upp ett nytt magnetfält med motsatt polaritet.
Effekt överförs från strömkällan till induktorn när ström och spänning har samma riktning. När ström och spänning har olika riktning, försöker induktorn i stället "ladda" strömkällan med energi från sitt kraftfält. Det pendlar därför effekt mellan strömkällan och induktorn, varvid effekten i ena riktningen är
lika stor som i andra riktningen.
Sett över en hel period upphäver därför
dessa effekter varandra. Följden blir att en
ideal induktor, i motsats till en resistor, inte
förbrukar någon aktiv effekt. Man säger att
en reaktans, här en induktor, arbetar med
reaktiv effekt.
l praktiken har kretsen även en viss resistans. Därför sätts reaktansens 90$\circ$ fasförskjutna ström samman med resistansens oo
fasförskjutna ström. Resultatet blir en ström,
som är mindre än 90$\circ$ ur fas, och det förbrukas då en viss aktiv effekt i resistansen.

Sedan strömmen passerat noll vid 180$\circ$ eller
0$\circ$/360$\circ$, bygger den upp ett nytt magnetfält
med motsatt polaritet.
Liksom med en induktor överförs effekt
från strömkällan till kondensatorn när ström
och spänning har samma riktning. När ström
och spänning har olika riktning, försöker
kondensatorn i stället "ladda" strömkällan
med energi. Det pendlar därför effekt mellan
strömkällan och kondensatorn, varvid effekten i ena riktningen är lika stor som i andra
riktningen.
Sett över en hel period upphäver därför
dessa effekter varandra. Följden blir att en
ideal kondensator, i motsats till en resistor,
inte förbrukar någon aktiv effekt. Man säger
då, att en reaktans, här en kondensator,
arbetar med reaktiv effekt.
l praktiken har kretsen även en viss resistans. Därför sätts reaktansens 90$\circ$ fasförskjutna spänning samman med resistansens o fasförskjutna ström. Resultatet blir en
spänning, som är mindre än 90$\circ$ ur fas, och
det förbrukas då en viss aktiv effekt i resistansen. Som framgår av bilden blir variationerna i tiden de omvända med kondensator
jämfört med induktor.

o

Fall c: Förloppen med en kondensator C.
Inte heller med en kondensator utvecklas
ström och spänning samtidig. Efter inkopplingen laddar strömmen upp kondensatorn,
d.v.s. bygger upp ett elektriskt fält med en
viss potential (spänning). Spänningen utvecklas långsammare än strömmen - den
blir fasförskjuten
Strömmen till (och från) en ideal kondensator är fasförskjuten 90$\circ$ före spänningen.
När kondensatorn är kopplad till en växelströmskälla, når strömmen toppvärdet vid
tidpunkten 90$\circ$ eller 270$\circ$ av perioden. Spänningen passerar då i båda fallen värdet noll.
När spänningen minskar, så sjunker strömmen och tar energi ur det elektriska fältet.

113-9

KRETSAR
Impedans

Bild II 3-12
Bilden visar en induktor, en kondensator
och en resistor som är kopplade i serie. När
man vill beräkna den resulterande impedansen i kretsen ("totala växelströmsmotståndet"), måste man ta hänsyn till att komponenternas spänningar eller strömmar inte är
i fas med varandra. De arbetar ju inte "i takt".
Att då addera max. värdena ger fel resultat. l stället söker man den s.k. resultanten
av de olika vektorer som motsvarar strömoch spänningsvärden.
Detta kan göras grafiskt eller beräknas.

Liten ordlista:
Impedans- hindra
(lat. impedire).
Resistans- motstå
(lat. resistere).
Del av impedansen,
kallas ibland ohmskt motstånd.
Reaktans- återverka
(lat. reagere).
Del av impedansen,
samlingsord för växelströmsmotstånd.
- Kapacitans- inrymma (lat. capax).
Del av reaktansen.
- Induktans- införa
(lat inducere).
Del av reaktansen.

Bild 113-13
Vi tänker oss att vektorerna i systemet
vrider sig moturs med hastigheten w= 2rc f,
där f är frekvensen och w ärvinkelhastighet.
Eftersom vektorerna har samma frekvens,
så är vektorernas lägen inbördes samma.
Ögonblicksvärdet av respektive vektorer följer en sinuskurva.
Spänningsvektorn i den "induktiva reaktansen" ligger 90$\circ$ före strömmen och spänningen i resistansen. Spänningsvektorn i
den "kapacitiva reaktansen" ligger 90$\circ$ efter
strömmen och spänningen i resistansen.
Vektorerna i dessa två reaktanser är således 2 · 90 = 180$\circ$ åtskilda, d.v.s. motriktade.
Det kallas att de är i mottas.

Hittills har storheterna resistans, induktans
och kapacitans behandlats var för sig, men
i praktiken förekommer de alltid tillsammans
och kallas impedans.
Resistansen är i princip oförändrad vid
ström- eller spänningsändringar. Men när
strömmen genom en ledare eller induktor
liksom spänningen över en kondensator ändras, så tillkommer en reaktans som motverkar förändringarna.
Reaktansen kan från fall till fall vara kapacitiv eller induktiv och ingår i impedansen.
Om ingen reaktans finns, så är impedansen
lika med resistansen.

Bild II 3-14
l bilden visas vektorerna för komponenterna i Bild II 3-12 samt hur man grafiskt
bestämmer inpedansen av dessa vektorer.
Vidare får man fasvinkeln mellan impedansens och resistansens vektor, varav den senare är den s.k. riktfasen för hela seriekretsen.

Bild II 3-12 Seriekrets av L+C+R

'

spänningsfall över R
= strammens fas 1 XL+ Xc +R

,spänningsfall over Xc.
spännir1gsfoll över XL
1

Bild II 3-13 Spänningar i seriekrets L+C+R

113-10

KRETSAR

Bild II 3-14 Impedansen och fasvinkeln i seriekrets L+C+R
Resistansen ritas som en vektor R, som
riktas vågrätt mot höger. Vektorns längd
motsvarar resistansens storhet i ohm.
Den induktivareaktansen ritas på liknande sätt med vektorn XL lodrätt uppåt. slutligen ritas den kapacitiva reaktansen Xc lodrätt neråt.
Man subtraherar de motverkande reaktiva vektorerna XL och Xc från varandra och
avsätter resultatet X på den vertikala axeln,
uppåt om XL är större och neråt om Xc är
större. Den resistiva vektorn R avsätts åt
höger på den horisontella axeln.
Man låter nu vektorerna X och R bilda
sidor i en rätvinklig rektangel. Längden på
rektangelns diagonal är den resulterande
impedansen Z. Fasvinkeln mellan impedans
och resistans kan också avläsas.
Eftersom vektordiagrammet bildar en rätvinklig triangel kan den resulterande spänningen U i kretsen även beräknas med
Pytagoras sats:

Tillämpad på ovanstående vektordiagram
kan satsen skrivas som

ULcR2

Termerna ersätts med följande ekvationer:

UR

= f. R

UL =J. XL =J. mL

1

Uc=f·Xc=l·-

mC

l' Z

2

=l

2

2

R + (hoL -l miG)'

eller

Z= ~R

'+(mL--dc;J eller

Z=~R

2
+(XL -Xct

l en seriekrets är den resulterande
reaktansen negativ (kapacitiv) om Xc är
större än XL och positiv (induktiv) om XL är
större än Xc.

Ohms lag vid växelström

l formler betecknas impedansen med bokstaven Z och reaktansen med bokstaven X.
l båda fallen är sorten Ohm [Q].
Vid beräkning av impedans är Ohms lag
inte direkt tillämplig, eftersom reaktansen i
en induktor eller kondensator uppträder annorlunda i tiden vid ström- respektive spänningsändring än vad resistansen gör.
Om impedansen Z sätts in i Ohms lag, så
fås följande samband som ofta kallas Ohms
lag för växelström, således

vett =lett.

= UR2 + (UL- Uc)2

ULRc =J. Z

z =R'+( mL- ~c)'

z

eller

Uett = lett. -.J R2 + )(2

eller

Uett=lett·~R2 +(XL -Xc) 2

o.s.v.

Av vad som framgått tidigare i detta avsnitt
kan även slutsatsen dras att:

2

Efter division med 12 fås

skenbar effekt=
= ~ (aktiv effekt) 2 +(reaktiv effekt) 2
113- 11

KR
LC- kretsar
Parallellkopplade LC-kretsar
Bild II 3-15
En parallellkopplad LC-krets är ansluten till
växelspänningen U från en signalgenerator
med inställbar frekvens f. Två fall studeras.
Fall i :

f=

fres

signalgeneratorns frekvens f ställs lika
med LC-kretsens resonansfrekvens fres· Då
visar kretsen hög impedans Z mot generatorn. En stark ström cirkulerar i svängningskretsen, men endast en svag ström flyter i
ledningen mellan generator och krets.
Jämför med modellförsöket på bild 3-000.
Fall2:

f> ~es

eller

f< ~es

Frekvensen f ställs högre eller lägre än
kretsens resonansfrekvens fres·
Svängningskretsen visar då en låg impedans Z mot generatorn. En svag ström cirkulerar i svängningskretsen, medan en starkare ström flyter i ledningen mellan generator och krets.
l praktiken finns även en resistans (belastning) parallellt över kretsen och en resistans i serie med induktansen. För enkelhetens skull bortses här från dessa resistanser.
l en parallellkopplad LC-krets är spänningen över induktans och kapacitans densamma. Spänningsvektorn U används därför som s.k. riktfas.

ström i XL

Riktfasen riktas på bilden åt höger. Strömmen le genom kondensatorn är fasförskjuten 90$\circ$ efter U och ritas rakt neråt (vektorerna roterar motsols). Strömmen IL genom
induktorn är fasförskjuten 90$\circ$ före U och
ritas rakt uppåt. Den resulterande reaktiva
strömmen genom kretsen är skillnaden mellan strömmarna le och IL, vilka är motriktade
varandra.
Formeln förparallellkopplade resistanser
kan även användas för parallellkopplade
re aktanser om man tillämpar Pytagoras sats
[A2 + 82 =
således

c2],

i
R

i
R1

1

-=-+-+ ....
R2

Gr ~(~r +(*J
~~ (~J +GJ ~~~, +;
eller

Med R försumbart kan den resulterande
reaktansen av kapacitansen Xe och den
vektormässigt motriktade induktansen XL beräknas på följande sätt
1
1
1
1 XL- X c
X= X c -XL d.v.s. X= -XL. X c eller

X= -XL ·Xc
XL -Xc
l en parallellkopplad LC-krets är den resulterande reaktansen negativ (kapacitiv) om XL
är större än Xe och positiv (induktiv) om XL är
mindre än Xe.

ström i XL

ström i Xc

l "'~~l

u

spänning

1--------IIB»

ström i Xc

Bild II 3-15 Parallellkopplad LC-krets

113- 12

u

Seriekopplade LO-kretsar

Bild II 3-16
En seriekopplad LC-krets ansluts till växelspänningen U från en signalgenerator med
inställbar frekvens f. Två fall studeras.
Fall 1:

f = f,es

signalgeneratorns frekvens f ställs lika
med svängningskretsens resonansfrekvens
fres· Impedansen Z i en seriekrets visar då ett
mycket lågt värde mot generatorn. Det flyter
en stark ström i ledningen mellan generator
och krets.
Fall2:

f< f,es eller f> f,es

Frekvensen f ställs lägre eller högre än
kretsens resonansfrekvens fres·
Eftersom svängningskretsen då visar hög
impedans Z mot generatorn, så flyter endast
en svag ström i ledningen mellan generator
och krets.
l praktiken finns även en resistans i serie
med induktansen liksom en parallellt över
kapacitansen. För enkelhetens skull bortses
här från dessa resistanse r.
Strömmen l är samma genom hela kretsen och strömvektorn l används därför som
s.k. riktfas. Den ritas i bilden åt höger. Om
serieresistansen R varit med, så skulle ett
spänningsfall UR varit inritad i samma riktning som l (i fas med 1). Spänningen över
re aktansen Xc ligger go o efter l och ritad rakt
neråt (vektorerna roterar motsols). Spänningen över reaktansen XL (induktorn) ligger
90$\circ$ före l och ritad rakt uppåt.

Thomson's svängningskrets
Bild II 3-17
Bilden visar en svängningskrets, som består
av en kondensator och en induktor med
förskjutbar järn kärna. En ändring av kärnans
tvärsnitt ändrar den magnetiska ledningsförmågan och därmed induktansen.
Med anordningen kan resonansfrekvensen alltså ställas in så att den blir högre, lika
med eller lägre än den anslutnaspänningens
frekvens. Tre fall undersöks:
XL > X c LA 1 och LA2 lyser upp, en kraftig
ström flyter genom kondensatorn,
XL < X c LA 1 och LA3 lyser upp, en kraftig
ström flyter genom induktorn,
XL= Xc LA2 och LA3 lyser upp, LA1 lyser
inte, en kraftig ström flyter i kretsen
men inte i tilledningarna
XL = X c kallas Thomson's svängningsformel, vilken beskriver resonansfallet
Då är de induktiva och kapacitiva
reaktanserna i kretsen lika stora och tar ut
varandra. Kvar är kretsens resistans, vilken
vi tills vidare betraktar som försumbar.
Således XL =Xc , där

XL

= 2 nfL
1

och

1 - sats
"t .tn.
Xc = 2nfC

2nfL=-2nfC

4n 2 f ·L· C= i

f-=1

f=

4n 2 LC

f [Hertz]

L [Henry]

1
C [Farad]

Formeln gäller både för parallell- och seriekretsar.

Bild II 3-16 Seriekopplad LC-krets

113-13

KRETSAR
Räkneexempel:
Strömriktning: 1 halvvågen -.....
2 halvvågen - - -

L= 100 nH C= 1O pF

f=

f=?

1
2n~1 00 ·1 o-9 ·1 o·1 o-12 = 2n1 o-9 =

109

= - ~ 159 MHz

2n

Impedansen i en resonant krets
En enkel framställning görs av hur impedans, re aktans och resistans förhåller sig
inbördes när en svängningskrets är i resonans. Som exempel används följande
kretsdata: Induktans 200 J.lH, kapacitans
200 p F, förlustresistans 1O Q.
Resonansfallet i en parallellkrets
Parallellkretsen består i sig själv av seriekopplade komponenter, varav XL och Xc
är reaktiva. Vid resonans är dessa lika
stora och motverkande. Inom kretsen är
således den resulterande reaktansen

Därför uppvisar samma krets en yttre
reaktans av

X= -XL ·Xc
XL -Xc

X= XL ·Xc
O

=oo

l praktiken finns i kretsen också en
resistans varför dessa extremvärden inte
uppstår. Inne i en parallellkrets i resonans cirkulerar alltså en stark ström, som
endast begränsas av kretsens resistans.
Bild II 3-18
Bilden visar en parallellkrets där
induktorn har resistansen rL och kondensatorn antas vara förlustfri. Vidare
förutsätts att kretsen är i resonans.
Vid resonans kan termen XL -Xc = O
bytas mot rL i formeln

X= -XL ·Xc
XL -Xc
Bild II 3-17 Thomson's svängningskrets

113-14

förutsatt att rL är försumbart jämfört med
XL.

ETSAR
Därtill är XL= 2nfL och Xc = ~fC
...
L
d .v.s. X L· X c= C som satts m.

2

Parallellkretsens impedans vid resonans kan
då skrivas
Z= XL ·Xc =-LIj
tj·C

Med ovanstående kretsdata blir Z= 100 kQ
Därav framgår, att impedansen i parallellkretsen är en funktion av det s.k. UC-förhållandet samt av kretsens resistiva förluster.

.
Z vtd

resonans:

XL~ Xc
l
--=-c
rL
rL·

l formeln

Z=~r/+(XL -Xc)
Z=~r/ +0 2

blirdå

=!j

Med ovanstående kretsdata blir resonansfrekvensen

1

~ 796kHz
2n LC
Vid resonansfrekvensen blir reaktansen
1000 Q både i induktansen och kapacitansen. Eftersom reaktansernas spänningsfall
är motriktade tar de ut varandra. Kretsens
impedans i resonans blir resistansen rL och
spänningsfallet över kretsen bestäms enbart av rL.

fo =

{[C

Antag att det alstras en spänning av 5 mV
i antennkretsen. Strömmen genom den vid
5
resonans blir då
m V= O. 5 mA.
10 Q
Av strömmen bildas reaktiva spänningar,
d.v.s. 0.5 mA·1 000 Q == 500 mV både över
induktans och kapacitans (som tar ut varandra) och 5 mV över resistansen.

Z vid
resonansfrekvens

2

resonans:

rL

ohm

Bild II 3-18 Resonansfallet i parallellkrets
Resonansfallet i en seriekrets
Bild II 3-19
När en seriekrets är i resonans, så är

XL =Xc

Serieresonans

i
d.v.s. mL = -

mC

eller

resonansfrekvens

frekvens

1

d.v.s. mL--=0

mC

Bild II 3-19 Resonansfallet i seriekrets

113-15

KR
Q-faktorn i en parallellkrets

Bild II 3-20
Godhetstalet Q (=Quality Facto r) kan ses
som den förmåga en svängningskrets har att
lagra energi, d.v.s. förhållandet mellan den
lagrade energin och energiförlusten i kretsen. Energiförlusten yttrar sig som värmeutveckling.

Q= n
2

lagrad energi i kretsen
energiförlusten per period
Energiförluster uppstår både i kretsens
kondensator och induktor, men moderna
kondensatorer har så låga förluster att
induktorn ensam kan anses bestämma Qvärdet, åtminstone i kortvågsområdet

Bandbredd

Bild II 3-21
Bilden visar med en kurva vilket impedansvärde kretsen har vid olika frekvenser.
Impedansens högsta värde är vid frekvensen fres och avtar vid frekvenser som är högre
eller lägre. Vid frekvenserna f1 och f2 är
impedansvärdet t. ex. 70$\circ$/o av maximalvärdet
Med bandbredden b förstås skillnaden mellan impedansvärdena i ett sådant frekvenspar, d.v.s. b= ~ - ~

z

Q= 2nfL =XL

R

R

En växelspänning U 1 ansluts till en
parallellkrets. l resonansfallet uppträder då
en spänning U2 över kondensatorn och
induktorn.
U2 är mycket större än U1 • Ju högre Q är
i kretsen desto större är förhållandet mellan
U2 och U1 •
l kortvågsområdet är det vanligt med ett
Q i storleksordningen 30 - 100.
Ju högre Q är, desto mindre är bandbredden.
När svängningskretsen är i resonans gäller sambandet

Q= ~es

b

Bandbredden ökar (avstämningsskärpan
minskar) vid ökande frekvens på grund av de
större kretsförlusterna.

u

f res
Bild II 3-20 Q-värden i parallellkrets
113- 16

f.

f
Bild 113-21 Bandbredd i parallellkrets

KRETSAR
3.2 Frekvensfilter

Frekvensfilter används inom radiotekniken
för många olika ändamål, t.ex. för att
• eliminera störande signaler,
• öka avstämningsskärpan (selektiviteten) i
mottagare och sändare,
• framhäva eller dämpa ett sidband i en AMsignal m.m.
Beroende på den s.k. frekvensgången,
så indelas filtren i flera "familjer", varav de
vanliga presenteras här.
Beroende på det tekniska utförandet finns
dels s.k. passiva filter vilka använder extern
energi för sin funktion, och dels aktiva filter
vilka i princip är förstärkare som likaledes
använder passiva kretsar. Här presenteras
för enkelhetens skull passiva filter.
Traditionella frekvensfilter är vad som
kallas analoga. Men nu i dataåldern börjar
även digitala filter vinna intåg. Sådana är
dock för komplicerade för att behandlas här.

Högpassfilter
Bild II 3-22

Ett högpassfilter släpper igenom signaler
med höga frekvenser och dämpar dem med
låga frekvenser.
Exempel: En frekvensberoende spänningsdelare som LC-högpassfilter.
.. Vid låga frekvenser är Xc stor och XL liten.
Over XL uppstår då ett litet spänningsfall- en
låg utgångsspänning ua. Resultatet blir att
låga frekvenser dämpas.
Vig höga frekvenser är Xc liten och XL
stor. Over XL uppstår då ett stort spänningsfall- en hög utgångsspänning ua. Resultatet
blir att höga frekvenser släpps igenom.
XL kan bytas ut mot en resister R, men då
blir passbandkurvan inte så brant.

Gränsfrekvens
Gränsfrekvensen f9 beror av kapacitansen
C, induktansen L samt resistansen R.

f

LC-högpass:

=

g

f9 [Hz]

C [Farad]

lågpassfilter

Bild II 3-23
Om induktor och kondensator respektive
resister och kondensator i ett högpassfilter
byter plats, så får man i stället ett LC-Iågpassfilter respektive ett RC-Iågpassfilter.
Ett lågpassfilter släpper igenom signaler
med låga frekvenser och dämpar dem med
höga frekvenser.
Exempel: En frekvensberoende spänningsdelare som LC-Iågpassfilter.
.. Vid låga frekvenser är Xc stor och XL liten.
Over XL uppstår då ett litet spänningsfall- en
hög utgångsspänning ua. Resultatet blir att
låga frekvenser släpps igenom.
Vig höga frekvenser är Xc liten och XL
stor. Over XL uppstår då ett stort spänningsfall -en låg utgångsspänning ua. Resultatet
blir att höga frekvenser dämpas.

Gränsfrekvens
Samma formler används vid beräkning av
gränsfrekvensen både i lågpass- och högpassfilter, således
LC-Iågpass: ~

f9 [Hz]

L [H]

RC-Iågpass: ~

f9 [Hz]

1

= 2 rc{LC
C [F]

1

= 2 rcRC

C [F]

R [Q]

C [Farad]

f =g

f9 = ?

1
10 3
~=
=-=79.62 Hz
2rc~ 4. 2rc ·1 o-B
4n
2) R= 1 kQ C= 1O n F f9 =?
1
10 5
~=
3
9 = = 15.934
2rc·1·10 ·10·102rc

1
2rc-fLC

L [Henry]
RC-högpass:

Räkneexempel:
1) L = 4 H C = 1 J.lF

1
-

2rcRC
R [Ohm]

113-17

KRETSAR

L

LC- HÖGPASS

Ua

Ekvivalentschema

Xc
R
RC -

HÖGPASS

Ekvivalentschema

Låga frekvenser dämpas

L

Höga frekvenser släpps igenom

Bild II 3-22 Högpassfilter

113-18

fg
Passbandskurva

f

KRETSAR

~©~CEPT

L

XL

Te

Ue

Uu

Ue
Xc

Ua

o

o
LC- LAGPASS

o

o

R
CJ

Ue

Ekvivalentschema

o

le

R

Ua.

I

o

Ue

Ekvivalentschema

RC- LAGPASS

L

o-------1 t ' r n r r •

o

IV Ile IV
o

Ua

Xc
o

-----o

Låga frekvenser släpps igenom

100

Ue

70

fg

o,..I..(-o

f

Passbandskurva

HöQa frekvenser dämoas

Bild II 3-23 Lågpassfilter

113- 19

KRETSAR
Bandpassfilter

Bild 3-24
Ett bandpassfilter släpper igenom signaler
bara inom ett frekvensområde medan signaler inom andra frekvensområden dämpas.
Bandpassfiltret består i enklaste fall av
två svängningskretsar av LC-typ, vilka är
avstämda till angränsande frekvenser. Kretsarna är kopplade induktivt, kapacitivt eller
galvaniskt.

Beroende på kopplingsgrad skiljer man
mellan underkritisk koppling (lös koppling),
kritisk koppling och överkritisk koppling (fast
koppling).
På bilden visas hur passbandet påverkas
bl.a. av kopplingsgraden. Lös koppling liten bandbredd. Kritisk koppling - större
bandbredd. Fast koppling- stor bandbredd.

KOPPLINGSSÄTT

Induktivt

Kapacitivt

Galvaniskt

fr
Underkritisk koppling

Bild II 3-24 Bandpassfilter

113-20

Kritisk koppling

f

fr
överkritisk koppfing

f

KRETSAR
Passfilter

Bild 113-25
Passkretsen stäms av till en viss frekvens
och erbjuder där en mycket låg impedans.
Passkretsen kopplas i serie med signalvägen och låter signaler med frekvenser inom
filtrets passband att passera.

fr
Bild II 3-25 Passfilter

Bandspärrfilter

Bild II 3-26
Om serie- och parallellkretsarna i ett bandpassfilter byter plats, så får man i stället ett
bandspärrfilter. Ett sådant spärrar signaler
inom ett visst frekvensområde, men släpper
igenom signaler utom detta område.

o

uin

I

o

I

CJ

I

I

uut
o

uut
o

Bild II 3-26 Bandspärrfilter

Spärrfilter

Bild II 3-27
Spärrkrets
Spärrkretsen stäms av till en viss frekvens
och erbjuder där en mycket hög impedans.
Spärrkretsen kopplas i serie med signalvägen och spärrar en signal med samma
frekvens som resonansfrekvensen.

Bild 113-27
sugkrets
sugkretsen stäms av till en viss frekvens och
erbjuder där en mycket låg impedans. sugkretsen kopplas parallellt med signalvägen
och kortsluter (suger bort) en signal med
samma frekvens som resonansfrekvensen.

113-21

KRETSAR

o

o

uin

o

uut

uut

·SPÄRRKRETS

I

o

I

uin
o

f

o

fr

uut
o

SUGKRETS

Bild II 3-27 Spärrfilter (2 sorter)

Kvartskristall
Bild II 3-28

Bandfilter med kvartskristaller

En kvartskristall, egentligen en slipad skiva

av kvarts, kan fungera som en elektromekanisksvängningskropp (resonator), vars egenskaper liknar dem i en LC-krets.

Den låga inre resistansen gör att Q-värdet i en kvartskristall är bättre än 10000.
Som jämförelse är Q-värdet i en LC-krets
oftast sämre än 1000.

Bild II 3-29
Kvartskristaller kan kombineras till filter med
önskad bandbredd. Även utföranden med
keramiska resonatorer finns.
Resonatorerna är avstämda till var sin
bestämda frekvens och hela komplexet bidrar på så sätt till att bilda passband eller
andra egenskaper på samma sätt som med
sammankopplade LC-kretsar.

j
[:=J

T
Schemasymbol

Ekvivalentschema

Bild II 3-28 Kvartskristall
113-22

Bild II 3-29 Bandfilter med kvartskristalle

KRETSAR
Mekaniska filter

Bild II 3-30
Med en elektromekanisk givare kan man få
en kropp (resonator) att svänga på sin resonansfrekvens. Med ännu en elektromagnetisk givare kan man känna av svängningarna och återvandla dem till elektriska signaler. Hela anordningen fungerar som en elektromekanisk resonator, vars egenskaper liknar dem i en LC-krets.
Resonatorerna kan kombineras till filterkomplex med önskad bandbredd där resenatorerna är avstämda till var sin bestämda frekvens. Hela komplexet bidrar på så
sätt till att bilda ett passband på samma sätt
som med sammankopplade LC-kretsar.
Beroende på tillämpningen finns olika frekvenslägen i intervallet 60-600 kHz.
Mekaniska filter användes mest förr som
mellanfrekvensfilter i högvärdiga radioutrustningar, men har numera till stor del ersatts
av bandfilter med kvartskristaller där arbetsområdet kan ligga avsevärt högre i frekvens.

Bild II 3-31 Kavitetsfilter
Inkommande och utgående signaler ansluts till filtrets mittledare över induktionskondensatorer eller direkt galvaniskt.
kavitetsfilter kan kopplas ihop för att
bilda bandfilter, frekvensdelare m.m ..

Helixfilter

Kavitetsfilter
Bild II 3-31

Svängningskretsars dimensioner minskar
med ökande frekvens. Vid mycket hög frekvens kan induktorns varvtal i en LC-krets ha
minskat till ett enda varv samtidigt som kapacitansen inom detta enda varv kan räcka
för önskad resonansfrekvens.
En sådan svängningskrets kan bl.a. ha
formen av en ledare mitt inne i en elektriskt
ledande kavitet. Ledarens längd tillsammans
med kavitetens insida bildar induktorn. Mellan ledaren och kavitetens insida råder en
kapacitans, som kan kompletteras/justeras
med en extra kondensator.

Givare

När ett kompakt kavitetsfilter behövs, så kan
man öka reaktansen i mittledaren både induktivt och kapacitivt genom att utforma den
som en spiral (helix). Detta är dock på bekostnad av Q-värdet. Flera kavitetsfilter kan
kopplas ihop för att bilda bandfilter, spärrfilter m.m ..

Resonanskroppar

Avkännare

]!~CJ··C) C)EJ]~
Bild II 3-30 Mekaniskt filter

113-23

KRETSAR
Pi-filter

Bild II 3-32
För att överföra H F-signaler med bästa verkningsgrad, så är det viktigt med god impedansanpassning mellan de olika funktionerna. Om anslutningsimpedansen är lika i båda
~~nktionerna, så behövs inga extra åtgärder.
Ar impedanserna däremot olika, så behövs
korrigeringsnät (filter).
Ofta är nätet Pi-format och består av
induktanser och kapacitanser. Ett Pi-format
nät kan sägas bestå av två L-formade nät
ställda mot varandra, där den seriella delen
är gemensam (på bilden en induktor).

T-filter

Bild II 3-33
Ett nät kan också vara T -format och bestå av
induktanser och kapacitanser. Ett sådant
nät kan sägas bestå av två L-formade nät
ställda "rygg mot rygg". Då är den parallella
delen gemensam. På bilden visas två alternativ.
När den parallella delen är kapacitiv, blir
huvudkaraktären ett lågpassfilter, men att
impedansanpassning också är möjlig med
en induktiv impedansdelning.
När den parallell delen är induktiv blir
huvudkaraktären ett högpassfilter, men att
impedansanpassning också är möjlig med
en kapacitiv impedansdelning.

Ett Pi- eller T-filter kan fungera som
• svängningskrets,
• impedanstransformator (anpassning),
• balansera ut en reaktans o.s.v.

o

·O

l

el
I

w f •

Bild II 3-32 Pi-filter
113-24

L

J t 'f , '

le
I

o

o

(

(

o

o

Bild II 3-33 T-filter

3.3

Kraftförsörjning

Den elektriska energi, som behövs för elektronikutrustningar, hämtas från det allmänna elektricitetsnätet, ett batteri eller en ackumulator. Vissa batterityper kan återuppladdas och kallas då ackumulator.
Batterier och ackumulatorer avger en
nominell spänning, som beror av de ingående materialen och givetvis av laddningstillståndet. Moderna utrustningar för amatörradio är utförda för 12 V likström och försörjs
vanligen från ett nätanslutet kraftaggregat
På så sätt kan mobila radioutrustningar även
försörjas från startackumulatorn i fordonet.
Handburna radioutrustningarförsörjs från
en inbyggd ackumulator, som laddas från
stationär laddare.
Äldre stationära radioutrustningar drivs
nästan alltid med nätanslutna kraftaggregat
med en eller flera transformatorer och likriktare. Alternativt kan samma transformators
sekundärsida vara försedd med flera lindningar för olika spänningar och strömkretsar.
Det allmänna elnätet i Sverige levererar
växelspänning med frekvensen 50Hz. Nätspänningen för hushållsändamål är numera
400/230 v.
Tidigare importerade utrustningar i marknaden kan vara utförda för andra nätspännings- och skyddsjordningssystem än vad
som nu tillämpas i Sverige. Försiktighet med
sådan utrustning rekommenderas.

Halvm och helvågslikriktning m. m.
Bild II 3-34
Likriktning av spänningar och strömmar i en
krets görs med "elektroniska ventiler", som
släpper igenom ström endast i den s.k. passriktningen och stoppas i spärriktningen. En
sådan strömventil kallas för diod och kan
vara av typen vakuumrör eller halvledare. l
moderna konstruktioner används uteslutande halvledardioder i likriktarkopplingar.

+

!>l

Passriktning

[>l

+

Spärriktning

Bild II 3-35
Halvvågslikriktning
Vid halvvågslikriktning släpps endast varannan halwåg av en växelspänning igenom. l
den strömkrets, som bildas av transformatorns sekundärlindning, dioden och belastningen, flyter därför ström endast under
varannan halvperiod.

Helvågslikriktning
l följande kopplingar med två respektive fyra

dioder släpps varje halwåg av transformatorns växelspänning igenom så att alla halvvågor får samma polaritet. Ström flyter genom belastningen i samma riktning under
varje halvperiod. Följande sätt att anordna
helvågslikriktning är vanliga:
• Med två dioder och mittuttag på transformatorns sekundärlindning. Den ena dioden och ena lindningshalvan släpper igenom ström till belastningen under ena halvperioden. Den andra dioden och andra
lindningshalvan under följande halvperiod
o.s.v.
• Med fyra dioder (s.k. Graetz-brygga) och
inget mittuttag på transformatorns sekundärlindning, släpper dioderna 1 och 3 igenom ström under den ena halvperioden.
Dioderna 2 och 4 släpper igenom ström
under följande halvperiod o.s.v.

Glättningskretsar
Bild II 3-36
Efter likriktningen har växelspänningen omvandlats till en pulserande likspänning som
kan "glättas". Efter likriktarna ansluts då ett
glättningsfilter, som t. ex. kan bestå av laddningskondensatorn CL, induktansen L (s.k.
drossel) och glättningskondensatorn C 8 .
Parallellt över denna kondensator ligger för
elsäkerhetens skull en urladdningsresister
med hög resistans alltid inkopplad.
Säkerhetsresistorn skall ladda ur kondensatorerna, när kraftaggregatet inte är
anslutet till strömförsörjningen och belastningen. Säkerhetsresistorn (eng. bleede()
skall vara av trådlindad typ och kunna tåla
fyra gånger sin egen effektförbrukning.

Bild II 3-34 Halvledardioder
113-25

KRETSAR

HALVVAGSLIKRIKTNING

u~~a:u~
HELVAGSUKRIKTNING

a - med 2 dioder

b- med 4 dioder
(Graetzkoppfing)

1 :a halvvågen

Diod 1 och 3 i passriktning

u
2 :a ha tvvågen

Diod 2 Diod 2 och 4 i passriktning

Bild II 3-35 Halv- och helvågslikriktning

113-26

KRETSAR

HALVVÅGSUKRIKTARE MED GLATTNINGSFILTER

GRAETZKOPPLING MED GLATTNINGSFILTER

Bild II 3-36 Glättning av likspänning
l obelastat tillstånd är spänningen över
gånger högre
laddningskondensatorn
än effektivvärdet på transformatorns sekundärspänning. När en transformator i tomgång har ett effektiwärde av 230 V över
sekundärlindningen, så blir spänningen över
= 325 V.
säkerhetsmotståndet 230

.J2

.J2

Spänningshöjande likriktarkopplingar
Vid likriktning av växelspänningar enligt någon av ovanstående metoder behövs en
sekundärspänning från transformatorn av
minst samma storlek som den önskade likspänningen. Önskas en högre likspänning,
t.ex. den dubbla, men med samma sekundärspänning på transformatorn, så måste en
speciell likriktarkoppling användas.
Bild 113-37
Bilden visar en spänningsfördubblande
koppling. Under 1 :a halwågen laddas kondensator C 1 upp. Under 2:a halwågen laddas kondensator C2 upp. Kondensatorerna
är kopplade i serie och den ena kondensatorn
hinner inte bli urladdad under tiden som den
andra kondensatorn blir uppladdad. Följden
blir att belastningen ser kondensatorernas
spänningar som seriekopplade och därmed

har en spänningsfördubbling erhållits. Det
finns även kopplingar för flerdubbling av
spänningar, vilket bl. a. används för att alstra
accelerationsspänningen för TV-bildrör.

Spänningsstabilisering

Bild II 3-38
Utspänningen från ett kraftaggregat tillåts i
många fall att endast variera mellan vissa
värden, fastän inspänningen och strömuttaget varierar mycket. Ett vanligt sätt att hålla
konstant spänning är att anordna en automatisk spänningsdelare efter glättningsfiltret
Glimlampan och zenerdioden har egenskapen att spänningsfallet över dem är i det
närmaste konstant inom ett visst strömområde. Glimlampor arbetar på högre spänningar och används i utrustningar med elektronrör. Zenerdioder arbetar på de lägre
spänningar som används i dagens elektronik.
stabiliseringen tillgår så att t. ex. zenerdioden får ingå som aktiv del i en spänningsdelare, som består av en resister i serie med
belastningen och zenerdioden parallellt med
den. Zenerdioden tar upp variationerna i
belastningsströmmen, varvid spänningen
113-27

KR

1 :a Halvvågen

2:a Halvvågen

Bild II 3-37 Likriktarkoppling med spänningsdubbling
över spänningsdelarens uttag blir stabiliserad. Vid större strömuttag kan zenerdioden
inte ensam ta upp hela den effekt som den
reglerar bort. l stället tas effekten upp av en
eller flera transistorer som i sin tur regleras
av zenerdioden.
l vissa fall behövs i stället en reglerad
utström från kraftaggregatet Även för detta
ändamål används kopplingar med zenerdioder och transistorer.
Senare utvecklingsformer är s.k. switchade aggregat. l sådana regleras spänningen eller strömmen genom sönderhackning (switching). Genom att förändra förhållandet mellan till- och frånslagstiderna kan
man skapa det önskade medelvärdet. Metoden ger hög verkningsgrad. Switch-frekvensen är i storleksordningen 20 kHz eller högre. På grund av den högre frekvensen
krävs mindre kondensatorer i switchade
aggregat. Sådana kraftaggregat kan emellertid ge störningar, varför effektiv avstörning behövs.

113-28

Ostabiliserad
spänning in

Stabiliserad
spänning ut

Bild II 3-38 Spänningsstabilisering

3.4 Förstärkare

Allmänt
Bild Ii 3-40
Elektronrör och transistorer är de aktiva komponenter som används i oräkneliga elektroniska kopplingar för alstring av signaler, för
förstärkning och blandning av signaler, för
multiplicering av signalfrekvenser etc.
Transistorn presenteras i avsnitt 2.6 och
elektronröret i avsnitt 2.7.
Först förekom endast elektronrör. Dessa
har emellertid på några få decennier nästan
helt ersatts av transistorer. Elektronrör används dock fortfarande, särskilt i effektförstärkare för sändare. Det finns därför skäl att
här behandla såväl elektronrör som transistorer.

Förstärkning
Med förstärkning avses här kvoten av amplituden i utgående och inkommande signal,
varvid frekvensgången har inverkan.
Frekvensgång
Förstärkare arbetar endast inom ett visst
frekvensområde, vilket kan skilja från fall till
fall.
Bandbredd
Det frekvensområde där förstärkaren arbetar med fulla data kallas bandbredd. Bandgränserna uttrycks som en nedre och övre
gränsfrekvens, där signalnivån avviker från
ett givet värde, vanligen med högst 3 dB.
För LF-förstärkare för amatörradiobruk
är kravet på bandbredd litet; inom ett band
av 300 Hz till 3 kHz uppnås godtagbar
återgivningskvalitet för taL Bandbredden
bestäms främst av kondensatorer i kretsen
avsedda för överföring och avkoppling.
HF-förstärkare används för signaler med
hög frekvens, typiskt 100kHz och däröver.
Det finns s.k. bredbandig a förstärkare för ett
stort frekvensområde, men även avstämda
förstärkare för smala frekvensband.

Huvudegenskaper hos förstärkare
LF- och HF-förstärkare
Bild II 3-41
Med LF-förstärkare menas förstärkare som
arbetar med signaler i det lägre frekvensområdet, typiskt upp till c:a i 00 kHz. LF-förstärkare är mycket vanliga såväl i mottagare
som sändare. Utöver de aktiva komponenterna (transistorer, elektronrör ) är kondensatorer och resistorer de viktigaste passiva.

Med H F-förstärkare menas förstärkare som
arbetar med signaler med högre frekvenser
än dem i LF-området. Även HF-förstärkare
är mycket vanliga såväl i mottagare som
sändare. De används t.ex. i mottagarnas
ingångs- och mellanfrekvenssteg, liksom i
sändarnas oscillatorer, signalberedningssteg
och slutsteg.
Utöver de komponenter, som även finns
i LF-förstärkare, används kombinationer av
frekvensberoende komponenter såsom
induktorer och kondensatorer.

Anod

·~·

Galler

.p

Katod

n

Kollektor
Bas
Emitter

npn
Bild II 3-40 Från elektronrör till transistor

113-29

KRETSAR

o[]

o[]
Katod kopp l i ng

Emitterkoppling

Bild II 3-41 Principen för förstärkare med elektronrör respektive transistor

Grundkopplingar för förstärkarsteg
Bild 113-42

l det föregående har redan visats att en av
polerna i ingången respektive utgången i en
förstärkare är gemensam. l ovanstående
bild är rörförstärkarens katod den gemensamma polen -därav namnet katodkoppling.
På liknande sätt är N PN-transistorns emitter
gemensam -därav namnet emitterkoppling.
På ett liknande sätt kan någon annan pol
vara gemensam. Man får då i stället en
baskoppling eller kollektorkoppling.
Beroende av kopplingsätt fås olika egenskaper. På nästa sida visas tre olika grundkopplingar för ett elektronrör (triod) respektive en NPN-transistor.
l praktiken känns en grundkoppling igen
på vilken elektrod som är avkopplad till Opotential över en kondensator.

Emitterkoppling används för LF och HF
när hög förstärkning eftersträvas. Eftersom
effektförstärkningen är produkten av
spännings- och strömförstärkningen, så fås
en effektförstärkning av mellan 200 till50000
gånger. Nackdelen med denna koppling är
den ibland låga ingångsimpedansen och
den relativt låga gränsfrekvensen.
Baskoppling använd som H F-förstärkare pågund av sin höga gränsfrekvens och
goda isolation mellan in- och utgång.
Kollektorkoppling används när hög ingångsimpedans och utgångsimpedans önskas. Denna koppling har emellertid ingen
spänningsförstärkning, men kan användas
för s.k. impedansomvandling.

Grundkopplingarnas typiska egenskaper vid NPN-transistor
Egenskap

Emitterkoppling

Baskoppling

Z in

medel

1 KQ

liten

son

stor

100 kn

Z ut

medel

10 kQ

stor

100 kQ

liten

50 kQ

ström-

stor

100 ggr

<1

0.9 ggr

stor

100 ggr

spänning-

stor

100 ggr

stor

100 ggr

<1

0.99 ggr

effekt-

mycket stor 10000 ggr

stor

100 ggr

stor

100 ggr

motfas

medfas

oo

medfas

oo

Kollektorkoppling

Förstärkning

Fasläge

113-30

180$\circ$

ETSAR

Ut
In

In

Katodkoppling

~

In

!

l

j

Emitterkoppling

l
In

Ut

Ut

Gallerkopp li ng

j

Ut

In

Ut

Baskoppling

~

In

!

In,

In

Anod kopp! in g

Kollektorkoppling

Bild II 3-42 Grundkopplingar för elektronrör och NPN-transistor

113-31

KR

EPT
stabilisering av arbetspunkten

För att en förstärkare skall kunna arbeta på
avsett sätt måste arbetspunkten, d.v.s.
arbetsströmmens vilavärde ställas rätt.
Det gör man genom att placera en förspänning över den styrande elektroden i
elektronröret eller transistorn i fråga.
l en katodkopplad rörsförstärkare innebär det att styrgallret skall ges en viss negativ spänning i förhållande till katoden. Det
kan man göra t.ex. med en separat spänningskälla eller vanligare med en avkopplad
res istor mellan katod och minuspolen (jord).
l en emitterkopplad transistorförstärkare
innebär det att basen skall ges en viss positiv
spänning i förhållande till emittern. Det kan
man göra t.ex. med en separat spänningskälla eller vanligare med en avkopplad resisto r mellan emittern ochminuspolen samt en
resistiv spänningdelare mellan plus- och
minuspolen.

Klass A-, B- och C-förstärkare
Arbetspunkt
Arbetspunkten för förstärkare väljs olika,
beroende på önskat arbetssätt. En olämpligt
vald arbetspunkt resulterar i förvrängning av
utsignalens form i förhållande till insignalens
form, s.k.distorsion. Distorsion uppstår även
vid överstyrning, d.v.s. när insignalens amplitud är för stor för att kunna återges med
oförändrad form, även om arbetspunkten är
rätt vald.
Med avseende på arbetspunktens läge
klassas därför förstärkare på sätt som framgår av följande diagram för elektronrör. En
emitterjordad NPN-transistor får anses mest
motsvara elektronrörkopplingen här nedan.
Anodströmmen la motsvaras då närmast av
kollektorströmmen le och styrgallerspänningen U~ 1 av spänningen UsE· Den stora skillnaden år att styrgallerspänningen i dessa fall
alltid är negativ medan bas/emitterspänningen är positiv. styrspänningens relativa läge
(arbetspunkten) mellan olika arbetsklasser
är emellertid lika.

113-32

KRETSAR
Klass A
Bild II 3-44
Klass A är ett arbetssätt i linjära LF- och HFförstärkarsteg, t.ex. i mottagare samt AMoch SSB-modulerade sändare. Vilavärdet
på strömmen i huvudkretsen, den s.k. arbetspunkten, placeras i mitten på den rakaste delen av styrkaraktäristikan (1=0.5 lmax).
Därmed fås låg distorsion. Verkningsgraden
är upp till 50 $\circ$/o.

ningskretsen. En resonanskrets med högt
Q-värde behövs som utgångskrets varvid
amplituddistorsion inte framstår som besvärande vid CW och FM. Med hjälp av en
utgångskrets kan frekvensmultiplicering utföras med förstärkare i klass C.
(På följande tre bilder är IR=anodviloström).

Klass AB
Klass AB är ett godtagbart linjärt arbetssätt
för AM- resp. SSB-modulering, men med en
lägre viloström. Arbetspunkten ligger mellan
den för klass A och B. Ett linjärt arbetssätt
enligt klass A är visserligen önskvärt vid
SSB, men verkningsgraden är lägre. Klass
AB är en kompromiss med bättre verkningsgrad utan en alltför stor distorsion.
Klass B
Bild II 3-45
Klass B är ett olinjärt arbetssätt med en låg
vilaström i förhållande till !max• d.v.s. arbetspunkten ligger i nederkant av styrkaraktäristikans nedre krökta del. Verkningsgraden
är upp till 67o/o. Trots det används klass 8 i
linjäraLF-och H F-förstärkarsteg t. ex. i SSBsändare.
Om klass B skulle tillämpas i ett slutsteg
med endast ett rör eller en transistor skulle
större delen av uteffekten förloras i splatter,
d.v.s. som förvrängda signaler långt vid sidan om den egentliga nyttosignalen. Ett sätt
att undvika det är att använda en avstämd
utgångskrets med högt Q-värde. Linjär förstärkning kan också erhållas med två mottaktkopplade rör eller transistorer i klass B.
Utgångskretsen behöver då inte vara avstämd av linjäritetsskäl.
Klass C
Bild 113-46
Klass C används i HF-förstärkarsteg i FM-,
CW- och AM-sändare. Arbetssättet är kraftigt olinjärt. Vilaströmmen är noll, d.v.s. arbetspunkten ligger på den negativa delen av
styrkaraktäristikan. Endast toppen av den
ena halvvågen av insignalen återges och i
starkt förvrängd form. Verkningsgraden är
upp till80$\circ$/o. Övertonerna dämpas av sväng-

~järt

l

Förvrängning
(distorsion)

genom fel arbetspunkt

Bild If 3-44 Förstärkare i klass A

113-33

KRETSAR

Bild II 3-45 Förstärkare i klass B

Frekvensmultiplicering
Bild 113-47

Frekvensmultiplicering kan användas för att
skapa en högre frekvens än den som avges
av oscillatorn. Oscillatorn följs då av ett eller
flera frekvensmultiplicerande förstärkarsteg
som arbetar i klass C.
l utgången av ett frekvensmultiplicerande steg måste finnas en svängningskrets,
som är avstämd till önskad frekvens, d.v.s.
överton av insignalen. Denna överton förstärks i efterföljande förstärkarsteg, vilket
också kan vara frekvensmultiplicerande.
Ju högre multiplikationsfaktorn är, desto
högre förspänning krävs för att svängningskretsen i utgången skall svänga obehindrat.
Med hög multipliceringsfaktor i ett enda steg
dämpas signalen då så mycket att en hög
förstärkning behövs i efterföljande steg. l
praktiken anordnas därför en kedja av frekvensdubblande och frekvenstripplanda
Den totala multipliceringsfaktorn är faktorerna för vartdera steget multiplicerat med varandra.
Som exempel visar bilden blockschemat
för en VHF-sändare med oscillatorkristaller i
8 MHz-området. Som räkneövning kan andra kristallfrekvenser sättas in för beräkning
av den slutliga sändningsfrekvensen. l frekvensmultiplicerande sändare kan även slutsteget arbeta i klass C, vilket är vanligt i
sändare förtelegrafi eller FM-telefoni. För att
då förhindra utsändning av alla de övertoner

113-34

Bild II 3-46 Förstärkare i klass C
som alstras i förstärkarkedjan, så förses
slutstegets utgång med en svängningskrets
som är avstämd till sändningsfrekvensen.
Övertonsdämpningen kan förbättras }'1terligare med ettefterföljande lågpassfilter. Overtoner för 144 MHz är 288 MHz, 432 MHz
o.s.v.
Frekvensmultiplicering behöver nödvändigtvis inte göras med ett förstärkarsteg i
klass C. En diod har nämligen olinjär karaktäristik och därmed alstras det övertoner i de
strömmar som passerar genom den. En av
dessa övertoner kan filtreras fram och förstärkas. T.ex. finns det frekvenstripplingssteg byggda kring en speciell typ av kapacitansdiod- varaktordiod. Vanliga frekvensområden för s.k. varaktortripplare är 144/
432 MHz och 432/1296 MHz.
Såväl signalen från en kristalloscillator
som den från en VFO kan multipliceras till en
högre frekvens.
Förr täckte VFO i amatörradiosändarna
oftast frekvensområdet 3.5-3.8 MHz. Med
en så vald VFO-frekvens kunde alla upplåtnafrekvensband för amatörradio nås med
frekvensmultiplicering. De ursprungliga amatörradiobanden i KV-området ligger fortfarande harmoniskt relaterade av detta skäl.
Således
2 • 3.5 = 7 MHz
2 · 2 · 3.5 = 14 MHz
2 • 3 · 3.5 = 21 MHz
2 · 2 · 2 · 3.5 = 28 MHz

KRETSAR

PT
-~ 1%1

T

CD

r·---{I>J---~--{g
Olinjär förstärkare

~····
f

\

f"' " 24 HHz

= 8 MHz

!f

8 MHz

+övertoner

~---·----

·-·lliJ--..--------..·--· ---

r

------------o

.,..............

n = 3 · 3 · 2 = 18

fs : : n. fa
fs = 18 · fa

[ii}

fq = 8, ... MHz

l

f :.: ..., ... MHz

f = B, ... MHz

= 24MHz

24 MHz

.
f re kvenstnpp 1are

[o

DJtAAAMAft

!Lfl[} .....

f= ... , ... MHz

f = .. , ... MHz

f = ... , ... MHz

Fyll i frekvensvärdena för styrkristallen och beräkna resterande frekvensvärde
i multiplikatorkedjan

Bild 113-47 Frekvensmultipliceringskedja
Vid frekvensmultiplicering flerfaldigas inte
bara oscillatorfrekvensen utan även variationerna i den. Om t.ex. VFO-frekvensen i
området 3.5 MHz ändras med 50 Hz, så
ändras utfrekvensen i området 28 MHz med
2 · 2 • 2 = 400 Hz. Alla frekvenser i signalen
multipliceras på detta sätt. Amplitudmodulerad telefoni kan därför inte överföras genom
en frekvensmultipliceringskedja utan att talet förvrängs.

Se5.3

113-35

KRETSAR
Sändarslutsteg
Slutsteg med en transistor
Bild 113-48

Transistorslutsteg för HF byggs vanligen
emitterkopplade p.g.a. den högre effektförstärkningen.
Bilden visar ett sådant förstärkarsteg.
Kollektorbelastningen består av en svängningskrets. För att anpassa transistorns kollektorimpedans till svängningskretsens impedans, har kollektorn anslutits till ett uttag på
svängningskretsens spole.
Orossel Dr och kondensator C fungerar
som en HF-mässig avkoppling av strömförsörjningen.
Uteffekten tas ut från svängningskretsen
över en kopplingslindning med samma impedans som belastningen.
För linjär återgivning krävs drift i klass A
eller möjligen klass AB.
Or* +U

Bild II 3-48 S!utsteg med en transistor

slutsteg med två transistorer
Bild 113-49
Ett mottaktkopplat (e ng. pus h-pull) förstärkarsteg i klass B har god verkningsgrad
samtidigt som det är nöjaktigt linjärt för SSB
i amatörradio. l ett slutsteg med endast en
transistor skulle denna behöva klara fyra
gånger så stor förlusteffekt
P.g.a. de låga impedansvärdena i transistoriserade förstärkarsteg används transformatorer, vilka inte är frekvensselektiva
och därför inte dämpar övertoner. Med mottaktkopplingen alstras dock inte jämna övertoner. För övertonsdämpning används fast
avstämda bandpassfilter, ofta ett per frekvensband, mellan drivsteg och slutsteg samt
mellan slutsteg och antenn.
För noggrann anpassning till antennen
behövs en antennkopplare - s.k. matchbox
- med ett n-, T- eller L-kopplat LC-filter.
Att ett slutsteg är "bredbandsavstämt" är
således en fråga om definitioner.
Högeffekts/utsteg med en tetrod
Bild II 3-50
Bilden visar ett effektslutsteg för HF med ett
elektronrör, en s.k. tetrod, i katodkoppling.
Det kan även vara en triod eller en pentod.
Med LC-kretsen i styrgallerkretsen filtreras (selekteras) önskade signalfrekvens ut
ur signalerna från föregående steg.
Orossiarna Dr spärrar HF och kondensatorerna C1 , C2 och C3 kortsluter (avkopplar)
HF till jord. Allt för att hindra HF att komma in
i kraftaggregatet

bredband
U2

Bild 113-49

113-36

omkopplingsbart för
vart och ett av banden

KRETSAR
HF-förstärkare kan råka i oönskad självsvängning. Orsakerna kan vara många, bl. a.
dålig avkoppling av matningsspänningar, induktiv och/eller kapacitiv återkoppling i kretsarna m.m.
Återkopplingsvägar både före och efter
röret kan bilda oavsiktliga svängningskretsar, som genererar självsvängning, ofta på
mycket höga frekvenser t.ex. i VHF-området. Sådana s.k. parasitsvängningar kan
stoppas/dämpas med UHF-drosslar (UHF
Dr) omedelbart intill röranslutningarna.

En åtgärd mot självsvängning i elektronrör är en motkopplingsväg från anod till styrgaller över en trimningsbar s.k. neutraliseringskondensator CN.
slutstegets utgångskrets kan utformas
på olika sätt. Bilden visar ett numera vanligt
sätt, det s.k. n-filtret (utläses pi-), som fungerar som
• en svängningskrets som är avstämd till
sändningsfrekvensen,
• ett övertonsdämpande lågpassfilter,
• anpassning mellan rörets utgångsimpedans och antenntilledningens impedans.

UKV-drossel

Cp

Bild II 3-50 Högeffekts/utsteg med en tetrod

- Ug1

I

I I

I
Bild II 3-51 Högeffekts/utsteg med två trioder

113-37

KRETSAR
Högeffekts/utsteg med två gallerjordade trioder (elektronrör)
Bild II 3-5i
Gallerjordad koppling innebär att elektronrörets styrgaller ligger på HF-mässig nollpotential medan styrsignalen matas in på
katoden. likspänningen mellan katod och
styrgaller väljs så att rörets arbetspunkt blir
den avsedda.
Gallerjordad koppling passar särskilt för
slutsteg med höga effekter, men fordrar en
högre styreffekt än andra kopplingar. l gengäld "överförs" styreffekten till utgången via
röret och ingår där i uteffekten. l gallerjordad
koppling är kapacitansen låg mellan katod
och anod, d.v.s.mellan in- och utgång. Därmed är risken för självsvängning betydligt
mindre än i ett katodjordat steg.
Uteffekten kan ökas genom att parallellkoppla två eller flera rör, som då skall ha så
lika data som möjligt. Uteffekten står i direkt
proportion till antalet rör.
Flera parallellkopplade rör medför emellertid ökade totala rörkapacitanser, ökade
kapacifanser i kopplingsledningarna m.m.,
vilket är till nackdel vid höga frekvenser.
Ett enda slutrör för hela effekten är emellertid dyrare än flera små med tillsammans
jämförbar effekt. Mottaktkoppling av två rör
(eng. "push-pull") i st.f. parallellkoppling har
en fördel i högre förstärkning, men nackdelar i mer komplicerad bandomkoppling av
svängningskretsar m.m. l moderna rörutrustade slutsteg för amatörradio förekommer
därför endast ett slutrör eller flera parallellkopplade. Utgångskretsen är i regel ett nfilter med manuell eller automatisk avstämning.

Slutsteg med elektronrör jämfört med
transistoriserade slutsteg

Ett slutsteg med transistorer är kompakt och
skaktåligt och använder bara klenspänningar. Det är därför särskilt vällämpat för portabelt och mobilt bruk.
Men transistorer är känsliga för överbelastning. Redan ytterst kortvarig överbelastning eller överspänning kan förstöra dem.
Transistorer är också känsliga för termisk
överbelastning. Särskilt vid höga effekter i
trånga utrymmen är det nödvändigt med god
kylning, eventuellt med fläkt.

113-38

Ett slutsteg med elektronrör är inte så
skaksäkert, men är mycket okänsligare i
övriga avseenden. En nackdel är att det
behövs extra effekt för uppvärmning av rörens katoder samt höga anodspänningar,
som är farliga vid ovarsamhet. P.g.a. behovet av flera olika spänningar är även strömförsörjningen för ett slutsteg med elektronrör
mer komplicerad och omfångsrik.

Bestämning av PEP-effekten
Bild 113-52
Moduleringsspänningens topp-toppvärde
Uss mäts lämpligen med ett oscilloscope.
För korrekt belastning vid mätningen används en konstlast
Med topp-toppvärdet känt kan man med
följande formler beräkna
toppvärdet (amplituden

U - Uss

effektiwärdet

Us
Uett= {2.

s- 2

och

Effekten vid moduleringstopparna, s.k.
PEP (Peak Envelope Power), kan beräknas
med följande formler

ue~
k.
P,PEP =
R respe t1ve
P,
PEP=

Us~
SR

PA

os c i lloscope

Bild II 3-52 Bestämning av PEP-effekten

KRETSAR
linjäritetskontroll vid SSB

Bild II 3-53
Linjäriteten i en SSB-sändare kan kontrolleras med ett oscilloscop. Sändaren moduleras då med två övertonsfria toner.

slutsteget bör först belastas med konstlast
upp till max tillåten effekt. Resultatet jämförs
därefter med antennen som last.

SSB-TVATONSSl GNAL
Oscillogram

u

Spektrum

Spektrumanalys

(tid/spänn i ngsdiagra m l
normal SSB-bandbredd

-----t

l.lL 

f

Idealisk linjär förstärkning (klass A- utan överstyrning)

Nästan linjär förstärkning (klass AB- utan överstyrning)

Olinjär förstärkning - för låg vilaström

överstyrning (klippning)
SSB-SIGNAL VID TAL ("aaah")

Linjär förstärkning

Överstyrning (klippning)

Bild II 3-53 Unjäritetskontro/1 vid SSB
113-39

PT

KRETSAR
Linjäritetens betydelse i förstärkare
Bild II 3-54
Förstärkningen bör ske med god verkningsgrad och minsta möjliga förvrängning, så att
det alstras ett minimum av oönskade frekvenser inom minsta möjliga bandbredd.
Linjär förstärkning innebär att den är lika
över hela det aktuella frekvensområdet Frekvensgången måste därför vara så rak som
möjligt. Med tilltagande olinjäritettillkommer
nämligen allt fler oönskade frekvenser.
Det uppstår blandningsprodukter av högre ordning vid olinjär förstärkning. Genom
förvrängning p.g.a. olinjär förstärkning uppstår ömsesidiga summa- och skillnadsfrekvenser av de modulerande frekvenserna.

Varje sådan blandningsprodukt blandar
sig additivt och subtraktivt med grundfrekvenserna till ytterligare blandningsprodukter
av näst högre ordning.
Dessa är:
0
blandningsprodukter i LF-området och
deras övertoner, vilka undertrycks i efterföljande HF-krets,
e
grundfrekvenserna och deras harmoniska
övertoner, som alla ner till 1 :a harmoniska dämpas kraftigt av efterföljande
H F-krets,
0
alla summa-och skillnadsfrekvenser av
de förstnämnda frekvenserna.

l ngångssignal

l  l 
Utgångssignal vid linjär förstärkning

Utgångssignal vid olinjär förstärkning

Olinjär karaktäristik

Linjär karaktäristik

la

Ug

Bild II 3-54 Linjäritetens betydelse

113-40

Ug

KRETSAR
l området för nyttafrekvenserna kallas
dessa produkter för intermodulationsprodukter och ger talförvrängning.
Utanför nyttafrekvenserna uppfattas
intermodulationsfrekvenserna som störningar och kallas splatter. På grund av det
lilla frekvensavståndet till nyttasignalen kan
den intermodulation, som alstrats i slutsteget
inte filtreras bort i efterhand.
Vid linjär drift uppträder grundfrekvensernas övertoner och intermodulationsfrekvenser endast svagt inom och utom överföringsbandet och kommer knappast att uppfattas som inkräktande på annan radiotrafik.
De svaga övertonerna kommer också att
dämpas tillräckligt i n-filtret och eventuella
ytterligare övertonsfilter.

tion till utstyrningsgraden. ALG-spänning
återförs till drivsteget och reglerar dess uteffekt så att överstyrning av slutsteget inte
sker. l transistoriserade slutsteg skapas ALGspänningen genom likriktning av slutstegets
utspänning. l rörslutsteg börjar styrgallret
dra ström, när styrgallerspänningen blir positiv i signaltopparna, vilket används för att
styra ALG-spänningen. När ALG-regleringen
sätter in, är överstyrningen således redan ett
faktum. Överstyrning kan ske både på LFoch HF-nivå.
En orsak till övermodulering är för stor
amplitud på den modulerande signalen. Detta
kan bl.a. avhjälpas med inställning av
mikrofonförstärkaren och riktig mikrofonhantering.

Utstyrningskontroll av slutsteg

slutstegets linjära utstyrningsområde överskrids, om ingångssignalens amplitud blirför
stor. Då ökar utgångssignalens amplitud inte
mycket mer, men utgångssignalens toppar
blir tillplattade (s.k. klippta). Det betyder att
slutsteget är överstyrt.
Vid överstyrning uppstår signalförvrängningar, som medför intermodulation, förvrängt tal, splatterstörningar och övertoner.
Den extra effektökning som uppnås med
överstyrning förbrukas i stort sett till signalförvrängning och kommer inte nyttasignalen
till godo. Överstyrning skall därför undvikas.
Drivstegets uteffekt får inte vara så stor
att slutsteget blir överstyrt. Ett slutsteg med
jordad katod blir fullt utstyrt redan vid en
driveffekt av ett fåtal watt. Ar uteffekten från
drivsteget större, än vad som behövs för full
utstyrning av slutsteget, och driveffekten inte
kan regleras ner, så måste en dämpsats
kopplas in mellan drivsteg och slutsteg. En
sådan dämpsats kan behöva ta upp en betydande effekt, från en vanligt förekommande
amatöradiosändare upp till1 00 watt PEP.
Ett slutsteg med jordat galler fordrar en
större driveffekt, varvid risken för överstyrning i slutsteget är något mindre och de
förebyggande åtgärderna inte så omfattande.
Linjära slutsteg innehåller oftast en funktion kallad ALG (Automatic Load Gontrol),
som kontinuerligt känner av driveffektens
inverkan på slutsteget När driveffekten blir
för hög, alstras en kontrollspänning i propor113-41

KRETSAR

113-42

KRETSA
3.5 Detektorer - Demodulatorer
Allmänt

Sändaren omvandlar informationen i lågfrekventa signaler till högfrekvens som kan
strålas ut från en antenn. l mottagningsanläggningen återvandlas informationen, vilket kallas demodulation eller demodulering.
l mottagare som är specialiserade för ett
sändningsslag, används bara en typ av demodulator medan mottagare för flera sändningsslag, AM, SSB/CW, FM etc. har flera
demodulatorer. Det finns många typer och
namn på demodulatorer, t.ex. detektor, diskriminator. Här beskrivs några av dem.

AM-detektorer
Dioddetektorn AM (A3E)
Bild II 3-55

Bilden visar en superheterodynmottagaredär den sista M F-kretsen är induktivt kopplad till demoduleringsdioden. Den amplitudmodulerade M F-signalen visas som ett amplitud/tid-diagram.

Dioden klipper antingen de negativa eller
positiva halvvågorna, beroende på hur den
är vänd - polariserad.
LF-signalen filtreras ut ur de högfrekventa pulserna med ett LF-Iågpassfilter.
LF-signalen är nu överlagrad på en likspänning. l talpauserna sänds bara bärvågen och då lämnar AM-demodulatorn bara
likspänning, som skiljs från LF-förstärkaren
med en kondensator. Kondensatorn släpper
bara igenom LF-signalen, som förstärks.

Produktdetektorn SSB (J3E)
Bild 113-56
Det finns flera metoder att dernodulera en
SSB-signal, såsom fasningsmetoden, filtermetoden och den s.k. tredje metoden. Filtermetoden är numera är den allra vanligaste
och beskrivs här.
En SSB-signal med undertryckt bärvåg
består av endast ett sidband. Det andra
sidbandet och bärvågen undertrycks i sändaren.

Blockschema på en
AM-super (A3E)

frän sista MF-steget

u

MF

t
A3E- demoduleringsförlopp

Bild II 3-55 Dioddetektorn

113-43

LSB
l . USB
9001,5 ~ ~ 8998,5
kHz
kHz

SSB - mottagare {J3E)

Produktdetektor

u

MF

u

M F-fi lterkurva

SSB-Slgnal
( USB}

LF

==>300Hz, 1kHz och 3kHz

8998,8;
8999,5 och

9 001,5 kHz

u

u

MF

LF

BFO

~~----~~~~,------~f

8998,5;
9001,5
9000,5 och
kHZ
9001,2 kHz

•

300 Hz, 1kHz och 3 kHz

Bild 113-56 Produktdetektor för AM (A3E) och CW (A 1A)

113-44

ETSAR
Vid dernoduleringen av SSB-signalen
alstras i mottagaren en signal som ersättning för den bärvåg som undertrycktes i
sändaren. Det undertrycktaandra sidbandet
ersätts inte.
l en mottagare med direktblandning blandas SSB-signalen med VFO-signalen, varvid en del av blandningsprodukterna faller ut
på LF-nivå.
l en superheterodynmottagare däremot,
blir SSB-signalen först blandad med en VFOsignal och som resultat erhålls en mellanfrekvens MF. Den till MF omvandlade signalen förstärks, filtreras och blandas med en
lokal BFO-signal i ytterligare en blandare,
kallad produktdetektor. Några blandningsprodukterna faller ut på LF-nivå. Ett lågpassfilter följer därför efter detektorn för att filtrera
ut LF-signalerna
Numera består produktdetektorn vanligen av en ringblandare, som i ett omvänt
förlopp även kan användas vid DSB-modulering i en sändare. Bilden visar dernoduleringen av en SSB-signal som innehåller tre
LF-toner
CW-/SSB-detektorer CW (A 1A)
Även telegrafi-signaler, även kallat CW, blir
demodulerade när M F-signalerna och BFOsignalen blandas i en produktdetektor.
Till skillnad från SSB är det vid CW inte nödvändigt med en given skillnad mellan MFoch BFO-frekvenserna. Frekvensskillnaden
påverkar bara överlagringstonens frekvens,
men inte läsligheten av CW-budskapet.
Många moderna mottagare har en fast
BFO-frekvens för CW, som ger en 800 Hzton vid rätt frekvensinstälining. l stället för
lågpassfiltret för SSB, används ibland ett
bandpassfilter, som bara släpper igenom
CW-signaler i frekvensområdet 800 Hz -en
idealfrekvens för god läsbarhet av morsetecken.

FM- och PM-detektorer
Bild 113-57
Vid vinkelmodulering överförs informationen
enbart genom frekvens- eller fasvariationer
i bärvågen. De amplitudvariationer som kan
uppstå före dernoduleringen är ej önskvärda
i detta sändningsslag. Av den anledningen
finns i FM-mottagare en amplitudbegränsare (limiter) före diskriminatorn (se följande
bild). Frekvensvariationerna i den FM-modulerade signalen omvandlas därefter av
detektorn till LF-spänning som motsvarar
det utsända talet.
Bild 113-58
Dernoduleringen skall ske med mottagaren inställd mitt på avsedd sändarfrekvens.
Ett hjälpmedel för det är en indikator, som vid
rätt inställning visar värdet noll. Positivt eller
negativt utslag anger att inställningen är för
högt respektive för lågt i frekvens. En sådan
indikator fanns i tidiga FM-mottagare. Nu
används i stället en AFC (Automatic Frequency Contro l) som själv ställer in mottagaren omsändarfrekvensen är tillräckligt nära.
Slope-detektorn - Diskriminatorn FM (F3E)
Bild II 3-59
Två svängningskretsar är kopplade induktivt
till den sista M F-kretsen. Resonansfrekvensen för dessa båda kretsar är något högre
respektive något lägre än mellanfrekvensen. De signalspänningar som uppträder
över svängningskretsarna likriktas och seriekopplas med varandra med motsatt polaritet.
När de båda svängningskretsarna matas
med samma frekvens, kommer likspänningarna att ta ut varandra. När frekvensen avviker uppåt i frekvens, kommer kretsen med
den högre resonansfrekvensen i kraftigare
svängning än den andra kretsen och avger
högre likriktad spänning. När frekvensen
awiker nedåt i frekvens, skiftar de båda
kretsarna roller, och den resulterande likriktade spänningen skiftar till motsatt polaritet.
Vid växelvisa frekvensändringar i MF,
överoch undervilofrekvensen, blir resultatet
en växelspänning ut från likriktarnas utgångsfilter, som är LF-signalen.

113-45

KRETSAR
F M-mottagare ! F3E)

MF-förstärkare och

2 · ( L1 f max
= 2 · ( 3 kHz

+ fLFmax>'·,.      /
+

3 kHz)

= 2 · 6 kHz = 12 kHz

Bild II 3-57 Amplitudbegränsning vid FM-mottagning

u
variationer i utgängsspänningen =
LF-växelspänning

f
frekvensvariationer i
den frekvensmodulerade bärvägen

Bild II 3-58 Ideal arbetslinje för diskriminator

MF-mittfrekvens

u

fres2

"'-...

''

'\

fres1

fres1

'

\

\

\

''

f

' ' ....
sista MFsvängkrets

Bild II 3-59 Slope-detektorn

113-46

till
L F-förstår kara

f res 2

ETSAR
Foster-Seeley diskriminatorn
Bild 113-60
Denna tidiga dernodulater har god linjäritet,
om den föregås av en god amplitudbegränsare, men har tämligen dålig känslighet.
Sista MF-förstärkarsteget avslutas med
en transformator vars båda lindningar ingår
i svängningskretsar avstämda till MF. MFsignalen överförs från primär- till sekundärsidan dels med induktion och dels med en
kondensatortill mitten av sekundärlindningen. Signalen delas på så sätt i två grenar
med en fasförskjutning av +90$\circ$ resp. -90$\circ$.
Signalerna i grenarna likriktas varför sig och
sammanlagras i ett RC-nät.
Om M F-signalen inte devierar så är LFspänningen i grenarna lika, men eftersom
grenspänningarna har motsatt polaritet så
tar de ut varandra och LF-signalen blir noll.
När M F-frekvensen devierar av modulering,
så ökar signalamplituden i den ena grenen
och minskar i den andra. LF-signalens amplitud blir då proportionell med frekvensdeviationen.

svängkrets

Bild II 3-60 Foster-Seeley detektorn

MF-förstärkare

och begränsare

från blandan~

Jl

Räknardiskriminatorn
Bild II 3-61
En mono-flopvippa påverkas att slå över av
fyrkantpulserna från de amplitudbegränsade
FM-signalerna.
En sådan vippa är en digitalkoppling som,
när den matas med en godtyckligt lång spänningspu!s, ändå kommer att leverera en spänningspuls med konstant längd. För varje
positiv halvvåg leverar mono-flopvippan en
impuls av konstant längd. Tidsavståndet
mellan pulserna kommer att vara proportionella till FM-signalens frekvens. Vid varierande frekvens kommer impulserna med
varierande tidsavstånd. Ett lågpassfilter filtrerar ut lågfrekvensen ur signalen och en
pulserande likspänning kvarstår. Med denna likspänning laddas kondensatorn upp till
ett medelvärde. Vid en högre frekvens av
lika långa pulser blir medelvärdet högre än
vid en lägre pulsfrekvens.
Svängningarna på likspänningen är LFsignalen, överlagrad på en likspänning. Utan
en mono-flopp med lika långa pulser hade
medelvärdet varit konstant. Man kan säga
att FM-signalen blivit omvandlad till en PLMsignal (pulslängdmodulerad signal).
PLL - demodulatorn
Bild II 3-62
Den frekvensmodulerade MF-signalen och
en VCO-signal matas in i en fasjämförare.
V GO-frekvensen följer frekvensändringarna
FM-signalen Avstämningsspänningen för
VCO är en likspänning. Den modulerande
LF-spänningen är överlagrad på denna likspänning.

monaflop

amptitud·
begränsad
FM-signal
medelvärde

Bild fl 3-61 Räknardiskriminatorn
113-47

KR
M F-förstärkare
och begränsare
frän blandare

avstämningsspänning

Bild 113-62 PLL-demodulatorn
LF-frekvenserna är för låga för att kunna
reglera VCO-frekvensen, men via en kondensator kan de styra LF-förstärkaren.
De båda sista metoderna lämpar sig speciellt för demodulering av F1-signaler. F1signaler kan "demoduleras" i en SSB-mottagare eftersom det uppstår en rytmisk svängning i tonhöjden. Denna frekvensmodulerade
ton kan sedan dernoduleras på sätt som
beskrivits.
Det finns ytterligare sätt att dernodulera
FM-signaler. Gemensamt för alla är, att de
fungerar bättre ju lägre mellanfrekvensen är.
Därför utförs de flesta FM-mottagare som
dubbel- eller trippelsuprar, med låg MF.

113-48

KRETSAR
3.6 Oscillatorer

Alstring av svängningar
Ordet aseiiiare (lat.) har betydelsen svänga
och den företeelse eller anordning som skapar en svängning kallas oscillator. Vid alla
slags svängningar sker växelverkan mellan
olika energiformer. Svängningar förekommer i olika former. Det kan vara vibrationer i
en kropp, en pendel som svänger, rörelser i
gaser och vätskor, elektriska laddningar i en
strömkrets o.s.v.
Mekanisk pendel
Bild II 3-63
Energiinnehållet i en pendel växlar mellan
lägesenergi (potentiell energi) och rörelseenergi (kinetisk energi). Lägesenergin är
störst i pendelns ytterlägen och minst i mittläget. Omvänt är rörelseenergin störst i mittläget och minst i ytterlägena.

Stöt till en pendel bara en gång så att den
börjar pendla, men utslagen blir allt mindre.
Pendeln utför en dämpad svängning därför
att det förbrukas energi under pendlingen.
Dämpningen kommer av att energi förloras av friktionen i upphängningspunkten och
av luftmotståndet.
Stöt nu till pendeln varje gång, som den
pendlar tillbaka. För lika stora utslag varje
gång- en odämpad svängning- fordras det
upprepade energitillskott som precis kompenserar förlusterna.
Villkoren för att en svängning skall fortgå
(vara odämpad) är att energitillskotten
• kommer vid rätt tidpunkt,
• har rätt riktning- polaritet,
• kompenserar förlusterna.

PENDEL

\

l
Epot

A

t
DAMPAD SVÄNGNING

ODÄMPAD SVÄNGNING

A

=

utslag från viloläge

Bild II 3-63 Svängningar

113-49

KRETSAR
Alstring av mekaniska svängningar
Bild II 3-64
En elektrisk ringklocka med självbrytande
kontakt är ett exempel på en enkel elektromekanisk oscillator. Denna bild visar hur
kontakten ersatts med ett elektronrör. En
mer "tidsenlig" lösning med transistor hade
naturligtvis också kunnat användas.
När anodspänningen kopplas till, så börjar anodström att flyta genom trioden från
katod till anod och genom elektromagneten.
Magneten drar då till sig bladfjädern, som
kopplar styrgallret till en negativ förspänning.
Den negativa förspänningen stryper anodströmmen och magnetfältet upphör. Bladfjädern släpper då från magneten och kopplar
bort styrgallret från förspänningen. Anodström börjat att flyta igen, varvid elektromagneten drartill sig bladfjädern o.s.v. Förloppet
kallas självsvängning.
Anordningen som alstrar svängningarna
kallas generator eller oscillator. Frekvensen
är det antal svängningar per sekund som
oscillatorn alstrar, i detta exempel bladfjäderns svängningshastighet.

Amperemetern
visar den pulserande
anodströmmen, som är en likström.
Amperemetern A 2 visar växelströmmen
svängningskretsen.
1 Halvvågen

Ua
POSITfVT GALLER:
- i anodkretsen flyter en ström
{A1 visar mot höger)
- i svängningskretsen flyter en ström
(A 2 visar mot höger)

2 Halvvågen

....--~

~--------~+

U9

,A1

,

-F-------~

V
~----~-

Bild II 3-64 Elektromekanisk oscillator

Alstring av elektriska svängningar
Bild II 3-65
En elektrisk svängningskrets är motsvarigheten till ett mekaniskt föremål i svängning.
Bilden visar en elektronisk oscillator med en
LC-krets i anodkretsen till en triod och som
är induktivt återkopplad till styrgallret

113-50

t

+~--~

Ua
NEGATIVT GALLER:
- ingen ström flyter i anodkretsen
(A 1 visar noll)
- i svängningskretsen flyter en ström
i motsatt riktning
(A2 visar mot vänster)

Bild 113-65 Elektronisk oscillator (Meissner)

KRETSAR
En elektronisk oscillator är en förstärkare, vars utsignal återförs till ingången så
att förstärkaren råkar i självsvängning -det
blir en s.k. positiv återkoppling.
Elektroniska oscillatorer används både i
mottagare och sändare. Funktionsprincipen
är lika i båda fallen. Vad som möjligen skiljer
är användningssättet Det finns många oscillatorkopplingar varav några beskrivs här.
Självsvängning kan demonstreras akustiskt med en mikrofon, en förstärkare och en
högtalare. Resultatet blir att det genereras
en ton. Ljudet från högtalaren är tryckvågor
som påverkar mikrofonen och omvandlas
där till elektriska signaler. Dessa går genom
förstärkaren tillbaka till högtalaren och omvandlas åter till tryckvågor som mikrofonen
uppfattar o.s.v. Det uppstår självsvängning,
ett tjut, som beror på akustisk återkoppling.
Om förstärkaren kompletteras med ett frekvensfilter i form av en svängningskrets, så
blir "tjutet" i stället en ton med samma frekvens som filtrets resonansfrekvens.

Bild II 3-66 Oscillator enligt Meissner
Bild II 3-67
Förstärkaren kan t.ex. vara en emitterkopplad transistorförstärkare enligt bilden.
Kopplingskondensatorerna Ck är nödvändiga för att förhindra kortslutning av de likspänningar som pestämmer arbetspunkten
för transistorn. A andra sidan kan växelspänningssignalerna passera till och från
transistorn.

:r
ck

r-

.,.--1}-----o.
utgång

LC-oscillatorer
Variabel frekvens oscillator- VFO
En oscillator med inställbar frekvens kallas
för VFO (variabel frekvensoscillator). Förutom frekvensstabilitet fordras också, att
noggrann inställning och avläsning av frekvensen skall kunna göras.
En Le-oscillator är urtypen för en oscillator med variabel frekvens. Meissner-kopplingen är lätt att urskilja och används här för
att beskriva grundprincipen för en oscillator
i stort. Bl.a. Colpitts- och Glapp-kopplingarna har emellertid bättre stabilitet och
inställbarhet i återkopplingsledet
Meissner-koppling
Bild 113-66
Bilden visar en Meissner-oscillator, som består av en LC-svängningskrets med återkopplingsspole och en förstärkare. Magnetfältet mellan induktansen i svängningskretsen och återkopplingsspolen är polariserat
så att en förändring i utsignalen medverkar
till självsvängning. (Motsatsen är motkoppling.)

f
Bild II 3-67 Emitterkopplad förstärkare
Bild II 3-68
Återkopplingsvägen görs i detta fall så, att
svängningskretsen kopplas parallellt över
förstärkaringången. Aterkopplingsspolen
fungerarsom förstärkarens kollektorresistor.

r

Bild II 3-68 Komplett Meissneroscillator

113-51

KRETSAR
Självsvängningsvillkoret

Bild II 3-69
Självsvängning i en förstärkare uppstår genom återkoppling. signalspänningen ain över
ingången blir förstärkt med faktorn A. När
som i bild 113-68 förstärkaren är emitterkopplad, blirutsignalen fasvriden 180$\circ$ i förhållande till insignalen. Fasvridningen a=180$\circ$ betecknas här med minustecken, alltså blir
förstärkningen -A.
På förstärkarens utgång fås en signalspä~ning out ~ed sambandet
Uut

=-A.

uin

En del av utsignalen återförs (återkopplas) till ingången. l en Meissner-oscillator
sker återkopplingen med en induktor, som är
induktivt kopplad till svängningskretsens induktor.
Kvoten k mellan den återkopplade signalspänningen ok och signalspänningen out
på förstärkarens utgång kallas återkopplingsfaktor. Den återkopplade spänningen
ok fasvrids så att den kommer "i fas med "
med insignalen. För den återkopplade signalen fås då sambandet

ok= -k. out

Tillräcklig signalspänning från utgången
måste återföras till ingången för att det skall
uppstå självsvängning. Det sker när den
återkopplade signalspänningen ak är minst
lika stor som ingångsspänningen Din och är i
rätt fasläge, d.v.s. i detta exempel
Ok~ Din
eller -k· Du!A eller k~ 1/A
Självsvängningsvillkoret blir
k~ 1l A eller k· A ~ 1
Ett k· A ::::: 3 är önskvärt för att oscillatorn
skall svänga igång snabbt.

Bild II 3-69 Svängningsvillkoret
113-52

Hartiey-koppling

Bild II 3-70
Återkopplingen sker galvaniskt över ett uttag
på induktorn i oscillatorns LC-krets.

Bild II 3-70 Hartiey-koppling
Huth-KO hn- eller TGTP-koppling (tuned grid
- tuned plate)

Bild II 3-71
Kopplingen är en förstärkare med LC-kretsar både på in- och utgång. Båda ~retsarna
är avstämda till samma frekvens. Aterkopplingen sker över de inre kapacitanserna
mellan elektronrörets elektroder resp. mellan transistorns materialskikt Denna koppling är av flera skäl inte särskilt vanlig.

{I];

----+---

Y. C>
Bild /13-71 TPTG-koppling
Golpitts-koppling

Bild 113-72
Återkopplingen sker över en kapacitiv spänningsdelare, som ingår som en del av oscillatorns LC-krets.

Bild II 3-72 Golpitts-koppling

Glapp-koppling

Bild 113-73
Denna koppling är en variant av Colpittskopplingen. Vridkondensatorn för frekvensinställningen är seriekopplad med spänningsdelarens kondensatorer. Glapp-oscillatorns
frekvensstabilitet är god.

Bild II 3-73a Glapp-koppling

Vi utvecklar denna beskrivning vidare.
Vridkondensatorn samt en fast och en trimningsbar kondensator är kopplade parallellt
med varandra. Alla tre kondensatorerna är i
sin tur seriekopplade med den kapacitiva
spänningsdelaren C3 C4 • Förstärkarens ingång är kopplad till den övre anslutningen av
C3 • Utgången från oscillatorns förstärkare
återkopplas över dämpresistorn Rct till mitten
av spänningsdelaren c3c4 (återkopplingskretsen).

..---+----+e v

Bild II 3-73b Förstärkare i Glappkoppling

Förstärkarens arbetspunkt bestäms av
spänningsdelaren R1 R2 • Ingen kopplingskondensator behövs eftersom det enbart
finns kondensatorer mellan förstärkaringång
och jord.
Kondensatorn C6 avkopplar kollektorn på
transistor T1 HF-mässigt till jord. Förstärkaren är alltså kollektorkopplad.

Kondensatorn C7 kopplar oscillatorns utsignal till buffertsteget För frekvensstabilitetens skull stabiliseras spänningen 8 V med
en lC-krets som avkopplas HF-mässigt med
en kondensator.

Frekvensinställning och bandspridning
Bild II 3-74
Att ställa in frekvensen i en Le-oscillator
gjordes förr oftast med en vridkondensator.
l moderna mottagare och sändare används
i stället en s.k. varicap, som styrs med en
likspänning.
Med en svängningskrets med endast en
induktor och en vridkondensator, skulle alla
amatörradiobanden endast vara smala områden utspridda på en mekanisk skala, d.v.s.
över vridkondensatorns hela kapacitansområde, varvid kapacitansen kan varieras med
förhållandet 1:5 a 1:1 O, tex. 10-50 pF a 10100 pF.
För att i stället få vart och ett av amatörradiobanden utspridda över större delen av
skalan kan man ordna med bandomkoppling
och s.k. bandspridning. Man parallellkopplar
då en relativt stor fast kapacitans med vridkondensatorns relativt lilla kapacitans. Den
totala kapacitansvariationen i LC-kretsen blir
då liten, trots att kondensatorns hela kapacitansområde utnyttjas. Resultatet blir en frekvensskala med större upplösning, d.v.s. bättre avläsningsnoggrannhet
Bandspridning kan också ordnas med
två seriekopplade kondensatorer, varav den
större görs variabel. Typiskt värde på vridkondensatorn i en kortvågsutrustning är då
100-500 pF och den fasta kondensatorn
mycket mindre än så.

nrno
Bild II 3-74 Bandspridning

113-53

113-54

ETSA
3.6 Kristalloscillatorer
Kvartskristaller i oscillatorkopplingar

En Le-oscillators frekvensstabilitet begränsas av de ingående komponenternas egenskaper. När mycket bättre stabiltet än så
krävs, speciellt inom stora temperaturområden, så är kvartskristallen en svängningskrets med bättre data. Kvartskristallens höga
Q-värde ger också en renare signal.
l en kristalloscillator (CO, Grystal Oscillator) är en kvartskristall det frekvensbestämmande elementet i stället för en LC-krets. l
övrigt kan samma kopplingsprinciper som
för en LC-VFO användas.
Kristallen kan utföras så att den svänger
antingen som en serie- eller parallellresonanskrets. Märk, att en kristall svänger på
något olika frekvens beroende på om den fås
att fungera som serie- eller parallellkrets.
Den högre frekvensen är den som vanligen
används.

Bild II 3-75
l parallellresonansalternativet kopplas
kristallen parallellt över oscillatorns återkopplingsled. Den minsta dämpningen av
den återkopplade signalen fås när signalens
frekvens är lika kristallens resonansfrekvens.
Kristallens reaktans är då som högst.
Parallellt över kristallens inre induktans
ligger dess inre seriekopplade kapacitanser
C och Cw Yttre kapacitanser (en trimbar och
två fasta kondensatorer i serie) är kopplade
parallellt över den inre anslutningskapacitansen Cw
Om den trimbara kapacitansen ändras,
så påverkas kristallens resonansfrekvens.
Man säger då att man "drar" kristallen inom
ett litet frekvensområde. Kristallens och oscillatorns egenskaper avgör hur stort området kan vara. Om kristaller dras för mycket,
så kan resonansfrekvensen bli ostabil.
Den relativa frekvensändringen uppgår
till högst 1o-4 = 0.01 $\circ$/o. Formel:

t, f

re a

IV

.. d .
absolut ändring
re vensan nng=
~ k
resonans1re vens

~ k

Övertonskristaller
Bild II 3-76
l serieresonansalternativet kopplas kristallen in i serie med oscillatorns återkopp!ingsled. Den minsta dämpningen av den
återkopplade utgångssignalen fås, när signalens frekvens är lika som kristallens resonanfrekvens. Kristallens reaktans är då som
lägst. S.k. övertonskristaller används för oscillatorfrekvenser över ca 20 MHz.

Bild II 3-75 Golpittsoscillator med kristall
i parallellresonansfallet

Bild II 3-76 Golpittsoscillator med kristall
i serieresonansfallet
113-55

KRETSAR

PT

Övertonskristallernas dimensioner är lika
grundtonskristallernas, men snittas ut annorlunda och slipas för att svänga på önskad
udda överton. En övertonskristall har övertonens frekvens instäm pi ad i höljet och kristallen förutsätts arbeta i oscillatorkopplingar
som seriekrets. Genom att låta kristaller
svänga på sin överton undviker man en svår
tillverkningsprocedur, nämligen att slipa
mycket tunna kristallskivor.
En övertonsoscillator måste alltid innehålla en svängningskrets som är avstämd till
den överton som anges på kristallen.
Modellförsök: En instrumentsträng sätts i
svängning på sin grundton genom en knäppning mitt på strängen. En knäppning på en
punkt bort från mitten får strängen att svänga
på en överton i stället.
Superheterodyn-VFO
Bild II 3-77
En enkel LC-VFO är inte tillräckligt frekvensstabil i ett högt frekvensläge, t.ex. 144-146
MHz. Man kan då använda en speciell koppling, som är en kombination av LC-VFO och
CO, kallad super-VFO.
l en super-VFO blandas en låg, variabel
frekvens från en VFO med en hög frekvens
från en CO. Ordet super kommer från
superheterodyne = överlagring, blandning.
En VFO arbetar stabilare på låg frekvens
medan en CO fortfarande arbetar stabilt
även på högre frekvenser, dock inte så högt
som vi behöver här. l vårt exempel arbetar
därför VFO i området 8-1 O MHz och CO på
17 MHz. VFO-signalen blandas med en fast
signalfrekvens, som är GO-signalen 17 MHz
multiplicerat med 8, d.v.s. 136 MHz.

VFO 8 -

10 MHz

co 17 MHz
Bild II 3-77 Superheterodyn- VFO

113-56

Ett bandpassfilter filtrerar fram den önskade blandningsprodukten, som ligger i
frekvensområdet 144-146 MHz. Resultatet
blir en hög frekvens, som är både variabel
och stabil.
Fördelar:
Frekvensstabiliteten hos en super-VFO
är mycket bättre än hos en enkel VFO, som
arbetardirekt i VHF-området. En super-VFO
är dessutom mycket brusfattigare än en
PLL-VFO, vilken beskrivs här nedan.
Nackdelar:
Vid frekvensblandning uppstår oönskade blandningsprodukter, vilka visserligen
dämpas av bandpassfilter, men som det är
omöjligt att undertryckta helt. Bl. a. alstras en
svag spegelfrekvens, som vandrar från 128
till 126 MHz, samtidigt som den önskade
blandningsprodukten vandrar från i 44 till
i 46 MHz. Risken för att spegelfrekvensen
förstärks och sänds ut måste elimineras,
vilket kan göras med effektiva bandpassfilter. Se vidare i avsnitt 3.8 om frekvensblandni ng.

3. 7

Oscillatorer med faslåsning - Pll

En kristalloscillator (CO) arbetar med god
frekvensstabilitet Frekvensen som är fast
bestäms av styrkristallen.
En LC-oscillator arbetar däremot inom
ett frekvensområde (VFO), som bestäms av
en LC-krets. Dennas frekvens är emellertid
mindre stabil än den med styrkristalL
l en PLL (Phase Locked Loop) kan god
frekvenstabilitet och stort frekvensområde
förenas. En PLL är en sluten krets för elektrisk styrning av en oscillator, så att dess
frekvens är både stabil och variabel.

Spänningsstyrd oscillator (VCO)
Bild 113-78
En VFO, vars frekvens kan styras med en
likspänning, kallas VCO (Voltage Controlied
Oscillator). l svängningskretsen i en VCO
fyller en kapacitansdiod (varicap, variable
capacitor) samma uppgift som den mekaniskt variabla kondensatorn i en VFO.
Bild 113-79
När en motriktad spänning läggs på dioden, så bildas ett spärrskikt i dioden, så att
zonerna med fria laddningsbärare isoleras
från varandra likt kondensatorplattor. Spärrskiktets tjocklek (c:a 1/1000 mm) beror av
spänningen över dioden. Vid hög spänning
är spärrskiktet tjockt, vilket motsvarar "stort
plattavstånd" och liten kapacitans. Vid låg
spänning är skiktet tunt, vilket motsvarar
"litet plattavstånd" och stor kapacitans.
Med en kapacitansdiod i svängningskretsen, i stället för en mekaniskt variabel kondensator, så behövs ytterligare två komponenter. D rossel n Dr hindrar högfrekvenssignalen att överlagras på styrkretsens likspänning. Då skulle bl.a. svängningskretsens godhetstal försämras (förlorad HF-energi innebär dämpning). Omvänt hindrar kondensatorn C att dioden och spärrspänningen kortsluts genom induktorn. Oscillatorfrekvensen ställs in med den variabla likspänningen U. Av en VFO har det blivit en VCO.

Bild II 3-78 VFO och VCO jämförs
Oscillator med PLL-styrning
Bild II 3-80
Människan jämför och reglerar förlopp utifrån givna fakta. Det kan liknas med PLLkretsens sätt att jämföra det inbördes fasläget mellan signalen från en VCO (är-värdet)
och signalen från en CO (bör-värdet).

......---:=::::;:::=! M .

. . Jiijo MHZJ/
Jamfora .
,
Låg spänning

stor kapacitans

Hög spänning
liten kapacitans

Varicap·
diod

Bild 113-79 Kapacitansdiod- Varicap

ata

:.~,:..c ~~go' O
Bild II 3-BOa Analogi

Människa-PLL

113-57

KRETSAR
Som resultat av jämförelsen justeras styrspänningen så att är- och bör-frekvenserna
hålls lika. En sådan reglerkrets består av
digitala komponenter.
Fasjämföraren levererar en cykliskt justerad styrspänning till kapacitansdioden i
VCO. Eftersom denna spänning ändras
språngvis, så avrundas förloppet så att frekvensändringarna blir mjuka. Avrundningen
sker med ett RC-filter där kondensatorn antar ett medelvärde av den pulserande utgångsspänningen frånjämföraren. Om VGOfrekvensen är för låg, så levererar jämföraren en positiv spänning. styrspänningen på
kapacitansdioden stiger då med en hastighet som bestäms av filtrets tidskonstant

Kapacitansen i kapacitansdioden minskar med ökande spänning, eftersom spärrskiktet blir tjockare och frekvensen på VCO
stiger.
När signalen från VCO åter är lik referenssignalen från CO, till fasläge och frekvens, så ökar utgångsresistansen i fasjämföraren. Lågpassfiltrets kondensator behåller
då sin laddning och styrspänningen till VCO
ändras inte t.v. Skulle frekvensen på VCO
vara för hög så blir jämförarens utgång
lågohmig och filtrets kondensator urladdas
med den hastighet som bestäms av tidskonstante n. Den sjunkande styrspänningen
medför att kapacitansdiodens spärrskikt blir
tunnare, kapacitansen tilltar och VCO-frekvensen sjunker tills en
ny fas- och frekvenslikhet uppnåtts.

~--------~--~----~

rv Utfrekvens

Lågpassfi !ter

fvco < fco

Referensfrekvens
(bör-värde)

fvco > fco

vco~~

co~~

VCO-si gnat och GO-signal, formad som kantvågor att jämföras i fasläge
Ua  

----.~

u
VCO-frekvensen för låg

VCO-frekvensen för hög

Bild 113-BOb Oscillator med PLL-styrning

113-58

PLL-oscillator i kombination med frekvensblandning
Bild II 3-81
Signalen f 1 från en VCO
alstrar en sändningsfrekvens i bandet 144146 MHz. Denna blandas med signalen f 2
(136 MHz), som är en
multiplicerad GO-frekvens. Blandningsprodukten f 1 - f2 filtreras
fram, d.v.s. en signal i
området 8-1 OMHz som
påförs en fasjämförare.
Utsignalen från en VFO,
som är variabel inom
samma frekvensområde 8-1 O MHz, påförs
också fasjämföraren.
Utsignalen från jämföraren är en likspänning, som beror av frekvensskillnaden mellan
blandningsprodukt och
VFO-signal. Jämförarens utsignal ändras
uppåt eller nedåt, beroende på frekvensfelets
riktning.

KRETSAR
V GO-frekvensen bestäms av en likspänningsnivå, som styrs av jämförarens utsignal. Vid varje frekvensändring i VCO, kommer systemet att sträva mot frekvensskillnaden noll i fasjämföraren vilket gör att sändningsfrekvensen hålls vid rätt värde.
Fördelar med en PLL-oscillator: Den har
samma frekvensstabilitet som en VFO eftersom denna även här arbetar på en låg frekvens. Till skillnad mot en super-VFO finns
inga sidafrekvenser i PLL-oscillatorn, eftersom VCO alstrar nyttafrekvensen direkt.
Nackdelar med en PLL-oscillator: Den har
högre brusnivå än en super-VFO. Frekvensstabiliteten är sämre än den för en PLLoscillator med CO och programmerbar
frekvensdelare.
PLL med programmerbar frekvensdelare
Bild II 3-82
Med PLL blir frekvensen på utsignalen från
en VCO låst till referensfrekvensen från en
CO. l princip fås en VCO med samma frekvensstabilitet som en CO, men också lika
svår att ändra frekvensen på. Med en frekvensdelare i fasregleringsslingan (PLL) kan
emellertid utfrekvensen ändras, medan CO
fortfarande avger samma referensfrekvens.
En frekvensdelare är en digital krets, som

räknar svängningar eller pulser upp till ett
valt tal för att återställas till i och börja om
igen. Vid varje återställning avges en utpuls.
Vid en delning med två avges en utpuls för
varannan inpuls. Vid delning med i 5 avges
en utpuls för var i 5:e in puls o.s.v.
Genom att välja delningstal i PLL kan
arbetsfrekvensen i VCO ställas in stegvis,
där varje steg är så stort som en referensfrekvens. signalfrekvensen från vco delas
med det valda delningstalet och resultatet
jämförs med referensfrekvensen från CO.
Varje avvikelse från likhet med referensfrekvensen kommer att medförajustering av
V GO-frekvensen.
Om man t. ex. vill täcka 2-metersbandet i
steg om 25 kHz, så väljer man den referensfrekvensen samt att delaren kan fås att dela
sändarens utfrekvens med vilket som helst
av talen 5760, 576i, 5762 o.s.v. upp till
5840. Om t.ex. delningstalet 5820 valts, så
kommer jämförarens styrspänning att styra
V GO-frekvensen till i 45500 kHz. Delarens
utfrekvens blir då i 45500/5820 = 25 kHz,
vilket motsvarar referensfrekvensen. l detta
exempel styrs alltså sändarens utfrekvens
så att den alltid blir i steg om 25kHz.

vco

144 - 146 MHz
.........
144 - 146 MHz
~~-------------------r--------------------Q

f avstämningsspänning

[Q

8 - 10 MHz
uppblandad VCO-frekvens

8- 10 MHz
referensfrekvens
LP-filter

fasjäm·
förare

d

~
VFO

Bild 113-81 PLL-oscillator kombinerad med frekvensblandning

113-59

KR
Men PLL-oscillatorn
brusar
förhållandevis
144 MHz till 146 MHz
starkt jämfört med en
VCO G !--------.---------·-------<>
VCO och speciellt jäm"""-'
rv 144 till 146 MHz
fört
med en CO. VCO}-----<>
svängningskretsen har
nämligen ett relativt lågt
Programmerbar delare i med tumhjul
n ( 5760 till 5840)
f eller annat
Avstämnings
godhetstal eftersom en
spänning
kapacitansdiod belastar kretsen mer än en
mekanisktvariabel kondensator.
Med det lägre god25kHz
hetstalet blir svängLågpass f i !ter
Referensfrekvens
ningskretsen ett mindre
bra filter för dämpning
Arbetsfrekvens Delningstal n Referensfrekvens
av oscillatorbruset Kapacitansdioden tillför
144 000 kHz
: 5760
= 25 kHz
: 5761
= 25 kHz
144 025 kHz
dessutom ett elektron145 500 kHz
: 5820
= 25 kHz
brus. Därtill kommerdet
146 000 kHz
: 5840
= 25 kHz
s.k. fasbruset från frekvensdelaren och PLL
VCO-frekvens
Med svängningskretsens låga godhetsVCO-frekvens,
tal är frekvensstabilite4-kantformad
ten i en VCO inte så bra
som den i en kristallosVCO-frekvens,
j 2-delad
cillator, utan faktiskt
sämre än den i en VFO.
Trots det är långtidsBild II 3-82 PLL med frekvensdelare
stabiliteten god i en
VCO, närden ingår i en
PLL, eftersom att frekvensen hålls ständigt
För- och nackdelar med PLL -oscillatorn
efterjusterad. PLL kan däremot inte åstadPLL-oscillatorn har nästan samma frekvenskomma en lika bra korttidsstabilitet Ett fasstabilitet som en kristalloscillator och frekvensen är inställbar i steg. Till skillnad mot jämförelseförlopp omfattar ju redan tiden för
en period av referensfrekvensen. Och det
en VFO med mekaniskt inställbar frekvens,
kommer att förflyta en multipel av denna
så är den PLL-styrda VGO-oscillatorns frekvens elektroniskt inställbar. Detta underlät- kortaste tid innan styrspänningen kan återställa VCO-frekvensen igen. Detta beror på
tar utformning och placering av reglage etc.
att kondensatorn i regleringsslingans lågför frekvensinställning, frekvensminne och
passfilter först måste laddas upp under ett
automatisk frekvensavsökning.
antal perioder innan reglering sker.
Först när den PLL-styrda oscillatorn kom
Dessa kortvariga frekvensavvikelser är
till användning i handapparater och mobila
en typ av frekvensmodulation som leder till
apparater blev det möjligt med frekvenstäckning över ett helt band med bibehållet fasbrus från PLL-oscillatorn och som kan
störa. Det är dock endast i extrema fall som
krav på små dimensioner. Som jämförelse,
skulle en inbyggnad av säg 80 till800 stycken fasbruset verkar störande eftersom det i
kanalkristaller i en traditionell kristallstyrd
moderna apparater reduceras till en accepapparat vara en mycket platskrävande, dyr- tabel nivå genom noggrann skärmning och
filtrering.
bar och opraktisk lösning.

1\ J\ 1\ ~1\ ;

r-uLflIlj
LJL 

113-60

Faktorer som påverkar frekvensstabilitet

Sändarens frekvens skall hållas så stabil
som möjligt. En ostabil sändare är inte godtagbar och skapar svårigheter inte bara för
de radiostationer som deltar i förbindelsen
utan även för radiotrafiken på närliggande
frekvenser.
En frekvensstabil oscillator skall ha följande
egenskaper:
stabil mekanisk uppbyggnad
Skakningar från underlaget t.ex. vid mobilt
bruk, vibrationer från en transformatorkärna
etc. kan försämra oscillatorns frekvensstabilitet
Frekvensbestämmande komponenter såsom fasta och variabla kondensatorer, spolar etc skall vara stabilt monterade, trimkärnorna i spolarna fixerade o.s.v.
Förbindningarna får inte tillåtas att böja
sig eller vibrera. Apparatstommen måste
vara tillräckligt styv för att inte ändra formen
och därigenom medföra frekvensändringar
vid hantering o.s.v.
God elektrisk uppbyggnad och högt Q-värde

i svängningskretsarna

Alla elektriska förbindningar måste vara så
korta som möjligt och löd- och kopplingsställen fullgoda. Induktorer och kondensatorer i
svängningskretsarna måste vara förlustfattiga och högvärdiga i övrigt så att signalen
blir så ren som möjligt från oönskade sidafrekvenser.
Återkopplingen i oscillatorn skall vara så
fast (kraftig) att självsvängningen är stabil.
Men för att få en renast möjlig signal får
kopplingen inte vara så fast, att svängningskretsarna blir alltför belastade och deras
godhetstal för lågt.
Avskärmande kapslingar
Svängningskretsar skall skärmas från yttre
kapacitanstillskott (t.ex.från en hand) Det
görs med skiljeväggar och komponentkapslingar av metall. Skärmningarna förhindrar också oönskad koppling mellan oscillatorn och efterföljande förstärkare genom elektriska och magnetiska fält.

stabila drivspänningar
Ostabila drivspänningar medför frekvensändringar. l en oscillator med transistorförstärkare beror ostabiliteten på förändringar
mellan skikten i en transistors diodsträcka.
Skikten fungerar nämligen som "kondensatorplattor" och spärrskiktet där emellan som
dielektrikum. Tjockleken av spärrskiktet och
därmed "plattavståndet" står i förhållande till
den spänning som läggs över transistorn.
Den spänningsberoende kapacitansen i transistorn är ansluten till svängningskretsen via
kopplingskondensatorn.
Eftersom kapacitansen i transistorn är en
del av svängningskretsen, så påverkar den
resonansfrekvensen. Denna egenskap kan
vara till besvär, men kan även användas för
att på ett enkelt sätt ändra oscillatorns arbetsfrekvens.
Se Kapacitansdiod och PLL-oscillatorn.
Buffertsteg
En oscillator i en radiosändare kan bestå av
ett enda förstärkarsteg, som alstrar högfrekventa elektriska svängningar. Vanligen tas
endast små effekter ut från en så enkel
sändare, normalt mindre än en watt. Utan
särskilda åtgärder, som t. ex. att använda en
styrkristall, är nämligen frekvensen inte särskilt stabil och olämplig för kommunikationsändamål.
Särskilt varierande belastning över oscillatorns utgång medför frekvensändring. Oscillatorn bör därför ges en så låg och stabil
belastning som möjligt. Ett buffertsteg kopplas därför in efter oscillatorn. Det bör ha hög
ingångsimpedans för att belasta oscillatorn
så lite som möjligt. Det skall också kunna
lämna tillräcklig driveffekt till efterföljande
förstärkare och bör därför ha låg utgångs impedans. Det måste dessutom arbeta linjärt
(se klass A-drift, bild 113-44) för att inte alstra
övertoner och därmed förvränga oscillatorsignalen. Bild II 3-42 visar ett buffertsteg i
kollektorkoppling, vilken har dessa egenskaper.

113-61

KRETSAR
Temperaturkompensation och termostater
Det alstras alltid förlustvärme i elektriska
apparater och även i en oscillator. Vid uppvärmningen utvidgas spolar och kondensatorer i svängningskretsarna, vilket leder till
frekvensändringar. Även spärrskiktskapacitansen i transistorerna är temperaturberoende. Det totala temperaturberoendet kan
kompenseras genom ett antal åtgärder.
Oscillatorn bör monteras så långt bort
som möjligtfrån övriga värmealstrande komponenter. Den avskärmande kapslingen omkring oscillatorn skall vara så tjockväggig
och värmeisolerande som möjligt. Inbyggnad i en termostatreglerad kapsling är ett
ännu bättre alternativ.
Komponenterna bör ha uppnått drifttemperaturför användningen. Oscillatorn bör
därför värmas upp under åtminstone 15 minuter.

Frekvensstabilitet och oscillatorbrus

Frekvensstabiliteten i kristalloscillatorer är
ca 100 gånger bättre än den är i LCoscillatorer. Likaså är utgångssignalen från
kristalloscillatorer renare från s.k. fasbrus
Gitter). Varje oscillator avger nämligen även
oönskade signaler med frekvenser som ligger omkring utgångssignalens nominella
frekvens.
Oscillatorn är ju en förstärkare, vars utgångsspänning delvis återkopplas till ingången i medtas. Detta innebär att utgångssignalen förstärks lavinartat till ett maximum, omväxlande med att den dämpas lavinartat till
ett minimum. Utan yttre påverkan befinner
sig alltså förstärkaren i ett självsvängningstillstånd mellan två yttervärde n. l återkopplingsvägen placeras ett filter som frekvensbestämmande element, t.ex. en LC-krets
eller en kvartskristall.
Återkopplingen blir starkast på filtrets
resonansfrekvens, vilket medför att oscillatorn svänger bäst där. Eftersom filtret oundvikligen har en viss bandbredd, så kommer
även ett spektrum av andra frekvenser tätt
omkring resonansfrekvensen att släppas
igenom. De oönskade frekvenserna omkring
den nominella kallas för brus.

113-62

l moderna konstruktioner används oftast
PLL-oscillatorer. På grund av sin funktion
pendlar deras frekvens alltid något. Hur
mycket beror bl.a. på loop-filtret. Alltså är
frekvensen egentligen ett mycket litet band
av flera frekvenser varav en framträder mest.
Försök:
Volymkontrollen i en lågfrekvensförstärkare
utan insignal vrids till maximum. Det kommer att höras ett brus i högtalaren, som
huvudsakligen kommer från ingångsstegets
transistorer. När en mikrofon ansluts måste
volymkontrollen vridas ner och då hörs bruset
mindre. Men bruset finns ändå där på en
lägre nivå och överlagras på insignalen från
mikrofonen.
Även i en högfrekvensoscillator överlagras bruset på insignalen. Men ju högre godhetstalet är i svängningskretsen, t.ex. en
kristall, desto smalare är filtrets bandbredd,
desto kraftigare blir brusundertryckningen
och desto merframhävs den önskade signalen. P.g.a. det större godhetstalet i svängningskretsen, och därmed den mindre bandbredden, så brusar alltså en kristalloscillator
mindre än en LC-oscillator.
En nackdel med kristalloscillatorn är att
dess frekvens inte kan ändras inom ettstörre
område. Önskas flera valbara frekvenser
från en kristalloscillator måste flera kristaller
användas tillsammans med något slags
omkopplingsanordning (kanalväljare).
Komponentmängden i en kristalloscillator
är mindre än i en VFO, men i apparater för
flera frekvenser uppvägs denna fördel av
merkostnaden för flera kristaller och kanalväljaren.
Kristalloscillatorn har många användningsområden, där en frekvensstabil och
brusfattig signal önskas och där platsbrist,
skakningar m.m. utesluter användning av en
LC-VFO.

ETSAR
3.8 Frekvensblandare

Grundprinciper

En anordning som blandar signaler för att
skapa andra kallas som namnet säger för
blandare. Blandare används både i mottagare och sändare och funktionsprinciperna
är lika i båda fallen. Vad som skiljer i stort är
hur de används.
Det finns många blandarkopplingar varav de vanligaste beskrivs här. Enkla typer
med vissa nackdelar ställs mot sådana som
är mer komplicerade, men har fördelar.

Bild II 3-83
När en linjär förstärkare matas med två
signaler så sammanlagras de. Den resulterande signalen vid varje tidpunkt är den
förstärkta summan av de inmatade signalerna.
När en olinjärförstärkare matas med två
signaler så blandas de med varandra. Förutom ingångssignalerna uppträder genom
blandningen ytterligare signaler på förstärkarutgången, så kallade blandningsprodukter.

ADDITION AV TVÅ SIGNALER
l ngångssignaler:
Växetspänningar med frekvensen f1

Utsigna!:

u

v.

u

och frekvensen f2

t U1 + U2

Frekvensspektrum

f,

v

l

f2

Endast ingångsfrekvenserna
uppträder i utsigna!en

BLANDNING AV TVÅ SIGNALER

u

u1~1ngångssignaler
fl

Utgångssignal

--·- -----··--.. ·-········,..-t

.....-

Frekvensspektrum

Det uppträder ytterligare frekvenser
utöver ingångsfrekvenserna

symbol för en blandare

f1~ Utgångbia f1, f2, f1
f2

t- f2, f2- f1

fz

>f1

Bild II 3-83 Principer för frekvensblandning

113-63

KR
Det finns ingen förstärkare i kopplingen.
signalspänningarna adderas genom att de
två transformatorernas sekundärlindningar
är seriekopplade. Dioden "förvränger" kraftigt summaspänningens kurvform. Beroende
av hur dioden är polariserad (vänd i kopplingen) blir den negativa eller den positiva
halwågen bortskuren.
Signalen på blandarens utgång, alltså
efter dioden, innehåller bl.a. frekvenserna f 1 ,
f2 , f2 +f1 , f2 -f1 • Den lägsta frekvensen f1 kan
lättast påvisas genom att ansluta ett lågpassfilter till blandarens utgång.

Två av blandningsprodukterna är särskilt
intressanta, det är summan och skillnaden
av ingångssignalernas frekvenser. De oönskade, övriga blandningsprodukterna filtreras bort med en avstämd krets eller ett
bandpassfilter.

Entaktsblandaren
Bild 113-84a
Vi kan övertyga oss om, att de fyra blandningsprodukterna verkligen uppstår. Först
undersöker vi den enklaste blandaren, som
är ett olinjärt element i form av en diod.

u1

l ngångssignaler:
f1

u

:J u, I~T
Ut+ U2

~J~ . .

Utgångssig nal

Ua

t

~~---t
u

(utan LP-filter)

l

U

Framtagning av lägsta frekvens fl

Bild II 3-84a Entaktsblandaren

113-64

Frekvensspektrum

L

b

f1

1 Lt.~

1 11
'.


1

KRETSAR

:J:

t

.
[) l "
N-r-,l· ·- ·r llillnnwnlliLI
n nn "

l

U

l

T.I} l"< lil~
lm~

=: - ---

-t

!

~~~~w~~

Frekvensspektrum

f2 tf 1

'---'-----'-....L--lll.-, f

(utan "'ngkmt.}

med "ängk""

-t

-t

Framtagning av frekvenserna f2, f1 + f2, f2 - f1

u

k~.
Al . -~ /'--- .-- /t+,r·~\ -· / /' . ,.
···'\ ::.

,

u

l

u

l

Signal med f2 - fl
(skilinadsfrekvens)

t

Signal med f2 + f1
(summafrekvims)

--~-

-::y-'(-

l

~-\---1--+-4-·-.~~~<}~ ~.-·.t···. ~-~---·-t ~~~na;;r :e11 f:d~r:~e
--+-- /

...

- 1T\--

UPtAJ\ f\ ~--t
u ,..--""

l

l
1

--,, .,

t--+---+-+-+--

'',..

-·

//

l
l·

-j:~·:"'-1---'

/

l

l

Signal med f2

/

:'.\, /

''< -..

-•·l

Alla tre signalerna adderade

l

Bild II 3-84b Entaktsblandaren

113-65

KRETSAR
Resultatet kan studeras med ett oscilloskop. Liksom på bilden ser man då att kondensatorn laddas upp till den positiva halvvågens toppvärde och med gott närmevärde
följer kurvformen på f 1 •
Bild II 3-84b
En svängningskrets med lämplig bandbredd, och som är avstämd till resonans
frekvensen f2 , ansluts nu till blandarens
gång. En signal med frekvensen f2 kan då
urskiljas och studeras i oscilloskopet. Svängningskretsen tillförs energi under de positiva
halvvågorna. Energin i svängningskretsen
kompletterar med den negativa halvvågen,
varvid en del av kretsens energi förbrukas.
Därför har de positiva och negativa halvvågorna inte samma amplitud (toppvärde).
Det syns i oscilloskopet hur amplituden
"svävar". Av detta dras slutsatsen att signalen består av fler frekvenser än f2 • Signalen
är sammansatt av f2 , f2 +f1 och f2 -f1 • Signalen
f1 ligger utanför svängningskretsens selektiva område och är därför bortfiltrerad (undertryckt). Blandningsprodukterna f2 +f 1 och f2 -f1
har båda en mindre amplitud än f2 •
Att det finns olika grundtoner och blandningsprodukter kan bevisas med en ännu
smalare svängningskrets med variabel frekvensavstämning, se bildens nedre del.
Vi har hittills utgått från entaktsblandaren.
Mer utvecklade blandartyper, såsom mottaktblandaren och ringblandaren, producerar färre blandningsprodukter.

Mottaktsblandaren
Bild II 3-85
Mottaktblandaren har två dioder, till skillnad
motentaktsblandarens enda diod ...................,,.,...
formatoremas ena lindning har mittuttag.
Ingången E1 ligger på den ena transformatorns primärlindning.lngången E2 1iggeröver
de båda mittuttagen. Utgången ligger på den
andra transformatorns sekundärlindning.
Ingången E1 matas med en signal med
en låg frekvens f. Eftersom en av de båda
dioderna alltid spärrar, så flyter det ingen
ström. De streckade pilarna visar endast i
vilken riktning strömmen kunde flyta, om de
spärrande dioderna vore öppna. Men så
länge som ingen signal ligger på ingång E2 ,
uppträder ingen signal på utgången.
113-66

Signalen på
avlägsnas och i stället
matas ingången
med en hög frekvens F.
Under den positiva halvvågen är de båda
dioderna öppna och genom båda flyter lika
stor ström. De båda transformatorernas
lindningshalvor genomflyts av lika ström i
motsatt riktning och då upphäver magnetfälten i lindningshalvorna varandraoch ingen
uppträder på utgången.
När signaler läggs
båda ingångarna
händer följande:
Dioderna öppnar och stänger i takt med
signalen på ingång E2 , med frekvensen F.
Den mycket svagare signalen på ingång E1 ,
med frekvensen f, kan alltefter polaritet passera diod D 1 eller D 2 • På återvägen överlagras signalen från E1 på signalen från E2 •
Strömmarna i lindningshalvorna är olika
stora. Då uppträder en signal på utgången.
Efter blandaren följer ett filter som endast
släpper igenom de önskade blandningsprodukterna F + f eller F - f.

Ringblandaren
Bild 113-86
Ringblandaren består av fyra dioder, som är
riktade åt samma håll i en "diodring".
ingången
matas med en signal med en
låg frekvens f. Till skillnad mot i mottaktsblandaren flyter en ström genom Di och D4
resp. D2 och 0 3 , men inte genom utgångstransformatorn. Ingen signal finns på utgången så länge som signalen F saknas.
Signalen på E 1 avlägsnas och i stället
matas ingången E2 med en hög frekvens F.
Till skillnad mot i mottaktsblandaren flyter en
ström genom dioderna Di och D2 resp. D3
och D4 och då upphäver magnetfälten i
transformatorernas lindningshalvor varandra. Ingen signal finns på utgången, så länge
som signalen f saknas.
När signaler läggs på båda ingångarna
händer följande:
De fyra dioderna kommer att öppna och
stänga parvis. Som i mottaktblandaren överlagras strömmen från ingång E1 på den
ström som dioderna öppnar för.
Här utnyttjas båda halvperioderna av F.
Strömmarna i lindningshalvorna blir olika
stora. På utgången uppträder då en signal.
Efter blandaren följer ett filter som släpper
igenom de önskade blandningsprodukterna.

KRETSAR
Bara signalen

e1

med frekvensen f ligger på
01

Bara signalen E2 med frekvensen f ligger på

båda dioderna öppnar

båda dioderna spärrar

Båda signalerna ligger på

Bild If 3-85 Mottaktsblandaren

113-67

R
Bara signalen E1 med frekvensen f ligger på

o,

Bara signalen E2 med frekvensen f ligger på

Båda signalerna ligger på

non no

lUlU

Bild II 3-86 Ringblandaren

113-68

KRETSAR

Signal utan svängkrets

Entaktsblandare
Frekvensspektrum

u

ut
nttöl\ 6tL.... n/\ nA~a

"

~t

~o~v~v~v~v~v~v~-~v~v\ v~v*v~v~o~v~~

li lL. . ~ ~~ ~ 11.
--
Signal med svängkrets

Signal utan svängkrets

Mottaktsblandare

u

.....

l

l..

+

LL

•

+

, 

f

Frekvensspektrum

u

eller

Signal utan svängkrets

Signal med svängkrets

;t;

il

M

Ringblandare

l

LL

M

-..
LL
M

ll

;;;

..

LL
M

... f

Frekvensspektrum

Bild 113-87 Jämförelse mellan olika blandare

113-69

KRETS R
Jämförelse av blandare

Bild 113-87
Bilden visar de tre beskrivna grundkopplingarna och de jämförs med avseende på frekvensspektrum på utgången.
Videntaktsblandaren uppträder summafrekvensen f+ F och skillnadsfrekvensen Ff, vidare ingångsfrekvenserna f och F, deras
övertoner 2f, 3f, 4f o s v, 2F, 3F, 4F o s v,
liksom deras blandningsprodukter F$\pm$ 2f, F$\pm$
3f o.s.v., 2F $\pm$f, 2F $\pm$ 2f, 2F $\pm$ 3f o.s.v.
Vid mottaktblandaren saknas frekvensen F och dess övertoner. Vidare bortfaller
de jämna övertonerna av frekvensen f.
Vid ringblandaren bortfaller ännu fler icke
önskvärda signaler, nämligen ingångssignalerna f och F och alla deras övertoner.
Endast blandningsprodukter av udda övertoner uppträder.
På bilden visas det fallet att frekvensen f
är mycket låg och då ligger blandningsprodukterna mycket nära varandra i frekvens.
Videntaktsblandaren filtrerar svängningskretsen ut frekvenserna F+ f, F- f, och F. Vid
mottakt- och ringblandaren saknas däremot
frekvensen F, den filtrerade signalen innehåller endast blandningsprodukterna F + f
och F- f. Om dessa båda blandningsprodukter är väl åtskilda eller svängningskretsen
har en bättre selektionsförmåga, då blir enbart summafrekvensen F + f eller skillnadsfrekvensen F- f framfiltrerad.
Vi har visat tre typer av blandare med
passiva komponenter. Sådana innehåller
olinjära dioder (germanium- eller kiseldiade r).
Det finns även blandare med aktiva komponenter, d.v.s. elektronrör eller transistorer
(bipolära, FET, MOSFET), men det skulle
leda för långt att gå in på alla olika lösningar.
l kapitlen 4. Mottagare och 5. Sändare beskrivs hur frekvensblandning används för
modulering och demodulering.

Icke önskade övertoner och blandningsprodukter

Varje olinjärt arbetande funktionssteg alstrar förutom nyttafrekvenser även icke önskade signaler med andra frekvenser. Både
önskade och icke önskade signaler kan bestå av övertoner eller blandningsprodukter
(skillnads- och summatoner) eller bådadera.
113-70

Vissa av signalerna filtreras fram för att
utgöra nyttosignaler. Andra signaler filtreras
bort, så att t. ex. utsändning inte sker på fel
frekvenser.
Bild 113-88
l ett tidigare avsnitt har vi beskrivit en s.k.
super-VFO. Vi skall nu undersöka vilka blandningsprodukter som uppstår i en sådan. De
två mest uppenbara frekvenserna är blandningsprodukterna (summan) i området 144146 MHz och (skillnaden) i området 128-126

MHz.

Ut från blandaren finner vi ingångsfrekvensen 136 MHz och dess övertoner 272
MHz, 408 MHz o.s.v. såväl som VFO-signalen och dess övertoner. På bilden är VFOfrekvensen 8 MHz och dess övertoner inritade, d.v.s. 16 MHz, 24 MHz, 32 MHz o.s.v.
Tyvärr bildar också de båda ingångssignalernas övertoner blandningsprodukter vilket bilden visar.
Bandpassfiltret släpper igenom nyttafrekvensen, men dämpar alla övertoner och
blandningsprodukter. Detta är enklare ju
längre ifrån nyttasignalen de icke önskade
signalerna ligger. l vårt exempel faller VFOsignalens övertoner inom bandpassfiltrets
passband på följande sätt:
15 · 9.6 = 144 MHz till15 · 9.733 = 146 MHz
16 · 9.0 = 144 MHz till16 · 9.125 = 146 MHz
17 · 8.471 =144 MHz till17· 8.588 = 146 MHz
18 · 8.0 = 144 MHz till18 · 8.111 = 146 MHz
Eftersom det här handlar om 15 :e-18 :e
övertonerna, så blir amplituderna så små, att
vi kan bortse från dem.
Det är viktigt med goda filter i signalbehandlande funktionssteg. En god regel är att
på ett tidigt stadium filtrera bort oönskade
övertoner och blandningsprodukter-helst i
varje steg -så att onödigt komplexa signaler
undviks. Det är också viktigt med frekvensvalet, så att oönskade blandningsprodukter
kommer så långt bort från nyttafrekvensen
som möjligt, liksom att endast mycket höga
övertoner med motsvarande små amplituder faller inom det nyttiga frekvensområdet

ETSAR

Super-VFO för 144-146 MHz

BLANDARUTGÄNGENs FREKVENsSPEKTRUM

U

f 1 =8MHz
passbandkurva
sidfrekvens

nyttasignal

f,df.
144 MHz

f2 -· f1
128 MHz

a) Förenklad framställning med VFO:n avstämd till 8 MHz

u

9,5 MHz

136MHz

126,5 MHz·--·----1 ffi--·-,.145,5 MHz

........... . .L111.

b) Förenklad framställning med VFO:n avstämd på 9,5 MHz

u

8MHz
16 MHz
24 MHz

l / )2

MHz osv

136 MHz
144 MHz

128 MHz

\

ll

1--11--"--..--A--.li--.S..-"-.L..-JL.........L

.....

27 2 MHz o s v

c) Ingångssignalens övertoner med VFO:n avstämd på 8 MHz

u

136

BMHz

16MHz
/24 MHz

/32MHz

l

osv

(136-BH1flz
( 136-16} M~z\

(136-2'dM!1zL

~1Hz

/

(136•8lt11fz
( 136+ 16) MHz

/(136+24lMHz

l

osv

( 2'72--B l ~1H7
(272+16lMHz\
\

272MHz
( 272+ Bl MHz

j(272+16lMHz

l

osv

l.LL--L---   .~~ aI...Jl,I......JII.-.L..-A(..I...JLI-'''---- f

d Blandningsprodukter från blandning av övertoner

Bild II 3-88 Frekvensspektrum från en
113-71

KRETS R

113-72

3.9 Modulatorer
Allmänt

När en signal (bärvågen) påverkas så att
den överför informationen i en annan signal,
sägs bärvågen bli modulerad. Detta förlopp
kallas modulation. Vad som då händer behandlas främst i kapitel 1, avsnitt 1.8, med
tillämpningar i kap. 5 och delvis i kap. 4 .
En anordning för modulation kallas för
modulator. En modulator kan ingå som en
funktion i sändare, mottagare m.fl. system.
Beroende på modulationsmetoden används
olika kombinationer av delkretsar som tillsammans utgör modulatorn.
l detta avsnitt ges exempel på några
vanliga modulatorer för sändare.

Amplitudmodulatorer

Med en amplitudmodulator påverkas bärvågens amplitud proportionellt med den modulerande signalens amplitud.
Vid sändningsslaget A 1A är amplituden
på den modulerande signalen antingen maximal eller ingen. Då består modulatorn av en
nycklingskrets, som påverkar t. ex. ett drivsteg i sändaren så att bärvågen släpps fram
helt eller inte alls.
Vid sändningsslaget A3E har den modulerande signalens amplitud ett analogt förlopp, t.ex. tal, med vilket bärvågens amplitud moduleras. Här beskrivs amplitudmodulation i en förstärkare med ett katodkopplat
elektronrör. En emitterkopplad transistorförstärkare kan moduleras på ett liknande
sätt. l båda fallen moduleras förstärkarens
arbetsspänning (anodspänning resp. kollektorspänning) med den modulerande signalens. Det som då händer är att två signaler
blandas på ett sätt som beskrivs i avsnitt 1 .8
med tillämpning påA3E.I vila är då bärvågsamplituden halva den möjliga inom arbetskurvans linjära del. Vid modulation kommer
bärvågens amplitud att variera mellan noll
och den möjliga amplituden.
Bild II 3-89
Bilden visar ett sändars lutsteg med en triod.
l serie med tilledningen för anodspänningen
finns sekundärlindningen av en modulationstransformator för LF-signalen.

Den LF-förstärkare som driver transformatorn måste för 100 $\circ$/o modulationsgrad
kunna avge bärvågens halva effekt. Eftersom uteffekten från en fullt utmodulerad
A3E-sändare är 150 $\circ$/o av den i vila, måste
slutsteget dimensioneras därefter. Utöver
den egna signalspänningen måste modulationstransformatorn även klara slutstegets
arbetsspänn ing.
Om som på bilden anodspänningen i ett
förstärkarsteg amplitudmoduleras, kan förstårkarsteget arbeta olinjärt, t.ex i klass C.
Varje följande förstärkarsteg måste däremot
arbeta linjärt, t.ex. i klass A.
På grund av den låga verkningsgraden
och det stora bandbreddsbehovet, används
i dagens amatörradiosändare knappast
"äkta" amplitudmodulering, d.v.s. A3E.I stället används i läget "AM" nästan alltid H3E,
d.v.s. enkelt sidband med full eller reducerad bärvåg (se nästa stycke). Trots det lägre
effektbehovet p.g.a. endast ett sidband och
ev. reducerad bärvågsamplitud kan av dimensioneringsskäl ändå inte de flesta H3Esändare avge sin fulla effekt kontinuerligt!
Som redan sagts i avsnitt 1.8, är det onödigt
sända ut två sidband, eftersom båda innehåller samma information. Det räcker med
ett sidband. Bärvågen innehåller inte någon
information. Den kan därför undertryckas
redan i sändaren för att ersättas i mottagaren. Därmed uppstår sändningsslaget J3E.

till
slutsleg

~--o+Ua

la

---1

lllg reaklans
för tonfrekvens

Dr

Bild II 3-89 A3E-modulator

ua

113-73

KR

I

--l
t

DSB-Signal

u

~6ft,, ~ftn,Molftftn..

Vllt! Wjj y VUllll" "t{lfl[V vIl"

/

•

t

Alstring av en DSB-signal (amplitudmodulering, dubbelt sidband med undertryckt bärvåg
i en ringblandare (balanserad modulator)
/

/u~
b''""'"d

modulew

~-----l

~/ ~SBSSBkristallfilter

Ul

lfr

'

fT ·fLF

:

fT · fLF

r---lL. 1

f

Framfiltrering av övre sidbandet

Bild II 3-90 Alstring av J3E (SSB)

113-74

/

DSBSi nal~Signa~

t

KRETSAR
Vid sändningsslaget J3E (SSB) sänds
således endast ett sidband. Det andra sidbandet och bärvågen undertrycks, vilket kan
göras på flera sätt. Numera är den s.k.
filtermetoden allra vanligast och den enda
som behandlas här.
Bild II 3-90
Med filtermetoden blandas HF- och LFsignalerna i en balanserad blandare där de
undertrycks medan blandningsprodukterna
med deras summa- och skillnadsfrekvenser
blir kvar, d.v.s. det övre och nedre sid bandet.
För att undertrycka det ena sidbandet
före sändningen, följs blandaren av ett bandpassfilter med bandbredd och frekvensläge
för avsett sidband. Den signal som sänds ut
innehåller på så sätt endast ett sidband
(Single Side Band).
Valet mellan USB och LSB kan göras på
två sätt. Antingen genom att välja mellan ett
separat passbandfilter för respektive sidband eller genom att använda ett enda filter
och flytta H F-signalen från ena sidan till den
andra av det filtret (se bild Il 1-28 i avsnitt
1.8).
En J3E-modulator enligt filtermetoden
består således av en balanserad blandareofta en s.k. ringblandare (se bild Il 3-87 i
avsnitt 3.8) samt ett bandpassfilter.
För att SSB-signalen skall få avsedd sändarfrekvens kan ytterligare frekvensblandning behövas (se kapitel 5).

Vinkelmodulation
Vinkelmodulation är samlingsnamnet för frekvensmodulation (FM) och fasmodulation
(PM).
Frekvensmodulation
Vid sändningsslaget F3E (även kallat FM)
varierar bärvågens frekvens i takt med den
modulerande signalens amplitud. Bärvågen
kommer på så sätt att pendla omkring en
nominell frekvens, d.v.s. vara frekvensmodulerad. Bärvågsamplituden ändras däremot
inte vid frekvensmodulation.
Likspänningsnivåer kan således överföras eftersom en frekvensawikelse (deviation) i bärvågen endast påverkas av den
modulerande signalens amplitud.
Vid F3E påverkas resonansfrekvensen i
den svängningskrets i oscillatorn som bestämmer dess arbetsfrekvens. Det görs enklast genom att tillföra en kondensator med
variabelt kapacitansvärde, en s.k. varicap
(se avsnitt 2.5).
Bild II 3-91
Bilden visar en LC-svängningskrets där
det ingår en varicapsom styrs av en likspänning med en överlagrad modulerande LFsignal. En likspänning tjänar som en ställbar
förspänning till varicap. På så sätt kan man
påverka arbetsfrekvensen. Med den överlagrade LF-signalen påverkas arbetsfrekvensen i takt med signalamplituden.

till oscillatorn

Alstring av FM:
Oscillatorn moduleras

Bild II 3-91 Alstring av F3E (FM)

113-75

Fasmodulation
Vid sändningsslaget G3E (även kallat PM)
varierar bärvågens fasläge i förhållande ,till

en referens, som är en omodulerad bärvåg.
Bärvågens amplitud ändras däremot inte.
Fasändringen -deviationen -är direkt proportionell till hur snabbt fasläget ändras och
till den totala fasändringen. Hastigheten på
fasändringen är direkt proportionell till frekvensen på den modulerande signalen och till
dess amplitud.
Det betyder att deviationen vid fasmodulation ökar både med amplituden och frekvensen på den modulerande signalen. Ändringar i likspänningsnivåer kan därför överföras endast om en fasreferens används.
Fasmodulation kan alstras t.ex. genom
att påverka resonansfrekvensen i en svängningskrets någonstans efteroscillatorn, d.v.s.
där oscillatorfrekvensen inte påverkas. Denna svängningskrets har i viloläge samma
resonansfrekvens som oscillatorn. När kretsen bringas ur resonans genom modulation
-samtidigt som kretsen påtrycks oscillatorsignalen -så uppstår i kretsen omväxlande
en induktiv och kapacitiv reaktans - detta
inom tiden förvarje halv period. Reaktansen
skapar därvid den fasförskjutning som innebär fasmodulation. Se även avsnitt 3.1, bilderna Il 3-18 och -19
Bild 113-92
Liksom vid frekvensmodulation kan t. ex.
en varicapanvändas för att med en modulerande signal påverka resonansfrekvensen i
en krets.

Från oscillatorn

J-l

~---

Alstring av PM:
Kretsens reaktans
moduleras

Bild 113-92

113-76

avG3E (PM)

~~?~:' ----~
fYYY"\r -J:

ID~- l~ ö, , , ~,"'

rrC>--:LJ stamnrngsspannrngen
,.

~ ~-J


\chapter{MOTTAGARE}
Energin i de elektromagnetiska magnetfält,
som omger oss, alstrar högfrekventa strömmar i alla metaflföremål. För att effektivt
fånga upp dessa fält används antenner.
Fastän energin i fälten kan få en lampa
att lysa om sändarantennen är tillräckligt
nära, så går det ändå inte att uppfatta den
information som fälten också kan innehålla.
För det behövs en radiomottagare för att
dels förstärka de oftast mycket svaga signalerna och dels uttyda informationen i dem.

Lyssna på amplitudmodulerade rundradiosändningar på mellanvåg kan man enklast
göra med hjälp av en detektormottagare
Speciellt under dygnets mörka timmar
vintertid kan man höra utländska sändare
med denna enkla mottagare, låt vara att det
hörs mycket svagt. l detektormottagaren
omvandlas fältens energi till elektricitet och
sedan till ljud. Så länge som ingen förstärkare används, förbrukas ingen annan energi
än den som fångas ur fälten -radiovågorna.

\

\

\

\ \
l

\

\
l

!

antenn selektering demodulering

tF-återgivning

Bild II 4-1 Detektormottagare

Raka mottagare
Mottagare med kristalldetektor
Bild II 4-1
Detektormottagaren består av ett mycket
litet antal komponenter. Princip och arbetssätt framgår av bilden. Samma princip används även i mer komplicerade mottagare,
mätinstrument etc. Antennkretsen består
av antenn, jordtag och däremellan en induktor (kopplingsspole), som överför energin
från antennen till en svängningskrets. Svängningskretsen används för att välja ut (selek-

tera) en bärvåg med önskad frekvens. Bärvågen kan naturligtvis inte höras, men av
kurvformen på bilden framgår att bärvågen
är amplitudmodulerad med en LF-signal.
För att återvinna LF-signalen utför man
en s.k. demodulering med hjälp av dioden.
Dioden klipper bort antingen de positiva
eller negativa halvvågorna i den mottagna
signalen, beroende på hur dioden är vändpolariserad. Kondensatorn, som är kopplad
parallellt över hörtelefonen, glättar de högfrekventa spänningstopparna till ett amplitud-

114-1

M
medelvärde (jämför med entaktsblandare i
Kapitel 3). Detta spänningsvärde varierar
på ett sätt, som motsvararden modulerande
spänning i sändaren som kommer av tal,
musik etc. Vi har nu demodulerat bärvågen,
återställt LF-signalen och kan höra den i
mottagaren.
Bild II 4-2
signalspänningen över svängningskretsen
är störst när dess resonansfrekvens och
antennströmmens frekvens är lika.
Överst i bilden ser man att mottagaren är
inställd på samma frekvens som sändare 2.
Även sändare 3 hörs eftersom bandbredden
i svängningskretsen är stor. Nederst i bilden
är svängningskretsen inställd på sändare 3,
men man hör också sändare 2 och 4.
Bandbredden i svängningskretsen blir
mindre ju mindre den belastas, d.v.s dämpas. l Bild II 4-1 består belastningen av
antennen (via kopplingsspolen), hörtelefonen och avkopplingskondensatorn (via dioden).
Mindre belastning kan åstadkommas på
två sätt; dels med "lösare" koppling mellan
antennkrets och svängningskrets och dels
med bättre impedansanpassning mellan
svängningskrets och diod. Båda sätten tillämpas i Bild II 4-3. Hur selektionen då
förbättras visas i Bild II 4-4, vilket skall
jämföras med Bild II 4-2.

demodulering

u

Bild II 4-2 Selektion i detektormottagare

u
550kHz

900kHz 1100kHz 1500kHz

Bild II 4-4 Förbättrad selektion

lågpass

volymkontroll

LFförstärkare

högtalare

}\
selektering

demodulering

lågpass

LFförstärkare

Bild II 4-3 Detektormottagare med LF-förstärkare

114-2

högtalare

f

GARE

Bild II 4-5 Förbättrade HF-egenskaper i detektormottagare

u

membranen i hörtelefonen knäpper
(ingen ton)

/\r---------------t
Bild II 4-6 Hög HF-selektion

Detektormottagare med förstärkare
Bild II 4-3
Om man vill höra sändningarna över högtalare, behövs högre effekt än vad som kan
fångas upp genom antennen. För ändamålet används en LF-förstärkare, som drivs av
en annan energikälla, t. ex. ett batteri. LFförstärkaren kan även minska belastningen
på svängningskretsen.
l bilden har ett LF- lågpassfilter satts in
efter HF-avkopplingskondensatorn. Det
dämparLF-signaler med högre frekvens än
vad som behövs för god mottagning.
Mottagare med bättre HF-egenskaper
Bild II 4-5
Ett sätt att minska bandbredden i en detektormottagare är att koppla flera svängningskretsar med samma frekvens efter varandra. Den större dämpningen av fler kretsar
kan kompenseras med en HF-förstärkare.
Sådana mottagare används för speciella
ändamål, t.ex. för övervakning av en enda
frekvens. Då är svängningskretsarna fast
avstämda. Kanske utnyttjas till och med en
kvartskristall som filterförden speciella frekvensen. Se Bild II 4-6 om hög selektion.

"knack" "knack" "knack"

\

"knack"

---------

membranen släpper

Bild 114-7 CW i detektormottagare

Detektormottagare och sändningsslag
l huvudsak fungerar detektormottagaren endast vid amplitudmodulering. Det innebär
sändningsslagen A3E och A2A, d.v.s. amplitudmodulerad telefoni resp. tonmodulerad
telegrafi, båda med full bärvåg.
Bild II 4-7 Däremot fungerar detektormottagaren inte vid A 1A, d.v.s. telegrafi med
endast bärvåg. En omodulerad bärvåg alstrar nämligen endast en likström i en
detektormottagare. Vid nyekling hörs då
endast knäppningar i hörtelefonen vid början och slutet av teckendelarna.
Detektormottagaren fungerar inte heller
vid J3E, d.v.s. SSB och övriga sändningsslag med undertryckt bärvåg. Ljud såsom tal
förvrängs nämligen kraftigt i en J3E-signal
eftersom bärvågskomponenten saknas.
l båda ovannämnda fal kan talet återställas med tillsats av en bärvåg.
Slutligen kan sändningsslag som innebär frekvens- och fasmodulering i princip
inte dernoduleras med detektormottagare.
114-3

PT

M

~2 +D
~l
fl

f1 =1830kHz

och dess

ö"ctoo"

f2- f1 = 1 kHz

f2 - f1 ::: 1kHz

/,
[>

t
~

f 2 = 1831 kHz

VFO

blandare

A
antenn

x

t:;+

b,

u

demodulering LF-Iågpassfilter

förse lek·
ter ing

[>
LF-för·
stärkare

högtalare

~
VFO

Bild II 4-8 Mottagare med direkt frekvensblandning
Mottagare med direkt frekvensblandning
För att demcdulera A 1A och J3E i en rak
mottagare- detektormottagare-måste den
kompletteras med en oscillator som alstrar
en intern bärvåg. Denna blandas med den
mottagna signalen. Det uppstår då en svävningston - beat frequency. Därav namnet
Beat Frequency Oscillator- BFO.
Förfarandet har givit mottagartypen sitt
namn- direktblandad mottagare.

114-4

Bild II 4-8
Ett sätt att komplettera den raka mottagaren
med BFO framgår av bilden. När BFO kopplas till och ställs in på en frekvens tillräckligt
nära mottagningsfrekvensen så uppstår en
hörbar ton.
Demodulatordioden tillförs alltså två HFsignaler, dels den från antennen och dels
den från BFO. Dessa båda signaler blandas
i dioden och skillnadsfrekvensen är den
hörbara tonen. Övriga blandningsprodukter
dämpas av ett lågpassfilter.

A ARE
HF på blandaringången

u~

LF-skillnadsfrekvens på
blandarutgången

VFO-

CWsignal
1830 kHz
l

si nal

1lb1 kHz

l1

... f

f1 f2

UrF01is2~1
/f-

u
kHz

CWsignal 1830 kHz

j

1 kHz

-JAI.I. f

1
~~~----------f

mo

f2 f1

u

f1- f2

119 kHz
~----------~~-~f-, f
V F O-

ur

si nal
1S29 2 kHz C;W,
sognal
1830kHz

1l

f

l

0,8 kHz

1

--f

fz f1

~-L------------f

f,- f2

Bild II 4-9 Demodulering i mottagare med direkt frekvensomvandling- C W-signaler

U

u~d~rtryckt
barvag

l SB

l

1832
kHz

~---

!l

i

1835kHz

l

fl\ 1

1834 1834,7
kHz kHz

u
1kHz
300 Hz 

.. f

r-1

111

3 kHz

1:
L------!l.-A--.L..----f

Bild II 4-1 O Demodulering i mottagare med direkt frekvensomvandling - SSB-signaler
Mottagning av telegrafi (CW)
Bild II 4-9
Då BFO (VFO) är inställd på frekvensen f2
=1831 kHz och den mottagna signalen f1 har
frekvensen 1830 kHz så hörs en svävningston med frekvensen 1000 Hz.
Samma resultat fås om BFO ställs in på
frekvensen f2 = i 829 kHz. Med BFO på

frekvensen f2 = 1830 kHz hörs ingenting av
signalen f 1 = 1830 kHz från sändaren.
Frekvensskillnaden är noll Hz.
De flesta föredrar en ton med frekvensen
c:a 800Hz för mottagning av telegrafi. BFOfrekvensen skulle i så fall ställas in på 1830.8
eller i 829.2 kHz om f 1 vore en telegrafisändning.

114-5

Selektering av blandningsprodukterna

[>
LF-lågpass
3kHz

förselektering
av ca 300kHz
bandbredd

Val av mottagarfrekvens

VFO

U HF
VFO-

frekvens

.H,

l r--4s
CW-Signal
1830 kHz

1829,2

1
,J

.

rll

~il

u

SSB-Signal
1835 .-·

lågpassfilterkurva

O - 3 kHz

1832 kHz

 L,

35 kHz

l

·f

1632 1834 1634,7
kHz kHz kHz

l
l

I....L-A.I..'-.L...ll.------ f

800 Hz 2,8 kHz

Vid mottagning av en CW-signal tillsammans med en SSB-signal hörs båda samtidigt

u
BOO Hz

4,8 kHz

5,5kHz

VFO
Förbättring av selekteringen med ett LF-CW-fiiter

Mottagning av J3E (SSB)
När en SSB-sändare sägs arbeta t. ex. på
frekvensen 1835 kHz, så innebär det frekvensen på den bärvåg som undertryckts i
sändaren redan före utsändningen.
Vad som uppfattas av mottagarens ingångskretsar är alltså det utsända sidbandet När en SSB-signal demoduleras, så
blandas den lokala bärvågen i mottagaren
med de mottagna modulationsprodukterna.
Vid blandningen uppstår blandningsproduk-

114-6

ter som består dels av LF, dels av andra
högre frekvenser som dämpas i ett låg passfilter.
Bild II 4-1 O
l nom amatörradio används för SSB det lägre sidbandet vid frekvenser under 1O MHz.
Med en frekvens av t. ex. 1835 kHz och ett
talspektrum av 300-3000 Hz kommer det
lägre sidbandet att finnas mellan 1834.7
och 1832.0 kHz. Tre modulerande frekvenser 300, 1000 och 3000 Hz visas på bilden.

TTA ARE
Med en bärvågsfrekvens av 1835kHz motsvaras dessa modulerande frekvenser av
utfrekvenserna1834.7, 1834 och 1832kHz.
VFO ersätter SSB-sändarens bärvåg och
skall ha samma frekvens-1835kHz- för att
kunna återge 300, 1000 och 3000 Hz.

selektionen i direktblandade mottagare

Bild II 4-11
Direktblandade mottagare kan ses som en
typ av detektormottagare, även kallad "rak"
mottagare. Begreppet "rak" kommer av att
HF-signalen från antennen passerar genom en selektiv krets och en eventuell HFförstärkare rakt fram till detektorn, utan att
frekvensen omvandlas.
l en detektormottagare är bandbreddenoftast rätt stor. Flera sändare hörs därför
samtidigt.
P.g.a. att blandningsdioden i en direktblandad mottagare även fungerar som AMde modulator, så hörs faktiskt alla sändare
inom förkretsens bandbredd. Detta kan undvikas till en del genom att dioden, som
fungerar som entaktsblandare, byts till en
mottaktblandare eller ännu hellre till en ringblandare. Sådana blandare undertrycker ingångsfrekvenserna och släpper endast igenom blandningsprodukter. Bara den sändarsignal hörs då, vars frekvens tillsammans med VFO-frekvensen ger blandningsprodukter, som faller inom LF-filtrets passband. Mottagningsfrekvensen är VFO-frekvensen. Svängningskretsen fungerar som
en ställbar förselektor och LF-Iågpassfiltret
ger den egentliga frekvensselektionen.
Vilka HF-signaler bildar blandningsprodukter med VFO-frekvensen och vilka av
dessa passerar sedan genom lågpassfiltret
efter nedblandning till LF-nivå?
Exempel:
En CW-sändare med frekvensen 1830 kHz
tas emot genom att mottagarens VFO ställs
in på frekvensen 1829.2 kHz. Från blandarutgången kommer då en ton med frekvensen 800Hz.
Men sändaren är inte ensam på bandet.
Kommer t. ex. SSB-sändaren på 1835, som
moduleras med 300, 1000 och 3000 Hz, att
störa mottagningen? (Bild II 4-1 0).

Förkretsen i mottagaren är så bred att
denna sändning passerar. SSB-sändarens
signalfrekvenser i det utsända sidbandet är
1834.7, 1834.0 och 1832kHz. Dessa frekvenser blandas med mottagarens VFO-frekvens 1829.2 kHz och alstrar blandningsprodukterna 5.5, 4.8 och 2.8 kHz. Eftersom
lågpassfiltret i mottagarens LF-förstärkare
har bandbredden 0-3000 Hz, så kommer
endast blandningsprodukten 2.8 kHz attvara
störande. För att förbättra CW-mottagningen, så kan lågpassfiltret bytas ut mot ett
bandpassfilter, som endast släpper igenom
ett smalt frekvensområde omkring mittfrekvensen 800 Hz.

Passband och spegelfrekvenser i direktblandare

Bild II 4-12
l exemplet i förra stycket blev problemet
med en störande ton löst med ett bandpassfilter med annan frekvensgång.
Men vilka frekvenser kan tas emot genom
ett lågpassfilter, 0-3000 Hz, om VFO-frekvensen är t.ex. 1829.2 kHz?
Experiment:
Ändra frekvensen på en CW-sändare långsamt från 1820 till 1840 kHz.
Såndarfrekvensen i 820 kHz hörs knappast eftersom överlagringstonen har frekvensen 9.2 kHz och den dämpas kraftigt av
lågpassfiltret Först när sändarfrekvensen
är 1826.2 kHz hörs en tydlig ton med frekvensen 3000 Hz. Fortsätter man att ändra
sändarfrekvensen, så sjunker tonens frekvens för att bli noll (svävningsnoll), när såndarfrekvensen är lika med mottagarens VFOfrekvens 1829.2 kHz. Om man nu fortsätter
med att höja frekvens, så blir överlagringstonens frekvens åter högre. Vid såndarfrekvensen 1832.2 är den 3000 Hz. Vid ännu
högresändarfrekvens dämpas överlagringstonen igen av lågpassfiltret
Slutsatsen av experimentet blir följande:
Vid en direktblandande mottagare med VFOfrekvensen i 829.2 kHz och ett 3 kHz lågpassfilter blir varje sändare hörbar, som har
en sändningsfrekvens mellan 1826.2 och
i 832.2, varvid överlagringstonen har frekvenser från 3000Hz, ner genom noll och upp
till 3000 Hz igen.

114-7

TTAGARE
HF

U

LF

u

.fvFo~
(---r-----,

-3 kHz

r-----.. ,

+3 kHz

l

l
l

l.f-----1-t

l
l

.

l

l

..

:

f

~--~----------f
3kHz

1826,2 1829,2 1832,2

kHz

kHz

kHz

6kHz HF-bandbredd vid3kHz LF-bandbredd

U

l - - . -:. .f.1. s.fvFO.L.~. . !\----f

1828,5 ·-·
1828,3 kHz

1829,2

kHz

1829,9 ---

1830,1 kHz

u

,-,

l

l

l
l
l

l
l
l

l

j

l

l

l

\

~~~----------f
700-900 Hz

Mottagningsfrekvens och spegelfrekvens
med ett LF-CW-filter

l

fvFO
r----r----~
l

-

!

183 2
kHz

1835

1838

kHz

-t

kHz

Mottagningsfrekvens och spegelfrekvens
med ett LF-Iågpassfilter

Bild II 4-12 Passbandbredd och spegelfrekvenser i direktblandade mottagare
Vår mottagare har bandbredden 6 kHz.
Varje annan sändare inom denna passbandbredd kommer att höras eller-om man
så tycker- störa mottagningen.
Tillbaka till exemplet med bandpassfiltret
Vilka frekvenser kan tas emot med ett
bandpassfilter 700-900 Hz (mittfrekvens 800
Hz), om VFO-frekvensen är 1829.2 kHz?
Jo, vi kan lyssna rätt ostört till vår CWsändares 800 Hz-ton på frekvensen 1830
kHz. Ändå kan en annan sändare med
frekvensen 1828.4 kHz störa mottagningen
därför att denna är spegelfrekvens till mottagningsfrekvensen 1830 kHz. Vid VFOfrekvensen 1829.2 kHz uppstår en överlagringston, inte bara vid sändarfrekve~sen
1830kHz utan också vid 1828.2 kHz. Aven
denna andra sändarfrekvens, liksom nytto114-8

frekvensen, släpps igenom bandpassfiltret
Spegelfrekvensmottagning är en principiell nackdel i mottagare med direktblandning. Nyttafrekvens och spegelfrekvens i
det senaste exemplet ligger 1.6 kHz (2 · 800
Hz) ifrån varandra, alltså dubbla värdet av
bandpassfiltrets mittfrekvens.
Vid 888-mettagning måste naturligtvis
hela LF-området upp till 3000 Hz kunna
släppas igenom. Utöver det önskade frekvensområdet 1832-1835 kHz, kommer även
spegelfrekvenser i området 1835-1838 kHz
att kunna tas emot.
Vid en LF-bandbredd av 3 kHz har således den direktblandade mottagaren en bandbredd av 6kHz, vilket är en god avstämningsskärpa i jämförelse med den 300 kHz breda
förkretsen.

M
För- och nackdelar med direktblandare
Enkel uppbyggnad, men trots det en god
känslighet och hygglig avstämningsskärpa.
VFO kan även användas till att styra en
sändare.
Spegelfrekvensmottagning är tyvärr
oundviklig. Vidare kan signaler från starka
sändare stråla in i den känsliga LF.:.förstärkaren och orsaka LF-detektering, om mottagaren är otillräckligt skärmad. Förbättrad
isolering mellan antenn och VFO kan dock
fås med en HF-förstärkare.
Entakts diodblandare är olämplig i en
direktblandad mottagare. Den tar emot alla
sändare inom förkretsens passband och en
del av VFO-signalen kommer att strålas ut i
antennen. Ingen av dessa nackdelar finns i
en mottakts-eller ringblandare.

f.MF

ARE

Superheterodyn mottagare
Superheterodynprincipen ger mycket större möjligheter, när önskemålet är en högselektiv mottagare för flera olika frekvenser.
Skillnaden mellan en direktblandad mottagare och en "super" är, att blandningsprodukterna i direktblandaren blir till LF direkt, medan de i supern först bildar en
mellanfrekvenssignal MF, vilken sedan
dernoduleras och blir till LF- detekteras.
l det följande kallas superheterodynmottagaren enbart SUPER. l supern blandas de mottagna signalerna med signalen
från en VFO. Före blandningen har HFsignalerna passerat ett selektivt försteg, som
dämpar spegelfrekvenser. För att inte störa
mottagningen placeras VFO-frekvensen alltid utanför det frekvensband, där man vill ta
emot signaler.
Bild 114-13
Alla mottagna signaler blandas med VFOsignalen. Mottagningsfrekvensen är vanligen skillnaden mellan en fast s.k. mellanfrekvens MF och VFO-frekvensen. Mellanfrekvensen är egentligen mittfrekvensen i
ett fast passband skapat av ett antal filter.

=455 kHz
M F-filter

DetektorIdemodulator

j~sfilter

fast avstämt bandpassfilter
VFO

fvro :: 4055 kHz

demodulator

LF-Iågpassfilter

LF

fMF = fvFo - fM (VFO-frekv. över mottagn.frekvensen)
eller
fMF = fM - fvFo (VFO-frekv. under mottagn.frekvensen)
Hög selektivitet, enkel avstämning {jämfört med en rak mottagare)

Bild II 4-13 Superheterodynmottagaren i princip

114-9

M TTA ARE
Dubbelsuperheterodynmottagare
Bild 114-14
Det är svårt att bygga enkla mellanfrekvensfilter för höga frekvenser, med liten bandbredd och branta flanker. Det är fallet för en
enkelsuper för kortvåg med en enda mellanfrekvens, t.ex. 9 MHz.
En god närselektion på höga frekvenser
är endast möjlig med relativt dyrbara kristallfilter. Däremot går det att få god närselektion
med enklare medel på lägre frekvenser.
En dubbelsuper, d.v.s. en super med
dubbel frekvensomvandling, möjliggör god
både när- och förselektion. l 1 :a blandaren
blandas den mottagna signalen med signalen från en 1 :a oscillator (VFO} till en hög
mellanfrekvens, t. ex. 9 eller 10.7 MHz.
Därmed kan en god spegelfrekvensdämpning erhållas. Första M F-filtret kan göras enklare och utan den höga selektivitet
som hade behövts i en enkelsuper. 1:a MF
blir sedan blandad ytterligare en gång i 2:a
blandaren till en 2:a MF, t.ex. 455kHz. För
den andra blandningen används en fast
oscillator. Filtret i 2:a MF kan lättare utföras
med en hög selektivitet, p.g.a. den lägre
frekvensen.
Exempel:
Trots att M F-filtret inte är en enkel
svängningskrets, kan ett "Q-värde" beräknas. Vid en passbandbredd av 6kHz och en
centerfrekvens av 455 kHz kan Q-värdet
anses vara

Bild II 4-13 visar en mottagare med mellanfrekvensen 455kHz, som är vanlig i äldre
mottagare. MF-filtret kan i enklaste fall bestå av ömsesidigt magnetiskt kopplade LCsvängningskretsar. Bättre avstämningsskärpa fås med resonatorer av keramik eller
kvarts eller de är elektromekaniska.
Exempel:
En sändning på frekvensen 3600 kHz
skall tas emot. Vi ställer då in VFO-frekvensen till 4055 kHz, eftersom mellanfrekvensen är 4055 - 3600 = 455 kHz. Den
mottagna signalen hamnar då mitt i MFfiltrets passband.
Signaler på angränsande frekvenser tas
också emot och alstrar blandningsprodukter.
Med ett mellanfrekvensfilter med t. ex. 3kHz
bandbredd (453.5-456.5 kHz), kan signalfrekvenser mellan 3598.5 och 3601.5 passera genom filtret. En signal med en närliggande frekvens t. ex. 3603kHz, och blandad
med den inställda VFO-frekvensen 4055
kHz, kommer att alstra en skillnadsfrekvens
av 452 kHz. Denna signal ligger utanför
filtrets passband och kommer att dämpas
och når inte detektorn.
VFO-signalen kan givetvis läggas under
i stället för över mellanfrekvensen.
Exempel: VFO-frekvensen 3145kHz kan
också användas för mottagning av frekvensen 3600 kHz, om mellanfrekvensen är 455
kHz (3600- 455 =3145kHz). Men för att
undvika att eventuella övertoner från VFOsignalen blandas med mottagna signaler är
det lämpligt att placera VFO-frekvensen över
mottagningsfrekvensen.
Efter M F-filtren följer bl.a. detektorer för
olika sändningsslag samt LF-förstärkare.
Jämför med Bild II 4-5 och Il 4-6

Q= f,es

b

CJ

Bild II 4-14 Dubbelsuperheteodynen i princip

o

6

l ett MF-filter med centerfrekvensen 9
MHz skulle det behövas ett nära 20 gånger
högre Q-värde för samma bandbredd6kHz

I

114- 1

= 455 =76

ARE
Q= ~es

b

= 9000 = 1500
6

Ett så högt Q-värde kan endast erhållas
med kristallfilter.
För högre mottagningsfrekvenser räcker
det, på grund av filterproblematiken, oftast
inte med en dubbel frekvensomvandling,
Om man antar en dubbelsuper-mottagare
för VHF-området 144-146 MHz enligt bilden, så skulle en i :a MF med frekvensen
i 0.7 MHz inte vara tillräckligt hög. Vid en
mottagningsfrekvens av 146 MHz är nämligen spegelfrekvensen i 46 + (2 • 1O. 7) =
i 67.4 MHz, alltså endast i .15 gånger mottagningsfrekvensen. Det hade alltså varit
lämpligt med en trippelsuper, d.v.s. en trefaldig frekvensomvandling, med en i :a MF
i frekvensområdet 70 MHz.

Jämförelse mellan supern och detektormottagaren
Principen för detektormottagaren är enkel. l

en sådan sker allt från antenn till dernodulering på samma frekvens, d.v.s. mottagningsfrekvensen. Signalen går utan frekvensomvandlinng rakt igenom mottagaren. Nackdelen är att det kan uppstå oönskade självsvängningar på grund av den höga förstärkningen i LF-förstärkaren. Vidare är det obekvämt att ställa infrekvensen om det finns
flera förselektionskretsar. Med ett kristallfilter som är en bättre selekteringskrets kan å
andra sidan mottagning endast ske på en
fast frekvens. Detektormottagare byggs inte
annat än för specialändamål eller i enkla
utföranden för t. ex. radiopejlorientering och
byggsatser.
En utveckling av detektormottagaren är
den direktblandade mottagaren, vilken
ler en uppgift i vissa enklare sammanhang.
Denna mottagartyp är liksom supern avstämbar med en VFO.
selektionen i den direktblandade mottagaren sker, i motsats till detektormottagaren
inte i förkretsen utan i ett LF-filter. En nackdel är fortfarande den oundvikligaspegelfrekvensmottagningen. Vidare kan HF utstrålas från VFO vid ett olämpligt val av blandarprincip. Principen med direktblandning används emellertid som demodu!eringsmetod
t.ex. i SSB-mottagare.

Superheterodynmottagaren är avstämningsbar på ett enkelt sätt med en VFO.
selektionen görs i den fast avstämda MFdelen. Spegelfrekvensdämpning görs med
förselektion i kombination med en lämpligt
vald mellanfrekvens.
En nackdel med en superheterodyn är
att den är mer komplicerad. Vidare kan även
i supern HF utstrålas från VFO om olämplig
blandarprincip väljs.
Men med en dubbelsuper kan spegelfrekvensmottagning lättare undvikas p.g.a.
en hög 1 :a MF samtidigt som en låg 2:a MF
medger en bättre närselektivitet
Fortfarande är risken för oönskade blandningsprodukter stor vid olämpligt valda oscillatorfrekvenser.
Fastän komplexiteten är relativt stor redan i en dubbelsuper så är den ännu större
i en trippelsuper.

Speciella mottagare
Panoramamottagare

Bild II 4-15
l en panoramamottagare visas på en oscilloskopskärm var det finns signaler inom ett
frekvensband. En panoramamottagare är
en superheterodyn. Mottagaroscillatorn är
en VCO (spänningsstyrd oscillator). Dennas frekvens styrs av en sågtandformad
likspänning, som stiger linjärt för att snabbt
falla tillbaka och återupprepas. VCO sveper
då över det önskade frekvensbandet med
ett antal gånger gånger per sekund. Med
samma sågtandspänning avlänkas strålen
på skärmen utmed x-axeln. Bild II 4-16
Den mottagna signalen dernoduleras och
översätts till en likspänning som skildrar de
mottagna signalernas styrka. Med denna
likspänning avlänkas strålen på bildskärmen utmed y-axeln. Strålens avstånd från
x-axeln anger alltså den mottagna stationens styrka och strålens läge utmed x-axeln
anger var stationen ligger i det frekvensområde som avsöks. Beroende på hur stort
frekvenssving som ges VCO, så kommer ett
större eller mindre frekvensområde att
avsökas och visas på skärmen. Området
kan vara så brett som ett amatörband eller
mer och ner till några få kHz.
Utöver övervakning av ett frekvensband
kan en panoramamottagare användas för

114-1 i

M TTA AR
studium t. ex. av signaler och sidefrekvenser
som alstras i den egna stationen. För noggranna mätningar behövs emellertid ett hjälpmedel av högre kvalitet, kallat spektrumanalysator. En sådan arbetar i grunden på
samma sätt som en panoramamottagare.

lil!ll

f

Frekvensspektrum

Bild II 4-16
En panoramamottagare kan anslutas till en
mottagare för att studera signalerna inom
MF-passbandet. Då är mottagningsfrekvensen i bildskärmens mitt. stationerna under
och över i frekvens visas till vänster respektive höger om den egna frekvensen.
Vid ändrad mottagningsfrekvens blir
denna fortfarande kvar mitt på skärmen.

sågtandsformad avstämningsspänning

Bild II 4-17 Signal- och svepspänningar

Bild II 4-15 Panoramamottagare

·-·-·-·-·-·-·-·-·-·-·-·!
[>

!
l
i

i
i

·-·-·-·-·-·-·-·:t.atio~~~tta~re ·-·-·j

9 MHz och
grannfrekvenser

Bild II 4-16 Anslutning av panoramamottagare till stationsmottagare

114-12

E
Mottagningskonvertern

Bild II 4-18
Konverter betyder i detta sammanhang frekvensomvandlare. När det är önskvärt att
flytta över alla signalerna inom ett helt frekvensområde till ett annat, så används en
mottag ningskonverter där frekvensblandning och frekvensfilter används.
Konvertern fungerar som tillsats före en
mottagare för att denna även skall kunna
användas inom ett annat frekvensområde. l
en konverter är oscillatorfrekvensen fast,
medan avsökningen av frekvensområdet
görs med VFO i mottagaren. Mellanfrekvensfiltret i mottagaren är så brett som hela
det frekvensområde som tas emot av konvertern och avsöks med mottagaren.
Exempel: l en KV -mottagare för området
28-30 MHz vill man även kunna lyssna i
området 432-434 MHz (UHF). Den i konvertern mottagna UHF-signalen förstärks för
att sedan blandas med 404 MHz, en frekvens som uppmultiplicerats från en kristalloscillator (CO) i konvertern. De blandningsprodukter som filtreras fram kommer att
ligga inom området 28-30 MHz oc.~ kan
alltså avlyssnas i KV-mottagaren. Ovriga
blandningsprodukter blir undertryckta i KVmottagarens ingångskretsar.
Blandningsfrekvensen 404 MHz i konvertern är beräknad på följande sätt:

Mittfrekvensen i UHF-bandet är
(432 + 434)/2 = 433 MHz = f 1 •
Mittfrekvensen i KV-mottagarens frekvensband är (28 + 30)/2 = 29 MHz.
Med vilken frekvens f2 måste 433 MHz
blandas för att erhålla en blandningsprodukt
av 29 MHz? 29 MHz är mindre än f 1 , alltså
kan endast skillnadsfrekvensen komma i
fråga. (Vid summafrekvens skulle blandningsfrekvensen bli högre än 433 MHz).
Vid användning av skillnadsfrekvensen
ges två möjligheter:
för f2 - f1 = f2 - 433 = 29 MHz är f2 = 462 MHz
för f 1 - f2 = 433 -f2 = 29 MHz är f2 = 404 MHz
Vi bestämmer oss för alternativet 404
MHz av ett speciellt skäl. Här motsvaras den
högsta UH F-frekvensen 434 MHz av 434 404 = 30 MHz och den lägsta UH F-frekvensen 432 MHz av 432-404 = 28 MHz. På så
sätt kan kHz-graderingen på KV-mottagarens skala användas direkt utan omräkning.
Fördelen med en konverter är att kostnaden för en sådan är låg jämfört med den
för en komplett mottagare för ett tilkommande
band. Förutsättningen är att en mottagare
redan finns.
Nackdelen är att mottagaren inte samtidigt kan användas för sin ordinarie funktion.

UH F-antenn

f1 432 - 434 MHz
432 MHz

eller f1 430 L------l

(> t - - - - - -

UH F-försteg

CJ

~

eller

f2 = 404MHz
f2 =
~1Hz

CJ

/I I~

44,889 MHz

4 .•... MHz

Bild II 4-18 Mottagningskonverter UHF till KV

114- 13

M TTA
Transvartern
Bild II 4-19
En transverter (transmitter-converter), är
en kombinerad frekvensomvandlare för både
sändning och mottagning. Den förflyttarbåde
mottagnings- och sändningssignaler mellan två frekvensområden.
Transvertern är ett bra exempel på hur
samma teknik kan användas både i mottagare och sändare. Om t.ex. en KV-transceiver redan finns, kan både mottagning
och sändning ordnas även på andra band
med en transverter som tillsats.

efter kristalloscillatorn CO kan användas för
sändning och mottagning.
Fördelen med en transverter är att kostnaden för en sådan är låg jämfört med den
för en komplett transeeiver även för det
tillkommande bandet. Förutsättningen är att
en transeeiver för något band redan finns.
Nackdelen är att den befintliga transeeivern inte samtidigt kan användas på några
andra frekvenser än de som används för
tillfället.

Exempel
En konverterförflyttar de mottagna UH Fsignalerna till kortvågsområdet Som huvudmottagare används en KV -transceiver i
mottagningsläge. Konvertern kan utökas till
att även fungera vid sändning och kallas då
transverter. Med KV -transceivern i sändningsläge flyttas dess signaler till UH F-området genom blandning i transvertern av
KV-signalen och en multiplicerad signal från
en lokaloscillator (LO). Den önskade blandningsprodukten i UHF-områdetfiltreras fram
och förstärks i efterföljande driv- och slutsteg. Samma frekvensmultipliceringskedja
UH F-antenn

Sändarblandare

s

Bandfilter

Drivsteg

-

f1 28-30 MHz

UH F-försteg

t
T 44,889MHz
D

CO

3

t2=

Bild 114-19 Transverter mellan UHF och KV
114-14

PA

404MHz

ARE

M T
Automatiskt förstärkningsreglering
(AGC) i mottagare
För att mottagaren skall fungera bra för
såväl mycket svaga som för mycket starka
in-signaler behövs en förstärkningsreglering
i signalvägen genom mottagaren. signalspänningen på mottagaringången kan vara
från delar av en mikrovolt upp till över 100
millivolt - ett spänningsförhållande på
1:100000. Det motsvarar mer än nio senheter, vilket är ett mått på signalstyrkan
(Appendix D).

Vid mottagning av en stark signal är det
inte tillräckligt med att bara minska LFförstärkningen. Förstärkarstegen i HF- och
M F-delen blir ändå överstyrda av den starka
insignalen och utsignalen förvrängs om inget ytterligare görs. Det är därför nödvändigt
att minska förstärkningen även i HF- och
MF-förstärkarstegen, ju mer desto starkare
insignalen är. Som hjälpmedel finns oftast
ett reglage för H F-förstärkningen (RF gain),
och därutöver en automatisk förstärkningsreglering- AGC (Automatic Gain Contro l).

AGC

Automatic

Gain

Contro l

Förstärkningsreglering i A3E - mottagare
till de reglerade förstärkarstegen

från sista

R

:n

till MF-förstärkare

fö'"''k"':sJ
lJ

u
LF-signal

överlagrad likspänning
= reglerspänning

Bild II 4-20 AGG vid AM-mottagning med superheterodynmottagare

114-15

M
En mottagare med god reglering kan
arbeta med signalstyrkor mellan mikrovolt
och volt. Beroende hur den mottagna signalen är modulerad (sändningsslaget), sker
AGC på olika sätt.
Både vid AM och SSB finns informationen i sidbanden. HF- och M F-stegen måste
därför arbeta i det linjära området och de får
inte överstyras. Förstärkningen i mottagaren måste alltså regleras med hänsyn till
detta.

AGG vid AM (A3E)
Bild II 4-20
Den likspänning som uppstårvid dernoduleringen av MF-signalen i en AM-mottagare
används till förstärkningsreglering - AGC.
Den LF-spänning som är överlagrad på
likspänningen undertrycks i ett RG-Iågpassfilter. Likspänningen över kondensatorn följer variationerna i den mottagna signalens
styrka med en tidskonstant av ca 0.1 sekunder. Likspänningen blir alltså inte påverkad
av de betydligt snabbare spänningsändringarna som kommer av moduleringen.
En stark bärvågssignal alstrar en hög
likspänning och en svag signal en låg likspänning, oberoende av moduleringen.
Denna likspänning återförs till de framförliggande HF- och MF-förstärkarstegen, vilka
är gjorda så att en hög reglerspänning sänker förstärkningen, medan en låg spänning
tillåter en hög förstärkning.
På så sätt kommer signalstyrkan efter de
reglerade stegen att hållas konstant samtidigt som mottagarens ingång inte överstyrs.

Den likspänning som filtrerats fram från
detektorn kallas reglerspänning eller AGGspänning. Diodens polarisering är inte viktig
för att få ut LF vid demoduleringen, men
däremot för att få rätt polaritet på AGGspänningen. i de flesta mottagare används
negativ AGG-spänning.

AGG vid SSB (J3E)
Bild II 4-21
l de flesta utföranden lämnar produktdetektorn en växelspänning utan överlagrad likspänning. Reglerspänningen alstras därför
genom likriktning av MF-spänningen med
hjälp av en separat demoduleringsdiod eller
genom likriktning av LF-växelspänningen.
Vid SSB alstras det ju ingen MF-spänning undertalpauserna, eftersom ingen bärvåg tas emot då. Tidskonstanten på lågpassfiltret för reglerspänningen måste därför vara längre än vid AM, d.v.s. 0.5 till 2
sekunder. En alltför snabb tillbakagång i
reglerspänningen p.g.a. en för kort tidskonstant skulle leda till mer mottagningsbrus i
tal pauserna. l moderna mottagare finns det
ofta en omkopplare för olika tidskonstanter.
Bild II 4-21
AGG vid CW (A 1A)
Metoden för att alstra AGG-spänning är
samma vid CW och SSB.

AGG vid FM (F3E)
FM-mottagare brukar inte regleras av den
anledningen att det vid FM inte finns någon
information i signalamplituden, utan finns i
stället i frekvensvariationerna i signalen.

.............

AGC
'l> O,Ss

Bild II 4-21 AGG vid SSB- och C W-mottagning med superheterodynmottagare

114-16

Helt avsiktligt läggs därför förstärkningen i
mottagaren så, att en sinussignal blir en
kantvåg p.g.a. överstyrning i förstärkarstegen. Ett eller flera sådana amplitudbegränsande steg, även kallat "limiter", placeras
före demoduleringssteget. Störningar av
amplitudvariationer kommer då att klippas
bort och inte störa mottagningen.
Störande signaler inom nyttabandbredden har dock ingen större inverkan så länge
som den önskade signalens styrka är en
halv s-enhet större än den störande signalens styrka. Likaså försvinner det störande
bruset vid mottagning av en FM-sändare
mycket snabbt över denna signal nivå. Amplitudmodulerade störningar, som t. ex. de från
tändgnistor i förbränningsmotorer, har liten
påverkan vid tillräckligt stark nyttasignaL

Signalstyrkemätare (Smmeter)
AGG-spänningen i en mottag re för AM, CW
och SSB kan även styra en S-meter, som
ger besked om hur stark signalen in i mottagaren är. (Se Appendix D)
Brusspärr
l en FM-mottagare hörs bara brus när det
inte kommer in en tillräckligt stark signal.
Bruset är genomträngande eftersom FMmottagare arbetar med hög förstärkning. En
brusspärr (eng. squelch) är en anordning
som stoppar signalerna till LF-förstärkaren
när signalerna ej uppnår en viss nivå. På så
sätt slipper man att höra på bruset. l mottagare förflera sändningsslag och därför även
AGC kan denna funktion styra brusspärren,
men i en ren FM-mottagare arbetar MFförstärkarna utan AGC. l det fallet behövs
någon annan anordning för att skilja mellan
en modulerad signal och brus. Ofta finns ett
reglage (squelch) för hur stark signal.en
skall vara innan spärren öppnar.

Med selektivitet menas en mottagares förmåga att skilja ut önskade signaler och
undertrycka övriga. Summariskt beskrivet
kallas avståndet mellan yttergränserna för
det önskade frekvensområdet för bandbredd. När det gäller superheterodynmotta.;
gare finns två selektivitetsbegrepp.
Det ena är förselekteringen för att dämpa
de spegelfrekvenser som uppstår i samband med blandning av mottagna signaler
och oscillatorfrekvenser i mottagaren.
Det är selektiviteten i en superheterodynmottagares MF- steg för att utskilja den
önskade signalen efter blandningsförloppen.

Spegelfrekvensproblemet vid mottagning
Bild II 4-22
Exempel:
En sändning på 3600kHz skall tas emot och
VFO-frekvensen är 4055 kHz. Mellanfrekvensfiltret undertrycker sändningar på så
närliggande frekvenser som t. ex. 3603 och
3597kHz. Denna egenskap kallas för närselektion.
Men tyvärr kan en sändning på så avlägsen frekvens som 451 O kHz ändå störa
mottagningen, den goda närselektionen till
trots. Avståndet mellan 451 O kHz och vår
mottagningsfrekvens3600kHz är 91OkHz.
Frekvensen 451 O kHz och VFO-signalen
bildar också en blandningsprodukt, som har
frekvensen 455 kHz. Vid en VFO-frekvens
av 4055 kHz och en mottagningsfrekvens
av 3600kHz benämns451OkHz som spegeifrekvensen. Avståndet mellan spegelfrekvens och mottagningsfrekvens är dubbla
värdet av mellanfrekvensen -i detta exempel 2 ·455kHz= 91OkHz.
Signaler på mottagningsfrekvensen och
spegelfrekvensen alstrar båda blandningsprodukter med VFO-frekvensen, som har
mellanfrekvensens värde. Mellanfrekvensfiltret kan därför inte undertrycka en främmande signal på spegelfrekvensen.
Bild II 4-23
Däremot kan en mottagaringång med
förselektering undertrycka den. En selektiv
krets före blandaren släpper igenom ett smalt
frekvensband med mittfrekvensen 3600kHz,

114-17

M

AR
/pegelfrekvens

u

fsp ::::

närselektion

fM

+

t

2·

/mellanfrekvens

fMF

"- mottagningsfrekvens

l

fvFO

IM

fsp

tfMF ~fMFj

3600
kHz

Mottagning av önskad sändare:

fvFo- fE

:::

Mottagning av ej önskad sändare: fsp - fvFo :::

4055
kHz

4510
kHz

4055 ---3600kHz = 455 kHz
4510 --4055kHz = 455kHz

Bild II 4-22 Enkelsuper med låg MF och ingen förselektion

u

MF =455kHz

Bild II 4-23 Enkelsuper med låg MF och med förselektion
men dämpar t. ex. frekvensen 451 O
kHz p.g.a. den stora frekvensskillnaden. En förselektion har alltså tillförts
som komplement till den närselektion
som erhålls med mellanfrekvensfiltret
Bild 114-24
Ju längre ifrån varandra nyttofrekvens och spegelfrekvens ligger,
desto bättre är förselektionen. Med en
mellanfrekvens av 455 kHz är alltså
detta avstånd 91 O kHz. l långvågsoch mellanvågsområdet är det tillräckligt för att man med enkla medel skall
kunna skapa tillräckligt selektiva filter.
Exempel:
Vid den högsta mottagningsfrekvensen på mellanvåg 1605 kHz är
spegelfrekvensen 2515 kHz, som ligger 1.57 gånger högre i frekvens och
med ett avstånd av 91OkHz. l kortvågsområdet dämpas inte en spegelfrekvens på avståndet91OkHz tillräckligt
kraftigt. Vid den högsta mottagnings-

114- 18

u

MF =900kHz

Bild II 4-24 Enkelsuper med hög MF och
med förselektion
frekvensen på kortvåg 30 MHz ligger nämligen
spegelfrekvensen 30.91 OMHz endast 1.03 gånger
högre i frekvens. Med antagandet, att förselektionskretsen har ett Q-värde av 30, blir bandbredden 53.5 kHz vid frekvensen 1605 kHz.
Med samma Q-värde blir bandbredden 1000
kHz vid frekvensen 30 MHz, vilket innebär att
förkretsen inte längre kan dämpa så närliggande
spegelfrekvenser på ett effektivt sätt.

ARE
l mottagare för högre frekvenser används
därför högre mellanfrekvens för att öka avståndet till spegelfrekvensen. l moderna
kortvågsmottagare är det vanligt med en
mellanfrekvens av 9 MHz eller högre. Vid en
mottagningsfrekvens av 30 MHz och en
mellanfrekvens av 9 MHz är spegelfrekvensen 48 MHz, vilket är 1.6 gånger mottagningsfrekvensen. Detta möjliggör förselektionsfilter med tillräcklig dämpning av spegelfrekvensen.
Bild II 4-25
Bilden visar hur när- och förselektion kompletterar varandra i ett frekvensspektrum.
Märk, att passbandbredden b i förselektionskretsen anger avståndet mellan de frekvenser där signalamplituden dämpats till 70
0
/o av toppvärdet. l exemplet här ovan har
antagits att förkretsen för kortvågsmottagning har samma Q-värde som förkretsen för
mellanvågsmottagning.

Vid högre frekvenser, i VHF- och UHFområdet, kan inte önskat Q-värde erhållas i
sådana kretsar som användsiKV-området
och lägre. Andra lösningar blir då nödvändiga, t.ex. kavitetsfilter och helixfilter.
MF-bandbredd vid AM (A3E)
Bild II 4-26
En amplitudmodulerad signals frekvensspektrum består av bärvågen och två sidfrekvenser - eller sidband om sidfrekvenserna är många.
Bandbredden i MF-kretsarnamåstevara
minst så stor att sidfrekvenserna längst bort
från bärvågen kan passera. Dessa frekvenser motsvarar de högsta modulerande tonerna. Vid rundradiosändningar på mellanvåg utsänds alla frekvenser upp till 4.5 kHz.
Detta motsvarar en bandbredd av 9 kHz.
För enbart talöverföring är en bandbredd av
6 kHz tillräcklig, vilket motsvarar en LFgränsfrekvens av 3 kHz.

u
HF

FÖRSELEKTERING:

Undertrycker spegelfrekvenser

önskad sändare
flera andra
annan sändare på
spegelfrekvensen

0.7

~~~~~~~ww~~~~~----f

l fM :
r--b---G<>j

u

MF-fittrets passband

1r

MF

NÄRSELEkTERING:

Undertrycker angränsande sändare

Bild II 4-25 Samtidig för- och närselektion i superheterodynmottagare

114- 19

M
Ett för smalt MF-filter skär bort de yttre
delarna av sidbanden. LF-signalerna kommer då att förlora de höga tonerna (diskanten). Om däremot filtret är för brett, kommer
närliggande utsändningar också att höras.
l vissa mottagare kan MF-bandbredden
anpassas till förhållandena. Det är alltså en
fråga om en kompromiss mellan bättre ljudkvalitet och mindre störd mottagning.

u

MF-bandbredd vid SSB (J3E)
Bild 114-27
Mellanfrekvensfiltret för SSB-mottagning
skall endast släppa igenom ett av de två
sidbanden, vars bredd är skillnaden mellan
högsta och lägsta överförda LF-frekvens.
Inom amatörradio är detta3kHz- 0.3 kHz =
2.7 kHz, alltså något mindre än hälften av
bandbredden vid AM.

MF

LF

u
fr

fLF

=

(fr t fLF ) -

fLF

E

fr -

fr och

( fr - f L F)

fL F
~r--~--~--~-------

8999
kHz

9 000

f

~~----------------~f

kHz

fL F

= 9001 -·· 9000kHz = 1 kHz

fL F

= 9000

····8999kHz

= 1 kHz

A3E- demodulering i frekvensspaktrat

u

u

MF
M F -filterkurva

MF

inställd sändare
med bärvåg och sidband

riktig MF-bandbredd

M F-bandbredden för smal, delar av
sidband bortklippta. Diskanten borta

b = 2 · fLFmax
b = 2 · 3 kHz = 6 kHz

u

MF

u

MF

Störning från angränsande sändare.
Undertryckning genom smalt MFfilter på bekostnad av diskant

För stor M F-bandbredd, alltför
flack filterkurva, störningar från
angränsande kanaler

MF-bandbredd vid A3E

Bild II 4-26 MF-bandbredd vid AM (A3E)
114-20

EPT

~©~

M TTA ARE

Ett alltför brett MF-filter skulle också
släppa igenom oönskaqe signaler från angränsande frekvenser. A andra sidan skulle
ett för smalt M F-filter skära bort signaler i det
önskade frekvensregistet och försvåra mottagningen. Smala filter kan å andra sidan
utnyttjas för att dämpa signaler, t. ex. från en
för nära liggande sändare eller som har för
stor bandbredd.

U

MF

När närliggande sändare stör mottagningen
ges följande möjligheter:
Att göra en liten snedavstämning, uppåt
eller nedåt i frekvens. Därigenom ändras frekvensläget på det mottagna talet, men vid små frekvensawikelser blir
förvrängningen liten. Läsligheten blir
sämre, men mottagningen på det hela
taget bättre.

U

MF-filterkurva
l

rätt MF-bandbredd
b = fLFmax - fLFmin
b = 3 --·- 0,3 kHz = 2,7 kHz

u

För stor MF-bandbredd, alltför
flack filterkurva, störningar från
angränsande kanaler

u

Minskning av grannstörning genom :
snedavstämning

Störning från grannkanalssändare

U

MF

u

den undertryckta
bärvågen

\l
l

l

normal BFO-frekvens

~~--~~==~~-----f

MF-skift

Avklippning av de lägre frekvenserna
genom förskjutning av 2:a MF-filterkurvan (passbands-tuning)

Bild fl 4-27 MF-bandbredd och passband-tuning vid SSB (J3E)

114-21

•

•

Vidare måste VFO, 1 :a BFO och 2:a BFO
kunna ställas in var för sig. Frekvensläget
på MF l och/eller MF Il kan då förskjutas
över respektive filters passband, oberoende av varandra. Därigenom uppstår
skenbart effekten att filterkurvorna skjuts
emot varandra. Samma effekt skulle fås
om kristallfiltren gick att avstämma, vilket
ju inte är möjligt.

M F-skift. Som just beskrivits kan en liten
snedavstämning göras. l vissa mottagare
är det ordnat så att också BFO-frekvensen kan förskjutas så att frekvensläget på
talet blir återställt igen. Därmed blir MFpassbandet skenbartbart förflyttat uppåt
eller nedåt i frekvens (M F-skift, IF-shift).
Det verkliga frekvensläget mellan nytto-:
signal och BFO behålls. l alla händelser
blirbasen ellerdiskanten på nyttasignalen
avskuren, beroende på var denna ligger
i frekvens.
Passband-tuning. Om det finns störande
sändare både över och under i frekvens,
går det inte att skära bort störningarna
med ett enkelt M F-skift, eftersom antingen den ena eller den andra störande
sändaren ändåskulle höras. Fördetfallet
erbjuder några moderna mottagare möjligheten att flytta MF-passbandets övre
och undre frekvensgräns oberoende av
varandra (bandpass tuning m.m.). Detta
förutsätter, att mottagaren är en trippelsuper med branta filter i varje MF-steg.

u

vid CW (A tA)
Bild II 4-28
En
har som bekant inte bandbredden non Hz, utan det handlar i grunden
om en amplitudmodulerad signal. Vid en
nycklingshastighet av 60 tecken per minut
är bandbredden c:a i 00 Hz och vid i 20
per minut den dubbla, c:a 200 Hz.
l vissa mottagare används ett SSB-filter
även för mottagning av CW. En vanlig bandbredd på ett SSB-filter är 2.7 kHz och då
kommer även stationer på närliggande frekvenser att höras. Låt vara att de flesta av
dessa stationer hörs med olika frekvens.

u

MF

LF

BFO
CW-Signa!

8

U

~~~

2

9

"-"'------f
~~~ b ca 100-200Hz

MF

""---------f

800Hz

U

MF
BFO .

f
CW-mottagning med SSB-filter

Bild II 4-28 Olika MF-bandbreder vid CW (A

114-22

CW-mottagning med 250 Hz CW-filter

PT
Fler än 20 CW-stationer får plats inom en
bandbredd motsvarande en SSB-kanal. Den
mänskliga hjärnan, kan med någon övning
koncentrera sig på en av dessa signaler
medan övriga uppfattas som störande.
Det tidigare nämnda LF-bandpassfiltret
skulle emellertid åstadkomma en bättre selektion och bekvämare avlyssning. Men om
en annan station inom passbandet är mycket
starkare än den station som är av intresse,
då blir MF-förstärkaren antingen överstyrd
av den starkare signalen eller AGC reglerar
ner förstärkningen så att den svagare signalen inte längre kan höras trots det smala LFfiltret. selektionen i en mottagare bör därför
sitta "så långt fram som möjligt". l det skildrade exemplet skulle ett smalt filter i MF vara
till bättre nytta vid CW-mottagning. Bandbredden på ett sådant filter är 250- 500 Hz,
således endast något bredare än CW-signalen.
Medettännu smalare CW-filterkan, p.g.a.
bristande frekvensstabilitet hos sändare och/
eller mottagare, svårigheter uppstå att finna
den önskade signalen. Välutrustade mottagare har passband-tuning även för CW,
steglös bandbreddsreglering eller stegvis
valbara filterbandbredder. Då kan mottagaren ställas in på den önskade signalen med
en stor bandbredd som därefter minskas.
För mottagning av RTTY (radiofjärrskrift)
med 170 Hz skift mellan de två frekvenserna, kan ett 500 Hz-filter användas. Smalare filter går däremot inte så bra.

Bandbredd vid FM (F3E)
En FM-sändare med frekvensdeviationen
~~ax och högsta modulerande LF-moduleringsfrekvensen fLFmax har bandbredden

b= 2( ~~ax + (Fmax) •

Inom amatörradio är det brukligt med en
maximal deviation av 3 kHz och en övre
gränsfrekvens av 3kHz, vilket motsvarar en
bandbredd av 12 kHz.
Fullgod mottagning är möjlig endast om
M F-filtren i mottagaren har minst den bandbredd, som sändaren har. Men vid för stor
mottagarbandbredd kan även stationer på
närliggande frekvenser uppfattas. Sedan
1996 är det av IARU Region 1 rekommenderade kanalavståndet 12.5 kHz vid FM-trafik
på VHF- och UHF-amatörradiobanden.

TTAGARE
Det är vanligare med för stor deviation på
FM-sändaren än att mottagaren är alltför
smaL En för stor deviation, avsaknad av
deviationsbegränsare och för hög LF-gränsfrekvens medför en onödigt stor bandbredd
på sändaren. Motstationen får då mottagningssvårigheter och stationer på angränsande kanaler blir också störda.
Det blir allt vanligare med 12.5 kHz kanalavstånd även för repeatrar, varför det är
viktigt att alla sändare är rätt inställda.

signalkänslighet och brus

Om man ställer in mottagaren på en ledig
frekvens, så hör man vid full förstärkning ett
brus likt det från ett vattenfall.
Bruset kommer från de svaga växelspänningar som uppstår när laddningsbärarna
rör sig genom de material som strömkretsen
består av. Beroende av bruskällan sträcker
sig frekvensspektrum från noll till nära nog
oändligt. På grund av egenskaperna skiljer
man mellan en rad specifika bruskällor:
• Resistorbrus, även kallat "vitt brus", som
uppstår i resistiva komponenter. Bruset
sträcker sig över hela det mätbara frekvensområdet varvid energifördelningen
är lika över hela området,
• Kretsbrus, som uppstår i resistanser i
svängningskretsar i resonans,
• Antennbrus, som är sammansatt av
bruset från antennens strålnings- och
förlustrasistanser samt av det galaktiska
brus som antennen tagit emot,
• Transistorbrus uppstår av laddningsbärarnas rörelser i halvledarmateriaL
Det bildas en sammanlagd brusspänning
som kan bestämmas. Man talar om ett brustal, som är ett mått på mottagningssystemets
egenbrus. Detta skall ställas mot styrkan på
den mottagna signalen. Man talar om ett
förhållande mellan signaleffekt och bruseffekt. Det finns flera metoder att mäta och
uttrycka detta förhållande som kallas S/N
(signal to noise ratio). För att uppfatta den
information som kommer ur en mottagares
LF-utgång måste nyttasignalen vara ett antal gånger starkare än bruset. Den lägre
gränsen för att uppfatta tal i kortvågsmottagare är ett brusavstånd i storleksordningen 1O dB.

114-23

M

ARE

l en broschyr på en kortvågsmottagare
kan man t. ex. läsa "Sensitivity SSB, CW:
lessthan 0.25 J! V for 1O dB SIN ... "
Termen S/N betyder Signai/Noise, d.v.s.
styrkeförhållandet signal/brus uttryckt i dB.
Det innebär att en signal kan läsas vid 25J.tV
signalnivå och ett S/N av mindre än 1O dB.
Utöver brusnivån i mottagaren spelar också
distorsionen en roll.

S+N+D
N
[dB]

Signalbrusförhållande
där S=Signalnivå
N=Brusnivå
D=Distorsionsnivå

UlllllJIDllll~N
3

f kHz

Spektrum

Bild II 4-29 SIN-värde

l en broschyr på en VHF-mottagare kan
man t. ex. läsa "Sensitivity FM: Lessthan
0.18 J.tV for 12 dB SINAD ... "
Termen SINAD betyder Signal, Noise
and Distorsion. Vid denna definition tar man
även hänsyn till distorsionsprodukter som
orsakas av den modulerande signalen.
SINAD= S+N+D
N+ D

[dB]

Nivå
(dB)

Bild 114-30 SINAD-värde

114-24

lntermodulation, korsmodulation

Utöver att en bra modern mottagare bör ha
tillräcklig frekvensstabilitet, känslighet och
selektivitet bör den även ha goda s. k. storsignalegenskaper.
Med storsignalegenskaper menar man
hur bra en relativt svag nyttasignal på mottagaringången motstår påverkan av starka
frekvensnära signaler med hög fältstyrka.
Störningar av detta slag uppstår genom icke
linjära förlopp i komponenter i mottagarens
ingångssteg, varvid mottagna signaler med
stor amplitud blir förvrängda.
Korsmodulation och intermodulation är
två begrepp som är förknippade med storsignalegenskaperna. Båda kan visserligen
definieras och bestämmas entydigt, men de
förväxlas ändå ofta.

Korsmodulation
Med korsmodulation menas, att den inkommande nyttasignalen amplitudmoduleras
med modulationsprodukter från en annan
frekvensnära amplitudmodulerad signal,
varvid korsmodulationen uppstår i olinjära
komponenter i mottagaringången (försteg,
blandare). När man med mottagaren i AMläge ställt in den på någon bärvåg så hörs
också andra starka, frekvensnära stationer.
Det måste alltså alltid finnas en nyttasignal på den inställda frekvensen för att det
skall uppstå korsmodulation. När nyttasignalen försvinner så försvinner även korsmodulationen.
lntermodulation
Vid s.k. intermodulation blandas två starka
inkommande signaler i olinjära komponenter
i mottagaringången. Deras blandningsprodukter faller på mottagningsfrekvensen
så att den störs, vare sig det finns en nyttosignal där eller inte.

Frekvensstab i l it et
Se kapitel 4, Oscillatorer


\chapter{SÄNDARE}

För att översiktligt beskriva en komplicerad
sändare eller transeeiver behövs ibland ett
enklare framställningssätt än detaljrika
principscheman. Då kan ett blockschema
vara till stor hjälp.
Hela apparaten kan ses som ett antal
funktionsblock. Hur de samverkar framgår i
stort av blockschemat. Där återfinns oscillatorer, blandare, förstärkare etc. f schemat
kan även finnas uppgifter om frekvenser
och spänningar m.m.
Det finns olika slags funktionsblock kretsar. Kombinationen av block ger apparater med olika egenskaper. Exempel är
s.k. raka sändare med samma frekvens
genom hela sändaren, superheterodynsändare där frekvensblandning används,
frekvensmultiplicerande sändare etc.

Rak sändare
Bild Ii 5-1

Den raka sändaren är det enklaste sändarkonceptet Då är oscillatorns frekvens samma som sändningsfrekvensen och ingen
frekvensomvandling sker i signalvägen. Om
en antenn kopplas till oscillatorn så blir den
en enkel enstegs sändare.

Oscillator

Bild II 5-1 Enstegs sändare

l>
Ose

Buffertsteg

l flerstegs raka sändare följs oscillatorn
av ytterligare funktioner på samma frekvens
som oscillatorn. Buffertsteg, drivsteg och
slutsteg kan vara sådana funktioner.
Bild 115-2
Bilden visar en rak sändare, som består
av oscillator+ buffertsteg 1 + buffertsteg 2 +
drivsteg +effektförstärkare.
Oscillatorn följs av ett avlastande buffertsteg 1. På så sätt blir oscillatorns frekvensstabilitet bättre. Buffertsteg 2 avlastar
ytterligare och matar dessutom ett effekthöjande drivsteg, som ger driveffekt till
slutsteget, samt slutsteget där den slutliga
effekthöjningen sker.
Raka sändare kan användas för CW,
FM, PM och AM, men inte DSB och SSB.
Fördelen med raka sändare är enkelheten.
Nackdelen är att alla steg arbetar på samma
frekvens, varvid risken för återverkan på ett
föregående funktionssteg är större. Oönskad återkoppling kan då bli följden. Genom
att i första hand bygga in VFO och buffertstegen i metallkapslingar, s.k. skärmar, så
minskas denna risk.

Sändare med frekvensmultiplicering

Helst väljer man en arbetsfrekvens för oscillatorn där den är mest frekvensstabiL
Om högre frekvens önskas på nyttasignalen, så kan man t. ex. multiplicera
oscillatorfrekvensen. l olinjära kretsar alstras övertoner, som ofta utnyttjas i detta
syfte.
Endast när kravet på frekvensstabilitet är
lågt används den frekvens, som VFO eller
CO arbetar på, även för nyttosignalen.

l>

l>

l>

Drivsteg

PA

Bild II 5-2 Flerstegs rak sändare

115-1

SÄNDA R
f Q =8.055...... MHz
sving = 55.55 .. Hz

ca

fo

Sändarirekvens f s= n· fQ

fs=435 MHz:

n =2 · 3 · 3 · 3 = 54

sving

=3000 Hz

Bild II 5-3 FM-sändare med frekvensmultiplicering
Bild II 5-3
Oscillatorn svänger här på en låg frekvens,
som multipliceras i olinjära förstärkarsteg till
en hög sändningsfrekvens. Oftast multipliceras frekvensen två eller tre gånger i vart
och ett av förstärkarstegen.
Bilden visar ett blockschema för en FMsändare för435 MHz (70 cm-bandet). Oscillatorfrekvensen är 8.056 MHz. l fyra av de
efterföljande förstärkarna multipliceras frekvensen 2, 3, 3 respektive 3 gånger, alltså
totalt 54 gånger. Sändningsfrekvensen blir
då 8.056 · 54 = 435 MHz.
Variationer i oscillatorfrekvensen blir också multiplicerade. l detta exempel blir sändningsfrekvensens deviation 54 gånger större än oscillatorfrekvensens deviation. En
deviation av max3000Hz från den nominella sändningsfrekvensen motsvaras av följande deviation från oscillatorfrekvensen,
!J. f=

3000
54

= 55. 6

FM-sändare för VHF, UHF och SHF utförs ofta med frekvensmultiplikation. Jämfört med en rak sändare är komponentbehovet större, men i stället ger den låga oscillatorfrekvensen god frekvensstabilitet, vilket
är en fördel. Risken för oönskade självsvängningar är mindre i en frekvensmultiplicerande än i en rak sändare, eftersom inoch utgångsfrekvenserna i flera av stegen
är olika.

De frekvensmultiplicerande stegen i bild
Il 5-3 arbetar i klass C, d.v.s. olinjärt, vilket
medför amplituddistorsion. Vid frekvens-och
fasmodulering saknar emellertid detta betydelse, eftersom amplituden i det fallet inte är
informationsbärande. Övertoner i nyttasignalen bör dock filtreras bort.

[H z]
Sändningsfrekvens

eller

=

fs =
Sändningsfrekvens
fs

=
=

kristallfrekvens - VFO-frekvens
fq -

fVFO

kristallfrekvens -

+

fq

fvFO

HD
fVFO

=

5 -

\

5,5 Mhz

14 -

14,5 MHz

4 -

3,5 MHz

och
VFO

Bild II 5-4 2-bands C W-sändare med frekvensblandning

115-2

VFO-frekvens

SÄND AR

Telegrafisändare (CW) för kortvåg
Bild II 5-4
En VFO är mest stabil på låga •ral.r\ /o,nct::>r
medan en CO har god stabilitet även på
högre frekvenser. När signalerna från dessa
blandas, bildas blandningsprodukter som är
skillnaden och summan av signalernas frekvenser. Bilden visar en telegrafisändare där
detta fenomen används för sändning inom
området 14.0-14.5 eller 3.5-4.0
beroende på passbandet i filtret efter blandaren.
Resultatet är en superheterodyn-VFO
med både variabel och stabil signaL På
bilden har valts ett filter med passband för
det övre av dessa frekvensområden.

len amplitudmoduleras i en balanserad blandare.! en sådan undertrycks bärvågen medan de båda sidbanden släpps fram. Det ena
sidbandet undertrycks med ett bandpassfilter. Denna SSB-signal flyttas till avsett frekvensband genom ännu en frekvensblandoch ytterligare filtrering.
exemplet är GO-frekvensen 9 MHz.
har frekvensområdet 5.0-5.5 MHz. Vid
blandningen fås blandningsprodukter inom
frekvensområdena 14.0-14.5 och 4.0-3.5
MHz. Genom att välja bandpassfilter kan
man sända i ett av dessa frekvensområden.
Efterföljande driv- och slutsteg utförs för
att arbeta i detta frekvensband, antingen
utan särskild avstämning- s.k. bredbandigt
utförande - eller genom avstämning på en
viss frekvens, vilket ger renaste signalen.

Telefonisändare (SSB) för kortvåg
Bild II 5-5
Bilden visar en SSB-sändare förtvå kortvågsband och bygger på sändaren i Bild 5-4.
Filtermetoden är den mest använda för
att bereda en SSB-signal. Oscillatorsigna-

Bild II 5-6
Bilden visar en SSB-sändare som liknar den
i bild 115-5. Den stora skillnaden är att signalfrekvensen kan flyttas till flera olika band
med hjälp av ännu en frekvensblandning.
Därför används fler valbara bandpassfilter.

Sändare med frekvensblandning

bandfilter
balanserad

SSB-

fs

=

14 -

14,5 MHz

HD
VFO

LF

fVFO "' 5 -

5,5 MHz

Bild II 5-5 2-bands SSB-sändare med frekvensblandning

HD

l

•

C02
'

l
l
l

l

1

--L---  ..J
14 -

14,5 MHz

l

"'•• bandomkopplare
C] f
02 valbar

T

Bild II 5-6 Flerbands SSB-sändare med frekvensblandning

115-3

SÄND AR
l en 88B-signal ligger all information i
amplituden, till skillnad från en FM-signal
där all information ligger i frekvensen. En
88B-signal får alltså inte förvrängas. Det
innebär att förstärkarstegen i 88B-sändare
måste arbeta linjärt, d.v.s. en utsignal skall
vara proportionell till insignalen i varje moment.

Pll-styrda sändare

PLL-styrning är inte ett sändarkoncept. Det
är ett sätt att styra frekvensen i en oscillator
och hålla den stabil med hjälp av en likspänning från en PLL - Phase Locked Loop vilket är en digitalt styrd krets.
En PLL kan användas t. ex. i raka sändare och heterodynsändare. l det första fallet
(bild Il 5-2) kan frekvensen i den enda ocillatorn styras av en PLL. l det andra fallet
(bild Il 5-6) kan frekvensen i någon av oscillatorerna styras av en PLL.
En närmare beskrivning av PPL-styrning
av dessa två sändarkoncept följer här.

PLL-styrd FM-sändare för 144- 146 MHz
Bild II 5-7
Bilden visar en PLL-styrd rak sändare med
en VCO (spänningsstyrd oscillator) och ett
PA (effektförstärkare).
VCO ingår som det frekvensstyrda elementet i en PLL. Utfrekvensen från VCO (ärvärdet) avläses och delas periodiskt med
talet 1O och matas in i en programmerbar
frekvensdelare. Eftersom frekvensområdet
för VCO är 144-146 MHz, kommer infrekvensen till den programmerbara delaren att
ligga i området 14.4-14.6 MHz. Delningstalet i denna delare kan progammeras in i steg
om 1 mellan talen 5760 och 5840.
Med den första delarens divisor 1O och
den andra delarens divisor inställd t.ex. på
5760, så avges ur delarkedjan en puls varje
gång som VCO har genomfört 57600 svängningar. Vid en VCO-frekvens av 144 MHz
(144000 kHz) motsvaras divisorn 57600 (=
1O· 5760) av en pulsfrekvens av 2.5 kHz ut
från räknarkedjan. På samma sätt kommer
en VCO-frekvens av 144025 kHz och divisorn 5761 O (= 1O · 5761) också att ge en
pulsfrekvens av 2.5 kHz, likaså 146 MHz
och divisorn 58400 o.s.v.

Avstämningsspänning
med överlagrad LF-

växelspänning

-·14A -14,6 MHz
LF

~

pragramarbar delare
·7 5760 till 7 5840

-o

knappsats
+

CPU

Avstämningsspänning

delare + 10
LP-filter

Bild 1/5-7 PLL-styrd FM-sändare för VHF

115-4

delad ned till 2,5 kHz
referensfrekvens

OH
co

fo =25 kHZ

referensosci Ilator

VCO-frekvensen låses alltså i intervall
om 25 kHz till närmaste delningstal, för att
uppnå en pulsfrekvens av 2.5 kHz. Om
V GO-frekvensen (är-värdet) awikerfrån det
inställda delningstalet (bör-värdet), så kommer pulsfrekvensen att bli högre eller lägre
än 2.5 kHz.
Pulsfrekvensen jämförs i en s.k. fasjämförare med en kristallstyrd referensfrekvens
som efter en delning med 1O också är 2.5
kHz. Utspänningen från jämföraren är en
likspänning, som intar ett medelvärde då
infrekvenserna är lika, men ett högre eller
lägre värde när de skiljer. Denna likspänning
används för att kontinuerlig styra V GO-frekvensen tilllikhet med börvärd et. Regleringsförloppets hastighet bestäms av tidskonstanten i ett lågpassfilter, det s.k.loop-filtret.

9 MHz

Sändningsfrekvensen regleras alltså med
styrspänningen. Med samma spänning går
det också att frekvensmodulera oscillatorn.
Det görs så, att LF-signalen från modulatorn
överlagras på styrspänningen genom additiv blandning (se sidan Il 3-63) via en kondensator. De variationer i reglerspänningen
som kommer av talet är snabbare än laopfiltrets tidskonstant Variationerna av talet
hinner därför inte uppfattas som frekvensavvikelser och blir därför inte utreglerade.
Orossel n efter laop-filtret förhindrar att moduleringssignalen kortsluts av filtrets kondensator.
Frekvensinställningen, d.v.s. programmeringen av delaren kan utföras på flera
sätt. Exempel på inställningsorgan är
tumhjuls-omkopplare, logikkretsar i kombination med en knappsats o.s.v.

70 MHz

HD
LF

% vco

co

'-

CJ

T

fG. = 61 MHz

70 MHz
bärvågstre kvens

40 -

69,5 MHz

l

-

avstämningsspänning

-

l

0,5-

30 MHz

referensoscillator

Bild 115-8 PLL-styrd SSB-sändare för kortvåg
115-5

SÄNDA R
PLL-styrd sändare för kortvåg
Bild II 5-8
Bilden visar ett avancerat koncept för en
kortvågssändare. SSB-signalen alstras på
frekvensen 9 MHz och blandas med 61 MHz
i 1:a blandaren.
Summafrekvensen 70 MHz filtreras fram
som mellanfrekvens. Den önskade sändningsfrekvensen fås genom att blanda 70
MHz MF med frekvensen från VCO och
därefter filtrera fram skillnadsfrekvensen.
VCO i detta exempel täcker frekvensområdet 40-69.5 MHz. Således blir sändarens
täckningsområde 1.5-30 MHz. För att filterfunktionen skall bli optimal, kan den delas
upp på flera valbara filtersektioner, t.ex. ett
per amatörband. Valet kan ske automatiskt
och styrt av frekvensläget på VCO.
Den absoluta ändringen mellan de två
extrema sändningsfrekvenserna är så stor
som 28.5 MHz eller 1:20. Frekvensändringen
i VCO är 29.5 MHz, men där är ändringsförhållandet mellan de extrema frekvenserna
endast 1:1.74, vilket kan täckas av en enda
VCO. Vid en lägre 2:a MF-frekvens skulle
det behövas flera omkopplingsbara VCO för
att täcka hela frekvensområdet
Exempel: Vid en MF på 9 MHz behöver
VCO-funktionen täcka 9.5-39 MHz, d.v.s.
1:4.11 , vilket är för mycket för en VCO.
SSB-signalen efter 2:a blandaren är inte
lämplig att använda i regleringsslingan i
PLL. Anledningen är att bärvågen är undertryckt i denna signal och att därför HFfrekvenserna i det resterande sidbandet
varierar i takt med de modulerande LFfrekvenserna.
l konceptet på bilden rekonstrueras bärvågen i en 1 :a kontrollblandare, genom
blandning av de två GO-frekvenserna 9 och
61 MHz. Den framfiltrerade bärvågen med
frekvensen 70 MHz blandas med VCO-frekvensen i 2:a kontrollblandare och ur denna
signal framfiltreras den rekonstruerade
bärvågen. Denna stämmer perfekt med den
undertryckta bärvågens frekvens och innehåller inga LF-signaler. Bärvågsfrekvensen
delas i en programmerbar frekvensdelare
och jämförs med frekvensen från en kristallstyrd referensoscillator CO. Ur fasjämföraren erhålls en likspänning som styr VCO via
ett loop-filter. Frekvensen ställs in genom
115-6

att programmera delaren i PLL.
l en modern sändare finns ofta en mikroprocessor, som erbjuder talrika möjligheter
bl. a. till frekvensinställning, minnen och avsökning av frekvenser.
Det beskrivna konceptet är avancerat.
Frekvensen i alla oscillatorer styrs av samma referensoscillator. Frekvensstabiliteten
beror alltså enbart på referensoscillatorns
stabilitet.
Omkopplingen mellan LSB och USB kan
göras antingen genom att behålla SSBfiltret och ändra frekvensen 9 MHz med ett
värde så att filtret blir verksamt i det motsatta sidbandet eller genom att behålla frekvensen 9 MHz och byta till ett SSB-filter som
är verksamt i det motsatta sidband et.
En PLL-styrd sändare har både kristalloscillatorns stabilitet och variabel frekvens
över ett stort frekvensområde trots ett litet
antal styrkristaller. En sådan sändare kan
relativt enkelt styras digitalt.
En principiell nackdel med alla sändare
med PLL-oscillator är fasbruset En annan
nackdel är den stora komponentmängden
(sidan 114-11).

SÄNDARE

l
En transeeiver - transmitter receiver - är
både en sändare och mottagare med delvis
gemensamma funktioner. Dessa kan t.ex.
vara oscillatorer, signalbehandlingskretsar,
filter, strömförsörjning o.s. v., vilket innebär
besparing av ingående komponenter, men
också vissa funktionella begränsningar.
Transceiverkoncept är numera vad som
används allra mest av radioamatörer. Eftersom man på olika vis önskar sig så många
sändar- och mottagarfunktioner som möjligt
inom samma skal, så kan det vara svårt att
undvika kompromisser. Så kan t. ex. en
specialiserad, separat mottagare ha bättre
eller fler egenskaper än i en transceiver.

Jämförelse mellan stationskoncept
Bild II 5-9 visar i stort en station med skilda
sändar- och mottagarfunktioner, men att
antennen är gemensam.
Bild II 5-1 O visar i stort en transeeiver där
VFO och antenn är gemensamma, men i
övrigt med skilda funktioner.
Bild II 5-11 visar samma transceiver, men
med ett mer detaljerat blockschema.

Bild II 5-9 Separat sändare och mottagare

l

l

l

l

.··~
ans l utntng
for fl

extra VFO
1
l

,.,.....

.
l

L--------------.J

Bild II 5-1 O Transeeiver med samma VFO

Bild II 5-11 Direktblandad transeeiver med gemensam VFO

115-7

CW-transceiver med direktblandare

Bild II 5-11
Bilden visar en enkel transeeiver för telegrafi. Sändaren är en raksändare och mottagaren arbetar med direktblandning. För ikanaltrafik räcker det med en gemensam
VFO för sändning och mottagning. Om motstationen svarar exakt på sändningsfrekvensen, vilken ju är VFO-frekvensen, så
erhålls svävningsnoll i mottagaren. För att
få hörbara morsetecken är mottagaren utrustad med RIT, som ändrar VFO-frekvensen med ca 800 Hz vid mottagning.
l konstruktionen finns en anordning kallad KOX (Key Operated Xmitter). Denna
kopplar om transeeivern till sändning när
telegrafnyckeln trycks ner och till mottagning igen efter en viss tid sedan nyckeln har
släppts upp. Telegrafnyckeln styr också en
tongenerator som ljuder i takt med de sända
morsetecknen, s.k. medhörning.
Denna transeeiver är utförd för endast
ett frekvensband och i övrigt mycket enkel.

Kristallstyrd FM-transceiver för VHF

Bild 115-12
Bilden visar en kristallstyrd FM-sändare med
frekvensomkopplare för kanalval inom 144146 MHz-bandet.
En kristallfrekvens av c:a 12 MHz multipliceras 12 gånger i en kedja av förstärkarsteg
för att ge sändningsfrekvensen. Bilden visar
räkneexempel för två frekvenskanaler. Det
frekvenssving i oscillatorn, som alstras av
modulatorn, multipliceras också med 12.
För ett sving av 3 kHz på bärvågen är
svinget på oscillatorn bara 250 Hz.
Efter mikrofonförstärkaren följer en amplitudbegränsare, som skall hålla deviationen inom ett givet maxvärd e, oavsett signalstyrkan från mikrofonen. Därefter följer ett
lågpassfilter, som dels dämpar de övertoner
som uppstårvid amplitudbegränsningen och
dels begränsar de höga frekvenserna i den
modulerade signalen. Båda åtgärderna begränsar bandbredden.
Mottagaren är en dubbelsuper. Den mottagna signalen passerar genom ett förselektionsfilter och en H F-förstärkare för att
i 1 :a blandaren blandas med en lokal signaL

Kanalomkopplare på i: simplextrafik på 145,500 MHz
Kanalomkopplare på 6: relätrafik på 145,025/145,625 MHz
145,SOOMHz : 12 ::::
12,125 MHz:

ro :
@

12 MHz

36 MHz 72 MHz

144 MHz

~~--------------~

r-iDH!

145,025 MHz : 12 :: 1
12, 0854 MHz
1

sving

l

l

l
l

r-------------------------------------------~

l

l
l

l

156,200 MHz : 12 =
13,0167 MHz
l

r-iD ,,
r-iDH

156,325 MHz : 12
13,0271 MHz

=

D 11,155 MHz

T

Bild II 5-12 Kristallstyrd 6-kana/s FM-transceiver för VHF

115-8

mellanfrekvensen i en UH F-mottagare väljas ytterligare tre gånger högre. Den relativt
låga 2:a mellanfrekvensen medger en god
närselektering redan med enkla bandfilter.
En eventuell MF-förstärkare ger tillräcklig
signalstyrka till FM-demodulatorn.
För denna lösning behövs det två styrkristaller för varje frekvenskanal, vilket av
kostnadsskäl kan vara en nackdel.

En kristallstyrd lokaloscillator med efterföljande frekvensmultipliceringssteg alstrar
denna signal.
Lokaloscillatorkedjans utfrekvens läggs
1O. 7 MHz över eller under mottagningsfrekvensen och mellanfrekvensen efter den i :a
blandningen blir då i 0.7 MHz. Skilda oscillatorer används vid sändning resp. mottagning varför styrkristalle rna för sändning resp.
mottagning på en given kanal får olika frekvens. Vid omkoppling till en annan kanal
väljs ett annat kristallpar, vilket lämpligen
sker med samma omkopplare.
Den relativt höga i :a mellanfrekvensen
i O. 7 MHz ger ett så stort avstånd till spegelfrekvensen, att bandbredden i förselektionsfiltren är tillräckligt smal för att undertrycka
spegelfrekvensen. Av samma skäl bör i :a

Pll=styrd FM=transceiver för VHF
Bild II 5-i 3
Den PLL-styrda sändare som redan beskrivits i bild 115-7 har här kompletterats med en
svingbegränsare och ett lågpassfilter i modulatorn. Liksom i den station med kanalkristalier, som beskrivits i bild Il 5-12, är mottagaren även i detta fall en dubbelsuper.

VCO-frekvens vid sändning
144 - 146 M Hz
VCO-frekvens vid mottagning 154,7 - 156,7 MHz

begränsare

M

D--{BD--[§}i~

delad
referensfrekvens 2,5 kHz

T

11,155MHz

Bild 115-13 PLL-styrd FM-transceiver för VHF

115-9

TRANSC
VCO används även som lokaloscillator i
mottagaren. Eftersom sändaren och mottagaren skall användas på samma frekvens
(simplextrafik), måste i detta koncept VCOfrekvensen vara olika vid sändning och mottagning. Eftersom mottagarens mellanfrekvens MF är 1O. 7 MHz måste nämligen VCO
ligga 1O. 7 MHz högre eller lägre vid mottagning än vid sändning. Vid sändning däremot, är VCO-frekvensen densamma som
sändningsfrekvensen.
Den programmerbara frekvensdelaren i
PLL-kretsen arbetar därför med olika delningstal vid sändning resp. mottagning. Inställningen av divisorn kan ske med kanalomkopplare, tumhjulssats, knappsats eller
"VFO-ratt" + digitalräknare o.s.v .. PLLstyrningen ger dessutom möjligheter, t.ex.
att ordna en automatisk avsökning över ett
önskat frekvensområde- s.k. scanning.
Sändning
QRG
Del n.MHZ
tal

Mottagning
QRG
vco
MHz
MHz

Simplexkanaler,
exempel
144.000 5760
144.025 5761

144.000 154.700 6188
144.025 154.725 6189

Repeaterkanaler,
exempel
145.000 5800
145.025 5801

145.600 156.300 6252
145.625 156.325 6253

Deln.tal

VCO-frekvensen är lika vid sändning och
mottagning medan delningstalet bestämmer
arbetsfrekvensen.

115- 1o

Kortvågstransceiver för SSB och CW
Bild II 5-14
Vi har redan beskrivit en KV -sändare och
KV-mottagare för SSB. l det koncept på en
kortvågstransceiver, som visas här, ingår
en super-VFO i signalberedningen. VFOsignalen (5 - 5.5 MHz) blandas med signalen från en kristallstyrd CO, vars frekvens är
valbar med en bandomkopplare. Samtidigt
kopplas ett bandpassfilter in efter blandaren
i super-VFO, som svarar till det aktuella
frekvensband et.
För t.ex. 21 MHz-bandet är VFO-filtrets
passband 12-12.5 MHz. När en VFO-signal
12-12.5 MHz blandas med en 9 MHz SSBmodulerad signal erhålls en frekvens i området 3-3.5 MHz och en frekvens i området
21-21.5 MHz. Den önskade av dessa frekvenser filtreras fram med omkopplingsbara
bandpassfilter, vilket sker med den bandkopplare som nämnts tidigare.
l den enkla kortvågssändare som beskrivits tidigare är det tillräckligt med en enda
sats av omkopplingsbara bandpassfilter. Det
större antalet filter i den här beskrivna utrustningen behövs för att även kunna använda super-VFO som en del i mottagaren,
vilken arbetar som enkelsuper. Eftersom en
MF på 9 MHz används även i mottagaren
kommer mottagning och sändning att kunna
ske på samma frekvens.
Mottagaren beskrivs inte närmare. Med
lämpliga omkopplingsanordningar kan vissa funktionsblock i transeeivern användas
både vid mottagning och sändning. Bilden
visar en SSB-transceiver där passbandfilter
i förkretsar, M F-filter och kristalloscillatorer
har dubbel användning. Funktionsblocken
visas inplacerade i sina alternativa funktioner, däremot inte omkopplingsanordningarna.
Vid sändning och mottagning av CW
förbikopplas den balanserade modulatorn
och kristallfiltret i signalbehandlingskretsarna för 9 MHz. För mottagning av CW ändras
BFO-frekvensen i mottagaren så att det
hörs en svävningston när en bärvåg tas
emot. Utan denna frekvensändring skulle
endast bärvågsbruset höras.
Även en RIT och en VOX (Voice Operated Xmitter, talstyrd sändnings-/mottagningsomkoppling) är inritade.

EIVER

9 MHz -

SSB-signalbehandling'

LSB USB CW

LF

omkopplingsbart
bandpassfilter

t

!
D

T

sändning

S1 -

S 5 gangade

( =manövreras samtidigt)
mottagning

omkopplingsbara
kristaller

* ,** ,***

Blockschema
Dessa delar är gemensamma för både sändning och mottagning, men visas med
båda placeringarna exkl. omkopplingsanordningar

Bild 115-14 SSB-transceiver för kortvåg

115- 11

EPT

9 MHz
SSB-

70 MHz-

M

*****

----

t
VCO 40-69,5 MHz
avstämningsspänning

för frekvensval

D

T
1-------------- för 3:e MF för FM

.--------l .A. GC

*****
D

*·**·***·****•*****:

T

61 MHz

Dessa delar är gemensamma för både sändning och mottagning, men visas med
båda placeringarna exkl. omkopplingsanordningar

Bild 115-15 PLL-styrd SSB-transceiver för kortvåg

115- 12

D

T

TRANSCEIVER
PLL-styrd kortvågstransceiver

Bild II 5-15
En modern transeeiver i den högre prisklassen, i s.k. "all-mode"-utförande, erbjuder många funktionella möjligheter. Flera av
dem kommer emellertid endast till användning i speciella situationer. Konceptet
för en sådan transeeiver beskrivs här i stort.
Huvudprincipen för signalbehandlingen kan
beskrivas som en PLL-styrd dubbelsuper.
SSB-signalen bereds på 9 MHz-nivån och
flyttas därefter upp till70 MHz-nivån genom
frekvensblandning och filtrering. De möjliga
sändningsfrekvenserna mellan 0.5 och 30
MHz skapas genom att blanda den fasta
SSB-signalen med en variabel frekvens från
VCO. Den steglösa frekvenstäckningen som
innefattar mellanvågs- och kortvågsområdet är emellertid endast avsedd för mottagningsfunktionen i transceivern. För sändningsfunktionen kan tillkomma blockeringskretsar, som förhindrar sändning utanför
tillåtna frekvensband.
Denna förenklade beskrivning omfattar
inte kristalloscillatorerna för 9 och 61 MHz i
fasregleringskretsen och inte heller SSBmodulatorn, FM-modulatorn och anordningarna för CW-sändning.
Mottagaren är en dubbelsuper med hög
1 :a M F-frekvens. Mottagare för höga frekvenser kan till och med utföras som en
trippelsuper. Samma bandpassfilter, blandare och kristallfilter används både vid sändning och mottagning.
Genom lämplig programmering av
frekvensdelaren kan sändning och mottagning ske på samma frekvens eller på
skilda frekvenser (split-trafik).
En extra VFO-funktion kan åstadkommas genom att frekvensdelaren programmeras med delningstal som hämtas från ett
digitalt minne. Den extra VFO-funktionen
kan sedan efterjusteras genom att ändra
delningstalet med frekvensratten. Minnet
blir ännu mer användbart, om det förutom
frekvenser också kan lagra uppgifter t.ex.
om sändningsslag och andra inställningar.

Sammanfattning

Till skillnad från den raka sändaren är den
här beskrivna PLL-styrda transeeivern
mycket komplicerad. Den tekniska utvecklingen går fort. Nya, bättre och mer invecklade apparater utvecklas ständigt. Men det
är inte alls nödvändigt att använda det senaste och mest avancerade inom apparattekniken för att utöva amatörradio. Det går
mycket bra att börja med enkla medel och
med liten ekonomisk insats.
Det finns ett stort utbud av begagnade
apparater som i olika avseenden är konkurrenskraftiga med senare konstruktioner. Det
ligger i amatörradions traditioner att ta tillvara tillgänglig utrustning och förbättradenna
efter bästa förmåga.
Ytterst beror resultatet och framgången
mest på radiooperatörens skicklighet, val av
frekvens, antenn och tillfälle.

115-13

TRANSC

115-14


\chapter{ANTENNSYSTEM}

Aldrig så förnämliga radioapparater kommer inte till sin fulla rätt utan ett effektivt
antennsystem. Det är en huvudförutsättning för framgångsrik radiokommunikation.
Antennen omsätter elektrisk energi från
sändaren till elektromagnetiska fält som strålas ut, d. v. s. radiovågor.
Vid mottagning fångar antennen upp
radiovågorna och omsätter dem till elektriska signaler som förs till mottagaren.
Antennsystemet består av den egentliga
antennen och transmissionsledningen mellan denna och sändaren respektive mottagaren. l antennsystemet ingår även impedansanpassningar, antennkopplare m. m.

Antenner- allmänt
Våghastighet
l vakuum breder elektromagnetiska vågor
ut sig med hastig heten c 0 , vilken mest kallas
ljushastigheten.
C0 :::< 300 ·1 Os [m/s]

l andra media än vakuum har samma
vågor utbredningshastigheten c
Formeln är då

c=~

[m/s]
!lo. s,
där !lo är relativa permeabilitetskonstanten
och Er är relativa dielektricitetskonstanten för
det medium som vågorna passerar igenom.
För enkelhetens skull sätts här !lo och er till
1 , alltså c 0 = c.
Sambandet mellan våghastigheten i vakuum, frekvensen och våglängden är förenklat
c= A. f
c [m/s] f[Hz] A [m]
och våglängden således A=

!j

[m]

Antennlängd
Elektriska längden
Längden för en resonant, ideal antenn som
är en våglängd lång kan beräknas med
ovanstående formel. Vi kallar denna längd
för den elektriska längden. Således le= A.

Elektriska längden (le) för en halwågsantenn (A/2) är hälften av den elektriska
längden för en helvågsantenn (A):

l =~

[m]
2f
Mekaniska längden
Man skiljer på antennens elektriska och
mekaniska längd. Av flera orsaker blir den
mekaniska antennlängden (lm) för samma
frekvens kortare än den elektriska {1 6 ). Det
beror bl. a. på våghastighet och ledningsförmåga i de material som ingår samt övriga
elektriska egenskaper beroende på antennens mekaniska utförande, påverkan från
jordplan och omgivning m. m.
Ett förhållande mellan längd och tjocklek
av 10000 ger t. ex. en c:a 2 $\circ$/o mekaniskt
kortare antenn. Förhållandet 30 ger en c:a 5
o/o kortare antenn. Det första värdet kan
passa för en 2 mm tjock halvvågsantenn för
7 MHz. Det andra värdet för en 3.5 mm tjock
halvvågsantenn för 145 MHz. Diagram för
den s.k. förkortningsfaktorn finns i de flesta
ante n nhandböcker.
l följande formel har den mekaniska längden (lm) för en fritt upphängd trådantenn
valts 2 $\circ$/o kortare än den elektriska längden.
e

l
m

=~ .
2f

O 98 = 147 . 1os
.
f

[m]

Exempel: Beräkna den elektriska och mekaniska längden på en halvvågsantenn med
resonansfrekvensen f= 7 MHz.
c= A, . f
c [m/s] f [H z] A [m]
Elektriska våglängden för 7 MHz är

A =E= 300. i os
f
7 ·1 os

= 300 = 42.86

[m]

7

Antennen är en halvvågsantenn, således är
elektriska längden

1 =~ =
e
2

42 86
·

2

= 21.43

[m]

och

mekaniska längden

l

m

= 3:
2f

. 0.98 =

42 86
· · 0.98 = 21

2

[m]

116-1

ANTENNSYSTE
)..

r--2-~~,

~
..........

Ström- och spännings-''-... ...... .!)-fördelning

Bild II 6-1 Spänning och ström i en
halvvågsantenn

l mpedansfördelning

Bild II 6-2 Matningsimpedansen
i en halwågsantenn

116-2

Ström och spänning i en halvvågsantenn
När en halwågsantenn matas med HF-energi på grundfrekvensen, så uppstår en stående våg med ett typiskt utseende.
Bild II 6-1 visar att i vardera änden av
antennen uppnår spänningen U ett maximum (en spänningsbuk), l mitten uppnår
strömmen I ett maximum (en strömbuk).
Antennen strålar mest där strömbuken finns .
Tag t.ex. en 40 meter lång metalltråd
som antenn. Dess grundresonansfrekvens
är ca 3.5 MHz, men den är även i resonans
på de harmoniska övertonerna (7, 14, 21,
28 MHz o.s.v.).
Bild II 6-3 visar ström- och spänningsfördelningen på antennen vid de respektive
övertonerna.
80 m (3.5 MHz):
l matningpunkten står ett spänningsminimum (en spänningsnod) och ett strömmaximum (en strömbuk). Strömmen är hög
därför att matningspunkten har låg impedans.
Samma antenn på 40 m, 20 m, 15 m, 1O
m (7, 14, 21, 28 MHz) har ett spänningsmaximum (spänningsbuk) och ett strömminimum (strömnod) i matningspunkten, som
då har hög impedans.
Ur horisontaldiagrammet för antennen
kan utläsas att ytterligare strålningskäglor
(strålningslober) utvecklas för varje överton
i den påmatade frekvensen. Samtidigt blir
strålningen alltmertill riktad längs med antennen.
Impedansen i antennens matningspunkt
Impedansen Z för varje punkt på en antenn
kan beräknas med Ohms lag Z = U/1.
Bild II 6-2
l mitten av en halwågsantenn på grundfrekvensen är impedansen Z låg eftersom
spänningen är låg där och strömmen hög.
Ute i ändarna är däremot impedansen hög
eftersom spänningen där är hög och strömmen låg.
Impedansen i mittpunkten är 73 Q på
grundfrekvensen, när antennen mätt i våglängder befinner sig mycket högt över jordytan, d.v.s. utan nämnvärd påverkan från
omgivningen. l praktiken kan impedansen
awika mycket från detta värde.

ANTENNSYSTE

3,5MHz {80m)

2:5',

..ooO!-o........
, ......
-------......

-- ---~y?/ 7
......

7 MHz (40m)

~

o<lli-------.. . . .
'

/

Grundfrekvens =
1 :a harmoniska

l::

t

.......

/

2:a harmoniska

l::2·-'}

4:e harmoniska

l= 4
21MHz(15m)
28 MHz (10m)

6

t

6:e harmoniska

l::6·t

Ä"> /'75:'> ..,.-?S> /. .
' . . SZ7 . .,szy ,\ Z./

zs:>. . <.;.(:;;,. -

8:e harmoniska

(::8·f

~---------------40m----------------~

-ström

HORISONTALDIAGRAM FÖR EN
HARMONISKA ÖVERTONER

- - - ... spänning

t-

DIPOL VID MATNING MED

Grundfrekvens

3:e harmoniska

2:a harmoniska

Bild II 6-3 Halwågsdipol matad med harmoniska övertoner

116-3

ANTENNSYSTE
Antenn och matningskabel måste vara
impedansanpassade till varandra för att det
inte skall skall uppstå vågreflexion i anslutningen.
Märk, att halwågsantennen är i resonans inte bara på grundtonen utan även på
jämna övertoner, 2:a, 4:e etc. harmoniska,
varvid matningspunkten har hög impedans.
Vid matning med en lågohmig koaxialkabel
uppstår då en kraftig missanpassning i anslutningen mellan antenn och kabel, vilket
måste åtgärdas på något sätt. Se avsnittet
Transmissionsledningar i detta kapitel.
Matningsimpedansen i några antenner
Med W3DZZ-antennen (se nedan) löses
hjälpligt anpassningsproblemet med mittmatade partier på 2:a harmoniska övertonen, d.v.s. dubbla grundfrekvensen. På 80och 40 m-banden är antennens matningsimpedans c:a 60 .Q och på de högre banden
ca 120 n. En kompromiss är att mata denna
antenn med en 75 n-kabel för att inte få
alltför stor missanpassning på något band.
Den omvikta dipolen (folded dipole):
Matningsimpedansen är c:a 240 n. En
bandkabel med impedansen 300 n kan användas alternativt en koaxialkabel med
impedansen 50 eller 75 n över en transformator med impedansomsättningen 4:1.
Jordplanantennen (GP-antennen):
Matningsimpedansen är 30-60 n. När
jordplanets spröt inte riktas horisontellt, utan
snett nedåt, erhålls en matningsimpedans
av 50 n, vilket passar bra för en koaxialkabel med 50 .Q impedans.
Yagi- och Quad-antenner:
En anpassningsanordning för anslutning
av 50-60 n koaxialkabel ingår oftast i fabriksgjorda riktantenner. En 50-60 n koaxialkabel kan då användas direkt.

Reaktansen i en icke resonant antenn
Den elektriska svängningskretsen behandlas i kapitel 3. Där framställs svängningskretsens grundegenskaper resistans R, induktans L och kapacitans C som koncentrerade till komponenter kallade resistor, induktor respektive kondensator.
116-4

Även en enkel tråd har dessa egenskaper, men utfördelade över hela tråden. Denna
kan därför ses som ett stort antal komponenter, som tillsammans bildar en svängningskrets, vilken naturligtvis kan fungera som
antenn.
När antennen matas med växelström
med samma frekvens som antennens resonansfrekvens, så svänger antennen med de
minsta förlusterna. Resonansfallet kan i
korthet beskrivas så att den induktiva och
kapacitiva reaktansen i antennen tar ut varandra medan resistansen kvarstår.
Impedansen är vektorsumman av resistansen och de kapacitiva och induktiva reaktanserna. l resonans är antennens impedans lika med resistansen, vilket är ett
specialfall. Antennströmmen har alltid sändarens frekvens. Om sändningsfrekvensen
är en annan än antennens resonansfrekvens, så händer endera av följande:
När antennströmmen har lägre frekvens
än antennens resonansfrekvens, så blir den
resulterande reaktansen negativ (kapacitiv),
d.v.s. Xc är större än XL.
När antennströmmen har högre frekvens
än antennens resonansfrekvens, så blir den
resulterande reaktansen positiv (induktiv),
d.v.s. XL är större än Xc.

Elektrisk "förlängning" och "förkortning"
Om sändarfrekvensen, awiker mycket från
antennens resonansfrekvens, så kan
reaktansen i antennen behöva elimineras
eller åtminstone minskas för en bättre
impedansanpassning mellan antenn och
matarledning. Den enklaste åtgärden är då
att försöka ändra antennlängden.
Bild II 6-4. Om detta inte låter sig göras,
så kan man i serie med en "för kort" antenn
sätta in en induktor - en s.k. elektrisk förlängning. Om i motsatt fall antennen är "för
lång", så kan man sätta in en kondensatoren s.k. elektrisk förkortning.
l amatörradio ändras sändarfrekvensen
mycket och ofta, varför antennsystemet bör
kunna stämmas av från marken/operatörsplatsen. Då kan en antennkopplare med
nödvändiga reaktiva komponenter behövas.
Se längre fram i kapitlet.

NSYSTE
dipol (l:::

eller

f)

~~

Dipol, elektriskt förlängd

Dipol, elektriskt förkortad

förkortning av antenner

Anpassning till sändarens impedans
Ett sändars lutsteg med elektronrör är vanligen utrustat med en avstämningsanordning
vid H F-utgången. Syftet är att kunna anpassa sändarens utgångsimpedans till impedansen i antennledningen. l moderna sändare består denna anordning mycket ofta av
ett s.k. n-filter, vars utgångsimpedans kan
variera mellan c:a 30-150 n.
Ett transistoriserat slutsteg är oftast utfört för en fast utgångsimpedans av 50 n
och är alltså i behov av en avstämningsanordning, om inte antennsystemet inom vissa
gränser håller samma impedans. Toleransgränsen för felanpassning brukar vara ett
SVF av storleksordningen 2:1 innan sändarens skyddskretsar styr ner uteffekten.
Vid lika impedans i sändarutgång, matarledning och antennanslutning uppträder
ingen stående våg på matarledningen och
mesta möjliga effekt överförs från sändaren
till antennen.

h ::::

Antennens strålningsdiagram
En antenns strålningsbild beskrivs bäst i tre
dimensioner. Bild II 6-3 visar bl. a. ett horisontaldiagram för en halvvågsantenn.
Bild II 6-5 visar strålningen i vertikalplanet som funktion av antennhöjden för samma
antenn. Vertikaldiagrammet kan ha mycket
olika utseende beroende på antennens utförande, dess elektriska höjd över mark och
omgivningens elektriska egenskaper. För
att överbrygga stora avstånd, måste antennen ha en flack utstrålning relativt markplanet. En horisontelit upphängd antenn med
en längd av ')J2 har övervägande flack utstrålning när den placeras på en höjd av AJ
2, A, 3 A/2, 2 A o.s.v. över mark. När en
horisontell antenn däremot placeras /J4, 3
A/4, 5 'A/4 o.s.v. över mark, är utstrålningen
övervägande vertikal, vilket inte skall förväxlas med polarisationen, som i detta fall är
horisontell.
Samma diagram gäller både för en
sändar-och mottagaranten n. styrkan på en
utstrålad signal motsvaras av styrkan på
mottagen signal.
Antennvinst
Med antennvinst G (eng. gain) menas förhållandet mellan effekten Pt i huvudstrålningsriktningen (framriktningen för en antenn med osymmetriskt utstrålad effekt) och
effekten från en definierad referensantenn.
En referensantenn som tänks vara oändligt liten och som strålar med exakt samma
effekt Pi i alla riktningar kallas isotropisk
antenn.
En isotropisk antenn är emellertid endast
teoretisk definierbar.

=A2

Bild II 6-5 Vertikaldiagram för halvvågsantenn

116-5

ANTENNSYSTE
Med effekten Pi från den isotropiska antennen som referens blir antennvinsten

P.

G= 1O log t
[dBi]
~
En i praktiken definierbar referens är
halwågsdipolen, vars huvudstrålning ärvinkelrätt ut från dipolen och runt omkring den.
Referenseffekten är då Pd och antennvinsten

P.

G=10 logL
pd

Di pol

[dBd]

Bild II 6-6

Bild II 6-6 Antennvinst dBd i effekt
Antennvinsten kan också definieras som
förhållandet mellan den elektriska fältstyrkan
uf i huvudstrålningsriktningen och referensfältstyrkan
(dipol).
Jämfört med A./2-dipol är antennvinsten

ud

u

ud

[dBd]

Ungefärlig antennvinst för olika antenner med en isotropantenn som referens
A./2-dipol
Isotrop
Isotrop antenn
-2.1 dBd O
dBi
GP, A/4
-1.8 dBd 0.3 dBi
Dipol, A/2
O
dBd 2.1
dBi
1.2 dBd 3.3 dBi
GP, 5/8 A
Dipol, 1/1 A
2-elements yagi
2-elements quad
3-elements yagi

Riktantenn

G= 20 log-'

6 dB antennvinst motsvarar en fördubblad fältstyrka [V/m], d.v.s. 1 S-en hets ökning
vid den mottagande stationen, liksom att
6 dB antennvinst motsvarar en 4-faldigad
sändareffekt [W/m 2 ).

Bild II 6-7

1.8

5

6
8

dBd
dBd
dBd
dBd

3.9
7.1
8
10.1

dBi
dBi
dBi
dBi

Effektivt utstrålad effekt

Effektivt utstrålad effekt (ERP - effective
radiated power) är den effekt som sändarantennen strålar ut i sin bästa strålningriktning. ERP beräknas som den effekt som
tillförs själva antennen, multiplicerat med
antennvinsten relativt en halwågsdipol. Förlusterna på vägen från sändaren ut till antennen är alltså borträknad före beräkningen
av ERP.

Fram-/backförhållande (antennvinst)

Di pol

Med fram-/backförhållande (F/B) för en riktantenn menas förhållandet mellan den utstrålade effekten i framriktningen P1 och
effekten i backriktningen Pb

Riktantenn

Bild II 6-7 Antennvinst dBd i spänning
Man använder uttrycket d Bi när antennvinsten anges i förhållande till en isotrop
antenn och d Bd i förhållande till en halvvågsanten n.
Se Appendix C om decibelbegreppet
Exempel på beräkning av antennvinst
U1 = 40 J-tV Ud= 20 J-tV
G=?

u

40
G= 20 log-r = 20 log-=
ud
20
=20 log 2=20·0.3=6

116-6

[dBd]

P.
pb

Fl B= 10 logL

Di pol

[dB]

Bild 6-8

Riktantenn

Bild II 6-8 F/B-förhållande i effekt
Fram/backförhållandet kan också definieras som förhållandet mellan elektriska
fältstyrkan uf i framriktningen och referensfältstyrkan ub i backriktningen

ANTE N
u
ub

Fl B= 20 log-'

Di pol

[dB]

Bild 6-9

Riktantenn

Bild II 6-9 F/B-förhållande i spänning

Exempel1
Ut= 40 ~-LV

Ub = 4 ~-LV

F l B= 20 log

u, = 20
ub

F/B= ?
log

40

=
4
=20 log 10=20·1=20 [dB]

F/B = 20 dB betyder att fältstyrkan Ut i
huvudriktningen är 1O gånger så hög som
referensfältstyrkan Ub.
Exempel2
Ut= 15 ~-LV Ub = 15 ~-LV

F/B= ?
15
Fl B= 20 log-' = 20 log-=
15

u
ub

=20 log 1=20·0=0

F/B= O dB betyder att Ut= Ub, d.v.s. att
fältstyrkorna i fram- och backriktning är lika
stora, vilket inträffar för en dipol.

Halvvärdesbredd
studera diagrammet för den horisontella
strålningen från en riktantenn.
Antennen avger sin största utstrålade
effekt Pt i huvudriktningen. Effekten avtar
utanför huvudriktningen. Fältstyrkan Ut förhåller sig på liknande sätt.
Med effekthalvvärdesbredd menas den
vinkel inom vilken nyttaeffekten är minst
hälften så stor som i huvudriktningen.
Bild II 6- i O
p
Observera, att
motsvarar ~

[T

d

2 u,

( ~ 0,7 Ut motsvarande 3 dB).
Med spänningshalvvärdesbredd menas
den vinkel inom vilken spänningen (fältstyrkan) är minst hälften så stor som den
största nyttaspänningen Ut. Spänningshalvvärdesbredden på en di pol är ungefär 90$\circ$.
Bild 116-10

[dB]

..öppningsvinkel (X

Umax

u
Effekthalwärde

,Öppningsvinkel

\

\

/

/

\c-6d8
0,5
J

Spänningshalwärde

l

/j;

Umax

u

l

Bild II 6-1 O Halvvärdesbredder

116-7

ANTE N
Vågpolarisation
Se även i kapitlen 1 och 7. Här nämns något
om polarisation vad gäller radioantenner.
En elektromagnetisk våg är sammansatt
av ett magnetiskt och ett elektriskt fält, vinkelrätt orienterade mot varandra.
Polariseringsriktningen för en elektromagnetisk våg definieras som den riktning
som dess elektriska fält har;
vertikalt elektriskt fält- vertikal polarisation,
horisontellt elektriskt fält- horisontell polarisation.
Polarisationsriktningen på de utsända
radiovågorna beror i främst på sändarantennens utförande.

Polarisation på HF- Kortvåg
För bästa mottagning skall användas en
antenn för samma polarisationsriktning som
i den infallande vågen. Vilken polarisation
man väljer är av mindre betydelse än att den
börvara lika både i sändar-och mottagarantennen. På kortvåg är det nödvändigtvis inte
samma riktning som den från sändarantennen, eftersom de utsända vågorna oftast
har reflekterats i jonosfären. Det kan då
uppstå en polarisationsvridning som inte
kan förutses. Att då kunna växla mellan
mottagaranten ner med olika polarisation kan
vara en fördel. Riktantenner för kortvåg monteras nästan alltid med horisontella element
-horisontell polarisation.

116-8

Polarisation på VHFIUHF!SHF

l dessa högre frekvensområden tilläm-

pas både horisontell, vertikal och cirkulär
polarisation.
Polarisationsriktningen ändras inte spontant under överföringen så länge som vågorna inte reflekterats på vägen. Jämför
med sändningar från rymdsatelliter då två
program sänds på samma frekvens, men
med olika polarisation. satelliten får då inte
ändra läge i förhållande till jorden.
För cirkulärt polariserade antenner, där
polarisationen vrider sig omkring utbredningsaxeln, gäller att överföringen är bäst,
när vridningens riktning är lika både i sändar-och mottagarantennen.
Bild II 6-11

Lämpligt antennarrangemang

Olämpligt antennarrangemang

Bild II 6-11 Inverkan av polarisation

NSYSTEM
Antenner för kortvåg
Mittmatad halvvågsantenn

Se föregående avsnitt

Ändmatad halvvågsantenn

Utstrålningen från en halvvågsantenn är i
princip lika hur den än matas. En än d matat
halvvågsantenn fungerar m.a.p. strålningsriktningar på samma sätt som en mittmatat
Vid längre antenner blir strålningskaraktären däremot en annan.
Skillnaden mellan änd- och mittmatade
halvvågsdipoler är att anslutningsimpedansen är mycket högre i ändarna än i
mitten. För att mata antennen längst ut i ena
änden behövs en transmissionsledning med
hög impedans, varvid ledningens ena part
ansluts till antennen och den andra parten
lämnas fri. En sådan anordning kallas zeppantenn och användes först i luftskepp, s.k.
zeppelinare.

Omvikt dipol (folded dipole)

Bild II 6-12
En omvikt di pol kan ses som två eller flera
parallella element, som är sammankopplade i ändarna. Mittpunkten på ett av elementen är ansluten till antennledningen.
Matningsimpedansen för en omvikt /J2dipol med två element är c:a fyra gånger
högre än den för en enkel dipol, d.v.s. 200
- 300 Q. Den omvikta dipolen, som endast
fungerar på grundfrekvensen och på dess
udda övertoner, är relativt bredbandig. Matningsimpedansen kan ändras med sinsemellan olika diametrar på de ingående elementen samt med antalet parallellkopplade
element.

Bild II 6-12 Omvikt dipol

Jordplanantenn

Bild II 6-13
Jordplanantennen eller GP-antennen (GP
av ground plane) består av en lodrätstrålare
som den ena polen och flera sammankopplade A./4-radialer eller markplanet som den
andra polen.
GP-antennen är rundstrålande och har
vertikal polarisering. Dess relativt flacka utstrålning, i jämförelse med en horisontell
antenn, gör den lämpad för långa distanser.
Av mekaniska skäl används den mest på
högre frekvenser (14 MHz och högre).
Med horisontella radialer som jordplan
är matningsimpedansen c:a 35 Q. För att få
god impedansanpassning, t.ex. till en 50 Q
koaxialkabel som matarledning, görs radialerna sluttande nedåt i en lämplig vinkel.
Koaxialkabelns innerledare ansluts till
antennen och kabelskärmen till radialerna.
Om antennen placeras omedelbart ovan
markytan, kan marken användas som jordplan, särskilt om dess elektriska ledningsförmåga är god.
Bild II 6-14
Om antennelementet inte har en elektrisk längd av ')J4, kan längden anpassas
elektriskt på liknande sätt som beskrivits
tidigare i detta kapitel för dipolantenner.

Bild II 6-13 GP-antenn

116-9

GP med
seriekondensator

l >

GPmed
toppkapacitans

t

Bild II 6-14 GP-antenner med elektrisk längdanpassning

Flerbands GP-antenner

Antennen fungerar som 'A/4 GP-antenn
åtminstone på de lägsta banden. Den mekaniska längden på en flerbands GP för
kortvåg blir kort, 4 6.5 meter, vilket på de
lägre banden innebär dålig verkningsgrad
och liten bandbredd. Jämför med SVF-kurvorna på bilden. Flerbands G P-antenner för
upp till sju kortvågsband tillverkas.

En GP-antenn kan fås att fungera på flera
band genom inbyggnad av en spärrkrets i
antennelementetför tillkommande band och
av jordplansradialer med anpassad längd
eller med spärrkretsar även i jordplanet för
de banden.
Bild II 6-15

a

SVF

SVF

'

l

1\ J

3.5

- 1 - - - --

-r---··- --- -·· - -

MHz

3

'l'..

5
1.u

.B

.1

svF

SVF
3

.6

28

2 .4

.6

.8 29 .2 .4

MHz

1

-

l

.05

J
MHz

.10

SVF
J

GP
( 80 /40 l 20 /15 /10m )

Bild II 6-15 SVF-kurvor för flerbands GP-antenn
116-10

1.5
1.2
1.0

i""'oo.

14

-

.2

...,..,.,.
.3

MHz

Det finns flera principer för denna
antenntyp. l den typ som visas här
används spärrkretsar.

Flerbands halvvågsantenner
Bild II 6-i6

En vanligt förekommande flerbandsantenn
är W3DZZ-antennen (namnet efter konstruktörens anropssignal). Den är en oftast horisontellt upphängd dipolantenn för 80, 40,
20, i 5 och i O m-banden.
W3DZZ-antennen är c:a 33.6 meter lång
och har två spärrkretsar, symmetriskt utplacerade omkring matningspunkten. Matningen sker med koaxialkabel och balun.

Antennen har en matnings impedans av
c:a 50 Q på 80- och 40-metersbanden På de
högre banden är anpassningen inte optimal
- matningsimpedansen stiger där upp till c:a
120 Q. Många använder bl. a. av den anledningen inte W3DZZ-antennen på höga kortvågsband utan föredrar där en flerbandig
GP-antenn eller en riktantenn (Yagi, quad
m.fl.).

Fysiska data
6,7tm

Praktiskt utförande
Balun1:1

spärrkrets

l

spärrkrets

om möjligt 6 mtr
lodrätt nedåt

koaxialkabel

Strömfördelning

80m

40m

./---o------------~~
~~~~

Bild II 6-16 W3DZZ-antennen

116-11

ANTENNSYSTE
W3DZZ-antennens arbetssätt:
80m-bandet
Hela antennen fungerar som en A/2dipol med resonansfrekvensen 3.7 MHz.
Den mekaniska längden är 2 · 16.8 meter
och förlängs elektriskt med induktanserna
i spärrkretsarna, vilka f.ö. är ur resonans
på detta band.
40 m-bandet
Spärrkretsarna är i resonans och "kopplar bort" antenndelen utanför dem. Delen
där innanför fungerar som en A/2-dipol
med resonansfrekvensen 7.05 MHz.
20m-bandet
Hela antennen fungerar som 3A/2-dipol
med resonansfrekvensen 14.1 MHz.
15m-bandet
Hela antennen fungerar som 5A/2-dipol
med resonansfrekvensen 21.2 MHz.
10m-bandet
Hela antennen fungerar som 7A/2-dipol
med resonansfrekvensen 28.4 MHz.

Riktantenner för kortvåg

Riktbar dipol-antenn
Bild II 6-17
En dipolantenn av måttlig mekanisk storlek
kan göras vridbar så att utstrålningen kan
riktas. Men eftersom en ensam di pol strålar
i många riktningar, låtvara mestvinkelrätt ut
från antennen, så är energin i flesta riktningarna att ses som "förlorad". När ett passivt
antennelement - en reflektor - placeras
bakom det aktiva elementet kan emellertid
bakåtstrålningen delvis vändas framåt och
man får i stället en viss riktverkan. För att det
skall fungera skall de båda elementen ha en
viss inbördes längd och ett visst avstånd
emellan.
Vagi-antenner
Bild II 6-18
Med ytterligare passiva antennelement
- s.k. direktorer- framför det aktiva elementet, blir riktverkan ännu bättre. Reflektorn är
alltid elektriskt längre än strålaren och
direktorerna alltid elektriskt kortare och allt
kortare ju längre bort från strålaren. En
sådan antenn kallas Yagi-antenn, efter sin
japanske upphovsman. Den är ursprungligen avsedd för ett enda frekvensband, en
s.k. monoband-beam.

116- 12

Om alla element förses med lämpliga
spärrkretsar, med W3DZZ-antennen som
förebild, fås en riktantenn som är användbar
på flera frekvensband, en s.k. multibandbeam. De vanligaste flerbandsantennerna
är konstruerade för 20 m-, 15m- och 1Ombanden och har två till tre element.
Bilden visar riktbara multibandantenner
med 2, 3 resp 5 element samt deras strålningsdiagram i horisontalplanet.
Matningen sker oftast med en koaxialkabel med 50 n karakteristisk impedans. Eftersom matningsimpedansen för själva riktantenn nästan aldrig är 50 n, så behövs
oftast en impedansanpassning mellan antenn och kabel.
Cubical Quadaantenner
Bild II 6-19
Cubical quad-antennen är en kvadratisk
helvågsstrålare med en sidlängd av /J4,
d.v.s. totalt 1 A.
En 2-elements quad-antenn består av
en strålare och en reflektor på ett inbördes
avstånd av 0.15 - 0.2 A. Det finns även 3och 4-elements quad-konstruktioner med
beaktansvärda dimensioner. Antennen görs
lämpligen vridbar och bör monteras åtminstone 3/4 "A över mark.
Matningsimpedansen är 50-70 n, beroende på elementavståndet Matningen sker
oftast med en koaxialkabel. Beroende på
hur matningspunkten placeras är det möjligt
att välja mellan kan horisontell eller vertikal
polarisering.
Det finns två utföranden av quad-antenner, det ena med en bärande bom med
spridare för att bära upp antennelementen
och det andra med bara spridare från ett
centralt fäste, den s.k. spider quad (spindel).
Quad-antennen byggs för vanligen för
20 m-, 15 m- och 1O m-banden. Spiderprincipen är att föredra vid flerbandsutförande, eftersom ett optimalt elementavstånd kan väljas för varje band utan att
spridarantalet behöver ökas.
Genom den flacka strålningsvinkeln är
quad-antennen en utmärkt DX-antenn. En
två-elements quad anses kunna ge ett resultat som en 3-elements Yagi-antenn.

NSYSTEM

Bild II 6-17 Riktbar dipol-antenn

----

~

/

Bild II 6-18 Yagi-antenner
116- 13

EPT

NSYSTE

1 - band Boom - Ouad

3- band Boom· Ouad

1 - band Spider - Ouad

3 - band Spider - Ouad

Bild II 6-19 Cubical Quad-antenner
För kortvågsbruk finns många antenntyper, såsom longwire-, zepp-, windom-, romb-, delta
loop-, quad laop-antenner etc. För mer information hänvisas till antennlitteratur.

116-14

ANTENNSYSTEM
Antenner för VHF/UHF/SHF

Allmänt
Alla antenner fungerar efter samma principer. Principerna för kortvågsantenner kan
därför tillämpas även för antenner för högre
frekvenser. Byggmåtten på en VHF/UHFantenn är betydligt mindre än för en motsvarande KV-antenn. JämförA= c:a2 m vid 145
MHz och A= c:a 80m vid 3.5 MHz. Det är
därför möjligt att bygga riktantenner med
rimliga dimensioner för VHF/UHF, även om
flera element används.
Om man bortser från rundstrålande vertikalantenner för trafik på korta avstånd och
mobil trafik, så används riktantenner främst
p.g.a. den större räckvidden. En riktantenns
egenskaper uttrycks i första hand i storheterna strålningsvinkel, antennvinst, fram/
backförhållande och halvvärdesbredd.
Eftersom polarisationsvridning sällan förekommer vid högre frekvenser, är det viktigt att sändar- och mattagarantenner har
samma polarisationsriktning.
Horisontell polarisation anses vara bättre
lämpad för långa distanser, eftersom vågor
med horisontell polarisation böjer av bättre
över horisontella formationer (bergryggar
etc). Även passage genom skogspartiergår
bättre med horisontellt polariserade vågor.
Antenner med horisontell polarisation används därför ofta för SSB- och CW-trafik på
långa avstånd och utmed markytan. Sådan
trafik sker i allmänhet från fasta stationer.
För mobil trafik och lokal fast trafik används dock antenner med vertikal polarisation. Vertikala antenner ger de önskvärda
rundstrålande egenskaperna för mobil trafik
och är bäst lämpade att montera på fordon.

Riktantenner

En A/2-antenn strålar vinkelrätt ut från
antennledaren och runt omkring den.
Placeras ett reflektorelement (längd ~Al
2 +5o/o) bakom antennen på ett avstånd av
::::: A/5 så reflekteras den bakåtriktade strålningen delvis framåt. En större del av energin kommer då att samlas i en riktning. Med
ett direktorelement (längd= A/2- 5o/o) framför det strålande elementet på ett avstånd
av~ A/1 O så kommer utstrålningsvinkeln att
bli mindre.

Vagi-antenner
Bild II 6-20
Den typ av riktantenn, som består av en
strålare, en passiv reflektor samt ett antal
passiva direktorer, kallas Yagi-antenn.
Vagi-antennen kan utföras med olika
antal direktorelement i kombination med
olika längd.
Det finns tre sätt att optimera en riktantenn, nämligen maximal riktverkan, minimum sidlober och maximalt fram/backförhållande. Dessa egenskaper är, emellertid ej möjliga att uppnå samtidigt. Okas t. ex.
antalet element, så ökar den s.k. antennvinsten genom att öppningsvinkeln på strålningen blir mindre, men samtidigt minskar
matningsimpedansen och den användbara
bandbredden.
Gruppantenner
Ordnas flera riktantenner vid sidan av och/
eller över varandra så erhålls en s.k. gruppantenn. Ett sådant arrangemang av s.k.
stackade antenner ger en ännu mindre
öppningsvinkel på strålningen vertikalt och/
eller horisontellt. Därigenom erhålls ytterligare antennvinst
Parabolantenner
Särskilt på frekvenser i mikrovågsområdet
och högre har radiovågorna i stort sett
samma utbredningsegenskaper som ljusets.
Behöver stor riktverkan uppnås på dessa
höga frekvenser, används ofta en parabolisk yta som spegel bakom själva antennen.
Jämför med reflektorn i en ficklampa.
Den egentliga antennen (den s.k. mataren), vars strålning är riktad mot parabolen
för att reflekteras, kan vara utformad på
många sätt. Eftersom parabolens storlek
står i omvänd proportion till frekvensen, så
används av praktiska skäl inte paraboliska
reflektorer på låga frekvenser.
Övriga antenntyper
Rundstrålande antenner: Ground plane,
A/4-,A/2-, 5A/8-antenner m.fl.
Riktantenner: Quad-, HB9CV-, helical-,
parabol- och hornantenner m fl.

116- 15

HORISONTALDIAGRAM

l

Di pol

-----::>

---r-·s

l--r-·

Yagiantenn

S = strårare
R == reffektor

:.::.::.:;;;;;>

s

R

D= direktor

:::::::::=;>

-Il-R <;

R

-ffi-

s

-«()-

o

l

D D

:=:::::;>

o

--~-

D

VERTIKALDtAGRAM

Di pol

--t- l

Dipol i

t A över jord

s
Vagi-

antenn

Yagi i

R

1'>

D

Bild II 6-20 Strålningsdiagram för horisontell Yagi-antenn

116-16

fÅ

över jord

ANTENNSYSTEM
Transmissionsledningar
En matarledning skall med så små förluster
som möjligt överföra den högfrekventa energin från sändaren fram till sändarantennen.
Omvänt skall den energi som fångats upp
av mottagarantennen transporteras till mottagaren med så små förluster som möjligt.

Bild II 6-22
Om matarledningen i stället är A/2 lång, så
uppstår i stället en spänningsnod och en
strömbuk i nedre änden av ledningen, vilket
innebär att matarledningen skall strömkopplas till sändaren.

Avstämd matarledning

Spännings- och strömkoppling
Bild II 6-21
En A,/2-dipol kopplas till sändarutgången via
en A,/4 matarledning. För tydlighetens skull
visas ledningen som en bandkabeL
Vid sändning uppstår en stående våg på
matarledningen och på dipolen. Även matarledningen svänger med och är avstämd
till resonans - därav namnet avstämd matarledning.
Vi följer ström- och spänningsfördelningen bakåt från dipolen till sändaren och finner
följande:
l vardera änden av A/2-dipolen uppträder en spänningsbuk (streckade linjer) och
i mitten av dipolen uppträder en strömbuk
(heldragna linjer). Den stående vågen, med
strömbuken på dipolens mitt, fortsätter ner
på A/4-matarledningen. l nedre änden av
matarledningen vid sändarutgången hardet
uppstått en strömnod och en spänningsbuk,
vilket innebär att matarledningen skall
spänningskopplas till sändaren.

r··-··------ . . ·2A

··-

··1

-.,.....:::""1~~--=::·~ ~--···---~--l

<llilliiioo.o........

. "if

Spänningskoppling

Bild If 6-21 Spänningskopplad Y2-dipol

Ä

2

n

l

Strömkoppling

Bild II 6-22 Strömkopplad }J2-dipol
Ström- och spänningsfördelningen kan
ritas upp för en A--di pol resp A/2-dipol i kombination med matarledningar med längderna
n · A-14 (med n = 1, 2, 3, ..... ). Med hjälp av
teckningen kan man avgöra om ström- eller
spänningskoppling måste användas.
Bild II 6-23
En A/2-dipol för 80 m-bandet ansluts till
en avstämd matarledning med längden A/2
=40 m.
Önskar man använda denna di pol för 80
m-bandet på 40-, 20- och 1O m-banden
måste en s.k. antennkopplare anslutas mellan sändaren och matarledningen. Kopplaren har alltid strömmatad ingång och valmöjlighet för ström- resp spänningsmatad
utgång. Se om antennkopplare sist i detta
kapitel.

116-17

ANTENNSYSTEM
f

=

f::::: 7 MHz

3,5 MHz

!--·-

--.....!

l

r·····

-~=40m

Å= 40m

2

n

Strömkoppling

J
Spänningskoppling

Bild II 6-23 Samma A/2-dipo/ på grundfrekvensen respektive 1:a övertonen
Oavstämd matarledning

Begreppet "oavstämd" syftar på ledningslängden, som under vissa bestämda förutsättningar kan vara godtyckligt lång. l motsats till den avstämda matarledningen behöver ledningslängden på en oavstämd matarledning inte stå i förhållande till våglängden A.. Som matarledning kan användas en
koaxialkabel eller öppen transmissionsledning.
Fördelar: Enkel uppbyggnad, mindre kritisk kabelfäring och längden kan väljas godtyckligt.
Nackdelar: Sändaren, matarledningen
och antennen måste alltid vara impedansanpassade till varandra. Dessutom måste
antenn- och kabelströmmarna balanseras. l
det följande visas hur dessa krav kan uppfyllas.
Som matarledning upp till mikrovågsområdet är koaxialkabeln vanligast.

116-18

Koaxialkabel

Bild II 6-24
Koaxialkabelns uppbyggnad framgår av bilden. l en koaxialkabel bildas ett radiellt
elektriskt fält mellan mittledaren och insidan
av ytterledaren. Av strömmen bildas också
ett magnetiskt koncentriskt fält mellan inner- och ytterledare n. Resultatet blir ett elektromagnetiskt fält, som breder ut sig i kabeln
som en TEM-våg (TE-våg = transversell
elektrisk, TM-våg = transversell magnetisk
och TEM-våg = transversell elektromagnetisk våg).
Koaxialkabeln består av en isolerad innerledare omgiven av en ytterledare, vars insida är kabelns andra strömledare. Ytterledaren förhindrar dessutom H F-utstrålning
och inkommande störningar. l motsats till
den symmetriskt uppbyggda bandkabeln,
tillhör koaxialkabeln de osymmetriska ledningarna.
Vanliga karakteristiska impedanser
koaxialkabel är 50 och 75 .Q.

ANTENNSYSTEM

skärm

isolerande hölje---·-·-····

En koaxialkabel har hastighetsfaktorn
11-{B, där c är den relativa dielektricitetskonstanten i isolationsskiktet Ett vanligt förekommande isolationsmaterial i koaxialkablar är polyetylen med dielektricitetskonstanten c = 2.25.
Hastighetsfaktorn v (velocity factor) blir
då

Bild II 6-24 Koaxialkabel

Bandkabel

Bild 6-25
Som framgår av bilden består bandkabeln
av två parallella ledare med samma dimensioner. Kabelns isolering håller samtidigt
ledaravståndet rätt. l ett kraftigare utförande
övergår denna ledningstyp till att bestå av
ett ledarpar med isolerade spridare på jämna
avstånd. Den kommer att likna en stege det ursprungliga utförandet på en matarledning.
Vanliga karakteristiska impedanser
bandkabel är 75 och 300 n.

Bild II 6-25 Bandkabel

Vågledare

Inom mikrovågsområdet är den vanligaste
typen av matarledning s.k. vågledare som
saknar mittledare. l en vågledare däremot,
matas energin fram enbart som speciella
elektriska och magnetiska fält (TEM) i möns ...
ter som kallas moder.

Hastighetsfaktor

Vid bestämning av den mekaniska längden
på en matarledning måste hänsyn tas till att
våghastigheten längs ledningen är lägre än
ljushastigheten. Man talarom en hastighetsfaktor relativt ljushastigheten. Hastighetsfaktorn beror på ledningens utförande och
ingående material.

1

1

1

V= -{B= ..j 2. 25 = tS = 0.666

1 meter av en sådan koaxialkabel är
1/0.666 =i .333 meter för en H F-signal.
Även bandkablar har naturligtvis en
hastighetsfaktor, vanligen 0.7- 0.85.

Karaktäristisk impedans Z i ledningar

Antag att en H F-sändare har kopplats till en
oändligt lång ledning. Om man undersöker
kvoten mellan spänning och ström på godtyckliga ställen utmed ledningen, så kommer man att finna samma kvot överallt.
Denna konstant uttrycks i ohm, om spänning och ström uttrycks i volt respektive
ampere. Konstanten kallas vågimpedans
eller karaktäristisk impedans.
Oändligt långa ledningar är ju orealistiska och då kan man i stället bestämma vågimpedansen genom ledningens geometriska uppbygg nad, dielektricitetskonstant
och dess induktivitet och kapacitet per längdenhet.
Exempel:
Vi undersöker elektriska karakteristika i
en kabel av typ RG 213/U.
På en provbit med längden 1 meter mäter vi en kapacitans av 97 p F mellan inneroch ytterledaren. När kabelns ena ände
kortsluts mäter vi en induktans av 262 nH.
Den uppmätta kapacitansen och induktansen bestämmer kabelns karaktäristiska
impedans Z, också kallat våg motstånd, som
är oberoende av ledningens längd.
Med ovanstående uppmätta värden blir
impedansen:

Z=\ 

z=

L [H]

262000 ·10-"
97 ·10-12

C [F]

Z [O]

=~ 262ooo =52
97

0

116-19

ANTENNSYSTE
Den karaktäristiska impedansen för en
matarledning, bestäms av dimensionerna i
ledningen och av isolationsmaterialets dielektricitetskonstant.
För en bandkabel är

Z= 276 ·log 2a

Fr

[Q]

d

[a = centrumavståndet mellan ledarna i
mm]
[d = ledardiametern i mm]
[er= dielektricitetskonstanten, överslags
värde 1.5]
[er för luft = 1.0]
För en koaxialkabel är
138
D
Z=-·log[Q]

Fr

d

[D = ytterledarens innerdiameter i mm]
[d = innerledarens ytterdiameter i mm]

Punkterna för maxima och minima beror
av belastning relativt vågresistansen och av
frekvensen.
stäende vågor uppträder inte bara i
antennkablar utan även i fasta material (trådar o. dyl.), i luft (ljud), i ljus (t.ex. laser), i
elektromagnetiska fält o.s.v.
Bild II 6-26
Spänningen utmed kabeln varierar regelbundet mellan
u max = uf + ub och umin = uf - ub
ståendevågförhållande (SVF)
(även SWR = standing Wave Ratio).
Med ståendevågförhållandet SVF menas
förhållandet mellan
umax och umin eller mellan !max och 'min

SVF = Umax
umin

Data, impedansdiagram och formler för
beräkning av transmissionsledningar finns
bl.a. i antennhandböcker.

eller

SVF=

=

U,+ Ub

u,-ub

/max
/min

Stående vågor
Både när sändarens och matarledningens
anslutningsimpedans är olika liksom när
matarledningens och antennens anslutningsimpedans är olika, så uppstår s.k. missanpassning som hindrar energitransporten.
Antag att matarkabelns och antennens
anslutningsimpedans är olika. En del av H Fenergin kommer då att strålas ut från antennen, men resten reflekteras tillbaka i matarledningen. På kabeln finns alltså en framåtgående våg mot antennen och samtidigt en
reflekterad våg tillbaka mot sändaren.
Den spänning och ström som man då
kan mäta var som helst på kabeln, är den
algebraiska summan av amplituden hos den
framåtgående och den reflekterade vågen.
Flyttar vi mätpunkten stegvis utmed kabeln, så kommer spänningen och strömmen
att stiga och sjunka på ett regelbundet sätt.
Den tillbakagående vågens spänning Ub
och den framåtgående vågens spänning Ut
överlagras på varandra. Kvoten för ström
och spänning är därmed inte konstant utmed matarledningen, utan får ett vågformat
förlopp- en stående våg.
116-20

ståendevågförhållandet SVF kan även
anges med hjälp av impedanserna i matarledningen (Z) och i antennens matningspunkt (Za)·

SVF = !

där Z > Za

~a

där Za > Z.

za

SVF =

eller

ståendevågmätning beskrivs i Kapitel 8.
Bild II 6-27
Bilden visar SVF-problemet enkelt sett
och vad en SVF-meter visar beroende på
var den kopplas in i kedjan sändare-ledning-antennkopplare-ledning-antenn.
Vid ett högre SVF-tal än 2:1 till 3:1 vid
sändarutgången, bör en antennkopplaresättas in efter sändaren för att skydda den från
(överhettning och) överslag. Antennkopplare har även andra benämningart.ex. matchbox, antennavstämningsenhet o.s.v. Bäst
är att göra sådana impedansanpassningar i
alla led, att antennkopplaren blir onödig.

NSYSTEM

Bild II 6-26 Ståendevåg på ledning

Pilarna visar HF-energins riktning
Sändare
Utgångson

SVFmeter

Koaxialkabel 50Q

Antenn
Koaxialkabel 50Q

Ingång SOn

SVF=1

HF-energin från sändaren

reflekteras delvis

men sänd s återigen till antennen
Sändare

Antenn

SVFmeter

Utgång
-:t-SOn

Ingång
-:t-SOn

Sändaren är efterinställd,
SVF
utgången är inte längre 50 Q,
>1
den inkommande reflekterade HFenergin sänds tillbaka till antennen
. f ron san d aren
HF-energrn
o

..

Sändare,\

9VIS7r

re fl e kteras d l.

SVFmeter

Utgång SOQ

Sändaren är inte efterinstätld,
den reflekterade HF-energin
stannar i sändarens slutsteg
och värmer det ytterligare

/

Antenn
Ingång
-:t-SOn

SVF>1

Antenn
Ingång
valfri

SVF=1

~

SVF>1

Lsönder den reflekterade energin
från antennen tillbaka igen

Bild II 6-27 SVF-problemet enkelt sett

efter DJ3XV. cq-DL 5/1978

116- 21

ANTENNSYSTEM
Effektförluster
l varje matarledning uppstår förluster, dels
av resistansen i ledarna och dels i isolationsmaterialet (dielektrikum) mellan ledarna samt i någon mån av fältutstrålning från
dem. De mest påtagliga effektförlusterna i
en ledning beror av förlusterna per längdenhet och därmed även av längden. Vidare
beror förlusterna av ståendevågförhållandet på ledningen på grund av dålig impedansanpassning.
Ett högt SVF-förhållande ger större ledningsförluster eftersom den reflekterade effekten då pendlar fler gånger på ledningen.
Den reflekterade effekt som återvänder till
ledningens inända är mindre när ledningen
har stora förluster än om den inte hade det.
Det innebär att det verkliga SVF-förhållandet i ledningens utände är större än vad som
syns på ett instrument i inänden.
Bal u ner- Balansering- Transformering
Balansering
Man skiljer mellan symmetriska ledningar
(bandkabel m.fl.) och osymmetriska (koaxialkabel), där dessutom den ena ledaren (skärmen) ofta är jordad.
På samma sätt finns det symmetriska
antenner (dipol, W3DZZ m .fl.) och osymmetriska (ground plane, Marconi m.fl.).
Vill man ansluta en symmetrisk (mittmatad) antenn till en osymmetrisk ledning
(koaxialkabel), så måste en strömbalansering göras i övergången. Om inte, så kommer matarledningen att stråla, vilket kan
medföra störningar på radio och TV. Utan
balansering kommer dessutom dipolens
strålningsbild inte att vara symmetrisk.
En balansering måste också göras i övergången mellan en bandkabel (symmetrisk)
och sändaren när den har anslutning för
koaxialkabel (osymmetrisk),
Balansering av impedans och därmed
ström sker med en anordning kallad BAL UN
(av de engelska orden BALanced-UNbalanced).
Bal uner kan utföras på flera sätt. Grundläggande har balunen lika in- och utgångsimpedans,
Exempel:
Ringkärnebalun 1 :1 för balansering.
Koaxialledare anordnad som balun 1:1.

116-22

Transformering
l samband med balanseringen kan en
impedanstransformering behövas och det
finns baluner (transformatorer) som både
balanserar och transformerar impedanser.
Bild II 6-28
Bilden visar en transformator med osymmetrisk ingång och symmetrisk utgång. Om
båda lindningarnas varvtal är lika så sker
ingen impedanstransformering. Om förhållandet mellan varvtalen är 1 :2 så blir förhållandet mellan impedanserna 1 :4. Se vidare
i Kapitel1.
Bilden visar också att matarledningens
impedans Z transformeras om så att den blir
lika antennens anslutningsimpedans Ra.
Denna transformering kan ske induktivt eller kapacitivt.
Exempel:
Ringkärnebalun 1 :4.
Koaxialledare anordnad som bal un 1:4.

BALANSERING

TRANSFORMERING

0118Ra
gz
~~: !Ra
Qz

i::

!Ra

Bild II 6-28 Balansering - Transtorrnerin

ANTENNSYSTE
Ringkärnebalun

Bild II 6-29
Ringkärnebalunen är en form av transformator. l den finns en ringkärna av hårt
sammanpressat järnpulver av en legering,
som tillsammans med lindningarnas utförande gör att frekvensbandbredden blir stor.
"1 = varvtal, primär
n2 = varvtal, sekundär

l den mellersta figuren är den översta
delen av matningskabeln en A./4 lång parallellsvängningskrets tillsammans med parallellt ansluten ledare (i detta fall en koaxialkabel som kortslutits i båda ändar). Den
nedrersta högra figuren i bilden visar den
kortslutna A./4-ledningen i tre varianter. l
samtliga fall uppstår HF-mässigt en strömbalancerande effekt mellan dipolhalvorna.
Dessutom hindras även antennströmmar från att komma ner på utsidan av matningskabelns skärm.

Osymmetrisk

ingång

Bild II 6-29 Ringkärnebalun

Koaxialledare som balun
Bild II 6-30
Balansering kan även göras med ett ett
koaxialkabelarrangemang, som i så fall är
starkt frekvensberoende. Bilden visar tre
utföranden, som alla arbetar enligt principen för en matarledning med en elektrisk
längd av A./4 och kortsluten i ena änden.
Den mekaniska längden är k· A/4, varvid
k är hastighetsfaktorn för våghastigheten i
kabeln. För de vanligaste koaxialkablarna
RG58 och RG213 är k= ca 0.66.
A./4-ledningen i den översta figuren fungerar som en parallellsvängningskrets med
mycket hög impedans Z i den öppna övre
änden.

----- --- - -·· . .

1·'

.A.
i ·--

Bild II 6-30 Koaxialledare som balun

116-23

ANTE N
- anpassning

)(

Delta - anpassning

Il
/il,r

!""------

anpassning

2Å

--------4

l= le~

l

i

Bild II 6-31 Sätt att ansluta en matningsledning

Sätt au ansluta en matningsledning

Bild II 6-31
T-, delta- och gamma-anpassning
Funtion: En mittmatad halvvågsdipol har
i fria rymden en impedans av c:a 73 n.
Flyttas matningspunkten bort från mitten, åt det ena eller andra hållet, så är
impedansen högre än i mitten.
Det finns alltid två symmetriskt liggande
punkter på antennen där impedansen är
precis lika stor.
T-, delta- och gamma-anpassning är användbar när matarkabelns
är högre än antennens mittpunktsimpedans. Matningsledningen kan anslutas till de punkter
på antennen som har samma impedans
som matarledningen. T-anpassning används för symmetriska matarledningar, gamma-anpassning för osymmetriska ledningar
och delta-anpassning för båda ledningstyperna.

A/4 -anpassningsledning - stub
Uppbyggnad: Antennen ansluts till en ')J
2 anpassningsledning och matarledningen
i sin tur till anpassningsledningen.
Funktion: Anpassningsledningen består
av en öppen 'A/4-matarledning. Den har
teoretiskt impedansen Z= Oi den ände som
är ansluten till antennen och Z = = i den
116-24

-Anpassningsledning

andra. Utmed anpassningsledningen finns
alltid en impedans som är lika matarledningens impedans.

},./2-fasningsledning
Bild II 6-32
Funktion: När t.ex. en omvikt dipol med
matningsimpedansen 240 n skall anslutas
till en 50 n-kabel, behövs en impedanstransformering med förhållandet 4:1. En 'A/2
lång fasningsledning kan användas fördetta
ändamål. Fasningsledningen har dessutom
en strömbalanserande verkan.
Observera: Med en 'A/2-fasningsledning
enligt bilden kan impedanstransformering
endast göras i förhållandet 4:1.

i- stinga
Bild II 6-32 )./2-fasningsledning

ANTENNSYSTEM
Transmissionsledningen
En transmissionsledning för radiofrekvent energi består av två elektriska ledare.
Den enklaste formen av en sådan ledning är
tvåparallella ledare. En annan form av transmissionsledning är koaxialkabeln, där den
ena ledaren löper inuti den andra.
Försök: Koppla en parallelledning till utgången på en VHF-sändare - t.ex. med
induktiv koppling. Ge ledningen passande
längd och mata ut högfrekvent energi på
ledningen. Nu kan fördelningen mellan spänning och ström på olika punkter utmed ledningen undersökas. När det finns en spänning mellan de två ledarna i ledningen alstras det ett elektriskt fält mellan dem.
Eftersom en glimlampa lyser när den
omges av ett elektriskt fält kan den användas som en enkel spänningsindikator.
När en elektriskt ledande krets - en induktionsslinga - omges av ett varierande
magnetiskt fält alstras det en ström i slingan.
Med en glödlampa inkopplad i slingan kan
den användas som en enkel strömindikator.
Öppen transmissionsledning
Bild II 6-33
Håll glimlampan nära intill ledningen.
Glimlampan tänds med jämna mellanrum
när den flyttas utmed ledningen.
När i stället en induktionsslinga med glödlampa hålls nära intill ledningen, kommer
glödlampan att lysa mitt emellan de ställen
där glimlampan lyser. Där glimlampan tänder har det bildats spänningsmaximum och
där glödlampan lyser har det bildats strömmaximum. Det har bildats en stående våg
på ledningen.
Bilden visar ström- och spänningsfördelningen för en öppen transmissionsledning med längden i = n · IJ4 med udda n =
1' 3, 5, ......
För bilden har valts n = 5.
Utmed ledningen uppstår omväxlande
elektriska och magnetiska fält allt efter som
svängningen fortsätter. Med en serie om
fyra figurer visas förloppet av en svängning,
en period. skillnaderna i den elektriska fältstyrkan framställs som olika långa fältlinjer.
Observera fältlinjernas riktning.

Skillnaderna i den magnetiska fältstyrkan
kan också utläsas ur bilderna i form av
antalet symboler "
resp " x ". Båda
tecknen betecknar elektromagnetiskt fält, "
i riktning ut ur papperet och " x " in i
papperet. För tydlighetens skull skildras
endast den elektromagnetiska fältstyrkan
mellan ledarna och inte utanför ledarparet
G

G

"

"

Kortsluten transmissionsledning
Bild II 6-34
På bilden visas såväl ström- och spänningsförhållandena som fältlinjeförloppen
på en avstämd, kortsluten transmissionsledning med längden l = IJ4 med jämna n =
2, 4, 6, 8, ... För bilden har valts n = 6.

A./4-ledning som svängningskrets
Bild II 6-35
Bilden visar ström- och spänningsfördelningen för en öppen resp. en kortsluten
transmissionsledning med längden l= IJ4.
Den öppna A./4-ledningen har en strömbuk i ingångsänden. En sådan ledning måste
således strömkopplas, d.v.s. den anslutande impedansen måste vara låg.
Den kortslutna A./4-ledningen har en
spänningsbuk i ingångsänden. En sådan
ledning måste spänningskopplas, d.v.s. den
anslutande impedansen måste vara hög.
En öppen A/4-ledning kan ses som en
seriekopplad LC-krets. När ledningen är i
resonans flyter en hög ström i ingången,
medan spänningen där är låg.
En kortsluten A./4-ledning kan ses som
en parallellkopplad LC-krets. När ledningen
är i resonans är spänningen hög över ingången, medan strömmen där är låg.

116-25

UPPBYGGNAD OCH INKOPPLING

SPANNINGS- OCH STRÖMFÖRDELNING

(l :::5· t A)

+++
Spänningsförlopp
(visat med glimlampa)

-ffi-

Strömförlopp
(visat med glödlampa
och slinqa

c®=J
FÖRLOPPEN FÖR DE ELEKTRISKA OCH MAGNETISKA FAL TEN

t=

o

' Elektriska fältlinjer

Magnetiska fältlinjer

Elektriska fältlinjer

Magnetiska fältlinjer

Bild II 6-33 Förlopp i öppen :1/4 transmissionsledning

116-26

UPPBYGGNAD OCH INKOPPLING

SPÄNNINGS- OCH STRÖMFÖRDELNING

( ( ::::

6.

t A)
Spänningsförlopp
{visat med glimlampa)

Strömförlopp
(visat med glödlampa
och slinga)

c®=l

FÖRLOPPEN FÖR DE ELEKTRISKA OCH MAGNETISKA FÄLTEN

+++

+++
Elektriska fältlinjer

Magnetiska fältlinjer

Elektriska fältlinjer

+++

+++

Magnetiska fältlinjer

Bild II 6-34 Förlopp i kortsluten A/4 transmissionsledning

116-27

ANTENNSYSTEM
STRÖM- OCH SPÄNNINGSFÖRDELNING

iT

a - öppen ledning

R=O

R::c:lO

'::---:::;:::::::>
1
u--

'X:;{

u
(l)

+---+----- f

fr

(l)xx
u

b- kortsluten ledning

+---1-----

FRÄN SERIESVÄNGNINGSKRETS Tl LL ÖPPEN

I:I
T

J

N'

t-

LEDNING

:::L,

::::=::::>
o

lfY'!

i

FRÄN PARALlEllSVÄNGNINGSKRETS Till KORTSlUTEN

1.

+

f·- LEDNING

...,  .Ä,

T

Bild II 6-35 JJ4 transmissionsledning som svängningskrets

116-28

J

--...-.j

f

NSYSTEM
Antennkopplare
Bild II 6-36
Bilden visar en antennkopplare för bandkabel av olika längder. storleken på kondensatorerna: C1 = C2 = 500 pF, C3 = 300pF
Avstämning vid spänningskoppling
c1 och c2 helt invridna eller kortslutna,
C 3 avstäms för resonanstillstånd
(parallellresonans).
Avstämning vid strömkoppling
helt utvriden,
C 1 och C2 avstäms för resonanstillstånd
(serieresonans), med maximal och likaström
i båda ledarna.
Matarledningen kan förlängas elektriskt
med induktanser när den är för kort för att
kunna avstämmas.

c3

Märk, att en antennkopplare mycket väl
även kan utformas för koaxialkabelutgång.

Antennkopplare för en
parallell-ledning med längden

l : : n·

:A.

4

För- och nackdelar med avstämd matarledning
När en matarledning är rätt avstämd
transporterar den energi utan att stråla själv.
När dipolen kopplas till en avstämd matarledning, kan den med hjälp av en antennkopplare arbeta på flera amatörradioband. Detta är en anledning till varför en
avstämd matarledning gärna används för
portabla installationer (t.ex. för field days).
lnjusteringen mot sändaren blir enklare.
Inom amatörradion används numera nästan uteslutande koaxialkabel som matarledning i st.f. bandkabeL Detta är av flera
skäl:
• En bandkabel måste hängas upp så fritt
som möjligt och den får inte komma för nära
murutsprång, takrännor o.s.v. Vidare måste
den isoleras väl vid genomföringar i väggar.
• De flesta såndaramatörer har inte plats
med långa matarledningar (n· /../4 med n=
1' 2, 3, ... ).
• Vid tvära bockar på ledningen kan det upp
stå oönskad utstrålning och därmed risk för
störningar på radio och TV m. m.

AAA.A.J

Q
,......,

Antennkopplare för en
parallell-ledning med längden

l.

* n ·-f:

Bild II 6-36 Antennkopplare

116-29

ANTENNSYSTEM

116-30

~©~

EPT

\chapter{VÅGUTBREDNING}

l
Elektromagnetisk vågutbredning är energitransport och förutsättningen för all radiokommunikation. Radiovågornas utbredning
på vägen mellan sändare och mottagare
påverkas emellertid på många sätt. Med
vetskap om radiovågornas utbredningssätt
kan man mer metodiskt försöka uppnå önskade radioförbindelser.

\section{Kraftfälten omkring antenner}

För att sända ut och ta emot radiovågor
behövs antenner. Mycket förenklat är en
antenn en elektrisk krets, som består av en
induktor och en kondensator.
Med kondensatorns elektroder helt isärdragna och förminskade har svängningskretsen fått ett mycket annorlunda mekaniskt utseende. Sedan induktorn i LC-kretsen tagits bort, så återstår mekaniskt sett
endast en enkel ledare, men elektriskt sett
finns kretsen ändå kvar. Ledaren med sin
utsträckning är fortfarande en induktor och
ytorna på dess motstående halvor är fortfarande elektroderna i kondensatorn med omgivningen som dielektrikum.
En elektrisk ledare, en stång, tråd etc. är
alltså en elektrisk svängningskrets, vars resonansfrekvens mest bestäms av längden
och tjockleken. Ledaren (antennen) kan kallas dipol- den har två poler. Se Bild II 7-1.
Vissa likheter finns mellan en mekanisk
pendel och en elektriskt svängningskrets.

Mekanisk pendel
Energin i en mekanisk pendel växlar mellan
två ytterlighetstillstånd. Det ena är när pendeln just vänder i ett ytterläge. Då innehåller
den mest lägesenergi och ingen rörelseenergi. När pendeln rör sig mot mittläget, så
omvandlas lägesenergin till rörelseenergi. l
mittläget, som är det andra ytterlighetstillståndet, innehåller pendeln mest rörelseenergi och ingen lägesenergi etc ..
Elektrisk svängningskrets
Den elektriska svängningskretsen kan sägas motsvara den mekaniska pendeln i den
meningen att det i båda fallen pågår en
ständig pendling mellan lägesenergi och rörelseenergi. Se Bild II 7-2
När strömmen i den elektriska svängningskretsen just upphört för att vända, så
innehåller kondensatorn mestladdning, d. v .s.
det starkaste elektriska fältet mellan elektroderna. Detta fält är lägesenergi. Den utjämningsström som följer, från den ena elektroden över till den andra, omges av ett
magnetiskt fält. Detta fält är rörelseenergi.
Förloppet visas i Bild II 7-2, där det framgår att dipolen omges av det starkaste elektriska fältet vid tidpunkten t=O samt vid t=1/
2T med omvänd polaritet, där T är periodtiden. Vidare att dipolen omges av det starkaste magnetiska fältet vid tidpunkten t=1/
4T samt vid t=3/4T med omvänd strömriktning och fältpolaritet

+

~~+

LTSluten
resonanskrets

Di pol

Bild II 7-1 Från sluten LC-krets till antenn

117- 1

VÅ

-

Max E- fält
Min H- fält

Min E- fält
MaxH-fält

Min E- fält
MaxH-fält

Bild II 7-2 Pendlingen mellan E-fält och H-fält
Med förklaringen av E- och H-fälten som
bakgrund följer nu en enkel framställning av
hur radiovågor uppstår ur dessa fält.
Maxwell påvisade i sina ekvationer bl. a.
sambandet mellan elektroner i rörelse i en
ledare och elektromagnetiska vågor i rummet. Vidare, att elektroner som rör sig med
avtagande eller tilltagande hastighet avger
elektromagnetisk energi.
Hur energi strålar från en ledare kan förklaras med en (tänkt) elementär dipol, som
genomflyts av växelström (Bild II 7-3).

-(])Bild II 7-3 Elementär dipol
Di polen består av två lika stora elektriska
laddningar med motsatt polaritet. När
len matas med en växelström, så rör sig
laddningarna ständigt, omväxlande emot
respektive ifrån varandra. Tänk på två kulor
i var sin ände av en spiralfjäder. Avståndet
mellan laddningarna ändras i takt med styrkan och riktningen på strömmen. Systemet
är alltså under ständig hastighetsändring
(ökning resp. minskning), vilket är förutsättningen för att energi skall strålas ut.
Först är laddningarna nära varandra på
grund av liten laddning. Vid ökande ström
ökar avståndet mellan laddningarna och det

117-2

byggs upp ett mer utbrett och energirikt Efält. Samtidigt byggs även ett H-fält upp
omkring dipolen, vinkelrätt mot E-fälteto.s.v ..
Detta gäller både för en elementär di pol och
en elektrisk ledare med många fria elektroner (verklig antenn).
Formeln för det resulterande s-fältet är
E. vilket visar att den lagrade energin
i di polens närmaste omgivning ökar när avståndet (potentialen) mellan dipolens laddningar ökar.
Bild II 7-4 visar hur ett E-fält byggs upp
omkring en dipol och avskiljs från den. De
visade kraftlinjerna är E- fältet. H-fältet visas
inte, men ligger vinkelrätt mot E-fältet, i cirklar omkring antennen. Se Bild II 7-5.
När dipolens laddningar ändrar riktning
och åter börjar att röra sig emot varandra,
börjar det E-fält som byggts upp att också
byta riktning. Men det kommer inte att falla
tillbaka till dipolens mitt utan sluts till ett eget
kmt!=:lcmn- Maxwells första ekvation. Jämför
med en såpbubbla som lämnat blåsröret.
Omkring dipolen har det nu bildats ett självständigt E-fält, som sin tur alstrar ett eget Hfält.
En period av en elektromagnetisk våg
(ett S-fält) har alstrats och fortsätter att utFörvarje följande period alstras ett
som separeras från antennen och
bildarett
H-fälto.s.v .. Varjegångbildas
alltså en ny "fältbubbla" inne i den föregående, vilken håller på att utvidgas.
Resultatet är ett elektromagnetiskt fält,
d.v.s. en radiovåg.

s=

VÅ UTBREDNING

Bild II 7-4 Ett självständigt E-fält skapas

Bild II 7-5 E-, H- och B-fälten omkring en antenn (förenklad framställning)
Som nämts består en radiovåg av ett högfrekvent elektromagnetiskt fält (S). Det är i
sin tur sammansatt av två andra fält, det
elektriska E- och det magnetiska H-fältet.
Energin i S-fältet fördelas lika mellan E-fältet
och H-fälten, vars krafter korsar varandra
vinkelrätt. S-fältet ligger i plan med både Eoch H-fälten och breder ut sig vinkelrätt mot
dem. s-fältets riktning beror av den inbördes
riktningen på E- och H-fälten.
När E-fältet är vertikalt, sägs vågen vara
vertikalt polariserad. När samma fält är horisontellt sägs vågen vara horisontellt polariserad. När E-fältet roterar i vågfrontens
plan, och därmed även H-fältet, sägs vågen
vara cirkulärt polariserad.

Fälten framställs i text och bild som s.k.
kraftlinjer med pilar som föreställer kraftriktningen. Linjernas längd föreställer fältets
styrka. Bild II 7-6 visar ett avsnitt av en
vågfront S med vertikal polarisation.
E

.E.

Bild II 7-6 E-, H- och S-fält

117-3


VÅGUTBREDNING
Radiovågornas egenskaper

Ett elektromagnetiskt fält, som alstras i ett
givet tidsmoment, breder ut sig åt alla håll i
rymden likt en ständigt växande sfär.
Fältstyrkan inom ett givet avsnitt av sfärens yta sjunker därför alltefter som avståndet från sändaren ökar. Det är därför som en
sändare hörs svagare ju mera avlägsen den
är ifrån mottagaren. Jämför med ljuset från
en rundstrålande lampa.
l rymden breder radiovågor ut sig mycket
långt. Det uppstår dockäven där utbredningsförluster i materia som finns i vägen.
När radiovågorna passerar genom jordatmosfärens olika skikt uppstår mycket större
utbredningsförluster än i rymden och därmed blir räckvidden kortare.
Elektromagnetiska fält från alla slags sändare (emittörer) genomkorsar alla slags material och alstrar strömmar i dem som är
elektriskt ledande.
Radiovågorna
• breder ut sig rätlinjigt i alla riktningari rymden med ljusets hastighet som är ca
300 000 km/s (se även 6.1 ),
e
tränger igenom fasta kroppar, som inte är
elektriskt ledande,
• dämpas eller reflekteras, bl.a. av metaller, joniserade vätskor och joniserade
atmosfärskikt,
e
är polariserade,
e
förstärker eller motverkar varandra.
Radiovågorna breder ut sig
• utmed jordytan,
• upp från jordytan,
e
upp från jordytan efter en första reflexion
mot denna.
Det första sättet kallas för markvåg och
de två senare kallas med ett samlingsbegrepp för rymdvåg.
Radiovågornas riktning kan böjas av genom
• reflexion eller splittring mot naturliga
reflektorer i atmosfären och i jordytan,
e
konstgjorda såväl passiva som aktiva
reflektorer (relästationer) på jordytan och
i rymden.
Radiovågorna kan dämpas
• i jordytan,
• i topografin,
e
i atmosfärsskikten.

117-4

~@~

EPT

Vågutbredningens natur är mycket sammansatt och kan inte enkelt beskrivas. Några starkt påverkande faktorer på vågutbredningen kan ändå urskiljas, t.ex.
• utbredningsvägens höjd över jordytan,
• radiovågens frekvens,
• solstrålningens jonisering av jordatmosfären,
• väderförhållandena..

Olika slags vågavböjning

Olika faktorer påverkar vågutbredningen
inom olika avsnitt i frekvensspektrum. Här
följer de viktigaste:

Reflexion
Reflexion innebär att vågorna böjs tillbaka
från den yta som de träffar. Ljus- och radiovågor reflekteras på samma villkor eftersom
att båda är elektromagnetiska till sin natur.
Den stora skillnaden är vågfrekvensen.
Reflektorns storlek uttrycks i termer av
antal våglängder vid den aktuella frekvensen. En 80-metersvåg reflekteras inte bra
mot en yta med bara någon meters sida.
Däremot reflekteras en 2-metersvåg mycket bättre mot en lika stor yta och en ljusvåg
7.7 J.lm) ojämförligt
(med våglängden 4
mycket bättre.
Olika materials förmåga att reflektera en
infallande radiovåg beror av vågens frekvens samt av materialets tjocklek och elektriska ledningsförmåga. Vågen tränger djupare in i materialet vid låg frekvens respektive vid låg ledningsförmåga.

a

Refraktion
Refraktion (brytning) innebär att vågen ändrar riktning, när den passerar gränsen mellan två media eller material med olika ledningsförmåga. När ledningsförmågan ändras successivt t.ex. i ett atmosfärskikt, blir
vågens avböjning mjuk.
Diffraktion
Diffraktion innebär att vågens infallsriktning
splittras upp i flera nya riktningar, närvågen
passerar nära över ett hinder. Det är p.g.a.
detta fenomen som radiosignaler i viss mån
kan höras även bortom en berg rygg. Diffraktionen tilltar med minskande frekvens.

VÅ UTBREDNING
Jonosfärskikten

Jonosfären har fått namnet från begreppet
jon, som är en fri elektron eller annan laddad
partikel. Jonisering- elektrisk uppladdningav jordatmosfären sker mellan c:a 40 till400
km över jordytan. Där är lufttrycket tillräckligt
lågt för att joner skall kunna röra sig fritt
under avsevärd tid utan att kollidera och
återförena sig till neutrala atomer.
När en radiovåg passerar genom ett
joniseratskikt i atmosfären, kanvågen ändra
riktning, vilket kallas för refraktion. För att
refraktion skall uppstå måste i första hand
två villkor uppfyllas, det är tillräckligt tät jonisering och tillräckligt lång våglängd. Under
"gynnsamma" omständigheter kan vågorna
till och med böjas av ner mot jorden, vilket är
den viktigaste förutsättningen för långväga
radioförbindelser på kortvåg.
Joniseringen av atmosfären är emellertid
oregelbunden och varierar bl. a. med höjden
över jordytan, solinstrålning, tidpunkt m.m.
Ett antal joniserade skikt kan definieras.
Se Bild II 7-7.

D-skiktet
D-skiktet förekommer under den ljusa delen
av dygnet på en höjd av c:a 50-90 km. På 7090 km höjd orsakas joniseringen huvudsakligen av röntgenstrålar från solen, medan
den kosmiska strålningen har störst påverkan på 50-70 km höjd. D-skiktet dämpar de
infallande radiovågorna, med största verkan
i kortvågsområdets lågfrekventa del och under de ljusaste timmarna under sommaren.
D-skiktet har dålig reflexionsförmåga och
verkar hindrande på långdistansförbindelser.

E-skiktet
E-skiktet (Kenelly-Heaviside-skiktet) är det
lägsta reflekterande jonosfärskiktet Det förekommer på en höjd av c:a 90-140 km och
är mest koncentrerat på c:a 11 Okm höjd. Eskiktet alstras av att ultraviolett strålning
joniserar syreatomer. Skiktet reflekterar vågor bäst i kortvågsområdets lågfrekventa del
och är kraftigast under den ljusa delen av
dygnet. På grund av D-skiktets dämpande
verkan under de ljusaste timmarna är Eskiktet mest användbart under grynings- och
skymningstimmarna.
Ett säsongmaximum i reflexionsförmågan inträffar under sommaren. Förbindelseavstånd på upp till 2000 km är möjliga.
Mögel-Dellinger-effekten
Strålning från gasutbrott på solytan kan jonisera D-skiktet så kraftigt, att alla radiovågor
med frekvenser över c:a 1 MHz dämpas helt.
Radiotrafik som baseras på vågutbredning
via jonosfären är då omöjlig att genomföra
under en tidsrymd av ett antal minuter upp till
flera timmar- det blir "black out".
Sporadiska E-skiktet
Den starkare solinstrålningen under sommaren orsakar en kraftigare jonisering i den
lägre jonosfären än under vintern. Inom Eskiktet bildas då sporadiska tunna molnlika
partier med mycket hög joniseringsgrad och
stor reflexionsförmåga, det s.k. sporadiska
E-skiktet (E 8 ). Vågutbredningen via Es är
mycket olika på olika latituder och är bäst
omkring 40:e breddgraden. Mycket goda
långväga förbindelser kan uppnås.

F2 -skikt

km

Avståndet jordyta - skikt icke måttriktigt avbildat

Bild II 7-7 Jonosfärskikten

117-5

VA

UTB EDNIN

F-skiktet
F-skiktet är det högst liggande jonosfärskiktet
Det förekommer såväl dag- som nattetid på
en höjd av 140-500 km. Den nedre del av
skiktet, 140-200 km, uppvisar andra variationer än den övre delen. F-skiktet beskrivs
därför som två skikt, F1 upp till ca 200 km
höjd och F2 över denna höjd.
Liksom E-skiktet, påverkas F1-skiktetkraftigt av instrålningen från solen. Det når sin
högsta joniseringsgrad ungefär en timme
efter högsta lokala solstånd och förekommer
endast under sommaren. Under natten förenar sig F1- och F2-skikten till ett enda Fskikt.
F2 -skiktet är det skikt som varierar mest i
tiden och rummet. Den högsta joniseringsgraden inträffar vanligen sent efter högsta
lokala solstånd, ibland under aftontimmarna.
Skiktets maximala jonisering är på 250-350
km höjd på mellanlatituder och på 350-500
km höjd vid ekvatorn. På mellanlatituder
ligger den största elektrontätheten i skiktet
högre under natten än under dagen. Vid
ekvatorn är förhållandet omvänt.
Reflexioner i F2 -skiktet möjliggör att stora
avstånd kan överbryggas (1 hopp = 30004000 km). Bild 117-8 omjonosfärsutbredning.

PT
Höjd till reflekterande skikt
När en radiovåg, som riktas rakt uppåt, träffar jonosfären kan den antingen
• absorberas, sugas upp,
• reflekteras,
e
tränga igenom.
Vilket som inträffar beror på den använda
frekvensen. Ju högre frekvensen är på den
uppåtriktade radiovågen, desto högre upp i
ett atmosfärskikt kommer avböjningen tillbaka att inträffa. Höjden till skiktet beräknas
ur radiovågens utbredningshastighet och
utbredningstid fram och åter mellan skiktet
och jordytan.
Kritisk frekvens
Vid en viss övre frekvens upphör reflexionen
i atmosfärskiktet och vågen går ut i rymden
i stället för ner till jordytan. Denna frekvens
kallas den kritiska frekvensen, som varierar
med joniseringsgraden i atmosfären. Den
kritiska frekvensen är högst vid högt solfläckstal, såväl i E- som i F-skikten, eftersom
joniseringsgraden då är störst. Den kritiska
frekvensen för E-skiktet varierar mellan c:a
1-4 MHz beroende på tidpunkt i solfläckscykeln och tid på dagen. Den kritiska frekvensen för F-skiktet varierar med tid på

Frankfurt/Main - Osaka, 9 000 km, beamriktning: NO, 3 hopp

Frankfurt/Mai n

Uralbergen

LANGA OCH KORTA VÄGEN

Avståndet jordyta - F -skiktet är icke
2
mättriktigt avbildat

Bild II 7-8 Jonosfärsutbredning

117-6

Bajkalsjön

Osa ka

EDNIN
dagen, årstid och skede i solfläckscykeln.
Den kan variera från 2-3 MHz
natten
3 MHz
under ett solfläcksminimum till 1
på dagen under ett solfläcksmaxi mu m.

Kritisk vinkel
Rymdvågen måste träffa ett joniserat atmosfärskikt med en tillräckligt flack vinkel för att
reflekteras, den s.k. kritiska vinkeln. Denna
vinkel är frekvensberoende. Allt eftersom
den utsända frekvensen ökas ytterligare över
den kritiska frekvensen, måste vågen träffa
atmosfärskiktet i en allt flackare vinkel för att
vågen skall reflekteras mot jordytan.
att sända ut vågen i mycket flack vinkel mot
F2 -skiktet kan långa distanser överbryggas
den
vid frekvenser som är upp till 3.5
kritiska frekvensen.
Så snart den kritiska frekvensen är högre
än frekvensområdet för ett amatörband är
det alltså möjligt att kommunicera över
våg på detta band. Det kan ske över alla
avstånd, allt ifrån skipavståndet till det som
avgörs av utbredningsförlusterna.
Högsta användbara frekvens (MUF)
Radiovågorna vandrar från sändaren till en
avlägsen mottagare genom att reflekteras
en eller flera gånger i jonosfären och jordytan. För detta kan frekvensen inte vara
högre än den högsta användbara frekvensen, Maximum Usab/e Frequency- MUFför en viss överföringssträcka.
MUF är högst mitt på dagen eller
eftermiddag. Allra högst är MUF under perioder av högt solfläckstal och kan då komma
upp till över 30 MHz. Under tidiga morgontimmar sjunker MUF ofta under 5 MHz.
De janesfäriska förlusterna är lägst nära
MUF och ökar snabbt under dagtid för lägre
frekvenser.
Aktuella MU F-data publiceras periodiskt
i olika media, men kan också överslagsberäknas med hjälp av speciella
gram.
Optimal trafikfrekvens (FOT)
l praktiken är det av intresse att veta det
frekvensområde där kommunikation bäst kan
genomföras.
Rekommenderad övre frekvensgräns för
en tillförlitlig radioförbindelse kallas optimal

traffic frequency -FOT- och väljs något under MUF som marginal för oregelbundenheter och turbulens i jonosfären, liksom för
korttidsavvikelser från det förutsagda månatliga medianvärdet för MUF. FOT är vaniigen ungefär 15 $\circ$/o lägre än MUF.
Lägsta användbara frekvens (LUF)
lägre sändningsfrekvens som väljs, desto
mer dämpas vågorna i jonosfären, intill den
frekvens då de inte kan uppfattas. Den lägsta användbara rekvensen Lowest Usab/e
Frequency- LUF- är den frekvens som ger
tillfredsställande kommunikation för en viss
utbredningsväg och vid en viss tidpunkt.
Vid frekvenser under LUF är mottagning
inte möjlig eftersom brusnivån då är för hög.
Ju mer frekvensen höjs över LUF, desto
bättre blir signal-brus-förhållandet.
Till skillnad från MUF, som endast påverkas av de janesfäriska förhållandena, kan
till en del påverkas genom utsänd effekt
och bandbredd. Generellt kan LUFsänkas
c:a 2 MHz för varje 1O dB ökning av E. R. P.
Vågutbredningsförutsägelser
Det görs regelmässiga förutsägelser av de
janesfäriska förhållandena. Fortlöpande fysiska observationer, statistisk och matematisk bearbetning ligger till grund för förutsägelserna, vilka bl.a. utnyttjas för att planera
radiotrafiken. Vågutbredningsförutsägelser
(propagation forecasts) görs av både civila
och militära institutioner och upplyser om de
lämpligaste frekvenserna och tiderna för olika förbindelsesträckor. Sådana förutsägelser meddelas i offentliga publikationer, men
även i andra, t.ex. tidskrifter och bulletiner
inom amatörradion.
Regional Warning Center (RWC) samlar
sol- och geofysiska data och sänder dagligen Ursigram per telex eller brev ( URSI =
Union Radio-Scientifique lnternationale).
Ursigram kan erhållas genom årsabonnemang (tyvärr till högt pris). De innehåller
aktuella mätvärden såsom solfläckstal R, i O
cm solflux F, magnetiskt index K, gränsdämpningsvärden, även anvisningar om särskilda händelser (flares,
magnetstormar, polarkalottabsorbtion, Mögei-Dellinger-effekter och liknande) liksom
korttidsprognoser och förvarningar.

117-7

:::i
l

co

<

~

et

):>o

:::::::::

)J

c:o

ffiJ

~

.g
a

cg

~
Ct
"'""\
tu

g
'"""i-

a:
~
ru

~

g.
~

~

"'O

!l>o

~

n3o

(Q

Radioprognos Juni 1997 SSN
Tid/

/GMr

1.8 MHz
000011111222
246802468024

3.5 MHz
000011111222
246802468024

SH
9H

l .. : ... llo2.

lo.: .... 1.11

41.: ••. o2222

621o ... o2244

~4

E: L
!!'
!!'G
JA
KH6
KH6-L

LU
OA
OD
py
T2

trAl
trA9

... : .... :o1o

... : .... : .o1

42o:12112426
•.. : ..•. :oo.

531oo2123546
. .. : .... :o ..

IfK
IfK-L
llU

l lo: .. 1232o.
... :o.o.l.ol
11.: .•. 1oool
••• : ••.. : •. o
o .. : .. 11: .. o
. o.: ... oo1o.

.lllo .. 11. ..
... :.o .. : ...

10 MHz
000011111222
246802468024
.o.: .... oll.
542211113455
2o.: ... o1123
3o.: .... :ool
334544554553
111: .... : .o1
•.• : .... llo.

1..: ... o1122
lolo .... :ool
.0000 ••• : •••

... : .... ooo.

ooo: .... : .. o
•.. : .... ll.o
•.. : .•. ooo .•

.o.: .... :

••••••••••••

•.• : ...• : .••

455455555554
453333445554
443212444554
442lo2334454

.00

. .o: .... : .. .

1\.ntarkt-W ••• : •••• : .••
!Ultarkt-E ... : •••. : •.•

SM 250
SM 500
SM 750
SM 1000

18 MHz
000011111222
246802468024
. .. : ... 11 ...
.0233212452.
.olll. .12o ..
•.. 11111:1..
.o222111221o

oo.: ...• : .. o

:rn

zs

14 MHz
000011111222
246802468024
.lo: .. olll ..
134o .. 355642
o21o.112331.
l .. : .... : ...
. .. o ... o121o
764221224456
222222222322
o .. : ..•. : .. o
••. :o.o.o11o
o .• :oolloo .•
oloooo .. l. .•
.o.: .... o .. .
o .. : .... ooll . . . : ... ooloo .lo: .. o.: .. .
o .. : .••. : ... llo: .... : .o1 .. lo .... : .11
21.: ••• oo123 522oo.112325 2o4llo.l2242
oo.: ••.. : .. o 11.: ..•. : .ol 1.. :oo .. ool2
. . . . . . . . . . . . . .. : .... oo.. .00000 . . oo ..
664333345667 12356665354321 122221122211
o .• : •••. 1221 loo:.ol22222 .111oo1llo .•
••. : ••.. oo .• ••• : .•. 111o.
••• : •••• :.00

W'2
W'6
lm

ZL
ZL-L

=6

7 MHz
000011111222
246802468024
... : .... :.o.
42l: •.. oo234
o .. : ..•. :ool

oo.: .... :.ol

... : .... :oo.
445455554554
454343445554
554332345664
543221234554

o11222112221
223333333432
234565554543
345566665654

ol.: .••• :ool
00.: . . . . :ooo
0000.00. :ooo
.. olo ... o1o.
122222222222
232332222332

. .. : ... o: .. .
0000000.0001
••. : ...• : •. o
•• oo .... olo.
112221112221

21 MHz
000011111222
246802468024

24 MHz
000011111222
246802468024

28 MHz
000011111222
246802468024

. . . : .0 . .

--1

m
m

:0

. .. 110 .. 22 ..

c

• .. oo ... 12 ..
.. l111o111o.

-z

• .• :1 .•• : ...

ooloo ... lo. o
• .. : .... :oo .
.o11ooo.: ...

c:

... : .... o ...
•• 0000 . . : • . .

:11.

. 2344224532.
... :l.o.ollo

.o2221o222 ..
... : .... :1 ..

.o121o.111 ..
• .. 33 ... : .. .

.. oo .... o ..•

... oo ... oo ..

.o.o .... : .. .

... : .... :1 ..

.oooo ... : ...

... : .... :.o .
.. ooooo1: .. .
. .. : ... oo .. .
llooooooooll
oo.: .•.. : .oo

... : .... :.o.

11ooooooool1

oo.: .... : .00

11oooooooo11
00.: .... : .00

1llooo1oo111
00.: .... : .oo

... : .... :o ..

Tabellen visar sannolikheten att få förbindelse för alla amatörband på kortvåg (1.8-28 Mhz) och varannan timme (02-24} GMT. Sannolikheten
anges i procent. "9" betyder 90-100 %, "8" 80-89 %, ... , "2" 20-29 %, "1" 10-19% och "o" 5-9%. Mindre an 5% markeras med"."(":" för timmarna
08 och 18). Vidare förklarina finns i QTC nr 1 1995 samt notis i QTC nr 4 1995. /SM510. Stio

~

©

VÅ UTBR DNIN
Bild II 7-9 visar en radioprognos ur SSAs
medlemstidning QTC. Ny prognos presenteras periodiskt och behandlar kortvågsspektrum. ObseNera det låga solfläckstalet SSN
(Sun Spot Number) på denna bild.
Sträckan SM - UA 1 på 1O MHz Juni -97
Klockan
Kod
Sannolikhet (0/o)
02
2
20-29
04
3
30-39
06
5
50-59
08
6
60-69
10
6
60-69
12
6
60-69
14
5
50-59
16
3
30-39
18
5
50-59
20
4
40-49
22
3
30-39
24
2
20-29

Bild II 7-1 O Detalj av radioprognos i Il 7-9

Solens inverkan på jonosfären
solaktivitet
Solen är ett gasklot, i vars inre pågår en

ständig kärnreaktion där väteatomer omvandlas till helium. Vid denna process frigörs en del av solmaterian som partikelstrålning och elektromagnetisk strålning inom ett
brett frekvensregister, bl.a. kortvågig radiostrålning, gammastrålning. solatmosfärens
yttre består av två skikt, kromasfären och
koronan. Vissa områden på solens yta har
en lägre temperatur och uppfattas som mörka fläckar - solfläckar. Från kromasfären
kastas det ut gasmassor, s.k. protuberanser, ofta från områden nära solfläckarna.
Det förekommer även kortvariga eruptioner, s.k. flares, som syns som lysande fläckar i närheten av solfläckarna. Fiares sänder
~t stark elektromagnetisk strålning och partiklar. Koronan är solatmosfärens yttersta
skikt. Från denna utstrålas partiklar i form av
atomer, elektroner och protoner, som fångas upp av jordens magnetfält och skapar
polarsken, s.k. aurora. Den ökade partikelstrålningen från fiares kan orsaka magnetiska oväder med åtföljande radiostörningar
och ökning av polarskenet. Antalet synliga
solfläckar står i samband med solaktiviteten.

Solfläckstal
Ett mått på solaktiviteten är antalet solfläckar, vilket det görs fortlöpande obseNationer på. Ur detta statistikmaterial beräknas
ett vägt solfläckstal R (Wolf-talet). Med stöd
av solobservationer under mer än 200 år har
det kunnat fastställas att solfläckstalet varierar någorlunda periodiskt mellan ungefär
200 och 5.
En solfläcksperiod varar mellan c:a 7.5
och 17 år, med ett medelvärde av c:a 11 år
-den s.k. 11-årscykeln. Vid utgången av år
1996 noterades ett så lågt solfläckstal som
5, vilket innebar slutet på cykel 22.
När cykel 23 nu börjar betyder det bättre
möjlighetertill DX på kortvåg under några år.
På senare tid har ännu en metod börjat
användas för mätning av solaktiviteten. Då
mäts styrkan av radiobruset från solen (solflux F) i våglängdsområdet 1O cm.
De båda mätmetoderna ger i huvudsak
samma tendenser och det finns ett statistiskt
samband mellan dem.
Vågutbredningen i jonosfären påverkas
av solaktiviteten. Under solfläcksmaximum
blir jonosfären starkt joniserad, speciellt Fskiktet under dagtid. Då reflekteras även
vågor med kortare våglängder mot jonosfären i stället för att passera igenom denna ut
i rymden. 20-metersbandet är då "öppet"
nästan dygnet runt, 15-metersbandet från
före gryningen till efter solnedgången och
10-metersbandet nästan varje dag till efter
solnedgången. Långa förbindelser med
mycket låga effekter är möjliga.
Under solfläcksminimum är det emellertid nödvändigt att använda avsevärt lägre
arbetsfrekvens än vid solfläcksmaximum.
20-metersbandet Jörblir t. ex. inte öppet under hela natten. Oppningar på 15-metersbandet uppstår endast tillfälligtvis och öppningar på 1O- metersbandet är sällsynta.
Goda antenner och högre effekter används
då för att i någon mån kompensera den
sämre vågutbredningen. Vid låg solaktivitet
kan de högre banden vara så tysta, att operatören kan undra om utrustningen verkligen
fungerar.

117-9

VÅGUTB
Vågutbredning på kortvåg
Markvåg
Markvågen breder ut sig längs jordytan utan
kontakt med atmosfären genom reflexion
eller refraktion.
Markvågen har vertikal polarisering och
en vertikal vågfront när jordplanets ledningsförmåga är god. Vid sämre ledningsförmåga
lutar vågfronten framåt.
Markvågens räckvidd står i forhållande
till den använda frekvensen, sändareffekten
och jordplanets ledningsförmåga.
Vid frekvenser under c:a 1O MHz är jordytan är en tämligen god ledare. Markvågsutbredning utnyttjas därför mest vid låga
frekvenser, t.ex. för rundradio i lång- och
mellanvågsbanden då räckvidden kan vara
i storleksordningen 1000 km. På kortvåg är
markvågsräckvidden i 80 m-bandet c:a 100
km och i i O m-bandet c:a 15 km.

Rymdvåg
Under vissa förutsättningar reflekteras radiovågorna mot joniserade atmosfärsskikt
och når åter jordytan på stort avstånd från
utsändningspunkten. Rymdvågsutbredning
utnyttjas mellan platser på jordytan med stort
avstånd.
För att bäst uppnå den önskade reflexionen måste man dels välja lämplig tidpunkt
och frekvens och dels utforma antennen så
att den har sin huvudriktning i en bestämd
vinkel mot det reflekterande skiktet .
Jonosfären är den del av atmosfären på
c:a 50 till 350 km höjd, där instrålningen från
solen skapar fria elektroner och joner i en
sådan mängd att det bildas skikt med god
elektrisk ledningsförmåga. Under vissa villkor reflekterar dessa skikt radiovågorna, men
kan under andra villkor även absorbera dem.

Sändningsfrekvens <övre gränsfrekvens

markvåg

Sändningsfrekvens

Bild II 7-11 Vågutbredning på kortvåg

117-10

>

övre gränsfrekvens

VÅ UTBREDNIN
När vågorna från jordytan reflekterats
mot de joniserade skikten, kan de återträffa
jordytan på ett avstånd av upp till 4000 km
från utsändningspunkten, beroende på frekvens och polarisering. Därefter kan de åter
reflekteras mot jordytan och upp i jonosfären o.s.v. (flerstegshopp). Under gynnsamma förhållanden når rymdvågen mycket
långt genom växelvisa reflexioner mellan
jordytan och jonosfären.
Död zon (skip zone) och skip-avstånd
Rymdvågorna böjs tillbaka mot jorden när
de träffar jonosfären i en vinkel som är
flackare än den s.k. kritiska vinkeln. När
vågorna träffar jonosfären med en brantare
vinkel än den kritiska vinkeln sker det ingen
avböjning utan vågorna passerar genom
jonosfären och rakt ut i rymden. Beroende
på den kritiska vinkeln för tillfället, kommer
därför reflekterade rymdvågor inte att höras
förrän på ett visst avstånd bort från sändaren. Detta avstånd kallas för skip-avstånd.
Men sändarens markvåg har också ett
visst täckningsområde och mellan detta och
zonen där rymdvågen kan höras bildar en
skymningszon-en skip zon e eller död zon.
Grålinjeutbredning -gra y Iine
Med gray Iine menas det smala bälte på
jordytan där det för tillfället råder gryning
eller skymning.
Tidintervallet för gray Iine varierar med
stationens latitud. Vid ekvatorn är det$\pm$ 5
minuter och i Skandinavien $\pm$ c:a 1 1/2
timme omkring tidpunkten för solens uppgång respektive nedgång.
När åtminstone en av två stationer befinner sig inom gray Iine kan kortvågsförbindelse erhållas över ett mycket större
avstånd än annars.
Kommunikation längs med gray Iine går
bäst på låga frekvenser, t.ex. på 3.5 MHz
amatörband, under det tidsintervall då Dskiktet just har börjat byggas upp (gryning)
respektive nästan har brutits ned (skymning). Då är joniseringen av D-skiktet liten
och en rymdvåg som träffar skiktet kommer
då snarare att böjas av i D-skiktet än att helt
dämpas. Vågutbredningen sker då både
genom refraktion i D-skiktet och reflexion i
E-skiktet.

Fädning eller signalbortfall
Fältstyrkan på de mottagna vågorna kan
variera kraftigt från ett ögonblick till ett annat
Fenomenet kallas fädning (e ng. fading, uttalas fejding).
Sådana interferensfenomen uppstår när
vågorna samtidigt vandrat flera vägar fram
till mottagarantennen, s.k. flervägsutbredning. När de träffar mottagarantennen kan
de vara tidsförskjutna sinsemellan, med utsläckningseffekter som följd (interferensförluster).
Andra typer av fädning är när
• polari-seringriktningen ändras p.g.a. oregelbundenheter i jonosfären (polariseringsförluster),
• överföringsvägen dämpar vågorna tidsmässigt oregelbundet (absorbtionsförluster),
• vågutbredningsriktningen ändras genom
reflexioner mot hus, bergväggar etc.
(reflexionsförluster, vid t.ex. mobil radiotrafik).

Om amatörradiobanden på kortvåg
1.8 MHz (160m):
Bandet kallas även "top-band": Räckvidden
är normalt relativt liten, nattetid undervintern
c:a 1200 km och i bästa fall några tusen km.
Men under solfläcksminimum kan räckvidden vara mycket större nattetid.
3.5 MHz (80 m):
Under dagtid är räckvidden ca 500 km och
under kvällstid 1000-1500 km. Tidigt på
morgonen under vintermånaderna, särskilt
under solfläcksminimum, är räckvidden tillräcklig för interkontinentala förbindelser (DX
= Iong distance). Under sommarmånaderna
har bandet hög atmosfärisk brus nivå. Döda
zoner förekommer normalt inte.

7 MHz (40 m):
Detta band har större räckvidd än 80 mbandet. Under dagtid har det en räckvidd av
1000-2000 km. Under natten, särskilt under
vintern, kan hela världen nås. Döda zoner är
100 km under dagen och 1000 km under
natten.

117- 11

vA
Vågutbredning på VHF, UHF, SHF

14 MHz (20m):
20m-bandet är ett säkert DX-band för stora
avstånd. Under kvällarna ökar räckvidden
på ett rymdvågshopp upp till ca 4000 km.
Särskilt gynnsam vågutbredning erhålls vid
kontakt genom en skymningszon, dvs där
den ena parten har dag och den andra har
natt. Döda zoner uppträder nästan alltid.

ocn EHF

21 MHz (15 m):
Vågutbredningen i 15 m-bandet är bäst vid
högt solfläckstaL Under solfläcksmaximum
är bandet nästan ständigt öppet för DXförbindelser.
Under solfläcksminimum är bandet i bästa
fall öppet kortare perioder på dagtid under
sommarmånaderna.
Bandet är dött nattetid. Vid reflexioner via
sporadiskt E-skikt kan avstånd av mer än
2000 km överbryggas.

På VHF och högre frekvenser (tidigare UKV)
förekommer sällan någon vågutbredning via
jonosfären annat än under tiden för maximal
sol aktivitet. l stället utnyttjas den lägre delen
av atmosfären och knappast högre än 4 5
km över jordytan. Denna del av atmosfären
kallas för troposfär och vågutbredningen
därför för troposfärisk vågutbredning.
All vågutbredning i troposfären förutsätter i princip optisk sikt. Emellertid förekommer en viss vågavböjning utmed jordytan,
varför den praktiska räckvidden utmed siktlinjen är något längre än till den optiska
horisonten. Man talar om radiohorisont.
Brytningsindex i atmosfären är en viktig
faktor för vågutbredning bortom frisiktsavståndet, speciellt vid frekvenser över 100
MHz. Även den splittring av vågorna som
uppstår när de träffar oregelbundenheter i
atmosfären kan utnyttjas för kommunikation
på avstånd som är flera gången frisiktsavståndet
Vid högre frekvenser begränsas emellertid räckvidden av atmosfärens dämpande
inverkan. likaså förloras vågenergi i den
topografi, vegetation och bebyggelse som
ligger i siktlinjen mellan sändare och mottagare. l gynnsamma fall är det dock möjligt att
överbrygga avstånd på upp till 1000 km
genom troposfären. Sådana avstånd kallas
för överräckvidd.

28 MHz (1 O m):
Bandet är lämpat för närkontakter upp till 50
km nattetid och för DX-kontakter dagtid,
dock ej dagar då E-skiktet är kraftigt joniserat
och skärmar av F-skiktet. Vågutbredningsvägen för DX är på den sida av jorden som
har dagsljus. Döda zoner på upp till4000 km
kan uppstå. Förbindelser över stora avstånd
är möjliga med låg effekt.
Under solfläcksminimum är bandet inte
användbart för DX-kontakter. Då är endast
kortvariga förbindelser på avstånd upp till
2000 km möjliga genom reflexioner via sporadiska E-skikt (short skip).
Bandet har i många fall VHF-karaktäroch
man kan ha kontakter via Aurora och andra
liknande utbredningsformer såsom AuroraE och dubbelt hopp på Auroraringen.
10, 18 och 24 MHz:
Vågutbredningsegenskaperna i dessa senast tillkomna amatörrradioband är ett mellanting av respektive närmast angränsande
amatörradioband.

117-12

Allmänt
Frekvensområdet 30-300000 MHz delas upp
i följande mindre avsnitt som kallas
VHF (Very High Frequency, 30-300 MHz),
UHF (Ultra High Frequency, 300-3000 MHz),
SHF (Super High Frequency, 3-30 GHz) och
EHF (Extra High Frequency, 30-300 GHz).

a

Troposfären - Troposcatter
När en kalluftfront nära jordytan stöter samman med en varmluftfront uppstårturbulenser
i luften med elektriska uppladdningar i gränsskiktet som följd.
Under sådana väderförhållanden kan
radiovågor i VHF-området och däröver att
brytas eller splittras upp när de träffar det
laddade gränsskiktet- troposcatter. Då kan
oväntade radiokontakter uppnås.

N
Temperaturinversion

När ett varmt luftskikt lägger sig över ett
kallare luftskikt uppstår en s.k temperaturinversion.
Vågor på VHF och UHF bryts då mot
gränsskiktet och böjs av mot jordytan. Om
det finns två inversionsskikt samtidigt, så
kan de bilda en slags vågledare, s.k. dukt
(eng. duct = ledning). En räckvidd på 6001300 km kan uppnås. Denna typ av vågutbredning förekommer ofta vid högt atmosfärstryck under sommaren.

Reflexion mot Es (sporadiskt E)

Joniseringen sker när partiklarna passerar genom E-skiktet och brinner upp. Eftersom joniseringen har en varaktighet av endast 0.1- i O sekunder måste MS-förbindelser planeras och förberedas väl. Förbindelserna begränsas vanligen till utbyte av
anropssignaler och signalrapporter med höghastighetstelegrafi med en hastighet av 3003000 tecken per minut. Under de större meteorskurarna kan kontakter uppnås utan överenskommelser på förhand (skeds), både på
telegrafi (CW) och telefoni (SSB).

EME-förbindelser

Vid stark solinstrålning bildas, på de lägre
breddgraderna, joniserade gasmoln på en
höjd av c:a 120 km och med en oregelbunden fördelning.
Den kritiska frekvensen är hög för Esskiktet och det kan även reflektera vågor på
VHF och UHF så effektivt att avstånd av
1000-4000 km kan överbryggas.

Radioförbindelse från en punkt på jorden till
en annan kan åstadkommas genom reflexion av VHF/UHF-signaler mot månen. EMEförbindelser (Earth-Moon-Earth) kallas även
Moon Bounce. EME-förbindelser kräver antenner med mycket hög riktverkan, mycket
hög sändareffekt och känsliga mottagare.

Aurora-reflexion

På VHF och högre frekvenser kan man, som
tidigare beskrivits, endast uppnå radiokontakter hitom den s.k. radiohorisonten.
För att överbrygga detta hinder används
relästationer. Den slags relästation, som allmänt kallas repeater, tar emot det den hör på
en viss fast frekvens och återutsänder detta
på en viss annan fast frekvens. Se frekvensplan i Appendix H.

Markbaserade relästationer

soleruptioner-fiares-utstrålar stora mängder ultraviolett ljus och kastar ut elektriskt
laddade partiklar, som efter 1-2 dagarfångas
upp av jordens m agnetasfär och tränger ner
i polarzonerna. När partiklarna kolliderar med
atmosfären bildas det polarsken i form av
lysande "draperier"- Aurora (kallat norrsken
på norra halvklotet) - samtidigt som atmosfären joniseras. Auroran är joniserade skikt
i samma plan som jordens magnetfält och
speciellt vågor med frekvenser över 30 MHz
reflekteras emot dessa.
VHF- och UHF-kommunikation kan ske
med hjälp av aurorareflexion. De signaler
som reflekteras av Aurora är kraftigt distorderade och har förlorat all ton. Den reflekterade signalen blir bred i frekvens, vilket emellertid gynnar kommunikation med telegrafi
när signalerna är svaga. Oftast är endast
telegrafiförbindelser i långsam takt möjliga.
Vid starkare Aurora går också SSB att använda.

Reflexion mot meteorer- Meteorscatter

Radiovågor på VHF och UHF reflekteras
mot joniserade spår efter det meteorgrus
som faller in i jordatmosfären. Detta fenomen kan utnyttjas för radioförbindelser.

Rymdsateflit-baserade relästationer

Radiovågor med tillräckligt hög frekvens kan
passera genom jonosfärskikten. Detta möjliggör radioförbindelser
VHF/UHF/SHF
mellan stationer på jorden med hjälp av relästationer i rymdsatelliter.
För amatörradiotrafik över rymdsatelliter
används vanligen den slags relästation, som
kallas transponder. En sådan tar emot allt
det den hör inom ett helt frekvensband och
återutsänder detta i ett helt annat frekvensband. På så sätt kan trafik över satellit ske på
ett jämförbart sätt som vid direktkontakt mellan jordbaserade stationer.
satellitbaserade lineartranspondrar med
amatörradioutrustning finns i OsCARsatelliterna (OSCAR = Orbiting Satellite
Carrying Amateur Radio). Dessa har konstruerats och byggts av amatörradiogrupper.

117-13

VÅ UTBREDNIN

E

OSCAR-satelliterna har många olika transpondrar i funktion, vilka var och en arbetar
med olika kombinationer av sändningsslag
(moder) och frekvensband. Detta kallas numera att de har olika konfiguration.
En vanlig konfiguration av transponder
ärCONFIG-V/U (f.d. MOD-J) därupplänken
är på VHF-bandet, t ex. 145.900-146.000
MHz och nerlänken på UHF-bandet t. ex.
435.800-435.900 MHz. Varje upplänk-frekvens motsvarar en bestämd nerlänk-frekvens, t. ex. upp 145.950 och ner 435.850
MHz. Trafiken övertranspondern kan därför
ske i full duplex.

FM- TRAFIK

PA

Man kan då prata och lyssa samtidigt i
båda riktningarna, vilket starkt förbättrar trafiken och gör samtalen roligare och intressantare.
En s.k. linjär transponder kan inte bara
överföra FM, utan även SSB, tontelegrafi
och SSTV. Dessutom även RTTY och andra
digitala trafiksätt
Nästan alla amatörradioband med tillräckligt hög frekvens används i olika kombinationer som upp- och nerlänkar i de olika
OSCAR-satelliterna.
AMSAT är den organisation, som fortlöpande informerar om amatörradiosatelliter.

"2 METER" (144-146 MHz) Exempel: R 1

Mobilstation A

Mobilstation B

Bild II 7-12 Markbaserad repeater

OSCAR-satellit

Markstation 1

Bild II 7-13 Transponder i rymdsatellit

117-14

Markstat i on 2

DNING
Amatörradion utvecklas mycket snabbt
genom den satellitbaserade verksamheten
och det kommer upp allt mer sofistikerade
OSCAR-satelliter. Tendensen är att man
efter hand går över till allt högre frekvensband och allt mera av digitala sändningsslag.
Med hjälp av satellit kan förbindelseavståndet bli mycket stort även med enkel
utrustning och små antenner. En fördel med
kommunikation över rymdsatellit är också
att den till största delen är oberoende av
vågutbredningsvillkoren.
Se Bild il 7-13

117-15

VA

117- 16

UTBR DNIN

E

\chapter{MÄTNING}

I forskning, utveckling och produktion är mätning en hörnpelare i verksamheten. Även
inom mättekniken sker en snabb utveckling och digitaltekniken kommer alltmer till
användning, men grunderna för mätning är desamma. I detta kapitel behandlas de viktigaste
mättekniska begreppen som radioamatörer kan behöva känna till.

\section{Att mäta}

Mäta likspänning
Vid spänningsmätning bestämmer man potentialskillnaden -spänningen - mellan två
punkter. Om det finns en spänning, så flyter
en motsvarande (mät)ström genom instrumentetinstrumentet presenterar mätströmmen som spänning.
Mätströmmen påverkar emellertid spänningsfördelningen i kretsen och då uppstår
ett mätfel, vilket inte framgår av det visade
mätvärdet. Med kännedom om kretsens och
instrumentets data kan man dock beräkna
mätfelet En voltmeter skall ha hög inre
resistans för att mätfelet skall bli litet.
Endast vid mycket noggrann mätning
kan man behöva räkna om det visade mätvärdet med hänsyn till voltmeterns inre resistans och förkopplingsresistansen - om
en sådan används).
På grund av den höga inre resistansen är
en voltmeter endast lämpad för spänningsmätning -INTE för direkt strömmätning f
Utöka mätområdet för en voltmeter
Bild II 3-1
Med hjälp av förkopplingsresistor i serie
med voltmetern kan man mäta högre spänning än den som voltmetern är gjord för.
Spänningen fördelas då proportionellt mellan förkopplingsresistorns resistans och instrumentets inre resistans.
När förkopplingsresistor används måste
mätvärdet räknas om med en skalfaktor eller
en skala med motsvarande gradering användas. En voltmeter med valbar förkopplingsresistor kan därför har flera skalor. l
digitala voltmetrar anpassas "skalan" oftast
automatiskt.

Mäta likström
Vid strömmätning bestämmer man strömstyrkan i en gren av en elektriskt strömkrets.
Amperemetern skall kopplas i serie med den
aktuella strömgrenen. Det visade mätvärdet
motsvarar strömstyrkan. Amperemeterns
inre resistans adderas emellertid till resistansen i strömgrenen och då uppstår ett
mätfeL En amperemeter skall ha låg inre
resistans för att mätfelet skall bli litet.
Endast vid mycket noggrann mätning kan
man behöva räkna om det visade mätvärdet
med hänsyn till amperemeterns inre resistans och resistansen i strömshunten-om en
sådan används.
På grund av den låga inre resistansen
skall en amperemeter ALDRIG användas för
spänningsmätning. Då förstörs den!
Utöka mätområdet för en amperemeter
Bild Ii 3-2
Med en strömshunt (en resister parallellt)
över amperemetern kan man mäta högre
ström än den som amperemetern är gjord
för.
Shunten dimensioneras så att större delen av strömmen leds förbi amperemetern.
Kvar är den mätström som behövs för att
amperemetern skall göra fullt utslag.
Mätströmmen fördelar sig omvänt proportionellt till instrumentets och shuntens
resistanser.
När en strömshunt används måste mätvärdet räknas om med en skalfaktor eller en
skala med motsvarande gradering användas. En amperemeter med valbar shuntresister kan därför har flera skalor. l digitala
amperemetrar anpassas "skalan" oftast automatiskt.

Mäta växelspänning och växelström
Grunderna för mätning av växelspänning
och växelström är samma som för likspänning och likström, men att bl.a. en instrumentlikriktare oftast behövs.
Beroende på frekvensen i strömkretsen
och vilket slags värde man vill mäta, används olika instrument.

118-1

ÄTNIN
Olika typer av instrument ger olika
ter, men också begränsningar.
Mjukjärnsinstrument utan ii!rl"i!r+.-,,..,.... kan
mäta växelströmmar ner till c:a 50 mA och
upp till c:a1 O A. Frekvensen får dock inte
vara högre än c:a 100 Hz.
Vridspoleinstrument används dels direkt
för likströmsmätning och dels med likriktare
även för växelströmsmätning.
Vridspoleinstrument med likriktare används ofta för frekvenser upp till c:a i O kHz
och strömmar nertill 0.1 mA. Noggrannheten
är sällan bättre än 1.5$\circ$/o av fullt utslag.
Beroende på funktionsprincipen kan det
skilja på hur instrument mäter, vilket nödvändigtvis inte är detsamma som hur mätvärdet
presenteras.
Mjukjärnsinstrument mäter effektiwärdet
av en växelström medan ett vridspoleinstrument med likriktare mäter likriktade medelvärdet. Som exempel kan skalan i ett instrument med likriktare även graderas för effektiwärdet för sinusformade förlopp.
För mätning av växelström används vanligen instrument med likriktare, men för HF
även instrument med termokors, vilka bygger på termogalvanisk spänning mellan metaller.

Frekvensens inverkan
Frekvensen på den mätta signalen inverkar
mer eller mindre på mätresultatet. Till en del
beror det på den instrumenttyp, som används. En faktor är instrumentets gränsfrekvens, d.v.s. hur högt i frekvens som instrumentet fortfarande är rimligt rättvisande. Detta
kallas instrumentets bandbredd, vilken bör
vara dokumenterad.
Vågformens inverkan
Även formen på den signal som mäts inverkar på mätresultatet och det är viktigt att veta
för vilken vågform som instrumentet presenterar mätvärdet. Det vanligaste är att vågen
förutsätts vara sinusformad, vilket ofta inte är
fallet i praktiken. Det innebär att fel värde
presenteras om vågformen är en annan än
den förutsatta.

Mäta resistans

Mätning av resistans är enklast att göra på en
fristående komponent, medan man vid mät-

118-2

ning
en resister i en strömkrets också
måste ta hänsyn till att andra komponenter i
kretsen kan påverka mätresultatet.
Resistans kan mätas på flera sätt. Det
grundläggande är att mäta strömmen genom resistorn och spänningen över den och
sedan beräkna resistansen med Ohms lag.
Därutöverfinns direktvisande instrument
för resistansmätning- s.k. ohm-metrar Sådana instrument innehållervanligen en egen
strömkälla i form av ett batteri.
1-tt,::.trttnl"'n"!lal"' vid likväxelström
(medel-, effektiv- och toppvärden)

Vid likström:

P= U ·l [W] (watt),
d.v.s. Joules lag

Vid sinusformad växelström:
medelvärde

p
 O.B·Ue/
medelR

effektiwärde

p

toppvärde

P,

Såndareffekt

eff

= ueff2
R

umax2
PEP=~

En sändares effekt kan mätas på olika sätt.
Förr, då radioamatören hade små möjligheter att mäta uteffekt, så var det naturligt att
föreskrifterna angav en mätmetod baserad
på ineffekt, vilket var enkelt och rättvisande
förtelegrafi m.fl. sändningsslag med bärvåg.
Även för SSB fick effektmätning göras så,
trots att resultatet var långt ifrån rättvisande.
Numera är instrument som mäter uteffekt
mer tillgängliga för radioamatören - även
toppvärdeskännande sådana för mätning av
p.e.p. Mot den bakgrunden anges nu (i 997)
i föreskrifterna såndareffekten som uteffekt.
Därvid måste även p.e.p. avses, fastän det
inte uttryckligen uttalas.
För N-licensen föreskrivs fortfarande att
uteffekten uttrycks i e.r.p. N-amatören kan
dock endast i undantagsfall mäta e.r.p. utan
den måste beräknas. Bakgrunden till detta
unika myndighetsbeslut var att se som en
anpassning till gällande regler för begränsning av radiostrålning.
Observera, att radioamatören måste beakta EMC-Iagen. Se vidare kapitel 9.

TNING
Metoder för mätning av sändareffekt

Tidigare har avhandlats effektberäkning i
allmänhet. Här nedan kommenteras mätning av såndareffekt i synnerhet.
Ett tillförlitligt sätt att mäta såndareffekt är
att ansluta sändaren till en konstlast med
samma resistans som sändarens utgångsimpedans och mäta spänningen över lasten
med ett osci!loscop med tillräcklig bandbredd. Då kan man se och mäta HF-spänningens topp-toppvärde och samtidigt se
signalens vågform.
Med spänningen och konstlastens impedans (resistans) bekanta så kan uteffekten
beräknas enligt formlerna på förra sidan
Den största HF-amplitud som uppstår
momentant vid modulering motsvarar PEPeffekten (PEP= Peak Enveiope Power).
En mindre exakt metod att mäta HFspänning är med voltmeter med likriktare.
Utifrån den uppmätta spänningen kan man
beräkna effekten över en belastning. På
grund av instrumentets tröghet visas emellertid bara ett "utjämnat" toppvärde, vilket
inte är det faktiska värde som instrumentet
"känner". Jämför med oscilloscopet som inte
har denna visningströg het.
Bild II 8-1
Bilden visar en voltmeter med likriktare,
som kopplats till en sändare över spänningsdelare. Två alternativa delare visas; den ena
består av resistorer och den andra av kondensatorer.
Den resistiva delaren är bättre i den meningen att den är frekvensoberoende och
inte belastar sändaren kapacitivt. Dessutom
dämpas övertoner som bildas vid likriktningen. l den kapacitiva delaren kan övertoner
passera lättare.
Denna mätmetod är noggrann bara när
impedansen är lika i sändaren, kabeln till
lasten och själva lasten. Lasten kan vara en
konstlast, en antenn etc och skall ha ett känt
värde för att effekten skall kunna beräknas.
Ett sätt att skaffa underlag för beräkning
av PEP-effekten är att mäta HF-strömmen
med ett termokorsinstrument och spänningen med en toppvärdesvisande voltmeter.
Utifrån dessa värden beräknar man effekten. Denna metod är dock inte så vanlig.

Direktvisande effektmetrar

Bild 118-6
Många föredrar direktvisande effektmätare.
En HF-voltmeter kan givetvis graderas för
att visa effekt i stället för spänning, men då
med den viktiga förutsättningen att impedansen måste ha en fastställt värde.
Om man avläser effekten genom en 75
Q-kabel på ett instrument för 50 Q, så är det
verkliga värdet ett annat än den avlästa.
De effektmetrar som förekommer i SVFinstrument är egentligen voltmetrar, men
med skalan graderad i effekt.

~Last
Sändare~

l

r~n
I
v

Bild II 8-1 Mätning av sändareffekt

118-3

ÄTNIN
Mäta ståendevågförhållande - SVF
När t. ex. en antennledning ansluts till en
antenn och deras impedanser inte är lika, så
kommer en del av inmatade effekten i ledningen att reflekteras tillbaka från antennen.
Det uppstår då en stående våg i ledningen. Förhållandet mellan inmatad och
reflekterad effekt uttrycks som ett ståendevågförhållande SVF (eng. SWR).
Med en SVF-meter som sätts in mellan
effektkälla och ledning kan man mäta hur
stor effekt som matas in i ledningen och hur
stor effekt som vänder tillbaka från slutet av
ledningen.
SVF-värdet kan bestämmas på något av
följande sätt:
• Man mäter framåt- respektive bakåtgående effekt var för sig med en riktningskänslig effektmete r. Man beräknar därefter SVF eller tar fram det ur ett diagram.
• Man använder ett instrument som beräknar eller visar SVF på något sätt.
studera vågformen
Vågformen för snabba växelströmsförlopp
studeras bäst med oscilloskop.
Mäta frekvens
Frekvensmätning gör man bäst med en s.k.
frekvensräknare, som är ett digitalt instrument. Man kan också använda en s.k. absorbtionsvågmeter, som är mycket enkel
och inte alls så exakt. Vid frekvensmätning
ansluter man instrumentet till mätobjektet
med en svag elektrisk eller magnetisk koppling.
Mäta resonansfrekvens
Mäta resonansfrekvensen för en passiv
svängningskrets gör man enklast med en
s.k. dip-meter. Även mer exakta metoder
finns.

118-4

Mätfel
Mätinstrument indelas i noggrannhetsklasser
efter största tillåtna felvisning. Klasserna är
0.1, 0.2, 0.5, 1.0, i .5, 2.5 och 5.0 varvid
klassen anges på instrumentet. Som exempel får ett instrument i klass 2.5 ha ett tillåtet
mätfel av$\pm$ 2.5 o/o av fullt utslag.
Mätresultatet bestäms av flera faktorer;
dels av instrumentets s.k.mätonoggrannhet,
dels av hur mätvärdet presenteras och slutligen av hur noga användaren läser av.
Vid analog visning presenteras mätvärdet med en visare mot en graderad skala
med en viss upplösning. Visaren kan vara
mekanisk eller optisk (ljusspalt). Vid snabba
mätvärdesändringar är instrumentets mekaniska tröghet en faktor att ta hänsyn till.
Vid digital visning presenteras mätvärdet
med siffror eller som längden på en pelare.
Det är förledande att se digital visning med
siffror som mer exakt än analog, men det är
inte alls säkert. Utöver instrumentets mätonoggrannhet, bestäms nämligen noggrannheten av hur många siffror som mätresultatet presenteras i.
En oberäknelig källa till mätfel är elektromagnetiska fält från apparater i närheten.
En ofta förbisedd felkälla är temperaturen i mätobjektet och/eller i instrumentet, det
kan vara av inkopplingstiden m.m ..
Visningströgheten är inget mätfel i sig
men kan till nackdel vid snabba förlopp.
Trögheten förekommersåväl vid analog som
digital visning. l det förra fallet är masströghet i instrumentets rörliga delar orsaken och
i det andra fallet är orsaken klackfrekvensen
för instrumentets mikroprocessor.

ÄTNIN
8.2 Mätinstrument

Presentation av mätvärden

Bild 118-2
Mätvärden kan presenteras på olika sätt. De
vanligaste sätten är optiska och då med
digital eller analog visning. Mätresultat kan
även överföras till datorerförvidare bearbetning och visning.

Multimeter

Bild II 8-2
Flera mätfunktioner kan utföras med samma
basinstrument Genom omkoppling mellan
olika tillsatser väljer man mätfunktion och
mätområde. Instrumentskalan utformas så
att olika slags mätvärden kan avläsas.
Kombinationer med elektroniska förstärkare
och digital visning etc. är nu vanligt.

Vridspoleinstrument

0

v·6"·n

.... "

A

-

O

+

MULTIMETER O

·D·.

.D:
Trigger

@

AY

in

Instrument med analog visning

A

.:Q:. Jololli3J.Jsl
DIGITALMETER

0

Bild II 8-3
Vridspoleinstrument kan bara användas för
likströmsmätning, eftersom visarutslaget
beror av strömriktningen. Instrumentet har
låg effektförbrukning och stor noggrannhet.
Visningen är vanligen linjär, men kan göras
annorlunda.
Funktion: En spole är upplagrad i fältet av
en hästskomagnet När den ström, som skall
mätas, passerar genom den vridbara spolen
så alstras ett magnetfält även i denna. De två
magnetfälten påverkarvarandra så att spolen
vrider sig. Spolen förses med en visare och
en returfjäder. Ju större ström det flyter genom spolen desto större blir visarutslaget

Volt
Ampere
Ohm

Instrument med digital visning

Bild II 8-2 Presentation av mätvärden

MODELL

----

Hästskomagnet

Vridspole
Visare\

Bild II 8-3 Vridspoleinstrument

118-5

MÄTNING

Spole

MODELL

Bild II 8-4 Mjukjärnsinstrument

Mjukjärnsinstrument

Bild II 8-4
Mjukjärnsinstrument kan användas för mätning av såväl lik- som växelström. Vid växelströmsmätning kommer effektiwärdet att
visas, oavsett strömmens kurvform.
Detta instrument har en relativt dålig precision, men är användbart för enklare ändamål. Det ersätts dock efter hand med billiga
digital instrument.
Funktion: l ett mjukjärnsinstrument sitter
två järnstycken B placerade. Det ena järnstycket sitter vridbart upphängt och är försett
med en returfjäder och en visare. Det andra
järnstycket sitterfast upphängt. Järnstyckena
magnetiseras av fältet i spolen. P.g .a. polariseringen kommer de alltid att stöta bort
varandra, oavsett strömriktningen i spolen.
Bortstötningskraften är ej proportionell mot
strömmen och skalan blir således olinjär.

Konstlast

Bild II 8-5
En konstlast (dummy load) bör ingå i varje
amatörradiostation. Vid mätning och inställning av t. ex. modulation och uteffekt, är det
lämpligt att belasta sändaren med dess nominella utgångsimpedans. För att då undvika att energi strålas ut bör en väl skärmad
konstlast användas.
l moderna amatörradiosändare med koaxialkabel utgång är utgångsimpedansen 50
n. Konstlasten skall då vara en 50 n res istor
utan reaktiva egenskaper. Den kan bestå av
en eller flera sammankopplade resistorer.
Sändareffekten skall kunna tas upp utan
att resistansen förändras nämnvärt. Det är
viktigt att resistorerna kyls effektivt med luft
eller vätska i ett kär! med tillräckligt utrymme, även när vätskan expanderar av värmen. Vätskan får inte vara lättantändlig eller
miljöfarlig. T. ex. är oljormed PCBförbjudna!

S!
r---------,

L  jO.Q.Sl

Bild fl 8-5 Konstlast

118-6

--j

IN

TX
Bild II 8-6 Fältstyrkemätare

Fältstyrkemätare
Bild II 8-6

styrkan av elektromagnetiska fält kan bestämmas med fältstyrkemätare.
En fältstyrkemätare är en högfrekvensdetektor, vars utspänning visas med ett instrument med skala. Den selektiva
kretsen kan bestå enbart av den ....,,u··+r>...,"'r~ ....
antennen, men även av ytterligare selektiva
kretsar. Instrumentet visar endast relativa
värden och används t. ex. för att bestämma
strålningsegenskaperna i sändarantenner
och för antennjustering. Mätresultatet påverkas även av utstrålning från andra sändare inom mätarens bandbredd. Bilden visar
en sändare och en fältstyrkemätare. Dessutom två enkla fältstyrkemätare.

Kalibreringsoscillator
Bild II 8-7

En kalibreringsoscillator används för attfrekvenskalibrera andra apparaters inställningsskalor. Den är kristallstyrd och avger särskilt
precisa och frekvensstabila signaler.
Oscillatorsignalen förvrängs avsiktligt, så
att det utöver grundfrekvensen även skapas
harmoniska övertoner. En
oscillator med t. ex. grundfrekvensen 25
avger på så sätt även frekvenserna 50
75 kHz, i 00 kHz, i 25 kHz o.s.v .. Man får
således en "kalibreringsfrekvens"
25kHz.
Detta övertonsspektrum kan sträcka
flera 100 MHz upp. Man "nollsvävar"
apparat mot närmaste kalibreringsfrekvens
och kan kalibrera t. ex. VFO-skalan.
Användningsområden:
• Kalibrering av mottagare och
• Gradering av nya skalor o.s.v.

trafikmottagare har VFO med
L C-krets och ofta en inbyggd kalibreringsoscillator. En kalibreringsoscillator kan i sin tur
behöva kalibreras. Det enklaste sättet är då,
att jämföra frekvensen på en känd rundradiosändare på mellanvåg med kalibreringsoscillatorn. Dagens mottagare och sändare
har syntesoscillator och då behövs normalt
ingen kalibreringsoscillator.

Bild 118-7 Kalibreringsoscillator i mottagare

Brusmätbrygga

Bild Ii 8-8
Brusmätbryggan används vid mätning i antennsystem. Den består av en brusgenerator och en Wheatstone-brygga för mätning
av resistans och reaktans.

Bild II 8-8 Brusmätbrygga
118-7

PT

MÄTNIN

f~

a---------.------n
~----------------------,
l
l

l
l
l
l

l

[[I[]J

~tonn

l

l

l
l

l

LJ --- - --- JJ
)l

)l

Bild II 8-9 SVF-meter, princip och inkoppling
Till bryggan ansluts en antenn som mätobjekt och en mottagare som nollindikeringsinstrument för brussignalen. Mottagaren
ställs in på den frekvens där mätvärden
önskas. Bruset hörs svagast när bryggan är
in justerad. Man kan då avläsas mätvärdena
för R och X. Mäter man vid flera frekvenser,
kan t.ex. ett impedansdiagram upprättas.
ståendevågmeter (SVFameter)
Bild 118-9
När en transmissionsledning eller apparat
ansluts till en annan med awikande impedans, kommer HF-energi att reflekteras i
övergången.
Med ståendevåg-förhållande (SVF) eller
standing W ave Ratio (SWR)) menas förhållandet mellan den effekt som flyter framåt
respektive bakåt i en transmissionsledning.

Användningområden för SVF-meter:
• Mätning av framåtgående effekt.
e Mätning av bakåtgående effekt.
e Bestämning av SVF.
• Bestämning av resulterande, relativ effekt.
Anmärkning: Vid bestämning av absolut
effekt måste anslutningsimpedansen vara
lika i instrument och transmissionsledning.
SVF-metern är ett av de mest användbara instrumenten vid HF-mätningar. En
SVF-meter kan ha separata instrument för
fram- respektive backeffekt eller ett gemensamt.
SVF-metern kan vara ständigt inkopplad
t.ex. mellan sändare och antenn, men skall
då kunna tåla effektutvecklingen. En SVFmeter kan alstra övertoner, vilka kan medföra störningar. Orsaken är olinjäriteten
halvledardioderna i instrumentet.

~--------------------,

l

i

~

l

i

t~;~-=t--~-J
Bild/18-10 Frekvensräknare

118-8

Bild II 8-11 Absorbtionsvågmeter

MÄTNIN
Frekvensräknare
Bild II 8-1 O
Frekvensräknaren, som är ett digitalt instrument, används för att bestämma oscillatorfrekvensen i sändare, mottagare m.m.
l frekvensräknaren räknas antalet svängningar i den aktuella inkommande signalen
under en bestämd tidsenhet. Först förstärks
signalen i en analog förstärkare och omvandlas till kantvågspulser. En elektronisk
"kontakt", en s. k. gate, släpper därefter den
behandlade ingångssignalen vidare till en
digital räknare under en viss tid. Detta sker
med stor precision och i ett periodiskt förlopp. Antalet pulser räknas under genomsläppsperioden. Resultatet motsvarar insignalens frekvens.
Resultatet visas som siffror i ett fönster.
Noggrannheten i den s.k. tidbasen erhålls
med en kristallstyrd oscillator, vars frekvens
delas ner till önskat värde.
Absorbtionsvågmeter
Bild II 8-11
Absorbtionsvågmetern används för att bestämma en oscillators arbetsfrekvens. Den
består av en resonanskrets med variabel
frekvens, som kan avläsas, och ett mätinstrument som resonansindikator.
Vågmetern kopplas induktivttill den krets,
vars frekvens skall bestämmas. När frekvensen i kretsen och vågmetern stämmer
överens, ger resonansindikatorn utslag. Frekvensen avläses då på vågmeterns skala.
Anmärkning: Frekvensmätning på en
passiv svängningskrets kan inte göras med
detta instrument, vilket däremot går med en
dip-meter. Principen för en absorbtionsvågmeter är annorlunda än den för en dip-meter,
men i regel kan en dip-meter också användas som absorbtionsvågmeter.

Bild II 8-12 Dip-meter

fj:obje~
Bild II 8-13 Mätning med dip-meter
Dip~meter

Bild II 8-12 och -i 3
Dipmetern är i princip en oscillator med
variabel frekvens och utbytbara induktorer
för olika frekvensområden.
Den används för att bestämma resonansfrekvensen på passiva och aktiva svängningskretsar samt vid bestämning av induktanser och kapacitanser.
Noggrannheten är c:a 3 $\circ$/o.
Funktion: Instrumentet avger alternativt
reagerarför en H F-signal med viss frekvens.
Resonansfrekvensen i dip-meterns svängningskrets är steglöst variabel och frekvensvärdet kan avläsas på en skala.

Y-förstärkare

X-ingång

Bild II 8-14 Oscilloscop

l
118-9

MÄTNIN
Vid mätning av resonansfrekvensen i en
passiv svängningskrets kopplas dip-meterns
induktor induktivt till kretsen. När resonansfrekvensen i kretsen och dip-metern överensstämmer, ändras belastningen i d lp-meterns svängningskrets varvid instrumentet
uppvisaren strömminskning-en "di p". Frekvensen avläses då på skalskivan.
Vid mätning på en aktiv svängningskrets,
d.v.s. som drivs av någon H F-källa, uppstår
i stället en strömökning vid resonans vilket
också visas på instrumentet.
Induktansen i en svängningskrets kan
bestämmas med dip-metern, om kapacitansenär bekant. På motsvarande sätt kan en
obekant kapacitans bestämmas om induktansen i svängningskretsen är bekant.
Namnet grid-dipmeter kommer från
elektronrörsepoken. Ändringar i gallerströmmen (g rid current) i ett oscillatorkopplat
elektronrör används som indikation på att en
svängningskrets är i resonans. Då minskar
gallerströmmen - det blir en "ström-dip".
Numera används en transistor i stället för
röret och instrumentet benämns dip-meter.

Oscilloskop
Bild II 8-14
Oscilloskopet är ett mycket användbart instrument. Mycket snabba förlopp kan med
fördel studeras på en oscilloskopskärm.
Spänningsförlopp kan visas som funktion av tiden. Tillsammans med andra instrument kan frekvenskaraktäristiken i filter,
modulationskvalitet o.s.v. åskådliggöras.
Oscilloskopet består av ett katodstrålerör, där styrningen av katodstrålen sker med
hjälp av X- och Y-förstärkare och en s.k.
triggerförstärkare. Den signal som skall mätas ansluts vanligen till Y-förstärkaren medan
en tidbasgenerator som alstrar en sågtandformad signal ansluts till X-förstärkaren.
Bildenvisar ett blockschema på oscilloscop.

118- i o

PT

\chapter{EMC}

Endast en del av frekvensspektrum av elektromagnetiska vågor används för radiosändningar. Samtidigt används detta utrymme av
allt fler intressenter och för allt fler ändamål.
Samhället blir alltmer tekniskt avancerat och
elektroniktätheten tilltar kraftigt. Den ökande mängden och komplexiteten hos apparaterna kräver därför regler, som styr både
utförande och användning med rimligt bibehållen säkerhet och funktion.

Störningar och störkänslighet
Om EMC-Iagen
Det kan inte längre ges enkla svar på vad
som är att vara störd och att störa. Internationella och nationella väl preciserade regler
för radio- och teletekniskt samexistens är
numera helt nödvändiga.
Samlingsbegreppet är Electromagnetic
Compatibility (EMC), d.v.s. en apparats förmåga att fungera tillfredsställande i sin elektromagnetiska omgivning utan att alstra elektromagnetiska störningar som överstiger en
nivå, som tillåter radio- och teleutrustning
och andra apparater att fungera som avsett.
Vidare skall apparater ha en sådan tillräcklig
inbyggd tålighet mot elektromagnetiska störningar, att de kan fungera som avsett.
Till skydd för liv, personlig säkerhet och
hälsa samt kommunikationer och näringsverksamheter har därför Lag om elektromagnetisk kompatibilitet införts. Denna lag
är anpassad efter av EG/EES utfärdade
direktiv angående bl.a. radiostörningar.
Förordning om elektromagnetisk kompatibilitet definierar nyckelbegreppen; apparater, EMC, elektromagnetiskstörning och
tålighet. Elsäkerhetsverket är ansvarig myndighet, med rätt att utfärda föreskrifter om
bl.a. skyddskraven, kontroll och märkning
samt om vissa undantag. Sådana föreskrifter är bl.a. ELSÄK-FS.

EPT

Post och telestyrelsens föreskrifter om innehav och användning av amatörradioanläggningar m.m. hänvisar till den överordnade Lag om radiokommunikation där följande finns om Åtgärder mot störningar:

Ur radiolagen

"Om en radiosändare stör användningen av
en annan radioanläggning skall den som har
tillstånd att inneha och använda radiosändaren ombesörja att störningen upphör eller i
görligaste mån minskas. Motsvarande skyldighet gäller för innehavare av radiomottagare som stör användningen av en annan
radiomottagare.
Den myndighet som regeringen bestämmerfårförelägga den som enligtförstastycket
är skyldig att vidta åtgärder mot en störning
att fullgöra denna skyldighet. Ett sådant föreläggande får förenas med vite."
" Elektriska eller elektroniska anläggningar
som, utan att vara radioanläggningar, är
avsedda att alstra radiofrekvent energi för
kommunikationsändamål eller industriellt,
vetenskapligt, medicinskt eller något liknande ändamål, får användas endast i enlighet
med föreskrifter som meddelas av regeringen eller den myndighet som regeringen bestämmer.
Den myndighet som regeringen bestämmer får meddela de förelägganden och förbud som behövs i ett enskilt fall för att föreskrifterna i första stycket skall följas. Ett
sådant föreläggande får förenas med vite.
Regeringen får meddela föreskrifter om
förbud mot att inneha elektriska eller elektroniska anläggningar som inte omfattas av
första stycket och som, utan att vara radioanläggningar, är avsedda att sända radiovågor."
l radiolagen definieras bl. a. radioanläggningsom en anordning för radiokommunikation eller radiobestämning genom sändning
(radiosändare) eller mottagning (radiomottagare) av radiovågor.

119-1

Utstrålning från amatörradiosändare
Vad som sägs i radiolagen innebär att
såndareffekten alltid skall anpassas så att
styrkan av utstrålade fält inte förorsakar störningar. Den enligt amatörradioföreskrifterna
högsta tillåtna effekten kan alltså inte användas hinders/öst. l samma paragrafstår också
att PTS i tillståndet kan besluta om andra
effektgränser, om det finns särskilda skäl.
Om störningarna inte kan avhjälpas kan
PTS komma att anvisa om restriktioner (begränsningar i sändningstillståndet), det kan
vara sändningsförbud under vissa tider, på
vissa frekvenser, över viss såndareffekt etc.

PM vid störningsproblem
• Störningar är alltid förenade med obehag
och ställer grannsämjan på prov. Håll Dig
väl med dem som bor i omgivningen!
• Om det väcks klagomål på Dig om störningar, skall Du först kontrollera Din egen
sändare och antennanläggning.
• Be därefter att få undersöka antennanläggning och apparater hos den som besväras av störningar.
• Om Du ser en lösning, berätta om vad
som kan göras. Kom överens om vad
som får göras. Ändra då inte något inne i
apparater, men provagärna ut yttre, kompletterande filter etc.
• Om det inte går att komma till rätta med
störningarna bör de som levererat och
installerat anläggningen anlitas.
• Störningsanmälan kan även ske till PTS
närmaste tillsynsområde.
Arbeta aktivt med avstörning
• Låna hem en av SSA:s avstörningslådor
och försök att finna en lösning. l lådan
finns ett sortiment av frekvensfilter för
avstörning,
• Undvik att störa i onödan. Sänk sändareffekten och begränsa sändningstiden
under utprovningen av en lösning.
•
•

Lyckas Du inte själv med att störa av
Ta gärna hjälp av en radioamatör med
erfarenhet av avstörning eller
Anlita annan sakkunnig hjälp.

119-2

9a 1 Störningar i elektronik

Liksom att radiomottagning kan "störas" av
sändningar som inte är av intresse, så kan
störningar i form av radiovågor från olika
slags elektrisk utrustning försvåra mottagning eller andra funktioner.
Utstrålning från t. ex. datorer, kabel-TV,
hushållsmaskiner, tändgnistorfrån oljebrännare, bilar och mopeder etc. är radiovågor.
Elektriska apparater kan alltså både störa
och störas genom radiovågor, även om de
inte är definierade som radioanläggning,
d.v.s. radiosändare och/eller radiomottagare.
Störningar som uppstår av elektromagnetiska fält kallas Electromagnetic lnterference - EMI. Känsligheten för sådana störningar kallas för Electromagnetic Susceptibility- EMS.
Äldre benämningar på störningar är t. ex.
BCI (broadcasting interference) och TVI (television interference). Dessa återfinns dock
inte i nu gällande terminologi.
Blockering
l de flesta radiomottagare finns en automatisk förstärkningsreglering. Om insignalerna
blir för starka, så räcker regleringen inte till.
Då överstyrs förstärkarstegen så att de arbetar olinjärt. Detta kallas blockering.
Ett sätt att undvika blockering är att koppla en dämpsats - attenuator- till mottagaringången. En sådan sänker dock signalstyrkan över hela frekvensområdet, inte bara för
en viss signalfrekvens.
Interferens
När den önskade signalen störs av en annan frekvensnära signal, kallas det interference. l mottagaringången finns frekvensfilter, som undertrycker ej önskade signaler,
om de inte ligger allför nära. Om ingången
inte är tillräckligt selektiv, kan det behövas
en tillsats som förbättrar selektiviteten.
lntermodulation (se även kapitel 4)
Blandningsprodukter av signaler i en mottagare eller sändare kallas för intermodulation.
lFmdetektering
HF-signaler kan komma in genom in- och
utgångarna för LF samt genom nätkabein.

EMC
Dessutom förekommer direktinstrålning
av radiovågor genom apparathölj et, om detta
inte har tillräckligt avskärmande verkan.
LF-detektering uppstår när HF-signaler
dernoduleras i diodsträckor i den störda ap~
paratens komponenter. Detta sker oavsett
vilken frekvens som sändaren eller mottagaren är inställd på.
LF-detektering uppstår särskilt vid AMeller SSB-modulerade sändningar samt av
transienter vid bärvågsnyekling av sändare.
Ofta är det inte möjligt att förhindra LFdetektering utan ingrepp i den störda apparaten. Ingrepp får endast göras av fackman.

\section{Störningsorsaker}

Störningar från sändare
HF-förstärkare, t.ex. i sändarslutsteg, kan
komma i oönskad självsvängning, vilket kan
uppstå av flera orsaker; det kan vara bristande avkoppling av matningsspänningar, induktiv och/eller kapacitiv återkoppling etc ..
Effektuförstärkare kan även överstyras.
Då uppstår intermodulation och övertoner
som strålas ut på oönskade frekvenser.
l många fall kan störningar undvikas med
en eller flera av följande åtgärder:
• Undvik att överstyra sändarslutsteget
(kontrolleras t.ex. med ALG-mätaren).
• Förse sändarutgången med lågpassfilter.
På så sätt undertrycks övertoner.
• Anpassa sändarens och antennanläggningens impedanser till varandra. Stäm
av sändarens re-filter och/eller en separat
antennanpassningsenhet rätt. En felinställd sändare kan medföra oavsiktligt
utstrålning.
• Koppla in balanseringsnät (bal un) mellan
osymmetriska antenntilledningar (koaxialkablar) och symmetriska antenner.
• Placera antennen högt och fritt och så
långt från personer och störningskänslig
utrustning som möjligt. Fältstyrkan är
nämligen högst närmast antennen.
• Undvik direkt HF-instrålning på belysningsnätet genom att använda nätfilter.
• Använd "mjuk" nyekling av bärvågen (avrundade telegrafitecken). Vid hård nyekling alstras övertoner i form av knäppar
som hörs långt vid sidan av sändningsfrekvensen.

Störningar på radiomottagning
l regel uppstår störningar på radiomottagning
först när utstrålade signaler uppnått en viss
styrka- immunitetsnivån för HF.
Man kan tala om tre slags HF-immunitet
hos mottagare;
• mot signaler genom antenningången,
• mot signaler genom övriga anslutna ledningar, t.ex. högtalar- och nätledningar,
• mot elektriska och/eller magnetiska fält
som strålar direkt in i apparaten
l de båda första fallen kan det hjälpa med
komplettering med hög- och/eller lågpassfilter och skärmströmsfilter.
Instrålningsstörningar är svårast att avhjälpa och fordrar ingrepp i mottagaren, vilket bör överlåtas till en fackman med tillgång
till tillverkarens serviceinstruktioner.
Störningar på TV-mottagning
Störningar från radiosändare kan yttra sig
t.ex. på följande sätt:
• Vid sändning av amplitudmodulerade signaler, t.ex. AM och SSB, uppstår ljudförvrängning i ljudkanalen samt ränder
m.m. i bilden,
• Vid sändning av FM och CW uppstår ljudstörningar samt kontrastvariationer, interferensmönster (moire-effekter) m.m. i
bilden.

Störningar i TV som orsakas av sändare
på lägre frekvenser kan i många fall avhjälpas med frekvensfilter. Det kan t. ex. uppstå
TV-störningar, när en amatörradiostation
sänder på 21 MHz-bandet. Dess 3:e överton
hamnar då på TV-kanal E-4 (61 - 68 MHz).
Ett lågpassfilter efter en KV-sändare kan
t.ex. dimensioneras att endast släppa igenom signaler under c:a 35 MHz.
Ett högpassfilter före en TV-mottagarekan t.ex. dimensioneras att endast släppa
igenom signaler med frekvenser över c:a 35
MHz.
Om inte mottagning i TV-band l (40- 68
MHz) är av intresse, så kan ett högpassfilter
med en gränsfrekvens av ca 160 MHz sättas
in. Det dämpar då oönskad utstrålning från
sändare i KV- och VHF-området, d.v.s. upp
t.o.m. 144-146 MHz amatörband. Däremot
släpps TV-band III (174- 230 MHz) och TVbanden IV och V igenom (470- 890 MHZ).
119-3

E
Ytterligare avstörningsmedel kan sättas
in om det uppstår störningar av amatörradiosändningar. Det kan vara skärmströmsfilter
på antennkablar, bandspärrar samt sug- och
spärrkretsar avstämda till störfrekvensen,
bandpassfilter avstämt till nyttofrekvensen.
Ett vanligt störningsfall är att en dåligt
skärmad och bredbandig antennförstärkare
blir överstyrd av starka sändare.

Det kan förekomma kraftiga spänningstransienter (spänningsstötar) på belysningsnätet. Dessa translenter kan leda till felfunktioner i anslutna apparater. För att förebygga
sådana fel kan man koppla in ett överspänningsfilter, som kan vara separat eller sammanbyggt med nätfiltret

Störningar på LF-apparater
Störningar av HF-instrålning i ljudbandspelare, LF-förstärkare, telefonapparater etc. kan
ofta stoppas med avkopplingskondensatorer och HF-drosslar. Moderna avstörningsdrassiar innehåller oftast något ferritmaterial
i form av rör, stavar eller ringar.

\section{Avstörningsmetoder}

Allmänt
För att prova ut ett filter, som bäst löser ett
visst radiostörningsproblem, kan man behöva tillgång till ett filtersortiment
Som exempel nämns bl. a. filter i SSA:s
avstörningslådor.
Nätfilter
Nätledningar kan fungera som antenn. l
såndarfallet kan HF-signaler komma ut i
elnätet genom nätledningen och störa andra
apparater både genom direktanslutning och
strålning. l mottagarfallet kan HF-signaler
uppfångas av nätledningen, ledas in i apparaterna och LF-detekteras där. För att förhindra sådana störningar behövs ett nätfilter.
Nätfiltret skall vara dimensionerat för den
nätström, som apparaten är avsäkrad för och
bör anslutas så nära apparaten som möjligt.
Om filtret inte kan placeras där, kan det vara
nödvändigt att även skärma nätledningen
mellan filtret och apparaten och jorda skärmen.
Om ledningen förses med t. ex. en serieinduktans -en drassel -så dämpas HFsignalerna. En drassel kan man göra t.ex.
genom att linda upp några varv av nätsladden
närmast apparaten på torcider eller en eller
flera sammanlagda ferritstavar. l svåra fall
kan det behövas ett bredbandigt nätfilter,
liknande det på bild Il 9-1.

119-4

lågpassfilter
Lågpassfilter släpper igenom signaler med
frekvenser under filtrets gränsfrekvens.
Ett lågpassfilter med lämpligt vald gränsfrekvens dämpar t. ex. övertonsutstrålningen
från en sändare, vars såndarfrekvens ligger
under filtrets gränsfrekvens medan övertonerna ligger över dess gränsfrekvens.
Övertoner kan dämpas med lågpassfilter. En överton är i detta sammanhanhang
en multipel av sändningsfrekvensen (grundtonen) exempelvis för 3.5 MHz
grundtonen = (1 :a harmoniska) 3.5 MHz,
1 :a överton = (2:a harmoniska) 7.0 MHz,
2:a överton = (3:e harmoniska) 10.5 MHz
o.s.v.
Viktigt för avsedd filterverkan är, att filtret
ansluts med korrekt impedansanpassning
och med kortast möjliga ledningar. Detta
gäller f.ö. alla filter.
Utstrålning utanför sändningsslagets tillåtna bandbredd anses som "icke önskad
utstrålning". Vidare gäller att sådan utstrålning från amatörradiosändare skall hållas så
låg som dagens amatörradioteknik medger.
Bild 119-2 visar principen för lågpassfiltret TP
30 för kortvåg, med gränsfrekvensen 36
MHz, att kopplas mellan sändaren och antennledningen. Med denna gränsfrekvens
dämpas övertoner från sändare så att risken
för TV-störningar minskar.

r---------------------------l koaxial-

koaxial- 1
kabel från~
sändaren 1

1

r:- kabel till
l

l
l

l

:
l

L---------------

Högpassfilter

l

Om en störande signal råkar finnas inom
passbandet för mottagaren kan man undertrycka - "spärra" - den signalen med ett
spärr- eller sugfilter. Vilket man väljer är inte
kritiskt.
Den störande signalen kan "spärras" med
en parallellresonanskrets i serie med
mottagaringången (Bild 9-5). Kretsen består av en induktans och en kapacitans.

.

koax1al-

:l

Bild II 9-2

l
---------J

Högpassfilter släpper igenom signaler med
frekvenser över filtrets gränsfrekvens.
Bild II 9-3 visar principen för högpassfiltret HP 40-S med gränsfrekvensen 47 MHz,
att kopplas in mellan antennledningen och
en mottagare för VHF eller högre frekvenser.
Störningar kommer inte alltid "utifrån".
De kan t.ex. alstras i bredbandiga antennförstärkare, vilka lätt överstyrs av alla slags
signaler från ett stort frekvensområde. Man
kan då koppla in ett högpassfilter före bredbandsförstärkaren, men en bättre lösning är
att byta till en väl skärmad passbands- eller
ännu hellre kanalförstärkare.
Koaxialkablar med täta skärmar och rätt
monterade anslutningskontakter är också
viktigt för en lyckad avstörning.

Spärrfilter och sugkretsar

antennen

:

Lågpassfilter
försändare

Om man använder enstub som resonanskrets- t. ex. en koaxialkabel- så skall den ha
längden fi./4 och vara "kortsluten" eller ha
längden A./2 och vara "öppen".
Man kan även kortsluta - "suga bort" den störande signalen med en serieresonanskrets parallellt över mottagaringången
(Bild 9-6). Om man då använder en stub, så
skall den ha längden ').)4 och vara "öppen"
eller ha längden ').)2 och vara "kortsluten".
Den störande signalen kan undertryckas
ytterligare med fler stubar, som ordnas som
i Bild 9-6. Filtret består då av öppna ').)4stubar, som utgör avgreningar från antennkabeln med ett avstånd av ').)4.
(Om stubarna i detta filter kortsluts, så
bildas ett bandpassfilter i stället).
Exempel på kommersieila spärrfilter är SF
145-S för 144 MHz och SF 435-S, för 435
MHz amatörband. De är avsedda att kopplas in före mottagare som störs av amatörradiosändningar.
SF 145-S spärrar amatörbandet 144 148 MHz och släpper igenom banden O-120
och 174- 870 MHz.
SF 435-S spärrar amatörbandet 430 440 M Hz och släpper igenom O- 350 och 470
-870 MHz.

r------------------------------, koaxial-

..~:-,

~~~:~~:~ ~~~'fTiT

l: l
i

1

11!11 ~~t~~~~ren
!

'--------------- --- --------------------.J
'

l

Bild II 9-3
HögpassfilterförVHF!UHF-mottagare

119-5

EMC

Z=~

Parallellresonanskrets

Kortsluten A/4-ledning

Öppen :A./2-Iedning

Serieresonanskrets

Öppen Ä/4-ledning

l Kortsluten A/2-ledning

z= o

Bild II 9-4 Ingångsimpedansen i resonanskretsar

från
antennen

till
mottagaren

från
antennen

till
mottagaren

Bild II 9-5 Spärrfilter för mottagare

Sugkre~

~x

T

Bild II 9-6 sugkretsar för mottagare

119-6

EMC

PT
Nät- och skärmströmfilter för mottagning
Bild II 9-7
Utsidan av antennkabelns skärm kan också
fungera som antenn. Särskilt i skärmskaNar
kan HF-strömmar läcka ut och in. De kan då
passera förbi eventuella antennförstärkare,
filter etc. och orsaka störningar.
l enkla fall kan yttre skärmströmmar stoppas med att linda upp kabeln några vaN på
ferritstavareller genom en storferritring som
på bilden. En nätkabel, s.k. sladdställ, får
inte kapas och skaNas.

Högtalarledningsfilter (EM 502-B)
Bild II 9-9
HF-instrålning på högtalarledningar kan ha
en störande påverkan. Detta kan undvikas
genom koppla in HF-drosslar i ledningarna.
Dessa d rosslar bör vara skärmade så att de
inte verkar som antenner istället.
l enklare fall kan det räcka med att byta till
skärmade högtalarkablar eller att linda upp
en sträcka av ledningarna på en ferritkärna.

Phono .. ingångsfilter (TBA 302)
Bild II 9-8
Störande påverkan från radiosändningar kan
uppstå om anslutningsledningarna till phono-ingången i LF-förstärkare är dåligt skärmade och avkopplade. Sådana störningar
kan avhjälpas med ett filter.

Ferritstav

Ferritring

'·
J,

Bild 119-7 Nät- och skärmströmfilter

o

från
mottagaren
o

l

r;n- l

rLt..

I

J

•l']
or

~

·r

o

till
högtalaren
o

Bild II 9-9 Högtalarledningsfilter

Avkoppling av HF-signaler
Med avkoppling av en signal menas att
den avleds från en signalväg till en
annan. Vid avstörning avkopplas vanligen den störande signalen till systemjord.
störimmuniteten i mottagare kan
alltså förbättras genom att LF-ingångarna H F-avkopplas med kondensatorer
och/eller HF-spärras med drosslar.
l svåra störningsfall kan det också
bli nödvändigt med H F-avskärmning av
LF-ingångsstegen, liksom med ytterligare avstörningsfilter inne i förstärkaren.
Sådana åtgärder innebär emellertid
att konstruktionsändringar har gjorts.
Apparatens elsäkerhetsmärkningar är
då ogiltiga.

r----------------,

l

l

phono

förstärkare

l

l

l

ll

r

1

~~~--~--~~~

l

--------.J

Bild II 9-8 Phonoingångsfilter

Bild II 9-1 Oa H F-avkopplat styrgaller

119-7

EM

Bild II 9-1 Ob H F-avkopplad bas på tre sätt

Dr

hårda CW-tecken

-llllllllllllllllllll-lllllllllllllllllllllllllllllllllllllllllllllllllllllllll-.

Bild II 9-11 Parasitfilter i H F-förstärkare
Bild 119-10a-b visar några sätt att avkoppla
en oönskad signal från styrgallret i ett elektronrör respektive från basen i en transistor.

Parasitfil ter
Bild Ii 9-11

Förstärkarsteg kan råka i självsvängning,
ofta på frekvenser i VHF/UHF-området. Ett
sätt att stoppa det är med s.k. parasitfilter.

mjuka CW-tecken

-I I I I I /J/ I i~-,i!I JI I I I I I I I I I I I I I I I I I I I I/ /I f!r-

Nyekling stilter

Bild II 9-12
När en bärvåg nyck! as, så bildas övertoner.
Blandningsprodukter av övertonerna och
bärvågen hörs som knäppar på omkringliggande frekvenser. Märk att övertoner uppstår vid all bärvågsnyekling - inte bara vid
morsetelegrafering!
När övergångstiden är kort (hård nyckling), så bildas fler övertoner än när den är
längre (mjuk nyckling). Knäpparna kan till en
del dämpas med ett nycklingsfilter där dels
insvängningsförloppet bromsas med en d rossel i serie med nycklingskontakten och dels
ursvängningsförloppet med en seriekrets av
en resister och en kondensator, kopplade
parallellt över nycklingskontakten.

119-8

Bild II 9-12 Nycklingsfilter

skärmning

HF-energi kan i olyckliga fall även stråla ut
genom sändarens hölje och in genom andra
apparaters hölje. Det medför att apparaternas skärmningar och jordning måste förbättras. Följ då elsäkerhetsbestämmelserna!
Se även kapitel 1.3, 1 .4 och 1O.

\chapter{FARLIGA STRÖMMEN}

\section{Människokroppen}

Elektrisk chock

Strömmens inverkan på människan

Människokroppen är ett komplicerat elektrokemiskt system, som främst kontrolleras av
hjärnan. Musklerna styrs av svaga elektriska
strömimpulser genom nervsystemet. Främmande strömmar genom kroppen kan störa
kroppsfunktioner och kan i olyckliga fall göra
stor skada. Styrkan och frekvensen på strömmarna avgör skadans art och omfattning.
Elektrisk chock kan döda av flera orsaker.
En orsak är att hjärtrytmen störs. Hjärtkammarflimmer och hjärtstillestånd kan lätt
uppstå. Flimmer innebär att hjärtat arbetar
okontrollerat och med kraftigt nedsatt eller
helt upphävd pumpfunktion. Hjärtstillestånd
inträffar lätt av hög spänning. Av otillräcklig
blodtillförsel blir det syrebrist i hjärncellerna,
som då skadas snabbt. Medvetslöshet inträder redan efter ett fåtal sekunder.
En annan orsak är andningsstillestånd
genom att andningscentum blockeras. Det
kan hända när strömmen från en högspänningskondensator i en sändare går genom
kroppen.

Strömstyrkan påverkar kroppen olika från
fall till fall och det är osäkert vilken strömstyrka som är farlig. Det finns både de som
överlevt höga strömmar och de som inte har
klarat några milliampere. Strömmar som går
genom hjärta eller hjärna är särskilt farliga.
Hjärtat sitter i strömvägen för vänster hand,
så när man arbetar med elektriska apparater
under spänning, bör man för säkerhets skull
hålla vänstra handen i fickan l
starka strömmar ger häftiga muskelkramper och/eller brännskador. Muskelkramp kan
förekomma redan vid strömmar under 1O
mA. För vuxna, friska människor är det direkt
farligt när strömmen överstiger detta värde.
För unga eller sjuka redan vid lägre värden.
Men även en "ofarlig" ström kan trots allt
vara ett indirekt faromoment. Vid en oväntad
"stöt" blir man rädd och gör okontrollerade
rörelser, vilket kan leda till fall eller oavsiktlig
beröring av spänningsförande föremål i närheten.

Resistansen genom människokroppen

Påverkan från elektromagnetiska fält

Vid kontakt med ett strömförande föremål
kommer kroppen att bli en del av strömkretsen. Det flyter då en främmande ström genom kroppen.
Strömstyrkan följer Ohms lag och beror
av strömkällans spänning och inre resistans
samt av övergångsresistansen i huden och
kroppens inre resistans.
Övergångsresistansen minskar med fuktigare hud samt med större kontaktyta och
större kontakttryck. Beröringsspänningen inverkar också. Vid spänningar över ca 75 V
minskar övergångsresistansen med ökande
spänning. Vid allvarliga förbränningar minskar övergångsresistansen särskilt mycket.
Den totala resistansen genom kroppen blir
då nära lika med dess inre resistans- ungefär 500 n.
VARNING. Experimentera inte med detta!

Undersökningar harvisat attvistelse i starka
elektromagnetiska fält kan kan påverka människan. Personer som har varit utsatta för
kraftig exponering av fält har bl.a. klagat över
svettningar och huvudvärk. Det forskas omkring dessa fenomen.
Elektromagnetiska fält kan förorsaka fel
i elektronikutrustningar. Halvledare är särskilt känsliga för kraftfält. Det är möjligt att
känsliga instrument, hjärtstimulatorer (pacemaker) etc. kan påverkas av högfrekventa
elektromagnetiska fältfrån radiosändare. När
du använder en sändare, mobiltelefon etc.
och någon får svårigheter med hjärta eller
andning så skall du omedelbart stänga av
din apparat helt! Med tiden utvecklas störningsokänsligare elektronik, men säker mot
störningar kan man aldrig vara.

111 o-1

FAR Ll

STR

EN

Normer för fältstyrkor

Det finns normer och rekommendationer för
elektromagnetiska fältstyrkor i olika syften.
Dels avses fältstyrkor som människor
och djur får utsättas för, dels fältstyrkor som
olika slags apparater skall kunna fungera i
respektive själva utsänder (EMC).
Samarbete mellan länderna utvecklas
inom båda dessa områden.

Konstgjord andning

Vid hjärtstillestånd, hjärtkammarflimmer och
andningsstillestånd skall hjärtmassage och
konstgjord andning sättas in omedelbart och
på ett kunnigt sätt. Obotliga hjärnskador av
syrebrist kan nämligen uppstå inom några få
minuter.
Livräddning vid e/skada är ett instruktivt
häfte från Energikontorets Förlagsservice,
101 53 Stockholm. studium rekommenderas.
Det är för sent när olyckan har skett. Häftet
kan beställas på fax 08 677 26 05.

\section{Allmänna elnätet}

Elektrisk energi levereras till förbrukarna över
transformatorstationerdär högspänning först
transformeras till lågspänning. Från transformatorstationerna förgrenas lågspänningsnätet till serviceskåp ute i kvarter och byar.
l Sverige är fördelningstransformatorns
sekundärlindningar oftast sammankopplade
till ett Y (s.k. Y- eller stjärnkoppling) där
mittpunkten är jordad.
De i Sverige vanligast förekommande 3fas lågspänningsnäten har huvudspänningen
400 V (tidigare 380V) och fasspänningen
230 V (tidigare 220 V). Spänningen mellan
fasledarna kallas för huvudspänning och
spänningen mellan respektive fasledare och
nollledaren kallas för fasspänning.
Bruksföremålen i huset ansluts oftast 1fasigt, d.v.s. mellan någon av fasledarnaoch
nolledaren. Någorlunda lika belastning mellan faserna är önskvärd. Mer effektkrävande
apparater som el-pannor och spisar ansluts
därför till alla tre faserna (3-fasigt). Amatörradioutrustningar ansluts oftast 1-fasigt.

Strömbrytare

Kraftförsörjningen av radiostationens apparater bör ske över en gemensam huvudströmbrytare, som lätt kan nås. En indikatorlampa får gärna markera att den brytaren är
tillslagen och att stationen är under spänning.
Informera familjen och övriga i din omgivning om hur den brytaren fungerar. Det är en
säkerhetsåtgärd om något skulle hända.
Apparaternas nätströmbrytare skall vara
utförda för den aktuella arbetsspänningen
och ha ett godkänt utförande.
Vid 1-fassystem skall nätströmbrytaren i
apparaterna vara 2-polig och bryta fas- och
N-ledare, men aldrig PE-Iedaren.
Vid 3-fassystem skall nätströmbrytaren
vara 3-polig och bryta fasledarna, men aldrig
N-ledare och PE-Iedare.

Kom ihåg, att behörig installatör skall
anlitas vid ingrepp i fasta installationer.

1110-2

FARLI
liten terminologi vid elinstallationer
•
•

•
•

•
•
•

Gruppcentral
Den säkringscentral som följer efter elmätaren, t.ex. i villor och lägenheter.
Gruppledningar
Ledningar efter en gruppcentral, d.v.s.
ledningar till belysning, el-spisar, uttag
m.m.
Fasledare
En ledare som för fasspänning.
Nolledare (N-ledare)
En ledare som är ansluten till elnätets s.k.
nollpunkt (nollskena) och som normalt
inte skall föra spänning till jord.
Skyddsledare (PE-Iedare)
De ledare i kablar och sladdar, som är
speciellt avsedda för skyddsjordning.
Bruksföremål
Ett i princip flyttbart elanslutet föremål,
t.ex. handverktyg och radioapparater.
Förstärkt isolering
Vissa bruksföremål tillverkas med en så
god isolering att de inte behöver skyddsjordas. Så isolerade får anslutningsledningen förses med en speciell stickpropp,
som passar i vägguttag, såväl med som
utan jorddon. Sådana bruksföremål är
märkta med symbolen IQI och får inte
ändras så att de kan skyddsjordas.

Färgkoder för fasa, noll- och skydds ledare.
Isoleringsmaterialet omkring gruppledarna i
fasta elinstallationer har färger som fyller en
viktig funktion. Dessa färger får därför aldrig
förväxlas.
Fasledaren har i regel svart färg. N-ledaren (nollan) har b lå färg.
Det är till fas- och N-ledarna i vägguttagen, som man kopplar apparaterna för att få
ström. Helst skall uttagen också ha jorddon
d.v.s. en extra kontakt- ett s.k. jordningsbleck. Detta bleck är anslutet till PE-ledaren
(skyddsjorden), som är färgad randigt gult/
grönt.
En gullgrön ledare är alltid en skyddsjordledare och får endast användas för det.
l äldre installationer kan emellertid skyddsledarens isolering vara t.ex. röd.

STR

MEN

Uttag och stickproppar med jorddon
Jorddonet ger förbindelse med elsystemets
skyddsjord (PE).
Det är rummets utförande, som avgör om
vägg- och lamputtagen där skall ha uttag
med jorddon. Bostadsrum är klassade som
inte särskilt riskfyllda och har därför tidigare
inte försetts medlamp-och vägguttag med
jorddon. Vid nybyggnation är emellertid numera alla uttag försedda med jorddon!
Kök och tvättstugor med ledande plåtbänkar, vattenkranar o.s.v. anses som riskfyllda rum och måste ha uttag med jorddon.
Samma gäller källare och liknande andra
rum med ledande golv, väggar och inredningar.
Det är tillrådligt att installera uttag med
jorddon för radiostationen. Observera då, att
alla uttag i det rummet skall ha jorddon!
skyddsjordning
Att jorda är det vanliga uttrycket för att ansluta ett föremål till ett jordtag. Metallhöljen
på apparater kan av olika anledningar bli
spänningsförande och är då en elsäkerhetsrisk. För att säkert ha nollpotential på höljena
kan de kopplas till jordskenan via PE-Iedaren- d.v.s. skyddsjordas. När man ansluter
apparathöljet till jorddonet, kommer
säkringen att bryta strömtillförseln om det
blir isolationsfel mellan en strömförande del
och höljet. PE-Iedaren får därför aldrig brytas!
Om skyddsjordning finns särskilda föreskrifter. Om du inte är säker på hur skyddsjordning skall utföras, fråga en behörig installatör.

Jordfelsbrytare
Jordfelsbrytare kallas en brytare som automatiskt bryter spänningen, när det uppstår
överledning till skyddsjordade detaljer- s.k.
jordfeL Brytaren mäter felströmmen och bryter spänningen innan strömmen uppnår ett
farligt värde, t. ex. 1O mA. Jordfelsbrytare får
inte ersätta skyddsjordning, men kan under
särskilda förutsättningar komplettera
skyddsjordningen som en extra säkerhetsåtgärd. Låt installera jordfelsbrytare l

111 o- 3

F RLI ASTR MM
Särjordning

Särjordning är ett uttryck för att jorda apparater till en separat jordpunkt, Det görs via
separat jordlina till ett jordtag, d.v.s. jordplåt
eller jordspett Särjordning skall ske på rätt
sätt eftersom det avsedda skyddet annars
kan bli en fara.
Särjordning får ske endast om skyddsjordning till PEN också har gjorts. Om du har
planer på särjordning, fråga en behörig installatör

Jordning av antennsystem

l brist på annan jordpunkt är det frestande att
ansluta antennjordledaren till PE-Iedarens
anslutningsbleck i vägguttaget med förhoppning att på så sätt få ett bättre HF-jordplan för
antennen. Detta är emellertid ett dåligt exempel påsärjordning, som både kan innebära säkerhetsrisker och medföra störningsproblem.

Snabba och tröga säkringar

Det finns snabba och tröga säkringar. Snabba
säkringar är det som normalt används. Tröga
säkringar för samma strömstyrka kan behövas för apparater som har speciellt hög startström, t.ex. stora nättransformatorer med
toroidkärna. Säkringarna skall kunna bryta
tillräcklig hög spänning, annars blir det en
kvarstående ljusbåge i dem vid säkringsbrott. Använd säkringar med rätta strömvärden. Det är förbjudet att laga säkringar,
vilket naturligtvis kan orsaka både brand och
andra faror.

\section{Faror}

Överhettning

Elektricitet kan lätt vålla både personskador
och materiella skador. Det är viktigt att veta
hur skador kan undvikas. Elektrisk utrustning skall vara beröringsskyddad med fullgod kapsling. Samtidigt får värmen inne i
kapslingen inte bli så hög att det innebär
brandrisk. Spontana fel kan trots allt uppstå.
Isolationsfel medför risk vid beröring och
brand kan utvecklas snabbt. När utrustning
under spänning lämnas obevakad, skall det
ske med särskild aktsamhet.
Hur elektriska apparater och anläggningar får utföras, regleras av lagar och föreskrifter. Elektriska apparater skall uppfylla
vissa krav för att få marknadsföras och användas. Utförande och ursprung skall vara
dokumenterat på föreskrivet sätt.
Även självbyggda apparater skall uppfylla kraven på elsäkerhet- d.v.s. säkerhet
mot e Ichock och brand - och byggaren bär
ensam ansvaret för att utförandet och
hanterandet av apparaterna är betryggande.
Den som bygger och använder en elektrisk apparat bör därför ha nödvändiga kunskaper om elsäkerhet.

Höga spänningar
Ingrepp i elektriska apparater under spän-

ning innebär personfara. Öppna aldrig en
apparat om spänningen är tillslagen. Vid
ingrepp t.ex. i sändare, mottagare och
kraftförsörjningsaggregat är det lätt att utsätta sig för höga likspänningar. l sändare
med elektronrör förekommer spänningar i
storleksordningen hundratals till tusentals
volt. Så är det också i bildskärmar.
Observera att även apparater som drivs
med batteri eller ackumulatorer kan innehålla kretsar som omvandlar den låga spänningen till direkt livsfarlig hög spänning. Exempel på det är likspänningsomvandlare.

Höga strömmar

Höga strömmar ger häftiga muskelkramper
och brännskador. Man vet att det skiljer
mellan skador av lik- respektive växelström.
Lågfrekvent växelström ger upphov till
muskelkramper, som kan göra det omöjligt
att släppa det strömförande föremålet.

111 o- 4

MMEN
Högfrekvent växelström i MHz-området
värmer upp kroppsvävnaderna, snarare än
att förorsaka muskel reaktioner.
Likström påverkar kroppen annorlunda
än växelström. Genom det elektriska motståndet i kroppens vävnader och vätskor
utvecklas det värme. Detta kan leda till brännskador både på huden och inne i kroppen.
Om likströmmen pulserar uppstår dessutom
muskelreaktioner på liknande sätt som vid
växelström.
Höga spänningar är alltid farliga. Det är
däremot inte så känt att även låga spänningar kan vara det. Ackumulatorer och anslutna apparater kan ge ifrån sig höga strömmar även om spänningen är låg. Oavsiktliga
strömvägar t.ex. kortslutning genom en
klocka eller fingerring kan medföra allvarliga
brännskador.
Antenner
Placera helst antennerna utom räckhåll för
obehöriga. På sändarantenner kan det nämligen uppstå höga H F-spänningar redan vid
låg sändareffekt. HF bränns vid beröring och
en reflexrörelse gör det lätt att tappa balansen och falla. Sätt gärna upp skyltar på eller
invid antennerna, med varning för högfrekvent spänning samt uppgift om ägarens
namn, adress och telefonnummer.
Antenner får inte korsa eller placeras
nära högspännings-, lågspännings- eller
telefonlinjer. Det är en olycksrisk om antenner och kraft- eller teleledningar av någon
anledning slår ihop. Det är också en olycksrisk om antenner faller ner över dessa ledningar.
Endast efter tillstånd från berörd myndighet
och/eller linjeägare får man dra ledningar av
något slag över väg eller offentlig plats även antenner.
Höga likspänningarfrån sändaren får inte
komma ut i antennen. Se till att antennernas
matarledningar är kopplade till god likströmsjord via HF-drosslar eller försedda med
överspänningsavledare. Som extra säkerhetsåtgärd bör sändaren anslutas till antennledningen över en stor kondensator.
Undvik att beröra antenner utan att de
jordats, särskilt vid vistelse på tak eller i träd.

Under åskväder, snöfall, regn eller dimma
då laddade partiklar är i rörelse, kan antennerna laddas upp till höga statiska spänningar. Arbetar man då med antennen kan man
överraskas av en elektrisk stöt. Det är då lätt
hänt att tappa taget och falla ner.
Ae~sti«:U'le!lnn·Hl i kondensatorer
Kondensatorer kan behålla en betydande
restladdning under många timmar sedan
kraften brutits.
Koppla urladdningsresistorer (bleeder)
över filterkondensatorer, så att de laddas
ur när matningen stängs av. Av säkerhetsskäl skall urladdningsresistorerna tåla
fyra gånger så stor effekt som de själva
förbrukar under drift.
• Varning: När du laddar ur en kondensakortslut den inte! Använd en resistor!
@l

säkerhetsåtgärder
Transformator med förstärkt säkerhet
• Om du är osäker på det elsäkerhetsmässiga utförandet på en apparat, t.ex. en
gammal sändare, använd då en skiljetransformator (fulltransformator) - helst
av klass Il (extraisolerad).
Vid reparation skall utrustningen vara spänningslös. Före arbetet skall du
e
Stänga av utrustningens nätströmbrytare,
• Dra ur stickproppen ur vägguttaget (dubbel säkerhet),
Om trimning eller felsökning måste ske under spänning skall följande iakttas:
• Arbeta inte med anläggningen när du är
trött eller omotiverad. Då är du minst
vaksam mot olyckor.
Se till att du inte får ström genom kropArbeta helst bara med höger hand
och håll den andra borta från den utrustning som du arbetar med. Stoppa gärna
den fria handen i fickan!
Ha inga hörtelefoner på huvudet. Använd
högtalare om du trimmar med hörseln.
®
Helst bör någon finnas i närheten när du
arbetar i ap pararter under spänning. Visa
var nätströmbrytaren sitter. Se gärna till
att han/hon kan elolycksfallshjälp.
@

@

111 o- 5

FARLI A STR MMEN
Vid arbete med ackumulatorbatterier
• Trots att spänningen är låg kan ackumulatorbatterier lämna mycket höga strömmarvid kortslutning. Tag därför av fingerringar, armbandsur m.m. Använd isolerade verktyg vid arbete med batteripoiskor.
• Akta dig för elektrolyten i ackumulatorbatterierna- den är starkt frätande.
Varning för explosionsrisk av knallgas
och syrastänk i ögonen.

PT
\section{Åska}
Faror

Vid åska utvecklas det mycket starka,
elektromagnetiska fält, som breder ut sig
och alstrar mycket korta spänningsstötar i
alla metallföremål, t.ex. i antenner. stötarna
vandrar genom kablarna in i apparaterna. Är
stötspänningen tillräckligt hög, kommer saker i strömvägen att förstöras på något sätt.
Förbränning och nersmältning är vanligt.
Men om åskurladdningen sker på långt håll,
kan stötspänningen bli så låg att man någorlunda kan undgå skada på apparater och
hus. Om åskurladdningen däremot sker
mycket nära antennen eller som direkt nerslag, då uppstår definitivt stora skador.

Skydd och jordning

Antenner och antennkablar kan man aldrig
skydda mot åsknerslag. De är ju till sin natur
en slags åskledare. Det man kan försöka att
göra är att leda en eventuell åskurladdning i
ett antennsystem bort från hus och människor. Observera, att man inte får "haka på"
husets ordinarie åskledare. Då gäller inte
husförsäkringen.
Antennkabel n, som fungerar som en (för
klent dimensionerad) åskledare, skall naturligtvis INTE l ONÖDAN dras in i huset utan
kortaste vägen utanför huset till en avgrening.
Från avgreningen fortsätter dels kabeln
in till apparaterna genom ett överspänningsskydd och dels en jordlina kortaste vägen
ner till jordtaget över en gniststräcka. Det
bästa sättet att skydda apparaterna mot
åska är fortfarande att koppla bort dem helt
från antennkabeln och vägguttag.
Om man bor i ett hyreshus är det tyvärr
oftast svårt att få vidta åtgärder som dem
härovan. Då får man nöja sig med att koppla
bortantennledningarna från apparaternaoch
lägga dem väl åt sidan - gärna utanför
husväggen.
Som permanent, men otillräckligt skydd
kan man förse de olika anslutningsställena
med lämpliga överspänningsskydd.
Att hoppas på skydd mot åsknerslag genom jordning i elsyste m et är naturligtvis helt
befängt!

1110-6

\part{REGLER OCH TRAFIKMETODER}

Naturlagar begränsar frekvensområdet för
radiosändningar. Alla radiosändningar måste
ske inom samma utrymme. En del sändningar är riktade till många lyssnare. Andra
sändningar är mellan två personer. Ingen
part vill bli störd av en annan. Intressena är
många. Eftersom radiosändningarna blir fler
och alla tar ett utrymme i anspråk, så är det
nödvändigt att de görs på ett frekvenseffektivt
sätt.

De frekvensband som tilldelas amatörradion är sålunda överenskomna vid internationella konferenser, där teleadministrationer och radiotjänster jämkar samman sina
intressen. Radioamatörerna, representerade av IARU, verkar därvid för sin tilldelning
av önskvärda frekvensområden. De svenska föreskrifterna för amatörradio påverkas
alltså påtagligt av internationella intressen
och överenskommelser.

För att styra detta samråder ländernas
administrationer och radiotjänster om hur
frekvenserna och utrymmet för radiokommunikation skall fördelas och användas.
Överenskommelserna omfattar inte bara
frekvenstilldelning utan även prioriteringar
om nyttjanderätt, sändningsslag, effekter,
räckvidder m.m.

Den internationella telekonventionen ITC - är den överenskommelse på vilken
verksamheten inom den internationella teleunionen -ITU- bygger. Gällande konven-

Liksom det finns lagar och trafikbestämmelser för flyg, sjöfart och landtrafik så regleras sedan mycket länge även radiotrafik av
alla de slag. Utöver nationella regler finns
det mellanstatliga (bilaterala), regionala och
internationella överenskommelser om radiotrafik. Detta gäller även amatörradiotrafik,
som är en internationell radiotjänst.

tion antogs 1982 och ratificerades (lagfästes) av Sverige 1985 (SÖ 1985: 66). Konventionen kompletteras av det internationella Radioreglementet (RR), vilket omfattar
huvudregler som överenskommits mellan
alla länder inom ITU.
Konventionen och radioreglementet är
bindande för alla stater som ratificerat
konventionen. De avsteg, som ett land vill
göra och övriga länder godtar, skrivs in som
s.k. fotnoter. Radioreglementet omfattar alla
radiotjänsters verksamhet, däribland Amatör- och amatörsatellittjänsterna.

OBSERVERA!
Som radioamatör är Du skyldig att följa gällande bestämmelser
för amatörradioanvändning i det land som Du vistas i.
Förvissa Dig om att Du har senaste utgåvan!
Vid osäkerhet- rådfråga PTS!

1111 -1

RE LE
III 1.. Nationella och internationella trafikbestämmelser
och procedurer
1.1 Fonetiska alfabeten
Ibland behöver man göra förtydliganden genom att bokstavera.
Svenska radioamatörer skall kunna två fonetiska alfabeten.

Det internationella fonetiska alfabetet

A

svenska

alfabetet

ö

Alfa
Bravo
Charlie
Delta
Echo
Foxtrot
Golf
Hotel
lndia
Juliett
Kilo
Lima
Mike
November
Oscar
Papa
Quebec
Romeo
Sierra
Tango
Uniform
Victor
Wiskey
X-ray
Yankee
Zulu
Alfa Alfa
Alfa Echo
Oscar Echo

ALL FA
BRA VO
TJAR Ll
DELL TA
ECK Å
FACKS
GÅLF
HÅ
IN DIA
DJO U
Kl LÅ
U MA
MAJK
NO VEM BÖ(RR)
ÅSSK A(RR)
PA PA
KE BECK
RÅ MIO
Sl ERR RA
TÄNG GÅ
JO NI FORM
VICK TÖ(RR)
OISS Kl
ECKS REJ
JÄNG Kl
LO
FA ALL
ALL FA ECK
ÅSSK A -

A
B
C
D
E
F
G
H
l
J
K
L
M
N
O
P
Q
R
S
T
U
V
W
X
Y
Z
Å
Ä

ö

Adam
Bertil
Cesar
David
Erik
Filip
Gustav
Helge
Ivar
Johan
Kalle
Ludvig
Martin
Niklas
Olof
Petter
Quintus
Rudolf
Sigurd
Tore
Urban
Viktor
Wilhelm
Xerxes
Yngve
Zäta
Åke
Ärlig
Östen

O
1
2
3
4
5
6
7
8
9

Zero
One
Two
Three
Four
Five
Six
Seven
Eight
Nine

ZE RO
O ANN
TO
TRI
FÅR
FAJV
SICKS
SE VEN
EJT
NAJ NÖ(RR)

O
1
2
3
4
5
6
7
8
9

Nolla
Ett (inte ETTA)
Tvåa
Trea
Fyra
Femma
Sexa
Sju (inte SJUA)
Åtta
Nia

B

C
D
E
F
G

H
l

J
K

L

M

N
O

P
Q
R
S

T
U
V

W
X
Y
Z
Å

Ä

Decimal
DE Sl MAL
Stop
STOPP
Ungefärligt uttal. Betona det understrukna.
1111 - 2

-

Komma
Punkt
Frågor kan förekomma i reglementsprovet.

R CH TRAFIKM
1m2 Q-koden
Bakgrund
Vid sändning med morsetelegrafi används
sedan år 1912 internationella "trafikförkortningar" enligt Q-koden, både för att minska
risken för mottagningsfel på grund av språksvårigheter, störningar m.m. och för att minska sändningstiden. En trafikförkortning i form
av Q-kod har en entydig innebörd, men kan
anpassas något till aktuell situation. Varje Qkod består av tre bokstäver i bokstavsserien
QAA- QZZ.

l CEPT-rekommendation T/R 61-02
nämns följande allmänna Q-förkortningar
som berör amatörradio.
Radioamatörerna använder emellertid i praktiken fler Q-förkortningar än
dessa. En lista kan beställas från SSA:s
kansiL

i reglamentsprovet för radioamatörcertifikat ingår frågor om Q-förkortningar.

Användning
1. Vissa Q-koder kan ges jakande betydelse
genom att bokstaven C (vid telefoni uttalad som CHAR U E) sänds omedelbart
efter förkortningen eller ges nekande betydelse med det engelska ordet NO omedelbart efter förkortningen.
2. Q-koder kan kompletteras med andra
lämpliga förkortningar, anropssignaler,
frekvenser, tidsuppgifter, person- och
ortsnamn, siffror, nummer o.s.v. l den beskrivande texten för vissa Q-koder lämnas inom en parentes plats för ytterligare
uppgifter. Dessa uppgifter skall då sändas i den ordning som anges i texten.
3. Q-koderna antar formen av fråga, då de
vid radiotelegrafering åtföljs av frågetecken liksom då de vid radiotelefonering
åtföljs av bokstäverna RQ (ROMEO
QUEBEC). När kompletterande uppgifter följer efter en uttalad fråga, skall ett
frågetecken respektive RQ följa efter
uppgifterna.
4. Q-koder med numrerade alternativa betydelser skall åtföljas av motsvarande
siffra. Siffran skall sändas omedelbart
efter förkortningen.
5. l internationell radiotrafik skall, då ej annat anges, tidpunkter anges i Universal
Time Coordinate (UTC) i stf. det tidigare
Greenwich Mean Time (GMT). Tidsformatet är fyra siffror, vilket även är militär
standard.

Q-kod

Fråga

Svar eller meddelande

QRK

Vilken uppfattbarhet har mina
(eller: ............. *:s) signaler?

Uppfattbarheten hos Dina
(eller ......... *:s) signaler är ... .
1. dålig
2. bristfällig
3. ganska god
4.god
5. utmärkt.

QRM

Är min sändning störd?

Störningarna på Din sändning är
1. obefintliga
2.svaga
3. måttliga
4. starka
5. mycket starka.

1111-3

R
QRN

Besväras Du av atmosfäriska
störningar?

Atmosfäriska störningar är
1. obefintliga
2.svaga
3. måttliga
4. starka
5. mycket starka.

QRO

Skall jag öka sändningseffekten?

Öka sändningseffekten.

QRP

Skall jag minska sändningseffekten?

Minska sändningseffekten.

QRS

Skall jag minska sändningshastig heten?

Minska sändningshastigheten
(sänd ...... ord i minuten).

QRT

Skall jag avbryta sändningen?

Avbryt sändningen.

(QRU) Har Ni något till mig?

Jag har inget till Dig.

QRV

Är Du redo?

Jag är redo.

QRX

När anropar Du mig igen?

Jag anropar Dig igen kl ... på ... kHz/MHz.

QRZ

Vem anropar mig?

Du anropas av ......... * (på ....... kHz/MHz).

(QSA)

Vilken styrka har mina
(eller: .... *:s) signaler?

Dina (eller: ...... *:s) signaler är
1. knappast uppfattbara
2.svaga
3. ganska starka
4. starka
5. mycket starka.

QSB

Varierar min signalstyrka?

Din signalstyrka varierar.

QSL

Kan Du ge mig kvittens?

Jag kvitterar.

QSO

Kan
få förbindelse med
.... * direkt?

Jag kan få förbindelse med .... * direkt.

QSY

Skall jag gå över till annan frekvens? Gå över till annan frekvens.

(QTC)

Hur många telegram har du att
sända?

Jag har telegram till Dig.

QTH

Vilket är Ditt geografiska läge?

Mitt geografiska läge är ....... .

QTR

Kan Du ge mig rätt tid?

Rätt tid är ........

* namn och /eller anropssignal

1111 - 4

1.3 Trafikförkortningar, vanliga i amatörradio
Utöver Q-koden och klartext används vid
morsetelegrafering även andra trafikförkortningar. Eftersom det internationella radiospråket är engelska, är förkortningar av engelska ord vanligast.
Förkortningar bör emellertid inte användas i onödan. En ovan operatör vid motstationen kan då få svårt att förstå meddelandet.

Urval för radioamatörer
l CEPT-rekommendation T/R 61-02 nämns
utöver Q-koden följande övriga trafikförkortningar, som berör amatörradio.
Radioamatörerna använder i praktiken
många fler trafikförkortningar än dessa. En
lista kan beställas från SSA:s kansli.

Ett exempel på en avsnitt ur en amatörradiosändning, där trafikförkortningar används
särskilt flitigt:
"gm es tnx vy much om fer ur rprt. u are
cmg in hr ufb. my tx is .... and rx .... anta 3
el beam . condx hr gud mni dx stns hrd . wl
nw nil so tks es 73 "
l klartext ser exemplet ut så här:
"good morning and thank you very much
Old Man for your report. You are coming in
here ultra fine business. My transmitter is .....
and receiver .. ... antenna is a 3 element
beam. Conditions here are good many
stations heard. Weil now nothing for you so
thanks and kindest regards "

l reglementsprovet för radioamatörcertifikat ingår frågor om trafikförkortningar.
Förkortning Engelskt uttryck

Svensk betydelse

BK
CQ

avbryt(-a) (sändningen)
allmänt anrop, till alla
telegrafi (A 1A)
från ..... (anropssignal)
"kom"
meddelande, telegram
var god (att .... )
allt uppfattat, mottaget
mottagare
sändare
din, ditt, dina, er

cw

DE
K
MSG
PSE
R
RX
TX
UR

break
"seek you"
continous waves
franska "de"
come
message
please
received
receiver
transmitter
your

Utöver ovanstående trafikförkortningar upptas i CEPT-rekommendationen även följande bokstavskombinationer, vilka används i
teleprintertrafik i stället för motsvarande
morsetecken, slagna utan tecken mellanrum.
(Strecket ovanför bokstäverna betecknar
att det inte finns något mellanrum).

Vidare upptas i CEPT -rekommendationen bokstavskombinationen RST som en
trafikförkortning. Denna får tydas som en
fråga eller anmodan om signal rapport.
Mer om detta under 13. stationsdagbok
och Appendix J.

AR
sluttecken
+
VA eller SK avslutningstecken @

1111 - 5

RE LE
1.4 Internationell nödtrafik och trafik vid naturkatastrofer
Nödsignaler
l ITU Radioreglemente (RR) framgår av Artikel 39 om "Distress Communications" hur
nödsignaler skall vara formulerade. Dessa
signaler är internationella och sänds när ett
skepp, flygplan eller annan farkost hotas av
allvarlig och omedelbarfara och begär hjälp.
Nödsignalen på morsetelegrafi består av
teckendelarna -- .. - - - .. -- sända i en följd,
där längden på de långa teckendelarna betonas så att de klart skiljer sig från de korta.
Signalen skrivs som bokstäverna SOS med
ett streck ovanför.
Nödsignalen på radiotelefoni består av
ordet MAYDAY uttalat som det franska uttrycket "m'aider".
På amatörradiofrekvenserna förekommer
även CQ EMERGENCY som internationellt
nq~anrop. l Sverige kan man även ropa
NODANROP på svenska.
Nödtrafik
Vid 1979 års världsradiokonferens (WARC)
antogs bl.a. Resolution 640, vilken avsåg
internationell radiokommunikation på frekvensband upplåtna åt amatörradion, i händelse av naturkatastrofer.
Resolutionen är inskriven i RR. l CEPTrekommendation T/R 61-02 nämns beslutspunkterna 1-5. För orientering återges resolutionen med dessa punkter kursiverade.
"l betraktande av
a. att i händelse av naturkatastrofer de normala kommunikationssystemen ofta är
överbelastade, skadade eller helt avskurna
b. att snabbt upprättande av kommunikationer är absolut nödvändigt för att möjliggöra världsomspännande hjälpaktioner
c. att amatörbanden inte är bundna av fasta
bandplaner eller kungörelser och därför
är vällämpade för korttidsanvändning vid
nödtillfällen
d. att internationell nödtrafik kan underlättas genom tillfällig användning av vissa
frekvensband upplåtna åt amatörradiotrafiken

1111 - 6

e. att i sådana situationer amatörradiostationer p.g .a. deras stora geografiska spridning och påvisade kapacitet kan hjälpa till
att upprätthålla viktiga radioförbindelser
f. att det existerar nationella och regionala
amatörnödtrafiknät som använder frekvenser inom gällande bandplan för amatör rad iot ra f ik
g. att i händelse av en naturkatastrof direkt
förbindelse mellan amatörradiostationer
och andra radiotjänster möjliggör överförandet av livsviktiga meddelanden tills
normala radioförbindelser åter kan upprättas.
Med insikt om att befogenheter och ansvar
för sådan radiotrafik vid naturkatastrofer vilar på berörda länders myndigheter, beslutar
konferensen
1. att frekvensbanden specificerade i No.
51 O *får användas av myndigheter inom
ramen för internationell nödtrafik
2. att sådan användning av amatörbanden
skall begränsas till nödtrafik i samband
med naturkatastrofer
3. att i sådana fall trafiken av icke-amatörradiostationer skall inskränkas till nödtillfället inom det speciella område som
anges av resp myndighet i det drabbade
landet
4. att nödtrafiken skall äga rum inom
katastrofområdet och mellan detta och
vederbörande hjälporganisations högkvarter
5. att sådan nödtrafik endast får upprättas
efter medgivande ifrån det drabbade landets myndighet
6. att nödtrafik ifrån länder utanför inte får
upphäva redan befintliga nationella eller
internationella amatörnödtrafiknät
7. att ett nära samarbete mellan amatörradiostationer och andra radiotjänster,
som i en framtid kan finna det nödvändigt
att för nödtrafik använda amatörbanden,
är önskvärt
8. att vid sådan internationell trafik såvitt
möjligt skall undvikas att störa amatörradiotrafik.

ET DER
Anhåller konferensen hos myndigheterna**
1. att skapa sådana förutsättningar som tillåter genomförandet av internationell nödtrafik
2. att i sina radioreglementen upptaga föreskrifter som tillåter genomförande av nödtrafik."

Om Du hör en nödsignal på radio

* 51 Oär den fotnot i frekvenstilldelningstabellen, som hänvisar till Resolution 640.
De amatörradioband som specificeras för
användning i händelse av naturkastrater
är 3.5, 7.0, 1O. i, i 4.0, i 8.068, 21.0, 24.89
och 144 MHz.
** Med myndigheterna avses respektive
lands teleadministration.

Du själv sänder nödsignal över radio

Avbryt omedelbart din egen sändning när du
hören nödsignal. Lyssna på nödmeddelandet
och SKRIV NER vad som sägs. Notera position, frekvens, tidpunkt etc. Anmäl vad du
hört på följande sätt.

Uppträd lugnt och sansat, när du kallar på
hjälp över radion. Tänk först och sänd sedan. Som ovan sagts måste den som svarar
dig och sedan ringer 112 (förut 90000) meddela larmoperatören att Ditt nödanrop kommit via radio.

Nödsignal från radioamatör i utlandet
Nödsignal från en radioamatör i ett katastrofområde utomlands ska anmälas till UD, d.v.s.
Utrikesdepartementet.
På dagtid kl. 8 - 17
te l. 08-40 55 950.
te l. 08- 40 55 001 .
På övrig tid

Nyckelordet för dina åtgärder är LARMA:
läge
Ange olycksplatsens läge. Du kan
ange gatu- eller vägnamn eller
riktmärken som t.ex. vägkorset,
gränsen, bron, järnvägen etc.
Analysera Gör en överblick över olycksplatsen och tala om vad som hänt.
Några skadade? Några innestängda? Brinner det? Släpps
farliga ämnen ut?
Ropa
Ropa på hjälp. Använd gärna en
repeater på 2-metersbandet så
att du når många, men även andra frekvenser kan användas.
Anropamed NÖDANROPFRÅN
SMXxxx. Fråga efter någon med
telefon. Ge inte upp om du inte får
svar genast.
Meddela Meddela när du fått kontakt med
någon med telefon, sänd NÖDTRAFIKPÅGÅR för att freda frekvensen och NÖDMEDDELANDET med de viktigaste uppgifterna. Begär att uppgifterna repeteras och ta löfte på att de sänds
vidare. Begär att få veta när så
har skett. Påminn annars!
Avvakta Vänta på platsen tills hjälp har
anlänt. Passa radion så att du
kan svara på frågor. Behövs inte
lätJgre din hjälp, avsll!~a då m~d
NODTRAFIK UPPHOR FRAN
SMXxxx .. KLART SLUT.

Nödsignal från svenskt landområde
l Sverige bör du ringa 112 (förut 90 000) för
att kalla på Ambulans, Polis, Räddningskår,
Sjöräddning, Flygräddning etc. Ditt telefonnummer visas automatiskt i larmoperatörens display.
För att undvika missförstånd och feldirigering av räddningsinsatserna MÅSTE
du meddela operatören att nödanropet kommit via radio. Själva olycksplatsen kan ju
ligga i ett helt annat riktnummerområde, än
som både Ditttelefonsamtal och nödanropet
kommer ifrån.
Nödsignal från fartyg eller luftfarkost
Om nödsignalen inte besvaras av någon
kust- eller markstatio n, ring 1 i 2 (förut 90000)
och begär Sjöräddning respektive Flygräddning och meddela dina iaktagelser. Du
kan även rapportera direkt till centralerna
Sjöräddning i Stockholm 08 - 601 79 00,
Sjöräddning i Göteborg 031 - 64 80 20 och
Flygräddning 031 - 64 80 00.
Vidarebefordra nödmeddelandet utan att
ändra på det!

1111 -7

R

l

H TRAFIKMET DER

~©~

PT

1.5 Anropssignaler
Anropssignalernas sammansättning
Varje land har unika anropssignaler för all
sin radiotrafik. Signalerna utformas enligt
radioreglementet (RR) på sätt, som beror på
syftet med varje särskild radiostation. l RR
finns definitioner för olika slags stationer,
t. ex. stationer för fast radio, landmobila stationer, stationer i fartyg, i sjöräddningsfarkoster, i flygplan, amatörradiostationero.s.v.

Identifiering av amatörradiostationer
En radiostation skall identifieras med den
anropssignal, som tilldelats av det egna landetsteleadministration (myndighet).! Sverige
är det Post- och telestyrelsen (PTS).
Anropssignalen meddelas i det tillstånd för
innehav och användning som utfärdats för
tillståndshavaren ifråga. Signalen gäller så
länge som tillståndet är giltigt.

Amatörradiosignaler är uppbyggda på följande sätt:
Antal kombinationer Anmärkningar
Teckenkombinationer
YOA - Y9Z
260
Första tecknet "Y" räcker ensamt
YOAA - Y9ZZ
6760
som nationell identitet om det är
YOAAA- Y9ZZZ
175760
B, F, G, l, K, M, N, R eller W.
XXOA - XX9Z
260
Signaler som börjar med en siffra,
XXOAA - XX9ZZ
6760
när andra bokstaven är O eller l,
XXOAAA - XX9ZZZ
175760
är dock ej tillåtna för amatörradio.
(XX är det två första tecknen i en tilldelad signalserie)
Sverige är tilldelat teckenkombinationer i serierna SAA - SMZ, 7SA - 7SZ och 8SA - 8SZ.
Anropssignalerna för svenska amatörradiostationer är uppbyggda på följande sätt, varvid
med distrikt avses amatörradiodistrikt
Amatörradiotillstånd (CEPT-tillstånd) för
SM + distriktssiffra + bokstäver,
radioamatörer
SK + distriktssiffra + bokstäver,
amatörklubbar
militära förband
SL + distriktssiffra + bokstäver,
Sl + distriktssiffra + bokstäver
(specialtillstånd),
amatörklubbar
amatörklubbar
SJ + distriktssiffra + bokstäver
(specialtillstånd),
7S + distriktssiffra + bokstäver
amatörklubbar
(specialtillstånd),
8S + distriktssiffra + bokstäver
amatörklubbar
(specialtillstånd),
SSA-tillstånd inom SSAs utbildningsverksamhet
SH + distriktssiffra + bokstäver (AAA- CZZ).
Sverige är indelat i amatörradiodistrikt med följande numrering och utsträckning:
Distrikt Utsträckning
o Stockholms (AB) län
1
Gotlands (l) län
2
Västerbottens (AC) och
Norrbottens (BD) län
Gävleborgs (X), Jämtlands (Z) och
3
Västernorrlands (Y) län
4
Örebro (T), Värmlands (S) och
Kopparbergs (W) län
5
Östergötlands (E), Södermanlands
(D), Västmanlands (U) och Uppsala
(C) län

1111-8

Distrikt Utsträckning
Hallands (N), Älvsborgs (P), Göte6
borgs och Bohus (O) län samt Skaraborgs (R) län
7
Malmöhus (M), Kristianstads (L),
Blekinge (K), Kronobergs (G),
Jönköpings (F) och Kalmar (H) län.
Distriktssiffran i signalen bestäms av det
län som hemadressen är belägen inom. Vid
sändning utanför hemadressen bör det
framgå av tillägg till signalen.

LER
l Post- och telestyrelsens föreskrifter sägs
dock inte vilken distriktssiffra som skall användas, när sändning sker från annan plats
än hemortsadressen.
Med stöd av praxis rekommenderar dock
SSA att följande regler tillämpas:
e
Vid trafik från en regelbundet använd
fritidsbostad kan i anropssignalen användas den distriktssiffra som utvisar var
fritidsbostaden är belägen.
e
Vid trafik från annan tillfällig plats bör
anropssignalen åtföljas av snedstreck och
siffran för det distrikt varifrån sändningen
görs.
Exempel: SMOXYZ/0, SMOXYZ/6 etc.
e
Vid trafik från mobil station bör den ordinarie anropssignalen även åtföljas av /M.
Exempel: SMOXYZ/6M.
e
Vid trafik från mobil station inom hemorten kan dock den extra distriktssiffran
utelämnas.
Exempel: SMOXYZ/M.
e
Vid trafik från sjöfarkost bör den ordinarie
anropssignalen åtföjas av /MM.
• Vid trafik från luftfarkost bör den ordinarie
anropssignalen åtföljas av /AM.
e
Vid trafik från svensk farkost på internationellt territorium kan distriktssiffran 8
användas.
• Vid sändning från ett annat lands territorium gäller det landets bestämmelser.
Vid osäkerhet- Skaffa upplysningar från
SSA:s reciprokfuntionär!
Utländsk radioamatör på besök i Sverige
skall använda sin anropssignal från det egna
landet, föregånget av SM*l där * motsvaras
av siffran för det svenska distrikt varifrån
sändningen görs.

Användning av anropssignaler

Både motstationens och den egna anropssignalen skall användas i början och slutet
av varje sändning.
Under sändningen skall anropssignalen
upprepas "med korta mellanrum", utan närmare precisering av mellanrummet.
Även om man inte har kontakt med en
motstation, skall den egna anropssignalen
anges vid varje sändning.
Se vidare i PTS föreskrifter.

CH TRAFIKMET DER
Hur man genomför en radiokontakt

Det finns många sätt att genomföra en radiokontakt, men det finns några grundregler för
hur man uppträder och utväxlar samtal. Ett
trevligt och kamratligt uppträdande är en
hederssak inom amatörradion. Det behöver
inte bli stelt för den skull!
Allmänt anrop är ett sätt att kalla på någon
-vem som helst- att kommunicera med.
På telegrafi låter det så här:
de SMOXYZ K, d.v.s anropet
först och därefter den egna signalen.
På telefoni låter det så här:
Allmänt anrop, allmänt anrop, allmänt anrop
från SMOXYZ Kom. Glöm inte Kom i slutet!

ca ca ca

Riktat anrop gör man, när man vill tala med
någon särskild station. Då sänder man först
signalen på den station, som man vill tala
med och därefter sin egen signal.
På telegrafi låter det så här:
SMOÅÄÖ SMOÅÄÖ SMOÅÄÖ de SMOXYZ
SMOXYZ SMOXYZ K
På telefoni låter det så här:
SMOÅÄÖ SMOÅÄÖ SMOÅÄÖ från SMOXYZ
SMOXYZ SMOXYZ Kom
Motstationen svarar förhoppniong~vis på
anropet, alltså SMOXYZ från SMOAAO Kom.
Upprättad förbindelse. När en station svarat
på anrop, lämnar man först sin signalrapport
enligt RST-koden och presenterar sig med
sitt förnamn och var man finns. Motstationen
kvitterartroligen med sina motsvarande uppgifter. Varje gång, som man överlämnar ordet till motstationen säger man först motstationens signal och därefter sin egen. Därefter säger man Kom och lyssnar. Om man
har en telegrafiförbindelse och bara vill tala
med den stationen kan man sända KN (kom
du och ingen annan (nobody else).
Avsluta förbindelse. När man så småningom
avslutar kontakten tackar man för sig på och
utbytter avskedhälsningar.
Då kan det låta så här:
.......... Tack för en trevlig förbindelse och på
återhörande. SMOÅÄÖ från SMOXYZ. Klart
Slut.
Träna med din instruktör på att klara olika
slags trafiksituationer!

1111-9

REG
1.6 Bandplaner
IARU:s bandplaner, syfte och ändamål
Det allra vanligaste är att en radiostation eller
ett nät av stationer tilldelas en eller ett fåtal
frekvenser samt väl preciserade villkor i övrigt. Amatörradio är däremot en radiotjänst,
som tilldelas inte bara enstaka frekvenser
utan hela frekvensband samt inom dessa
band förhållandevis stor frihet till personligt
val av frekvens, sändningsslag etc.
Därvid kan den enskilde radioamatören
inte ställa anspråk på ostörda frekvenser. l
stället är det upp till radioamatörerna, att
själva samråda och rekommendera varandra om hur de tilldelade frekvensbanden bör
fördelas på olika slags användning. Denna
fördelning av trafiken kallas bandplan.

Internationella Amatörradiounionen IARU är det enda organ på internationell
nivå, där samråd om amatörradions intressen sker regelbundet, dels i arbetsmed olika inriktning och dels i
generella konferenser.
IARU har som syfte att
• verka för att av ITU tilldelade frekvensband för amatörradio bevaras,
• förbättra amatör- och amatörsatellittjänsternas status inom tilldelade
frekvensband,
• verka för tilldelning av ytterligare
frekvensband för amatörradio,
e
frekvensplanera amatörradiotrafiken
inom tilldelade amatörradioband genom samråd och rekommendationer.

Syftet med en bandplan är att ge utrymme för alla aspekter inom amatörradio självträning, kommunikation och tekniska
undersökningar.
Radioamatörernas bandplaner siktar på
att ge möjlighet till så många olika amatöraktiviteter som möjligt, såväl sändningsslag
som tekniker, både nu och i framtiden. För
att utnyttja banden på bästa sätt är det normalt att minsta möjliga bandbredd samt optimal sändarutrustning och teknik används.
För att alla skall kunna utöva amatörradio
med ett minimum av störningar, förutsätts att
man använder utrustningar som är "state of
the art".
God insikt i frekvensplanering, tillräckliga
resurser, gott anseende samt internationellt
samarbete behövs för att främja amatörradion. De flesta nationella amatörradioorganisationer har sedan många år ett världsomfattande samarbete genom sitt organ The
International Amateur Radio Union -IARUsom är organiserat som tre regioner. Dessa
regioner sammanfaller geografiskt med ITU :s
regioner. Region 1 omfattar Afrika, Europa
och västra Asien.

Svenska bandplaner, sändningsslag
Tilldelningen av frekvensband för amatörradioanvändning sker enligt överenskommelser mellan telemyndigheterna i de länder som är anslutna till ITU. Tilldelningen är
därvid i stort sett lika i de flesta länder. Av
olika skäl förekommer dock skillnader såväl
mellan ITU-regioner som länder.
l Sverige regleras amatörradioanvändningen främst genom Radiolagen och Postoch telestyrelsens föreskrifter. l anslutning
till frekvenstilldelningen anges tillåtna sändningsslag och amatörradiostatus i respektive band. Inom denna ram är det upp till
radioamatörerna själva att utnyttja sina möjligheter
bästa sätt.
bandplaner fungerar som radioamatörernas rekommendationer till varandra. Endast i minsta utsträckning medverkar
PTS till reglering inom dessa planer.

Se Appendix F

Se Appendix G.

llli- i O

Föreningen Sveriges SändareamatörerSSA - företräder de svenska radioamatörerna i IARU Region 1.

RE

R CH TRAFIKMET DER

III 2.. Nationella och internationella bestämmelser för
Amatör- och Amatörsatellittjänsterna
Tekniskt sett kan radioamatörerna världen
över, med hjälp av sina radiostationer, tämligen lätt skapa kontakt med varandra. Därvid krävs att reglerna i de länder som berörs
vid kontakten respekteras.
En hel serie både internationella och nationella regler styr radiokommunikationerna
i en nation. Varje radioamatör skall känna till
och följa dessa regler så långt de har anslutning till amatörradio. Vissa länder - t.ex.
CEPT-länderna - har i någon utsträckning
harmoniserat sina bestämmelser inbördes.
Nationella avvikelser förekommer likväl och
reglerna i det land, som man gör radiosändningar ifrån, skall alltid följas.

2.1 ITU Radioreglemente (RR)

Amatör- och Amatörsatellittjänsterna är radiokommunikationstjänster med syfte att tillhandahålla nödvändig kommunikation i händelse av naturkatastrofer, träna operatörer
och tekniker i radio- och telekommunikationsteknik till ingen kostnad för stat och
samhälle, bidra till att tidsenlig radiokommunikation främjas och att förbättra internationell förståelse och välvilja.

Artikel1 (RR) Termer och definitioner

Si .56 (RR) Amatörtjänst

En radiokommunikationstjänst avsedd för
självutbildning, inbördes kommunikation och
tekniska undersökningar bedrivet av amatörer, det vill säga av behörigen godkända
personer intresserade av radioteknik, endast av personligt intresse och utan ekonomiskt syfte.
S 1.57 (RR) Amatörsatellittjänst
En radiokommunikationstjänst som använder rymdstationer på jordsatelliter för samma
ändamål som för Amatörradiotjänsten.
81.96 (RR) Amatörradiostation
Radiostation inom amatörradiotjänst

Artikel S25 (RR) (f.d. Artikel 32)
Sektion l. Amatörtjänst
825. i \  1. Radiokommunikation mellan
amatörstationer i olika länder skall vara förbjuden, om administrationen i en av de berörda nationerna har meddelat att den är
emot sådan radiokommunikation.
825.2 \  2. (1) Närsändningarmellan amatörstationer i olika !änder är tillåtna, skall det
ske på klart språk och begränsas till meddelanden av teknisk natur i samband med prov
och till personliga kommentarer, som på
grund av sin oviktighet inte är skäl nog för att
ta den allmänna telekommunikationstjänsten i anspråk.
(2) Det är absolut förbjudet att
825.3
använda amatörradiostationer för internationell radiokommunikation för tredje parts
räkning.
825.4
(3) De föregående bestämmelserna får ändras genom särskilda överenskommelser mellan administrationerna i berörda länder.
825.5 \ 3
(1) Varje person som söker en
licens för att använda apparaterna i en
amatörradiostation skall bevisa sin förmåga
att för hand sända rätt och med hörseln rätt
ta emot texter i form av morsesignaler. Berörda administrationer får emellertid bortse
från detta krav för stationer som endast
används på frekvenser över 30 MHz.
825.6
(2) Administrationerna skall vidta sådana åtgärder som de finner nödvändiga för att kontrollera de handhavandemässiga och tekniska kvalifikationerna hos varje
person som önskar använda apparaterna i
en amatörradiostation.
825.7 \ 4
Den högsta effekten från en
amatörstation skall fastställas av berörda
administrationer, med hänsyn till operatörernas tekniska kvalifikationer och under vilka förhållanden dessa stationer skall användas.

1112- 1

LER

C TRAFIKMET DER

S25.8 \ 5
(1) Alla allmänna regler i överenskommelsen och de i denna artikel skall
tillämpas på amatörradiostationer. Särskilt
den utsända frekvensen skall vara så stabil
och så fri från sidafrekvenser som den tekniska utvecklingen för sådana stationer medger.
S25.9
(2) Under loppet av sändningarna skall amatörstationer sända sina anropssignaler med korta mellanrum.

Sektion Il. Amatörsatellittjänst
S25.1 O \ 6 Bestämmelserna i Sektion 1 i
denna artikel skall gälla i all tillämplig omfattning även för amatörsatellittjänst
S25.1 O \ 7. Rymdstationer i amatörsatellittjänst, som arbetar i band som delas med
andra tjänster, skall förses med lämplig utrustning för att kontrollera utstrålningen om
skadlig störning rapporteras, allt i överensstämmelse med den procedur som föreskrivs i Artikel S15 *.Administrationer som
godkänner sådana rymdstationer skall informera RRB (Radio Registrations Board)
och skall tillse att tillfredställande jordkontrollstationer upprättas före uppskjutningen för
att säkerställa att varje rapporterad skadlig
störning skall kunna avbrytas omedelbart av
den bemyndigande administrationen. Se
S22.1 **.
* S15 behandlar "lnterference"
** 822 behandlar "Space Services"

Bild III 2-1 ITU Regionkarta (ur RRB-2)

1112-2

~©

Artikels (RR 8..1) Frekvenstilldelning
Inledning
391 § 1. l Unionens alla dokument där
termerna allocation, alfatment och assignment används skall de ha den betydelse
som ges i nummer 17 till19, varvid termerna
på de tre arbetsspråken skall vara som följer
(franska, engelska och spanska):
Frekvensfördelning till:
Tjänster
Allocation (tilldelning)
Allotment
(fördelning)
Områden
stationer
Assignment (anvisning) .... etc.
(För enkelhetens skull återges här endast
betydelserna på engelska språket).
sektion l. Regioner och områden
392 § 2. Förtilldelning av frekvenserhar
världen delats in i tre Regioner så som visas
på följande karta och som beskrivs i 393 till
399 .... etc.
Det innebär att tilldelning, fördelning och
anvisning av frekvenser mycket väl kan skilja
mellan ITU-regionerna. Skillnaderna förklaras t.ex. av regionalt olika behovsstruktur,
befolkning etc.
Det förekommer också likheter. På nedanstående karta har markerats en tropisk
zon, vilket förklaras av den annorlunda vågutbredningen där. T.ex. behöver särskild
hänsyn tas vid frekvenstilldelning (allokering) till rundradiotjänsten i zonen.

RE LE

CH

K ET

2.2
Begreppet CEPT
Vid sidan av folkrättsligt bindande avtal såsom den internationella telekonventionen
(ITC) - har det internationella samarbetet
lett till överenskommelser som inte är tvingande. Sådana avtal görs bl.a. inom CEPT.
CEPTbetyder Conference Europeanne des
Administrations des Poste s et des Telecommunications, d.v.s. Europeiska konferensen
förpost- och teleadministrationerna. "Konferens" är att förstå som ett ständigt arbetande
samarbetsorgan.
Arbetet inom CEPT har huvudsakligen
karaktär av ömsesidiga programförklaringar
mellan länder. Trots att dessa viljeförklaringar
eller rekommendationer inte är bindande har
de visat sig värdefulla för utvecklingen av det
internationella samarbetet.
Länder anslutna till CEPT förenklar handläggningen av ärenden bl.a. rörande amatörradio genom att ömsesidigt bekräfta
rekommendationer inom området.

CEPT-rekommendationerna
Länder anslutna till CEPT förenklar numera
handläggningen av tillståndsärenden om
amatörradio genom att ömsesidigt bekräfta
och inom sitt land tillämpa rekommendationer som länderna utformat i samråd. Det
innebär att svenska amatörradiobestämmelser kan "harmoniseras" till andra länders.
För kompetenskrav vid examinering av radioamatörer finns CEPT-rekommendationerna TIR 61-01 och TIR 61-02.

CEPT-rekommendation TIR 61-02
Rekommendationen T/R 61-02 innebär att
administrationerna i CEPT-Iänder utger ömsesidigt erkända Harmoniserade Amatörradio Examinerings Certifikat (HAREC) till de
personer som vid nationella prov uppfyller
rekommendationens kunskapskrav motsvarande nivå A respektive B. Dessa HARECnivåer motsvarar kraven för de svenska certifikatsklasserna CEPT i respektive CEPT
2. Radioamatörer med ett sådant certifikat
får utöva amatörradio i annat CEPT-Iand,
som godkänt T/R 61-02 och får tilldelas ett
CEPT-certifikat av det landet utan att behöva genomgå ytterligare kunskapsprov.
Det medger också att en person som
uppvisar ett CEPT-certifkat (HAREC), utfärdat av ett annat CEPT-Iand, tilldelas ett
motsvarande tillstånd vid återkomsten till
hemlandet utan att behöva genomgå ytterligare kunskapsprov.
Rekommendationen godkändes år 1990
och reviderades år 1994 med målsättning
att möjliggöra för icke CEPT-Iänder att delta
i systemet.
Sverige tillämpar T/R 61-02.

CEPT-rekommendation TIR 61-01
Rekommendationen T/R 61-01 möjliggörför
radioamatörer från CEPT-länderna att utöva
amatörradio under korta besök i andra CEPTländer, utan att behöva ett tillfälligt tillstånd
från det besökta CEPT-Iandet. Den godkändes år 1985. Erfarenheterna med detta system är goda. År 1992 reviderades rekommendationen med målsättning att möjliggöra för icke CEPT-Iänder att delta i systemet.

1112-3

REG

1112-4

R CH

FIK ET

R

D R
2.3 Svenska lagar, bestämmelser

Lagar, föreskrifter och anvisningar
tillämpas för amatörradioanvändning.
Märk, att ändringar kan tor'eKt'JmJma.
Använd
lagen om radiokommunikation m.fl.
Denna lag reglerar all
i
Sverige. Tillstånd behövs i princip för all
slags radiosändning.
Post- och telestyrelsen - PTS - är enligt
Förordning om radiokommunikation den
svenska myndigheten (administrationen) för
telekommunikation. PTS skall bland annat
svara för att möjligheterna till radiokommunikationer utnyttjas effektivt och har därvid att
beakta den internationella regleringen inom
området. Regleringen av amatörradioanvändningen begränsas nu till den minsta
omfattning som följer av internationella avtal
och europeiska rekommendationer,
CEPT-rekommendationer.

Post- och telestyrelsens föreskrifter om
innehav och användning av amatörrad i om
anläggningar m.m.
Post- och telestyrelsens styrelse beslutar
om Post- och telestyrelsens föreskrifter om
innehav och användning av amatörradioanläggningar. Dessa föreskrifter är anpassade tilllagen om radiokommunikation.
Enligt radiolagen kan ett tillstånd att
inneha och använda radiosändare +r. ..."'"''"'"
med villkor angående kompetenskrav för
den som skall handha radioanläggningen.
För att få ett amatörradiotillstånd måste
man ha ett radioamatörcertifikat, som är ett
kompetensbevis från Post- och telestyrelsen. Med stöd av lagen ställer PTS
tenskrav, att jämföras med artikel S25 i
internationella radioreglementet
åberopar därvid CEPT-rekommendationen
T/R 61-02 som kunskapsnorm för de svenska klasserna CEPT 1 och CEPT 2.
Det innebär att PTS numera tillhandahåller endast dessa två certifikatsklasser.

tillståndsvillkor
Föreningen Sveriges Sändareamatörers
1995: 1 om innehav och
av amatörradioanläggningar
CEPT-rekommendationer för nybörjarcertifikat däremot, finns inte f.n. (år 1997).
CEPT-Iänderna är nämligen meningarna delade om de krav som skulle
rekommenderas. Eftersom CEPT-rekommendationer för nybörjarcertifikat ej finns,
tillgodoses behovet av svenska sådana certifikat på annat sätt.
Med stöd av Post- och telestyrelsens
föreskrifter har Föreningen Sveriges Sändareamatörer- SSA - tillstånd att inneha
och använda amatörradiosändare för Föreningens utbildningsverksamhet inom amatörradioområdet Mot denna bakgrund beslutar SSA:s styrelse en serie anvisningar
där villkoren för SSA-certifikat och SSAtillstånd specifiseras.
.,,.., ..,,.,l!"'',fti!"!!U"'\ 11"1!

litteraturhänvisning om lagar och föreskrifter
Följande kan lånas på de flesta större bibliotek eller kan beställas från Fritzes Förlag
eller Föreningen Sveriges Sändareamatörer:
e Telelag,
• Lag om radiokommunikation,
e Förordning om radiokommunikation,
• Post-och telestyrelsens föreskitter om godkännande av provförrättare m.m.,
• Post- och telestyrelsens föreskifterom innehav och användning av amatörradioanläggningar m. m.,
• Föreskrifter om ändring i Post- och telestyrelsens föreskitter om innehav och användning av amatörradioanläggningar
m.m.,
e Post- och telestyrelsens föreskrifter om
avgifter för certifikat m.m. inom radioområdet
Följande kan beställas från
~o-r.v·on1nncln Sveriges Sändareamatörer:
• SSA:s anvisningar om amatörradioanläggningar vid SSA-tillstånd,
e SSA:s anvisningar om kunskapskrav för
SSA-certifikat,
• SSA:s anvisningar om provförrättning för
SSA-certifikat.

1112-5

REG

1112-6

R CH TRAFIKM

R CH
3.1 stationsdagbok (loggbok)
Ändamål
Dina radioförbindelser och övriga händelser
med radiostationen bör antecknas i en
stationsdagbok (loggbok)
Amatörradioverksamheten bygger på förtroende och då är det viktigt att själv kunna
dokumentera sin verksamhet t. ex. i störningssituationer m.m .. Loggen används också för
att kunna visa när man har varit aktiv.
Helt i eget intresse är det ju också trevligt
med en loggbok. Tänk bara på hur bra det är
att ha alla underlag för tävlingar och diplom
m.m. dokumenterade.
Kunna visa hur man för en loggbok
Bilden på nästa sida visar ett exempel på
hur en loggsida (förminskad) kan se ut.
Fundera på följande:
1. Halv tre på eftermiddagen den tionde
oktober gör Ulrik (SM7LQQ) ett allmänt anrop på den lokala repeatern 2-metersbandet.
Karin (SM7UBM) som är på väg hem från
skolan svarar. Ulrik berättar att han precis
har byggt sitt nya slutsteg på 25 W färdigt
och frågar Karin om det hörs någon skillnad
när han kopplar ur det. Efter lite småprat om
allt möjligt säger de 73 till varandra och då
har det gått sju minuter sen de började.
Fyll i loggboken åt U/rik!
2. Gör ett låtsas-QSO med en kurskamrat. Bokstavera era "signaler". För in i loggen.

Allmännauppgifter om motstationen, t.ex.
signalrapport, namn, QTH, motpartens utrustning, QSL-adress o.s.v. brukar också
vara bra att ha med.
Man bör också skriva upp när man har
gjort allmänt anrop, sänt ut bärvåg för prov,
experiment och annat som kan vara av intresse.
Om någon annan radioamatör använder
din station ska du också skriva upp hans/
hennes namn och anropssignal.

Rapportkoder
Man blir ofta ombedd av motstationen att
lämna en s.k. signalrapport på dennes sändning. Omvänt är det bra att få en signalrapport på den egna sändningen.
För rapportering mellan radioamatörer
används RST -koden
För lyssnarrapporter t.ex. till rundradiostationer, förekommer ett kodsystem, som
kallas för SINPO eller SINPFEMO.
Se Appendix J.

Föra in data
Det man skriver upp i loggen är
e
Tiden i början och i slutet av förbindelsen.
Glöm inte datum!
• Motstationens signal.
• Din effekt (ineffekt, PEP eller utstrålad
effekt)
• Frekvensband, ev frekvens.
• Sändningsslag (FM, SSB, CW, paketradio etc).
e
Uppgift om varifrån man sände (eget
QTH).
e
signalrapporter (rapportkoder)

1113- 1

RE LER

~j]

-~~~~~~~~~

ll----+--+--!--+---i-+-+--+--+---!--+---+--+---l---l---!---l--1----+---+--l-i---l---l

1113-2

\appendix

\chapter{Sätt att uttrycka måttenheter}

Inom fysiken förekommer allt mellan
mycket höga och mycket låga värden på
frekvens, spänning, ström, resistans etc.
l en radiomottagares antenningång
är signalspänningen ofta mindre än 11.1V.
l slutsteget i en amatörradiosändare
kan anodspänningen vara mer än 2 kV
och uteffekten upp till1 kW.
l spektrum för elektromagnetiska vågor finns mycket höga frekvenser.
1 000
1 000
1 000
1 000
1 000
1 000
1 00

000
000
000
000
000
W

För att ange storheten på måttenheter används ofta ett prefix före måttenheten (av latinets pre, före och fixare,
att tillägga). Med prefixet anges från fall
till fall vilken multiplikations- ellerdivisionsfaktor (talfaktor) som används. Det finns
ett antal prefix.
Märk, att enhetens sort inte har något
att göra med själva prefixet. Nedan ges
sorterna Hz, W, V, F etc. som exempel.

000
000
000
000
W

000 000 000 Hz = 1 EHz = 1 · 10 18 Hz (E är Exa)
000 000 Hz
= 1 PHz = 1 · 1015 Hz (P är Peta)
000 Hz
= 1 THz = 1 · 10 12 Hz (T är Tera)
W
= i GW = i · i 09 W (G är Giga)
= 1 MW = 1 · 1 06 W (M är Mega)
= 1 kW = 1 · 103 W (k är kilo}
= 1 · 102 W (h är hecto)
1O
= i · i 0 1 W (da är deca)
i V
= 1 V = 1 · 10$\circ$ V (1 = 10$\circ$ är grundenhet)
= 1 . 1o- 1
(d är deci)
1:1o
1 : 1 oo
= 1 · 1o-2
(c är cent i)
1 : 1 000 V
= i mV = 1 · 1o-3 V (m är milli)
i : 1 000 000 V
= 1 )l V = 1 · i o-e V (!l är mikro)
1 : 1 000 000 000 F
= 1 nF = 1 · 1o-9 F (n är nano)
1 : 1 000 000 000 000 F
= i pF = 1 · 1o- 12 F (p är pico)
1 : 1 000 000 000 000 000
=1 f
= 1 · 1Q- 15
(f är femto)
1 : 1 ooo 000 ooo ooo 000 000 = 1 a
= 1 · 1o- 18
(a är atto)
Exponenter, t. ex. siffran 6 i uttrycket 106 , förklaras i appendix B.

Mer om att uttrycka måttenheter
En decimal talstorhet uttrycks ofta med ett s.k. tekniskt flyttaL Decimaltecknet placeras då så att den
visade tio-exponenten i talet blir en multipel av 3. Se
exempel i ovanstående uppställning.
Decimaltecknet kan även placeras så att tioexponenten är något annat än en multipel av 3.Talstorheten uttrycks då med ett s.k. tekniskt flyttaL
l miniräknare m.m. visas ofta exponenten som
bokstaven E, åtföljt av ett värde. Ibland utelämnas
själva bokstaven medan exponentvärdet står kvar.
Ex. 1000 visas som 1 · 103
eller 1 E+03
125 visas som 1.25 · 102 eller 1.25 E+02
1O
visas som 1 · 10 1
eller 1 E+01
1
visas som 1 · 10$\circ$
eller 1 E+OO
0.1
visas som 1 · 1o-1
eller 1 E -01
0.01 visas som 1 · 1o-2
eller i E -02
0.012 visas som 12 · 1o-3 eller 12 E-03

Metallers resistivitet
Ämne
Resistivitet
vid 20$\circ$ C
Q· mm 2

m
Aluminium
Bly
Guld
Järn
Koppar
Kvicksilver
Nickel
Platina
Silver
Tenn
Volfram
Zink

0.028
0.22
0.024
0.105
0.018
0.958
0.078
0.1 08
0.016
0.115
0.056
0.058

A-1

Bokstäver ur bl. a. grekiska alfabetet används som symboler för tekniska begrepp.
Märk, att samma symboler används olika inom olika teknikområden.
Här anges några användningar inom elektroniken.
Versaler
"stora"
bokstäver

A

B

a
~

r

y

E

z

E

e

11

A

'A

il

I

M
N

ö

s

v

t
K

Jl

v

.!::..

~
o

p

1t

o
L
T

p

()

't

y
cil

u

x
n

<p

'P

A-2

Gemena
"små"

x

'V
ro

Uttal

Alpha
Beta
Gamma
Delta
Delta
Epsilon
Z eta
tE ta
Teta
Jota
Kappa
lambda
My
Ny
Xi
Om ikron
Pi
Rh o
Sigma
T au
Ypsilon
Fi
Fi
Chi
Ps i
Omega
Omega

Användningsexempel

ledningsförmåga
Del av .. storhet
Förlustvinkel etc.
Dielektricitetskonstant
Verkningsgrad
Vinklar
Kopplingskoefficient
Våglängd
Permeabilitet
Frekvens

3.14159 ...
Resistivitet
Summa
Tidskonstant
Magnetiskt flöde
Fasvinkel
Resistans
Vinkelfrekvens

\chapter{Grundläggande matematik för radioamatörer}

GRUNDLÄGGANDE
Detta avsnitt omfattar några matematiska
begrepp, ekvationer och formler som kan
vara till hjälp vid studium inför radioamatörcertifikat. Svårighetsgraden spänner över
grundskolans och gymnasiets nivåer.
Genomgången av exponentiella tal och
logaritmer ligger till grund för förklaringen av
begreppen decibel och s-enhet, vilka ofta
förekommer i radiotekniska sammanhang.

Ekvationer
Ekvation är ett annat ord för likhet. Vid matematiska beräkningar ställs storheterna upp i
en eller flera ekvationer.
l en s.k. sann ekvation har resultatet av
de uppställda storheterna samma värde på
båda sidor om likhetstecknet.
Ex. 3 · 5 = 15 (3 multiplicerat med 5 är 15)
4 + 7- 1 = 1O (4 plus 7 minus 1 är i O)

~ =3
5

(15 dividerat med 5 är 3)

(Multiplikationstecknet bör skrivas som
en höjd punkt · och inte som x. Då undviks
förväxlingar med bokstaven x i ekvationer,
där okända tal betecknas med bokstäver).
För att resultatet skall bli rätt måste givna
regler alltid följas vid behandlingen av storheterna i uppställningarna.
Vid multiplikation och addition kan storheterna hanteras i godtycklig ordning, men
däremot inte vid division och subtraktion.
Resultatet blir 15, antingen vi skriver 3 • 5
eller 5 · 3.
Likaså är resultatet 8, antingen vi skriver
3 + 5 eller 5 + 3.
Däremot blir resultatet annorlunda när man
skriver

3
15
i stället för
15

5

Likaså blir resultatet annorlunda när man
skriver 15 - 5 i stället för 5 - 15
Vid division kan talen ställas upp som s.k.
bråktal. De kan skrivas på något av sätten
15:3 eller 15/3 eller

~
5

Talet före kolon, före snedstrecket respektive över bråkstrecket kallas för täljare.
Talet efter kolon, efter snedstrecket respektive under bråkstrecket kallas för nämnare.
För att tydligare beskriva allmängiltiga samband mellan storheterna i en ekvation, kan
storheterna uttryckas med bokstäver i st. f.
med siffror. En sådan ekvation kallas för
formel.
Sökta eller okända storheter brukar betecknas med bokstäver från slutet av alfabetet, t.ex. x, y eller z. Givna eller kända
storheter brukar betecknas med bokstäver
från början av alfabetet, t. ex. a, b eller c.
Antag två tal a och b, vars produkt är c.
Formeln är då

a·b=c

Sätts c= 15, så är a·b=15. Då kan a· b
vara 3 · 5 eller 5 · 3 eller 7.5 · 2 eller vilka
andra tal som helst vars produkt blir 15.
Likheten .!! =!! kan enligt de matematiska

y

b

reglerna skrivas på något av följande sätt:

b·X=a·y
b·X
a

l=!!
x a
b·X
a=
-

y

!!=l
a b
b-- a·
x

X=a·y
b

Att alla dessa sätt är varianter av en och
samma ekvation kan bevisas, genom att
multiplicera den ursprungliga likheten
b·x=a·y med b·y på båda sidor om
likhetstecknet,

x a
b·y·-=-·b·y
y b

d.v.s.

Detta visar den s.k. diagonalregeln, som
innebär korsvis uppmultiplicering av nämnarna till täljarna.
Vid multipliceringen fås samma resultat
för var och en av varianterna, vilket visar att
de är likvärdiga.

B-1

E

APPENDIX
Räkneexempel med 1 obekant storhet:
Om tredjedelen av ett tal är 8 enheter större
än femtedelen av samma tal, vilket är då
talet?
Det sökta, okända talet kallas t. ex. för x.
Tredjedelen av x är !. och femtedelen är !..

3

5

När 8 läggs till femtedelen fås tydligen två
lika tal, och en ekvation (likhet) kan skrivas

!=B+!

3

5

Vi kan multiplicera, dividera, addera eller
subtrahera godtyckligt på ena sidan om likhetstecknet om vi också gör samma operationer på den andra sidan.
Likhetsvillkoret får aldrig äventyras.
För att kunna utläsa vilket tal som motsvarar x, gäller det att få x ensamt- "fritt"på den ena sidan om likhetstecknet.
Vi multiplicerar alla termer på båda sidorna med 3 i ovanstående formel.
3 ·x= 3. 8 + 3 ·x

3
5
3·X
X=24+5

vilket kan avkortas till

Därefter multipliceras termerna på båda
sidorna med 5.

3·X·5
5·X=5·24+-5
5·X= 120+3·X

d.v.s.

Båda sidor om likhetstecknet minskas
därefter med 3 · x
således 5·x-3·x= 120+3· x-3· x
Multiplikationstecknet brukar inte skrivas
ut, varken mellan tal och bokstäver eller
mellan bokstavsgrupper. Alltså

5x-3x= 120+3x-3x
5x-3x= 2x och 3x-3x= O
Kvar blir då 2x = i 20,
där x är detsamma som 1 · x eller 1x.

8-2

Den sist erhållna ekvationen divideras med
2 på båda sidor om likhetstecknet

-2x = -120
2

2

vilket ger x = 60

Det sökta talet är alltså 60
Kontroll:

60
60
=8+
3
5

20 = 8 + 12; 20 = 20 vilket skulle bevisas.
l det första exemplet använde vi diagonalregeln. De två exemplen visar, att det går att
göra omflyttningar när man löser en ekvation. Ett tal med positivt eller negativt förtecken, och som står på ena sidan om likhetstecknet, kan t.ex. "flyttas" över till andra
sidan om likhetstecknet, om förtecknet byts
till det motsatta.
5x = 120 + 3x kan också skrivas

+5x= +120+ 3x
5x-3x= 120
5x- 3x-120 =O
Kom ihåg: 5x-x-x är samma som
5x-x-x eller 5x-(x+x), d.v.s. 3x
Räkneexempel med 2 obekanta storheter
{ekvationssystem):
Endast en obekant storhet har behandlats i
föregående exempel och en ekvation har
varit tillräcklig för det.
Två eller flera obekanta storheter kan
inte behandlas med bara en ekvation.
Antag, att vi skall beräkna

7x+6y=34
Det går det inte att lösa denna ekvation
entydigt, eftersom x och y kan ha många
olika olika värden, som uppfyller ekvationens villkor- satisfierar den.
Men när ännu en ekvation ställs upp, blir
det möjligt att göra en entydig lösning.
Således

!:

{;::~~=~:

eller

G::~:=~~

Nu passar endast ett och samma x- respektive y-värde in i båda ekvationerna.

APPENDIX
Om x "löses" genom att de båda
ekvationerna skrivs om fås:
34-

1.

X=---"-

7

29-9y

x

2.

5

Vi kan nu göra en ekvation 3 där det bara
finns en obekant, y, som är lätt att beräkna.
3.

34-6y- 29-9y

7

-

eller

5

170- 30y= 203- 63y
33y= 33 d v. s. y= 1

eller

Allmänt gäller att det behövs minst lika
många ekvationer som antalet obekanta storheter.
Exempel:
Vi vet, att ytan i en rektangel är produkten av
dess längd och bredd.
Om en husgrund är 1Ometer lång och har
en yta av 50m 2 , så får vi bredden b genom
att dividera ytan med längden,
50

b= 10 = 5 Bredden är således 5 meter.
Om ytan av ett hus är 300 m2 och bredden är
en tredjedel av längden, vilken bredd och
längd har då huset?
Antag, att längden är x meter. Bredden är då
en tredjedels x och vi får alltså ekvationen

x

x · x kan också skrivas >t vilket uttalas
"kvadraten på x" eller "x upphöjt till 2".
x· x· x kan också skrivas x' vilket uttalas
"kuben på x" eller "x upphöjt till 3".
När vi som i ovanstående exempel har
>(- = 900 och vill veta värdet på x, måste vi
"dra kvadratroten ur" 900.
Detta skrivs x= ,)900 = 30
Ett tal kan även vara negativt, men det
behövde vi inte beakta i detta exempel.
Annars skriver man x= $\pm$,)900 = $\pm$30
Potenser, digniteter

Värdet på y sätts in i ekvationerna 1. och
2., varefter även värdet på x beräknas.
Pröva själv! Svaren är y= 1 och x= 4

X·3=300

B

X·X=300·3, d.v.s. >(- =900

Hur stort är då x?
Vi prövar med olika tal och först med x=
2~, men 20 · 20 = 400 vilket är ett för lågt
varde. Sedan prövar vi med x = 40, men 40
· 40 = 1600 vilket är ett för högt värde. Sätt x
= 30. Eftersom 30 · 30 = 900, så är det sökta
talet x = 30.
30
Huset är 30 meter långt och
=1
meter brett.
3

o

Produkten av två eller flera exakt lika stora
2
faktorer kallas potens. l uttrycket x kallas
faktorn x för bas. Det antal gånger, som
faktorn ingår i produkten, kallas för exponent. Om exponenten är ett positivt helt tal
kallas produkten av faktorerna för dignitet.
Uttrycket X 2 är t. ex. 2:a digniteten av x.
Ett annat exempel är 5 · 5 · 5 = 125
Faktorn är 5 och produkten i 25 är 3:e digniteten av 5.
Det är opraktiskt att skriva många faktorer efter varandra. Man skriver därför faktorn
en gång och exponenten med en liten siffra
till höger ovanför faktorn.
Produkten 5 · 5 · 5 kan i stället skrivas 53 •
Basen är 5 och exponenten är 3. Digniteten
utläses 5 upphöjt till 3.
i O · 1O skrivs 1 02 och läses i O upphöjt till 2
5
2 · 2 · 2 · 2 · 2 skrivs 2 och läses 2 upphöjt
till 5
Om vi går över till bokstavsbeteckningar
gäller allmänt att

an= a· a· a ... n gånger= a upphöjt till n
Faktorn a kallas potensens bas och faktorernas antal kailas potensens exponent.
5
Om vi nu skriver 2 · 2 · 2 · 2 · 2 som 2

'

hur skrivs då

?

2·2·2·2·2 .

Vi kan s~riva ; men det är mer praktiskt
2
att skriva 2
5
Minustecknet anger att 2 står i nämnaren, alltså under bråkstrecket
B-3

EPT

APPENDIX

1

På samma sätt kan vi skriva
1

2- i stället för

i

2

5 i stället för

2

2- i stället för

2~

$\pm$

o.s.v.

6

10 anger att 1O skall multipliceras med sig
självt 6 gånger, d.v.s. resultatet är 1 miljon.
6
10- anger på samma sätt i miljondel
Hur beräknas uttrycket a 3 . a 2 ?
Eftersom a 3 =a· a· a och a 2 =a· a
är tydligen a 3 • a 2 =a· a· a· a· a= a 5
Produkten av två digniteter med samma
bas är lika med basen upphöjd till summan av exponenterna.
Allmänt uttrycks detta
På samma sätt beräknas am
an
Exempel:

a·a·a·a·a
2
----=a·a=a
a· a· a

således

När potenser med samma bas skall divideras med varandra, fås resultatet genom att den gemensamma basen "upphöjs till" skillnaden mellan exponenterna,
d.v.s.
Är n större än m får exponenten negativt
tecken t.ex. för m = 5 och n = 7

as

-=a

-2

al

0

Alla tal upphöjt till noll blir= 1 t.ex. a = i
Med m = n i den föregående formeln får vi

an
an

-=1
an

Vi får också -

an

8-4

=a

n-n

=a

o

r

r

Att upphöja en produkt till en potens görs
så, att var och en av faktorerna upphöjs
till potensen, varefter resultaten multipliceras med varandra.
3

am= a 5 =a· a· a· a· a
an = a 3 = a. a. a

2

r

r

(O.OOOOOi ).

1

T.ex. 10$\circ$ = 1 ingår i serien 10- , i0$\circ$, 10 , 10
.... ,vilket är ett annat sätt att skriva O, 1, 1, 1O,
100 .... etc
Uttrycket a. bn betyder att a skall multipliceras med "b upphöjt till n".
Uttrycket (a· b betyder att a och b skall
multipliceras med varandra n gånger:
(a. b = (a . b) . (a. b) . .. o. s. v. n gånger.
l det senare fallet kan parenteserna slopas,
utan att resultatet förändras:
(a. b = a. b. a. b . . . o. s. v. n gånger.
Samlar vi alla a respektive alla b var för sig
fås an . bn = (a. b
Skilj noga mellan abn = a. bn
och
(abr = (a·br

3

3

Exempel1 . (4 · 5) = 4 · 5 = 64 · 125 =8000
2
2
2
2
Exempel 2. (3a) = 3 · a = 9a
På samma sätt kan ett bråk upphöjas till en
potens genom att upphöja täljaren och
nämnaren
a a a
an
b =b·b·b ... o.s. v. n gånger= bn

a)n
(
2)

(3

3

2 2 2- 2

3

8

-3'3'333 27
-

-

Rötter
Roten ur ett tal är den faktor, vars kvadrat är
talet.
Tidigare behandlade vi uttrycket i = 900 för
att få fram värdet på x. Vi "drog kvadratroten
ur 900".
Tecknet
kallas rottecken.

r-

x= ~900 skall egentligen skrivas !\ /900,
men tvåan brukar uteslutas när det gäller
kvadratroten. l övriga fall är det nödvändigt
att skriva ut rottermen, t. ex. ~1 00 (uttalas
som 3:e roten ur) eller ~ (uttalas som
6:e roten ur).
Kom ihåg följande allmänna regler

~=-fä·{b

och

~={a
~b {b

APPENDIX
Den första regeln förenklar dragning av roten ur stora tal,
~ 1225 = ~ 25. 49 = {25. {49 = 5. 7 = 35
Ett större tal kan alltså delas i flera mindre tal,
vars respektive rotvärden är lättare att få
fram. Rotvärdena kan erhållas ur matematiska tabeller eller miniräknare.
Roten ur tal kan blir ändlös, t. ex.
{2. = 1. 414.... och {3 = 1. 732....

logaritmer

Beräkningar kan göras enklare med användning av logaritmer.
Först studerar vi följande tabell över digniteter av talet 2,
4
9
t.ex. 2 = 16 och 2 = 512.
Produkten eller kvoten av tal kan beräknas
med addition respektive subtraktion sedan
talen omvandlats till exponentiella tal med
samma bas.
Exempel
4
9
4 9
13
1) 16. 512 = 2 .2 =2 + = 2 = 8192
2) 2048 =~=211-6 =25 =32
64
26
Här är sambandet mellan exponent och dignitet för basen 2:
Expo- DigniExpo- Digninent
tet
nent
tet

1
2
3
4
5
6
7

8
9

2

4
8

4

16 (=2 )
32
6
64 (=2 )
128
256
9
512 (=2 )

10
11
12
13
14
15
16
17
18

1 024
11
2 048 (=2 )
4 096
13
8 192 (=2 )
16 384
32 768
65 536
131 072
262 144

En sådan tabell har emellertid begränsad
användbarhet vid behandling av godtyckliga
taluppställningar. Begreppet logaritm är däremot mera användbart.

Med logaritmen för ett tal menas den
exponent, som basen skall upphöjas till,
för att potensens värde skall bli talet.

B

Exempel
l ekvationen 2x = 31 säger man att x är
logaritmen för talet 31 i det logaritmsystem,
vars bas är 2.
Detta skrivs x= 2!og 31 och läses x= tvålogaritmen för talet 31 .
2
Kvadraten på talet 1Oär 100, d.v.s. 10 =
100. Talet 2 är alltså den exponent som talet
1Oskall upphöjas med för att digniteten skall
bli 100. Således 1Olog 100 = 2
Vid omvandling mellan decimala tal och
deras logaritmer används s.k. logaritmtabeller eller miniräknare (inte de allra enklaste). För överslagsberäkningar används
även diagram och skalor (t. ex. räknestickan).
Så här räknar man med logaritmer:
När decimala tal skall multipliceras med varandra, omvandlar man dem först till logaritmer. Man adderar dessa och återvandlar
resultatet till decimala tal igen.
När decimala tal skall divideras med varandra, omvandlar man dem först till logaritmer. Man subtraherardessaoch återvandlar
resultatet till decimala tal igen.
Exempel
Talen 100, 100, 100,2 och 2 skall multipliceras med varandra.
Det decimala förfarandet är:
6
100. 100. 100 . 2. 2 = 4000000 = 4. 10
Förfarandet med logaritmer är att manomvandlar talen till deras respektive 1Ologaritm, vilken är 1Olog x, varefter logaritmerna adderas. Dessa räkneoperationer kan
göras t.ex. med en miniräknare. Då fås
2 + 2 + 2 + 0.301 03 + 0.30103 = 6.60206 som
är summan av logaritmerna för talen.
För att uttrycka svaret som ett decimalt
tal omvandlas logaritmen till antilogaritm,
6 60206
6
~ 4 · 10
vilken är 1Ox= 10 .
(samma somvid det decimala förfarandet).
Skulle talen ha dividerats så skulle deras
respektive logaritmer ha subtraherats från
varandra i stället.
Här är sambandet mellan en serie decimala tal och deras 1O-logaritmer (1 Olog x).

B-5

APPENDIX
Antilogaritmen
för tal med
10-bas och
exponenten
x d.v.s. (i Ox)
1.00
1.25
i.6
2.0
2.5
3.2
4
5
6
7
8
9
iO
20
30
50
iOO
500
1 000
5 000
10 000
100 000
1 000 000

8-6

~©

Dignitet

Logaritmen
för 1Olog x
(avrundade
tal)

1.00 . 10$\circ$
1.25 . 10$\circ$
1.6. i 0$\circ$
2.0. 10$\circ$
2.5. 10$\circ$
3.2. 10$\circ$
4. 10$\circ$
5. 10$\circ$
6. 10$\circ$
7. i0$\circ$
8. i0$\circ$
9. 10$\circ$
1
i .i0
1
2. i0
1
3. 10
1
5. i0
2
i .i0
2
5. i0
3
i . 10
3
5. 10
4
i . 10
5
1 . 10
6
1 . 10

0.00
0.097 ~O. i O
0.204 ~ 0.20
0.301 ~ 0.30
0.398 ~ 0.40
0.505 ~ 0.50
0.602 ~ 0.60
0.699 ~ 0.70
0.778 ~ 0.80
~ 0.85
0.903 ~ 0.90
~ 0.95
i .00
1.301 ~ 1.30
1.477 ~ 1.50
1.699 ~i .70
2.00
~ 2.70
3.00
~3.70

4.00
5.00
6.00

\chapter{OMRÄKNING MELLAN dB OCH KVOTEN AV TAL}

Benämningen Bel kommer från namnet på
amerikanen Alexander Graham Bell, som år
1876 uppfann den första praktiskt användbara telefonen efter ideer från tysken Philipp
ReiB.
Inom teletekniken används begreppet
decibel för att beskriva förlopp av effekt,
ström och spänning. Begreppet förekommer
även i andra sammanhang, t. ex. akustik där
det istället är fråga om ljudtryck.
Måtten i det metriska systemet är alldagliga och ingen finner det märkligt att det
t.ex. gårtio decimeter på en meter. Däremot
är begreppet decibel ovant för många.
Räkning med decibel grundas på användning av logaritmer, som är ett bekvämt
sätt att uttrycka och behandla talvärden.
Detta har har i korthet förklarats i Kapitel1 .9.
Här beskrivs ett omräkningsförfarande med
hjälp av tabeller.

Decibel är ett dimensionslöst uttryck för
graden av dämpning alternativt förstärkning.
Dämpning är följden av att vissa komponenter bromsar elektrisk ström.
Förstärkning innebär att en aktiv komponent kan styra en större elektrisk ström och
därmed större effekt än den själv styrs med.

Bestämning av dB ur effektförhållandet

i) Dela upp effektförhållandet i faktor och
i O-potens
2) Bestäm dB-talet för 1O-potensen
3) Bestäm dB-talet för faktorn med nomogram, tabell eller räknare
4) Addera dB-talen till ett slutresultat
Exempel: dB-talet för 300-faldig effektförstärkning
i) 300 = 3·i00
2) i 00 motsvarar 20 dB
3) 3 motsvarar 4.8 dB
4) 20 + 4.8 = +24.8 dB

l

2
i ji l l

o

iJl
3

i

i

l

l

4
l

l

6

Exempel: dB-talet för 70 000-faldig effektminskning
i) 70 000 = 7 . i 000
2) i O 000 motsvarar 40 dB
3) 7 motsvarar 8.5 dB
4) 40 + 8.5 =- 48.5 dB

o

Exempel: En Vagi-antenn omfördelar den
utstrålade effekten så att den i bästa riktningen blir 56-faldigt bättre än från en
referensantenn. Hur många dB motsvarar
det?
56= 5.6. iO
vilket motsvarar 7.5 + 1O= 17.5 dB
Exempel: l en koaxialkabel förloras hälften
av den inmatade effekten. Hur stor är
dämpningen i dB?
Effektförhållandet är faktor 2 (inverterat),
d.v.s 3 dB dämpning=- 3 dB

Bestämning av effektförhållandet ur dB

i) Dela upp dB-talet i i-tal och i O-tal
2) i-talet dB ger en faktor
3) i O-talet dB ger antal nollor bakom faktorn
(ställvärdet)
4) Multiplicera faktorn med ställvärdet
(antalet nollor).
Exempel: Vilket effektförhållande motsvarar
i3 dB?
1) i 3 dB = i O dB + 3 dB
2) 3 dB motsvarar faktor 2
3) 1O dB ger 1 nolla
4) 2·i O= 20, d v s 20-faldig förstärkning
Exempel: Vilket effektförhållande motsvarar
-i15d8?
1) ii5 dB = i 1O + 5 dB
2) 5 dB motsvarar faktor 3.2
3) 11 O dB ger ii nollor
4) Efter decimaltecknet i faktor 3.2 följer ett
ställvärde av 11 ,
d.v. s. 320 000 000 000-faldig dämpning.

8
l

i

l

l

l

l

9

i

l

10 ggr

l

Effekt

10 dB

C-1

APPENDIX
Exempel: Om ineffekten till ett slutsteg är
100 W och det har en förstärkning av 1odB.
Hur stor är uteffekten?
1O dB motsvarar en 1O-faldig förstärkning.
Uteffekten från slutsteget blir således 10·1 00
= 1000 w.

Exempel: 300-faldig spänningsförstärkning
1) 300 = 3 . 100
2) 100 motsvarar 40 dB, d.v.s två (nollor)
gånger 20 = 40
3) 3 motsvarar 9.5 dB
4) 40 dB+ 9.5 dB = 49.5 dB

Sambandet mellan effektförhållande och dB

Bestämning av ström.. eller spännings..
förhållandet ur dB
1) Dela dB-värdet med 20 dB varvid erhålls
en del och en rest
2) Delen ger 1O-potensen, d.v.s antalet nollor bakom faktorn (ställvärdet)
3) Gå in i nomogrammet och omvandla "resten" till en faktor
4) Multiplicera faktorn med ställvärdet

dB

o

o

1
1
1
1
2

1
2
3

4

5
6
7

8
9

10

o

2
5
9
5
1
9

3
3
5 . o
6
3
7
9

20

o

5
8
9
1
6
8
1

o

4

30

o

8
4
5
1
2
1
1
9
3

40

o

9
8
2
8
2

o

8

5
2

50

o

2
9
6
8
7
7
7
7
8

60

o

5
3
2
6
8
2

2
3

2

Kolumnen längst till vänster upptar 1-tal dB
från Otill 9 och den översta raden upptar 1Otal dB-tal från Otill 60.
Med tabellen kan effektförhållanden bestämmas ur dB-tal från Otill 69 eller omvänt.
Det motsvarar effektförhållanden från 1:1 till
1:8 millioner.
Var uppmärksam på decimaltecknets placering. Avkorta till önskat antal decimaler.
Exempel: Vilket effektförhållande motsvarar
+7dB?
?ligger mellan Ooch 9. Sök därför OdB i den
översta raden. Gå sedan rakt nedåt i kolumnen till raden för 7 dB. Vi kommer då till första
siffran i kvoten för 7 dB. Decimaltecknet står
till höger om denna ruta (mellan kolumnerna
för O och 1O dB).
l sifferfältet kan nu utläsas en effektförstärkning (kvot) av 5.011872.
Bestämning av dB ur ström- eller
spänningsförhållandet
1) Dela upp ström- eller spänningsförhållandet i faktor och 1O-potens
2) Bestäm dB-talet för 1O-potensen
3) Bestäm dB-talet förfaktorn ur nomogrammet
4) Addera dB-talen till ett slutresultat

C-2

Exempel: Vilket spänningsförhållande motsvarar -115 dB?
1) 115 dB/20 dB= 5 rest 15
2) 5 (nollor) motsvarar 100 000
3) 5 dB motsvarar 5.5
4) 5.5 ·1 00 000 =550 000-faldig spänningsdämpning
Tabeller för sambandet mellan ström- eller spänningsförhållande och dB
Kolumnerna längst till vänster upptar dBtalen O- 9. l tabellernas översta rad är dBtalen listade i jämna 1O-tal dB från O - 120
respektive i udda 1O-tal dB från 10-130.

Med tabellerna kan ström- och spänningsförhållanden bestämmas ur dB-tal från Otill
139 dB och omvänt.
Detta motsvarar förhållanden från 1:1 till
1:8.9 millioner.
dB

Jämna 1O-tal dB från O till 120 dB
O 20 40 60 80 100 120

o

o

1
2
3
4
5
6
7
8
9

1
1
1
1
1
1
2
2
2

o o o o o o

1
2
4
5
7
9
2
5
8

2
5
1
8
7
9
3
1
1

2
8
2
4
8
5
8
1
8

o

9
5
8
2
2
7
8
3

1
2
3
9
7
6
2
8
8

8
5
8
3
9
2

1

6
3

PPENDIX
Udda 10-tal dB från 10 till 130 dB
10 30 50 70 90 110 130

o

1
5

3
3
3

1
2
3

9

4
5
5
6
7
7
8

4
5

6

7

8
9

6
4
8
6
1
2

4

o
6

o

7
4
1

9
9

7

8
4
2
6
2

3

o

7

8
8
4
5
4
2
5

3
9
9
3
2

o

3

2
1

2
8
1
6
1

Decibel över 1 m W vid 50 Q [dB(m)]
Som nu beskrivits är uttrycket decibel ett
logaritmiskt mått för hur två effekter förhåller
sig till varandra. När de jämförda effekterna
uppträder över lika stora impedanser, kan
även förhållandet mellan två spänningar eller två strömmar uttryckas i decibel. l samtliga fall rör det sig om förhållandet mellan två
storheter - aldrig absoluta storheter.

3
7
1
7
5
8

3
3
8
2

o

Exempel
Ett drivsteg i en sändare drivs med 1 watt och
avger 1Owatt. Effektförhållandet är 10:1 och
effektförstärkningen är 1O gånger eller 1O
dB. slutförstärkaren i samma sändare drivs
med 1O watt från drivsteget och avger 100
watt till antennen. Även i detta fall är effektförhållandet 10:1 och effektförstärkningen
1O gånger eller 1O dB.
slutförstärkaren hanterar en 1 gånger
så hög effektnivå som drivsteget och ändå är
förstärkningen 1OdB i båda fallen. Decibel är
m.a.o. dimensionslöst
Men om en av två jämförda effekterna
alltid är densamma och väl definierad så
medges nya möjligheter. Den effekt som
skall kvantifieras kan nu ställas i förhållande
till den kända referenseffekten. Med denna
förutsättning kan även de absoluta effektnivåerna, t.ex. genom en sändare uttryckas
i decibel. Detta tillgår på följande sätt.
Det är mycket vanligt att in- och utgångarna i HF-utrustningar utförs med en impedans av 50 n. För god anpassning väljs
då koaxialkablarna mellan apparaterna med
en karaktäristisk impedans av 50 n.

9

Exempel:
En förstärkare med lika in- och utgångsimpedans förstärker spänningen 350-falt.
Hur många dB är det?
Vi söker närmaste 3-ställiga tal i de två
ovanstående tabellerna. l den nedersta tabellen finner vi talet 354 på andra raden.
Över entalet 4 finner vi 50 dB i den översta
raden. Till vänster om 354 finner vi 1 dB.
Som ett närmevärde är alltså förstärkningen
50+ 1 =51 dB.

o

Om kvoten i nedanstående nomogram är en
eller flera 1O-potenser högre än 1O (ggr), så
kan nomogrammet utökas enligt följande
tabell.
Kvot*

Analys
1 har O nollor
1O har 1 nolla
100 har 2 nollor
1 000 har 3 nollor
1O 000 har 4 nollor

1

10
100
1 000
10 000

1

1.1

l

1.2

l

1.3

Skriv

dB

1 . 20
2. 20
3. 20
4. 20

= 20
= 40
= 60
= 80

o. 20 = o

1.4

1.5

1.6

1.7

1.8

1.9

2.0

l

ggr

l
l ....a...........~.........r.....,.l...~..-....\&.........\&................,1~"""'--"-........~......,..j.--l
l
!
l
l
l
.......!..........l......,..j.r-1

l

!

....,...~,.,...........,.1....a...........~...........~.rl

l

i

Spänning
dB

O
2

1

l1PI 1 l 1 1
l

l

0246

8

3
l
l

l
10

l

i

4
l

5

i

l
l

12

14

7

6
l

i

l

l
16

l

li

8

l
l

18

9
i

l

10 g gr

l

20

Spänning
dB
C-3

APPENDIX

C

Det har utbildats en praxis, att referensvärdet vid jämförelse av signalnivåer i radiosystem skall vara en milliwatt (1m W) utvecklad i en belastning med impedansen 50 Q.

Signalnivåer över belastningen 50 Q kan
uttryckas i dB(m), där (m) står för milliwatt,
varvid referenseffekten 1 mW är OdB( m) vid
50 Q.
Det spänningsfall som bildas över belastningen 50 Q vid effektnivån O dB( m) är

U=-.JP·R=1·10-s·50~0.224 V
Den ström som flyter genom belastningen
50 Q vid effektnivån O dB(m) är

l=

(P =~ 1 . 10 -s =0.0045 mA

~Fi

5o

Strömmen 4.5 mA genom belastningen 50 Q
motsvarar således O dB( m).
Varje annan effekt, spänningsfall och ström
som uppstår vid en belastning av 50 Q kan
jämföras med respektive referensvärden
1 mW, 0.22 V och 4.5 mA.

Sambandet mellan spänning över 50 Q
och dB(m)
dB(m) V

-40
-30
-20
- 10

o

1
2
3
4
5
6
7
8
9
10

0.00224
0.00707
0.0224
0.0707
0.224
0.251
0.282
0.316
0.354
0.398
0.446
0.501
0.562
0.630
0.707

dB(m)

V

11
12
13
14
15
16
17
18
19
20

0.793
0.890
0.999
1.121
1.257
1.411
i .583
1.776
1.993
2.236

dB(m) är ett absolut och logaritmiskt mått.

dB(W) är ett annat absolut mått.

Effekt

Effektnivåer över en belastning kan också
uttryckas i dB(W), där (W) står för watt.
Referenseffekten är då 1 W, d.v.s. OdB(W).
Liksom med dB( m) anges impedansen i den
belastning, som effekten utvecklas över.

a [dB(m)]=10 log·

r~nJ

1 m

a

f'so =1

[mW]·1010

Ström
O dB(m) = 4.47 mAsa

l
4.47

a [dB( m)]= 20 log §Q

Isa= 4.47 ·1 0 20
Spänning
O dB(m) = 0.223 Vso

u.

a[dB(m)] = 20 log o.d~
Uso = 0.223 ·1 0 20

C-4

3

sow

J

Exempelvis motsvarar 26 dB(W) 398 W
(se tabellen för sambandet mellan effektförhållande och dB).

\chapter{S-ENHETER OCH dB}

I kommunikationsradiomottagare brukar det
nästan alltid finnas en anordning som mäter
och visar styrkan av mottagna signaler.
Eftersom spänningen från antennen in i
mottagaren kan variera mycket, är det praktiskt att uttrycka styrkevärdena i en
misk måttenhet, s.k. s-enhet.
signalspänningen mäts över en impedans av 50 n.
Eftersom s-enheten är logaritmisk, så
motsvarar t. ex. signalstyrkan S8 halva signalspänningen, d.v.s. 25 ~V eller -6
jämfört
med S9. Om halveringen fortsätts, fås att so
(noll) motsvarar en kvarvarande signalstyrka
av0.1J.LV.
l en kortvågsmottagare alstras det ett
interntbrus meden nivå av åtminstone 0.1 J.LV.
Detta brus blandas med den inkommande
signalen. En insignal med en styrka under
under brusnivån kommer alltså inte att kunna
höras, alltså SO. Vid högre signalstyrkor än
S9 anges styrkan som S9 +ett antal dB. Det
är då frågan om mycket starka signaler.
Följande tabell gäller för det ideala sambandet mellan s-enheter och signalstyrkor
över två alternativa brusnivåer.

signalstyrkan mäts vid mottagarens antenningång, varför skillnaden i signalstyrkan
olika antenner och mottagningsriktningar samt dämpningen i antenn och nedledning kan behöva bedömas.
l kortvågsområdet (under 30 MHz) uppträder ett atmosfäriskt bredbandigt brus tillsammans med bruset från den stora mängden rundradio- m.fl. andra starka sändare.
Detta brus är mer dominerande än mottagarens interna brus. l praktiken har de flesta
KV-mottagare en högre brusnivå än 0.1 J.LV.
Över 30 MHz däremot, är det mest mottagarens interna brus som sätter gränsen för
hörbarheten av svaga signaler. Med samma
s-skala som för kortvågsom rådet, börjar man
uppfatta signaler i bruset utan att s-metern
ger utslag.
Vid IARU Region i-konferensen 1978 i
Miskolcz föreslog de nationella föreningarna
VERON (Nederländerna) och RSGB (Storbritannien) en annan s-skala över 30 MHz.
Vid konferensen 1981 i Brighton antogs förslaget som rekommendation.
Mätningar skall i båda fallen göras med
en kvasi-toppvärdesdetektor med en stigtid
av 1O ms $\pm$0.2 ms och en falltid av 500 ms.

s-enheter, rekommenderade normvärden inom IARU Region 1

S-meter
värde
s

dB m

( vid 50 .Q)

9+ 40 dB
9+ 30 dB
9+ 20 dB
9+ 10 dB
9
8
7
6
5
4
3
2
1

-33
-43
-53
-63
-73
-79
-85
-91
-97
-103
-109
-115
-121

5.0 mV
1.6 mV
500
~v
160
J.LV
50
J.LV
25
~v
12.6 ~v
6.3 J.LV
3.2 JlV
1.6 J.LV
0.8 J.LV
0.4 JlV
0.21 JlV

Över 30 MHz

Under 30 MHz
dB~ V

dB m

(U vid 50 .Q)

dBJ.LV

74
64
54
44
34
28
22
16
10
+4
-2

-53
-63
-73
-83
-93
-99
-105
-111
-i 17
-123
-129
-135
-141

500
160
50
16
5
2.5
1.26
o.63
o.32
0.16
0.08
0.04
0.02

54
44
34
24
14
8
+2
-4
-1 o
-16
-22
-28
-34

-8

-14

J.LV
J.LV
J.LV
J.LV
J.LV
J.LV
J.LV

~v
~v

J.LV
J.LV
JlV
J.LV

D-1

APPENDIX

D-2

D

\chapter{BESKRIVNINGSKOD FÖR SÄNDNINGSSLAG}

Ett resultat av World Administrative Radio
Conferece (WARC) som avhölls år 1979 var
en ny beskrivningskod för radiosändningar.
Före WARC 79 uttrycktes den nödvändiga bandbredden i termer av kHz. Till exempel användes då 0.1 för 1 00 Hz och 6000 för
6 MHz.
l dataåldern behövs ett elegantare system för taluppställningar. Det nya systemet
går ut på att uttrycka den nödvändiga bandbredden med tre siffror och en bokstav.
Bokstaven placeras på platsen för decimaltecknet och representerar enheten för bandbredd. Bokstäverna H (Hz), K (kHz), M (MHz)
och G (GHz) används, medan varken Oeller
K, M eller G får vara det första tecknet.
Numeriska värden med mer än tre signifikanta siffror rundas av till tre.
Detta benämningssystem är dock inte
utan problem. Det tar mer hänsyn till metoden hur en signal alstras, snarare än hur en
signal helt enkelt ser ut när den sänds.
Följdaktligen skall direkt modulation av
huvudbärvågen benämnas på ett sätt, medan
modulation av en underbärvåg i en enkelt
sidbandssändare med undertryckt bärvåg
skall benämnas på ett annat sätt. Om man
t. ex. nycklarett RTTY-modem med en direktskrivande fjärrskrivare och sedan bytertill en
dator för att göra samma sak, så ändras
benämningen av sändningsslaget
Bandbredd
Basbandetärdetfrekvensområde, som upptas av signaler innan de modulerar bärvågen. Signaler i basbandet ligger vanligen
mycket lägre i frekvens än bärvågen. l den
låga änden av basbandet kan frekvensen
närma sig eller vara likström (O Hz). l den
höga änden beror frekvensen på det värde
då information finns liksom att det finns underbärvågor eller andra speciella signaler
inom basbandet Det finns ett basband för
alla typer av signaler, vare sig de är analoga
eller digitala. Det skall också förstås, att
termen basband är relaterad till den modulation som avses från fall till fall.
Det kan finnas mer än ett basband i en
komplett modulationsprocess. Till exempel,
en nycklad ton som går till sändaren genom

mikrofoningången är dess analoga basband
medan nycklingspulserna till tongeneratorn
är dess digitala basband.
Sidband alstras alltid när en bärvåg moduleras. De är blandningsprodukter på båda
sidor om bärvågen, som resultat av att signaler från basbandet modulerar bärvågen
på något sätt. Det övre sidbandetkallas USB
(av upper sideband, eng.) och det undre
sidbandet LSB (av lower sideband, eng.).
l system för amplitudmodulation är bredden på sidbanden i stort lika med den högsta
frekvenskomposanten i basbandet Sidbanden är spegelbilder av varandra och innehåller exakt samma information. För att spara
bandbredd räcker det alltså med att överföra
ena sidbandet, varvid det andra sidbandet
kan undertryckas, liksom även bärvågen.
l andra modulationssystem än för amplitudmodulation däremot, kan bredden på sidbanden mycket överstiga den högsta frekvenskomposanten i basbandssignalen.
Använd bandbredd är avståndet mellan
översta och nedersta delen av ett spektrum,
där medeleffekten är lägre än 0.5$\circ$/o (-23 dB)
av den totala medeleffekten. l vissa fall kan
en annan relativ effektnivå specificeras; t ex.
i USA 0.25$\circ$/o (-26 dB) för reglering av digital
kommunikation. För amatörer är det inte
alltid lätt att bestämma den använda bandbredden. Den kan mätas med en spektrumanalysator, men ett sådant instrument är
svårtillgänglig för de flesta amatörer. Den
använda bandbredden kan även beräknas,
men det kräver matematikkunskaper i informationsteori och behandlas inte här.
Nödvändig bandbredd är den del av den
använda bandbredden, som räcker för att
säkra informationsöverföringen i den omfattning och kvalitet som krävs. Förenklade
sätt att beräkna nödvändig bandbredd vid
specifika modulationssystem finns i kapitel1.
Tilldelat frekvensband är den nödvändiga bandbredden plus två gånger den absoluta frekvenstoleransen.
Frekvenstolerans (uttryckt i del per 106 ,
procent eller i Hz) är den maximalt tillåtna
frekvensavvikelsen från den korrekta frekvensen.

E-1

(Exempel)

6M25
C3F MN
---.-- ~~~

m
l

l

1\ )

Bandbredd - - - - - - . . : .•

Huvudbärvågens modulation
N Ingen modulation
Utsändning där huvudbärvågen är linjärt modulerad
(även i fall med vinkelmodulerad underbärvåg)
A Dubbla sidband
H Enkelt sidband, full bärvåg
R Enkelt sidband, reducerad bärvåg eller bärvåg av varierande nivå
J Enkelt sidband, undertryckt bärvåg
B Sinsemellan oavhängiga sidband
c stympat sidband
Utsändning där huvudbärvågen är vinkelmodulerad
F Frekvensmodulation
G Fasmodulation
D Utsändning vars huvudbärvåg är amplitudoch vinkelmodulerad antingen samtidigt eller
i viss förutbestämd följd.
Utsändning av huvudbärvågen som tåg av pulser not.
P Omodulerade pulser
K Amplitudmodulerade pulser
L Längdmodulerade pulser
M Faslägesmodulerade pulser
Q Vinkelmodulerad bärvåg under pulsens varaktighet
V Kombination av ovanstående eller alstrat på
annatsätt
Övriga fall där utsändningens huvudbärvåg är modulerad, antingen samtidigt eller i förutbestämd följd på
två eller flera av sätten amplitud-, vinkel- eller pulsmodulering
W
övriga fall

x

not. Utsändning där huvudbärvågen är direkt

modulerad av en signal, vilken är kodad
i kvantisarad form (t.ex. pulskodmodulation) skall hänföras till amplitudeller vinkelmodulation

1

Den modulerande signalens karaktär

0 Ingen modulerande signal
En enda kanal med
1 kvantiserad eller digital information, utan användning av modulerande underbärvåg
2 kvantiserad eller digital information, med användning av modulerande underbärvåg
3 analog information
Två eller flera kanaler med
7 kvantiserad eller digital information
8 analog information
Sammansatta system av
9 en eller flera kanaler med kvantiserad ellerdigital information samt
en eller flera kanaler med analog
information
Övriga fall

x

l fråga om bassignalens karaktär skiljer
man å ena sidan på kanaler för kvantiserad eller digital information, d.v.s. där signalen växlar språngvis mellan vissa givna tillstånd, och på kanaler för analog information, där signalen kan variera kontinuerligt inom givna gränser.
Att fastställa arten av huvudbärvågens modulation kan kräva viss eftertanke. l många fall
får den information som skall överföras, modulera en underbärvåg, som i sin tur påtrycks
modulatorn för huvudbärvågen.

M l' l

f''rf
d
u t1p ex o aran e
Närmare signalbeskrivning

'

Informationens form

N Ingen överförd information
A Telegrafi
för hörselmottagning
B Telegrafi
för automatisk mottagning
C Faksimil
D Dataöverföring,
fjärrmätning,
fjärrstyrning
E Telefoni,
även ljudrundradio
F Television, video
W Kombination av
ovanstående fall
X Övriga fall

m
m

00

"<z
Jl

z

G)

:t>
"'C
"'C

m
c

z

x

00

"oc
00

)>:

z

c

z
z
G)
00
00

r

)>

G)

Telegrafisignaler är kvantiserade (till/från,
mark!space). Telefonisignaler har mestadels
varit analoga, men är allt oftare kvantiserade
(digitala). Faksimilsignaler är analoga eller
kvantiserade, beroende på om gråtoner överförs eller ej.

Obligatoriska kännetecken
för sändningsslag enligt
ITU radioreglemente (RR)

~

©

-a

-1

(Exempel)

-. l

6M25
C3F MN
---.-.....
Utsändningens bandbredd
En fullständig kodbeteckning för en radioutsändning är uppbyggd av en niostäflig teckenföljd, t. ex.
6M25 C3F MN. l teckenföljden är de första fyra
tecknen (t.ex. 6M25) bandbreddsangivelsen.
Bandbredden ska/f anges med tresiffror samt en
bokstav som decimaltecken.
Bokstaven anger även vilken enhet som bandbredden har, nämligen H för Hz, K för kHz, M för
MHz och G för GHz.
Det första tecknet får inte vara noll, K, M eller G.
Decimaltecknen används på följande sätt:
Bandbredd 0.001-999 Hz (decimaltecken H),
bandbredd 1.00-999 kHz (decimaltecken K},
bandbredd 1.00-999 MHz (decimaftecken M},
bandbredd 1.00-999 GHz (decima!tecken G).
Exempel:
O. 002 Hz skrivs H002
12.5 kHz skrivs 12K5,
O. 1 Hz
skrivs H 100 2.4kHz
skrivs 2K40,
25.3 Hz skrivs 25H3 6 kHz
skrivs 6KOO,
180. 7kHz
skrivs 181K
6.25 MHz
skrivs
6M25.

m
l

w

Det är särskift viktigt att komma ihåg bandbredden vid utsändningar nära bandgränsema. T. ex.
kommer sidbandet (USB) i en telefonisignal med
bärvågsfrekvensen 29.699, att tydligt överskrida
den övre bandgränsen för 10-metersbandet. Bandgränserna får INTE överskridas och det gäller även
sidbanden f
A v Post- och telestyrelsens föreskrifter för amatörradio framgår de tillåtna bandbreddema. Därutöver gäller att bandbredden bör hållas så smal som
möjligt med hänsyn till sändningsslaget

Obligatorisk del

Närmare beskrivning av signalen

m

m

":c<
00

Arten av mulliplex

Tvåtillståndskod med element av
N Ingen multip/ex
A Olika antal och/eller olika varaktigMultiplex med
z
het- morsetelegrafi
C Koddelning
B Samma antal och varaktighet, utan
F Frekvensdelning
C)
felkorrigering - fjärrskrift
T Tiddelning
00
C Samma antal och varaktighet, med
W Kombination av F och T
felkorrigering- fjärrskrift, AMTOR,
X Andra arter av multip/ex
paketradio m.m.
Fyratillståndskod där
D Varje tillstånd företräder ett tillstånd
Se föregående sida om
om ett antal bitar
den obligatoriska delen
00
):>:
Flertillståndskod där
av kännetecknen!
E Varje tillstånd företräder ett signalelement om ett antal bitar
F Varje tillstånd eller kombination av
tillstånd företräder ett tecken
C)
Ljud av rundradiokvalitet
00
00
G Monatoniskt
r
H stereofoniskt eller kvadrafoniskt
Ljud av kommersiell kvalitet
J Alla fall utom K och L enligt nedan
K Med användning av frekvensinversion eller banduppdelning
L Medsärskilda frekvensmodulerade signaler för styrning av den demodulerade signalens nivå
Video
M Monokrom
N Färg
Kompletterande kännetecken
Kombination av ovanstående fall

z

~

©

~

o

"oc
z
z
z
c

>

"

w
Övriga fall
x

för sändningsslag enligt
ITU radioreglemente (RR)

)'>
'1J
'1J

m
c

z

x

BESKRIVNINGSKOD

SÄNDNINGSSLAG

Exempel på fullständigt beskrivna sändningsslag
N0N
1OOH A 1A AN

Omodulerad bärvåg, ingen överförd information.
Morsetelegrafi genom nyekling av bärvåg, 125-takt, bandbredd 100 Hz, s.k. CW.
16KO F2A AN Morsetelegrafi, frekvensmodulation med nyekling av ton,
t.ex. i repeater, s.k. tontelegrafi.
254H F1 B BN Fjärrskrift genom frekvensskiftnyckling av bärvåg (FSK),
utan felkorrigering, hastighet 50 Bd, frekvensskift i 70Hz,
s.k. RTTY.
254H J2B BN Fjärrskrift genom frekvensskiftnyckling av modulerande
tonpar (AFSK), vid sändning av enkelt sidband med undertryckt bärvåg, bandbredden beroende av hastighet och
frekvenser i tonparet
Jfr 254H Fi B BN,
304H Fi B CN Fjärrskrift genom frekvensskiftnyckling av bärvåg (FSK),
med felkorrigering, hastighet 100 Bd, frekvensskift 170
Hz, t.ex. AMTOR. Jfr 254 Fi B BN.
6KOO A3E JN Telefoni, amplitudmodulation med dubblasidband och full
bärvåg, bandbredd 6kHz, s.k. AM.
2K70 J3E JN Telefoni, enkelt sidband och undertryckt bärvåg, bandbredd 2.7 kHz, s.k. SSB.
16KO F3E JN
Telefoni, frekvensmodulation, bandbredd 16 kHz, s.k.
NBFM (smalbands-FM).
2K12 F3C MN Faksimil med halvtoner (telefoto), kooperationsindex 264,
avsökningshastighet 90 linjer/minut, frekvensmodulering
med $\pm$ 400 Hz skift.
6M25 C3F MN Television i svartvitt enligt det europeiska 625-linjerssystemet.
3KOO F3F MN Smalbandstelevision enligt amatörradiostandard, s.k. ATV.

Exempel på sändningsslag utan ITU beskrivningskod enligt ovan
Telefoni, amplitudmodulation med dubbla sidband och
reducerad bärvåg.
En enda kanal med analog information.
Sändningsslaget tillämpas i effektbesparande syfte bl.a.
för rundradiosändningar på "AM", varvid traditionella
rundradiomottagare fortfarande kan användas.

E-4

\chapter{IARU Region 1 bandplan}

Sammanfattad av SM3AVQ Lars

Denna bandplan reviderades vid lA RU Region i -konferensen i Tel-Aviv 1996.
Den vänstra delen är själva bandplanen, medan den högra delen rekommenderar användning/mötespunkter.
(PTS bandplan och status för amatörradio i Sverige, framgår av Kapitel III 1.6 samt Appendix G och H.)

Band
MHz

Segment
kHz

Trafiksätt

1.8

i 81 o - 1838

CW enbart
Digitala trafiksätt men ej Packet Radio, CW
Digitala trafiksätt men ej Packet Radio, Telefoni, CW
Telefoni, CW
(i Sverige 1842-1850)

1838- 1840
1840- 1842
1842-2000

o

3.5

3500- 351
3500-3560
3560-3580
3580-3590
3590-3600
3600-3620
3600-3650
3650-3775
3700-3800
3730-3740
3775-3800

CW enbart, DX-fönster för interkontinentala kontakter
CW enbart, segment för CW-tester
CW enbart
Digitala trafiksätt, CW
Digitala trafiksätt företrädesvis Packet Radio, CW
Telefoni, Digitala trafiksätt, CW
Telefoni, Segment för Telefoni-tester, CW
Telefoni, CW
Telefoni, Segment för Telefoni-tester, CW
SSTV \& FAX, Telefoni, CW
Telefoni, DX-fönster för interkontinentala kontakter

7

7000-7035
7035-7040
7040-7045
7045-7100

CW enbart
Digitala trafiksätt men ej Packet Radio, SSTV \& FAX, CW
Digitala trafiksätt men ej Packet Radio, SSTV \& FAX, Telefoni, CW
Telefoni, CW

10

10100-10140
10140-10150

CW enbart
Digitala trafiksätt men ej Packet Radio, CW

14

14000- 14070
14000 - 14060
14070 - 14089
14089 - 14099
14099-14101
14101-14112
14112 - i 4125
14125-14300
14230
14300 - 14350

CW enbart
CW enbart, Segment för CW-tester
Digitala trafiksätt, CW
Digitala trafiksätt företrädesvis Packet Radio, CW
Exklusivt fyrband IBP
Digitala trafiksätt företrädesvis Packet Radio forwarding, Telefoni, CW
Telefoni, CW
Telefoni, Segment för Telefoni-tester, CW
SSTV \& FAX anropsfrekvens
Telefoni, CW

18

18068 - 181 00
18100-18109
18109- 18111
18111 - i 8168

CW enbart
Digitala trafiksätt, CW
Exklusivt fyrband l BP
Telefoni, CW

21

21 000 - 21 080
21 080 - 21 i 00
21100-21120
21120-21149
21149-21151
21 i 51 -21450
21340
24890 - 24920
24920 - 24929
24929- 24931
24931 - 24990

CW enbart
Digitala trafiksätt, CW
Digitala trafiksätt företrädesvis Packet Radio, CW
CW enbart
Exklusivt fyrband IBP
Telefoni, CW
SSTV \& FAX anropsfrekvens
CW enbart
Digitala trafiksätt, CW
Exklusivt fyrband l BP
Telefoni, CW

24

F-1

APPENDIX
Trafiksätt

Band
MHz

segment
kHz

28

28000 - 28050
CW enbart
28050- 28120
Digitala trafiksätt, CW
28120- 28150
Digitala trafiksätt företrädesvis Packet Radio, CW
28150- 28190
CW enbart
28190-28199
Regionella fyrar med tidsdelningsschema !BP
28199- 28201
Världstäckande fyrnät med tidsdelnings-schema !BP
28201 - 28225
Kontinuerligt sändande fyrar l BP
28225- 29200
Telefoni, CW
28680
SSTV \& FAX anropsfrekvens
29200- 29300
Digitala trafiksätt (NBFM Packet Radio), Telefoni, CW
29300 - 2951 o
satellit utfrekvens (nerlänk)
2951 O- 29700
Telefoni (29 MHz FM-band, se nedan), CW
FM Frekvensuppdelning
2951 O
Del bandkant, används ej
29520 - 29550
FM Simplex
29560 - 29590
Repeater infrekvenser, 1O kHz frekvensavstånd
29600
Anropsfrekvens
2961 O - 29650
FM Simplex
29660 - 29690
Repeater utfrekvenser, 1O kHz frekvensavstånd
29700
Bandkant, används ej

29

ANMÄRKNINGAR
Prioritet:
När flera trafiksätt förekommer på samma frekvenssegment, har det trafiksätt företräde, som här nämns
först. Detta sker dock under vad som kallas Noninterference Basis, NIB (icke störande grundval).
Digitala trafiksätt:
Omfattar BaudoVRTTY, AMTOR, PACTOR, CLOVER,
ASCII, Packet Radio.
Observera undantagen för 1.8, 7 och i OMHz där Packet
Radio ej ingår i Digitala Trafiksätt
Telefoni:
Alla slag av detta trafiksätt inkluderas. Upp till 1O MHz
skalllägre sidbandet (LSB) användas och över 1O MHz
det övre sidbandet (USB).
3500-351 O och 3775-3800 kHz:
Interkontinental trafik skall ges företräde på dessa segment.
segment för tester:
Då DX-trafik ej är involverad skall testsegmenten ej
innefatta 3500- 351 O eller 3775- 3800kHz. Medlemsföreningarna tillåts sätta andra, smalare, segment för
sina nationella tester (inom testsegmenten).
Rekommendationen om testsegment gäller inte tester
med digitala trafiksätt.
Testaktivitet skall ej äga rum på 1O, 18 och 24 MHzbanden.
7 och 10 MHz:
Användande av Packet Radio på 7 och 1OMHz avråds.
7035-7045 får under dygnets ljusa timmar användas av
Packet Radio forwarding-stationer i Afrika söder om
ekvatorn.
1O MHz-bandet:
Vid nödtrafik får även SSB användas på detta band.

F-2

Obemannade stationer som använder digitala trafiksätt
skall undvika att använda 1O MHz- bandet.
Nyhetsbulletiner skall ej sändas på 1O MHz- bandet.
10120- i 0140 kHz får under dygnets ljusa timmar användas av SSB-stioner i Afrika söder om ekvatorn.
SSTV l FAX:
Frekvenserna 14230, 21340 och 28680 bör användas
som anropsfrekvenser för SSTV- och FAX-operatörer.
Efter att kontakt erhållits skall dessa flytta till annan
ledig frekvens inom telefonidelen av bandet.
satellitbandet 29300-2951 O kHz:
Medlemsländerna skall råda amatörerna att inte sända
FM på frekvenser mellan 29300 och 2951 O kHz. Detta
för att undvika interferens med satelliternas nerlänk.
Obemannade sändare:
IARU:s medlemsländer äro uppmanade att begränsa
denna aktivitet på kortvågsbanden.
Obemannade stationer på kortvåg skall endast aktiveras under en operatörs kontroll.
Med detta menas att stationer ej skall aktiveras av
exempelvis ett program i en dator. Aktivering skall ske
av SYSOP eller av en uppropande station efter att de
avlyssnat frekvensen och funnit att den är ledig.
Undantag gäller för fyrar och speciella experimentstationer.
Sändarfrekvenser:
l bandplanen angivna frekvenser är "sändarfrekvenser"
(inte frekvensen för den undertryckta bärvågen).
NBFM Packet Radio på 29 MHz-bandet:
Rekommenderade frekvenser på varje '1 O kHz f.o.m.
2921 Ot.o.m. 29290 kHz. En deviation av plus/minus 2.5
kHz skall användas med max 2.5 kHz modulationsfrekvens.

\chapter{IARU Region 1 bandplan VHF/UHF/SHF/EHF}

Sammanfattad av SM?GVF Kjell

Denna bandplan reviderades vid IARU Region i-konferensen i Tel-Aviv i 996.
Den vänstra delen är själva bandplanen, medan den högra delen rekommenderar användning/mötespunkter.
(PTS bandplan och status för amatörradio i Sverige, framgår av Kapitel !111.6 samt Appendix G och H.)

50MHz
50,000

l

Användning: Experimentband, rundradio primär, landmobil radio tillåten

cw

50,100

50,500

Alla smalbandsmoder
(CW, SSB, AM, RTTY,
SSTV, ETC)
Smalband = 6 kHz

Alla moder
50,190
51,210

l

51,390
51,410

l

51,590
51,810

l

51,990
52,000

50,020 - 50,080
50,090

Fyrar
CW aktivitetscenter

50,100-50,130
50,11 o
50,150
50,185
50,200

SSB/CW internationellt (interkontinentalt)
DX anropsfrekvens interkontinentalt
SSB aktivitetscenter
Aktivitetscenter för crossband
MS aktivitetscenter

50,51 o
50,550
50,600
50,620- 50,750

SSTV (AFSK)
FAX arbetsfrekvens
RTTY (FSK)
Digital kommunikation

RF81
NBFM repeater infrekvenser, 20kHz kanaldelning, i O kHz kanalbredd, varannan kanal används
RF99
F41
NBFM, simplex
51,510
NBFM anropsfrekvens
F59
RF81
NBFM repeater utfrekvenser, 20 kHz kanaldelning, 1O kHz kanalbredd, varannan kanal används
RF99

ANMÄRKNINGAR
Sändningsslag

Telegrafi är tillåtet över hela bandet, men är exklusivt i området 50,000 - 50,100 MHz

Anropsfrekvenser

50,11 O MHz är interkontinental DX anropsfrekvens och bör inte användas för trafik inom Europa.

Kanaltrafik

För kanaltrafik är kanaldelningen 20 kHz, förskjutet 1O kHz.

Tillståndskrav i Sverige

För amatörradiotrafik i bandet 50 - 52 MHz, krävs i Sverige utöver amatörradiotil/ståndet ett särskilt tillstånd !

F-3

APPENDIX F2
144 MHz

Användning: Amatörradio primär

144,000

CW(a)
144,150
SSB

l

144,400

144,000- 144,035
144,050
144, i 00
144,140 - i 44,150
i 44,150 - 144,160
144,195 - i 44,205
144,300
144,390 - 144,400

EME (månstuds) exklusiv användning
CW anropsfrekvens
CW MS referensfrekvens, random
CW, FAl (Field Aligned lrregularities)
SSB, FAl (Field Aligned lrregularities)
SSB MS (Meteorscatter), Random
SSB anropsfrekvens
SSB MS (Meteorscatter), Random

144,490
144,500
144,525
144,600
144,700
144,750

SAREX uplink, temporär
SSTV anropsfrekvens
ATV SSB talk back center
RTTY anropsfrekvens
FAX anropsfrekvens
ATV anropsfrekvens

Fyrar (b)

l

144,490
144,500
Alla moder (c)

144,800

l

144,990
145,000

l

145, i 875
145,200

l

i45,5875
145,600

l

i45,7875
145,800

l

i46,000

Digital kommunikation (d)
RV48
NBFM repeater infrekvenser, 12,5 kHz kanalseparation, 600kHz skift (e)
RV 63
145,200
Bemannad rymdtrafik, upplänk
V16
RTTY lokal
i 2,5 kHz NBFM
i45,300
(Mobil) anropsfrekvens
i45,500
simplex
V47
RV 48
NBFM repeater utfrekvenser, 12,5 kHz kanalseparation, 600 kHz skift
RV63
145,800
Bemannad rymdtrafik, nerlänk
satellitservice

ANMÄRKNINGAR
Generella
i) l Europa skall inga in- eller utfrekvenser för NBFM repeatrar förekomma inom segmentet 144 - 145 MHz.
ii) Med undantag för satellitsegmentet tillåts inte in- eller utfrekvenser i 2-metersbandet för repeatrar i andra band.
iii) Inga nya nät för packet radio skall sättas upp i 2-metersbandet.
Access från nät i 2-metersbandet till nät i andra band skall inte förekomma. Emellertid får detta förekomma
under begränsad tid i delar av Region 1 för att där introducera packet radio. De delar av regionen som avses är
där amatörtätheten är låg och/eller i regionens utkanter där sådan access inte påverkar trafiken i de delar av
regionen där stort tryck på tillgång till spektrum motiverar att bandplanen följs metodiskt. Denna andra del av
fotnoten skall aldrig användas för att legitimera att första stycket ignoreras för avsevärd tid.
iv) Fyrar skall oavsett ERP ligga i fyrbandet
Särskilda
(a) Telegrafi är tillåtet över hela bandet, exklusivt i segmentet 144,035- 144,150 MHz.
(b) Fyrar med ERP över 50 W koordineras av IARU Region 1 fyrkoordinator. Förfyrar med 1OW eller mer skall denne
meddelas. Under begränsad tid- inte längre än att noviser i Nederländerna har detta segment tillgängligt- är
även SSB och CW tillåtet i detta segment.
(c) Inga obemannade stationer skall användas i all mode-segmentet.

F-4

F2

APPENDIX

(d) stationer i nätverk för digital trafik skall använda den digitala delen av bandet och tillåtas för en begränsad tid.
Dessa bör ha access till portar på andra VHF-, UHF- eller mikrovågsband och bör inte använda 2-metersbandet
för forward-trafik till andra nätverksstationer. Nya nätverksstationer uppmuntras inte. Obemannade stationer
tillåts endast i segmentet 144,800 - 144,990 MHz. Utanför detta segment skall sidband inte överstiga -60dB i 12
kHz band bredd. Hänsyn skall tas till den använda bandbredde n, så att sidband ej faller utanför segmentet. För
stationer med 12,5 kHz bandbredd betyder det att kanalerna 144,8125 - 144,975 MHz då kan användas.
(e) En övergång till genuint i 2,5 kHz kanalsystem uppmuntras.
144,140 - 144,160 MHz utgör ett alternativt segment för EME.

432 MHz
432,000

l

432,150

l

432,500

l

432,600

Användning: Amatörradio och radiolokalisering delat primär
CW (a)

432,000 - 432,025
432,050

Månstuds
CW aktivitetscenter

SSB/CW

432,200
432,350

SSB aktivitetscenter
Mikrovågor "talk-back" center

Linjära transpondrar, in

432,500
432,600
432,700
432,700- 432,775

SSTV (smalband)
RTTY (FSK/PSK)
FAX (FSK)
Digital kommunikation, ej mer än
25 kHz kanalseparation

Linjära transpondrar, ut

l

432,800

l

432,990
433,000

l

433,3875
433,400

l

433,5875
433,600

Fyrar (b)
RU368
NBFM repeater infrekvenser, 12,5 kHz kanalseparation, 1,6 MHz skift
RU399
U272
Simplex, i 2.5 kHz
SSTV (FM/AFSK)
433,400
433,500
(Mobil) FM anropskanal
U287

Alla moder

434,575
434,600

433,600
433,625 - 433,775
433,700
434,450 - 434,575

RTTY (FM)
Digital kommunikation
FAX (FM/AFSK)
Digital kommunikation, ej mer än
25 kHz kanalseparation

l

RU 368
NBFM repeater utfrekvenser, 12,5 kHz kanalseparation, i ,6 MHz skift
RU 399

l

satellitservice

434,9875
435,000
438,000

Förslag till packet duplex frekvenser: 432,700/434,500 - 432,775/434,575 MHz.
ANMÄRKNINGAR

Generella

i) l Europa skall inga in- eller utfrekvenser för NBFM repeatrar förekomma inom segmentet 432 - 433 MHz.
ii) Fyrar skall oavsett ERP placeras i fyrbandet

Särskilda

(a) Telegrafi är tillåtet över hela bandet, exklusivt i segmentet 144,035- 144,150 MHz.
(b) Fyrar med ERP över 50 W koordineras av IARU Region 1 fyrkoordinator.

F-5

APPENDIX
1296 MH:z
1240.000

Användning: Amatörradio sekundär
Alla moder

1240,000-1241,000
1242,025- 1242,700
i 242,725- 1243,250

Digital kommunikation
Repeater ut, RS1 - RS28
Packet duplex, RS29 - RS50

l

Amatörtelevision

1258, i 50- i 259,350

Repeater ut, R20 - R68

l

satellitservice
1270,025- 1270,700
1270,725- 1271,250

Repeater in, RSI - RS28
Packet duplex, RS29 - RS50

l

1243,250
1260,000
1270,000

l

1272,000

l

12l000
1291,475
1291,500

Alla moder

Amatörtelevision
RMO
(används i Sverige)
NBFM repeater infrekvenser
25 kHz kanalseparation, 6 MHz skift
RM19

l

Alla moder

i 293,150 - i 294,350

Repeater in, R20 - R68

l

CW(a)

i 296,000 - 1296,025

Månstuds

1296,200
1296,400 - 1296,600
1296,500
1296,600
i 296,600 - i 296,800
1296,700

Smalbands aktivitetscenter
Linjär transponder infrekvens
SSTV
RTTY
Linjär transponder utfrekvens
FAX

1296,000
1296,150

SSB

1296,800

l

1296,990
1297,000

l

1297,475
1297,500

l

1297,975
1298,000

1300,000

Fyrar

RMO
(används i Sverige)
NBFM repeater utfrekvenser,
25 kHz kanalseparation, 6 MHz skift
RM19
SM20
NBFM simplex kanaler,
25 kHz kanalseparation,
SM39

1297,500

FM aktivitetscenter

Alla moder

1298,025- 1298,700
i 298,500 - i 300,000
1298,725- 1299,000

Repeater ut, RS1 - RS28
Digital kommunikation
Packet duplex, RS29 - RS40

ANMÄRKNINGAR
Särskilda
(a) Telegrafi är tillåtet över hela smalbandsegmentet, exklusivt i segmentet 1296,000 - i 296, i 50 MHz.
(b) Fyrar med ERP över 50 W koordineras av IARU Region 1 fyrkoordinator.

F-6

APPENDIX F2
2300 Mhz
2300,000

l

Användning: Amatörradio sekundär
Subregional planering

2320,000

l
l

CW

2320,000- 2320,025

Månstuds

CW/SSB

2320,200

SSB aktivitetscenter

2320,150
2320,800

l

Fyrar

2320,990
2321,000

Simplex och repeater, NBFM

2322,000
Alla moder

2400,000

l

2322,000- 2355,000
2355,000- 2365,000
2365,000- 2370,000
2370,000- 2392,000
2392,000 - 2400,000

Amatörtelevision
Digital kommunikation
Repeatrar
Amatörtelevision
Digital kommunikation

satellitservice

2450,000
5650 MH:z
5650,000

l

5670,000
5661'000
567,,000
5700,000

l
l
5760,000
l
5762,000
l
5790,000
l

5720,000

Användning: Amatörradio sekundär
Satellitservice, upplänk
Smalband, CW/SSB/FM

5668,200

Aktivitetscenter

5760,200

Aktivitetscenter

Digital kommunikation
Amatörtelevision
Alla moder
Smalband, CW/SSB/FM
Alla moder
Satellitservice, nerlänk

5850,000

10000 MH:z
10000,000

l

10150,000

l
l
10350,000

10250,000

Användning: Amatörradio sekundär
Digital kommunikation
Alla moder: ATV, data, FM simplex/duplex/repeatrar
Digital kommunikation

l
l

Alla moder

l

Alla moder

l

satellitservice

10368,000
10370,000
10450,000

Smalband CW/SSB/fyrar

10368,200

Aktivitetscenter

10500,000

F-7

PPENDIX

F2

24000 MHz Användning: Amatörradio sekundär
24000,000

l

satellitservice

l

CW/SSB/fyrar

24048,200

Aktivitetscenter, smalbandsmoder

Alla moder

24125,000

Aktivitetscenter,
bredbandiga moder

24048,000
24050,000

l
24250,000

47000 MHz Användning: Amatörradio primär
47000,000

l

47200,000

F-8

47088,000

Aktivitetscenter, smalbandsmoder

\chapter{SVENSKA BANDPLANER}
Vart och ett lands teleadministration utfärdar föreskrifter för amatörradio i sitt land. Dessa
föreskrifter griper naturligtvis över IARU :s bandplaner, vilka endast är rekommendationer för
hur tilldelade frekvensband bör disponeras.
Post- och telestyrelsens föreskrifter PTSFS 1994: 5, (utdrag ur bil. 1-2)
FREKVENSBAND

1810-1850
10100-10150
18068 - 18168
24890 - 24990

kHz
kHz
kHz
kHz

3500-3600
3600-3800
7000-7040
7040-7100
14000 - 141 00
141 00 - 14350
21000-21150
21150 - 21450
28000 - 28200
28200 - 29700

kHz
kHz
kHz
kHz
kHz
kHz
kHz
kHz
kHz
kHz

144-146
432-438
1240- 1300
2300-2450
5650 - 5850
10,0- 10,5
24,0 - 24,25
47,0- 47,2
75,6- 76,0
76-81
142-144
144- 149
241 -248
248-250

MHz
MHz
MHz
MHz
MHz
GHz
GHz
GHz
GHz
GHz
GHz
GHz
GHz
GHz

SÄNDNINGSKLASSER
Tillstånd enligt
certifikatsklass
CEPT1
l, Il (endast J3E)
l, Il, III

1,11,111

l, Il, III

l
1,11,111
l

l, Il, III
l
l, Il, III
l
1,11,111
l

1,11,111, IV
Tillstånd enligt
certifikatsklass
CEPT 1 och CEPT 2
l, Il, III, IV
l, Il, III, IV, V
l, Il, III, IV, V
l, Il, III, IV, V, VI
l, Il, III, IV, V, VI
l, Il, III, IV, V, VI
l, Il, III, IV, V, VI
l, Il, III, IV, V, VI
l, Il, III, IV, V, VI
l, Il, III, IV, V, VI
l, Il, III, IV, V, VI
l, Il, III, IV, V, VI
l, Il, III, IV, V, VI
l, Il, III, IV, V, VI

UTEFFEKT

AMATÖRRADIOSTATUS

1000 w
150 w
1000 w
1000 w

Primär
sekundär
Primär
Primär

1000W
1000 w
1000 w
1000 w
1000 w
1000 w
1000 w
1000 w
1000 w
1000W

Primär
Primär
Primär
Primär
Primär
Primär
Primär
Primär
Primär
Primär

1000 W
1000 W
1000 W
1000 W
1000 W
1000 W
1000 W
1000 W
1000 W
1000 W
1000 W
1000 W
1000 W
1000 W

Primär
Delad primär
sekundär
sekundär
sekundär
sekundär
sekundär
Primär
Primär
sekundär
Primär
sekundär
sekundär
Primär

Primär tjänst har företräde före tjänst med lägre status. Observera att flera tjänster kan ha
delad primär status i ett band, som till exempel i 3500 kHz- och 432 MHz-bandet.
Ovanstående är den föreskrivna statusen för amatörradio i Sverige när denna bok trycktes.
Mer om sändningsslagen per sändningsklass härovan på följande sida och i Appendix E.

G-1

APPENDIX
SSA:s anvisningar 1995: 1, bilaga 1, avseende SSA-tillstånd klass UN och UC
Med stöd av ett särskilt amatörradiotillstånd för SSA, utger SSA anvisningar SSA 1995: 1
avseende utbildningstrafik i de frekvensband som anges i anvisningarnas bilaga 1. Bandplanen m. m. framgår här nedan.
För ändamålet meddelar SSA på anvisade villkor SSA-tillstånd klass UC respektive UN.
Därutöver skalllARV Region 1 bandplan följas i tillämpliga delar.
Märk, att sändning under SSA-tillstånd endast får ske från Sverige!

SÄNDNINGSKlASSER

FREKVENSBAND

UTEFFEKT

AMATÖRRADIOSTATUS

Tillstånd enligt
certifikatsklass

uc

3500-3800
7000-7100
21000- 21450
28000 - 28200
28200 - 29700

144- 146
432-438

kHz
kHz
kHz
kHz
kHz

l
l
l
l
l' Il

100W
100W
100W
100W
100W

Primär
Primär
Primär
Primär
Primär

MHz
MHz

Tillstånd enligt
certifikatsklass
UC och UN
l, Il, III, IV
l, Il, III, IV, V

100W
100W

Primär
Delad primär

Primär tjänst har företräde före tjänst med lägre status. Observera att flera tjänster kan ha
delad primär status i ett band, som till exempel i 3500 kHz- och 432 MHz-bandet.

Sändningsslag per sändningsklass

l PTS föreskrifter (och i tillämpliga fall i SSA:s anvisningar) anges tillåtna sändningslag för
vart och ett frekvensband med avseende på operatörens tillståndsklass. Sändningsslagen
är grupperade i sändningsklasser.
Respektive sändningsklass (grupp) omfattar följande sändningsslag:
Grupp!
A1A,J2B,J2D,F18,F1D,
Grupp Il
A3E, H3E, R3E, J2C, J3E, 8KOOF3E, 8KOOG3E,
Grupp III
H3C, R3C, J3C, J3D, 3KOOC3F, 3KOOF3F,
Grupp IV
A2A,A2D,H2D,R2D,J2D,F1A,F2A,F28,F2D,A3C,F3C, 16KOF3E,
3KOOG3E,
Grupp V
C3F, F3F, 36KOF3E, 36KOG3E,
Grupp VI
F3E, G3E, P1 A, P2A, K1 E.
l Appendix E beskrivs sändningsslagen närmare.

G-2

\chapter{Frekvenser för svenska amatörradiorepeatrar}
Vid direktförbindelser på höga frekvenser är
räckvidden begränsad, särskiltvid låg effekt
och små antenner. Med repeatrar
tio ner) med högt belägna antenner
räckvidden förbättras, vilket underlättar kommunikation med rörliga (mobila) radiostationer.
Eftersom sändaren och
i en
repeater arbetar samtidigt, måste avståndet
mellan deras arbetsfrekvenser vara så stort
att det inte uppstår ömsesidiga störningar.
Dessa arbetsfrekvenser kallas frekvenspar eller kanal och avståndet mellan dem
kallas repeaterskift, vilket är enhetligt inom
repeaterbandet

Frekvensparet i en repeater måste arbeta med omvänt frekvensläge i förhållande
till det i de stationer som den betjänar.
Kanalavståndet mellan repeatrarna i ett
band är också enhetligt och sändningar över
repeatrarna måste naturligtvis ha mindre
bandbredd än kanalavståndet
Inom IARU har man enats bl.a. om frekvensparen för smalbandiga FM-repeatrar.
Se IARU:s bandplaner i Appendix F.
Frekvensplaner finns för repeatrar inom
banden 51-52 MHz (6 m), 145-146 MHz (2
432-438 MHz (70 cm), 1240-1300 MHz
cm) samt 28000-29700 kHz (1 O m).

Nya kanalnumreringsmetoden
l och med införandet av i
kHz kanalavstånd på 2 meters- och 70 cm-banden har
ett nytt enkelt system införts. Man börjar med
en bokstav som talar om vilket band det är:
F för 51 MHz, kanalavstånd 1O
V för 145 MHz, kanalavstånd 12,5 kHz,
kHz.
U för 430 KHz, kanalavstånd 1
Kanalnumret börjar med 00 på varje sådant band och ökar med ett (1) för varje kanal
i bandet. På 51 och 145 MHz används
siffrig numrering och på 430 MHz tresiffrig.
För repeaterfrekvenser sätts ett R före bandbokstaven.

70-centimetersbandet (skift 1600kHz)
Kanal
Din sändarDin mottagarnr
frekvens MHz frekvens MHz
433,000
RU368
434,600
433,0125
RU369
434,6125
RU370
433,025
434,625
RU371
433,0375
434,6375
433,050
RU372
434,650
434,6625
433,0625
RU373
434,675
433,075
RU374
434,6875
433,0875
RU375
434,700
433,100
RU376
433,1125
434,7125
RU377
433,125
434,725
RU378
433,1375
434,7375
RU379
RU380
433,150
434,750
RU381
433,1625
434,7625
RU382
433,175
434,775
RU383
433,1875
434,7875
433,200
434,800
RU384
433,2125
434,8125
434,825
RU386
433,2375
434,8375
RU387
433,250
434,850
RU388
433,2675
434,8675
RU389
RU390
434,875
433,2875
RU391
434,8875
RU392
433,300
434,900
433,3125
434,9125
RU393
433,325
434,925
RU394
433,3375
434,9375
RU395
433,350
434,950
RU396
434,9675
RU397
433,375
434,975
RU398
433,3875
434,9875
RU399

2-metersbandet (repeaterskift 600
Kanal
Din sändarDin mottagarnr
frekvens MHz
frekvens MHz
RV48
145,000
145,600
RV49
145,0125
25
1
RV 50
145,025
1
RV51
145,0375
i
RV 52
145,050
145,650
RV 53
145,0625
1
RV 54
i45,075
1
RV55
145,0875
1
RV 56
145,100
i
RV 57
145,1125
1
RV 58
145,125
1
RV59
145,1375
i
RV60
145,150
1
RV61
145,1625
1
RV62
145,175
1
RV63
145,1875
i

H-1

APPENDIX

E

H

23-centimetersbandet (skift 6000kHz)
Kanal
nr

RMO

RM1
RM2
RM3
RM4
RM5
RM6
RM7
RM8
RM9
RM10
RM11
RM12
RM13
RM14
RM15
RM16
RM17
RM18
RM19

Din sändarfrekvens MHz
1291,000
1291,025
1291,050
1291,075
1291 '1 00
1291 '125
1291 '150
1291,175
1291,200
1291,225
1291,250
1291,275
1291,300
1291,325
1291,350
1291,375
1291,400
1291,425
1291,450
1291,475

Din mottagarfrekvens MHz
1297,000
1297,025
1297,050
1297,075
1297,100
1297,125
1297,150
1297,175
1297,200
1297,225
1297,250
1297,275
1297,300
1297,325
1297,350
1297,375
1297,450
1297,475
1297,450
1297,475

Repeaterband med speciella egenskaper

6-metersbandet (skift 600kHz)
Din sändarDin mottagarKanal
nr
frekvens MHz frekvens MHz
RF81
51.21 O
51.81 O
RF83
51.230
51.830
RF85
51.250
51.850
RF87
51 .270
51 .870
RF89
51 .290
51 .890
RF91
51.31 O
52.91 O
RF93
51.330
52.930
RF95
51 .350
52.950
RF97
51.370
52.970
RF99
51.390
52.990
(dvs endast udda kanalnummer används).
Observera att det i Sverige, utöver amatörradiotillståndet, t.v. krävs särskilda tillstånd
för amatörradioanvändning i detta band.
På grund av den relativt låga frekvensen
uppnås ofta överräckvidder p.g.a. sporadisk
vågutbredning via E-skiktet. Man kan då
uppnå förbindelser utan hjälp av repeater.
Observera, att i Sverige f.n. inga repeatrar
finns i 6-metersbandet.
1O- metersbandet (skift 100kHz)
Kanal
Din sändarDin mottagarnr
frekvens kHz
frekvens kHz
29560
29660
29570
29670
29580
29680
29590
29690
På grund av den relativt låga frekvensen
uppnås stora räckvidder genom jonosfärisk
vågutbrednin.g, särskilt under år med högt
solfläckstaL Aven sporadisk vågutbredning
via E-skiktet förekommer. l båda fallen bör
repeatertrafik undvikas.

H-2

\chapter{RAPPORTKODER}
Det finns olika sätt och system att rapportera hur en radiostation hörs.

Amatörradiotrafik
l amatörradiotrafik används RST -koden vid
rapportering av telegrafisignaler och RSMkoden för telefonisignaler. Namnet kommer
av begynnelsebokstäverna i de engelska
orden
Readability
(läsbarhet),
Signal strenght
(signalstyrka),
Tone
(ton)
Modulation
(modulation).
R-skala (läsbarhet)

1 Oläsbar

2
3
4
5

Knappt läsbar, enstaka ord tydbara
Läsbar med stor svårighet
Läsbar med obetydlig svårighet
Helt läsbar

s-skala (signalstyrka)
1 Sigalerna knappt uppfattbara
2 Mycket svaga signaler
3 Svaga signaler
4 Något svaga signaler
5 Ganska goda signaler
6 Goda signaler
7 Mycket goda signaler
8 starka signaler
9 Mycket starka signaler

Kommersiell sjö- och luftradiotrafik

1kommersiell sjö- och luftradiotrafik används
t.ex. Q-förkortningarna QSA (signalstyrka),
ORM (störningar från annan station), QRN
(atmosfäriska störningar), QSB (fädning) och
ORK (uppfattbarhet) åtföljda av en siffra för
graden i skala 1-5. Jämför med SINPOkoden.
Exempel:
"QSA 5, ORK 3, QRN 1", vilket betyder
"ljudstyrka mycket god, uppfattbarhet ganska god, störningar från andra stationer
måttliga, atmosfäriska störningar obefintliga".
Se vidare i avsnitt 111.1.2

M-skala (modulation)
1 Moduleringen oförståelig
2 Mycket dålig modulering, p.g.a.
parasitsvängningar eller annan orsak
3 Dålig modulering, p.g.a. obehöriga
frekvensändringar hos bärvågen i takt
med moduleringen
4 Ganska god modulering, låter klippt
p.g.a. övermodulering, överstyrning
etc
5 God modulering, helt felfri.

T-skala (ton)
1 Mycket rå växelströmston, ostabil och omusikalisk
2 Mycket rå växelströmston, stabil men musikalisk
3 Rå växelströmston, ostabil och omusikalisk
4 Rå växelströmston. stabil och någorlunda musikalisk
5 Tydligt växelströmsmodulerad ton, ostabil men musikalisk
6 Tydligt växelströmsmodulerad ton, stabil och musikalisk
7 Nästan ren likströmston, ostabil och med tydligt brum
8 Nästan ren likströmston, med spår av brum eller ojämnheter
9 Abalut ren likströmston, stabil

Dessutom kan följande tillägg till T-skalan förekomma:
x Absolut ren likströmston, mycket stabil, kristallklar, mjuka tecken utan knäppar
c Absolut ren, men ostabil likströmston vid nycklingen
k Knäppar alstras vid nycklingen

J-1

APPENDIX
Rundradiosändningar m.m.

För rapportering till rundradiostationer m.m.
förekommer ett system som kallas SINPO
eller SINPFEMO-koden.
Förr användes SINPO för radiotelegrafi
och SINPFEMO för radiotelefoni. Numera
används enbart SINPO-koden.
Namnet på koden kommer av begynnelsebokstäverna i orden
Signal strength (signalstyrka),
lnterference (störningar från annan radiosändning),
Noice (atmosfäriska störningar),
Propagation disturbance (vågutbredningsstörningar),
Frequency of fading (fädningsfrekvens),
Emission quality (modulationskvalitet)
Modulation depth (modulationsgrad),
Over all merit (sammanfattande omdöme).

Rapporten inleds med koden SINPO eller
SINPFEMO följd av fem resp. åtta siffror,
vilka var och en i tur och ordning graderar
egenskaperna i skala 1-5. För icke bedömda
egenskaper skall bokstaven x sättas i stället
för en siffra.
Kod

Grad

s

1
2
3
4
5

Koder Grad

l,

p

1
2
3
4
5

Kod
F

Grad

1
2
3
4
5

Kod

E

1
2
3
4
5

Kod
M

Grad

Kod

Grad
1

o

J-2

Grad

1
2
3
4
5

2
3
4
5

Bedömning
Knappt uppfattbar
Dålig
Tillfredsställande
God
Utmärkt
Bedömning
Mycket stark
stark
Måttlig
Svag
Ingen
Bedömning
Mycket snabb
Snabb
Måttlig
Långsam
Ingen
Bedömning
Mycket dålig
Dålig
Tillfredsställande
God
Utmärkt
Bedömning
ständig övermodulering
Dålig eller ingen
T ilitredsställande
God
Maximal
Bedömning
Oanvändbar
Dålig
Tillfredsställande
God
Utmärkt

Bl

KNI

Grupp l
Bild 1.1
Bild 1.2
Bild 1.3
Bild 1.4
Bild 1.5
Bild 1.6

Morsetecknens uppbyggnad ...................................................... ~ ................... 1
Inlärningsordning för morsetecken ................................................................ 2
Rätt sittställning sett framifrån ....................................................................... 3
Rätt sittställning sett från sidan ...................................................................... 4
Rätta handledsrörelser .................................................................................. 4
Telegrafnyckel ................................................................................................ 5

GRUPP Il
Kapite11Bild II 1-1
Bild 111-2
Bild II 1-3
Bild 111-4
Bild II 1-5
Bild 111-6
Bild II 1-7
Bild II 1-8
Bild II 1-9
Bild II 1-1 O
Bild 111-11
Bild II 1-12
Bild 111-13
Bild 111-14
Bild II 1-15
Bildll1-16
Bild 111-17
Bild 111-18
Bild 111-19
Bild 111-20
Bild II 1-21
Bild 111-22
Bild II 1-23
Bild 111-24
Bild 111-25
Bild II 1-26
Bild II i -27
Bild 111-28
Bild 111-29
Bild 111-30
Bild 111-31
Bild 111-32

Atomernas uppbyggnad ................................................................................. 1
Tankeförsök med kulor i ett rör ...................................................................... 3
Potential och spänning i en strömkrets .......................................................... 5
Formelsnurra" för Ohms och Joules lagar ..................................................... 6
Elektriska kraftfält ........................................................................................ 16
Elektrisk fältstyrka ........................................................................................ 17
Kraftfält omkring magneter .......................................................................... 20
Magnetiska fält omkring strömledare ........................................................... 21
Tillämpade e lektromgneter .......................................................................... 22
Vågor längs en linje ..................................................................................... 25
Vågutbredning på en yta .............................................................................. 25
Vågutbredning i rummet .............................................................................. 25
Elektromagnetiskt spektrum ........................................................................ 26
Polarisation av elektromagnetiska vågor ..................................................... 28
Våginterferens ............................................................................................. 29
Alstring av en sinusformad signal ................................................................ 31
Vektorer och fasförskjutning ........................................................................ 33
Ren sinusvåg och övertonshaltig våg .......................................................... 35
Uppdelning av en signal i grundton och övertoner ...................................... 36
Uppdelning av en fyrkantvåg i grundton och övertoner ............................... 37
Överlagrade spänningar .............................................................................. 38
Modulerade signaler .................................................................................... 40
Modulerande signaler .................................................................................. 42
Sidband vid A3E-modulation ........................................................................ 43
A3E-modulation med toner med olika styrka och frekvens .......................... 44
Amplitudmodulation med morsetecken ........................................................ 46
Sidband vid DSB .......................................................................................... 47
Sidbandsval vid SSB ................................................................................... 48
Sidband!ägen vid SSB ................................................................................. 49
Frekvensmodulation .................................................................................... 51
Sidbandsspektrum vid FM-modulering med 1 sinuston ............................... 53
Effektförhållande .......................................................................................... 57
Bilder -1

l
GRUPP Il

Kapitel2..

Bild II 2-1
Bild 112-2
Bild 112-3
Bild 112-4
Bild II 2-5
Bild 112-6
Bild II 2-7
Bild 112-8
Bild II 2-9
Bild II 2-1 O
Bild II 2-11
Bild 2-12
Bild 2-13
Bild 2-14
Bild 2-15
Bild 2-16
Bild 2-17
Bild 2-18
Bild 2-19
Bild 2-20
Bild 2-21
Bild 12-22
Bild 112-23
Bild 112-24
Bild 112-25
Bild 112-26
Bild 112-27
Bild 12-28
Bild 112-29
Bild 2-30
Bild 2-31
Bild 2-32
Bild 2-33
Bild 2-34
Bild 2-35
Bild 2-36
Bild 2-37
Bild 2-38
Bild 2-39
Bild 2-40
Bild 112-41
Bild 112-42
Bild 112-43
Bild II 2-44
Bild 112-45

Bilder-2

Schemasymboler för resistorer ...................................................................... 4
Schemasymboler för kondensatorer .............................................................. 7
Försök med induktion .................................................................................. 1O
Schemasymboler för induktorer ................................................................... 11
Schemasymboler för transformatorer .......................................................... 15
Obelastad transformator .............................................................................. 16
Belastad transformator ................................................................................ 16
Sparkopplad transformator .......................................................................... 17
Strömtransformator ...................................................................................... 18
Högspänningstransformator ........................................................................ 18
Klenspänningstransformator ........................................................................ 18
Spärrskiktet i en halvledardiod ..................................................................... 19
Halvledardiodens karaktäristik ..................................................................... 20
Schemasymboler för dioder ......................................................................... 20
Dioders polarisering i kretsen ...................................................................... 21
Schemasymboler ......................................................................................... 23
Skikten i en bipolär transistor ....................................................................... 23
Emitterkopplad transistor ............................................................................. 24
Karaktäristika för transistor BC 107 ............................................................. 25
Schemasymbol för en FET .......................................................................... 26
Skikten i en N-kanal FET ............................................................................. 26
Skikten i en N-kanal MOS-FET .................................................................... 26
Karaktäristikför N-kanal FET ........................................................................ 27
Schemasymboler för dioder ......................................................................... 29
Edisoneffekten ............................................................................................. 29
Diodens karaktäristik ................................................................................... 29
Halvvågslikriktning ....................................................................................... 30
Helvågslikriktning ......................................................................................... 30
Likriktande funktion ...................................................................................... 31
Symboler för triod och pentod ...................................................................... 31
Elektronstömmen i en triod .......................................................................... 31
Karaktäristika för elektronrör ........................................................................ 32
Branthet ....................................................................................................... 33
Inre resistans ............................................................................................... 33
Transistorn som analog förstärkare respektive digital strömställare ............ 35
NOT-gate ..................................................................................................... 35
OCH-grind (AND-gate) ................................................................................ 36
ELLER-grind (OR-gate) ............................................................................... 37
OCH INTE-grind (NANO-gate) ..................................................................... 37
INTE ELLER-grind (NOR-gate) ................................................................... 38
Inverterad ingång ......................................................................................... 38
Exklusiv ELLER-grind (EXOR-gate) ........................................................... 39
Exklusiv INTE ELLER-grind (EXNOR-gate) ............................................... 39
DTL-Iogik ..................................................................................................... 40
TTL-Iogik ...................................................................................................... 40

NG
GRUPP Il

Kapitel3=

Bild II 3-1
Bild II 3-2
Bild II 3-3
Bild II 3-4
Bild II 3-5
Bild II 3-6
Bild 113-7
Bild II 3-8
Bild 113-9
Bild 3-1 O
Bild 3-11
Bild 3-12
Bild 3-13
Bild 3-14
Bild 3-15
Bild 13-16
Bild II 3-17
Bild 113-18
Bild II 3-19
Bild II 3-20
Bild 3-21
Bild 3-22
Bild 3-23
Bild 3-24
Bild 3-25
Bild 3-26
Bild 3-27
Bild 3-28
Bild 3-29
Bild 3-30
Bild 3-31
Bild 3-32
Bild 3-33
Bild 13-34
Bild 3-35
Bild 3-36
Bild 3-37
Bild 3-38
Bild 3-40
Bild 3-41
Bild 3-42
Bild 3-44
Bild 3-45
Bild 3-46
Bild 3-47
Bild 3-48

Seriekopplade resistorer ................................................................................ 1
Parallellkopplade resistorer ........................................................................... 1
Resistiv spänningsdelare ............................................................................... 2
Wheatstone 's brygga ..................................................................................... 2
Parallellkopplade kondensatorer ................................................................... 3
Seriekopplade kondensatorer ........................................................................ 3
Magnetiskt kopplade induktorer ..................................................................... 4
Uppladdning av en kondensator .................................................................... 5
Urladdning av en kondensator ....................................................................... 6
Inkoppling av en induktor ............................................................................... 7
Faslägen och effekter i LC-kretsar ................................................................. 8
Seriekrets av L+C+R ................................................................................... 1O
Spänningar i seriekrets L+C+R .................................................................... 1 O
Impedansen och fasvinkeln i seriekrets L+C+R .......................................... 11
Parallellkopplad LC-krets ............................................................................. 12
Seriekopplad LC-krets ................................................................................. 13
Thomson's svängningskrets ........................................................................ 14
Resonansfallet i parallellkrets ...................................................................... 15
Resonansfallet i seriekrets ........................................................................... i 5
Q-värden i parallellkrets ............................................................................... i 6
Bandbredd i parallellkrets ............................................................................ i 6
Högpassfilter ................................................................................................ 22
Lågpassfilter ................................................................................................ 23
Bandpassfilter .............................................................................................. 24
Passfilter ...................................................................................................... 25
Bandspärrfilter ............................................................................................. 25
Spärrfilter ..................................................................................................... 26
Kvartskristall ................................................................................................ 26
Bandfilter med kvartskristaller ...................................................................... 26
Mekaniskt filter ............................................................................................. 27
Kavitetsfilter ................................................................................................. 27
Pi-filter .......................................................................................................... 28
T-filter ........................................................................................................... 28
Halvledardioder ............................................................................................ 29
Halv- och helvågslikriktning ......................................................................... 30
Glättning av likspänning ............................................................................... 31
Likriktarkoppling med spänningsdubbling .................................................... 32
Spänningsstabilisering ................................................................................. 32
Från elektronrör till transistor ....................................................................... 33
Principen för förstärkare med elektronrör respektive transistor ................... 34
Grundkopplingar för elektronrör och NPN-transistor ................................... 35
Förstärkare i klass A .................................................................................... 37
Förstärkare i klass B .................................................................................... 38
Förstärkare i klass C .................................................................................... 38
Frekvensmultipliceringskedja ....................................................................... 39
Slutsteg med en transistor ........................................................................... 40
Bilder- 3

BILD F

TECKNING

Bild II 3-49 Mottaktskopplat slutsteg med transistorer ................................................... 40
Bild II 3-50 Högeffekts lutsteg med en tetrod .................................................................. 41
Bild II 3-51 Hög effektslutsteg med två trioder ................................................................ 41
Bild 3-52 Bestämning av PEP-effekten ....................................................................... 42
Bild 3-53 Linjäritetskontroll vid SSB ............................................................................ 43
Bild 3-54 Linjäritetens betydelse ................................................................................. 44
Bild 3-55 Dioddetektorn .............................................................................................. 47
Bild 3-56 Produktdetektor för AM (A3E) och GW (A 1A) .............................................. 48
Bild 3-57 Amplitudbegränsning vid FM-mottagning .................................................... 50
Bild 3-58 Ideal arbetslinje för diskriminator ................................................................. 50
Bild 3-59 Slope-detektorn ........................................................................................... 50
Bild 3-60 Foster-Seeley detektorn .............................................................................. 51
Bild 3-61 Räknardiskriminatorn ................................................................................... 51
Bild 3-62 PLL-demodulatorn ....................................................................................... 52
Bild 3-63 Svängningar ................................................................................................. 53
Bild 3-64 Elektromekanisk oscillator ........................................................................... 54
Bild 3-65 Elektronisk oscillator (Me i Bner) ................................................................... 54
Bild l 3-66 Oscillator enligt Meissner ............................................................................. 55
Bild II 3-67 Emitterkopplad förstärkare ........................................................................... 55
Bild II 3-68 Komplett Meissneroscillator ......................................................................... 55
Bild II 3-69 Svängningsvillkoret ...................................................................................... 56
Bild II 3-70 Hartiey-koppling ........................................................................................... 56
Bild 3-71 TPTG-koppling ............................................................................................. 56
Bild 3-72 Golpitts-koppling .......................................................................................... 56
Bild 3-73b Förstärkare i Glappkoppling ......................................................................... 57
Bild 3-73a Glapp-koppling ............................................................................................. 57
Bild 3-7 4 Bandspridning .............................................................................................. 57
Bild 3-75 Golpittsoscillator med kristall i parallellresonansfallet .................................. 59
Bild 3-76 Golpittsoscillator med kristall i serieresonansfallet ...................................... 59
Bild l 3-77 Superheterodyn-VFO ................................................................................... 60
Bild 3-78 VFO och VGO jämförs ................................................................................. 61
Bild 3-79 Kapacitansdiod- Varicap ............................................................................ 61
Bild 3-80a Analogi Människa-PLL ............................................................................. 61
Bild 3-80b Oscillator med PLL-styrning ........................................................................ 62
Bild 3-81 PLL-oscillator kombinerad med frekvensblandning .............................. ~ ...... 63
Bild l 3-82 PLL med frekvensdelare .............................................................................. 64
Bild II 3-83 Principer för frekvensblandning ................................................................... 67
Bild II 3-84a Entaktsblandaren ......................................................................................... 68
Bild 3-84b Entaktsblandaren ......................................................................................... 69
Bild 3-85 Mottaktsblandaren ....................................................................................... 71
Bild 3-86 Ringblandaren ............................................................................................. 72
Bild 3-87 Jämförelse mellan olika blandare ................................................................ 73
Bild 3-88 Frekvensspektrum från en Super-VFO ........................................................ 75
Bild 3-89 A3E-modulator ............................................................................................. 77
Bild 3-90 Alstring av J3E (SSB) .................................................................................. 78
Bild 3-91 Alstring av F3E (FM) .................................................................................... 79
Bild 3-92 Alstring av G3E (PM) ................................................................................... 80

Bilder-4

KNING
GRUPP Il
Kapitel4m
Bild II 4-1
Bild 114-2
Bild 114-3
Bild II 4-4
Bild II 4-5
Bild 114-6
Bild 114-7
Bild II 4-8
Bild 114-9
Bild 114-10
Bildll4-11
Bild 4-12
Bild 4-13
Bild 4-14
Bild 4-15
Bild 4-16
Bild 4-17
Bild 4-18
Bild 4-19
Bild 4-20
Bild 4-21
Bild 4-22
Bild 4-23
Bild 4-24
Bild 4-25
Bild 4-26
Bild 4-27
Bild 4-28
Bild 4-29
Bild 4-30

Detektormottagare ......................................................................................... 1
selektion i detektormottagare ........................................................................ 2
Detektormottagare med LF-förstärkare ......................................................... 2
Förbättrad selektion ....................................................................................... 2
Förbättrade HF-egenskaper i detektormottagare .......................................... 3
Hög H F-selektion ........................................................................................... 3
CW i detektormottagare ................................................................................. 3
Mottagare med direkt frekvensblandning ....................................................... 5
De modulering i mottagare med direkt frekvensomvandling - CW-signaler ... 4
De modulering i mottagare med direkt frekvensomvandling- SSB-signaler .. 4
selektionen i direktblandade mottagare ........................................................ 6
Passbandbredd och spegelfrekvenser i direktblandade mottagare ............... 8
Superheterodyn mottagaren i princip .............................................................. 9
Dubbelsuperheteodynen i princip ................................................................ 1O
Panoramamottagare .................................................................................... 12
Anslutning av panoramamottagare till stationsmottagare ............................ 12
Signal- och svepspänningar ........................................................................ 12
Mottagningskonverter UHF till KV ................................................................ 13
Transverter mellan UHF och KV .................................................................. 14
AGG vid AM-mottagning med superheterodynmottagare ............................ 15
AGC vid SSB- och CW-mottagning med superheterodynmottagare ........... 16
Enkelsuper med låg MF och ingen förselektion ........................................... 18
Enkelsuper med låg MF och med förselektion ............................................. 18
Enkelsuper med hög MF ochmed förselektion ............................................ 18
Samtidig för- och närselektion i superheterodynmottagare ......................... 19
MF-bandbredd vid AM (A3E) ....................................................................... 20
MF-bandbredd och passband-tuning vid SSB (J3E) ................................... 21
Olika MF-bandbreder vid CW (A i A) ............................................................ 22
SIN-värde ..................................................................................................... 24
SINAD-värde ................................................................................................ 24

GRUPP Il
KapitelsBild II 5-1
Enstegs sändare ............................................................................................ 1
Bild II 5-2
Flerstegs rak sändare .................................................................................... 1
Bild II 5-3
FM-sändare med frekvensmultiplicering ........................................................ 2
Bild II 5-4
2-bands CW-sändare med frekvensblandning .............................................. 2
Bild II 5-5
2-bands SSB-sändare med frekvensblandning ............................................. 3
Bild II 5-6
Flerbands SSB-sändare med frekvensblandning .......................................... 3
Bild II 5-7
PLL-styrd FM-sändare för VHF ...................................................................... 4
Bild II 5-8
PLL-styrd SSB-sändare för kortvåg ............................................................... 5
Bild II 5-9
Separat sändare och mottagare .................................................................... 7
Bildll5-10 Transeeiver med samma VFO ....................................................................... 7
Bild II 5-11 Direktblandad transeeiver med gemensam VFO ........................................... 7
Bild II 5-12 Kristallstyrd 6-kanals FM-transceiver för VHF ............................................... 8
Bilder- 5

Bl

RTE

IN

~©~

EPT

Bild II 5-13 PLL-styrd FM-transceiver för VHF ................................................................. 9
Bild II 5-14 SSB-transceiver för kortvåg ......................................................................... 11
Bild il 5-15 PLL-styrd SSB-transceiver för kortvåg ........................................................ 12

GRUPP Il

Kapitel 6Bild II 6-1
Bild II 6-2
Bild Ii 6-3
Bild II 6-4
Bild II 6-5
Bild II 6-6
Bild II 6-7
Bild II 6-8
Bild II 6-9
Bild II 6-1 O

Bild II 6-11
Bild II 6-12
Bild II 6-13
Bild 116-14
Bild II 6-15
Bild 116-i6
Bild II 6-i 7
Bild II 6-i 8
Bild II 6-19
Bild II 6-20
Bild II 6-21
Bild II 6-22
Bild II 6-23
Bild II 6-24
Bild II 6-25
Bild II 6-26
Bild II 6-27
Bild II 6-28
Bild II 6-29
Bild II 6-30
Bild II 6-3i
Bild II 6-32
Bild II 6-33
Bild II 6-34
Bild II 6-35
Bild II 6-36

Bilder- 6

Spänning och ström i en halvvågsantenn ...................................................... 2
Matningsimpedansen i en halwågsantenn ................................................... 2
Halvvågsdipol matad med harmoniska övertoner .......................................... 3
Elektrisk förlängning och förkortning av antenner .......................................... 5
Vertikaldiagram för halwågsantenn ............................................................... 5
Antennvinst d Bd i effekt ................................................................................. 6
Antennvinst d Bd i spänning ........................................................................... 6
F/B-förhållande i effekt ................................................................................... 6
F/B-förhållande i spänning ............................................................................. 7
Halwärdesbredder ........................................................................................ 7
Inverkan av polarisation ................................................................................. 8
Omvikt di pol ................................................................................................... 9
GP-antenn ..................................................................................................... 9
G P-antenner med elektrisk längdanpassning .............................................. i O
SVF-kuNor för flerbands G P-antenn ........................................................... i O
W3DZZ-antennen ........................................................................................ i1
Riktbar dipol-antenn ..................................................................................... 13
Vagi-antenner .............................................................................................. 13
Cubical Quad-antenner ................................................................................ i 4
Strålningsdiagram för horisontell Vagi-antenn ............................................. 16
Spänningskopplad A./2-dipol ........................................................................ 17
Strömkopplad A./2-dipol ................................................................................ 17
Samma A/2-dipol på grundfrekvensen respektive 1:a övertonen ................ i 8
Koaxialkabel ................................................................................................ 19
Bandkabel .................................................................................................... i 9
ståendevåg på ledning ................................................................................ 21
SVF-problemet enkelt sett ........................................................................... 21
Balansering -Transformering ...................................................................... 22
Ringkärnebalun ............................................................................................ 23
Koaxialledare som bal un ............................................................................. 23
Sätt att ansluta en matningsledning ............................................................. 24
A./2-fasningsledning ...................................................................................... 24
Förlopp i öppen A./4 transmissionsledning ................................................... 26
Förlopp i kortsluten A./4 transmissionsledning .............................................. 27
A./4 transmissionsledning som svängningskrets .......................................... 28
Antennkopplare ............................................................................................ 29

F RTECKNING
GRUPP Il

Kapitel 7-

Bild
Bild
Bild
Bild
Bild
Bild
Bild
Bild
Bild
Bild
Bild
Bild
Bild

Il 7-1
Från sluten LC-krets till antenn ..................................................................... 1
Il 7-2
Pendlingen mellan E-fält och H-fält .............................................................. 2
Il 7-3
Elementär di pol ............................................................................................. 2
Il 7-4
Ett självständigt E-fält skapas ....................................................................... 3
Il 7-5
E-, H- och s-tälten omkring en antenn (förenklad framställning) .................. 3
Il 7-6
E-, H- och S-fält ............................................................................................ 3
Il 7-7
Jonosfärskikten ............................................................................................. 5
Il 7-8
Jonosfärsutbredning ..................................................................................... 6
Il 7-9
Radioprognos för amatörradiobanden på kortvåg ........................................ 8
Il 7-1 O Detalj av radioprognos i Il 7-9 ...................................................................... 9
Il 7-11 Vågutbredning på kortvåg .......................................................................... 1O
Il 7-12 Markbaserad repeater ................................................................................ 14
Il 7-13 Transponder i rymdsatellit .......................................................................... 14

GRUPP Il

KapitelsBild II 8-1
Bild II 8-2
Bild II 8-3
Bild II 8-4
Bild II 8-5
Bild II 8-6

Mätning av sändareffekt ............................................................................... 3
Presentation av mätvärden ........................................................................... 5
Vridspoleinstrument ...................................................................................... 5
Mjukjärnsinstrument ...................................................................................... 6
Konstlast ....................................................................................................... 6
Fältstyrkemätare ........................................................................................... 7
Bild II 8-7
Kalibreringsoscillator i mottagare ................................................................. 7
Bild II 8-8
Brusmätbrygga ............................................................................................. 7
Bild II 8-9
SVF-meter, princip och inkoppling ................................................................ 8
Bild II 8-1 O Frekvensräknare ........................................................................................... 8
Bild II 8-11
Absorbtionsvågmeter .................................................................................... 8
Bildll8-12 Dip-meter ...................................................................................................... 9
Bild II 8-13 Mätning med dip-meter ................................................................................. 9
Bild II 8-14 Oscilloscop ................................................................................................... 9
GRUPP Il

Kapitei9Bild II 9-1
Bild II 9-2
Bild II 9-3
Bild II 9-4
Bild II 9-5
Bild II 9-6
Bild II 9-7
Bild II 9-8
Bild II 9-9

Nätfilter ......................................................................................................... 4
Lågpassfilter för sändare .............................................................................. 5
Högpassfilter för VHF-/UHF-mottagare ........................................................ 5
ingångsimpedansen i resonanskretsar ......................................................... 6
Spärrfilter för mottagare ................................................................................ 6
sugkretsar för mottagare .............................................................................. 6
Nät- och skärmströmfilter .............................................................................. 7
Phonoingångsfilter ........................................................................................ 7
Högtalarledningsfilter .................................................................................... 7
Bild II 9-10a H F-avkopplat styrgaller ................................................................................. 7
Bild II 9-10b HF-avkopplad bas på tre sätt ....................................................................... 8
Bilder-?

BILD F
Bild II 9-11 Parasitfilter i HF-förstärkare ........................................................................... 8
Bild II 9-12 Nycklingsfilter ................................................................................................. 8

GRUPPIII

Kapitei2Bild III 2-i

Bilder- 8

ITU Regionkarta (ur

........................................................................... 2

SAKREGISTER
A
AiA IIi -36

A3E Iii -36
Absorbtionsvågmeter 118-9
Aut. förstärkn.reglering (AGC) 114-15, 114-16
Allmänna elnätet 111 0-2
Amplitudmodulation 111-35, 113-73
Analog IC 112-41
Anod 112-29
Anodspänning 112-29
Anodström 112-29
Anropssignal 1111-8
Anropssignalers sammansättning 1111-8
Antenner 111 0-5
Antennavstämningsenhet 116-20
Antennkopplare 116-20, -29
Antennlängd 116-1
Antennsystem 116-1
Antennvinst 116-5
Arbetspunkt 113-32
Atomkärna 111-1
Atomstruktur 111-1
Aurora-reflexion 117-13
Avkoppling 119-7
Avstämd matarledning 116-17
Avstörning 119-2
B
Backspänning 112-20
Backström 112-20
Balun 116-22
Balansering 116-22
Bandbredd 111-35, 113-16, -29
Bandfilter med kvartskristaller 113-22
Bandkabel 116- i 9
Bandpassfilter 113-20
Bandspridning 113-53
Bandspärrfilter 113-21
Barkhausen 112-34
Basband 111-35
Baskoppling 113-30
BCI (broadcasting interference) 119-2
Beskrivningskod för sändn.slag 111-36, E-1
Bestämning av PEP-effekt 113-38
Blandningsprodukt 113-64
Blockering 119-2
Branthet 112-33
Bromsgaller 112-32
Bruksföremål 111 0-3
Brusmätbrygga 118-7
Brusspärr 114-17
Bågmått IIi -27

c

CEPT 1112-3
GEPT-rekommendationerna 1112-3
Chi p (i IG-) 112-41
Glapp-koppling 113-53
Golpitts-koppling 113-52
Goulomb 111-11
D
Decibel 111-53, G-1
Delta-anpassning 116-24

Demodulator 113-43
Detektor 113-43
Detektormottagare 114-1
Dielektrikum 112-5
Diffraktion 117-4
Digital IG 112-41
Digital krets 112-35
Dioden 112-i 9
Dignitet B-3
Dioddetektorn 113-43
Dip-meter 118-9
Direktblandad mottagare 114-4
Diskriminator 113-45
Dipol 117-1
DTL-Iogik 112-40
D-skiktet 117-5
Dubbelsuperheterodynmottagare 114-1 O
Död zon (skip zone) 117-11
E
Edisoneffekt 112-2
Effekt 111-53
Effektdämpning 111-53
Effektförhållande 111-53
Effektförstärkning 111-53
Effektändring uttryckt i dB 111-53
Effektiwärde 111-27
Effektivt utstrålad effekt- ERP 116- 6
E-fält 117-2
Ekvation B-1
Electromagnetic Gompatibility (EMG) 119-1
Electromagnetic lnterference- EMIII9-2
Electromagnetic Susceptibility- EMS 119-2
Elektrisk chock 111 0-1
Elektrisk effekt- Enheten Watt 111-6
Elektrisk förkortning 116- 4
Elektrisk förlängning 116-4
Elektrisk längd 116-1
Elektrisk fältstyrka 111-11
Elektrisk laddning 111-11

Ord-1

SAKRE ISTE
Elektrisk ledare 111-1
Elektrisk spänning 111-4
Elektriskt arbete- Enheten Joule 111-6
Elektriskt kraftfält 111-11
Elektromagnet 111-15
Elektromagnetiskt fält 111-21, 22, 117-3
Elektromagnetisk våg 111-23, 117-2, A-1
Elektromotorisk kraft- EMK 111-9
Elektron 111-1
Elektronrör 112-29
Elektronskal 111-1
Elementär di pol 117-2
Elementarladdning 111-11
Elementarpartikel 111-1
ELLER-grind eller OR-gate 112-36
EMC-Iagen 119-1
EME-förbindelse 117-13
Emitterkoppling 113-30
Energi 111-53
Enkelsuperheterodynmottagare 114-18
Entaktsblandare 113-64
E-skiktet 117-5
Exklusiv ELLER-grind (EXOR-gate) 112-39
Exklusiv INTE ELLER-grind (EXNOR gate)
112-39
F
F3E 111-37
Farad 112-5
Faslåsning- PLL 113-57
Förluster i kärnmaterial 112-13
Förlustvinkel 112-6
Fasförskjutning 111-28
Fasförskjutning i en induktor 112-12
Fasförskjutning i en kondensator 112-6
Fasledare 1110-3
Fasläge 111-27
Fasmodulation 111-35, 113-76
Flerbands GP-antenn 116-1 O
Flerbands halwågsantenn 116-11
Fonetiska alfabeten 1111-2
Formelsnurra 111-6
Formler B-1
Foster-Seeley diskriminatorn 113-47
Fram-/backförhållande (antennvinst) 116- 6
Framström 112-20
Frekvens 111-28
Frekvensblandare 113-63
Frekvensdeviation 111-48
Frekvenseffektivitet 111-35
Frekvensfilter 113-17
Frekvensgång 113-29
Frekvensinställning 113-53
Ord-2

Frekvensmultiplikation 113-34
Frekvensmodulation 111-35, 113-75
Frekvensräknare 118-9
Fri elektron 111-1
F-skiktet 117-6
Fysikalisk strömriktning 111-4
Fädning eller signalbortfall 117-11
Fälteffekttransistor- FET 112-23, -26
Fältstyrkemätare 118-7
Färgkoder 1110-3
Förstärkningsfaktorn p, 112-34
Förkopplingsresistans 118-1
Förselektion 114-19
Förstärkning 113-29
Förstärkningsfaktor 112-24
Förstärkt isolering IIi 0-3
G
G3E 111-37
Gallerjordad koppling 113-38
Gallerspänning 112-31
Galvaniskt kopplade induktorer 113-4
Gamma-anpassning 116-24
Gauss 111-19
Germanium 111-2
Glimmerkondensator 112-7
Glättningskretsar 113-25
Glödtråd 112-29
Grekiska alfabetet A- 2
Grundläggande matematik B-1
Grundton 111-31
Grundämne 111-1
Gruppantenn 116-15
Gruppcentral 1110-3
Gruppledning 1110-3
Grålinjeutbredning = gray Iine 117-11
Gränsfrekvens 113-17
Jordplanantenn - GP 116-4

H
H-fält 117-2
Halvledardiod 112-19
Halvledare 111-1
Halwågsantenn 116-1
Halwågslikriktning 112-30, 113-25
Halwärdesbredd 116-7
Harmoniska övertoner 116-2
Hartiey-koppling 113-52
Hastighetsfaktor 116- i 9
Helixfilter 113-23
Helvågslikriktning 112-30, 113-25
Henry 111-19, 112-11

SAKR
Hertz 111-28
HF-förstärkare 113-29
Huth-Kuhn-koppling 113-52
Huvudbärvåg 111-35
Höga spänningar 1110-4
Höga strömmar 111 0-4
Högsta användbara frekvens (MUF) 117-7
Högfrekvens HF 111-28
Högpassfilter 113-17, 119-5

l
IARU:s bandplaner 1111-1 O
IARU Region 1 bandplan HF F-1
IARU Region 1 bandplan VHF/UHF/SHF/
EHF F-3
IC 112-41
Icke sinusformad signal 111-31
Impedans 113-1 O
Impedans i resonant krets 113-14
Impedansanpassning 111-55
Impedansomsättning 112-15
Impedansomvandling 113-30
Impedanstransformator 112-15
Induktans 112- 9
Induktiv reaktans 112-11
Induktorn 112-9
Inkoppling av induktor 113-6
Integrationsgrad 112-41
Integrerad krets (IC) 112-41
lntermodulation 114-24
Internationell nödtrafik 1111-7
Internat. Amatörradiounionen 1111-1 O
Internatradioreglementet (RR) 1111-1. 1112-1
Internationella telekonventionen -ITC 1111-1
Internationella teleunionen-ITU 1111-1
Inre resistans 1129, -33
INTE ELLER-grind eller NOR-gate 112-38
Inverterad ingång 112-38
Inverterande grind 112-35
Isolator i 11-1
Isotropisk antenn 116-5

J

J3E 111-37
Jon 111-4
Jonosfärskikt 117-5
Jordfelsbrytare 111 0-3
Jordning av antennsystem 1110-4

K

Kalibreringsoscillator 118-7
Kantvåg 111-31

ISTER

Kapacitans 112-5
Kapacitansdiod (VariCap) 112-21
Kapacitiv reaktans 112-6
Karaktäristisk impedans Z 116-19
Katod 112-29
Kavitetsfilter 113-23
Kirchhoffs lagar 111-6
Kirchhoff's 1:a lag 113-1
Kirchhoff's 2:a lag 113-1
Kisel 111-2
Klass A 113-33
Klass AB 113-33
Klass B 113-33
Klass C 113-33
Koaxialkabel 116-18
Kolfilmsresistor 112-2
Kollektorkoppling 113-30
Kondensator, pappers-, plast-, elektrolyt-,
keramisk 112-7
Kondensatorn 112-5
Kondensatorn i växelströmskretsen 112-6
Konduktivitet 111-1
Konstgjord andning 1110-2
Konstlast 118-6,
Korsmodulation 114-24
Kortslutningsström 111-9
Kraftförsörjning 113-25
Kristalldetektor 114-1
Kristalloscillatorer 113-55
Kritisk frekvens 117-6
Kritisk vinkel 117-7
Kvartskristall 113-22

l
Laddningsmängd 111-11
Lag om EMC 119-1
Lagen om radiokommunikation 11125
Le-oscillatorer 113-51
Likspänning 111-4
LF-detektering 119-2
LF-förstärkare 113-29
Lineartransponder 13
Linjäritetskontroll vid SSB 113-39
Linjär och olinjär potentiometer 112-3
Ljusberoende resistor, fotoresister 112-2
Ljusvågor 111-23
Logaritmer B-5
Lysdiod 112- 21
Lågfrekvens LF 111-28
Lågpassfilter 113-1 7. 119-4
Läckström 112-6, -20
Lägsta användbara frekvens (LUF) 117- 7
Ord-3

S KRE ISTER
M

Magnetisk flödestäthet 111-1g
Magnetisk fältriktning 111-15
Magnetisk fältstyrka 111-1g
Mag netiskt flöde 111-1g
Mag netiskt fält 111-15
Magnetiskt kopplade induktorer 113-4
Magnetism 111-15
Magnetfältberoende resister 112-2
Markbaserad relästation- repeater 117-i 3
Markvåg 117-1 O
Massaresistar 112-2
Match-box 116-20
Matningsimpedans 116-2
Meissner-koppling 113-51
Mekanisk längd 116-1
Mekaniskt filter 113-23
Mellanfrekvens 114-1 O
Metallers resistivitet A-1
Metallfilmresistor 112-2
Metalloxidresistor 112-2
MF-bandbredd 114-1g
MF-skift 114-22
Miikrofarad 112-5
Mikroprocessor 112-41
Minnesfunktion 112-41
Missanpassning 116-4
Mittmatad halwågsantenn 116- g
Mjukjärnsinstrument 118-2, -6
Modulation 111-35
Modulationsindex 111-48
Modulationssystem 111-35
Modulatorer 113-73
Modulera 111-35
Modulerad signal 111-35
Modulerande signal 111-35
Momentanvärde 111-27
Morsesignalering 11-1
Morsetecknen 11-i
Mottagare 114-i
Mottagningskonverter 114-13
Mottaktsblandare 113-66
Måttenheter A-1
Människokroppen 1110-1
Märkning av kondensator 112-7
Märkning av resister 112-4
Mögel-Dellinger-effekten 117-5

N

Nanofarad 112-5
Nervbaning 11-2
Neutroner 111-1
Ord-4

N-ledning 111-2
Nolledare 1110-3
NOT-gate 112-35
NPN-transistor 112-23
Nukleon 111-1
Nycklingsfilter IJg-8
Närselektion 114-1g
Nödsignaler 1111-7
Nödtrafik 1111- 7

o

Gavstämd matarledning 116-18
OCH INTE-grind eller NANO-gate 112-36
OCH-grind eller AND-gate 112-36
Ohms lag 111-6
Ohms lag vid växelström 113-1 i
Omvikt dipol (folded dipole) 116-4, g
Operationsförstärkare 112-41
Optimal trafikfrekvens (FOT) 117-7
OSCAR-satellit 117-14
Oscillatorer 113-4g
Oscillator med PLL-styrning 113-57
Oscilloskop 118-1 O,

p
Panoramamottagare 114-11
Parabolantenn 116-15
Parallellkopplade kondensatorer 113-3
Parallellkopplade LC-kretsar 113-12
Parallellkopplade resistorer 113-1
Parasitfilter ng-8
Passband-tuning 114-22
Passfilter 113- 2i
Passriktning 112-20
Period Il i -28
Periodtid Il i -28
Permanentmagnet l l i -15
Permeabilitetstal (fältkonstant) 111-i g
Pi-filter 113-24
Pikofarad 112-5
P-ledning- "hålledning" 111-4
PLL- dernodulatom 113-47
PLL-styrd FM-transceiver 115-g
PLL-styrd sändare 115-4
PNP-transistor 112-23, -24
Polspänning 111-g
Post- och telestyrelsen- PTS 1112-5
Post- och telestyrelsens föreskrifter 1112-5
Potenser B-3
Primärlindning 112-i 5
Produktdetektor 113-43
Proton 111-1

ISTER
Pull-down 112-36
Pull-up 112-36
Pulsmodulation 111-35
Påverkan från elektromagnetiska fält 111 0-1
Q

Q-faktorn i en parallellkrets 113-16
Q-faktor- godhetstal 112-12
Q-koden l l 11-3
Quad-antenn 116-4, -12

R

Radioanläggning
Radiolagen 119- i
Radiotrafik vid naturkatastrofer 1111-7
Radiovågor Il i -23
Rak mottagare 114-1
Rak sändare 115-1
Rapportkoder 1113-1, J-1
Reflexion 117-4
Reflexion mot Es (sporadiskt
117-13
Reflexion mot meteor- Meteorscatter 117-13
Refraktion 117-4
Regler och trafikmetoder III i -1
Resistans 112-1
Resistansen i människokroppen 1110-1
Resistiva material 112-1
Resistivitet 112-1
Resister 112-1
Resolution 640 vid WARC 1979 1111-7
Resonansfallet i en parallellkrets 113-14
Resonansfallet i en seriekrets 113-15
Restladdning i kondensatorer 1110-5
Ringblandare 113-66
Ringkärnebalun 116-23
Rymdsatellit-baserad relästation 117-13
Rymdvåg 117- i O
Räknardiskriminatorn 113- 47
Rötter B-4

s

Schemasymboler
för dioder 112-20
för induktorer 112-11
för kondensatorer 112-7
för resistorer 112-4
för transformatorer 112-15
för transistorer 112-23
för elektronrör 112-31
sekundärlindning 112-15
selektion 114-2
selektivitet 114-17

s-enheter D-1
S-fäit 117-2
Seriekopplade kondensatorer 113- 3
Seriekopplade LC-kretsar 113-13
Seriekopplade resistorer 113-1
signalkänslighet 114-23
Signalstyrkemätare (S-meter) 114-17
SINAD 114-24
IC'TI"\.-rn~lf"t signal 111-27
Självinduktion 112-9
Självsvängningsvillkoret 113-52
Skin-effect (yteffekt) 112-12
Skip-avstånd 117-11
Skydd och jordning IIi 0-6
skyddsjordning 1110-3
skyddsledare 111 0-3
Skärmgaller 112-32
Skärmning av elektriskt fält 111-13
Skärmning av magnetiskt fält 111-19
Slope-detektorn 113-45
S/N 114-24
Snabba och tröga säkringar 1110-4
solaktivitet 117-9
Solfläckstal 117-9
Sparkopplad transformator 112-17
Spegelfrekvensdämpning 114-1 O
Spegelfrekvenser i direktblandare 114-7
Spegelfrekvensproblemet 114-17
Sporadiskt E-skikt 117-5
Spänningsnod 116- 2
Spänningsberoende resister 112-2
Spänningsdelare 113-2
Spänningshöjande likriktarkopplingar 113-27
Spänningsstabilisering 113-27
Spänningstransformator 112-15
Spänningsstyrd oscillator (VCO) 113-57
Spänningsändring uttryckt i dB IIi -54
Spärrfilter
, 119-5
Spärriktning
Spärrskikt 112-19
Spärrzon 112-23
stationsdagbok (loggbok) 1113-1
Strålningslob 116-2
Strömbrytare l l 10-2
Strömbuk 116-2
Strömkrets 111-5
Strömshunt 118-1,
Strömtransfermater 112-15
Strömändring uttryckt i dB 111-54
sändning 11
styrgaller 112-31
ståendevågmeter (SVF-meter) 118-8
Ord-5

SAKRE ISTER
Stående vågor 116-20
Störningsproblem 119- 2
Sugkrets, sugfilter 113-21, 119-5
Superheterodynmottagare 114-9
Superheterodynsändare 115-3
Superheterodyn-VFO 113-56
Svenska bandplaner G-1
Svenska repeaterfrekvenser H-1
Svängningar 113-50
säkerhetsåtgärder 111 0-5
Switchade aggregat 113-28
Sändare med frekvensmultiplicering 115-1
Sändarslutsteg 113-36
Sändningsslag 111-35
Särjordning IIi O- 4

T

T-anpassning 116-24
Teckendelii-i,
Teknisk strömriktning lli-4
Temperaturinversion 117-i3
Temperaturberoende resistor 112-3
Temperaturkoefficient 112-7, -i2
Temperaturkompensation 113-62
Tesla 111-i9
T-filter 113-24
Thomson' s svängningskrets 113-i3
Tjockfilmsresistor 112-2
Topp-till-toppvärde lli-27
Toppvärde eller amplitud lli-27
Toppvärdeseffekt P. E. P. IIi-56
Transistorn 112-23
Transeeiver 115-7
Transformatorn 112-i5, 116-22
Transmissionsledning 116-i7, -25
Transverter 114-i4
Troposfären - Troposcatter 117-i2
Trådlindad resistor 112-2
TTL-Iogik 112-40
Tunnfilmsresistor 112-2
TVI (television interference) 119-2

u

Underbärvåg 111-35
Uppladdning av kondensator 113-5
Urkoppling av induktor 113-6
Urladdning av kondensator 113- 5
Utbredningsförlust 117-4
Uttag och stickproppar med jorddon 1110-3

Ord-6

PT
v

Vakuumdiod 112-19, -29
Vakuumtriod 112-31
Vakuumpentod 112-31
Valenselektron 111-1 , -2
Variabel frekvens oscillator- VFO 113-51
VariCap 112-21
Varvtalsomsättning 112-15
Vektor 111-28
Vridspoleinstrument 118-2
Våghastighet 116-1
Våginterferens 111-25
Vågledare 116-i9
Vågpolarisation lli-23, 116- 8
Vågutbredning 111-21,117-1, -10
Vågutbredningsförutsägelser 117-7
Vågutbredningsmodeller 111-21
Vänsterhandsregeln 111-15
Växelspänning 111-4
Växelströmskretsar 113- 8

w

W3DZZ-antennen 116-4
Weber 111-19
Wheatstone's brygga 113-2

x

Inget sökord

y
Vagi-antenn 116-4, -12
Yteffekt (skin-effect) 112-12

z

Zenerdiod 112-21

Å

Åska 1110-6
Ä
Ändmatad halvvågsantenn 116-9

ö

Överhettning 111 0-4
Överlagrad spänning 111-34
Överton 111-31
Övertonskristaller 113-55

RAT UR
The American Radio Relay League, lnc. (ARRL):
The ARRL Handbook for Radio Amateurs.1994, Seventy-First Ed., ISBN 0-87259-171-9
The ARRL Radio Amateur's Library.
Cuno, Hans. H.:
Vorbereitung auf die Amateur-Funk-Lizenzprilfung.
frech-verlag, 1993, 16. Auflage. ISBN 3-7724-5402-X.
Deutscher Amateur-Radio-Ciub e. V. (DARC):
Ausbildungsunterlagen.
DARC Verlag, 1983.
Ekbom, Lennart:
Tabeller och formler N T Te.
Esselte Studium, 2. upplagan, 1986, ISBN 91-24-34594-6.
Experimenterende Danske Radioamat0rer (EDR):
Vejen til sendetilladelsen.
EDR, 6. udgave, 2. oplag, ISBN 87-85149-02-0.
Follbring, Tommy:
Radioteknik för sändareamatörer.
Ljudbandsinstruktioner AB.
Föreningen Sveriges Sändareamatörer (SSA):
Populär Amatörradio. SSA, 1952.
Grundläggande Amatörradioteknik. SSA, 2. upplagan, 1970.
Hall, T; Pederby, Bo; Elmgren, Bo:
Fysikboken för högstadiet.
Esselte Studium, 2. upplagan, 1974, ISBN 91-24-69278-6.
Haraldsson, Tore:
Radioteknik för radioamatörcertifikat
Radio TV KB Haraidsson \& Söner, 8. upplagan 1989, ISBN 91-970362-1-8.
lsännäinen, Antti:
Amatöörtekniikkaa PerusJuakan Kursseille.
Suomen Radioamatööriliitto r.y. (SRAL), 1987, ISBN 951-96056-1-4.
Lindkvist, Olle:
Antenner. 1993, ISBN 99-0830753-3
Radiosändare. 1989
Ljudbandsinstruktioner AB
Lundqvist, Hans; Roos, Olle:
Elektronik.
Esselte Studium. 2. upplagan 1982, ISBN 91-24-32173-7.
Moltrecht, Eckart K. W.:
Amateurfunk-Lehrgang.
Teil1, 2. Auflage 1987,
Teil2, 2. Auflage 1989,
Teil 3, 2. Auflage 1987,
Teil 4, 2. Auflage 1989,
frech-verlag.

ISBN
ISBN
ISBN
ISBN

3-7724::5386-4,
3-7724-6387-8,
3-7724-5388-0,
3-7724-5389-9.

Litteratur- 1

LITTER
Norsk Radio Relre Liga (NRRL):
Radioamaterens ABC. Laarebok i radioteknikk.
NRRL, 1995, 1. utgave- 2. opplag.
Pietsch, Hans-Joachim:
Kurzwellen-Amateurfunktechnik.
Franzis Verlag, 1984, ISBN 3-7723-6592-2.
Rothammel, Karl:
Antennenbuch.
Franckh' sche Verlagshandlung, 1978, ISBN 3-440-04498-X.
de Sousa Pires, Jorge:
Electronics Handbook.
Studentlitteratur, 1989, ISBN 91-44-21021-3.
Svenska Elverksföreningen, Elektriska Installatörsorganisationen (EIO), Elsäkerhetsverket
och Röda Korset:
Livräddning vid e/skada.
Elsäkerhetsverkets Publikationsservice, 1996, ISBN 91-88924-00-9.
Svenska Elverksföreningen, Elektriska Installatörsorganisationen (EIO):
E/kunskap för vardagsbruk, 1994, ISBN 91-76221-04-0.
Händig med el, utg. 2, rev. 1995:08.
Energikontorets Förlagsservice.
Södra Vätterbygdens Amatörradioklubb (SVARK):
Möt världen genom etern.
Föreningen Sveriges Sändareamatörer (SSA), 1993, ISBN91-86368-07-9.
Wallander, Per:
Bli sändaramatör.
PERANT Per Wallander AB, 3. omarbetade upplagan 1995, ISBN 91-86296-06-X.
Lennart:
EL-LARA och RADIOTEKNIK.
TextdeiiSBN 91-86368-05-2.
Bilddel ISBN 91-86368-04-4.
Föreningen Sveriges Sändareamatörer (SSA), 1990.
Wibe~g,

ÖVSV ADXB-OE (Österrike):
Amateurfunk-Lizenzlehrgang. Der Weg zum Amateurfunk.
Orbit, 1982, 25. Auflage, ISBN 3-85216-001-4.

Litteratur- 2


\listoffigures
\listoftables

\backmatter

\printindex

\end{document}
