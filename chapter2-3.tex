\section{Induktorn}
\textbf{HAREC a.\ref{HAREC.a.2.3}\label{myHAREC.a.2.3}}

Allmänt
När elektrisk ström flyter genom en ledare,
så alstras ett magnetfält omkring den. Så
snart strömmens styrka eller riktning ändras,
uppstår en motsvarande s.k. elektromotorisk kraft (EMK), som motverkar ändringen.
Kraften finns i magnetfältet, som är lagrad
magnetisk energi.
Självinduktion - induktans
Magnetfältets förmåga att alstra en motverkande EMK kallas självinduktion eller induktans. Ordet induktans kommer från latinets
inducere, som betyder införa.
När en ledare, som ingår i en sluten krets,
rör sig i ett magnetfält, så kommer en ström
att flyta genom ledaren på grund av den
EMK (spänning) som alstras. Varje ändring
av strömmen motverkas av det magnetfält
som strömmen själv alstrar.
När det uppstår självinduktion i en ledare,
så kallas ledaren induktor. Självinduktionen
är jämnt utbredd över ledarens hela längd.
När ett större induktansvärde behövs på
något särskilt ställe i strömkretsen, så kan
ledarens längd ökas just där och lindas upp
till en spole med lämplig form. Hela spolen
kallas då för induktor.
Det att ett motverkande magnetiskt fält
alstras omkring en ledare när strömmen i
den ändras, påverkar kretsens egenskaper
och därmed utformning på olika sätt. Vid
snabba strömändringar, t.ex. vid hög frekvens, är motverkan större än vid långsamma
ändringar. Vid konstant likström uppstår
däremot ingen motverkan- självinduktion.
Induktansen är efter resistansen och
kapacitansen den vanligaste egenskapen i
en strömkrets.
Försök med induktion
Försök 1
Bild II 2-3 överst
Ett känsligt vridspoleinstrument kopplas till
en induktor. Instrumentet bör ha noll på
skalans mitt, så att strömriktningen syns. En
permanentmagnet används för att visa att

självinduktion uppstår när magneten förs
fram och tillbaka genom induktorn.
Instrumentet ger utslag när magneten är
i rörelse. Utslaget blir större vid snabbare
hastighetsändring. Utslagsriktningen växlar,
när magneten förs in i respektive dras ut ur
induktorn - det uppstår en växelström.
En växelspänning uppstår över induktorn,
även när den ingår i en strömkrets som sluts
och bryts-alltså utan en magnetsom rör sig.
Försök 2
Bild II 2-3 mitten
Permanentmagneten byts nu mot ännu en
induktor. Utöver den första induktorn, som vi
nu kallar sekundärlindning, kallar vi den nya
induktorn för primärlindning.
När vi släpper ström genom primärlindningen så alstrar den ett magnetfält. Först är
strömmen noll för att sedan ändras till ett
högt värde och därefter återgå till noll. Det
blir en strömstöt
Varje ändring alstrar en mot-emk, som
bygger upp ett magnetfält, först i en riktning
och sedan i den andra. l båda fallen passerar
fältet genom båda lindningarna. Fältet från
primärlindningen inducerar en spänningsstöt i sekundärlindningen. Stöten har en riktning, när primärlindningens strömkrets sluts
och motsatt riktning när den bryts, d.v.s. det
blir en växelspänning. När sekundärlindningen ingår i en sluten krets, uppstår en
växelström genom sekundärlindningen.
Försök 3
Bild II 2-3 nederst
Vad händer när primärlindningen i försök 2
ansluts till en växelspänning, t.ex. med nätfrekvensen 50 Hz? Använd för säkerhets
skull en skyddstransformator mellan nätet
och lindningen!
l sekundärlindningen uppstår då spänningsstötar, vars polaritet i detta fall växlar
100 gånger per sekund. Det uppstår alltså en
växelspänning över sekundärlindningen och
om denna ingår i en sluten strömkrets uppstår det en motsvarande växelström.

112-9

N

R

E

VÄXELSTRÖM

STRÖMSTÖT

STRÖMSTÖT

Fältspole

Primärkrets

Induktionsspole

sekundärkrets
STRÖMSTÖT

l ndu kti ensspole

Pr i märkrets

sekundärkrets
VÄXELSTRÖM

Bild II 2-3 Försök med induktion
112- 1o

K MP NENTER

2

Allmän symbol,
induktor utan kärna
2 Induktor med kärna
3 Trimbar induktor
4 ställbar induktor

4

3

Bild II 2-4 Schemasymboler för induktorer

Olika utföranden
Elektromagneter, drosslar, induktorer för
svängningskretsar, ramantenner o.s.v.
Enheten Henry
Måttenheten för självinduktion är Henry (H)
1 Henry (1 H) är självinduktionen i en induktor, som alstrar en motspänning av 1 volt vid
en strömändring av 1 ampere under 1 sekund.
l formler betecknas induktans med L
Sambandet är
Volt= Henry · Ampere/sekund
1 H är en stor måttenhet. För elektroniktillämpningar används därför ett mer hanterligt format.
Exempel:
1 H= 1000 mH
1 mH = 1 · 1o-3 H
1 mH = 1000 J..LH
1J..LH = 1 · 1o-3 m H= 1 · 1o-s H

Hur induktansen påverkas
Induktansen beror på induktorns mekaniska
dimensioner, antalet lindningsvarv och materialet i kärnan.
Induktansen i en cylindrisk induktor är
proportionell mot tvärsnittsytan, omvänt proportionell mot längden och proportionell mot
kvadraten på lindningsvarvtalet
Induktansen ökar, om induktorn förses
med en kärna av järn och minskar med en
kärna av omagnetisk, ledande metall, t.ex.
koppar, mässing eller aluminium.

Induktiv reaktans
Till skillnad från när en resistor ansluts till en
spänning, så blir strömökningen i en induktor fördröjd. Orsaken är att en induktor inte
bara har en resistans, vilken ju inte påverkas
av strömvariationer, utan har även en induktiv reaktans XL. Ordet reaktans kommer från
latinets re (åter) agere (verka).
Reaktans - växelströmsmotstånd eller
skenbart motstånd - uppträder så länge
som strömmen genom induktorn ändras. En
induktor gör således också motstånd mot
varje strömändring och detta motstånd ökar
med ökande ändringshastighet
En fullbordad pendling i en växelström
kan ses som ett varv i en cirkel - 360° -och
en fullbordad pendling kallas en period.
En period motsvarar omkretsen i en cirkel med radien r, där omkretsen är 2 · n · r
(n = 3.141593 .. ). När strömmen växlar 1
gång/sekund har pendlingen en frekvens [f]
av 1 Hertz [Hz]. Vid 50 växlingar/sekund har
pendlingen en frekvens av 50 Hz o.s.v.
Induktiva reaktansen XL - växelströmsmotståndet i en induktor- är en funktion av
strömmens s.k. vinkelhastighet m= 2 · rc • f
och av storheten av induktansen L.
Den induktiva reaktansen är proportionell mot strömmens frekvens och mot
induktorns induktansvärde. Inga förluster
uppstår i en ideal induktor, d.v.s. en som
teoretiskt saknar resistans.
Sambandet är
X L = 2 · rc · f· L = mL
eller

XL [Q]

f [Hz] L [H]
[MQ]
[MHz] [mH]
(exempel på prefix)

112- 11

K M N
Exempel:
1.
L= 1H
f = 50 Hz
XL = 2 . Jr. 50 . 1= 3 i 4 .Q
2.

L= 1H
f = 5 kHz
XL = 2 . Jr . 5000 . 1= 31400 .Q

Fasförskjutning mellan
och
ström i en induktor
Bild II 3-000 (i kapitel 3)
Med fasförskjutning menas den tidsmässiga förskjutningen mellan ström- och
spänningsförlopp. Strömmen genom en induktor, når inte sitt toppvärde samtidigt som
spänningen över den. Orsaken är växlingarna mellan elektrisk och magnetisk energi i
induktorn.
l en ideal induktor är spänningen fasförskjuten 90° före strömmen. l praktiken är
dock förskjutningen något mindre än 90° på
grund av resistansen i induktorn.
Q-faktor- godhetstal
Q-faktorn kan avse två olika saker, som inte
skall förväxlas. Det är Q-faktorn för en komponent respektive den för en hel strömkrets.
Q-faktorn för en induktor är kvoten av
dess reaktans och serieresistans.
Q

komponent -

xkomponent

R

komponent

Q-faktorn för en hel svängningskrets beror däremot på bredden på det frekvensband som en viss komponentkombination
ger. Q-faktorn för en resonant svängningskrets är därför ett mått på dess selektivitet
(se kapitel 3).
Medan Q-faktorn för en ingående komponent påverkar Q-faktorn för en hel krets,
så gäller inte det omvända.

Yteffekt {skin-effect)
l en ledare av homogent material fördelar sig
en likström likaöverhela tvärsnittet. Strömtätheten för en växelström däremot, minskar i
ledarens mitt och ökar i stället vid ytan. Ju
högre frekvensen är, desto större är strömtätheten vid ytan. Fenomenet kallas yteffekt
(på engelska skin effect) och uppträder i alla
ledare.

112- 12

Det djup i ledarmaterialet där laddningstätheten sjunkit till 37% av värdet vid ytan
kallas skin depth. För koppar är detta djup
c:a 70 mm vid 100 Hz. Vid 1 MHz har djupet
minskat till 0.07 mm och vid 100 MHz till
0.0067 mm. På grund av yteffekten är alltså
materialet i mitten av homogena ledare elektriskt mindre verksamt vid höga frekvenser.
Resistansen blir alltså större för växelström
än för likström, om ledaren är samma
Utöver frekvensen påverkas yteffekten
av ledarmaterialets elektriska och magnetiska ledningsförmåga. För att få låg resistans
i ledare för högfrekvent ström är det viktigt att
omkretsen är stor och att materialskiktet vid
ytan har hög ledningsförmåga. Det är därför
som induktorerna i sändarslutsteg ofta är
försilvrade och består av rör med stor diameter eller av breda band.

Temperaturkoefficient
Liksom med resistorer, så påverkas induktansen av temperaturen. Att sambandet
mellan induktans och temperatur är viktigt,
förstås av att temperaturkoefficienten i den
frekvensbestämmande induktorn i en oscillatorkrets påverkar frekvensstabiliteten.
Eftersom metallen koppar utvidgar sig
vid temperaturökning och induktorns tvärsnittsyta då blir större, så är temperaturkoefficienten vanligen positiv.
Temperaturkoefficienten aL anger induktansändringen per grad temperaturändring.
Induktansändringen blir då
!lL =\(\pm\)aL· Lk ·llfJ

varvid Lkär induktansvärdet vid den lägre
temperaturen (oftast 20 °C) och fl{} är
temperaturändringen i oKelvin.
Kelvin [K] är den normerade måttenheten
för absolut temperatur. En ändring med 1 K
motsvarar en ändring med 1 oc.
Induktorer kan innehålla kärnor av någon
metallegering, vars egenskaper också är
temperaturberoende.
l praktiken kan man knappast påverka
temperaturkoefficienten i en induktor. Eftersom en svängningskrets för det mesta även
innehåller kondensatorer, så kan man t.ex.
kompensera en positiv temperaturkoefficient
i induktorn med en negativ i en kondensator.

PT

NENTER

Förluster i kärnmaterial
När ett magnetiskt växelfält passerar ett
kärnmaterial så kommer atomerna (som är
permanentmagneter) att ständigt inta nya
lägen i materialet i takt med fältets frekvens.
Då uppstårvirvelström mar, s.k. järnförluster,
som dels påverkar materialets ledningsförmåga och dels höjer temperaturen i kärnan och därmed i hela induktorn.
