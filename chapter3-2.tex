\section{Frekvensfilter}

Frekvensfilter används inom radiotekniken
för många olika ändamål, t.ex. för att
• eliminera störande signaler,
• öka avstämningsskärpan (selektiviteten) i
mottagare och sändare,
• framhäva eller dämpa ett sidband i en AMsignal m.m.
Beroende på den s.k. frekvensgången,
så indelas filtren i flera "familjer", varav de
vanliga presenteras här.
Beroende på det tekniska utförandet finns
dels s.k. passiva filter vilka använder extern
energi för sin funktion, och dels aktiva filter
vilka i princip är förstärkare som likaledes
använder passiva kretsar. Här presenteras
för enkelhetens skull passiva filter.
Traditionella frekvensfilter är vad som
kallas analoga. Men nu i dataåldern börjar
även digitala filter vinna intåg. Sådana är
dock för komplicerade för att behandlas här.

Högpassfilter
Bild II 3-22

Ett högpassfilter släpper igenom signaler
med höga frekvenser och dämpar dem med
låga frekvenser.
Exempel: En frekvensberoende spänningsdelare som LC-högpassfilter.
.. Vid låga frekvenser är Xc stor och XL liten.
Over XL uppstår då ett litet spänningsfall- en
låg utgångsspänning ua. Resultatet blir att
låga frekvenser dämpas.
Vig höga frekvenser är Xc liten och XL
stor. Over XL uppstår då ett stort spänningsfall- en hög utgångsspänning ua. Resultatet
blir att höga frekvenser släpps igenom.
XL kan bytas ut mot en resister R, men då
blir passbandkurvan inte så brant.

Gränsfrekvens
Gränsfrekvensen f9 beror av kapacitansen
C, induktansen L samt resistansen R.

f

LC-högpass:

=

g

f9 [Hz]

C [Farad]

lågpassfilter

Bild II 3-23
Om induktor och kondensator respektive
resister och kondensator i ett högpassfilter
byter plats, så får man i stället ett LC-Iågpassfilter respektive ett RC-Iågpassfilter.
Ett lågpassfilter släpper igenom signaler
med låga frekvenser och dämpar dem med
höga frekvenser.
Exempel: En frekvensberoende spänningsdelare som LC-Iågpassfilter.
.. Vid låga frekvenser är Xc stor och XL liten.
Over XL uppstår då ett litet spänningsfall- en
hög utgångsspänning ua. Resultatet blir att
låga frekvenser släpps igenom.
Vig höga frekvenser är Xc liten och XL
stor. Over XL uppstår då ett stort spänningsfall -en låg utgångsspänning ua. Resultatet
blir att höga frekvenser dämpas.

Gränsfrekvens
Samma formler används vid beräkning av
gränsfrekvensen både i lågpass- och högpassfilter, således
LC-Iågpass: ~

f9 [Hz]

L [H]

RC-Iågpass: ~

f9 [Hz]

1

= 2 rc{LC
C [F]

1

= 2 rcRC

C [F]

R [Q]

C [Farad]

f =g

f9 = ?

1
10 3
~=
=-=79.62 Hz
2rc~ 4. 2rc ·1 o-B
4n
2) R= 1 kQ C= 1O n F f9 =?
1
10 5
~=
3
9 = = 15.934
2rc·1·10 ·10·102rc

1
2rc-fLC

L [Henry]
RC-högpass:

Räkneexempel:
1) L = 4 H C = 1 J.lF

1
-

2rcRC
R [Ohm]

113-17

KRETSAR

L

LC- HÖGPASS

Ua

Ekvivalentschema

Xc
R
RC -

HÖGPASS

Ekvivalentschema

Låga frekvenser dämpas

L

Höga frekvenser släpps igenom

Bild II 3-22 Högpassfilter

113-18

fg
Passbandskurva

f

KRETSAR

~©~CEPT

L

XL

Te

Ue

Uu

Ue
Xc

Ua

o

o
LC- LAGPASS

o

o

R
CJ

Ue

Ekvivalentschema

o

le

R

Ua.

I

o

Ue

Ekvivalentschema

RC- LAGPASS

L

o-------1 t ' r n r r •

o

IV Ile IV
o

Ua

Xc
o

-----o

Låga frekvenser släpps igenom

100

Ue

70

fg

o,..I..(-o

f

Passbandskurva

HöQa frekvenser dämoas

Bild II 3-23 Lågpassfilter

113- 19

KRETSAR
Bandpassfilter

Bild 3-24
Ett bandpassfilter släpper igenom signaler
bara inom ett frekvensområde medan signaler inom andra frekvensområden dämpas.
Bandpassfiltret består i enklaste fall av
två svängningskretsar av LC-typ, vilka är
avstämda till angränsande frekvenser. Kretsarna är kopplade induktivt, kapacitivt eller
galvaniskt.

Beroende på kopplingsgrad skiljer man
mellan underkritisk koppling (lös koppling),
kritisk koppling och överkritisk koppling (fast
koppling).
På bilden visas hur passbandet påverkas
bl.a. av kopplingsgraden. Lös koppling liten bandbredd. Kritisk koppling - större
bandbredd. Fast koppling- stor bandbredd.

KOPPLINGSSÄTT

Induktivt

Kapacitivt

Galvaniskt

fr
Underkritisk koppling

Bild II 3-24 Bandpassfilter

113-20

Kritisk koppling

f

fr
överkritisk koppfing

f

KRETSAR
Passfilter

Bild 113-25
Passkretsen stäms av till en viss frekvens
och erbjuder där en mycket låg impedans.
Passkretsen kopplas i serie med signalvägen och låter signaler med frekvenser inom
filtrets passband att passera.

fr
Bild II 3-25 Passfilter

Bandspärrfilter

Bild II 3-26
Om serie- och parallellkretsarna i ett bandpassfilter byter plats, så får man i stället ett
bandspärrfilter. Ett sådant spärrar signaler
inom ett visst frekvensområde, men släpper
igenom signaler utom detta område.

o

uin

I

o

I

CJ

I

I

uut
o

uut
o

Bild II 3-26 Bandspärrfilter

Spärrfilter

Bild II 3-27
Spärrkrets
Spärrkretsen stäms av till en viss frekvens
och erbjuder där en mycket hög impedans.
Spärrkretsen kopplas i serie med signalvägen och spärrar en signal med samma
frekvens som resonansfrekvensen.

Bild 113-27
sugkrets
sugkretsen stäms av till en viss frekvens och
erbjuder där en mycket låg impedans. sugkretsen kopplas parallellt med signalvägen
och kortsluter (suger bort) en signal med
samma frekvens som resonansfrekvensen.

113-21

KRETSAR

o

o

uin

o

uut

uut

·SPÄRRKRETS

I

o

I

uin
o

f

o

fr

uut
o

SUGKRETS

Bild II 3-27 Spärrfilter (2 sorter)

Kvartskristall
Bild II 3-28

Bandfilter med kvartskristaller

En kvartskristall, egentligen en slipad skiva

av kvarts, kan fungera som en elektromekanisksvängningskropp (resonator), vars egenskaper liknar dem i en LC-krets.

Den låga inre resistansen gör att Q-värdet i en kvartskristall är bättre än 10000.
Som jämförelse är Q-värdet i en LC-krets
oftast sämre än 1000.

Bild II 3-29
Kvartskristaller kan kombineras till filter med
önskad bandbredd. Även utföranden med
keramiska resonatorer finns.
Resonatorerna är avstämda till var sin
bestämda frekvens och hela komplexet bidrar på så sätt till att bilda passband eller
andra egenskaper på samma sätt som med
sammankopplade LC-kretsar.

j
[:=J

T
Schemasymbol

Ekvivalentschema

Bild II 3-28 Kvartskristall
113-22

Bild II 3-29 Bandfilter med kvartskristalle

KRETSAR
Mekaniska filter

Bild II 3-30
Med en elektromekanisk givare kan man få
en kropp (resonator) att svänga på sin resonansfrekvens. Med ännu en elektromagnetisk givare kan man känna av svängningarna och återvandla dem till elektriska signaler. Hela anordningen fungerar som en elektromekanisk resonator, vars egenskaper liknar dem i en LC-krets.
Resonatorerna kan kombineras till filterkomplex med önskad bandbredd där resenatorerna är avstämda till var sin bestämda frekvens. Hela komplexet bidrar på så
sätt till att bilda ett passband på samma sätt
som med sammankopplade LC-kretsar.
Beroende på tillämpningen finns olika frekvenslägen i intervallet 60-600 kHz.
Mekaniska filter användes mest förr som
mellanfrekvensfilter i högvärdiga radioutrustningar, men har numera till stor del ersatts
av bandfilter med kvartskristaller där arbetsområdet kan ligga avsevärt högre i frekvens.

Bild II 3-31 Kavitetsfilter
Inkommande och utgående signaler ansluts till filtrets mittledare över induktionskondensatorer eller direkt galvaniskt.
kavitetsfilter kan kopplas ihop för att
bilda bandfilter, frekvensdelare m.m ..

Helixfilter

Kavitetsfilter
Bild II 3-31

Svängningskretsars dimensioner minskar
med ökande frekvens. Vid mycket hög frekvens kan induktorns varvtal i en LC-krets ha
minskat till ett enda varv samtidigt som kapacitansen inom detta enda varv kan räcka
för önskad resonansfrekvens.
En sådan svängningskrets kan bl.a. ha
formen av en ledare mitt inne i en elektriskt
ledande kavitet. Ledarens längd tillsammans
med kavitetens insida bildar induktorn. Mellan ledaren och kavitetens insida råder en
kapacitans, som kan kompletteras/justeras
med en extra kondensator.

Givare

När ett kompakt kavitetsfilter behövs, så kan
man öka reaktansen i mittledaren både induktivt och kapacitivt genom att utforma den
som en spiral (helix). Detta är dock på bekostnad av Q-värdet. Flera kavitetsfilter kan
kopplas ihop för att bilda bandfilter, spärrfilter m.m ..

Resonanskroppar

Avkännare

]!~CJ··C) C)EJ]~
Bild II 3-30 Mekaniskt filter

113-23

KRETSAR
Pi-filter

Bild II 3-32
För att överföra H F-signaler med bästa verkningsgrad, så är det viktigt med god impedansanpassning mellan de olika funktionerna. Om anslutningsimpedansen är lika i båda
~~nktionerna, så behövs inga extra åtgärder.
Ar impedanserna däremot olika, så behövs
korrigeringsnät (filter).
Ofta är nätet Pi-format och består av
induktanser och kapacitanser. Ett Pi-format
nät kan sägas bestå av två L-formade nät
ställda mot varandra, där den seriella delen
är gemensam (på bilden en induktor).

T-filter

Bild II 3-33
Ett nät kan också vara T -format och bestå av
induktanser och kapacitanser. Ett sådant
nät kan sägas bestå av två L-formade nät
ställda "rygg mot rygg". Då är den parallella
delen gemensam. På bilden visas två alternativ.
När den parallella delen är kapacitiv, blir
huvudkaraktären ett lågpassfilter, men att
impedansanpassning också är möjlig med
en induktiv impedansdelning.
När den parallell delen är induktiv blir
huvudkaraktären ett högpassfilter, men att
impedansanpassning också är möjlig med
en kapacitiv impedansdelning.

Ett Pi- eller T-filter kan fungera som
• svängningskrets,
• impedanstransformator (anpassning),
• balansera ut en reaktans o.s.v.

o

·O

l

el
I

w f •

Bild II 3-32 Pi-filter
113-24

L

J t 'f , '

le
I

o

o

(

(

o

o

Bild II 3-33 T-filter
