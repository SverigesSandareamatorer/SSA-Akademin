\chapter{El-lära}

\section{Elektriska grundbegrepp}

Elektrisk laddning, spänning och ström hänger samman med hur materian är uppbyggd.
Den förmåga ett material har att leda laddningar, d.v.s. ström, kallas konduktivitet.

\subsection{Grundämnen}

Det finns många former av materia. Ofta är en form av materia sammansatt av andra
former med enklare uppbyggnad.
Sammansatt materia kan sönderdelas på kemisk väg, men
däremot inte de enklaste formerna. All materia är uppbyggd av atomer. De enklaste
materieformerna, som kallas grundämnen, innehåller endast ett slags atomer. Över
100 grundämnen är kända.
Vart och ett av grundämnena har sin speciella atomuppbyggnad och därmed en
materialstruktur, som skiljer sig från varje annat grundämne.
Tre fjärdedelar av alla grundämnen är metaller (elektriska ledare) medan de flesta
övriga är icke-metaHer (isolatorer). Det finns även en liten mellangrupp som kallas för
halvledare.

\subsection{Atomernas uppbyggnad}

Länge ansågs atomerna vara de minsta
beståndsdelarna i materian. Men omkring
sekelskiftet upptäcktes att atomerna i sin tur
består av ännu mindre beståndsdelar, s.k.
elementarpartiklar såsom protoner, neutroner, elektroner m .fl. Det gemensamma namnet för alla dessa partiklar är nukleoner.
En atom består dels av en kärna, som är
sammansatt av protoner och neutroner, dels
av elektroner, som kretsar omkring kärnan.

Protonerna är positivt (+)laddade.
Neutronerna är neutrala, ej laddade.
Elektronerna är negativt (-) laddade
Bild II i -1 Elektronerna kretsar i banor
omkring atomkärnorna, liksom planeterna
kretsar i banor omkring sina solar.

Atomkärna
Proton
'--\-~~- Neutron
-

Elektronskal

Bild II 1-1 Atomernas uppbyggnad
Banor med samma avstånd till atomkärnan är på samma energinivå och sägs bilda
ett elektronskal.
Det kan finnas flera elektronskal. Ju fler
elektroner som finns i ett elektronskal, desto
starkare är elektronerna i skalet bundna till
atomen. Det yttersta skalet kan emellertid
aldrig innehålla fler än 8 elektroner.
Elektronerna i det yttersta skalet kallas
för valenselektroner, vilka används även av
angränsande atomer vid den kemiska bindningen till atomstrukturer, molekyler och
ämnen. För bindningen behövs ett visst
antal valenselektroner.

De valenselektroner som ej behövs för
bindningen kan röra sig fritt genom materia/strukturen. De kallas fria elektroner och är vad vi kallar elektrisk ström.
Valenselektronerna är alltså inte bara av
betydelse för materialets kemiska struktur
utan också för dess elektriska egenskaper.
Atomernas massa och volym är ytterst
liten. Tag som exempel en kopparkub med
volymen 1 cm 3 och vikten 8.9 gram. Den
består av c:a 8.5 · 1025 kopparatomer, d.v.s.
85 000 000 000 000 000 000 000 000
stycken.

IIi- i

ElVarje elementarpartikel har en massa och
en atoms totala massa är summan av
elementarpartiklarnas massor. Den enklaste
atomen är väteatomen med en proton och
en elektron. Väteatomens totala massa har
kunnat beräknas till 1.66 . 1o- 24 gram.
Nästan hela massan i atomen är samlad
till kärnans protoner och neutroner. Var och
en av dem har en massa som är ungefär
2000 gånger större än massan i en elektron.
1 cm 3 av koppar innehåller t. ex. 10 23 stycken
fria elektroner.

\subsection{Elektrisk laddning och kraftverkan}

Enligt sägnen upptäckte Thales från Milteus
redan för 2500 år sedan, att en bit bärnsten
drog till sig små grässtrån, sedan stenen
gnidits mot en bit ylle. Det grekiska ordet för
bärnsten är ELEKTRON och de krafter som
uppstod kom att kallas "elektriska". Av den
elektriska spänningen mellan kroppar med
olika laddning, verkar krafter mellan dem
och deras omgivning. Krafterna kallas för
elektriska fält och är det som gör att elektriskt
laddade kroppar kan komma i rörelse.
Ett exempel får man varje gång man
kammar sig med en kam av isolerande material. Då kommer håret att dras mot kammen därför att håret och kammen har fått
olika slags elektriska laddningar. Samtidigt
har hårstråna sinsemellan samma slags laddning och stöter bort varandra- håret "reser
sig".
Lika laddningar stöter bort varandra.
Olika laddningar drar varandra till sig.

\subsection{Konduktivitet - Ledare, halvledare och isolator}

En elektrisk ström sägs flyta, när de fria
laddningsbärarna i ett material -en strömledare - fås att röra sig samtidigt i samma
riktning. Hur många som rör sig beror på
strömledarens egenskaper och spänningen
mellan ledarens ändar.
Alla material har någon grad av elektrisk
ledningsförmåga som beror på materialets
atomstruktur, dimensioner och temperatur.
Vissa material (t.ex. metaller, kol, halvledare) leder elektrisk ström bättre än andra
(t.ex. glas, gummi, plast). Mängden av fria
laddningsbärare i materialet begränsar hur
stor strömmen kan bli.

111-2

EPT
\subsubsection{Ledare}
Metaller har god elektrisk ledningsförmåga
och kallas ledare. Bäst ledande är de metaller, vars atomer har det minsta antalet valenselektroner i det yttersta elektronskalet. Koppar-, silver- och guldatomerna har en enda
valenselektron och därmed mycket god ledningsförmåga. Järn, zink och magnesium
har två valenselektroner och därmed något
sämre ledningsförmåga. Ännu sämre ledare
är de s.k. halvledarna med 3 till 5 valenselektroner.

\subsubsection{Isolatorer}
Glas, plast, porslin, pertinax, vissa mineraler
etc. har mycket dålig ledningsförmåga och
kallas isolatorer. Isolatorerna är dåliga ledare därför att de har så många valenselektroner. Det största möjliga antalet är 8
stycken.
l icke ledande material är elektronerna
mycket hårt bundna till sitt valensskal och
därför svåra att flytta. l fasta material är
också positiva laddningar svåra att flytta,
eftersom de är bundna i atomkärnorna. Atomerna är i sin tur bundna i en struktur som
kännetecknar vart och ett material.

\subsubsection{Halvledare}
Till exempel en ren kristall av mineralen
germanium [Ge] eller kisel [Si] leder inte
elektrisk ström. Båda dessa mineral är därför isolatorer. Men om några atomer av ett
främmande material blandas in i deras kristaller, så blir de i någon mån elektriskt ledande- de blir halvledare. Inblandningen är
1 eller 2 främmande atomer per 100 millioner
germanium- eller kiselatomer. Liknande resultat fås med andra material. Beroende på
materialen och i vilka proportioner de blandas fås olika egenskaper.

\subsubsection{N-ledning}
Man talar om N-ledande material respektive
N-ledning- "elektronledning".
Germanium, kisel m.fl. halvledare har
fyra elektroner med "fasta platser" i valensskalet - förutsatt att materialet är helt
rent. Då finns det inga fria elektroner för
laddningstransport.
För att skapa fria elektroner kan det rena
materialet förorenas- dopas - med atomer
av t. ex. arsenik [As] eller antimon [Sb]. Båda
dessa material är 5-värdiga. De har 5 elektroner i valensskalet
4 elektroner är fast bundna medan den 5:e är
löst bunden till atomen. Den 5:e elektronen
kan lossgöras från atomen med yttre
kan, t. ex. värme eller elektrisk spänning och
då skapas en fri elektron. När en spänning
läggs på materialet kommer den fria elektronen att vandra mot den positiva polen. Materialet är N-ledande.

Bild II 1-2 Tankeförsök med kulor i ett rör

\subsubsection{P-ledning - "hålledning"}
När germanium eller kisel dopas med indium
[In] eller gallium [Ga] så blir de P-ledande.
Indium och gallium är 3-värdiga - deras
valensskal innehåller 3 elektroner. Men för
en fast bindning med germanium eller kisel
saknas det en elektron och det uppstår då ett
"hål" - en "bristelektron". Hålet kan fyllas ut
av en elektron från en annan atom. l den
atom som elektronen lämnar bildas det i sin
tur ett hål o.s.v. När en spänning läggs på,
kommer "hålet" att vandra mot den negativa
polen. Materialet är då P-ledande.

\subsection{Elektrisk spänning - Enheten Volt}
Bild II 1-2

I ett tankeförsök med ett rör med kulor i, tänks materialet i röret motsvara
atomstrukturen i en strömledare och kulorna de fria elektronerna. Tänker man sig ett slag
mot en ände av röret så flyttar det sig av den energi som tillförs. På grund av
obundenheten till röret så följer av masströgheten kulorna inte med röret, utan hamnar i dess ena ände.

Att kulorna samlas i ena änden av röret tänks motsvara ett elektronöverskott i ena
änden av en ledare och ett motsvarande underskott i den andra änden.

Man kallar änden med elektronöverskott för minuspol och änden med elektronunderskott för
pluspol. Olika stora elektriska laddningar vid polerna innebär att de sinsemellan har
olika potential. Potentialskillnaden kallas spänning.

Likspänning innebär ett överskott av elektroner och alltid vid samma
anslutningspol.

Växelspänning innebär ett överskott av elektroner, omväxlande vid den ena
anslutningspolen och den andra.

Måttenheten för spänning är Volt [V].
I formler betecknas spänning med
U för effektivvärdet
u för momentanvärdet (ögonblicks-)
û för toppvärdet (amplitud-)

Spänningen över ändpunkterna på en
strömledare är 1 Volt [V], då ledaren
genomflyts av en likström av 1 Ampere
[A] under effektutvecklingen 1 Watt (W].

\subsection{Symboler}

När man ritar scheman för elektriska kretsar, används symboler. Följande symbol visar
ett elektriskt batteri med en enda cell.

Förtydligande kommentarer och skrivtecknen invid symbolen förekommer. Ofta
referar dessa till en komponentlista. Se f.ö. i kapitel 2. Komponenter.

\subsection{Elektrisk ström - Enheten Ampere}

När en sluten strömkrets innehåller en spänningskälla, så kan en laddningsutjämning en
ström- ske genom kretsen. Det innebär att fria elektroner förflyttar sig genom kretsen i
riktning från spänningskällans minuspol till dess pluspol. Vid pluspolen är det nämligen
brist på negativa laddningar och naturen söker alltid en utjämning. Under
utjämningsförloppet är spänningskällan även en strömkälla.

I gaser och elektrolyter (elektriskt ledande vätskor och geler) samt i halvledare består
strömmen av joner (positiva eller negativa laddningar); i metaller däremot av elektroner
(negativa laddningar).

Av tradition anses strömriktningen vara positiv i jonströmmens riktning - den s.k.
tekniska strömriktningen - medan elektronströmmens riktning är den motsatta - den
s.k. fysikaliska strömriktningen.

Måttenheten för ström är Ampere [Aj.
I formler betecknas ström med
I för effektivvärdet,
i för momentanvärdet (ögonblicks-),
i för toppvärdet (amplitud-).

Bild II 1-3 Potential och spänning i en strömkrets

Strömmen är \(1 A\), när \(6.25 \cdot 10^{18}\) elektroner per sekund flyter genom ett givet
ledartvärsnitt, vilket motsvarar laddningen 1 Coulomb.

\subsection{Strömkrets}

Bild II 1-3
En elektrisk strömkrets består av en eller
flera energikällor och energiförbrukare. Källor kan vara batterier, nätaggregat etc. Förbrukare kan vara lampor, ledningar etc.
Varje energiförbrukare har en resistans
och de elektriska laddningarna "köar" före
förbrukaren. strax efter förbrukaren finns
ingen kö.
Det uppstår en skillnad i laddningsmängd
(en potentialskillnad) mellan varje punkt i en
strömkrets, när det flyter ström. Man talar om
spänningsfall.

\subsection{Strömförlopp}

Likströms- och växelströmsförloppen kan vara sammansatta av ett huvudförlopp och
underordnade förlopp.

Likström kan ha konstant styrka eller den kan variera enligt något förlopp,
men växlar aldrig riktning. Växelström kan variera enligt något
visst förlopp, t.ex. sinusvåg, fyrkantvåg, och växlar ständigt riktning.

\(R_1 \cdot l_1 + R_2 \cdot I_2 + \cdots R_n \cdot I_n = U_1 + U_2 + \cdots U_n\)

\subsection{Resistans - Enheten Ohm}

Närfria elektronertvingas fram genom atomstrukturen i en ledare, t.ex. glödtråden i en
lampa, så avgår energi i form av värme.
Detta fenomen kallas för resistans (av latinets resistere som betyder att motstå).
Resistansen och därmed förlusterna i en
strömkrets fördelas i förhållande till de ingående materialen och deras dimensionering.

Resistans uttrycks i enheten Ohm och betecknas med den grekiska bokstaven
omega (\(\Omega\)).
I formlerbetecknas resistansen i en elektrisk krets eller en del av den med R.

Resistansen i en resistor är 1 Q, när en
spänning av 1 V driver en ström av 1 A
genom den resistorn.

\subsection{Ohms lag}
Ohms lag beskriversambandet mellan grundbegreppen ström I [ampere], spänning U
[volt] och resistans R [ohm].
Sambandet gäller både för likspänning och effektiwärdet för växelspänning och
växelström.

I en ledare med resistansen R är strömstyrkan l genom resistansenproportionell
mot den pålagda spänningen U.

\(\begin{array}{ccc}U=I \cdot R & I=\frac{U}{R} & R=\frac{U}{I}\end{array}\)

\subsection{Kirchhoffs lagar}

Den tyske fysikern G R Kirchhoff (1824-1887) formulerade sina välkända lagar, först
1845 och sedan 1847.

\subsubsection{Kirchhoffs strömlag}

Den algebraiska summan av alla strömmar, som flyter till eller från varje punkt i
en elektrisk krets, är lika med noll.

\(I_1 + I_2 + I_3 + \cdots + I_n = 0\)

\subsubsection{Kirchhoffs spänningslag}

I varje sluten strömkrets är den algebraiska summan av alla spänningskällor lika
med det totala spänningsfallet i alla resistorer.

Uttryckt på ett annat sätt är algebraiska summan av spänningarna i en strömkrets lika med
noll.

\subsection{Elektrisk effekt - Enheten Watt}

När en ström flyter genom en resistans utvecklas värme. Värme är en form av effekt,
som är högre ju starkare strömmen och högre spänningen är.
Måttenheten voltampere [VA] för elektrisk effekt härleds ur produkten av volt [V]
och ampere [A].
För effekt som alstras av likström används enheten Watt [W] i stället för voltampere [VA]. Vid sidan om grundenheten 1 W används delar och multipler av denna.

\(1 volt [U] \cdot 1 ampere [l]= 1 watt [P]\)

Effektformeln \(P = U \cdot I\) kan skrivas om på
flera sätt. Den gäller i första hand för likström,
men även för växelström, om ström och
spänning inte är fasförskjutna, vilket är fallet
när belastningen är resistiv

\(
\begin{array}{lll}
U = R \cdot I & I = \frac{U}{R} & R = \frac{U}{I} \\
P = U \cdot I & P = \frac{U \cdot U }{R} & P = \frac{U^2}{R} \\
&U = \sqrt{P \cdot R} & \\
P = R \cdot I \cdot I & P = R \cdot I^2 & P = R \cdot I^2
\end{array}
\)

Med hjälp av dessa formler kan effekten beräknas ur resistans- och strömvärdena
respektive ur resistans- och spänningsvärdena.

\subsection{Elektrisk arbete - Enheten Joule}

Energi finns i olika former, alltid och överallt.
Energi kan varken skapas eller förstöras,
bara omvandlas från en form till en annan.
Formen kan vara mekanisk, kemisk, elektrisk etc.
Arbete är omvandlingsprocessen från
en energiform till en annan.
Arbetsmängden i alla energiformer kan
mätas med samma enhet- Joule [J].

Bild II 1-4 "Formelsnurra" för
Ohms och Joules lagar

1 Joule motsvarar det arbete som utvecklas när ett föremål förflyttas 1 meter
med kraften 1 Newton [N], d. v. s. 1 Newtonmeter [Nm].
Arbetet [W=Work] är mer ju längre tid [s]
en viss effekt [P=Power] utvecklas.

\subsection{Joules lag}

\(Arbete = Effekt \cdot tid\)

\([W] = [P] \cdot [s]\)

Eftersom effekten uttrycks som \(P = U \cdot I\)
så kan det elektriska arbetet uttryckas som
\(W = U \cdot I \cdot t\), vilket också är Joules lag.
Om grundenheterna för volt [U], ampere
[l] och sekund [s] sätts in i formeln fås en
måttenhet, uttryckt som voltamperesekunder
[VAs] eller wattsekunder [W s] eller joule [J].
Måttenheten för elektriskt arbete är 1
Joule per sekund, som vanligen kallas 1
wattsekund [1 Ws] eller helt enkelt watt [W].
Vid sidan avgrundenheten används multipler
av denna.
Exempel:
\(
\begin{array}{lll}
1 kilowattsekund & = 1 kWs & = 1 000 Ws \\
1 wattimme & = 1 Wh & = 3600000 Ws \\
 & & = 3.6 · 10^6 Ws \\
1 kilowattimme & = 1 kWh & = 1 000 Wh \\
 & & = 3.6 · 10^9 Ws
\end{array}
\)

\subsection{Formelsnurran}

Bild 111-4

Så här finner man rätt formel i "snurran":
Välj ett segment med önskad storhet I, U, R
eller P som det första ledet i formeln. Inom
valt segment finns tre alternativ för det andra
ledet i formeln. Välj det alternativ som innehåller två kända storheter.

Exempel:

\subsubsection{Ohm's lag}

R söks, U och I är kända;
Om \(U = 230 V\) och \(I = 2 A\), så blir

\(R=\frac{U}{I}=\frac{230}{2}=115 \Omega\)

\subsubsection{Joule's lag}

P söks, U och I är kända;

Om \(U = 230 V\) och \(I = 2 A\), så blir

\(P = U \cdot I = 230 \cdot 2 = 460 W\)

\subsection{Amperetimmar (Ah) och batterikapacitet}

Det finns flera sätt att lagra energi. Ett sätt är att göra det i kemisk form i speciella
celler, där man kan ta ut energin i elektrisk form.

Det finns celler som kan laddas upp och laddas ur upprepade gånger, s. k. ackumulatorer.
Det finns också sådana celler som endast kan användas en gång och som inte
kan laddas upp igen, s. k. primärceller.

Energi i form av en elektrisk laddning kan även lagras i en kondensator. Energin kan
då lagras och tas ut utan omvandling.


Kapaciteten i en elektrisk cell uttrycks som produkten av den ström [A] som
cellen avger och under den tid [s, h] detta kan ske.
Uttryckt med tidsenheten timmar blir då kapaciteten Ah.


Den kapacitet som anges i en cells produktdata är den nominella. Denna kapacitet
gäller endast under vissa normerade förhållanden såsom celltemperatur, strömstyrka
och urladdningstid.

Den praktiska kapaciteten i en cell begränsas av användningen. En elektrisk cell
avger sålunda regelmässigt mindre energimängd, desto högre urladdningsströmmen
är. Kapaciteten i en elektrisk cell skiljer sig i det avseendet från den i t.ex. en
oljetank, där man kan ta ut lika mycket energimänd som man häller i och oberoende av hur
fort man gör det.


Elektriska celler kan samlas till s.k. batterier, varvid cellerna oftast seriekopplas.
Batteriets polspänning är då summan av cellernas polspänningar.

Hur stort arbete ett batteri avger, beror då såväl på hela batteriets polspänning som på
de enskilda cellernas kapacitet.
Exempel.
Ett batteri med polspänningen \(12 V\) och cellkapaciteten \(100 Ah\) kan nominellt avge
\(P = U \cdot I = 12 \cdot 100 = 1200 VAh = 1.2kWh\).

Hur länge batteriet "räcker" per laddning beror som sagt bl.a. på vilken strömstyrka
man tar ut. Tar man ut \(1 A\) ur \(100 Ah\)-cellen här ovan, så blir urladdningstiden
nominellt \(t = 100 Ah/1 A = 100 h\).

\cleardoublepage

\section{Elektriska kraftkällor}

\subsection{Elektromotorisk kraft - EMK}

Det som driver ström genom en elektrisk strömkrets är kretsens elektromotoriska kraft
(EMK).
Måttenheten för EMK är Volt [V]. EMK är summan av de potentialökningar som uppstår i
kretsen.

De vanligaste slagen av emk är
\begin{itemize}
\item elektromagnetisk emk som uppkommer i
strömledare i magnetfält som varierar
(ex. lindningarna i en roterande generator),
\item elektrokemisk emk som uppkommer i
beröringsytan mellan en metallisk ledare
och en elektrolyt (ex. battericell),
\item elektrostatisk emk, t. ex. i kondensatorer,
\item kontaktemk i beröringsytan mellan metaller med olika termoelektrisk potential
eller mellan metall och luftens syre (ex.
korrosion mellan metaller),
\item termoemk som uppkommer i en strömkrets där två sammanlödda metaller med
olika temperatur ingår (ex. termokors för
strömmätning).
\end{itemize}

\subsection{Polspänning}

Den spänning, som kan mätas mellan kretsens anslutningspoler då kretsen är öppen.

\subsection{Inre resistans}

Liksom att komponenterna i strömkretsen har en viss resistans, så har också en strömkälla
en inre resistans. Den inre resistansen i en strömkälla ingår i kretsens totala resistans.

\subsection{Kortslutningsström}

Om man på kortaste väg förbinder strömkällans anslutningspoler så blir kretsen totala
resistans lika med källans inre resistans.

Den kortslutningsström som då uppstår, begränsas enbart av strömkällans polspänning och
inre resistans.

Eftersom den inre resistansen oftast är mycket liten blir kortslutningsströmmen
motsvarande hög.

\subsection{Serie- och parallellkopplade kraftkällor}

\subsubsection{Seriekopplade kraftkällor}

För att uppnå en högre total spänning (emk)
kan flera kraftkällor (delspänningar) kopplas
i en slinga efter varandra. Detta kallas seriekoppling.

Seriekopplade delspänningarverkar med
eller mot varandra, beroende på deras
inbördes polariteter.

Den totala spänningen över kopplingen
är summan av de ingående de/spänningarna, med hänsyn taget till deras
polariteter.

\subsubsection{Parallellkopplade kraftkällor}

För att erhålla högre ström, kan flera svagare kraftkällor parallellkopplas. Vid
parallellkoppling erhålls däremot inte högre spänning.

Vid parallellkoppling av kraftkällor måste
deras polaritet vara lika.

För minsta utjämningsström mellan parallellkopplade kraftkällor bör även deras
polspänning och inre resistans vara så lika
som möjligt.

\cleardoublepage

\section{Elektriskt fält}

\subsection{Potential}

Potentialskillnaden - spänningen - mellan olika laddade kroppar, skapar krafter mellan
varandra samt mellan dem och deras omgivning. Detta fenomen kallas elektriskt kraftfält och är orsaken till att elektriskt laddade kroppar kan komma i rörelse.

\subsection{Elektrisk laddning}

Elektriska laddningar är grunden för elektricitetsläran. Varje proton i atomkärnan är
bärare av en positiv laddning. Neutronerna i atomkärnan är elektriskt neutrala. Antalet
protoner i kärnan bestämmer därför ensamt kärnans totala positiva laddning, kallat för
kärnladdningstalet Elektronerna som kretsar omkring atomkärnan är bärare av var sin
negativa laddning.

Elementarladdningen [ e ] är den laddning som finns i en elektron och har länge
ansetts vara den minsta möjliga laddningen. Nutida elektronfysik konstaterar ännu
mindre enheter, men detgår vi inte in på här.

Antalet protoner och elektroner i en atom är lika och elektronernas samlade negativa
laddning blir då lika stor som protonernas samlade positiva laddning. När laddningar med
olika polaritet är lika stora väger de ut varandra och blir elektriskt neutrala till sin
omgivning.

Måttenheten för elektrisk laddning är Coulomb [C].

Laddningsmängden 1 Coulomb motsvarar 6.25 trillioner (\(6.25\cdot10^{18} \)) elementarladdningar.

Sambandet mellan laddning och ström är

\(Q = I \cdot t\)

Laddning [Q] är ström [l] under tiden [t]

\(1 C= 1 A ·1 s= 1 amperesekund [1 As]\)

\(1 Coulomb = 1 Ampere·1 sekund\)

\subsection{Kraftfält omkring elektriska laddningar}

Bild II 1-5
Mellan elektriska laddningar bildas krafter.

\begin{itemize}
\item Varje laddning är omgiven av ett elektriskt kraftfält.
\item Mellan positiva (+) elektriska laddningar
och (-) negativa laddningar bildas krafter.
\item Fältkrafternas styrka och riktning symboliseras som linjer mellan positiva och
negativa laddningar, där styrkan är densamma utmed respektive linje.
\end{itemize}

(även 1.1)

Kroppar med olika slags laddningar dras
till varandra

Kroppar med lika slags laddningar stöter bort varandra

Oladdade kroppar påverkas inte och ger ingen kraftverkan.

\subsection{Elektrisk fältstyrka}

I en trådformad ledare, som det flyter likström igenom, fördelas strömmen lika över
tvärsnittet. Om ledaren i stället är ett tunt plan, så blir strömfördelningen annorlunda.
Bilden visar ett plan med två elektroder, som anslutits till en spänningskälla. Utmed
sträckan mellan elektroderna fördelas strömmen över planet så som strömlinjerna på bilden.
Fördelningen beror på elektrodernas utformning och polaritet. Strömtätheten är inte lika
över hela planet, eftersom planet kan ses som många parallellkopplade resistorervars
resistanser ökar med tilltagande strömlinjelängd.

Strömtätheten i planet är större där resistansen mellan elektroderna är liten. Närmast
elektroderna där alla strömlinjer samlas är strömtätheten extremt hög. Där strömtätheten
är som störst finns den största potentialskillnaden (spänningen) per längdenhet
strömlinje. Man kan mäta potentialerna i planet. Spänningen mellan två punkter utmed en
tänkt strömlinje är därvid proportionell med linjens längd mellan punkterna. Halva
spänningen finner man mitt emellan punkterna.

Bild II 1-5 Elektriska kraftfält

Elektriska fält är upplagrad energi. Fältstyrkan kan bli så hög, att det blir en
urladdning mellan polerna. Korona från ändarna av en antenn är ett annat tecken på hög
fältstyrka. För att försvåra urladdning kan man öka elektrodytan, t. ex. göra den
klotformad. Omvänt kan man medverka till urladdning genom att minska elektrodytan.
Ett exempel är åskledarens spets.

Bild II 1-6
I diagrammet U = f (l) visas spänningarna utmed "mittströmslinjen" l genom plus- och
minuspolerna. Kurvutseendet är typisk även för omkring liggande linjer, oavsett längd.

Bilden framställer en ledare som ett idealt plan, medan den i praktiken är en volym.
För att efterlikna en volym föreställer vi oss att bilden roterar omkring
mittströmslinjen, med fältlinjerna oförändrade. Även om resistansen i den rotationskropp
som uppstår är så hög att ingen ström flyter, så är spänningsbilden fortfarande densamma.

Spänningsbilden gäller även för isolerande fasta material, gaser och vakuum.
Det finns alltså spänning mellan olika punkter även i "friska luften". Denna
spänningfältstyrka- kan mätas med särskilda instrument, s. k. fältstyrkemätare.

Av brantheten på spänningskurvan i bilden framgår vilken delspänningen är per dellängd av
en spänningslinje. Kvoten av delspänning och avståndet mellan mätpunkterna kallar man för
elektrisk fältstyrka.

I formler betecknas elektrisk fältstyrka med bokstaven E.
Elektrisk fältstyrka mäts i volt per meter.

\(
\begin{array}{cc}
E=\frac{\Delta U}{\Delta l} & \frac{[volt]}{[meter]}
\end{array}
\)

Bild II 1-6 Elektrisk fältstyrka

\subsection{Skärmning av elektriska fält}

I grunden finns det två slags fält, det elektriska och det magnetiska. Dessutom finns det
även elektromagnetiska fält, som är sammansatt av båda dessa. Fält kan vara statiska
eller dynamiska, varav här avses dynamiska. Ett dynamiskt elektriskt fält genererar ett
magnetiskt fält. Omvänt generar ett dynamiskt magnetiskt fält ett dynamiskt elektriskt
fält. Denna växelverkan gör att fälten kan hållas igång av varandra med tillskott av
yttre energi.

Fält i rörelse alstrar elektromagnetisk strålning, som påverkar omgivningen. När
påverkan inte är önskvärd måste fältet skärmas av. Ett sätt att skärma av ett elektriskt
fält är en metallisk kapsling som anslutits till apparatens jordreferens. Skärmen behöver
inte vara tät, men utförd så att all magnetiskt inducerad ström i den bryts. (Jfr 1.4)

\cleardoublepage

\section{Magnetiskt fält}

\subsection{Magnetism}

Enligt den romerske författaren Plinius lär, vid tiden ungefär 160 år f. K. herden Magnes
en dag ha känt hur järnstiften i sandalerna häftade vid en viss sorts sten. Det kunde ha
varit svart järnmalm, som grekerna i äldsta tider benämnde Lithos herakleia efter staden
Herakleia i Lydien, där sådan malm förekommer. Staden fick sedermera namnet Magnesia och
man kan tänka sig att stenen kom att kallas Magnetes. En hel mineralgrupp med liknande
egenskaper, såsom järn, nickel m. fl. kallas magnetiska.

Magnetism uppstår av elektriska laddningar i rörelse. Elektronernas rörelser i en atom
skapar nämligen magnetfält. Det gör att atomerna var för sig fungerar som en magnetisk
dipol - en magnet. I de flesta material är atomerna orienterade så att deras magnetiska
kraftertar ut varandra. Materialet som helhet är då omagnetiskt och utövar inga yttre
krafter. Men vid påverkan från ett yttre magnetfält kan dipolerna (atomerna) i ett
material orienteras i samma riktning och deras magnetfält kommer då att
samverka. Hela materialet blir då magnetiskt. När det yttre magnetfältet avlägsnas,
kvarstår orienteringen endast delvis- magnetisk remanens. l terrornagnetiska legeringar
kvarstår en större del av orienteringen, även om påverkan från det yttre magnetfältet har
upphört. Materialet är då permanentmagnetiskt

\subsection{Kraftfält i och omkring magneter}

Bild 111-7

Varje magnet omges av ett magnetiskt kraftfält Magnetfältets fördelning, styrka och
riktningar beskrivs som kraftlinjer med slutna kretslopp.

Utanför magneten går kraftlinjerna från nord- till sydpol och inne i magneten i motsatt
riktning. Kraftriktningen i varje punkt av fältet är den som nordändan på en kompassnål
skulle peka åt. Om man hänger upp en magnet i en tråd, så kommer den att inta
samma riktning som jordens magnetfält.

Poler med samma polaritet stöter bort varandra (repellerar).

Poler med olika polaritet dras till varandra (attraherar).

\subsection{Magnetiska fält omkring strömbanor}

Bild II 1-8

Omkring varje ledare, som det flyter en elektrisk ström igenom, alstras det ett
magnetiskt kraftfält.

Magnetiska kraftlinjerna fördelar sig koncentriskt omkring en rak ledare och vinkelrätt
mot denna.

Mellan ändarna av en ledare med bågformad utsträckning bildas kraftlinjer som verkar med
varandra.

En strömgenomfluten cylindrisk spole induktor- uppvisar samma magnetiska fältbild som en stavformad permanentmagnet

\subsection{Bestämma magnetiska fältriktningen}

Magnetfältets riktning omkring en ledare kan enkelt bestämmas med vänsterhandsregeln

När en ledare fattas med vänster hand och med tummen i strömmens riktning, så
kommer fingrarna att peka i fältriktningen.

När en ledare formas som en spole och en elektrisk ström flyter genom den, kommer
magnetfältet att ha ett utseende som liknar det omkring en permanentmagnet

En sådan spole kallas elektromagnet.

Magnetfältets riktning i en spole kan också bestämmas med vänsterhandsregeln.
När en spole fattas med vänster hand och med fingrarna i strömmens riktning, så
kommer den utsträckta tummen att peka mot spolens nordpol.

Fälten omkring alla slags magneter, såväl permanentmagnetiska som e!~ktromag­
netiska, återverkar på varandra. Aven enkla
elektriska ledare är elektromagneter.

Bild II 1-7 Kraftfält omkring magneter

Bild II 1-8 Magnetiska fält omkring strömledare

\subsection{Exempel på elektromagneter}

Bild II 1-9

\subsubsection{Elektromagnet}
Det bildas ett magnetfält genom en spole så länge som det flyter ström genom den. En
järnkärna i spolen koncentrerar fältet p.g.a. den större magnetiska ledningsförmågan.

Elektromagneter används för att sätta magnetiska material i rörelse eller hålla fast
dem.

\subsubsection{Elektrisk ringklocka}
Anordningen består av en elektromagnet och en järnplatta på en fjäder. På plattan
sitter en självbrytande kontakt samt en kläpp som kan slå på en klocka.

Kontakten åstadkommer en växelvis brytning och slutning av strömmen genom
elektromagneten. Armaturen med kläppen kommer då i svängning och slår på klockan.

\subsubsection{Telefon}
I en enkel telefon finns bl.a. en mikrofon, ett batteri och en hörtelefon.

Särskilt i äldre telefoner består mikrofonen av en kolkornskammare med ett membran.
Trycksvariationer (ljud) får membranet att vibrera, varvid resistansen genom kolkornen
varierar i motsvarande grad. Därmed varierar talströmmen genom mikrofonen.

Hörtelefonen består av en elektromagnet och ett membran av mjukjärn. Variationer i
talströmmen genom mikrofonen passerar även hörtelefonen får dess magnetfält att variera.
Hörtelefonens membran alstrar då trycksvariationer, d.v.s. ljud.

\subsubsection{Elektromagnetiskt relä}
Reläet består av en elektromagnet, en järnplatta (ankare) på en fjäder och en elektrisk
kontakt. Med en svag ström/låg spänning genom spolen i manöverkretsen, så kan
man med reläets arbetskontakt styra starkare ström/högre spänning i huvudkretsen.

Bild II 1-9 Tillämpade elektromagneter

\subsection{Magnetisk fältstyrka}

Som magnetisk fältstyrka Henry \([H]\) förstår man flödet per meter fältlinje, d.v.s.

\(H=\frac{\Phi}{l} = \frac{I \cdot N}{l}\)

\(H [A/m]\) \(I [A]\) \(N [varvtal]\) \(l [fältlinjelängd]\)

Magnetisk fältstyrka uttrycks således som Ampere per meter flödesväg.

\subsection{Magnetisk flödestäthet}

Den magnetiska flödestätheten mäts i enheten Tesla \([T]\) (förut Gauss).

Formeltecknet är \(B\).
Formeln är \(B = \mu_0 \cdot HH\)

Flödestäthet \(B [Vs/m^2]\) Fältstyrka \(H [A/m]\)

\(\mu_0\) är permeabilitetstalet (fältkonstanten) för den magnetiska ledningsförmåga för
luft och omagnetiska material.

För järn eller annat magnetiskt ledande material tillkommer permeabilitetstalet \(\mu_r\).
Det anger hur många gånger bättre än luft etc., som materialet det leder ett magnetisk
flöde.

Formeln är \(B = \mu_0 \cdot \mu_r \cdot H\)

\subsection{Magnetiskt flöde}

Det magnetiska flödet är produkten avflödestätheten \(B\) och tvärsnittsytan \(A\) av flödesvägen, således

\(\Phi = B \cdot A\)
\(\Phi [Weber eller Vs]\) \(B [T eller Tesla]\) \(A [m^2]\)

\subsection{Skärmning av magnetiska fält}

I grunden finns det två slags fält, det elektriska och det magnetiska. Det finns även
elektromagnetiska fält, som är sammansatt av båda dessa. Fält kan vara permanenta eller
rörliga, varav här avses de rörliga. Ett rörligt magnetiskt fält genererar ett elektriskt
fält.
Omvänt generar ett rörligt elektriskt fält ett rörligt magnetiskt fält. Denna växelverkan
gör att fälten kan hållas igång med tillförsel av yttre energi.

Fält i rörelse alstrar elektromagnetisk strålning, som påverkarfunktioner i omgivningen.
När påverkan inte är önskvärd, måste fältet skärmas av. Ett sätt att skärma magnetiska
fält är en metallisk kapsling. Kapslingenskall vara tät och bilda en sluten magnetisk
krets. Kapslingen skall vara utförd i ett material som är en god ledare av magnetiskt
flöde.
(Jämför 1.3)

\cleardoublepage

\section{Elektromagnetiskt fält}

\subsection{Vågutbredning}

En tillståndsändring i ett medium innebär att energi tillförs eller tas bort. Om detta
sker växelvis, så uppstår förlopp såsom pendling, svängning, vågbildning etc.
Eftersom naturen söker jämvikt, så brederförloppet ut sig genom mediet efter någon modell.

Energi kan inta olika tillstånd. I en pendel växlar energin mellan lägesenergi och
rörelseenergi. Vågor på en vätskeyta liksom fjädring i fasta material är exempel på
detta. Det kan även innebära trycksvängningar i gaser o.s.v.

I detta avsnitt behandlas elektromagnetiska fält. Sådana uppstår av svängningar i
elektriska och magnetiska fält. För att förklara pendling och utbredning används här
modeller.

\subsection{Utbredningsmodeller}

\subsubsection{Vågutbredning längs en linje}

Bild II i-10

När änden av en tråd sätts i pendling med en frekvens f, så kommer till sist hela tråden i
svängning med den frekvensen. Den pendling, som först skapades, vandrar längs tråden med
utbredningshastigheten v. Våglängden är Å (lambda), som är avståndet mellan två
närliggande punkter med samma svängningsläge och svängningsriktning.

Bild II 1-10 Vågor längs en linje

Bild II 1-11 Vågutbredning på en yta

\subsubsection{Vågutbredning på en yta}

Bild II 1-11

När ett föremål släpps genom en vätskeyta, så bildas vågor som breder ut sig som cirklar
i varandra (koncentriska).

De punkter på vågen, som för ögonblicket har samma svängningsläge, och är lika långt från
energikällan, kallas för vågfront

Sambandet mellan utbredningshastighet \(v\), våglängd \(\lambda\) och frekvens \(f\) är

\(v = \lambda \cdot f\) \(v [m/s]\) \(\lambda [m]\) \(f [Hz=1/s]\)

Bild If 1-12 Vågutbredning i rummet

Exempel: När våglängden \(\lambda = 2 m\) och antalet svängningar per sekund \(f = 10 Hz\),
så breder vågen ut sig med hastigheten \(v = 20 m/s\).

\subsubsection{Vågutbredning i rummet}

Bildll1-12

Ljud är energi i form av tryckvågor i luften. När en mekanisk kropp sätts i svängning
(stämgaffel, dricksglas etc), överförs svängningarna till den omgivande luftmassan som
börjar att svänga med. I luftmassan bildas det omväxlande över- och undertryckszoner, som
breder ut sig åt alla håll. De mekaniska svängningarna i ljudkällan omvandlas alltså till
tryckvågor.

Det mänskliga örat uppfattar tryckvågor inom frekvensområdet c:a \(15-18000 Hz\) som ljud.
Dessa vågor kallas ljudvågor. Utbredningshastigheten för ljudvågor är \(v = c:a 340 m/s\) vid 
15\(\circ\)C och normalt lufttryck.

\subsection{Elektromagnetiska fält}

Bild 111-13

I detta avsnitt görs i huvudsak endast jämförelse mellan ljusvågor och radiovågor, vilka
båda är elektromagnetisk strålning. Hur ett elektromagntiskt fält frigörs från en ledare,
framgår av kapitel 7 Vågutbredning.

Elektromagnetiska fält är energi, som är sammansatt av mycket snabbt svängande elektriska
och magnetiska fält. När elektrisk ström genom en ledare ändras i styrka, så bildas ett
magnetfält omkring ledaren. Detta magnetfält alstrar en elektromotorisk kraft (EMK), som
är motriktad den som driver fram strömmen. Magnetfältet motverkar således strömändringen.
På liknande sätt alstrar en ändring av magnetfältet omkring ledaren en EMK i form av ett
elektriskt fält. Detta driver en motriktad ström och därmed ett motverkande magnetiskt
fält.

Både det elektriska och det magnetiska fältet har således alstrats av ändringar i det
andra och existerar därför bara tillsammans.

De båda fälten kombineras till ett elektromagnetiskt fält, som har egenskapen att kunna
stråla (breda ut sig) i alla tre dimensioner. Beroende på frekvensen har
elektromagnetiska fält olika egenskaper och användning, vilket framgår av bilden.


\begin{center}
\begin{tabular}{|rl|rl|l|}
\hline
\multicolumn{2}{|c|}{\multirow{2}{*}{Frekvens}} & \multicolumn{2}{|c|}{\multirow{2}{*}{Våglängd}} & \multicolumn{1}{|c|}{Egenskaper/} \\
 & & & & \multicolumn{1}{|c|}{användning} \\ \hline
300 & Hz  & 100 & mil & \\
  1 & kHz & 300 & km & ULF \\ \cline{5-5}
  3 & kHz & 100 & km & \\
 10 & kHz &  30 & km & VLF \\ \cline{5-5}
 30 & kHz &  10 & km & \\
100 & kHz &   3 & km & LF \\ \cline{5-5}
300 & kHz &   1 & km & \\
  1 & MHz & 300 & m & MF \\ \cline{5-5}
  3 & MHz & 100 & m & \\
 10 & MHz &  30 & m & HF \\ \cline{5-5}
 30 & MHz &  10 & m & \\
100 & MHz &   3 & m & VHF \\ \cline{5-5}
300 & MHz &   1 & m & \\
  1 & GHz & 300 & mm & UHF \\ \cline{5-5}
  3 & GHz & 100 & mm & \\
 10 & GHz &  30 & mm & SHF \\ \cline{5-5}
 30 & GHz &  10 & mm & \\
100 & GHz &   3 & mm & EHF\\ \cline{5-5}
300 & GHz &   1 & mm & \\\
  1 & THz & 300 & \(\mu\)m & Infrarött \\
  3 & Thz & 100 & \(\mu\)m & ljus \\
 10 & THz &  30 & \(\mu\)m & (värme- \\
 30 & THz &  10 & \(\mu\)m & strålning) \\
100 & THz &   3 & \(\mu\)m & \\ \cline{5-5}
300 & THz &   1 & \(\mu\)m & Synligt ljus \\ \cline{5-5}
  1 & PHz & 300 & nm & \\
  3 & PHz & 100 & nm & Ultraviolett \\
 10 & PHz &  30 & nm & ljus \\ \cline{5-5}
 30 & PHz &  10 & nm & \\
100 & PHz &   3 & nm & Rönt-\\
300 & PHz &   1 & nm & gen-\\
  1 & EHz & 300 & pm & strålning\\ \cline{5-5}
  3 & EHz & 100 & pm & \\
 10 & EHz &  30 & pm & Gamma-\\
 30 & EHz &  10 & pm & strål-\\
100 & EHz &   3 & pm & ning\\
300 & EHz &   1 & pm & \\
\hline
\end{tabular}
\end{center}

Bild II 1-13 Elektromagnetiskt spektrum

\subsubsection{Ljusvågor}

Ögat uppfattar elektromagnetisk strålning bara inom ett visst frekvensområde som ljus.
Ljusets utbredningshastig het beror av vilket material, som det passerar igenom. I
vakuum är hastigheten störst, c= 299793077 m/s (= ca \(3 \cdot 10^8 m/s\)). I tätare ämnen är
hastigheten lägre, t. ex. i glas ca 200 000 000 m/s. Det för människan synliga ljuset har
våglängder mellan \(7.7 \cdot 10^{-7}\) och \(3.9 \cdot 10^{-7} m\), motsvarande 7.7 till 3.9
tiotusendels mm.

Sambandet mellan ljusets utbredningshastighet c i vakuum, frekvensen f och våglängden A är

\(c = \lambda \cdot f = 3 \cdot 10^3\)

\(c [m/s]\) \(f [Hz  i/s]\) \(A [m]\)

\subsubsection{Radiovågor}

Även radiovågor är elektromagnetisk strålning, men inom ett lägre frekvensområde än
det för ljus. Men utbredningshastigheten för radiovågor genom olika material följer ändå
samma lagar som de för t. ex. ljusets utbredning.

Radiovågor anses omfatta ett frekvensområde från ca 1O kHz (\(\lambda = 30 km\)) till 300
GHz (\(\lambda = 1 mm\)).

Rundradio tilldelas frekvenser i intervallet 100 kHz till 1 000 MHz. Amatörradio tilldelas
ett antal frekvensområden i intervallet 1.8 MHz till 250 GHz.

Att märka är att elektromagnetiska fält, som sagts ovan, förekommer så långt ner i
frekvens som ett fåtal kHz. Detta skall självklart inte förväxlas med ljudtryck med samma
frekvens.

\subsubsection{Egenskaper hos elektromagnetiska vågor}

Elektromagnetiska vågor med högre frekvens än radiovågor uppfattas som värmestrålning,
vågor med ännu högre frekven som ljus etc., men fortfarande är huvudegenskaperna samma.
Som exempel kan nämnas polariserade vågor. Dessutom kan
man finna motsvarigheten till sådana egenskaper som interferens, överlagring, även i
andra vågtyper, t.ex. i ljud.

\subsection{Vågpolarisation}

Bild II 1-14

\subsubsection{Vågor längs en linje (tråd el. dyl.)}
En vågrörelse i ett plan kallas linjärt polariserad. Om änden på en horisontell tråd sätts
i rörelse uppåt-nedåt, uppstår på tråden en linjärt polariserad vågrörelse i vertikalplanet
-vertikal polarisering.
Om tråden sätts i rörelse höger-vänster kommer dess svängning att vara horisontellt
polariserad.
Om tråden sätts i svängning i ett plan och detta plan ständigt vrider sig, kommer även
vågrörelsen utmed tråden att vrida sig. En vågrörelse, vars polarisering vrider sig
roterar - kallas för cirkulärt polariserad. Vridning mot- respektive medurs kallas för
vänster- respektive högervriden polarisering.

Bild II 1-14 Polarisation av elektromagnetiska vågor

\subsubsection{Elektromagnetiska vågor}

De magnetiska och elektriska fälten omkring en ledare är vinkelrätt orienterade mot
varandra. Det elektromagnetiska fält, som de bildar tillsammans, bildar en vågfront som är
vinkelrätt orienterad mot dem.

Polariseringsriktningen för en elektromagnetisk våg definieras som den riktning dess
elektriska fält har.
Vertikalt elektriskt fält- vertikal polarisering.
Horisontellt elektriskt fält - horisontell polarisering.

\subsubsection{Ljusvågor}

Ljus är elektromagnetiska vågor. När dagsljus, som f.ö. är opolariserat, belyser ett
polariseringsfilter, så passerar endast de vågkomposanter genom filtret, som har samma
polarisering som filtret.

När det polariserade ljuset därefter sänds mot ett efterföljande filter, så passerar ljuset
genom det filtret endast när det har samma polarisering som ljuset. När de båda filtren är
vridna 90\(\circ\) i förhållande till varandra, passerar inget ljus alls.

\subsubsection{Radiovågor}
Radiovågor är elektromagnetiska vågor inom
det frekvensområde som lämpar sig för radiokommunikation.

Beroende på sändarantennens utformning så avger den vågor med en polarisation. På samma
sätt är en mottagarantenn mest mottaglig för vågor med en viss polarisation.
Överföringsförlusterna blir lägst mellan antenner med samma polarisation.

I det högre frekvensområdet för radio (VHF, UHF, SHF) är polariseringsvridning under
överföringen mindre vanlig. Genom att utforma antennerna med horisontell, vertikal eller
cirkulär (höger- alternativt vänstervriden) polarisation, så fås överföringsegenskaper för
olika syften.

Cirkulärt polariserade antenner ger lägst överföringsförluster när polariseringsriktningen
är lika i sändar-och mottagarantennen.

I det lägre frekvensområdet för radio (HF och lägre) utnyttjas oftast rymdvågsutbredning.
Eftersom de utsända vågorna då reflekteras mot jonosfärskikt, uppstår
polariseringsvridningar som inte kan förutses. Då är det en fördel att kunna växla mellan
antenner med olika polarisation.

\subsection{Väg interferens}

Bild II 1-15

När vågor från olika energikällor blandas med varandra (överlagras), så kommer de att
antingen samverka eller motverka. Beroende av det tidsmässiga läget mellan vågorna och
deras amplituder, så blir resultatet en förstärkning eller en försvagning. Om har samma
frekvens och lika stora, motriktade amplituder, så uppstår en utsläckning, vilket kallas
fädning (eng. fading).

Denna vågmekanism är liknande i gaser (luft), vätskor, elektromagnetiska fält etc. Ett
försök kan göras med en stämgaffel som man slår an och håller intill örat. När man
vrider stämgaffeln runt sin längdaxel, så kommer avståndet mellan vart och ett av
gaffelbenen och örat att variera. Då uppstår en växelvis med- och motverkan mellan tonerna
från gaffelbenen och därmed varierande tonstyrka.

Detta fenomen utnyttjas bl.a. i antenner för riktad sändning respektive mottagning av
radiovågor.

Bild II 1-15 Våginterferens

\cleardoublepage

\section{Sinusformade signaler}

Bild II 1-16 Alstring av en sinusformad signal

I detta avsnitt behandlas några grundbegrepp inom växelströms/äran. Förloppen
framställs med vektor- och linjediagram.

För närmare beskrivning används sådana begrepp som momentanvärde, toppvärde, topp- till
toppvärde, effektiwärde, fasläge, fasförskjutning och båghastighet.

\subsection{Momentanvärde}

Momentanvärdet är storheten på en spänning \(u\), en ström i etc. vid en viss tidpunkt \(t\).
(Storheter som ändrar sig som en funktion av tiden kännetecknas ofta med gemena bokstäver.)

Bilden visar en sinusformad växelspänning med frekvensen \(50 Hz\). Spänningen \(u\) är \(+230 V\) vid tidpunkten 2.5 millisekunder efter en positiv nollgenomgång. Efter totalt \(5 ms\)
uppnås toppvärdet \(u\) d.v.s. \(+325 V\). Efter totalt \(1O ms\) sker en neagativ
nollgenomgång. Efter totalt \(12.5 ms\) är spänningen \(-u\), d.v.s. \(-230 V\) o.s.v.

\subsection{Toppvärde eller amplitud}

Toppvärdet u är det högsta värdet över eller under noll. På bilden är de högsta
värdena \(+325 V\) och \(-325 V\).

\subsection{Topp-till-toppvärde}

Topp-till-toppvärdetuss är summan av toppvärdena över och under noll. På bilden är
detta värde 650 V.

\subsection{Effektivvärde}

Effektivvärdet av en växelspänning \(u\) är det värde, som medför samma effektutveckling
som en likspänning \(U\).

För ett sinusformat förlopp gäller följande samband mellan toppvärdet och effektivvärdet
(det s.k. kvadratiska medelvärdet), vilket motsvarar amplituden vid vinklarna 45,
135, 225 och 270\(\circ\).

\(U=\frac{\hat{u}}{\sqrt{2}}\) \(I=\frac{\hat{i}}{\sqrt{2}}\) (\(\sqrt{2} = 1.414\))

\subsection{Fasläge}

Fasläget är när inom en period, som ett givet momentanvärde uppträder. Tidpunkten för
varje momentanvärde motsvarar en andel av 360\(\circ\) elektriska grader. T.ex. uppnås
värdet volt vid 0\(\circ\), 180\(\circ\) och 360\(\circ\) (= 0\(\circ\)).

\subsection{Bågmått}

I beräkningar av växelströmskretsar används ofta inte vinkelmått för fasläget (gradtal)
utan i stället begreppet bågmått.

I en s.k. enhetskrets med radien \(r = 1\) motsvaras vinkeln 360\(\circ\) av en båge med
längden \(2 \cdot \pi \cdot r= 2 \cdot \pi \cdot 1 = 2 \pi =\) omkretsen
Vid \(f\) perioder per sekund blir båglängden \(= 2\pi f\). Denna storhet kallas båghastigheten
och betecknas med ro (uttalas omega).
\(\omega= 2\pi f\) \([1/s]\)

\subsection{Period}

En period har passerat, när en storhet (spänning, ström o.s.v.) återtagit samma tillstånd
eller värde efter att ha gjort en fullständig växling, t.ex. en hel pendelrörelse eller ett
helt varv vid rotation.

\subsection{Periodtid T}

Periotid T är den tid som åtgår för att strömmen ellerspänningen skall genomlöpa
en period. Periodtiden är det inverterade värdet av frekvensen.

Måttenheten för periodtid är sekund [sj

Periodtid

\((T) = \frac{1}{f}\)

T [s] f[Hz] eller
T [ms] f [kHz] eller
T [ms] f [MHz]

Exempel:

\begin{center}
\begin{tabular}{lll}
\(T_1=\frac{1}{10}\) s & = 0.100 s & = 100 ms (f = 10 Hz)\\
\(T_2=\frac{1}{50}\) s & = 0.020 s & = 20 ms (f = 50 Hz)\\
\(T_3=\frac{1}{1000}\) s & = 0.001 s & = 1 ms (f = 1 kHz)\\
\(T_4=\frac{1}{1000000}\) s & = 0.000001 s & = 1 \(\mu\)s (f = 1 MHz)\\
\end{tabular}
\end{center}

\subsection{Frekvens}

Frekvens är antalet perioder per tidsenhet.

Följande begrepp demonstreras med hjälp av pendeln:

Period = en fullständig fram- och tillbakasvängning i ett system, t.ex. pendelns väg
mellan punkterna 2- 3- 2- 1 - 2- 3- o.s.v.

Periodtid T = tidsåtgången för en fullständig svängning.

Amplitud A = den största avvikelsen från viloläget.

Frekvens f = antal svängningar/tidsenhet.

Sambandet mellan frekvensen f och periodtiden T är

\(f=\frac{1}{T}\) t. ex.

\(5 [H z] = \frac{1}{5} [sekunder]\)

\subsection{Enheten Hertz}

Måttenheten för frekvens är Hertz [Hz].
l formler betecknas frekvensen med f.

\begin{center}
\begin{tabular}{ll}
1 Hz      & = 1 period per sekund (p/s) \\
1O Hz     & = 1O perioder per sekund \\
50 Hz     & = 50 perioder per sekund \\
1 000 Hz  & = \(10^3\) Hz = 1 kHz (kilohertz) \\
1 000 kHz & = \(10^6\) Hz = 1 MHz (megahertz) \\
1 000 MHz & = \(10^9\) Hz = 1 GHz (gigahertz) \\
\end{tabular}
\end{center}

Nätfrekvensen för elkraft är i Europa 50 Hz.

Andra nätfrekvenser förekommer, t.ex. 60Hz i USA.

Frekvensområdet vid överföring av kvalitativt tal och musik, lågfrekvens LF, är mellan ca 16Hz och 16kHz.

Frekvensområdet för talöverföring, t.ex. över telefonlinjer eller kommunikationsradio,
är c:a 300 till 3000 Hz.

Frekvensområdet för radioöverföring, högfrekvens HF, är i huvudsak mellan 50 kHz, s.k.
långvåg, och 100-tals GHz, s.k. mikrovåg.

\subsection{Fasförskjutning}

Bild II 1-17

Med fasförskjutning menas tidsskillnaden mellan förlopp, t.ex. spänningar och/eller
strömmar. Fasförskjutningen mellan vektorerna kallas även fasvinkel och uttrycks
som ett gradtal mellan O och 360\(\circ\).

\subsection{Vektorer}

En spänning, ström, kraft o.s.v. kan beskrivas som en vektor med en storhet och riktning.
På bilden har vektorerna \(X_L\), \(R\) och \(X_C\) en inbördes fasförskjutning av 90\(\circ\).
De motsvarar spänningsfallen i en krets med en induktor, en resister och en kondensator
kopplade i serie.

Antag att vektorerna roterar i ett oförändrat inbördes läge och med en vinkelhastighet
av \(\omega= 2\pi f\). Systemet roterar då \(360\circ = 2\pi radianer = 1 varv/period\).

Vid varje tidpunkt har vektorsystemet uppnått en viss vridningsvinkeL Momentanvärdet på
vektorernas spänningar avsätts till höger i bilden. Avståndet mellan en vektorspets och
noll-linjen är vektorns momentana värde, som kan vara positivt eller negativt.

Bild II 1-17 Vektorer och fasförskjutning

\cleardoublepage

\section{Icke sinusformade signaler}

\subsection{Grundton, övertoner- Kantvågor}

Bild II 1-18

Ett sinusformat förlopp med en enda frekvens- en enda ton- sägs vara spektralt ren.
En sådan svängning kallas för grundton.

Varje signal, som inte är sinusformad, är sammansatt av flera sinussvängningar. Det är
signalens grundton samt dess harmoniska övertoner, vilka kan ha 2, 3 o.s. v. gånger högre
frekvens än grundtonen. Den inbördes styrkan på grundton och övertoner avgör signalens
form. Om signalen ligger inom det hörbara området, kan man märka hur den ändrar karaktär
beroende på övertonshalten. Man kan säga att övertonerna modulerar grundtonen.

Bild II 1-19

Oscillatorsignalen i exemplet på bilden har 1 volts amplitud på grundtonen f0 (1:a
harmoniska), 0.7 volts amplitud på de n 1 :a övertonen (2:a harmoniska) och 0.2 volts
amplitud på den 2:a övertonen (3:e harmoniska). Den totala amplituden blir emellertid inte
summan av 1, 0.7 och 0.2 volt eftersom de olika delspänningarnas toppvärden inte uppträder
samtidigt. I stället måste delspänningarna adderas vid varje tidpunkt för sig.

Bild II 1-20

Det finns olika karaktärer av förlopp såsom sinusvåg, triangelvåg, sågtandsvåg,
fyrkantvåg o.s.v.

Fyrkantvågen är sammansatt av sinusvågor med grundfrekvensen och dess udda övertoner,
varvid amplituderna fördelar sig som 1/1, 1/3, 1/5, 1/7, 1/9, 1/11 o.s.v. Teoretiskt når
övertonsspektrum upp till oändligt höga frekvenser, medan de motsvarande amplituderna
minskar till oändligt små värden.

En ideal fyrkantvåg, vilken inte kan uppnås i praktiken, skulle bestå av ett oändligt
antal udda övertoner med fallande amplitud. Ju fler av de högre övertonerna som filtreras
bort, desto mer lutar fyrkantvågens flanker, desto rundare blir hörnen på vågen och
desto vågigare blir kurvans topp.

Bild II 1-18 Ren sinusvåg och övertonshaltig våg

Bild II 1-19 Uppdelning av en signal i grundton och övertoner

Bild II 1-20 Uppdelning av en fyrkantvåg i grundton och övertoner

\subsection{Överlagrade spänningar
(likspänningskomposant)}

Bild ll 1-21

I signalkretsar förekommer det mycket ofta, att växelspänning överlagras på likspänning
eller omvänt. Likspänningen kallas då för likspänningskomposant Olika åtgärder behövs för
att överlagra spänningar på varandra och att sedan skilja dem åt.

Bilden visar ett avsnitt av en AM-mottagare. Från vänster hämtas en AM-modulerad signal
från MF-förstärkaren för att demoduleras, d.v.s. för att återvinna den modulerande
LF-signalen. MF-signalen halvvågslikriktas. Kvar blir den positiva delen av MF-signalen
och den modulerande LF-signalen, sammanlagrade. LF-signalen skall nu skiljas ut och
förstärkas. Alltså filtreras MF-komposanten bort. Kvar blir LF-signalen, men överlagrad på
en likspänning. Likspänningen stoppas och kvar blir slutligen LF-signaien som förstärks.

Bild II 1-21 Överlagrade spänningar

\cleardoublepage

\section{Modulation}

\subsection{Allmänt}

Modulera (lat. modulari, rytmiskt avmäta) är att med hjälp av en oftast högfrekvent
elektrisk signal (bärvågen) överföra informationen i en lågfrekvent signal. På så sätt kan
lågfrekvens, t.ex. tal och musik, först omvandlas till en elektrisk signal, som får 
påverka (modulera) en högfrekvent elektrisk signal. Denna modulerade signal strålas ut från
antennen som ett elektromagnetiskt fält.

Den signal som innehåller informationen kallas modulerande signal eller basband eller
underbärvåg.

Den signal som informationen överförts till kallas modulerad signaleller huvudbärvåg.

\subsection{Modulationssystem}

Den största gruppen av modulationssystem är definierad med avseende på hur huvudbäNågen är
modulerad. Vanligast är då amplitud- och vinkel modulation. Av vinkelmodulation finns
främst två slag, frekvensmodulation och fasmodulation. Därutöver finns system för
pulsmodulation

\subsection{Sändningsslag}

Sätten att modulera kallas sändningsslag. Gemensamt för sändningsslagen är att en
givare-det kan vara en mikrofon, en telegrafnyckel, en fjärrskriftmaskin, en dator, en
TV-kamera o.s.v.- alstrar en analog eller digital signal. Denna styr underbärvågen så att
huvudbärvågen moduleras med den avsedda informationen och sänds ut.

Det enklaste sändningsslaget får anses vara morsetelegrafi med "nycklad bärvåg".
Då förekommer bara två tillstånd, nedtryckt och icke nedtryckt telegrafnyckel, d.v.s.
antingen bärvåg med någon varaktighet eller ingen bärvåg alls. Kombinationer av
bärvågselement med olika längd motsvarar skrivtecken.

För att återge tal, musik etc. behövs en noggrannare tillståndsstyrning av bärvågen.
Det innebär att bärvågen måste moduleras av en underbärvåg och att denna motsvarar
lufttrycksvariationerna i ljudet.

\subsection{Kännetecken för modulerade signaler}

Bild 111-22

En modulerad signal kännetecknas av dess amplitud, frekvens och fasläge.

Vid amplitudmodulation påverkas huvudbärvågens amplitud, så att den i varje tidpunkt
motsvarar den modulerande signalens variation.

Vid frekvensmodulation påverkas huvudbärvågens frekvens, så att den i varje tidpunkt
motsvarar den modulerande signalens variation.

Vid fasmodulation, som är besläktad med frekensmodulation, påverkas i ställettörfrekvensen
huvudbärvågens fasläge i förhållande till en referenssignal, så att fasläget i varje
tidpunkt motsvarar den modulerande signalens variation.

Frekvens- och fasmodulation liknar varandra och kan sammanfattas som vinkelmodulation,
eftersom fasvinkeln mellan bärvågens spänning och ström varierar i båda fallen.

Vid pulsmodulation används pulståg (korta upprepade bärvågspaket); t. ex. pulsamplitud-,
pulslängds-, pulsläges- och pulskodmodulation. Pulskodmodulation används t.ex. vid
samtidig överföring av flera telesamtal på samma linje, bärvåg etc.

\subsection{Bandbredd vid olika sändningsslag}

Varje radiosändning tar upp plats omkring den nominella bärvågsfrekvensen- tillsammans
bandbredden.

Radioamatören måste veta detta "platsbehov", främst för att inte sända utanför de
frekvensband som är tilldelade för amatörradioanvändning, men även för att kunna
umgås med annan trafik inom banden.

I alla sändningsslag ökar den använda bandbredden med ökad modulation. Eftersom största
frekvenseffektivitet alltid skall eftersträvas så upptar en sändare med kraftigare
modulation än vad som behövs för en överföring, alltid onödigt frekvensutrymme.

Bild II 1-22 Modulerade signaler

\subsection{Beskrivningskod för sändningsslagen}

Vid 1979 års radioförvaltningskonferens (WARC 79) i Geneve reviderades det internationella
radioreglementet (RR), som i huvudsak trädde i kraft 1982. Däri ingår bl. a. ett nytt
system för klassindelning och beteckning av sätten att utsända information över
radio m.m. Reglementet har reviderats senare, men i detta stycke gäller det ännu.

Indelningen i sändningsslag behövs för att känneteckna utsändningarna, t. ex. i
frekvenslistor, författningar och föreskrifter. Indelningen är också av stort värde vid
teknisk beskrivning av apparater och system för radiokommunikation.

Emellertid används av många även äldre benämningar, vilka lever kvar i litteraturen, i
märkning av manöverdonen på sändare och mottagare o.s.v ..

Dessa äldre benämningar är dock inte entydiga och skapar lätt missförstånd, varför
beskrivningskoden enligt WARC 79 bör användas för tydlighetens skull.

Här följer avkortade koder enligt WARC 79 för några av de sändningsslag, som amatörer
använder mest, samt för jämförelse även de benämningar som fortfarande används jämsides
(se vidare i Appendix E).

\begin{description}
\item[NON] Bärvåg utan modulerande signal. Ingen information.

\item[A1A] Bärvåg med dubbla sid band. En enda kanal med kvantiserad bärvåg. Ingen
modulerande underbärvåg. Telegrafi. Även kallat nycklad bärvåg (CW).

\item[A3E] Linjärt modulerad huvudbärvåg. Dubbla sidband. En enda kanal med
analog information. Telefoni.

Även kallat amplitudmodulation (AM).

\item[J3E] Linjärt modulerad huvudbärvåg. Ett sidband med undertryckt bärvåg. En
enda kanal med analog information. Telefoni.

Även kallat enkelt sidband (Single Side Band-SSB).

\item[F3E] Vinkelmodulerad bärvåg. Frekvensmodulering. En enda kanal med analog
information. Telefoni.

Även kallat frekvensmodulering (FM)

\item[G3E] Vinkelmodulerad bärvåg. Fasmodulering. En enda kanal med analog information.
Telefoni.

Även kallat fasmodulering (PM)
\end{description}

Såväl A1A, A3E som J3E är sändningsslag där amplituden moduleras. Därför är
termen amplitudmodulation inte tillräcklig för att beskriva flera likartade sändningsslag.

\subsection{Modulerande signaler}

\subsubsection{Basband}

Basband är ett frekvensområde för en modulerande signal. Det finns ett basband för
alla slags modulerande signaler, vare sig de är analoga eller digitala. Det kan finnas mer
än ett basband i en komplett modulationsprocess. Till exempel är en nycklad ton, som
går till sändaren genom mikrofoningången, dess analoga basband medan nycklingspulserna
till tongeneratorn är dess digitala basband.

Bild 111-23

Ett vanligt sätt att överföra information över radio är med telefoni, d.v.s. tal.

Frekvensområdet 300-3000 Hz räcker för god förståelighet av tal. Dels är örat känsligast
inom det området och dels finns där den mesta energin i talet.

Mikrofonen tar upp de lufttrycksvariationer, som uppstår när man talar, och omvandlar dem
till elektriska svängningar. Svängningarna varierar mellan positiva och negativa
spänningsvärden.

\subsubsection{Försök}

\begin{enumerate}
\item Anslut en mikrofon till ett oscilloskop och studera spänningsförloppen för olika slags
ljud, toner, tal o.s.v. som funktion av tiden. På bilden är dessa svängningar mycket
förenklade, t.ex. sinusformade.

\item Anslut en högtalare och ett oscilloskop till en LF-generator, vars frekvens och amplitud
kan ändras. Lyssna på ljud med låg och hög frekvens samt på svaga och starka ljud. En
baston har låg frekvens och en diskantton har hög frekvens. En svag ton har liten
amplitud och en stark ton har stor amplitud.
\end{enumerate}

\subsection{Sändningsslaget A3E (även kallat AM)}

Bild 111-24

Bilden visar frekvensspektrum av en signal vid amplitudmodulation med

\begin{enumerate}[label=\alph*.,noitemsep]
\item en sinuston,
\item en blandning av tre sinustoner,
\item ett frekvensspektrum.
\end{enumerate}

\subsubsection{Försök}

Modulera en A3E-sändare med en 3 kHzsignal. Med en mottagare utrustad med ett
smalt filter för telegrafi, kan man urskilja och påvisa bärvågen och de båda sidbanden.

\subsubsection{A3E-modulation med en ton}

Bild 111-25

En omodulerad bärvåg har konstant amplitud. En amplitudmodulerad signal är i grunden
resultatet av svävning mellan frekvenser eller av icke linjär blandning av frekvenser. När 
bärvåg och basband blandas, så är särskilt tre blandningsprodukter av intresse.

Dessa är

\begin{enumerate}[label=-,noitemsep]
\item bärvågen,
\item det lägre sidbandet (förkortat LSB) och
\item det övre sidbandet (förkortat USB).
\end{enumerate}

AM-signalen består således inte bara av bärvågsfrekvensen fHF utan även av övre
och nedre sidofrekvenser, vilka är summan och skillnaden av bärvågsfrekvensen \(f_{HF}\) och
den modulerande frekvensen \(f_{LF}\). Alltså \(f_{HF} + f_{LF}\) (övre sidfrekvens) och
skillnadsfrekvensen \(f_{HF} - f_{LF}\) (undre sidfrekvens).

Bild II 1-23 Modulerande signaler

Bild II 1-24 Sidband vid A3E-modulation

Eftersom tal inte bara omfattar en enda frekvens utan ett helt frekvensspektrum (c:a
0.3 - 3 kHz), så uppstår inte bara två sidfrekvenser utan två sidband, det lägre sidbandet
(LSB, Lower Side Band) och det övre (USB, Upper Side Band).

LF-signalens frekvens bestämmer sidfrekvensens avstånd från bärvågen. Bandbredden på en
amplitudmodulerad signal med full bärvåg och två sidband är dubbelt så stor som den högsta
modulerande LFfrekvensen:

\(b= 2 \cdot f_{LFmax}\)

Om de modulerande LF-frekvenserna är mellan 0.3 och 3 kHz, så blir sändningens
totala bandbredd 6 kHz.

LF-signalernas amplitud påverkar sidbandens och sidfrekvensernas amplitud. Vid
maximal modulation (100 \% modulationsgrad) varierar signalamplituden mellan noll
och dubbla värdet av det för en omodulerad bärvåg.

Som mest kan vardera sidbandet överföra en fjärdedel så mycket effekt som bärvågen, d.v.s.
en sjättedel av den totalt utsända effekten. Då avger sändaren dubbelt så stor medeleffekt
som utan modulation. Toppeffekten (PEP, Peak Envelope Power) är till och med fyra gånger
så stor.

slutförstärkaren och kraftförsörjningen måste dimensioneras för toppeffekten vid
full modulation eller att modulationsgraden anpassas så att överbelastning inte sker.

Bild II 1-25 A3E-modulation med toner med olika styrka och frekvens

\subsubsection{Fördelar med A3E-modulation}

En A3E-sändare är enkel jämfört med en J3E-sändare, vilken har en mer komplicerad
signalbehandling.

\subsubsection{Nackdelar med A3E-modulation}

Eftersom samma information finns i båda sidbanden och ingen finns i bärvågen, så sänds
effekten i bärvågen och ett av sidbanden ut till ingen nytta. I talpauser sänds endast
bärvågseffekten och till ingen nytta. Aven frekvensutrymme slösas bort. Då en annan,
alltför närliggande sändares bärvåg blandas med den egna, så alstras renstoner i
mottagarna.

\subsection{Sändningsslaget A1A (även kallat CW)}

Bild 111-26

Man kan överföra meddelanden med morsetelegrafi på olika sätt. Det enklaste sättet är att
koppla in och ur sändarens bärvåg i takt med teckendelarna i morsetecknen. Man kan kalla
det för bärvågstelegrafi. Förfarandet kallas sedan mycket länge även för CW (continous
waves), vilket egentligen anger att bärvågen svänger med konstant amplitud, om man bortser
från att den nycklas. Detta i motsats till de dämpade bärvågssvängningar som var fallet i
sedan mycket länge förbjudna s.k. gnistsändare.

Fastän en sändare "moduleras utan ton", har den en viss bandbredd. Det beror på att den
takt, som sändaren nycklas med, egentligen är en ton - låt vara med låg frekvens. Antag
att sändaren nycklas med en serie korta morsetecken. Vid telegraferingshastigheten
60 tecken/minut alstrar bärvågspulserna en kantvåg med frekvensen 5 Hz. Som tidigare
beskrivits, består en sådan kantvåg av summan av sinussignaler med frekvenserna 5 Hz,
15 Hz, 25 Hz, 35 Hz o.s.v.

Det innebär att det uppstår sidfrekvenser över och under bärvågens frekvens och med
ett avstånd till bärvågen av 5 Hz, 15 Hz, 25, 35Hz o.s.v .. Telegrafisändaren har alltså
liksom vid A3E en bandbredd, som dels står i förhållande till nycklingshastigheten och
dels till "kantigheten" på tecknen, vilket bestämmer övertonshalten i bärvågen. Vid s.k.
mjuk nyekling kan den 9:e övertonen antas vara den högsta som uppfattas av en motstation.
Med en nycklingsfrekvens av 5 Hz blir bandbredden inte större än
\(2 \cdot 10 \cdot 5 = 100Hz\).

En hård (kantig) och snabb teckengivning ökar bandbredden och kan resultera i att s.k.
nycklingsknäppar kan uppfattas långt vid sidan om sändningsfrekvensen. Ju hårdare
nycklingen är, desto längre bort från bärvågsfrekvensen hörs nycklingsknäpparna. Detta
stör andra stationer.

Kännetecken för sändningsslaget A1A, telegrafi genom nycklad bärvåg:

Mycket liten bandbredd, extremt gott utnyttjande av såndareffekten, stor
överföringssäkerhet, lång räckvidd, enkla sändare.

Bild II 1-26 Amplitudmodulation med morsetecken

\subsection{Sändningsslaget J3E (även kallat SSB)}

\subsubsection{Princip}

Som sagts är det onödigt sända ut två sidband, eftersom båda innehåller samma information.

Signaler med endast ett sidband och undertryckt bärvåg kan alstras på flera sätt.
Numera är den s.k. filtermetoden i särklass vanligast och den enda som behandlas här.

Bild II i-27

Med filtermetoden blandas HF- och LFsignalerna i en speciell blandare. Där undertrycks
båda dessa signaler medan blandningsprodukterna med deras summa- och skillnadsfrekvenser
blir kvar, d.v.s. det övre och nedre sidbandet.

Utsignalen från blandaren benämns DSBsignal (Double Side Band). Till skillnad från
i A3E-signalen saknas dock bärvågen i DSBsignalen. För att även undertrycka det ena
sidbandet före sändningen, så följs blandaren av ett bandpassfilter med bandbredd
och frekvensläge för avsett sidband.

Den signal som sänds ut innehåller därför endast ett sidband (Single Side Band).

\paragraph{Exempel}

Bild II 1-28

Ett SSB-filter har ett passband av 9000.39003 kHz. Vid bärvågsfrekvensen 9000kHz
sträcker sig det övre sidbandet från 9003.39003 kHz och släpps igenom. Däremot blir
bärvågsfrekvensen undertryckt.

Det undre sidbandet 8997-8999.7 kHz faller utanför filtrets passband och blir också
undertryckt.

Skall däremot det undre sidbandet kunna passera igenom samma filter, så måste
bärvågsfrekvensen höjas med 3 kHz, alltså till 9003 kHz. Då faller det undre sidbandet,
9002.7-9000.0 kHz inom filtrets passband.

Det övre sidbandet 9003.3-9006.0 kHz faller nu utanför passbandet och blir undertryckt.

Bild II i -29

LF-signalens amplitud bestämmer amplituden på sidfrekvensen.

LF-signalens frekvens bestämmer sidfrekvensens avstånd från bärvågsfrekvensen (bärvågen
undertryckt).

Bandbredden på den utsända signalen är skillnaden mellan högsta och lägsta
modulerande frekvens i signalen:

t.ex. \(b = 3kHz - 0.3 kHz = 2.7 kHz\)

\subsubsection{Fördelar med J3E-modulation}
Bra verkningsgrad vid J3E-modulation jämfört med vid A3E-modulation (traditionell AM).
Effekten i det utsända sidbandet motsvarar den i ett av sidbanden vid A3E. Hela den
utsända effekten finns alltså i ett enda sidband, som överför hela informationen.

I sändningspauserna sänds ingen effekt ut. Bandbredden är mindre än hälften av den
vid A3E. Vid mottagning av en J3E-sändning (SSB) är det mindre besvär med interferenstoner
från J3E-sändningar på närliggande frekvenser, eftersom ingen bärvåg och endast ett
sidband sänds ut.

\subsubsection{Nackdelar med J3E-modulation}
J3E-modulation medför mera komplicerade apparater, både för mottagning och sändning.
En J3E-signal blir förvrängd och hörs i fel tonläge, om mottagaren inte är inställd på
exakt rätt frekvens.

Bild II 1-27 Sidband vid DSB

Bild II 1-28 Sidbandsval vid SSB

Bild II 1-29 Sidbandlägen vid SSB

\subsection{Vinkelmodulation}
Termen vinkelmodulation är samlingsnamnet för frekvensmodulation (FM) och fasmodulation
(PM). Ofta sägs utrustningar vara för frekvensmodulation, när de antingen är för frekvens-
eller fasmodulation. Det finns alltså skillnader och likheter mellan dessa system, vilka
emellertid inte är oberoende av varandra, eftersom frekvensen i en signal inte kan
varieras utan att fasen också varieras, och vice versa.

Hur effektiv kommunikationen då är, beror mest på mottagningsmetoderna. I båda fallen
uppfattas ändringar i den mottagna signalens frekvens och fasläge. Amplitudändringar
uppfattas däremot inte. De flesta störningar - särskilt pulserande sådana som från
tändningssystem - kommer att därför att skiljas bort.

För att effektivt utnyttja fördelarna med vinkelmodulation, antingen det är frekvenseller
fasmodulation, behövs tillräckligt frekvensutrymme. Det innebär att främst högre
frekvensband kommer i fråga.

\subsection{Frekvensmodulation (även kallat FM)}

Bild II 1-30 (överst och i mitten)

Vid frekvensmodulation varierar bärvågens frekvens i takt med den modulerande signalens
amplitud och polaritet. På bilden ökar bärvågens frekvens när den modulerande signalen är
positiv (första halvperioden) och minskar när den modulerande signalen är negativ (andra
halvperioden). Bilden visar att perioderna i den modulerade bärvågen tar kortare tid (har
högre frekvens), när den modulerande signalen är positiv, och mertid (har lägre frekvens)
när den modulerande signalen är negativ. Bärvågen kommer alltså att pendla omkring ett
medelvärde, d.v.s. vara frekvensmodulerad.

Frekvensawikelsen L1f (deviationen) från bärvågens vilafrekvens är vid varje tillfälle
proportionell till den modulerande signalens amplitud. Sålunda är deviationen liten när
den modulerande signalens amplitud är liten och störst när amplituden når sitt toppvärde,
antingen amplituden är positiv eller negativ. Vid en modulationsfrekvens av 300 Hz
varierar bärvågsfrekvensen 300 gånger per sekund, vid 3kHz - 3000 gånger per sekund.

Likspänningsnivåer kan överföras med FM, eftersom en motsvarande frekvensavikelse kan
framställas.

Bilden visar också vad som oftast sägs, att bärvågsamplituden inte ändras av modulationen.
Detta är emellertid bara delvis sant, eftersom såväl bärvågsamplitud som sidbandsamplitud
varierar med modulationsindex, vilket förklaras nedan.

\subsubsection{Sidbanden vid vinkelmodulation}

Vid AM produceras endast ett sidbandspar med samma inne hål!, ett över och ett under
bärvågsfrekvensen. Vid vinkelmodulation, både vid FM och PM, produceras däremot flera
sidbandspar över och under bärvågsfrekvensen. Dessa sidband uppträder på multiplerna av
varje modulerande frekvens. Vid basband med samma frekvensomfång har därför en
vinkelmodulerad signal större bandbredd än en AM-signal.

Vid vinkelmodulation beror antalet sidband på sambandet mellan den modulerande frekvensen,
frekvensdeviationen och modulationsindex.

\subsubsection{Bandbredden vid vinkelmodulation}

Bild II 1-30 (nederst)

Vi gör tankeexperimentet att en FM-sändare moduleras med en fyrkantvåg. Frekvensen
kommer då att hoppa växelvis mellan frekvenserna \(f\) och \(f + \Delta f\). Sättet kallas FSK
(frekvensskiftnyckling) och används t. ex. vid sändning av radiofjärrskrift (RTTY, AMTOR,
Paketradio etc.).

Vi föreställer oss två sändare, som sänder varannan gång, varav den ena sänder frekvensen
\(f\) och den andra sänder \(f + \Delta f\). Båda sändarnas HF-signaler kommer då att bilda
ett frekvensspektrum, som förutom \(f\) och \(f + \Delta f\) även innehåller sidfrekvenser.

Bredden på detta spektrum beror bl. a. på nycklingsfrekvensen. Eftersom en fyrkantvåg
innehåller summan av dess grundfrekvens och övertoner, kommer alla dessa toner att
modulera vardera sändaren. De högsta modulerande LF-frekvenserna alstrar sidfrekvenserna
längst ut från vilofrekvensen. LF-signalens frekvensspektrum påverkar alltså
HF-signalens bandbredd.

Spektrum nederst i bilden är en förenklad framställning av frekvensskiftnyckling.

Bild II 1-30 Frekvensmodulation

Vid modulation med en sinussignal istället för med en fyrkantsignal, uppstår ett
frekvensspektrum som på överst i bilden.

\paragraph{Frekvensdeviation och modulationsindex}

Bild II 1-31

Vid vinkelmodulation uppstår talrika sidefrekvenser, som beror av den modulerande
frekvensen \(f_{LF}\). Amplitudfördelningen mellan sidfrekvenserna står i förhållande till
deviationen, varvid deras amplitud blir mindre desto längre bort från bärvågen de är.

I praktiken anses en sidfrekvens försumbar när dess amplitud är mindre än 1 \% av
amplituden för omodulerad bärvåg.

För beräkning av bandbredden används begreppet modulationsindex m, vilket är kvoten av
maximal deviation \(\Delta f\) och högsta frekvensen \(f_{LF}\).

\(m = \frac{\Delta f_{max}}{f_{LFmax}}\)

Inom amatörradion är det vanligt att arbeta med \(\Delta f_{max} = 3 kHz\) och
\(f_{LFmax} = 3 kHz\), d.v.s. \(m = 1\).

Vid modulationsindex \(m = 1\), gäller följande
formel för bandbredden \(b\)

\(b = 2 \cdot ( \Delta f_{max} + f_{LFmax}) = 2 \cdot \Delta f_{max} + 2 \cdot f_{LFmax}\)

Med ovan nämnda värden blir bandbredden \(b = 2 \cdot (3 kHz + 3 kHz) = 12 kHz\)

Bandbredden ökar således både med ökande deviation och ökande modulerande frekvens. För
att inte interferera med trafik på grannkanalerna måste såväl deviation som frekvensen på
den modulerande signalen begränsas. En deviationsbegränsare begränsar amplituden på denna
signal. Ett lågpassfilter reducerar den distorsion, som uppstår av begränsningen. Vidare
undertrycks modulerande frekvenser högre än 3 kHz, vilket är tillräckligt för överföring
av tal.

\paragraph{Jämförelse}

En VHF-rundradiosändare är tilldelad ett större frekvensutrymme och kan därför använda
mycket större bandbredd

Där är \(\Delta f_{max} = 75 kHz\) och \(f_{LFmax} =15 kHz\), därmed är \(m = \frac{75}{15} = 5\)
och \(b = 2 \cdot (75 + 15) = 180 kHz\).

Som framgår av tabellen på nästa uppslag varierar bärvågens liksom sidfrekvensernas
inbördes amplitud med modulationsindex. Detta skall jämföras med AM där bärvågens amplitud
är konstant och endast sidbandens amplitud varierar.

Vid vinkelmodulation utsläcks bärvågen \(A_0\) vid modulationsindex 2.404. Den blir sedan
"negativ" vid högre index, vilket betyder att den återkommer, men att dess fasläge blir
omvänt. I vinkelmodulation tas energin i sidbanden från bärvågen, vilket innebär att
den totala effekten förblir densamma oavsett modulationsindex.

\paragraph{Kännetecken för sändningsslaget F3E (FM)}

Fördelar: F3E-sändaren är enkel till sin uppbyggnad och hög överföringskvalitet
uppnås vid stor bandbredd, störningar från amplitudmodulerade signaler t. ex. tändgnistor
undertrycks i mottagaren.

Nackdelar: En relativt stor bandbredd behövs för överföring av ett basband med
stort frekvensomfång. Sändaren måste avge full effekt, även när modulation inte sker.

Bild 111-31 sidbandsspektrum vid FM-modulering med 1 sinuston

\begin{table*}[h]
\begin{center}
\begin{tabular}{ll|l|l|l|l|l|l|l|l|}
\cline{3-9}
&\multicolumn{1}{l}{}  & \multicolumn{7}{|c|}{Modulationsindex} \\ \cline{3-9}
&\multicolumn{1}{l|}{}  &   1   &   2   &    3   &    4   &    5   &    6   &    7   \\ \hline
\multicolumn{1}{|c|}{\multirow{11}{*}{\rotatebox[origin=c]{90}{Relativ amplitud på}}}&\(A_0\) & 0.765 & 0.224 & -0.260 & -0.397 & -0.178 &  0.151 &  0.300 \\
\multicolumn{1}{|c|}{}&\(A_1\) & 0.440 & 0.577 &  0.334 & -0.066 & -0.328 & -0.277 & -0.005 \\
\multicolumn{1}{|c|}{}&\(A_2\) & 0.115 & 0.353 &  0.486 &  0.364 &  0.047 & -0.243 & -0.301 \\
\multicolumn{1}{|c|}{}&\(A_3\) & 0.020 & 0.129 &  0.309 &  0.430 &  0.365 &  0.115 & -0.168 \\
\multicolumn{1}{|c|}{}&\(A_4\) &       & 0.034 &  0.132 &  0.281 &  0.391 &  0.358 &  0.158 \\
\multicolumn{1}{|c|}{}&\(A_5\) &       & 0.016 &  0.043 &  0.132 &  0.261 &  0.362 &  0.348 \\
\multicolumn{1}{|c|}{}&\(A_6\) & \multicolumn{2}{c|}{} &  0.011 &  0.049 &  0.131 &  0.246 &  0.339 \\
\multicolumn{1}{|c|}{}&\(A_7\) & \multicolumn{3}{c|}{} &  0.015 &  0.053 &  0.130 &  0.234 \\
\multicolumn{1}{|c|}{}&\(A_8\) & \multicolumn{4}{c|}{}           &  0.018 &  0.057 &  0.128 \\
\multicolumn{1}{|c|}{}&\(A_9\) & \multicolumn{4}{c}{Tomma fält för \(A_n\) under 0.01 (1 \%)} &        &  0.021 &  0.059 \\
\multicolumn{1}{|c|}{}&\(A_{10}\) & \multicolumn{5}{c}{} &  &  0.024 \\ \hline
\end{tabular}
\end{center}
\caption{Relativa amplituden på bärvåg \(A_0\) och sidfrekvenser \(A_1\)-\(A_{10}\) vid
modulationsindex 1-7 (Vid omodulerad bärvåg är modulationsindex 0. Då är
bärvågens relativa amplitud 1.0)}
\end{table*}

\subsection{Fasmodulation (även kallat PM)}

Vid fasmodulation varierar bärvågens fasläge i förhållande till ett
referensvärde. Vid PM är frekvensändringen - deviationen - direkt proportionell
till hur snabbt fasläget på den modulerande frekvensen ändras och till den
totala fasändringen. Hastigheten på fasändringen är direkt proportionell till
frekvensen på den modulerande frekvensen och till den momentana amplituden på
den modulerande signalen.

Det betyder att deviationen i PM-system ökar både med den momentana amplituden
och frekvensen på den modulerande signalen. Detta att jämföras med FM-system där
deviationen är proportionell till den momentana amplituden på den modulerande
signalen.

I PM-system uppfattar demodulator i mottagaren endast momentana ändringar i
bärvågsfrekvensen. Till skillnad från vid FM, så kan därför ändringar i
likspänningsnivåer överföras endast om en fasreferens används.

Med konstant amplitud på insignalen till modulatorn, så är vid PM
modulationsindex konstant oavsett modulerande frekvens, medan vid FM
modulationsindex varierar med den modulerande frekvensen.

\subsection{Frekvens- och fasmodulation jämförs}

\begin{itemize}

\item Frekvensmodulation (FM) alstras genom att sändarens oscillatorfrekvens
varieras (devieras) i takt med den modulerande signalen (t.ex. tal). Det gör man
genom att variera resonansfrekvensen i den svängningskrets som styr
oscillatorfrekvensen.

\item Fasmodulation (PM) alstras vanligen genom att efter sändarescillatorn
variera den modulerande signalens fasläge i förhållande till en omodulerad
bärvåg - s.k. fasmodulering. Det gör man genom att variera resonansfrekvensen i
en svängningskrets efter oscillatorn- d.v.s. utan att påverka
oscillatorfrekvensen.

\item I båda fallen ändrar man alltså resonansfrekvensen i en svängningskrets i
takt med frekvensen i den modulerande spänningen, men att denna krets har olika
placering i FM-sändare respektive PM-sändare.

\item I sändaren alstras det i båda fallen utfrekvensersom devierar från
oscillatorns vilofrekvens. Graden av deviation skiljer emellertid vid FM och PM.
Vid FM är deviationen proportionell mot amplituden på den modulerande
underbärvågen medan deviationen vid PM är proportionell mot produkten av den
modulerande underbärvågens amplitud och frekvens.

\item Den hörbara skillnaden mellan FM och PM är därför en annorlunda
frekvensgång. Vid samtidig användning av PM-sändare och FM-mottagare är det
alltså lämpligt att justera frekvensgången i PM-sändarens modulator, lämpligen
med 6 dB dämpning per oktav ökad frekvens.

\end{itemize}

\subsection{Pulsmodulation}

Pulsmodulation används mest i mikrovågområdet. Pulsmodulerade signaler sänds
vanligen som en serie korta pulser åtskilda av relativt långa pauser utan
modulering.

En typisk sändning kan bestå av pulser med en längd av 1 \(\mu\)s och en frekvens
av 1000 Hz. Toppeffekten på en pulssändning är därför mycket högre än dess
medeleffekt.

Före WARC 79 var symbolen för all pulssändning P. Därefter används P endast för
omodulerade pulståg. Annan pulsmodulation har följande symboler

\begin{description}
\item[K] - puls-/amplitudmodulation (PAM)
\item[L] - pulsviddmodulation (PWM)
\item[M] - pulsposition/fasmodulation (PPM)
\item[Q] - vinkelmodulation under pulsen
\item[V] - kombination av dessa eller annat sätt.
\end{description}

\begin{table*}[h]
\begin{center}
\begin{tabular}{|l|l|l|l|l|}
\hline
Sändningsslag & Amplituden på & Tonhöjden på & Bandbredden b      & För stor amplitud \\
              & LF-signa!en   & LF-signalen  & förhåller sig till & på LF-signalen \\
              & påverkar      & påverkar     &                    & medför \\ \hline
A3E (AM) & amplituden i   & sidfrekvenser- & LF-signalens    & övermodulering \\
         & båda sidbanden & nas avstånd    & högsta frekvens & och för stor bandbredd \\
         &                & från bärvågen  & & \\
J3E (SSB)& amplituden på  & sidfrekvenser- & skillnaden mellan & för stor bandbredd,\\
         & utsänt sidband & nas avstånd    & LF-signalens      & överstyrning av\\
         &                & från bärvågen  & högsta och lägsta & förstärkarsteg\\
         &                &                & frekvens          & \\
F3E (FM) & deviationen    & hastigheten på & dubbla summan     & för stor deviation,\\
         &                & bärvågens      & av största devia- & för stor bandbredd\\
         &                & frekvens-      & tion och högsta   & \\
         &                & ändring        & LF-frekvens       & \\ \hline
\end{tabular}
\end{center}
\caption{Jämförelse mellan några. vanliga. sändningsslag inom amatörradio}
\end{table*}

\cleardoublepage

\section{Effekt och energi}

\subsection{Effekt i en sinusformad signal}
För beräkning av effekten av en sinusformad signal använder man effektiwärdet
av spänning och ström.

\(U_{eff} = \frac{U_{max}}{\sqrt{2}}\) och \(I_{eff} = \frac{I_{max}}{\sqrt{2}}\)

\(P = U_{eff} \cdot I_{eff}\)

Bild II 1-32 Effektförhållande

\subsection{Effektändring uttryckt i dB}
Måtten i det metriska systemet är alldagliga och ingen finner det märkligt att
det t. ex. går tio decimeter på en meter. Däremot är begreppet decibel ovant för
många.

I detta avsnitt förklaras det mycket användbara begreppet decibel. Decibel (dB)
är en tiondedel av grundenheten Bel (B).

Räkning med decibel grundas på logaritmer, som är ett bekvämt sätt att uttrycka
och behandla talvärden.

Decibel är ett dimensionslöst uttryck för graden av dämpning alternativt
förstärkning.

Effektdämpning är följden av att vissa komponenter bromsar elektrisk ström. Den
bromsande faktorn kan vara en resistans R, induktans L, kapacitans C eller
sammansatta nätverk av R, L och C.

Effektförstärkning innebär att en transistor, ett elektronrör eller annan s.k.
aktiv komponent kan styra en större elektriskström och därmed större effekt än
den själv styrs med. Vad som förorsakar effektförändringarna går vi inte in på i
detta sammanhang, utan byggdelarna betraktas som "svarta lådor" med
anslutningsklämmor.

En byggdel med två ingångs- och två utgångsklämmor kallas för "fyrpol".

Antag attden inmatade effekten P är 1 W. Om effekten inte ändras vid passagen
genom fyrpolen, så är även den uttagna effekten 1 W.

Effektförhållandet mellan in- och utgångarna är då

\(\frac{P_{in}}{P_{ut}} = \frac{1 watt}{1 watt} = 1 (kvoten = 1)\)

Oförändrad effekt varken dämpas eller förstärks, varför både dämpningen och
förstärkningen har talvärdet 0. Enheten på talvärdet är Bel, dämpningen eller
förstärkningen är således 0 Bel. En tiondel därav är 0 decibel (0 dB).

Omräkning av kvoten av en effektändring till dB görs så, att 1O-logaritmen för
kvoten söks och resultatet blir effektändringen uttryckt i Bel (B). Om
resultatet uttrycks i dB, skall Bel-värdet multipliceras med 10.

Logaritmer förklaras i Appendix B.

För att förenkla beräkningen av dB-talet divideras det högre effekttalet med det
lägre. Bokstaven a i följande formler betyder antingen förstärkning (+ a) eller
dämpning (-a) beroendet på vilket förtecken som sätts.


\(a[B] = \log \frac{P_{hög}}{P_{låg}}\)

\(a[dB] = 10\log \frac{P_{hög}}{P_{låg}}\)

Att addera eller subtrahera värden på en logaritmisk skala, motsvarar att
multiplicera resp. dividera värden på en linjär skala. Huvudskalorna på en
räknesticka är logaritmiska. (Räknestickan är ett enkelt, förut mycket använt
hjälpmedel).

Med hjälp av följande nomogram kan en effektändring, uttryckt som kvot
(effekterna dividerade med varandra), omvandlas till decibel och omvänt.



Följande avrundade värden kan utläsas:
O dB = 1
1 dB = 1.25 2 dB = 1.6
3 dB = 2
4 dB = 2.5
5 dB = 3.2
6 dB = 4
7 dB = 5
8 dB= 6.3
9 dB = 8
1O dB = 1O 11 dB = 12.5
d. v. s. vid ökning fördubblas effekten för var
3:e dB och vid minskning halveras effekten
för var 3:e dB.

Om kvoten är en eller flera 1O-potenser
högre än 1O, så kan nomogrammet utökas
enligt följande tabell.
Kvot av Analys
PhögiPiåg
1
1 har O nollor
1O
1O har 1 nolla
100
100 har 2 nollor
1 000
1 000 har 3 nollor
1O 000 1O 000 har 4 nollor

Skriv

dB

o. 10 = o
1·10=10

2. 10

= 20

3. 10 = 30
4. 10 = 40

\subsection{Strömändring uttryckt i dB}
Förhållandet mellan strömmar liksom mellan spänningar kan även uttryckas i dB, men
annorlunda än mellan effekter. En fyrpol
med inbördes lika ingångs- och utgångsimpedans är förutsättningen för jämförelse.
Enligt Joules lag är P= l

2

•R

(P= U·~

En jämförelse uttryckt i dB kan endast göras
under samma förutsättningar; här att impedanserna (resistanserna) är lika,

111 -54

a[ dB]= 1Olog 'h'og22
flåg

Eftersom log

x2

=

2 ·log x, fås slutligen

a[ dB] = 20 log /hög
flåg

\subsection{Spänningsändring uttryckt i dB}
Förhållandet mellan spänningar kan uttryckas i dB på ett liknande sätt som med
strömmar.

~

2

Enligt Joules lag är P=

(P=

U·~

Två effekter kan ställas i förhållande till varandra på följande sätt:

~ög

-=
~åg

uhög2:R
2
ulåg

:R

R avkortas och efter omskrivning fås en

formel som liknar den för strömmar

~ög

2
uhög

~åg

ulåg

-=--2

log Uhög
qåg

R kan avkortas om in- och utgångsimpedanserna (resistanserna) är lika.

således

Effektförhållandet eller kvadratvärdet på
strömförhållandet kan uttryckas logaritmiskt
i B eller dB

a[ dB] = 20

således

Effekt

~ög

lhög

~åg

flåg

2

-=-2

Med följande nomogram kan kvoten av en
ström- eller spänningsändring omvandlas till
decibel och tvärt om.
Följande avrundade värden kan utläsas:
O dB = 1
1 dB ~ 1.12 2 dB ~ 1.25
3 dB ~ 1.4
4 dB ~ 1.6
5 dB ~ 1.8
6 dB ~ 2
7 dB ~ 2.24 8 dB ~ 2.5
9 dB ~ 2.8
1O dB ~ 3.2 11 dB ~ 3.6
d. v. s. vid ökning fördubblas strömmen resp.
spänningen för var 6:e dB och att vid minskning halveras strömmen resp. spänningen
för var 6:e dB.

L-LÄRA
1

1.2

1.1

l

l

o

i
1

2

lijl l i l i

l
o2 4 6 8
l

1.3
!

l

3
l
i

l
10

l

l
2

l

l

4
l
l
12

l

J

l
3

5
l
l
14

1.6

1.5

1.4
l

l

i

6
l

l

l
4

7
li

l

l
16

Om kvoten är en eller flera 1O-potenser
högre än 1O, så kan nomogrammet utökas
enligt följande tabell.
Skriv
Kvot*
Analys
20
1 har O nollor
1
1 . 20
10
1O har 1 nolla
2. 20
100
100 har 2 nollor
1 000
1 000 har 3 nollor 3. 20
10 000 1 O 000 har 4 nollor 4. 20
*kvot av Uhö/U 1å9 resp. lhö/l,å9

o.

=

dB

o

= 20

= 40

= 60
= 80

Se Appendix C för beräkning med tabeller.

\subsection{Ändring uttryckt i dB vid förstärkande eller
dämpande anordningar kopplade i serie}

Ett räkneexempel på effektändringar:
Fråga:
Vi har en enkel sändaranläggning med
ett drivsteg med en in effekt av 1O W. Drivsteget förstärker med 6 dB. Vidare har vi ett
effektslutsteg som förstärker med 1O dB.
Antennkabeln dämpar med 1 dB.
Med vilken effekt matas själva antennen?
Svar: (två sätt att lösa uppgiften)
1) Drivsteget förstärker fyra gånger, slutsteget förstärker tio gånger och kabeln
dämpar 1/1.25 = 0.8 gånger. Antennen
matas då med 10 · 4 · 10 · 0.8 = 320 W.
2) Drivstegets 6 dB plus slutstegets 1O dB
minus antennkabelns 1 dB = 15 dB.
15 dB är 1O+ 5 dB d.v.s. 1O· 3.2 = 32 ggr.
Antennen matas med 1OW· 32 = 320 W.

1.8

1.7

l

l

l
5

8
l
l
18

2.0 g gr

1.9
l

l

l

l

6

Spänning
dB

10 ggr

9
l
i

Spänning
20

dB

\subsection{Impedansanpassning}
Impedansanpassning är av stor betydelse
inom kommunikationsstekniken. Normaltvill
man nämligen överföra mesta möjliga effekt
från energikällan (t. ex. sändaren) till förbrukaren (t.ex. antennen).
Varje spänningskälla har en inre resistans Ri. Det innebär som först att källan inte
kan avge oändligt stor ström. För att förenkla
det hela antar vi nu att en sändare med den
inre resistansen Ri ansluts direkt till en antenn med resistansen Ra.
Målet med anpassningen är att finna det
optimala förhållandet mellan sändarresistansen och antennresistansen för att kunna
överföra maximal effekt. Vi har de två ytterlighetsfallen obelastad sändare respektive
kortsluten sändare. Sändarens elektromotoriska kraft (EMK) betecknas som E [V]
och sändarens utspänning som U [V].
Fall1.
En obelastad sändare avger ingen ström när
ingen antenn eller en med oändligt stor resistans har anslutits.
Alltså vid obelastad sändare:

Ra =oo

l= O U= E

Fall2.
När sändarutgången är kortsluten, d.v.s.
belastningen (antennresistansen) är noll
ohm, avger sändaren en ström som beror av
EMK och inre resistans. Eftersom såndarutgången är kortsluten är utspänningen Unoll.
Alltså vid kortsluten sändare:

Ra=O !=E
Ri

U=O

111 -55

EL-LÄRA
l båda ytterlighetsfallen är den effekt som
omsätts i Ra lika med noll. För att få ut någon
effekt måste man alltså söka ett värde på Ra
som ligger mellan ytterlighetsvärdena.
Enligt formeln för spänningsdelare är
utspänningen

U=E·

Ra
Ra+Ri

Formeln för uteffektens effektivvärde är

u2
p=ut

R

a

Efter insättning får man
p = p ·Ra

ut

(Ra +Ri)2

För att finna det optimala förhållandet
mellan Ri och Ra, d.v.s. när Ra tar upp maximal effekt, måste man differentiera formeln
med d Pa/dRa, men vi hoppar över denna
utflykt i matematiken.
l stället konstaterar vi helt enkelt att
maximal effektöverföring sker när Ri= Ra.

\subsection{Förhållandet mellan in- och uteffekt uttryckt som \% verkningsgrad}

Antag att en antennkabel har en effektförlust
av 1 dB. Det innebär en effektdämpning av
1.25 gånger, d.v.s. 0.8. Nu matar vi in 1O W
i kabeln och får alltså ut 8 W. Hur stor
verkningsgrad har kabeln uttryckt i o/o ?
Lösning:

8

1J = 1Q ·1 00 = 80o/o

\subsection{Toppvärdeseffekt P.E.P.}

Uteffekten från en sändare kan mätas över
en konstlast (dummy load). En konstlast är
en res istor som kan omsätta sändarens hela
effekt till värme. Med HF-mätprob och en
detektordiod eller en HF-voltmeter kan man
mäta effektivvärdet på spänningen över
konstlasten och beräkna uteffekten med
formeln

u2

p
ut =RU = HF-spänningens effektiwärde
R = resistansen i konstlasten

111-56

På grund av utsignalens karaktär kan
man inte mäta effektiwärdet av uteffekten
från SSB-sändare.
Med oscilloskop kan man emellertid mäta
utspänningen på den största modulationstoppen.
Med detta toppvärde kan man beräkna
spänningen över konstlasten.
Uteffekten definierad som P.E.P. (Peak
Envelope Power) är "den medeleffekt som
matas in i en antennmatarledning under det
högsta effektvärdet inom en frekvenscykel
och mätt under normal drift".

f12

P.E.P.=R
där Oär momentanspänningen på den största
modulationstoppen.


