\chapter{CEPT HAREC krav}

Den här upplagan av KonCEPT är baserad på CEPT T/R 61-02 Harmonised Amatuer Radio
Examination Certificate (HAREC) Edition 4 June 2016. För att underlätta att se
att alla HAREC-kraven finns täckta samt se vilka HAREC krav som en viss textdel
uppfyller.

För den studerande så kan detta hjälpa till att förstå hur omfattande kraven är
från den internationella överenskommelsen som CEPT HAREC är. Det är genom dessa
gemensamma minimumregler som jämförbarheten i kunskap länder emellan sedan kan
utföras.

Själva kunskapskraven i HAREC finns beskrivna i ``Annex 6: Examination syllabus and requirements for a HAREC''. Alla krav i den finns med här, i HAREC
orginalformulering, enbart omarbetad med avseende på format. För varje krav
redovisas sedan en eller flera referenser i den övriga texten där det kravet
anses uppfyllas.

\section{Introduction}

\renewcommand{\theenumii}{\arabic{enumii}}
\renewcommand{\labelenumii}{\theenumi.\theenumii}
\begin{enumerate}[label=\alph*.]
\item Where quantities are referred to, candidates should know the units in which these quantities are expressed, as well as the generally used multiples and sub-multiples of these units. (\ref{myHAREC.I.a})\label{HAREC.I.a}
\item Candidates must be familiar with the compound of the symbols. (\ref{myHAREC.I.b})\label{HAREC.I.b}
\item Candidates must know the following mathematical concepts and operations:
\begin{enumerate}
\item adding, subtracting, multiplying and dividing (\ref{myHAREC.I.c.1})\label{HAREC.I.c.1}
\item fractions (\ref{myHAREC.I.c.2})\label{HAREC.I.c.2}
\item powers of ten, exponentials, logarithms (\ref{myHAREC.I.c.3})\label{HAREC.I.c.3}
\item squaring (\ref{myHAREC.I.c.4})\label{HAREC.I.c.4}
\item square roots (\ref{myHAREC.I.c.5})\label{HAREC.I.c.5}
\item inverse values (\ref{myHAREC.I.c.6})\label{HAREC.I.c.6}
\item interpretation of linear and non-linear graphs (\ref{myHAREC.I.c.7})\label{HAREC.I.c.7}
\item binary number system (\ref{myHAREC.I.c.8})\label{HAREC.I.c.8}
\end{enumerate}
\item Candidates must be familiar with the formulae used in this syllabus and be able to transpose them. (\ref{myHAREC.I.d})\label{HAREC.I.d}
\end{enumerate}

\section{Technical Content}

\renewcommand{\labelenumi}{\theenumi.}
\renewcommand{\theenumii}{\arabic{enumii}}
\renewcommand{\labelenumii}{\theenumi.\theenumii}
\begin{enumerate}
\item ELECTRICAL, ELECTRO-MAGNETIC AND RADIO THEORY
\begin{enumerate}[noitemsep]
\item Conductivity; (\ref{myHAREC.a.1.1})\label{HAREC.a.1.1}
\item Sources of electricity; \label{HAREC.a.1.2}
\item Electric field; \label{HAREC.a.1.3}
\item Magnetic field; \label{HAREC.a.1.4}
\item Electromagnetic field; \label{HAREC.a.1.5}
\item Sinusoidal signals; \label{HAREC.a.1.6}
\item Non-sinusoidal signals, noise; \label{HAREC.a.1.7}
\item Modulated signals ; \label{HAREC.a.1.8}
\item Power and energy; \label{HAREC.a.1.9}
\item Digital signal processing (DSP). \label{HAREC.a.1.10}
\end{enumerate}
\item COMPONENTS
\begin{enumerate}[noitemsep]
\item Resistor; \label{HAREC.a.2.1}
\item Capacitor; \label{HAREC.a.2.2}
\item Coil; \label{HAREC.a.2.3}
\item Transformers application and use; \label{HAREC.a.2.4}
\item Diode; \label{HAREC.a.2.5}
\item Transistor; \label{HAREC.a.2.6}
\item Heat dissipation; \label{HAREC.a.2.7}
\item Miscellaneous. \label{HAREC.a.2.8}
\end{enumerate}
\item CIRCUITS
\begin{enumerate}[noitemsep]
\item Combination of components; \label{HAREC.a.3.1}
\item Filter; \label{HAREC.a.3.2}
\item Power supply; \label{HAREC.a.3.3}
\item Amplifier; \label{HAREC.a.3.4}
\item Detector; \label{HAREC.a.3.5}
\item Oscillator; \label{HAREC.a.3.6}
\item Phase Locked Loop [PLL]; \label{HAREC.a.3.7}
\item Discrete Time Signals and Systems (DSP-systems). \label{HAREC.a.3.8}
\end{enumerate}
\item RECEIVERS
\begin{enumerate}[noitemsep]
\item Types; \label{HAREC.a.4.1}
\item Block diagrams; \label{HAREC.a.4.2}
\item Operation and function of the following stages; \label{HAREC.a.4.3}
\item Receiver characteristics. \label{HAREC.a.4.4}
\end{enumerate}
\item TRANSMITTERS
\begin{enumerate}[noitemsep]
\item Types; (\ref{myHAREC.a.5.1})\label{HAREC.a.5.1}
\item Block diagrams; (\ref{myHAREC.a.5.2})\label{HAREC.a.5.2}
\item Operation and function of the following stages; (\ref{myHAREC.a.5.3})\label{HAREC.a.5.3}
\item Transmitter characteristics. (\ref{myHAREC.a.5.4})\label{HAREC.a.5.4}
\end{enumerate}
\item ANTENNAS AND TRANSMISSION LINES
\begin{enumerate}[noitemsep]
\item Antenna types;
\item Antenna characteristics;
\item Transmission lines.
\end{enumerate}
\item PROPAGATION
\item MEASUREMENTS
\begin{enumerate}[noitemsep]
\item Making measurements;
\item Measuring instruments.
\end{enumerate}
\item INTERFERENCE AND IMMUNITY
\begin{enumerate}[noitemsep]
\item Interference in electronic equipment;
\item Cause of interference in electronic equipment;
\item Measures against interference.
\end{enumerate}
\item SAFETY
\end{enumerate}