\section{Elektronrör}
Allmänt
Ett elektronrör består av två eller flera elektroder i en lufttom glaskolv.

Direktupphettad
katod

Indirekt upphettad katod

Allmän
symbol

Bild II 2-24 Schemasymboler för dioder

Vakuumdioden (två.elektrodröret)
Bild 112-24
Dioden innehåller två elektroder
a anod
k katod, med f f= glödtråd (filament)
Anoden skall dra elektronerna från katoden.
Katoden skall avge elektronerna och måste
därför hettas upp.
Upphettningen av katoden görs på något
av följande sätt:
Direkt upphettning, d.v.s. katoden är i sig
själv en glödtråd. En 4- till 6-volts strömkälla
är vanligt.
Indirekt upphettning, d.v.s. en glödtråd
omsluter och hettar upp ett speciellt katodmaterial. En 1.5 till12.6 volts glödströmkälla
är vanligt.
Ed i soneffekten
Bild 112-25
När katoden upphettas lossnar fria elektroner från den och bildar ett moln. Med en
spänning mellan anod och katod, med
anoden positiv, så kommer elektronerna att
dras mot anoden. En s.k. anodström börjar
att flyta.

uh

la/Ua .. karaktäristikan för en vakuumdiod
Bild II 2-26
Bild II 2-25 Edisoneffekten
När anoden ges positiv potential (anodspänning), flyter en elektronström från katod
la l Ua- karaktäristik
till anod (anodström).
la
Om anodspänningen
ua ökar' så ökar anodströmmen la. Varje
par av talvärden representerar en punkt
i ett diagram, som det
på bilden. När anodspänningen ökattill ett
Ua
visst värde, så ökar
(
Al B
inte anodströmmen
l
ytterligare.
l ett melA: Initialströmsområde
lanområde är kurvan
B: Den linjära delen
i det närmaste rak.
C: Mättnadsområde
Bild II 2-26 Diodens karaktäristik

112-29

P NENTER
likriktarverkan

När anoden i en vakuumdiod ges positiv
potential i förhållande till katoden, flyter en
s.k. anodström förutsatt att katoden upphettas så att den avger fria elektroner.
När anoden ges en negativ potential i
förhållande till katoden flyter däremot ingen
anodströ m.
Vakuumdioden kan därför användas för
likriktning av växelströmmar. Den har en
likriktande funktion.

En anodström flyter

Halwågslikriktning.
Bild 112-27
När anoden ges en omväxlande positiv och
negativ potential, en växelspänning, så flyter
anodström under varje positiv halvperiod av
växelspänningen. En likströmspuls uppstår
under varannan halvperiod.
Helvågslikriktning.
Bild 112-28
Med ett elektronrör med dubbla anoder och
en transformator med mittuttag på sekundärlindningen, kan växelspänningens båda
halvperioder utnyttjas så, att anodström flyter i samma riktning under alla halvperioder.

Det flyter ingen ström

En pulserande likström flyter

Bild II 2-27 Halwågslikriktning

1 :a halvvågen

Bild II 2-28 Helvågslikriktning

112-30

2 :a halvvågen

K MP NENTER
Ua
VÄXELSPÄNNING
PAANODERNA

HALVVAGsLIKRIKTNING

laf

!@.

f@.

~

HELVAGsLIKRIKTNING

r
F

Bild II 2-29 Likriktande funktion

Vakuumtrioden (treelektrodröret)

Triodens funktion

g2

TRIOD

PENTOD

Bild II 2-30 Symboler för triod och pentod
Trioden innehåller tre elektroder
a anod
g 1 styrgaller
k katod, med f f= glödtråd (filament)

Det flyter både anod- och gallerström

Bild II 2-31
En triod fungerar som en diod, när styrgallret
ges samma potential som katoden. Valet av
förspänning avgör triodens arbetssätt. styrgallret kan ges positiv, neutral eller negativ
potential (förspänning) i förhållande till katoden. Med styrgallret positivt ökar anodströmmen. Med gallret negativt minskar den.
Trioden har en förstärkande funktion eftersom anodströmmen kan styras med styrgallret. En liten ändring av gallerspänningen
medför stor ändring av anodströmmen. Vid
positiv förspänning flyter det gallerström,
som inte får bli för hög. Vanligen väljs en
negativ förspänning.

Det flyter anodström men ingen gallerström

Bild II 2-31 Elektronstömmen i en triod

112- 31

K MP NENTE
Triodens strömkretsar och strömkällor

Glödströmskrets Anodkrets Gallerkrets
Glödbatteri
Anodbatteri Gallerbatteri
Glödspänning Ut Anodsp. Ua Gallersp. U91
Glödström lt
Anodstr. la Gallerstr. 191
Vanligen används nätdrivna strömkällor i
stället för batterier.
Valet av gallerförspänning är avgörande för
triodens arbetssätt.

Tetroden (fyraelektrodröret)

Denna rörtyp innehållerfyra elektroder. Uppbyggnaden är densamma som pentodens,
men bromsgallret saknas.

Pentoden (femelektrodröret)

Pentaden innehåller fem elektroder.
a
anod
g3
bromsgaller
g2
skärmgaller
styrgaller
g1
k
katod, med f f = glödtråd (filament)
Bromsgallret förbinds med katoden. Skärmgallret ges en potential som är något lägre än
anodspänningen. Broms- och skärmgallren
förhindrar elektronerna att studsa tillbaka till
styrgallret efter anslaget mot anoden.

- - - - - - · - - - Ua.
Bild II 2-32 Karaktäristika för elektronrör

112-32

Karaktäristika för elektronrör
Bild II 2-32

1iU91 -diagram för en triod eller pentod, vid Ua
=konstant
liUa- diagram för en triod, vid U~ 1 =konstant
liUa - diagram för en pentoa, vid U91 =
konstant
Tre kurvor visas i VUa- diagrammen, med
olika värden på U91 = konstant (U 91 är s.k.
parameter).

la

l ug1 - karaktäristik för en triod eller pentod

la

l

18

l U8 - karaktäristik för en pentod

ua- karaktäristik för en triod

Branthet S och inre resistans Ri
Bild 112-33

•

Om man (vid konstant anodspänning)
ändrar gallerförspänningen med värdet
ilU 91 så ändrar sig anodströmmen med
värdet illa.

la

il/
Branthet S = a
ilU91
S [mA/V]

illa [mA]

ilU 91 [V]

Bild 112-34
• Om man (vid konstantgallerförspänning)
ändraranodspänningen medilUasåändras anodströmmen med värdet illa
Inre resistans
Ri [kQ]
•

BRANTHET

R.= ilUa
' illa
ilUa [V]

3mA

Om man vill ändra anodströmmen med
ett yärde illa , så ges det två möjligheter:
- Andra gallerförspänningen med värqet ilU 91
- Andra anodspänningen med värdet
il Ua
Med ändring av gallerförspänningen
med värdet ilU 1 kan man åstadkommasamma anodströmsändring illa som
med en ändring av anodspänningen
med värdet ilUa.

:::;. ... ,...J....J..---Ug1
-4 V
-2V

Bild II 2-33 Branthet

9mA

INRE MOTSTAND Rl
8m A

R;

=

v

10
1 mA

v

10
0,001 A

: : - = - - - = 10000

A

:: 10 000 1l

Bild II 2-34 Inre resistans

112-33

K MP NENTER

PT

Barkhausen's elektronrörsformler
Förstärkningsfaktorn J.l
Följande samband gäller mellan de s.k. rörkonstanterna

J1

= S·Ri

Exempel:
Beräkna~,

om S = 2 mA/V
Svar:~=

20

R = 1o kQ
(~är

~

=?

dimensionslös)

Transistorer jämfört med elektronrör
Transistorer
Fördelar:
Lågt pris- små dimensioner -lång livslängd
-enkel strömförsörjning (g lödström behövs
inte) -låg driftspänning (6V, 12V ...... ).
Nackdelar:
Känslighet för överbelastning och höga
temperaturer.
Elektronrör
Fördelar:
Tålighet mot överbelastning
Nackdelar:
Behöver hög anodspänning
Behöver glödström
Utrymmeskrävande

Ett användningsområde, där elektronrör
ännu är vanliga, är i större sändarslutsteg.
Transistorer ersätter numera nästan helt
elektronrören, men man bör ändå känna
elektronrörens egenskaper och arbetssätt.
