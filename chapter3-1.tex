\section{Komponenter i serie}

parallellt

Seriekopplade resistorer
Bild II 3-1
Den totala resistansen av seriekopplade
resistorer är summan av resistanserna

1

1

1

1

R=R-1 +R-2 +R-3 ... Rn
För två resistorer gäller

1
1
1
eller
R R1 R2
För tre resistorer gäller

-=-+-

Strömmen är lika stor genom alla seriekopplade resistorer i strömvägen (ingen avgrening)

l= ~ + /2 + /3 + .....
Det totala spänningen över seriekopplade
resistorer är summan av spänningen över
var och en av dem

U=U1 +U2 +U3 + .....
Spänningen över var och en av seriekopplade
resistorerförhåller sig som deras resistanse r.
För två resistorer gäller

R1

u1
u2

1

1

1

R=------R1~·-R~2·R~3~---R1·R2 + R1·R3 + R2 ·R3
Strömmen förgrenar sig mellan parallellkopplade resistare r. Den totala strömmen är
summan av grenströmmarna
(Kirchhoff's 1 :a lag)
Spänningen är lika stor över var och en av
parallellkopplade resistorer

u= u1 = u2 = u3 =

-=-

R2

.. .

un

(Kirchhoff's 2:a lag)

Parallellkopplade resistorer
Bild II 3-2
Den totala resistansen av parallellkopplade
resistorer är lägre än den lägsta enstaka
resistansen

Grenströmmarna genom parallellkopplade
resistorer fördelar sig omvänt proportionellt
till deras respektive resistanser.
För två resistorer gäller

l

12

R2
R1

,

(r

1

- = - + - + - eller
R R1 R2 R3

-l

11

+

l '
lz

l,....

R1

u1

:r2
l

Bild 1/3-1 Seriekopplade resistorer

u

Gu

~

,,

u1[ R,

,.....

l

1

z

RZ

J11

JUz

J 'z

Bild II 3-2 Parallellkopplade resistorer
113-1

KRETSAR
Spänningsdelare
Bild II 3-3
Spänningsdelare förekommer i flera former.
Bilden visar en spänningsdelare med
resistorer där spänningen U delas upp i
spänningen U1 över resistorn R1 respektive
U2 över R2 • Man kan då t. ex. använda spänningen
för något ändamål.
Ett alternativ till spänningsdelning med
resistorer med fasta värden är patentiometern Det är en variabel spänningsdelare i
form av en resister med ett uttag som kan
flyttas mellan ändanslutningarna.
Om man nu ansluter en apparat parallellt
över R2 , t. ex. ett instrument vars inre resistans motsvaras av Ry, så kommer spänningarna över R1 och R2 att påverkas.
Om Ry är mycket större än R2 , så kan
man bortse från påverkan. För att beräkna
U2 kan man då använda följande formel för
en obelastad resistiv spänningsdelare.

.....l

u1[ R1

~l

u2

U2 =
U
eller
U2 =U·----"--R2 R1 +R2
+R2
Om Ry däremot är av samma storleksordning eller lägre än R2 , så måste man för
att beräkna U2 använda en formel för en
belastad resistiv spänningsdelare, t. ex.
R2 ·Rr
U2 =U·

u
~ lz

21

U

1.,...

~ly
Ry

R2

~ lz

+ly

Bild II 3-3 Resistiv spänningsdelare
Det
ström mellan X och Y när det
finns en potentialskillnad- spänning- däremellan. Bryggan är då i obalans.
Det flyter däremot ingen ström där när
det inte finns en potentialskillnad, d.v.s. när
bryggan är i balans. Balans (mätvärdet) får
man genom justering av den graderade
potentlometern till noll ström. Då gäller sambandet

R2 +Rr
R2 ·Rr
R1+--=-R2+Rr

Härav förstås att t. ex. en spänningsmätning ger olika resultat beroende på den inre
resistansen i voltmetern.

Wheatstone's brygga
Bild 113-4
En speciell tillämpning av spänningsdelare
är en s.k. brygga (Wheatstone's brygga),
som används för att jämföra spänningar.
Bryggan kan ses som två parallellkopplade spänningsdelare varav den ena är en
potentiometer med en skala graderad t. ex.
i n. Den andra spänningsdelaren består av
en resister med känd resistans och en resisto r med okänd resistans, d.v.s. mätobjektet
l ledningen som förbinder de respektive mittuttagen X och Y, finns en amperemeter som
nollströmsindikator.

113-2

JUz

u

Bild 113-4 Wheatstone's brygga

KRETSAR
Spänningsdelare och bryggor har tagits
med för att påvisa att apparater påverkar
varandra när de kopplas samman, vilket är
fallet även vid mätningar.
Spänningsdelning kan även utföras med
kondensatorer och induktorer förutsatt att
det är fråga om en växelströmskrets.
Parallellkopplade kondensatorer
Bild II 3-5
l stället för en enda kondensator kan man
parallellkoppla flera kondensatorer för att
uppnå önskad total kapacitans.
Den totala kapacitansen för parallellkopplade kondensatorer är summan av de
enskilda kapacitanserna.
C=C1 +C2 +C3 ... Cn

Seriekopplade kondensatorer
Bild II 3-6
Den totala kapacitansen för seriekopplade
kondensatorer är lägre än kapacitansen för
kondensatorn med det minsta värdet.

1

C1 = 5 11 F C2 = 1o 11F
C = C1 + C2 = 5 + 1O= 15 J..LF

2.

C1 = 1 nF

eller
C= C1. C2
C1 c2
C1+C2
För tre kapacitanser gäller

c
1

1

C=

1

c1c2c3

C1C2 + C1C3 + C2C3

C1 = 5 J.tF

C2 = 5 pF

C2 = 1o11 F

1
1
1
-=-+c1 c2

C= 5·10 =3! F
5+10
3 Jl

c

+
+

+
+

+

+
+

+

+

+

+
l+

1

-=-+-+-eller
C C1 C2 C3

+

+

1

Räkneexempel:

C= C1 + C2 = 1+ 0.005 = 1.005 nF

+

1

! = 1 +-1-

Räkneexempel:

1.

1

C = C + C + C + . . . . . .. . . ;
2
3
1
För två kapacitanser gäller

+ +l

(1

l+ +

+l

(z

Bild II 3-5 Parallellkopplade kondensatare

+

-· (1

+

Bild II 3-6 Seriekopplade kondensatorer

113-3

KRETSAR
Sammankopplade induktorer
Galvaniskt kopplade induktorer

Induktansvärdetför galvanisktsammankopplade induktorer kan i princip beräknas på
samma sätt som för motsvarande sammankoppling av resistorer.

Galvaniskt seriekopplade induktorer
Förutsatt att magnetfälten från de respektive
induktorerna inte återverkar på varandra d.v.s. inte "kopplar magnetiskt till varandra"
-så gäller:

L= L1 + L2 + L3 + . . . Ln
Räkneexempel:
L1 = 20 mH L2 =50 mH L= ?
L = L 1 + L2 = 20 + 50 = 70 mH

Galvaniskt parallellkopplade induktorer
Förutsatt att magnetfälten från de respektive
induktorerna inte återverkar på varandrad.v.s. inte "kopplar magnetiskt till varandra"
-så gäller:
1
1
1
1
- = - + - + ... -

L

L1

L2

Ln

För två induktorer gäller:
1
:! = ! + eller
L = L; · L2
L L1 L2
L1 +L2
Räkneexempel:
L1 = 50 mH L2 = 60 mH L = ?

Medverkande magnetfält

L1

-

L2

~:~-'Olflf;Motverkande magnetfält

L1

L2

~~M~

-

Bild 113-7 Magnetiskt kopplade induktorer
Bild 113-7
Bilden visar seriekopplade induktorer, vars
magnetfält kopplar till varandra på olika sätt.
"Pricken" vid änden av induktorerna på
bilden markeraratt magnetfälten där har inbördes polarisering.

Magnetiskt kopplade induktorer i serie
Formel:
L=L 1 +L2 \(\pm\)2M
Räkneexempel:
Två induktorer har en impedans av 20 resp.
1O J..LH och en ömsesidig induktans av 2 JlH.
Induktorerna är kopplade och placerade så
att deras magnetfält verkar med varandra.
Vardera induktansen ökas därför med
M=2J..LH

L=L1 +M+L 2 +M

L= L1 • L2 = 50 . 60 = 3000 ~ 27 m H
L1 +L2
50+60
110

L = 20 + 2 + 1O + 2 J..LH = 34 J..LH

Magnetiskt kopplade induktorer

Räkneexempel:
Två induktorer har en impedans av 20 resp
1O !lH och en ömsesidig induktans av 2!-LH.
Induktorerna är kopplade och placerade så
att deras magnetfält verkar mot varandra.
Vardera induktansen minskas därför med
M=2J..LH

l praktiken anordnas ofta induktorer så, att
deras respektive magnetfält kan återverka
på varandra- s.k. magnetisk koppling.
En ömsesidig induktans M uppstår i
induktorerna på grund av denna koppling.
Den ömsesidiga induktansen ökar eller minskar det resulterande induktansvärdet beroende på om induktorernas magnetfältverkar
med eller mot varandra.
Beräkningen av värdet på "M" är emellertid relativt komplicerad och behandlas ej här.
l stället görs en förenklad framställning.
113-4

L= L1 -M+L 2 -M

L = 20 - 2 + 1O - 2

= 26 JlH

Magnetiskt kopplade induktorer i parallell
Formel:
L= L; ·L2 ·M 2
L1 +L 2 \(\pm\)2M

~©rNJ

EPT

KRETSAR

Upp- och urladdning av en kondensator

Uppladdning

Bild 113-8
En kondensator C seriekopplas med en resistans R och kopplas in över spänningen U.
Spänningen över kondensatorn stigerfrån
volt till umax
Laddningsströmmen sjunker från lmax till
noll ampere.

u

Uc

o

Spänningen över kondensatorn ökar
exponentiellt uppladdningen.

(1- e-~ J

uC = Umax ·

e

spänningen över kondensatorn efter
en given inkopplingstid
slutspänningen efter minst t= 5-r
inkopplingstiden
2. 718 (e = basen för den naturliga
logaritmen)

l förloppet ingår storleken av resistans
och kapacitans enligt följande samband, som
kallas tidskonstant
r= R· C
C [F] R [Q] s [sek] -r [tidskonstant i sek]
Efter tiden t= 1r från inkopplingsögonblicket har spänningen över kondensatorn
ökat från noll till 63°/o av maxvärdet
Efter tiden t= 5-r är kondensatorn uppladdad till 99 o/o.

Strömmen från kondensatorn minskar

exponentiellt under uppladdningen.

ic
lmax

strömmen från kondensatorn efter en
given inkopplingstid
begynnelseströmmen

Efter tiden t= 1r från inkopplingsögonblicket har strömmen till kondensatorn
minskat till 37°/o av maxvärdet
Efter tiden t= 5 r återstår 1 °/o av strömmens maxvärde.

Uc
100%---

le

U max

-~------:-:::::;..:;;:;;;-~---

o

2T

l max
100% - - - - - - - - - - - - - - - -

o
Bild If 3-8 Uppladdning av en kandensa to
Urladdning

Bild 113-9
En kondensator C urladdas över resistor R.

Spänningen över kondensatorn minskar
exponentiellt under urladdningen.
t

uC =Umax ·e--:r
Strömmen från kondensatorn minskar

exponentiellt under urladdningen. Strömriktningen är motsatt den vid uppladdningen.
t

iC =-lmax · e--:r
Efter tiden t= 1r är kondensatorn urladdad Så, att 37 °/o av lmax respektive umax
återstår.
Efter tiden t= 5 r är kondensatorn urladdad så, att mindre än i o/o av lmax respektive
umax återstår.

113-5

KRETSAR
Exempel på beräkning av tidskonstanten:
R = 1 kQ
1. C = 1O J.lF

r= R. c = 1·1 0 3 ·1 o·1 o-e = 1o·1 o-3

d.v.s. 1/100 sekund

2. C

= 1000 J.lF

1: =R·

In- och urkoppling av en induktor
Inkoppling
Bild II 3-1 O
En induktor L i serie med en resistans R
kopplas in över en likspänning U.
Spänningen över induktorn ökar från O till

u max·

R = 1 kQ

C= 10 3 ·1 0 3 ·1 o-e = 1 sekund

(Egentligen, induktorns motspänning
minskar så att. .. )
Strömmen genom induktorn ökar från O
till umax·

Strömmen genom induktorn ökar exponentiellt efter inkopplingen

iL

=lmax ·

(1- e-~ J

iL strömmen efter en given inkopplingstid
lmax slutströmmen efter minst t= 5-r

t

e

inkopplingstiden
2.718 (e = basen för den naturliga
logaritmen)

l förloppet ingår storleken av resistans
och induktans enligt följande samband, som
kallas tidskonstant

u

L
R

1:=-

L [H]

Uc

U max

le
Bild II 3-9 Urladdning av en kondensator

113-6

R [O]

s [sek]

1:

[tidskonstant]

Efter en tid av t= 11: från inkopplingsögonblicket har strömmen genom induktorn
ökat från noll till 63°/o av lmax och motspänningen över induktorn minskat till 37o/o
av maxvärdet

Urkoppling
Spänningskällan kopplas bort från samma
induktor som ovan. En resister är inkopplad
över induktorn. Energin i induktorn avleds
genom resistorn som en ström med motsatt
riktning än vid inkopplingen. Strömmen är
vid urkopplingstillfället lmax = iL och minskar
därefter exponentiellt.

ETSAR
iL
lmax

e

t

strömmen genom induktorn efter en
given urkopplingstid
strömmen i urkopplingsögonblicket

2.718

tiden efter urkopplingsögonblicket

Efter en tid av t= 1r från urkopplingsögonblicket har strömmen genom induktorn
minskat till 37°/o av maxvärdet
Teoretiskt kan spänningarna och strömmarna aldrig nå ett noll- eller maxvärd e, men
för praktiskt bruk anses detta inträffa efter en
tid av minst 61:.
All den energi som lagras i en induktor
finns i dess magnetfält. När strömmen bryts
eller minskas så återgår energin omedelbart
till kretsen. i en induktor kan det således inte
finnas någon kvarstående energi, vilket det
däremot kan göra i en kondensator.

Under den tid som magnetfältet i en induktor avvecklas eller byggs upp, så induceras en motspänning i den. Denna spänning
är högre än den som finns över induktorn
innan strömmen bryts eller ändras och är
proportionell till den hastighet som ändringen har. När en en strömkrets med induktor
bryts är det vanligt att det i brytögonblicket
bildas en gnista eller ljusbåge över brytarens
kontakter.
Om induktansen är stor och kretsströmmen hög, så skall en stor mängd energi
frigöras på mycket kort tid. Det är därför inte
ovanligt att brytarkontakter bränns eller smälter. l likströmskretsar kan gnistan eller ljusbågen minskas eller undertryckas genom att
en kondensator i serie med en resistor kopplas över kontaktstället Kondensatorn fångar
upp en del av energin i induktorn och resistorn minskar hastighetsändringen.

l~
L

u
'------{ /

1-----4-----l

lL

l Max

37:'

47:'

10~0--------------~~.

'(

Bild II 3-1 O Inkoppling av en induktor

113-7

KRETSAR

PT

Växelströmskretsar
Komponentegenskaper vid växelström
Inom radiotekniken används mycket ofta
svängningskretsar bestående av kondensatorer och induktorer, som är kopplade i serie
eller parallellt med varandra. När svängningskretsens egenfrekvens sätts lika med
frekvensen på den signal som tillförs kretsen, så får kretsen särskilda egenskaper
som används på olika sätt.
För att förstå hur "LC-kretsar" fungerar,
beskrivs först hur de ingående komponenternas resistans, induktans och kapacitans
förhåller sig till varandra, när de kombineras
och kopplas til en växelströmkälla.
Bild II 3-11
Bilden visar amplituden av spänning och
ström vid ett sinusformat förlopp samt den
effekt som då utvecklas. Tidsaxeln är graderad O - 360° per period.

Fall a: Förloppen med en resister R.

Med en resister följer ström- och spänningskurvorna varandra tidsmässigt, även
vid riktningsändring. När kurvorna följs åt på
det sättet, sägs de vara i fas med varandra.
Effekt överförs från strömkällan till resistorn. Den effekt som utvecklas i resistorn är,
vid varje tidpunkt av perioden, produkten av
strömmmen och spänningen just då. Eftersom storheterna av spänning och ström är
antingen positiva eller negativa samtidigt, så
blir produkten alltid positiv. Det betyder att
den effekt som utvecklas pulserar två gånger per period mellan ett noll- och maxvärde.
Fall b: Förloppen med en induktor L.

Med en induktor är utvecklingen av ström
och spänning inte samtidig. Vid inkopplingen stiger spänningen genast till maxvärdet
medan strömmen stiger långsammare och
bygger under tiden upp ett magnetfält i induktorn och omkring övriga ledare i kretsen.

u~
a

u~

p

b

u~
c
Bild II 3-11 Faslägen och effekter i L C-kretsar

113-8

p

ETSAR
Strömmen fördröjs alltså i förhållande till
spänningen. Eftersom kurvornas max- och
nollvärden inträffar vid olika tidpunkter, så
heter det att de är ur fas eller fas förskjutna.
En växelström genom en ideal induktor
ärtasförskjuten 90° efterspänn ingen. Strömmen når toppvärdet vid tidpunkten 90° av
perioden, när spänningen nått ner till noll.
När spänningen minskar, så sjunker strömmen och tar med sig energin i magnetfältet.
Först vid 180°, när spänningen har nått maxvärdet åt andra hållet, ändrar också strömmen riktning och bygger upp ett nytt magnetfält med motsatt polaritet.
Effekt överförs från strömkällan till induktorn när ström och spänning har samma riktning. När ström och spänning har olika riktning, försöker induktorn i stället "ladda" strömkällan med energi från sitt kraftfält. Det pendlar därför effekt mellan strömkällan och induktorn, varvid effekten i ena riktningen är
lika stor som i andra riktningen.
Sett över en hel period upphäver därför
dessa effekter varandra. Följden blir att en
ideal induktor, i motsats till en resistor, inte
förbrukar någon aktiv effekt. Man säger att
en reaktans, här en induktor, arbetar med
reaktiv effekt.
l praktiken har kretsen även en viss resistans. Därför sätts reaktansens 90° fasförskjutna ström samman med resistansens oo
fasförskjutna ström. Resultatet blir en ström,
som är mindre än 90° ur fas, och det förbrukas då en viss aktiv effekt i resistansen.

Sedan strömmen passerat noll vid 180° eller
0°/360°, bygger den upp ett nytt magnetfält
med motsatt polaritet.
Liksom med en induktor överförs effekt
från strömkällan till kondensatorn när ström
och spänning har samma riktning. När ström
och spänning har olika riktning, försöker
kondensatorn i stället "ladda" strömkällan
med energi. Det pendlar därför effekt mellan
strömkällan och kondensatorn, varvid effekten i ena riktningen är lika stor som i andra
riktningen.
Sett över en hel period upphäver därför
dessa effekter varandra. Följden blir att en
ideal kondensator, i motsats till en resistor,
inte förbrukar någon aktiv effekt. Man säger
då, att en reaktans, här en kondensator,
arbetar med reaktiv effekt.
l praktiken har kretsen även en viss resistans. Därför sätts reaktansens 90° fasförskjutna spänning samman med resistansens o fasförskjutna ström. Resultatet blir en
spänning, som är mindre än 90° ur fas, och
det förbrukas då en viss aktiv effekt i resistansen. Som framgår av bilden blir variationerna i tiden de omvända med kondensator
jämfört med induktor.

o

Fall c: Förloppen med en kondensator C.
Inte heller med en kondensator utvecklas
ström och spänning samtidig. Efter inkopplingen laddar strömmen upp kondensatorn,
d.v.s. bygger upp ett elektriskt fält med en
viss potential (spänning). Spänningen utvecklas långsammare än strömmen - den
blir fasförskjuten
Strömmen till (och från) en ideal kondensator är fasförskjuten 90° före spänningen.
När kondensatorn är kopplad till en växelströmskälla, når strömmen toppvärdet vid
tidpunkten 90° eller 270° av perioden. Spänningen passerar då i båda fallen värdet noll.
När spänningen minskar, så sjunker strömmen och tar energi ur det elektriska fältet.

113-9

KRETSAR
Impedans

Bild II 3-12
Bilden visar en induktor, en kondensator
och en resistor som är kopplade i serie. När
man vill beräkna den resulterande impedansen i kretsen ("totala växelströmsmotståndet"), måste man ta hänsyn till att komponenternas spänningar eller strömmar inte är
i fas med varandra. De arbetar ju inte "i takt".
Att då addera max. värdena ger fel resultat. l stället söker man den s.k. resultanten
av de olika vektorer som motsvarar strömoch spänningsvärden.
Detta kan göras grafiskt eller beräknas.

Liten ordlista:
Impedans- hindra
(lat. impedire).
Resistans- motstå
(lat. resistere).
Del av impedansen,
kallas ibland ohmskt motstånd.
Reaktans- återverka
(lat. reagere).
Del av impedansen,
samlingsord för växelströmsmotstånd.
- Kapacitans- inrymma (lat. capax).
Del av reaktansen.
- Induktans- införa
(lat inducere).
Del av reaktansen.

Bild 113-13
Vi tänker oss att vektorerna i systemet
vrider sig moturs med hastigheten w= 2rc f,
där f är frekvensen och w ärvinkelhastighet.
Eftersom vektorerna har samma frekvens,
så är vektorernas lägen inbördes samma.
Ögonblicksvärdet av respektive vektorer följer en sinuskurva.
Spänningsvektorn i den "induktiva reaktansen" ligger 90° före strömmen och spänningen i resistansen. Spänningsvektorn i
den "kapacitiva reaktansen" ligger 90° efter
strömmen och spänningen i resistansen.
Vektorerna i dessa två reaktanser är således 2 · 90 = 180° åtskilda, d.v.s. motriktade.
Det kallas att de är i mottas.

Hittills har storheterna resistans, induktans
och kapacitans behandlats var för sig, men
i praktiken förekommer de alltid tillsammans
och kallas impedans.
Resistansen är i princip oförändrad vid
ström- eller spänningsändringar. Men när
strömmen genom en ledare eller induktor
liksom spänningen över en kondensator ändras, så tillkommer en reaktans som motverkar förändringarna.
Reaktansen kan från fall till fall vara kapacitiv eller induktiv och ingår i impedansen.
Om ingen reaktans finns, så är impedansen
lika med resistansen.

Bild II 3-14
l bilden visas vektorerna för komponenterna i Bild II 3-12 samt hur man grafiskt
bestämmer inpedansen av dessa vektorer.
Vidare får man fasvinkeln mellan impedansens och resistansens vektor, varav den senare är den s.k. riktfasen för hela seriekretsen.

Bild II 3-12 Seriekrets av L+C+R

'

spänningsfall över R
= strammens fas 1 XL+ Xc +R

,spänningsfall over Xc.
spännir1gsfoll över XL
1

Bild II 3-13 Spänningar i seriekrets L+C+R

113-10

KRETSAR

Bild II 3-14 Impedansen och fasvinkeln i seriekrets L+C+R
Resistansen ritas som en vektor R, som
riktas vågrätt mot höger. Vektorns längd
motsvarar resistansens storhet i ohm.
Den induktivareaktansen ritas på liknande sätt med vektorn XL lodrätt uppåt. slutligen ritas den kapacitiva reaktansen Xc lodrätt neråt.
Man subtraherar de motverkande reaktiva vektorerna XL och Xc från varandra och
avsätter resultatet X på den vertikala axeln,
uppåt om XL är större och neråt om Xc är
större. Den resistiva vektorn R avsätts åt
höger på den horisontella axeln.
Man låter nu vektorerna X och R bilda
sidor i en rätvinklig rektangel. Längden på
rektangelns diagonal är den resulterande
impedansen Z. Fasvinkeln mellan impedans
och resistans kan också avläsas.
Eftersom vektordiagrammet bildar en rätvinklig triangel kan den resulterande spänningen U i kretsen även beräknas med
Pytagoras sats:

Tillämpad på ovanstående vektordiagram
kan satsen skrivas som

ULcR2

Termerna ersätts med följande ekvationer:

UR

= f. R

UL =J. XL =J. mL

1

Uc=f·Xc=l·-

mC

l' Z

2

=l

2

2

R + (hoL -l miG)'

eller

Z= ~R

'+(mL--dc;J eller

Z=~R

2
+(XL -Xct

l en seriekrets är den resulterande
reaktansen negativ (kapacitiv) om Xc är
större än XL och positiv (induktiv) om XL är
större än Xc.

Ohms lag vid växelström

l formler betecknas impedansen med bokstaven Z och reaktansen med bokstaven X.
l båda fallen är sorten Ohm [Q].
Vid beräkning av impedans är Ohms lag
inte direkt tillämplig, eftersom reaktansen i
en induktor eller kondensator uppträder annorlunda i tiden vid ström- respektive spänningsändring än vad resistansen gör.
Om impedansen Z sätts in i Ohms lag, så
fås följande samband som ofta kallas Ohms
lag för växelström, således

vett =lett.

= UR2 + (UL- Uc)2

ULRc =J. Z

z =R'+( mL- ~c)'

z

eller

Uett = lett. -.J R2 + )(2

eller

Uett=lett·~R2 +(XL -Xc) 2

o.s.v.

Av vad som framgått tidigare i detta avsnitt
kan även slutsatsen dras att:

2

Efter division med 12 fås

skenbar effekt=
= ~ (aktiv effekt) 2 +(reaktiv effekt) 2
113- 11

KR
LC- kretsar
Parallellkopplade LC-kretsar
Bild II 3-15
En parallellkopplad LC-krets är ansluten till
växelspänningen U från en signalgenerator
med inställbar frekvens f. Två fall studeras.
Fall i :

f=

fres

signalgeneratorns frekvens f ställs lika
med LC-kretsens resonansfrekvens fres· Då
visar kretsen hög impedans Z mot generatorn. En stark ström cirkulerar i svängningskretsen, men endast en svag ström flyter i
ledningen mellan generator och krets.
Jämför med modellförsöket på bild 3-000.
Fall2:

f> ~es

eller

f< ~es

Frekvensen f ställs högre eller lägre än
kretsens resonansfrekvens fres·
Svängningskretsen visar då en låg impedans Z mot generatorn. En svag ström cirkulerar i svängningskretsen, medan en starkare ström flyter i ledningen mellan generator och krets.
l praktiken finns även en resistans (belastning) parallellt över kretsen och en resistans i serie med induktansen. För enkelhetens skull bortses här från dessa resistanser.
l en parallellkopplad LC-krets är spänningen över induktans och kapacitans densamma. Spänningsvektorn U används därför som s.k. riktfas.

ström i XL

Riktfasen riktas på bilden åt höger. Strömmen le genom kondensatorn är fasförskjuten 90° efter U och ritas rakt neråt (vektorerna roterar motsols). Strömmen IL genom
induktorn är fasförskjuten 90° före U och
ritas rakt uppåt. Den resulterande reaktiva
strömmen genom kretsen är skillnaden mellan strömmarna le och IL, vilka är motriktade
varandra.
Formeln förparallellkopplade resistanser
kan även användas för parallellkopplade
re aktanser om man tillämpar Pytagoras sats
[A2 + 82 =
således

c2],

i
R

i
R1

1

-=-+-+ ....
R2

Gr ~(~r +(*J
~~ (~J +GJ ~~~, +;
eller

Med R försumbart kan den resulterande
reaktansen av kapacitansen Xe och den
vektormässigt motriktade induktansen XL beräknas på följande sätt
1
1
1
1 XL- X c
X= X c -XL d.v.s. X= -XL. X c eller

X= -XL ·Xc
XL -Xc
l en parallellkopplad LC-krets är den resulterande reaktansen negativ (kapacitiv) om XL
är större än Xe och positiv (induktiv) om XL är
mindre än Xe.

ström i XL

ström i Xc

l "'~~l

u

spänning

1--------IIB»

ström i Xc

Bild II 3-15 Parallellkopplad LC-krets

113- 12

u

Seriekopplade LO-kretsar

Bild II 3-16
En seriekopplad LC-krets ansluts till växelspänningen U från en signalgenerator med
inställbar frekvens f. Två fall studeras.
Fall 1:

f = f,es

signalgeneratorns frekvens f ställs lika
med svängningskretsens resonansfrekvens
fres· Impedansen Z i en seriekrets visar då ett
mycket lågt värde mot generatorn. Det flyter
en stark ström i ledningen mellan generator
och krets.
Fall2:

f< f,es eller f> f,es

Frekvensen f ställs lägre eller högre än
kretsens resonansfrekvens fres·
Eftersom svängningskretsen då visar hög
impedans Z mot generatorn, så flyter endast
en svag ström i ledningen mellan generator
och krets.
l praktiken finns även en resistans i serie
med induktansen liksom en parallellt över
kapacitansen. För enkelhetens skull bortses
här från dessa resistanse r.
Strömmen l är samma genom hela kretsen och strömvektorn l används därför som
s.k. riktfas. Den ritas i bilden åt höger. Om
serieresistansen R varit med, så skulle ett
spänningsfall UR varit inritad i samma riktning som l (i fas med 1). Spänningen över
re aktansen Xc ligger go o efter l och ritad rakt
neråt (vektorerna roterar motsols). Spänningen över reaktansen XL (induktorn) ligger
90° före l och ritad rakt uppåt.

Thomson's svängningskrets
Bild II 3-17
Bilden visar en svängningskrets, som består
av en kondensator och en induktor med
förskjutbar järn kärna. En ändring av kärnans
tvärsnitt ändrar den magnetiska ledningsförmågan och därmed induktansen.
Med anordningen kan resonansfrekvensen alltså ställas in så att den blir högre, lika
med eller lägre än den anslutnaspänningens
frekvens. Tre fall undersöks:
XL > X c LA 1 och LA2 lyser upp, en kraftig
ström flyter genom kondensatorn,
XL < X c LA 1 och LA3 lyser upp, en kraftig
ström flyter genom induktorn,
XL= Xc LA2 och LA3 lyser upp, LA1 lyser
inte, en kraftig ström flyter i kretsen
men inte i tilledningarna
XL = X c kallas Thomson's svängningsformel, vilken beskriver resonansfallet
Då är de induktiva och kapacitiva
reaktanserna i kretsen lika stora och tar ut
varandra. Kvar är kretsens resistans, vilken
vi tills vidare betraktar som försumbar.
Således XL =Xc , där

XL

= 2 nfL
1

och

1 - sats
"t .tn.
Xc = 2nfC

2nfL=-2nfC

4n 2 f ·L· C= i

f-=1

f=

4n 2 LC

f [Hertz]

L [Henry]

1
C [Farad]

Formeln gäller både för parallell- och seriekretsar.

Bild II 3-16 Seriekopplad LC-krets

113-13

KRETSAR
Räkneexempel:
Strömriktning: 1 halvvågen -.....
2 halvvågen - - -

L= 100 nH C= 1O pF

f=

f=?

1
2n~1 00 ·1 o-9 ·1 o·1 o-12 = 2n1 o-9 =

109

= - ~ 159 MHz

2n

Impedansen i en resonant krets
En enkel framställning görs av hur impedans, re aktans och resistans förhåller sig
inbördes när en svängningskrets är i resonans. Som exempel används följande
kretsdata: Induktans 200 J.lH, kapacitans
200 p F, förlustresistans 1O Q.
Resonansfallet i en parallellkrets
Parallellkretsen består i sig själv av seriekopplade komponenter, varav XL och Xc
är reaktiva. Vid resonans är dessa lika
stora och motverkande. Inom kretsen är
således den resulterande reaktansen

Därför uppvisar samma krets en yttre
reaktans av

X= -XL ·Xc
XL -Xc

X= XL ·Xc
O

=oo

l praktiken finns i kretsen också en
resistans varför dessa extremvärden inte
uppstår. Inne i en parallellkrets i resonans cirkulerar alltså en stark ström, som
endast begränsas av kretsens resistans.
Bild II 3-18
Bilden visar en parallellkrets där
induktorn har resistansen rL och kondensatorn antas vara förlustfri. Vidare
förutsätts att kretsen är i resonans.
Vid resonans kan termen XL -Xc = O
bytas mot rL i formeln

X= -XL ·Xc
XL -Xc
Bild II 3-17 Thomson's svängningskrets

113-14

förutsatt att rL är försumbart jämfört med
XL.

ETSAR
Därtill är XL= 2nfL och Xc = ~fC
...
L
d .v.s. X L· X c= C som satts m.

2

Parallellkretsens impedans vid resonans kan
då skrivas
Z= XL ·Xc =-LIj
tj·C

Med ovanstående kretsdata blir Z= 100 kQ
Därav framgår, att impedansen i parallellkretsen är en funktion av det s.k. UC-förhållandet samt av kretsens resistiva förluster.

.
Z vtd

resonans:

XL~ Xc
l
--=-c
rL
rL·

l formeln

Z=~r/+(XL -Xc)
Z=~r/ +0 2

blirdå

=!j

Med ovanstående kretsdata blir resonansfrekvensen

1

~ 796kHz
2n LC
Vid resonansfrekvensen blir reaktansen
1000 Q både i induktansen och kapacitansen. Eftersom reaktansernas spänningsfall
är motriktade tar de ut varandra. Kretsens
impedans i resonans blir resistansen rL och
spänningsfallet över kretsen bestäms enbart av rL.

fo =

{[C

Antag att det alstras en spänning av 5 mV
i antennkretsen. Strömmen genom den vid
5
resonans blir då
m V= O. 5 mA.
10 Q
Av strömmen bildas reaktiva spänningar,
d.v.s. 0.5 mA·1 000 Q == 500 mV både över
induktans och kapacitans (som tar ut varandra) och 5 mV över resistansen.

Z vid
resonansfrekvens

2

resonans:

rL

ohm

Bild II 3-18 Resonansfallet i parallellkrets
Resonansfallet i en seriekrets
Bild II 3-19
När en seriekrets är i resonans, så är

XL =Xc

Serieresonans

i
d.v.s. mL = -

mC

eller

resonansfrekvens

frekvens

1

d.v.s. mL--=0

mC

Bild II 3-19 Resonansfallet i seriekrets

113-15

KR
Q-faktorn i en parallellkrets

Bild II 3-20
Godhetstalet Q (=Quality Facto r) kan ses
som den förmåga en svängningskrets har att
lagra energi, d.v.s. förhållandet mellan den
lagrade energin och energiförlusten i kretsen. Energiförlusten yttrar sig som värmeutveckling.

Q= n
2

lagrad energi i kretsen
energiförlusten per period
Energiförluster uppstår både i kretsens
kondensator och induktor, men moderna
kondensatorer har så låga förluster att
induktorn ensam kan anses bestämma Qvärdet, åtminstone i kortvågsområdet

Bandbredd

Bild II 3-21
Bilden visar med en kurva vilket impedansvärde kretsen har vid olika frekvenser.
Impedansens högsta värde är vid frekvensen fres och avtar vid frekvenser som är högre
eller lägre. Vid frekvenserna f1 och f2 är
impedansvärdet t. ex. 70°/o av maximalvärdet
Med bandbredden b förstås skillnaden mellan impedansvärdena i ett sådant frekvenspar, d.v.s. b= ~ - ~

z

Q= 2nfL =XL

R

R

En växelspänning U 1 ansluts till en
parallellkrets. l resonansfallet uppträder då
en spänning U2 över kondensatorn och
induktorn.
U2 är mycket större än U1 • Ju högre Q är
i kretsen desto större är förhållandet mellan
U2 och U1 •
l kortvågsområdet är det vanligt med ett
Q i storleksordningen 30 - 100.
Ju högre Q är, desto mindre är bandbredden.
När svängningskretsen är i resonans gäller sambandet

Q= ~es

b

Bandbredden ökar (avstämningsskärpan
minskar) vid ökande frekvens på grund av de
större kretsförlusterna.

u

f res
Bild II 3-20 Q-värden i parallellkrets
113- 16

f.

f
Bild 113-21 Bandbredd i parallellkrets

