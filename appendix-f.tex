\chapter{IARU Region 1 bandplan}

Sammanfattad av SM3AVQ Lars

Denna bandplan reviderades vid lA RU Region i -konferensen i Tel-Aviv 1996.
Den vänstra delen är själva bandplanen, medan den högra delen rekommenderar användning/mötespunkter.
(PTS bandplan och status för amatörradio i Sverige, framgår av Kapitel III 1.6 samt Appendix G och H.)

Band
MHz

Segment
kHz

Trafiksätt

1.8

i 81 o - 1838

CW enbart
Digitala trafiksätt men ej Packet Radio, CW
Digitala trafiksätt men ej Packet Radio, Telefoni, CW
Telefoni, CW
(i Sverige 1842-1850)

1838- 1840
1840- 1842
1842-2000

o

3.5

3500- 351
3500-3560
3560-3580
3580-3590
3590-3600
3600-3620
3600-3650
3650-3775
3700-3800
3730-3740
3775-3800

CW enbart, DX-fönster för interkontinentala kontakter
CW enbart, segment för CW-tester
CW enbart
Digitala trafiksätt, CW
Digitala trafiksätt företrädesvis Packet Radio, CW
Telefoni, Digitala trafiksätt, CW
Telefoni, Segment för Telefoni-tester, CW
Telefoni, CW
Telefoni, Segment för Telefoni-tester, CW
SSTV \& FAX, Telefoni, CW
Telefoni, DX-fönster för interkontinentala kontakter

7

7000-7035
7035-7040
7040-7045
7045-7100

CW enbart
Digitala trafiksätt men ej Packet Radio, SSTV \& FAX, CW
Digitala trafiksätt men ej Packet Radio, SSTV \& FAX, Telefoni, CW
Telefoni, CW

10

10100-10140
10140-10150

CW enbart
Digitala trafiksätt men ej Packet Radio, CW

14

14000- 14070
14000 - 14060
14070 - 14089
14089 - 14099
14099-14101
14101-14112
14112 - i 4125
14125-14300
14230
14300 - 14350

CW enbart
CW enbart, Segment för CW-tester
Digitala trafiksätt, CW
Digitala trafiksätt företrädesvis Packet Radio, CW
Exklusivt fyrband IBP
Digitala trafiksätt företrädesvis Packet Radio forwarding, Telefoni, CW
Telefoni, CW
Telefoni, Segment för Telefoni-tester, CW
SSTV \& FAX anropsfrekvens
Telefoni, CW

18

18068 - 181 00
18100-18109
18109- 18111
18111 - i 8168

CW enbart
Digitala trafiksätt, CW
Exklusivt fyrband l BP
Telefoni, CW

21

21 000 - 21 080
21 080 - 21 i 00
21100-21120
21120-21149
21149-21151
21 i 51 -21450
21340
24890 - 24920
24920 - 24929
24929- 24931
24931 - 24990

CW enbart
Digitala trafiksätt, CW
Digitala trafiksätt företrädesvis Packet Radio, CW
CW enbart
Exklusivt fyrband IBP
Telefoni, CW
SSTV \& FAX anropsfrekvens
CW enbart
Digitala trafiksätt, CW
Exklusivt fyrband l BP
Telefoni, CW

24

F-1

APPENDIX
Trafiksätt

Band
MHz

segment
kHz

28

28000 - 28050
CW enbart
28050- 28120
Digitala trafiksätt, CW
28120- 28150
Digitala trafiksätt företrädesvis Packet Radio, CW
28150- 28190
CW enbart
28190-28199
Regionella fyrar med tidsdelningsschema !BP
28199- 28201
Världstäckande fyrnät med tidsdelnings-schema !BP
28201 - 28225
Kontinuerligt sändande fyrar l BP
28225- 29200
Telefoni, CW
28680
SSTV \& FAX anropsfrekvens
29200- 29300
Digitala trafiksätt (NBFM Packet Radio), Telefoni, CW
29300 - 2951 o
satellit utfrekvens (nerlänk)
2951 O- 29700
Telefoni (29 MHz FM-band, se nedan), CW
FM Frekvensuppdelning
2951 O
Del bandkant, används ej
29520 - 29550
FM Simplex
29560 - 29590
Repeater infrekvenser, 1O kHz frekvensavstånd
29600
Anropsfrekvens
2961 O - 29650
FM Simplex
29660 - 29690
Repeater utfrekvenser, 1O kHz frekvensavstånd
29700
Bandkant, används ej

29

ANMÄRKNINGAR
Prioritet:
När flera trafiksätt förekommer på samma frekvenssegment, har det trafiksätt företräde, som här nämns
först. Detta sker dock under vad som kallas Noninterference Basis, NIB (icke störande grundval).
Digitala trafiksätt:
Omfattar BaudoVRTTY, AMTOR, PACTOR, CLOVER,
ASCII, Packet Radio.
Observera undantagen för 1.8, 7 och i OMHz där Packet
Radio ej ingår i Digitala Trafiksätt
Telefoni:
Alla slag av detta trafiksätt inkluderas. Upp till 1O MHz
skalllägre sidbandet (LSB) användas och över 1O MHz
det övre sidbandet (USB).
3500-351 O och 3775-3800 kHz:
Interkontinental trafik skall ges företräde på dessa segment.
segment för tester:
Då DX-trafik ej är involverad skall testsegmenten ej
innefatta 3500- 351 O eller 3775- 3800kHz. Medlemsföreningarna tillåts sätta andra, smalare, segment för
sina nationella tester (inom testsegmenten).
Rekommendationen om testsegment gäller inte tester
med digitala trafiksätt.
Testaktivitet skall ej äga rum på 1O, 18 och 24 MHzbanden.
7 och 10 MHz:
Användande av Packet Radio på 7 och 1OMHz avråds.
7035-7045 får under dygnets ljusa timmar användas av
Packet Radio forwarding-stationer i Afrika söder om
ekvatorn.
1O MHz-bandet:
Vid nödtrafik får även SSB användas på detta band.

F-2

Obemannade stationer som använder digitala trafiksätt
skall undvika att använda 1O MHz- bandet.
Nyhetsbulletiner skall ej sändas på 1O MHz- bandet.
10120- i 0140 kHz får under dygnets ljusa timmar användas av SSB-stioner i Afrika söder om ekvatorn.
SSTV l FAX:
Frekvenserna 14230, 21340 och 28680 bör användas
som anropsfrekvenser för SSTV- och FAX-operatörer.
Efter att kontakt erhållits skall dessa flytta till annan
ledig frekvens inom telefonidelen av bandet.
satellitbandet 29300-2951 O kHz:
Medlemsländerna skall råda amatörerna att inte sända
FM på frekvenser mellan 29300 och 2951 O kHz. Detta
för att undvika interferens med satelliternas nerlänk.
Obemannade sändare:
IARU:s medlemsländer äro uppmanade att begränsa
denna aktivitet på kortvågsbanden.
Obemannade stationer på kortvåg skall endast aktiveras under en operatörs kontroll.
Med detta menas att stationer ej skall aktiveras av
exempelvis ett program i en dator. Aktivering skall ske
av SYSOP eller av en uppropande station efter att de
avlyssnat frekvensen och funnit att den är ledig.
Undantag gäller för fyrar och speciella experimentstationer.
Sändarfrekvenser:
l bandplanen angivna frekvenser är "sändarfrekvenser"
(inte frekvensen för den undertryckta bärvågen).
NBFM Packet Radio på 29 MHz-bandet:
Rekommenderade frekvenser på varje '1 O kHz f.o.m.
2921 Ot.o.m. 29290 kHz. En deviation av plus/minus 2.5
kHz skall användas med max 2.5 kHz modulationsfrekvens.
