\section{Transformatorn}
\textbf{HAREC a.\ref{HAREC.a.2.4}\label{myHAREC.a.2.4}}

Allmänt

En transformator består av en eller flera
lindningar eller spolar av elektriska ledare.
Lindningarna är magnetiskt kopplade till varandra. Det innebär att de är anordnade så,
att ett magnetfält, som alstrats i någon av
lindningarna, även passerar genom övriga
lindningar.
När en växelspänning läggs över en lindning, kallas den primärlindning. l och omkring primärlindningen alstras då ett magnetiskt fält som växlar i takt med spänningen.
Primärfältet passerar även genom övriga
lindningar- sekundärlindningarna-och alstrar där spänningar och strömmar.
Den s.k. kopplingsfaktorn mellan lindningarna varierarförolika frekvenser, sämre
vid låga frekvenser (hundratals Hz) och bättre vid höga frekvenser (tusentals Hz). Speciellt vid låga frekvenser behövs en bättre
koppling för att avsedd effekt skall kunna
överföras mellan lindningarna. Då kan ledningsförmågan i den magnetiska flödesvägen ökas med hjälp av en järnkärna.

Terminologi
primärkrets
sekundärkrets
primärlindning
sekundärlindning
primärspänning u 1 sekundärspänning u2
primärström i1
sekundärström i2
lindningsvarvtal n primärt n1 sekundärt n2
varvtalsomsättning = !i eller n2

n2

impedansomsättning

ni

z rt

= z = ,i
1

2

2

Den ideala (förlustfria) transformatorn
Transformering av spänning och ström
Transformatorn är obelastad när sekundär-

kretsen är bruten.
Bild 112-6
När primärlindningen ansluts till en växelspänning, induceras det växelspänningar
både över primär- och sekundärlindningarna. Det uppstår även en ström i primärlindningen, men däremot inte i sekundärlindningen när sekundärkretsen är bruten. För
den obelastade transformatorn gäller sambandet

!!J..=!i
u2

n2

d.v.s. spänningen över lindningarna är proportionell med lindningsvarvtalet

2
1 Allmänna symboler
2 Transformator med järnkärna

Bild II 2-5 Schemasymboler för
transformatorer

Utföranden

Transformatorn kan utföras för olika ändamål, t.ex. som
spä n n ingstransformator,
strömtransformator eller
impedanstransformator
Utförandet påverkas även av överförd effekt
och frekvens.

Transformatorn är belastad när sekundärkretsen är sluten.
Bild II 2-7
När någon av transformatorns sekundärlindningar ingår i en sluten strömkrets, uppstår en sekundärström där.
sekundärströmmen alstrar ett magnetfält, som motverkar primärströmmens fält,
hindrar dess växlingar och tar ut energi från
primärkretsen.
Strömförbrukningen på primärsidan ökar
således i proportion till strömförbrukningen
på sekundärsidan. Transformatorn reglerar
själv hur mycket energi som den tar från
strömkällan och lagrar i fältet för att föra över
till sekundärkretsen.

112-15

K MP NENTER
För den belastade transformatorn gäller, att
strömmen genom lindningarna är omvänt
proportionell med lindningsvarvtalet

i1 n2
-=-

i2

n1

Av föregående formler följer att:

Bild II 2-6 Obelastad transformator

Bild II 2-7 Belastad transformator

112-16

Av~=

u1 ·i1 och ~ = u1 ·i1 följer att~=~

Om man bortser från förlusterna i transformatorn, så är den effekt som den tar från
kraftkällan lika med den effekt den avger.

Transformatortillämpningar
Sparkopplade transformatorer

Bild II 2-8
Här ovan har transformatorn beskrivits så att
primär- och sekundärlindningarnas enda
förbindelse med varandra är över ett magnetfält, alltså utan galvanisk förbindelse.
Varje lindning kan emellertid förses med
godtyckliga uttag. Över
av
finns då en spänning i proportion till det
lindningsvarvtal som finns mellan uttagen.
Detta är en metod att spara in på antalet
lindningar. För att t.ex. omsätta
ningen 230 V till 115 V används
spartransformator.

Med en spartransformator kommer olika
strömkretsar i galvanisk förbindelse med
varandra och särskild försiktighet skall därför iakttas vid användning av sparkopplade
transformatorer, p.g.a. risken för elolycksfall. Spartransformatorer bör därför inte användas i amatörradiosamman hang. Säkrast
ärtransformatorer med galvaniskt skilda ledningar och dessutom med speciellt bra isolering och kapsling- s.k. skyddstransformatorer.

Bild II 2-8 Sparkopplad transformator

Strömtransformatorer

Hög sekundärström under låg sekundärspänning kännetecknar en strömtransformator.
Strömtransformatorer används i elektriska svetsningsutrustningar, induktionsugnar
o.s.v. Strömtransformatorer används även
för mätning av höga växelströmmar.
Bild 112-9
Bilden visar principen för en induktionsugn, som är en transformator med en sekundärlindning med endast ett fåtal varv omkring en smältdegel.

Högspänningstransformatorer

Hög sekundärspänning under förhållandevis låg sekundärström kännetecknar en
spänningstransformator.

Högspänningstransformatorer används i
distributionsnät, neonskyltar, tändsystem för
förbränningsmotorer, anodspänningsaggregat för sändare o.s.v.
Bild II 2-i O
Bilden visar en transformator med ett gnistgap i sekundärkretsen för tändning av gas.

och klenspänningstransformatorer

2-1 i
Lågspänningstransformatorer används i lokala distributionsnät, vanligen med spänningen 400/230 V. För ökad säkerhet mot
elektrisk chock krävs dock att vissa apparater drivs med en s.k. klenspänning av högst
50 V över en s.k. skyddstransformator med
förstärkt isolering.

112-17

K MP NENTER

PT
h

12

nz

= n1

= 500

220 Vrv

Bild II 2-9 Strömtransformator

Uz
U1 : : 220 V
n1 =500 ~.....-   

-.J

~

Uz:::: 4 400 V

nz =10 000

Bild II 2-1 O Högspänningstransformator

u,

n1

u2

nz

- : : : : - ::::

Uz

n1 ::: 1000 ,   

--.J

n2

u, : : 220 v

1000
so
--20

~U2::::

1

4,4 v

= 20

Bild II 2-11 Klenspänningstransformator

Sambandet mellan varvtal och impedans

Transformatorn kan även användas för anpassning av impedanser. Impedansen Z i en
lindning är proportionell mot kvadraten av
dess lindningsvarvtal n.

112- 18

Om effekten i sekundärlindningen är lika
stor som i primärlindningen, gäller formeln

zp

n/

zs

ns

-=-2
