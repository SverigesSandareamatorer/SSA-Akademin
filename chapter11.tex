\chapter{Nationella och internationella trafikbestämmelse och procedurer}

\section{Fonetiska alfabeten}
Ibland behöver man göra förtydliganden genom att bokstavera.
Svenska radioamatörer skall kunna två fonetiska alfabeten.

Det internationella fonetiska alfabetet

A

svenska

alfabetet

ö

Alfa
Bravo
Charlie
Delta
Echo
Foxtrot
Golf
Hotel
lndia
Juliett
Kilo
Lima
Mike
November
Oscar
Papa
Quebec
Romeo
Sierra
Tango
Uniform
Victor
Wiskey
X-ray
Yankee
Zulu
Alfa Alfa
Alfa Echo
Oscar Echo

ALL FA
BRA VO
TJAR Ll
DELL TA
ECK Å
FACKS
GÅLF
HÅ
IN DIA
DJO U
Kl LÅ
U MA
MAJK
NO VEM BÖ(RR)
ÅSSK A(RR)
PA PA
KE BECK
RÅ MIO
Sl ERR RA
TÄNG GÅ
JO NI FORM
VICK TÖ(RR)
OISS Kl
ECKS REJ
JÄNG Kl
LO
FA ALL
ALL FA ECK
ÅSSK A -

A
B
C
D
E
F
G
H
l
J
K
L
M
N
O
P
Q
R
S
T
U
V
W
X
Y
Z
Å
Ä

ö

Adam
Bertil
Cesar
David
Erik
Filip
Gustav
Helge
Ivar
Johan
Kalle
Ludvig
Martin
Niklas
Olof
Petter
Quintus
Rudolf
Sigurd
Tore
Urban
Viktor
Wilhelm
Xerxes
Yngve
Zäta
Åke
Ärlig
Östen

O
1
2
3
4
5
6
7
8
9

Zero
One
Two
Three
Four
Five
Six
Seven
Eight
Nine

ZE RO
O ANN
TO
TRI
FÅR
FAJV
SICKS
SE VEN
EJT
NAJ NÖ(RR)

O
1
2
3
4
5
6
7
8
9

Nolla
Ett (inte ETTA)
Tvåa
Trea
Fyra
Femma
Sexa
Sju (inte SJUA)
Åtta
Nia

B

C
D
E
F
G

H
l

J
K

L

M

N
O

P
Q
R
S

T
U
V

W
X
Y
Z
Å

Ä

Decimal
DE Sl MAL
Stop
STOPP
Ungefärligt uttal. Betona det understrukna.
1111 - 2

-

Komma
Punkt
Frågor kan förekomma i reglementsprovet.

R CH TRAFIKM
\section{Q-koden}
Bakgrund
Vid sändning med morsetelegrafi används
sedan år 1912 internationella "trafikförkortningar" enligt Q-koden, både för att minska
risken för mottagningsfel på grund av språksvårigheter, störningar m.m. och för att minska sändningstiden. En trafikförkortning i form
av Q-kod har en entydig innebörd, men kan
anpassas något till aktuell situation. Varje Qkod består av tre bokstäver i bokstavsserien
QAA- QZZ.

l CEPT-rekommendation T/R 61-02
nämns följande allmänna Q-förkortningar
som berör amatörradio.
Radioamatörerna använder emellertid i praktiken fler Q-förkortningar än
dessa. En lista kan beställas från SSA:s
kansiL

i reglamentsprovet för radioamatörcertifikat ingår frågor om Q-förkortningar.

Användning
1. Vissa Q-koder kan ges jakande betydelse
genom att bokstaven C (vid telefoni uttalad som CHAR U E) sänds omedelbart
efter förkortningen eller ges nekande betydelse med det engelska ordet NO omedelbart efter förkortningen.
2. Q-koder kan kompletteras med andra
lämpliga förkortningar, anropssignaler,
frekvenser, tidsuppgifter, person- och
ortsnamn, siffror, nummer o.s.v. l den beskrivande texten för vissa Q-koder lämnas inom en parentes plats för ytterligare
uppgifter. Dessa uppgifter skall då sändas i den ordning som anges i texten.
3. Q-koderna antar formen av fråga, då de
vid radiotelegrafering åtföljs av frågetecken liksom då de vid radiotelefonering
åtföljs av bokstäverna RQ (ROMEO
QUEBEC). När kompletterande uppgifter följer efter en uttalad fråga, skall ett
frågetecken respektive RQ följa efter
uppgifterna.
4. Q-koder med numrerade alternativa betydelser skall åtföljas av motsvarande
siffra. Siffran skall sändas omedelbart
efter förkortningen.
5. l internationell radiotrafik skall, då ej annat anges, tidpunkter anges i Universal
Time Coordinate (UTC) i stf. det tidigare
Greenwich Mean Time (GMT). Tidsformatet är fyra siffror, vilket även är militär
standard.

Q-kod

Fråga

Svar eller meddelande

QRK

Vilken uppfattbarhet har mina
(eller: ............. *:s) signaler?

Uppfattbarheten hos Dina
(eller ......... *:s) signaler är ... .
1. dålig
2. bristfällig
3. ganska god
4.god
5. utmärkt.

QRM

Är min sändning störd?

Störningarna på Din sändning är
1. obefintliga
2.svaga
3. måttliga
4. starka
5. mycket starka.

1111-3

R
QRN

Besväras Du av atmosfäriska
störningar?

Atmosfäriska störningar är
1. obefintliga
2.svaga
3. måttliga
4. starka
5. mycket starka.

QRO

Skall jag öka sändningseffekten?

Öka sändningseffekten.

QRP

Skall jag minska sändningseffekten?

Minska sändningseffekten.

QRS

Skall jag minska sändningshastig heten?

Minska sändningshastigheten
(sänd ...... ord i minuten).

QRT

Skall jag avbryta sändningen?

Avbryt sändningen.

(QRU) Har Ni något till mig?

Jag har inget till Dig.

QRV

Är Du redo?

Jag är redo.

QRX

När anropar Du mig igen?

Jag anropar Dig igen kl ... på ... kHz/MHz.

QRZ

Vem anropar mig?

Du anropas av ......... * (på ....... kHz/MHz).

(QSA)

Vilken styrka har mina
(eller: .... *:s) signaler?

Dina (eller: ...... *:s) signaler är
1. knappast uppfattbara
2.svaga
3. ganska starka
4. starka
5. mycket starka.

QSB

Varierar min signalstyrka?

Din signalstyrka varierar.

QSL

Kan Du ge mig kvittens?

Jag kvitterar.

QSO

Kan
få förbindelse med
.... * direkt?

Jag kan få förbindelse med .... * direkt.

QSY

Skall jag gå över till annan frekvens? Gå över till annan frekvens.

(QTC)

Hur många telegram har du att
sända?

Jag har telegram till Dig.

QTH

Vilket är Ditt geografiska läge?

Mitt geografiska läge är ....... .

QTR

Kan Du ge mig rätt tid?

Rätt tid är ........

* namn och /eller anropssignal

1111 - 4

\section{Trafikförkortningar, vanliga i amatörradio}
Utöver Q-koden och klartext används vid
morsetelegrafering även andra trafikförkortningar. Eftersom det internationella radiospråket är engelska, är förkortningar av engelska ord vanligast.
Förkortningar bör emellertid inte användas i onödan. En ovan operatör vid motstationen kan då få svårt att förstå meddelandet.

Urval för radioamatörer
l CEPT-rekommendation T/R 61-02 nämns
utöver Q-koden följande övriga trafikförkortningar, som berör amatörradio.
Radioamatörerna använder i praktiken
många fler trafikförkortningar än dessa. En
lista kan beställas från SSA:s kansli.

Ett exempel på en avsnitt ur en amatörradiosändning, där trafikförkortningar används
särskilt flitigt:
"gm es tnx vy much om fer ur rprt. u are
cmg in hr ufb. my tx is .... and rx .... anta 3
el beam . condx hr gud mni dx stns hrd . wl
nw nil so tks es 73 "
l klartext ser exemplet ut så här:
"good morning and thank you very much
Old Man for your report. You are coming in
here ultra fine business. My transmitter is .....
and receiver .. ... antenna is a 3 element
beam. Conditions here are good many
stations heard. Weil now nothing for you so
thanks and kindest regards "

l reglementsprovet för radioamatörcertifikat ingår frågor om trafikförkortningar.
Förkortning Engelskt uttryck

Svensk betydelse

BK
CQ

avbryt(-a) (sändningen)
allmänt anrop, till alla
telegrafi (A 1A)
från ..... (anropssignal)
"kom"
meddelande, telegram
var god (att .... )
allt uppfattat, mottaget
mottagare
sändare
din, ditt, dina, er

cw

DE
K
MSG
PSE
R
RX
TX
UR

break
"seek you"
continous waves
franska "de"
come
message
please
received
receiver
transmitter
your

Utöver ovanstående trafikförkortningar upptas i CEPT-rekommendationen även följande bokstavskombinationer, vilka används i
teleprintertrafik i stället för motsvarande
morsetecken, slagna utan tecken mellanrum.
(Strecket ovanför bokstäverna betecknar
att det inte finns något mellanrum).

Vidare upptas i CEPT -rekommendationen bokstavskombinationen RST som en
trafikförkortning. Denna får tydas som en
fråga eller anmodan om signal rapport.
Mer om detta under 13. stationsdagbok
och Appendix J.

AR
sluttecken
+
VA eller SK avslutningstecken @

1111 - 5

RE LE
\section{Internationell nödtrafik och trafik vid naturkatastrofer}
Nödsignaler
l ITU Radioreglemente (RR) framgår av Artikel 39 om "Distress Communications" hur
nödsignaler skall vara formulerade. Dessa
signaler är internationella och sänds när ett
skepp, flygplan eller annan farkost hotas av
allvarlig och omedelbarfara och begär hjälp.
Nödsignalen på morsetelegrafi består av
teckendelarna -- .. - - - .. -- sända i en följd,
där längden på de långa teckendelarna betonas så att de klart skiljer sig från de korta.
Signalen skrivs som bokstäverna SOS med
ett streck ovanför.
Nödsignalen på radiotelefoni består av
ordet MAYDAY uttalat som det franska uttrycket "m'aider".
På amatörradiofrekvenserna förekommer
även CQ EMERGENCY som internationellt
nq~anrop. l Sverige kan man även ropa
NODANROP på svenska.
Nödtrafik
Vid 1979 års världsradiokonferens (WARC)
antogs bl.a. Resolution 640, vilken avsåg
internationell radiokommunikation på frekvensband upplåtna åt amatörradion, i händelse av naturkatastrofer.
Resolutionen är inskriven i RR. l CEPTrekommendation T/R 61-02 nämns beslutspunkterna 1-5. För orientering återges resolutionen med dessa punkter kursiverade.
"l betraktande av
a. att i händelse av naturkatastrofer de normala kommunikationssystemen ofta är
överbelastade, skadade eller helt avskurna
b. att snabbt upprättande av kommunikationer är absolut nödvändigt för att möjliggöra världsomspännande hjälpaktioner
c. att amatörbanden inte är bundna av fasta
bandplaner eller kungörelser och därför
är vällämpade för korttidsanvändning vid
nödtillfällen
d. att internationell nödtrafik kan underlättas genom tillfällig användning av vissa
frekvensband upplåtna åt amatörradiotrafiken

1111 - 6

e. att i sådana situationer amatörradiostationer p.g .a. deras stora geografiska spridning och påvisade kapacitet kan hjälpa till
att upprätthålla viktiga radioförbindelser
f. att det existerar nationella och regionala
amatörnödtrafiknät som använder frekvenser inom gällande bandplan för amatör rad iot ra f ik
g. att i händelse av en naturkatastrof direkt
förbindelse mellan amatörradiostationer
och andra radiotjänster möjliggör överförandet av livsviktiga meddelanden tills
normala radioförbindelser åter kan upprättas.
Med insikt om att befogenheter och ansvar
för sådan radiotrafik vid naturkatastrofer vilar på berörda länders myndigheter, beslutar
konferensen
1. att frekvensbanden specificerade i No.
51 O *får användas av myndigheter inom
ramen för internationell nödtrafik
2. att sådan användning av amatörbanden
skall begränsas till nödtrafik i samband
med naturkatastrofer
3. att i sådana fall trafiken av icke-amatörradiostationer skall inskränkas till nödtillfället inom det speciella område som
anges av resp myndighet i det drabbade
landet
4. att nödtrafiken skall äga rum inom
katastrofområdet och mellan detta och
vederbörande hjälporganisations högkvarter
5. att sådan nödtrafik endast får upprättas
efter medgivande ifrån det drabbade landets myndighet
6. att nödtrafik ifrån länder utanför inte får
upphäva redan befintliga nationella eller
internationella amatörnödtrafiknät
7. att ett nära samarbete mellan amatörradiostationer och andra radiotjänster,
som i en framtid kan finna det nödvändigt
att för nödtrafik använda amatörbanden,
är önskvärt
8. att vid sådan internationell trafik såvitt
möjligt skall undvikas att störa amatörradiotrafik.

ET DER
Anhåller konferensen hos myndigheterna**
1. att skapa sådana förutsättningar som tillåter genomförandet av internationell nödtrafik
2. att i sina radioreglementen upptaga föreskrifter som tillåter genomförande av nödtrafik."

Om Du hör en nödsignal på radio

* 51 Oär den fotnot i frekvenstilldelningstabellen, som hänvisar till Resolution 640.
De amatörradioband som specificeras för
användning i händelse av naturkastrater
är 3.5, 7.0, 1O. i, i 4.0, i 8.068, 21.0, 24.89
och 144 MHz.
** Med myndigheterna avses respektive
lands teleadministration.

Du själv sänder nödsignal över radio

Avbryt omedelbart din egen sändning när du
hören nödsignal. Lyssna på nödmeddelandet
och SKRIV NER vad som sägs. Notera position, frekvens, tidpunkt etc. Anmäl vad du
hört på följande sätt.

Uppträd lugnt och sansat, när du kallar på
hjälp över radion. Tänk först och sänd sedan. Som ovan sagts måste den som svarar
dig och sedan ringer 112 (förut 90000) meddela larmoperatören att Ditt nödanrop kommit via radio.

Nödsignal från radioamatör i utlandet
Nödsignal från en radioamatör i ett katastrofområde utomlands ska anmälas till UD, d.v.s.
Utrikesdepartementet.
På dagtid kl. 8 - 17
te l. 08-40 55 950.
te l. 08- 40 55 001 .
På övrig tid

Nyckelordet för dina åtgärder är LARMA:
läge
Ange olycksplatsens läge. Du kan
ange gatu- eller vägnamn eller
riktmärken som t.ex. vägkorset,
gränsen, bron, järnvägen etc.
Analysera Gör en överblick över olycksplatsen och tala om vad som hänt.
Några skadade? Några innestängda? Brinner det? Släpps
farliga ämnen ut?
Ropa
Ropa på hjälp. Använd gärna en
repeater på 2-metersbandet så
att du når många, men även andra frekvenser kan användas.
Anropamed NÖDANROPFRÅN
SMXxxx. Fråga efter någon med
telefon. Ge inte upp om du inte får
svar genast.
Meddela Meddela när du fått kontakt med
någon med telefon, sänd NÖDTRAFIKPÅGÅR för att freda frekvensen och NÖDMEDDELANDET med de viktigaste uppgifterna. Begär att uppgifterna repeteras och ta löfte på att de sänds
vidare. Begär att få veta när så
har skett. Påminn annars!
Avvakta Vänta på platsen tills hjälp har
anlänt. Passa radion så att du
kan svara på frågor. Behövs inte
lätJgre din hjälp, avsll!~a då m~d
NODTRAFIK UPPHOR FRAN
SMXxxx .. KLART SLUT.

Nödsignal från svenskt landområde
l Sverige bör du ringa 112 (förut 90 000) för
att kalla på Ambulans, Polis, Räddningskår,
Sjöräddning, Flygräddning etc. Ditt telefonnummer visas automatiskt i larmoperatörens display.
För att undvika missförstånd och feldirigering av räddningsinsatserna MÅSTE
du meddela operatören att nödanropet kommit via radio. Själva olycksplatsen kan ju
ligga i ett helt annat riktnummerområde, än
som både Ditttelefonsamtal och nödanropet
kommer ifrån.
Nödsignal från fartyg eller luftfarkost
Om nödsignalen inte besvaras av någon
kust- eller markstatio n, ring 1 i 2 (förut 90000)
och begär Sjöräddning respektive Flygräddning och meddela dina iaktagelser. Du
kan även rapportera direkt till centralerna
Sjöräddning i Stockholm 08 - 601 79 00,
Sjöräddning i Göteborg 031 - 64 80 20 och
Flygräddning 031 - 64 80 00.
Vidarebefordra nödmeddelandet utan att
ändra på det!

1111 -7

R

l

H TRAFIKMET DER

~©~

PT

\section{Anropssignaler}
Anropssignalernas sammansättning
Varje land har unika anropssignaler för all
sin radiotrafik. Signalerna utformas enligt
radioreglementet (RR) på sätt, som beror på
syftet med varje särskild radiostation. l RR
finns definitioner för olika slags stationer,
t. ex. stationer för fast radio, landmobila stationer, stationer i fartyg, i sjöräddningsfarkoster, i flygplan, amatörradiostationero.s.v.

Identifiering av amatörradiostationer
En radiostation skall identifieras med den
anropssignal, som tilldelats av det egna landetsteleadministration (myndighet).! Sverige
är det Post- och telestyrelsen (PTS).
Anropssignalen meddelas i det tillstånd för
innehav och användning som utfärdats för
tillståndshavaren ifråga. Signalen gäller så
länge som tillståndet är giltigt.

Amatörradiosignaler är uppbyggda på följande sätt:
Antal kombinationer Anmärkningar
Teckenkombinationer
YOA - Y9Z
260
Första tecknet "Y" räcker ensamt
YOAA - Y9ZZ
6760
som nationell identitet om det är
YOAAA- Y9ZZZ
175760
B, F, G, l, K, M, N, R eller W.
XXOA - XX9Z
260
Signaler som börjar med en siffra,
XXOAA - XX9ZZ
6760
när andra bokstaven är O eller l,
XXOAAA - XX9ZZZ
175760
är dock ej tillåtna för amatörradio.
(XX är det två första tecknen i en tilldelad signalserie)
Sverige är tilldelat teckenkombinationer i serierna SAA - SMZ, 7SA - 7SZ och 8SA - 8SZ.
Anropssignalerna för svenska amatörradiostationer är uppbyggda på följande sätt, varvid
med distrikt avses amatörradiodistrikt
Amatörradiotillstånd (CEPT-tillstånd) för
SM + distriktssiffra + bokstäver,
radioamatörer
SK + distriktssiffra + bokstäver,
amatörklubbar
militära förband
SL + distriktssiffra + bokstäver,
Sl + distriktssiffra + bokstäver
(specialtillstånd),
amatörklubbar
amatörklubbar
SJ + distriktssiffra + bokstäver
(specialtillstånd),
7S + distriktssiffra + bokstäver
amatörklubbar
(specialtillstånd),
8S + distriktssiffra + bokstäver
amatörklubbar
(specialtillstånd),
SSA-tillstånd inom SSAs utbildningsverksamhet
SH + distriktssiffra + bokstäver (AAA- CZZ).
Sverige är indelat i amatörradiodistrikt med följande numrering och utsträckning:
Distrikt Utsträckning
o Stockholms (AB) län
1
Gotlands (l) län
2
Västerbottens (AC) och
Norrbottens (BD) län
Gävleborgs (X), Jämtlands (Z) och
3
Västernorrlands (Y) län
4
Örebro (T), Värmlands (S) och
Kopparbergs (W) län
5
Östergötlands (E), Södermanlands
(D), Västmanlands (U) och Uppsala
(C) län

1111-8

Distrikt Utsträckning
Hallands (N), Älvsborgs (P), Göte6
borgs och Bohus (O) län samt Skaraborgs (R) län
7
Malmöhus (M), Kristianstads (L),
Blekinge (K), Kronobergs (G),
Jönköpings (F) och Kalmar (H) län.
Distriktssiffran i signalen bestäms av det
län som hemadressen är belägen inom. Vid
sändning utanför hemadressen bör det
framgå av tillägg till signalen.

LER
l Post- och telestyrelsens föreskrifter sägs
dock inte vilken distriktssiffra som skall användas, när sändning sker från annan plats
än hemortsadressen.
Med stöd av praxis rekommenderar dock
SSA att följande regler tillämpas:
e
Vid trafik från en regelbundet använd
fritidsbostad kan i anropssignalen användas den distriktssiffra som utvisar var
fritidsbostaden är belägen.
e
Vid trafik från annan tillfällig plats bör
anropssignalen åtföljas av snedstreck och
siffran för det distrikt varifrån sändningen
görs.
Exempel: SMOXYZ/0, SMOXYZ/6 etc.
e
Vid trafik från mobil station bör den ordinarie anropssignalen även åtföljas av /M.
Exempel: SMOXYZ/6M.
e
Vid trafik från mobil station inom hemorten kan dock den extra distriktssiffran
utelämnas.
Exempel: SMOXYZ/M.
e
Vid trafik från sjöfarkost bör den ordinarie
anropssignalen åtföjas av /MM.
• Vid trafik från luftfarkost bör den ordinarie
anropssignalen åtföljas av /AM.
e
Vid trafik från svensk farkost på internationellt territorium kan distriktssiffran 8
användas.
• Vid sändning från ett annat lands territorium gäller det landets bestämmelser.
Vid osäkerhet- Skaffa upplysningar från
SSA:s reciprokfuntionär!
Utländsk radioamatör på besök i Sverige
skall använda sin anropssignal från det egna
landet, föregånget av SM*l där * motsvaras
av siffran för det svenska distrikt varifrån
sändningen görs.

Användning av anropssignaler

Både motstationens och den egna anropssignalen skall användas i början och slutet
av varje sändning.
Under sändningen skall anropssignalen
upprepas "med korta mellanrum", utan närmare precisering av mellanrummet.
Även om man inte har kontakt med en
motstation, skall den egna anropssignalen
anges vid varje sändning.
Se vidare i PTS föreskrifter.

CH TRAFIKMET DER
Hur man genomför en radiokontakt

Det finns många sätt att genomföra en radiokontakt, men det finns några grundregler för
hur man uppträder och utväxlar samtal. Ett
trevligt och kamratligt uppträdande är en
hederssak inom amatörradion. Det behöver
inte bli stelt för den skull!
Allmänt anrop är ett sätt att kalla på någon
-vem som helst- att kommunicera med.
På telegrafi låter det så här:
de SMOXYZ K, d.v.s anropet
först och därefter den egna signalen.
På telefoni låter det så här:
Allmänt anrop, allmänt anrop, allmänt anrop
från SMOXYZ Kom. Glöm inte Kom i slutet!

ca ca ca

Riktat anrop gör man, när man vill tala med
någon särskild station. Då sänder man först
signalen på den station, som man vill tala
med och därefter sin egen signal.
På telegrafi låter det så här:
SMOÅÄÖ SMOÅÄÖ SMOÅÄÖ de SMOXYZ
SMOXYZ SMOXYZ K
På telefoni låter det så här:
SMOÅÄÖ SMOÅÄÖ SMOÅÄÖ från SMOXYZ
SMOXYZ SMOXYZ Kom
Motstationen svarar förhoppniong~vis på
anropet, alltså SMOXYZ från SMOAAO Kom.
Upprättad förbindelse. När en station svarat
på anrop, lämnar man först sin signalrapport
enligt RST-koden och presenterar sig med
sitt förnamn och var man finns. Motstationen
kvitterartroligen med sina motsvarande uppgifter. Varje gång, som man överlämnar ordet till motstationen säger man först motstationens signal och därefter sin egen. Därefter säger man Kom och lyssnar. Om man
har en telegrafiförbindelse och bara vill tala
med den stationen kan man sända KN (kom
du och ingen annan (nobody else).
Avsluta förbindelse. När man så småningom
avslutar kontakten tackar man för sig på och
utbytter avskedhälsningar.
Då kan det låta så här:
.......... Tack för en trevlig förbindelse och på
återhörande. SMOÅÄÖ från SMOXYZ. Klart
Slut.
Träna med din instruktör på att klara olika
slags trafiksituationer!

1111-9

REG
\section{Bandplaner}
IARU:s bandplaner, syfte och ändamål
Det allra vanligaste är att en radiostation eller
ett nät av stationer tilldelas en eller ett fåtal
frekvenser samt väl preciserade villkor i övrigt. Amatörradio är däremot en radiotjänst,
som tilldelas inte bara enstaka frekvenser
utan hela frekvensband samt inom dessa
band förhållandevis stor frihet till personligt
val av frekvens, sändningsslag etc.
Därvid kan den enskilde radioamatören
inte ställa anspråk på ostörda frekvenser. l
stället är det upp till radioamatörerna, att
själva samråda och rekommendera varandra om hur de tilldelade frekvensbanden bör
fördelas på olika slags användning. Denna
fördelning av trafiken kallas bandplan.

Internationella Amatörradiounionen IARU är det enda organ på internationell
nivå, där samråd om amatörradions intressen sker regelbundet, dels i arbetsmed olika inriktning och dels i
generella konferenser.
IARU har som syfte att
• verka för att av ITU tilldelade frekvensband för amatörradio bevaras,
• förbättra amatör- och amatörsatellittjänsternas status inom tilldelade
frekvensband,
• verka för tilldelning av ytterligare
frekvensband för amatörradio,
e
frekvensplanera amatörradiotrafiken
inom tilldelade amatörradioband genom samråd och rekommendationer.

Syftet med en bandplan är att ge utrymme för alla aspekter inom amatörradio självträning, kommunikation och tekniska
undersökningar.
Radioamatörernas bandplaner siktar på
att ge möjlighet till så många olika amatöraktiviteter som möjligt, såväl sändningsslag
som tekniker, både nu och i framtiden. För
att utnyttja banden på bästa sätt är det normalt att minsta möjliga bandbredd samt optimal sändarutrustning och teknik används.
För att alla skall kunna utöva amatörradio
med ett minimum av störningar, förutsätts att
man använder utrustningar som är "state of
the art".
God insikt i frekvensplanering, tillräckliga
resurser, gott anseende samt internationellt
samarbete behövs för att främja amatörradion. De flesta nationella amatörradioorganisationer har sedan många år ett världsomfattande samarbete genom sitt organ The
International Amateur Radio Union -IARUsom är organiserat som tre regioner. Dessa
regioner sammanfaller geografiskt med ITU :s
regioner. Region 1 omfattar Afrika, Europa
och västra Asien.

Svenska bandplaner, sändningsslag
Tilldelningen av frekvensband för amatörradioanvändning sker enligt överenskommelser mellan telemyndigheterna i de länder som är anslutna till ITU. Tilldelningen är
därvid i stort sett lika i de flesta länder. Av
olika skäl förekommer dock skillnader såväl
mellan ITU-regioner som länder.
l Sverige regleras amatörradioanvändningen främst genom Radiolagen och Postoch telestyrelsens föreskrifter. l anslutning
till frekvenstilldelningen anges tillåtna sändningsslag och amatörradiostatus i respektive band. Inom denna ram är det upp till
radioamatörerna själva att utnyttja sina möjligheter
bästa sätt.
bandplaner fungerar som radioamatörernas rekommendationer till varandra. Endast i minsta utsträckning medverkar
PTS till reglering inom dessa planer.

Se Appendix F

Se Appendix G.

llli- i O

Föreningen Sveriges SändareamatörerSSA - företräder de svenska radioamatörerna i IARU Region 1.

RE

R CH TRAFIKMET DER
